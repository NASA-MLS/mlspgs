\documentclass[11pt]{article}
\usepackage[fleqn]{amsmath}\textwidth 6.25in
\oddsidemargin -0.25in
%\evensidemargin -0.5in
\topmargin -0.6in
\textheight 9.30in

\begin{document}

%\tracingcommands=1
\newlength{\hW} % heading box width
\settowidth{\hW}{\bf wvs-005r1}
%\settowidth{\hW}{Page \pageref{lastpage}\ of \pageref{lastpage}}
\makeatletter
\def\@biblabel#1{#1.}
\newcommand{\ps@twolines}{%
  \renewcommand{\@oddhead}{%
    12 June 2001\hfill\parbox[t]{\hW}{{\bf wvs-005r1}\newline
                          Page \thepage\ of \pageref{lastpage}}}%
\renewcommand{\@evenhead}{}%
\renewcommand{\@oddfoot}{}%
\renewcommand{\@evenfoot}{}%
}%
\makeatother
\pagestyle{twolines}

\vspace{-10pt}
\begin{tabbing}
\phantom{References: }\= \\
To: \>Regina Glover\\
Subject: \>Continuation of consulting agreement 1202403\\
From: \>Van Snyder\\
\end{tabbing}

\parindent 0pt \parskip 3pt
\vspace{-20pt}

This memorandum address the reasons to continue consulting agreement
1202403 with Dr. Fred T. Krogh.

\section{Project}

Dr. Fred Krogh has been advising the Level 2 software development team
for the Microwave Limb Sounder instrument on the EOS AURA project
concerning computational mathematics.

The mathematical problem immanent in the level 2 retrieval is a nonlinear
least-squares problem.  Nonlinear least-squares problems can be solved by
iterative methods based on local linearization; they require to solve
linear problems at each iteration.  In the case of the level 2 problem,
the linear problem that results has an unusual structure, the exploiting
of which results in substantial computational economy.

Dr. Krogh has an unique combination of knowledge, skills and talents in
computational mathematics and mathematical software.  In particular, he
has written two software packages to solve nonlinear least-squares
problems.  One of these packages is used by the trajectory planning
group; it routinely and efficiently solves problems that no other package
is able to solve at all.

Dr. Krogh also has a deep understanding of the relation between linear
algebra, software to solve problems in linear algebra, and the solution
of nonlinear least-squares problems.  In particular,  he has experience
in developing software to solve linear algebra problems having unusual
structure, and incorporating such into software to solve nonlinear
least-squares problems.

Dr. Krogh has already given the team substantial advice on methods and
software to solve nonlinear least-squares problems, and software and
algorithms to solve the linear problems that arise in solving nonlinear
least-squares problems.  In particular, he has given the team software,
that already has many of the features we need in the level 2 program, to
solve nonlinear least-squares problems.  His advice has to date been
critical in the development of the level 2 program.  To the extent that
past experience is an indication of future expectations, continuing to
exploit his knowledge, skills and talents should save several work-months
of effort, and result in software of superior reliability, accuracy and
performance.

Pursuant to this consulting agreement, Dr. Krogh would provide mathematical
development, advice on mathematical approaches, advice on computational
approaches, advice on specialized software organization strategies,
advice on selection of existing software, and advice on modification of
existing software.

\section{Availability of specific expertise at JPL}

The specific expretise described above was available at JPL until Dr.
Krogh was forced to retire prematurely, due to lack of funding.  It is no
longer available.

\section{Duration of consulting agreement}

This consulting agreement shall expire on 23 September 2002.  The
agreement shall provide for 20 days of work.  Account 100668-B.B.5 shall
provide funds.

\label{lastpage}
\end{document}
% $Id$
