\documentclass[11pt]{article}
\usepackage[fleqn]{amsmath}\textwidth 6.5in
\oddsidemargin -0.25in
%\evensidemargin -0.5in
\topmargin -0.25in
\textheight 9in

\newcommand{\docname}{wvs-094r2}
\newcommand{\docdate}{20 May 2010}

\usepackage{alltt}
\usepackage{longtable}

\begin{document}

%\tracingcommands=1
\newlength{\hW} % heading box width
\newlength{\pW} % page number field width
\settowidth{\hW}{\bf\docname}
\settowidth{\pW}{Page \pageref{lastpage}\ of \pageref{lastpage}}
\ifdim \pW > \hW \setlength{\hW}{\pW} \fi
\makeatletter
\def\@biblabel#1{#1.}
\newcommand{\ps@twolines}{%
  \renewcommand{\@oddhead}{%
    \docdate\hfill\parbox[t]{\hW}{{\hfill\bf\docname}\newline
                          Page \thepage\ of \pageref{lastpage}}}%
\renewcommand{\@evenhead}{}%
\renewcommand{\@oddfoot}{}%
\renewcommand{\@evenfoot}{}%
}%
\makeatother
\pagestyle{twolines}

\vspace{-10pt}
\begin{tabbing}
\phantom{References: }\= \\
To: \>Bill, Paul, Dong, Nathaniel\\
Subject: \>Usage of {\tt Mie\_Tables} program\\
From: \>Van Snyder\\
Reference: \>wvs-065, wvs-066, wvs-067, wvs-068, wvs-070, wvs-071
\end{tabbing}

\parindent 0pt \parskip 6pt
\vspace{-10pt}

\section{Introduction}

The {\tt Mie\_Tables} program
computes the emission cross section
         $\beta_{c\_e} = 2\pi \int_0^\infty n(r) r^2 \xi_e(r)\, \text{d}D$,
the scattering cross section
         $\beta_{c\_s} = 2\pi \int_0^\infty n(r) r^2 \xi_s(r)\, \text{d}D$,
where $\xi_e(r)$ and $\xi_s(r)$ are computed by {\tt Mie\_Efficiencies}, q.v.,
their derivatives w.r.t. temperature and IWC, the total Ice Water Content
        IWC$_{\text{total}}$ =
          $\frac83 \pi \rho_{\text{ice}} \int_0^\infty n(r)\, r^3 \text{d}D$,
the integrated phase function
$P(\theta) =
 \frac{\lambda^2}{2\pi\beta_{c\_s}} \int_0^\infty n(r) p_0(\theta,r) \text{d} r$,
and the derivatives of all of these quantities w.r.t. temperature and IWC.

The program is in the CVS archive in \dots/{\tt Mie}.

\section{Input}

All input uses Fortran NAMELIST, with name IN.  Each namelist block begins
on a new line, with an ampersand followed by the namelist name (case
insensitive), i.e., in this case {\tt \&in}, followed by \emph{name =
value} fields preceded by blanks and an optional comma, terminated by
slash (/).  The program can process several input blocks in a single
invocation.

Namelist input can occupy as many lines as desired, the only restriction
being that end-of-line cannot appear within a name or value.

The \emph{value} part can be a single value, or several values, each
followed by blanks and an optional comma if \emph{name} is an array.
\emph{name} can also be an array element or array section designator. If
\emph{name} designates an array (or array section) and \emph{value}
specifies fewer values than \emph{name} designates, those values are
transferred in array element order and remaining elements of the array
are not affected.

If a \emph{name = value} field does not appear for a Namelist name, the
variable value is not changed.

Anything after an exclamation point (!) not part of a character value, up
to the end of the line upon which it appears, is a comment.  Blank lines
are allowed before and within a namelist block.

Names are not case sensitive.  The names that can appear in the {\tt
\&in} namelist and their use in the program are explained in the
following table.  Names are scalars unless otherwise specified.

\begin{longtable}{|l|p{0.95in}|p{4in}|}
\hline
Name & Default value & Usage \\
\hline
\endfirsthead
\hline
Name & Default value & Usage \\
\hline
\endhead
IWC\_Min   & -4.0   & Minimum value for $\log_{10}$ of IWC. \\
IWC\_Max   & 0.0    & Maximum value for $\log_{10}$ of IWC. \\
N\_IWC     & 5      & Number of equally spaced values of $\log_{10}$ of IWC. \\
T\_Min     & -100.0 & Minimum temperature in degrees Celsius (not Kelvin!). \\
T\_Max     & -15.0  & Maximum temperature in degrees Celsius (not Kelvin!). \\
N\_T       & 5      & Number of equally spaced values of temperature. \\
F\_s       & all -1.0 & A 100-element array of frequencies in GHz.  This
                      is reset to all -1.0 before reading each input. 
                      Calculations are carried out starting with the first
                      frequency, and continuing until a negative one is
                      found. \\
R\_Min     & 1.0    & Minimum ice particle size in $\mu$m. \\
R\_Max     & 2000.0 & Maximum ice particle size in $\mu$m. \\
Theta\_Min & 10.0   & Minimum value for $\theta$ in degrees. \\
Theta\_Max & 170.0  & Maximum value for $\theta$ in degrees. \\
N\_Theta   & 2      & Number of equally spaced values of $\theta$. \\
Half       & {\tt .false.} or {\tt F}
                    & The range of $\theta$ covers a half circle; used for
                      blunder testing. \\
N\_Cut     & 100    & Maximum order for Bessel functions. \\
File       & blank  & Output to this file if not blank.  Enclose value
                      between quotes or apostrophes.  Maximum 1023
                      characters.  See {\tt HDF}. \\
HDF        & {\tt .true.} or {\tt T}
                    & Output using HDF if true, else output to a Fortran
                      unformatted file.  See {\tt File}. \\
IOPT       & 0, 6, 2, 0, 10, 0, 9, 2000000, 0, 0, 10*0
                    & A 20-element array used to provide options to the
                      {\tt DINT} quadrature routine from Math77.  Ask me
                      for details if you need them. \\
NFMAX      & 2000000 & Maximum number of integrand values for any
                      integrand.  Equivalenced to {\tt IOPT(8)}. \\
WORK       & Undefined & 20-element array.  First element is the variable
                      of integration.  Other elements are used in
                      conjunction with {\tt IOPT}.  Ask me for details if
                      you need them. \\
Warn       & {\tt .false.} or {\tt F}
                    & Print warnings from {\tt DINT}. \\
Level      & 2      & Print level for the {\tt CADRE} quadrature routine.
                      Ask me for details if you need them. \\
Tan\_u     & {\tt .false.} or {\tt F}
                    & Do $x = \tan(u)$ change of variable in integrals. 
                      Ask me for details if you need them. \\
Only       & blank  & If not blank, only do this integral.  Meaningful
                      values are {\tt P}, {\tt dP\_dIWC}, {\tt dP\_dT},
                      enclosed between quotes or apostrophes.
                      These values are case sensitive. \\
Capture    & 0      & Write values of integrands on this Fortran output
                      unit if $>$ zero. \\
Derivs     & {\tt .false.} or {\tt F}
                    & Calculate derivatives. \\
Details    & 0      & Printing level.  $<$0 means print nothing. \newline
                      0 Means totals only. \newline
                      $>$0 Means error estimates and numbers of function values
                          for each integral. \newline
                      $>$1 Means orders of Bessel functions. \newline
                      $>$2 Means coefficients $\alpha$, $\mu$, $\sigma$, etc. \\
Diffs      & {\tt .false.} or {\tt F}
                    & Print differences, for checking derivatives. \\
Norm       & {\tt .false.} or {\tt F}
                    & Report quantities divided by total IWC. \\
Progress   & {\tt .false.} or {\tt F}
                    & Print summary after each integral. \\
WantBeta   & {\tt .true.} or {\tt T} & Compute $\beta_{c\_e}$ and
                      $\beta_{c\_s}$. \\
WantIWC    & {\tt .true.} or {\tt T} & Compute total IWC. \\
WantP      & {\tt .true.} or {\tt T} & Compute $P(\theta)$. \\
Test\_Tol  & 3.0    & Number of standard deviations for blunder testing. \\
Blunder\_Details &  0 & Print level for blunder detection.
                        $<$1 means no printing. \newline
                      $>0$ means print outliers. \newline
                      $>1$ means print {\tt avg}, {\tt avg2}, {\tt
                      stdev2}, {\tt maxloc}, replacements. \newline
                      $>2$ means print differences. \\
Blunder\_Order  & 5 & Order of least-squares polynomial fit to use for
                      blunder detection. \\
\hline
\end{longtable}

Here are sample inputs:

\begin{alltt}
&in
    r_max = 1250
    f_s       = 118, 190, 240, 640, 2500           ! GHz
    n_cut     = 200 ! Max order for Bessel functions
    n_iwc     = 10 iwc_min = -4.0 iwc_max = 0.0    ! Log10 IWC
    n_t       = 10 t_min = -100.0 t_max = -15.0    ! Celsius
    n_theta   = 361 theta_min = 0 theta_max = 180  ! Degrees
    half      = t ! theta covers a half circle
    derivs    = t ! Need derivatives
    file      = 'runs/Mie-2009-08-04.hdf5'
    HDF       = t ! Output is HDF, not Fortran unformatted
    details   = 0 ! Minimal stdout output
    progress  = t ! More verbose stdout
    iopt(9:10) = 5, 8 ! Specify integrand relative error in WORK(8)
    work(8) = 1.0e-9 ! Integrand relative error
    blunder_details=0
    blunder_order=7
/
\end{alltt}

\begin{alltt}
&in f=63   theta_min=0 theta_max=180 n_theta=19 half=t derivs=t /
&in f=118  /
&in f=190  /
&in f=203  /
&in f=240  /
&in f=640  /
&in f=2500 /
\end{alltt}

\section{Output}

Output quantities include those shown in the following table.  Quantities
that are not requested do not appear in the output.

\begin{longtable}{|l|p{4in}|}
\hline
Quantity & Description \\
\hline
\endhead
R\_Max  & From input. \\
R\_Min  & From input. \\
N\_Cut  & From input. \\
IWC\_s(:) & As specified by input. \\
T\_s(:) & As specified by input. \\
Theta\_s(:) & As specified by input. \\
F\_s(:) & From input.  Size is the number of positive frequencies before
          the first negative one. \\
Beta(:,:,:,:) & $\beta_{c\_e}$ and $\beta_{c\_s}$.  Dimensions are
Temperature $\times$ IWC $\times$ Frequency $\times$ 2.  Final subscript =
1 for $\beta_{c\_e}$ and 2 for $\beta_{c\_s}$.\\
Eest(:,:,:,:) & Error estimates for $\beta$ integrals; same dimensions as
          {\tt Beta}. \\
nFunc(:,:,:,:,:) & Quadrature status flags and numbers of integrand values.
          Dimensions are 2 $\times$ Temperature $\times$ IWC $\times$
          Frequency $\times$ 6.  Subscripts for the first dimension are 1
          for the number of integrand values and 2 for the quadrature
          status flag.  Subscripts for the final dimension are for
          $\beta_{c\_e}$, $\beta_{c\_s}$,
          $\partial\beta_{c\_e}/\partial$IWC,
          $\partial\beta_{c\_s}/\partial$IWC,
          $\partial\beta_{c\_e}/\partial T$ and
          $\partial\beta_{c\_s}/\partial T$. \\
MaxOrd(:,:,:,:) & Maximum order of Bessel functions used; same dimensions
          as {\tt nFunc}.\\
P(:,:,:,:) & Phase function.  Dimensions are Temperature $\times$ IWC
          $\times$ $\theta$ $\times$ Frequency.\\
E\_P(:,:,:,:) & Error estimates for {\tt P}.  Same dimensions as {\tt P}.\\
nFuncP(:,:,:,:,:,:) & Quadrature status flags and numbers of integrand
          values for computing $P(\theta)$.  Dimensions are 2 $\times$
          Temperature $\times$ IWC $\times$ $\theta$ $\times$ Frequency
          $\times$ 3.  Subscripts for the first dimension are as for {\tt
          nFunc}.  Subscripts for the last dimension are $P$, $\partial
          P/\partial T$ and $\partial P/\partial$IWC. \\
MaxOrdP(:,:,:,:,:) & Maximum order of Bessel functions used in computing
          {\tt P}.  Dimensions are Temperature $\times$ IWC $\times$
          $\theta$ $\times$ Frequency $\times$ 3.  Subscripts for the last
          dimension are the same as for {\tt nFuncP}. \\
WantBeta & From input. \\
WantIWC  & From input. \\
WantP    & From input. \\
dBeta\_dIWC(:,:,:,:) & $\partial\beta/\partial$IWC.  Dimensions are
         the same as for {\tt Beta}. \\
E\_dBeta\_dIWC(:,:,:,:) & Error estimate in computing
         $\partial\beta/\partial$IWC.  Dimensions are the same as for {\tt
         Beta}. \\
dBeta\_dT(:,:,:,:) & $\partial\beta/\partial T$.  Dimensions are
         the same as for {\tt Beta}. \\
E\_dBeta\_dT(:,:,:,:) & Error estimate in computing 
         $\partial\beta/\partial T$.  Dimensions are the same as for {\tt
         Beta}. \\
dP\_dIWC(:,:,:,:) & $\partial P/\partial$IWC.  Dimensions are the same as
         for {\tt P}. \\
E\_dP\_dIWC(:,:,:,:) & Error estimate in computing $\partial
         P/\partial$IWC.  Dimensions are the same as for {\tt P}. \\
dP\_dT(:,:,:,:) & $\partial P/\partial T$.  Dimensions are the same as
         for {\tt P}. \\
E\_dP\_dT(:,:,:,:) & Error estimate in computing $\partial
         P/\partial T$.  Dimensions are the same as for {\tt P}. \\
\hline
\end{longtable}

\label{lastpage}
\end{document}

% $Id$

% $Log$
% Revision 1.2  2010/05/21 00:02:19  vsnyder
% State that the program can handle several input blocks
%
% Revision 1.1  2010/05/20 23:49:17  vsnyder
% Initial commit
