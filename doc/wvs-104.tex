\documentclass[11pt]{article}
\usepackage{alltt}
\usepackage[fleqn]{amsmath}
\usepackage{longtable}
\usepackage{xr}
\externaldocument[100-]{\logdir/wvs-100}
\externaldocument[095-]{\logdir/wvs-095}

\textwidth 6.5in
\oddsidemargin -0.25in
%\evensidemargin -0.5in
\topmargin -0.25in
\textheight 9in

\newcommand{\docname}{\bf wvs-104}
\newcommand{\docdate}{6 March 2011}

\ifx\pdfoutput\undefined
  \pdfoutput=0
  \usepackage[hypertex,plainpages,hyperindex=true]{hyperref}
  \hypersetup{%
    hypertexnames=false%
  }
  % Specify the driver for the color package
  \ExecuteOptions{dvips}
  %\ExecuteOptions{xdvi}
\else
  \ifnum\pdfoutput>0
    \usepackage[pdftex,plainpages,hyperindex=true,pdfpagelabels]{hyperref}
    \hypersetup{%
      hypertexnames=false,%
      colorlinks=true,%
      linktocpage=true,%
    }
    % Specify the driver for the color package
    \ExecuteOptions{pdftex}
  \else
    \usepackage[hypertex,plainpages,hyperindex=true]{hyperref}
    \hypersetup{%
      hypertexnames=false%
    }
    % Specify the driver for the color package
    \ExecuteOptions{dvips}
    %\ExecuteOptions{xdvi}
  \fi
\fi

\hyperbaseurl{}
\ifx\dvidir\undefined
  \newcommand\hr[1]{\href{#1.dvi}{dvi} \href{#1.pdf}{pdf}}
\else
  \newcommand\hr[1]{\href{\dvidir/#1.dvi}{dvi} \href{\pdfdir/#1.pdf}{pdf}}
\fi
\newcommand\h[1]{#1 (\hr{#1})}
\newcommand\hh[2]{#1 \hr{#2}}
\newcommand\hrpdf[1]{\href{#1.pdf}{pdf}}
\newcommand\hpdf[1]{#1 \hrpdf{#1}}

\begin{document}

%\tracingcommands=1
\newlength{\hW} % heading box width
\newlength{\pW} % page number field width
\settowidth{\hW}{\docname}
\settowidth{\pW}{Page \pageref{lastpage}\ of \pageref{lastpage}}
\ifdim \pW > \hW \setlength{\hW}{\pW} \fi
\makeatletter
\def\@biblabel#1{#1.}
\newcommand{\ps@twolines}{%
  \renewcommand{\@oddhead}{%
    \docdate\hfill\parbox[t]{\hW}{{\hfill\docname}\newline
                          Page \thepage\ of \pageref{lastpage}}}%
\renewcommand{\@evenhead}{}%
\renewcommand{\@oddfoot}{}%
\renewcommand{\@evenfoot}{}%
}%
\makeatother
\pagestyle{twolines}

\vspace{-10pt}
\begin{tabbing}
\phantom{References: }\= \\
To: \>Van, Bill, Mike\\
Subject: \>Revisiting $2\times2$ matrix exponential and derivatives, for polarized model\\
From: \>Van Snyder\\
Reference: \>\h{wvs-015}
\end{tabbing}

\parindent 0pt \parskip 6pt
\vspace{-10pt}

The exponential of a $2\times2$ matrix, and its derivative, are needed in
the polarized forward model.  The method currently used is described in
wvs-015.  This memo describes a simpler method.  It would be useful if it
were correct.  Unfortunately, Equation (\ref{one}) doesn't hold because
the matrices in Equation (\ref{rho}) do not commute.

In calculating polarized radiative transfer, we evaluate
$\alpha_{\sigma_-}$, $\alpha_\pi$, and $\alpha_{\sigma_+}$.  We then form
$\delta = \alpha_{\sigma_-} \rho_{-1} + \alpha_\pi \rho_0 +
\alpha_{\sigma_+} \rho_{+1}$, where

\begin{equation}\begin{split}\label{rho}
{\bf\rho}_{-1} =& \left [ \begin{array}{cc}
                    \cos^2 \phi + \sin^2 \phi \cos^2 \theta &
                     -\sin\phi \cos\phi \sin^2 \theta - i \cos \theta \\
                     -\sin\phi \cos\phi \sin^2 \theta + i \cos \theta &
                     \sin^2 \phi + \cos^2 \phi \cos^2 \theta
                   \end{array} \right ] \\
{\bf\rho}_0 =& \sin^2 \theta \left [ \begin{array}{cc}
                    \sin^2 \phi & \sin\phi \cos\phi \\
                    \sin\phi \cos\phi & \cos^2 \phi
               \end{array} \right ] \\
{\bf\rho}_{+1} =& \left [ \begin{array}{cc}
                    \cos^2 \phi + \sin^2 \phi \cos^2 \theta &
                     -\sin\phi \cos\phi \sin^2 \theta + i \cos \theta \\
                     -\sin\phi \cos\phi \sin^2 \theta - i \cos \theta &
                     \sin^2 \phi + \cos^2 \phi \cos^2 \theta
                   \end{array} \right ] = {\bf\rho}^T_{-1} \,.
\end{split}\end{equation}

Then we exponentiate $\delta \Delta s$.

If we could write $\exp(\delta \Delta s)$ in three parts, viz.

\begin{equation}\label{one}
\exp(\delta \Delta s) =
\exp(\alpha_{\sigma_-} \rho_{-1} \Delta s)
\exp(\alpha_\pi \rho_0 \Delta s)
\exp(\alpha_{\sigma_+} \rho_{+1} \Delta s)
\end{equation}

we would have three separate matrix exponentiation problems.  This
appears to be a more difficult problem, but it's actually simpler.  If we
look at Sylvester's formula for the exponential of a $2\times2$ matrix

\begin{equation}
\exp(\mathbf{A}) =
 \frac{\exp(\lambda_1)(\mathbf{A} - \lambda_2 \mathbf{I})}
      {\lambda_1-\lambda_2} +
 \frac{\exp(\lambda_2)(\mathbf{A} - \lambda_1 \mathbf{I})}
      {\lambda_2-\lambda_1}
\end{equation}

we are led to inquire the eigenvalues $\lambda_1$ and $\lambda_2$ of
$\mathbf{A}$.  The eigenvalues of $h \mathbf{A}$ for some scalar $h$ are
$h \lambda_1$ and $h \lambda_2$.  Therefore, in the factors of Equation
(\ref{one}) we are led to consider the eigenvalues of $\rho_{-1}$,
$\rho_0$, and $\rho_{+1}$.  Since $\rho_{+1} = \rho_{-1}^T$, their
eigenvalues are the same.  Those eigenvalues can be calculated explicitly:

\begin{equation}
\begin{array}{ll}
\text{Eigenvalues}(\rho_{-1}) =
 \left[ \begin{array}{c} 1+\cos^2\theta \\ 0 \end{array} \right]
&
\text{, and Eigenvalues}(\rho_{0}) =
 \left[ \begin{array}{c} \sin^2\theta \\ 0 \end{array} \right]
\end{array} \,.
\end{equation}

Since $\lambda_2 = 0$ in both cases, Sylvester's formula simplifies to

\begin{equation}\label{four}
\exp(\mathbf{A}) =
 \frac{\exp(\lambda_1)-1}{\lambda_1} \mathbf{A} - \mathbf{I} \,.
\end{equation}

Equation (\ref{one}) can therefore be evaluated

\begin{equation}\begin{split}\label{six}
\exp(\delta \Delta s) =\,& (F_{\sigma_-}\rho_{-1} - \mathbf{I})
                           (F_\pi\rho_0 - \mathbf{I})
                           (F_{\sigma_+}\rho_{+1} - \mathbf{I})
\text{ where}\\
F_p =\,& \frac{\exp(\lambda_p)-1}{\lambda_p}\text{ for }
p \in \{ \sigma_-, \pi, \sigma_+ \}\,, \\
\lambda_{\sigma_-} =\,& \alpha_{\sigma_-} \Delta s ( 1 + \cos^2\theta )\,, \\
\lambda_\pi = \,& \alpha_\pi \Delta s \sin^2\theta \text{, and} \\
\lambda_{\sigma_+} =\,& \alpha_{\sigma_+} \Delta s ( 1 + \cos^2\theta )
\,. \\
\end{split}\end{equation}

$F_p$ is well-behaved as $\lambda_p \rightarrow 0$: $\lim_{\lambda_p
\rightarrow 0} F_p = 1$.  One needs to take care for small $\lambda_p$,
but that is not difficult (use Taylor's series).

Computing the factors as in Equation (\ref{six}) might be useful if we
apply Trotter's multiplication formula

\begin{equation}\label{Trotter}
\exp(\mathbf{A}+\mathbf{B}+\mathbf{C}) = \lim_{n\rightarrow\infty}
\left( \exp(\mathbf{A}/n)\exp(\mathbf{B}/n)\exp(\mathbf{C}/n)\right)^n\,,
\end{equation}
%
but this depends upon how large $n$ must be for sufficient accurary, and
how much cancellation occurs while computing the right-hand side. 
Evaluation is expensive if $n \neq 2^j$.

Since the matrices in Equation (\ref{rho}) do not depend upon mixing
ratios or temperature, the derivatives of each factor of $\exp(\delta
\Delta s)$ with respect to mixing ratios or temperature are also easy to
compute using

\begin{equation}
\frac{\partial F_p}{\partial x} =
 \frac{(\lambda_p-1)F_p + 1}{\alpha_p} \,
 \frac{\partial \alpha_p}{\partial x}\,.
\end{equation}

These are well-behaved as $\alpha_p \rightarrow 0$: $\lim_{\alpha_p
\rightarrow 0} \frac{\partial F_p}{\partial x} = \frac{\lambda_p}{2
\alpha_p} \frac{\partial \alpha_p}{\partial x}$.

This would simplify the magnetic model, and reduce its run time (if it
were correct).

The second derivatives are also simple to compute from

\begin{equation}
\frac{\partial^2 F_p}{\partial x \partial y} =
 \frac{((\lambda_p-1)^2+1) F_p - 2 + \lambda_p}{\alpha_p^2} \,
 \frac{\partial \alpha_p}{\partial x}
 \frac{\partial \alpha_p}{\partial y} +
\frac{(\lambda_p-1) F_p + 1}{\alpha_p}
 \frac{\partial^2 \alpha_p}{\partial x \partial y} \,.
\end{equation}

Indeed, these are far simpler than computing the second derivative of
$\exp(\delta \Delta s)$ starting from the expression for the first
derivative in wvs-015.

These are well-behaved as $\alpha_p \rightarrow 0$: $\lim_{\alpha_p
\rightarrow 0} \frac{\partial^2 F_p}{\partial x \partial y} =
\frac{\lambda_p^2}{3 \alpha_p^2} \frac{\partial \alpha_p}{\partial x}
\frac{\partial \alpha_p}{\partial y} + \frac{\lambda_p}{2 \alpha_p}
\frac{\partial^2 \alpha_p}{\partial x \partial y}$.

\label{lastpage}
\end{document}

% $Id$
