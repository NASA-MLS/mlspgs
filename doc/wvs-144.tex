\documentclass[11pt]{article}
\usepackage{alltt}
\usepackage[fleqn]{amsmath}
\usepackage{floatflt}
\usepackage{graphicx}
\usepackage{longtable}
\usepackage[strings]{underscore}

\textwidth 6.5in
\oddsidemargin -0.25in
%\evensidemargin -0.5in
\topmargin -0.5in
\textheight 9in

\newcommand{\docname}{wvs-144r1}
\newcommand{\docdate}{29 September 2017}

\ifx\pdfoutput\undefined
  \pdfoutput=0
  \usepackage[hypertex,plainpages,hyperindex=true]{hyperref}
  \hypersetup{%
    hypertexnames=false%
  }
  % Specify the driver for the color package
  \ExecuteOptions{dvips}
  %\ExecuteOptions{xdvi}
\else
  \ifnum\pdfoutput>0
   
\usepackage[pdftex,plainpages,hyperindex=true,pdfpagelabels]{hyperref}
    \hypersetup{%
      hypertexnames=false,%
      colorlinks=true,%
      linktocpage=true,%
    }
    % Specify the driver for the color package
    \ExecuteOptions{pdftex}
  \else
    \usepackage[hypertex,plainpages,hyperindex=true]{hyperref}
    \hypersetup{%
      hypertexnames=false%
    }
    % Specify the driver for the color package
    \ExecuteOptions{dvips}
    %\ExecuteOptions{xdvi}
  \fi
\fi

\hyperbaseurl{}
\newcommand\hr[1]{\href{#1.dvi}{dvi}, \href{#1.pdf}{pdf}}
\newcommand\h[1]{#1 (\hr{#1})}

\begin{document}

%\tracingcommands=1
\newlength{\hW} % heading box width
\newlength{\pW} % page number field width
\settowidth{\hW}{\bf\docname}
\settowidth{\pW}{Page \pageref{lastpage}\ of \pageref{lastpage}}
\ifdim \pW > \hW \setlength{\hW}{\pW} \fi
\makeatletter
\def\@biblabel#1{#1.}
\newcommand{\ps@twolines}{%
  \renewcommand{\@oddhead}{%
    \docdate\hfill\parbox[t]{\hW}{{\hfill\bf\docname}\newline
                          Page \thepage\ of \pageref{lastpage}}}%
\renewcommand{\@evenhead}{}%
\renewcommand{\@oddfoot}{}%
\renewcommand{\@evenfoot}{}%
}%
\makeatother
\pagestyle{twolines}

\vspace{-10pt}
\begin{tabbing}
\phantom{References: }\= \\
To: \>Van\\
Subject: \>Sparse-matrix representation of interpolation coefficients\\
From: \>Van Snyder\\
%Reference: \\
\end{tabbing}

\parindent 0pt \parskip 6pt
\vspace{-20pt}

Interpolation coefficients in the full forward model are represented by a
sparse matrix, in which each row provides the coefficients to interpolate
from all coordinates or state-vector quantity elements to one point on the
path of integration, and each column provides the coefficients to
interpolate from one coordinate or state-vector element onto all points on
the path of integration.

\vspace*{0.1in}

\includegraphics{Sparse}

The number of rows is equal to the number of points along the path of
integration.  The number of columns is equal to the size of the basis,
which could be one-dimensional ($\phi$) or ($\zeta$), two-dimensional
($\zeta \times \phi$), three-dimensional ($\zeta\, \times$ QTM or $\zeta\,
\times$ lat $\times$ lon), or could include frequency as an additional
first dimension.

Each element contains its value ({\tt V}), row index ({\tt R}), and column
index ({\tt C}).  The value is never zero.

For interpolation from a one-dimensional horizontal ($\phi$) or vertical
($\zeta$) basis, there are at most two nonzeros per row.  For
interpolation from a rectangular two-dimensional ($\zeta \times \phi$ or
lat $\times$ lon) basis, there are at most four nonzeros per row.  For
interpolation from a QTM two-dimensional horizontal basis, there are at
most three nonzeros per row.  For a three-dimensional quantity with a QTM
horizontal basis, there are at most six nonzeros per row.  For a
three-dimensional quantity with a lat $\times$ lon horizontal basis,
there are at most eight nonzeros per row.  If the basis includes
frequency, the number of nonzeroes is doubled.  For higher-dimensional
bases, especially for ones with QTM or lat $\times$ lon horizontal bases,
the size of a non-sparse representation would be very much larger than
the sparse representation, and consist mostly of elements with zero
values.

To traverse a row, one starts with the value in the {\tt Rows} array,
which gives the index within the {\tt E} (for ``element'') array of the
last element in the row.  The {\tt NR} component of the last {\tt E}
element provides the index of the first element in the row, and the index
of the next element in the row of each other element.

To traverse a column, one starts with the value in the {\tt Cols} array,
which gives the index within the {\tt E} array of the last element in the
column.  The {\tt NC} component of the last {\tt E} element provides the
index of the first element of the column, and the index of the next
element in the column of each other element.

Each row or column list is circular, which allows both to append at the
end of the list, and traverse from the beginning, by representing only the
position of the last element in each row or column in the {\tt Rows} or
{\tt Cols} array.  If the first element were represent in the {\tt Rows}
or {\tt Cols} array, it would be necessary to traverse the list to find
the last element.

To interpolate a quantity from a state vector to a point on the path of
integration, compute the inner product of the row corresponding to the
point on the path, and the state vector quantity.

To compute derivatives for integration of variational equations along the
path of integration, traverse the column of the representation
corresponding to the state-vector element.  The row index in each element
indicates the path position where a nonzero interpolating coefficient
exists.

Higher-dimensional matrices of interpolation coefficients are constructed
by composing lower-dim\-en\-sion\-al matrices of coefficients, by
computing the outer products of corresponding rows.  The number of rows of
the constructed matrix is the same as the numbers of rows of the factors
(which must be the same).  The number of columns is the product of the
numbers of columns of the factors.  The element order of the array from
which values are to be interpolated is assumed to correspond to the first
factor followed by the second factor.  For example, temperatures and
mixing ratios are stored in arrays that have $\zeta \times \phi$ bases. 
Therefore, when a $\zeta \times \phi$ interpolator is formed it is
constructed as the product of a $\zeta$ interpolator and a $\phi$
interpolator, not vice-versa.  The extents of the dimensions of the bases
corresponding to the columns are preserved in the representation, so that
one can compute, e.g., separate $\zeta$ and $\phi$ subscripts from the
column index of an element, which is assumed to correspond to the
array-element order of a one-dimensional representation of the state
vector quantity.

\label{lastpage}
\vspace*{-0.1in} % Somehow, this causes lastpage to be defined
\end{document}

% $Id$

% $Log$
% Revision 1.1  2017/09/22 23:59:59  vsnyder
% Initial commit
%
