\documentclass[11pt]{article}
\usepackage{alltt}
\usepackage[fleqn]{amsmath}

\textwidth 6.5in
\oddsidemargin -0.25in
%\evensidemargin -0.5in
\topmargin -0.25in
\textheight 9in

\newcommand{\docname}{wvs-095}
\newcommand{\docdate}{20 May 2010}

\ifx\pdfoutput\undefined
  \pdfoutput=0
  \usepackage[hypertex,plainpages,hyperindex=true]{hyperref}
  \hypersetup{%
    hypertexnames=false%
  }
  % Specify the driver for the color package
  \ExecuteOptions{dvips}
  %\ExecuteOptions{xdvi}
\else
  \ifnum\pdfoutput>0
    \usepackage[pdftex,plainpages,hyperindex=true,pdfpagelabels]{hyperref}
    \hypersetup{%
      hypertexnames=false,%
      colorlinks=true,%
      linktocpage=true,%
    }
    % Specify the driver for the color package
    \ExecuteOptions{pdftex}
  \else
    \usepackage[hypertex,plainpages,hyperindex=true]{hyperref}
    \hypersetup{%
      hypertexnames=false%
    }
    % Specify the driver for the color package
    \ExecuteOptions{dvips}
    %\ExecuteOptions{xdvi}
  \fi
\fi

\hyperbaseurl{}
\newcommand\hr[1]{\href{#1.dvi}{dvi}, \href{#1.pdf}{pdf}}
\newcommand\h[1]{#1 (\hr{#1})}

\begin{document}

%\tracingcommands=1
\newlength{\hW} % heading box width
\newlength{\pW} % page number field width
\settowidth{\hW}{\bf\docname}
\settowidth{\pW}{Page \pageref{lastpage}\ of \pageref{lastpage}}
\ifdim \pW > \hW \setlength{\hW}{\pW} \fi
\makeatletter
\def\@biblabel#1{#1.}
\newcommand{\ps@twolines}{%
  \renewcommand{\@oddhead}{%
    \docdate\hfill\parbox[t]{\hW}{{\hfill\bf\docname}\newline
                          Page \thepage\ of \pageref{lastpage}}}%
\renewcommand{\@evenhead}{}%
\renewcommand{\@oddfoot}{}%
\renewcommand{\@evenfoot}{}%
}%
\makeatother
\pagestyle{twolines}

\newcommand{\TS}{T_\text{scat}}
\newcommand{\TSs}[1]{T_{\text{scat}_{#1}}}
\newcommand{\DB}{\overline{\Delta B}}
\newcommand{\MT}{\mathcal{T}}
\newcommand{\hMT}{\hat{\mathcal{T}}}

\vspace{-10pt}
\begin{tabbing}
\phantom{References: }\= \\
To: \>Bill, Van\\
Subject: \>Radiance and derivatives incorporating $\TSs{i}$\\
From: \>Van Snyder\\
Reference: \>\h{wvs-066}, \h{wvs-080}
\end{tabbing}

\parindent 0pt \parskip 6pt
\vspace{-10pt}

When $\TS$ is taken into account, the factor $\Delta B_i$ that appears in
the equation giving radiance at frequency $\nu_n$, \emph{viz.}

\begin{equation}
I_n = \sum \Delta B_{in} \MT_{in}\,,
\end{equation}

where $2 \Delta B_{in} = B_{i+1,n} - B_{i-1,n}$, is replaced by

\begin{equation}\begin{split}\label{two}
2 \DB_{in}
 =\,& ( 1 - \omega_{0_{i+1,n}} ) B_{i+1,n} + \omega_{0_{i+1,n}} \TSs{i+1,c} -
      ( 1 - \omega_{0_{i-1,n}} ) B_{i-1,n} - \omega_{0_{i-1,n}} \TSs{i-1,c} \\
 =\,& \left[
      ( 1 - \omega_{0_{i+1,n}} ) B_{i+1,n} -
      ( 1 - \omega_{0_{i-1,n}} ) B_{i-1,n} \right] +
      \left[
      \omega_{0_{i+1,n}} \TSs{i+1,c} - \omega_{0_{i-1,n}} \TSs{i-1,c}
      \right] \\
 =\,& 2 \Delta B_{in}^g + 2 \Delta B_{in}^s\,,
\end{split}\end{equation}

where $2\Delta B_{in}^g = ( 1 - \omega_{0_{i+1,n}} ) B_{i+1,n} -
                          ( 1 - \omega_{0_{i-1,n}} ) B_{i-1,n}$,
      $2\Delta B_{in}^s = \omega_{0_{i+1,n}} \TSs{i+1,c} -
                          \omega_{0_{i-1,n}} \TSs{i-1,c}$,
      $c$ is a channel number,

\begin{equation}\label{three}
\omega_0 = \frac{\beta_{c\_s}}{\beta_{e}} \text{, and }
\beta_e = \alpha_\text{gas} + \beta_{c\_s} + \beta_{c\_a} =
          \alpha_\text{gas} + \beta_{c\_e}\,.
\end{equation}

When frequency averaging is carried out, the radiance in channel $c$ is

\begin{equation}\begin{split}\label{four}
I_c = \sum_{n=1}^{n_f} \phi_{nc} \Delta \nu_{nc} I_n
    =\,& \sum_{n=1}^{n_f} \phi_{nc} \Delta \nu_{nc}
          \sum_{i=1}^{n_p} \Delta B_{in}^g \MT_{in}
       + \sum_{n=1}^{n_f} \phi_{nc} \Delta \nu_{nc}
          \sum_{i=1}^{n_p} \Delta B_{in}^s \MT_{in} \\
    =\,& \sum_{n=1}^{n_f} \phi_{nc} \Delta \nu_{nc}
          \sum_{i=1}^{n_p} \Delta B_{in}^g \MT_{in}
       + \sum_{i=1}^{n_p} \sum_{n=1}^{n_f}
          \phi_{nc} \Delta \nu_{nc} \Delta B_{in}^s \MT_{in}\,.
\end{split}\end{equation}

In the final term, substitute the definition of $\Delta B_{in}^s$ from
Equation (\ref{two}) and exchange the order of summation, giving

\begin{equation}\begin{split}\label{five}
\sum_{i=1}^{n_p} \sum_{n=1}^{n_f}
          \phi_{nc} \Delta \nu_{nc} \Delta B_{in}^s \MT_{in}
=\,&
\sum_{i=1}^{n_p} \sum_{n=1}^{n_f}
          \phi_{nc} \Delta \nu_{nc}
            ( \omega_{0_{i+1,n}} \TSs{i+1,c} 
            - \omega_{0_{i-1,n}}  \TSs{i-1,c} ) \MT_{in} \\
=\,&
\sum_{i=1}^{n_p} \TSs{i+1,c} \sum_{n=1}^{n_f}
 \phi_{nc} \Delta \nu_{nc} \omega_{0_{i+1,n}} \MT_{in}
-
\sum_{i=1}^{n_p} \TSs{i-1,c} \sum_{n=1}^{n_f}
 \phi_{nc} \Delta \nu_{nc} \omega_{0_{i-1,n}} \MT_{in}\,.
\end{split}\end{equation}

Defining the frequency-averaged values

\begin{equation}\label{w0hat}
\hMT_{i+1,c} =
\sum_{n=1}^{n_f} \phi_{nc} \Delta \nu_{nc} \omega_{0_{i+1,n}} \MT_{in}
\text{ and }
\hMT_{i-1,c} =
\sum_{n=1}^{n_f} \phi_{nc} \Delta \nu_{nc} \omega_{0_{i-1,n}} \MT_{in}
\,,
\end{equation}

Equation (\ref{four}) becomes

\begin{equation}\label{six}
I_c = \sum_{n=1}^{n_f} \phi_{nc} \Delta \nu_{nc}
       \sum_{i=1}^{n_p} \Delta B_{in}^g \MT_{in} +
      \sum_{i=1}^{n_p} (\hMT_{i+1,c} \TSs{i+1,c} -
                        \hMT_{i-1,c} \TSs{i-1,c})\,.
\end{equation}

From Equation (\ref{six})

\begin{equation}\begin{split}
\frac{\partial I_c}{\partial x} =\,&
 \sum_{n=1}^{n_f} \phi_{nc} \Delta \nu_{nc}
  \sum_{i=1}^{n_p} \left( \frac{\partial \Delta B_{in}^g}{\partial x} \MT_{in}
                        + \Delta B_{in}^g \frac{\partial \MT_{in}}{\partial x}
                   \right) \\
+\,&
  \sum_{i=1}^{n_p} \left( \frac{\partial \hMT_{i+1,c}}{\partial x}
                          \TSs{i+1,c} +
                          \hMT_{i+1,c}
                           \frac{\partial \TSs{i+1,c}}{\partial x}
                        - \frac{\partial \hMT_{i+1,c}}{\partial x}
                          \TSs{i-1,c} -
                          \hMT_{i-1,c}
                           \frac{\partial \TSs{i-1,c}}{\partial x}
                   \right)\,.
\end{split}\end{equation}

From Equation (\ref{w0hat})

\begin{equation}
\frac{\partial \hMT_{ic}}{\partial x} =
 \sum_{n=1}^{n_f} \phi_{nc} \Delta \nu_{nc}
  \left( \frac{\partial \omega_{0_{in}}}{\partial x} \MT_{in} +
         \omega_{0_{in}} \frac{\partial \MT_{in}}{\partial x}
  \right).
\end{equation}

From Equation (\ref{three})

\begin{equation}
\frac{\partial \omega_0}{\partial x} =
 \frac1{\beta_e^2} \left( \frac{\partial \beta_{c\_s}}{\partial x} -
                          \frac{\partial \beta_e}{\partial x} \right)
=
 \frac1{\beta_e^2} \left( \frac{\partial \beta_{c\_s}}{\partial x} -
                          \frac{\partial \beta_{c\_e}}{\partial x} -
                          \frac{\partial \alpha_\text{gas}}{\partial x}
                   \right) \,.
\end{equation}

$\alpha_\text{gas}$ depends upon frequency, temperature, and atmospheric
constituent mixing ratios.

$\beta_{c\_s}$ and $\beta_{c\_e}$ depend upon frequency, temperature and IWC.

$B$ depends upon frequency and temperature.

$\TS$ depends upon frequency, temperature, IWC and whatever molecules are
specified when it is computed.

When frequency averaging
\label{lastpage}
\end{document}

% $Id$

% $Log$
