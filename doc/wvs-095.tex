\documentclass[11pt]{article}
\usepackage{alltt}
\usepackage[fleqn]{amsmath}
\usepackage{floatflt}
\usepackage{graphicx}

\textwidth 6.5in
\oddsidemargin -0.25in
%\evensidemargin -0.5in
\topmargin -0.5in
\textheight 9in

\newcommand{\docname}{wvs-095r4}
\newcommand{\docdate}{8 June 2010}

\ifx\pdfoutput\undefined
  \pdfoutput=0
  \usepackage[hypertex,plainpages,hyperindex=true]{hyperref}
  \hypersetup{%
    hypertexnames=false%
  }
  % Specify the driver for the color package
  \ExecuteOptions{dvips}
  %\ExecuteOptions{xdvi}
\else
  \ifnum\pdfoutput>0
    \usepackage[pdftex,plainpages,hyperindex=true,pdfpagelabels]{hyperref}
    \hypersetup{%
      hypertexnames=false,%
      colorlinks=true,%
      linktocpage=true,%
    }
    % Specify the driver for the color package
    \ExecuteOptions{pdftex}
  \else
    \usepackage[hypertex,plainpages,hyperindex=true]{hyperref}
    \hypersetup{%
      hypertexnames=false%
    }
    % Specify the driver for the color package
    \ExecuteOptions{dvips}
    %\ExecuteOptions{xdvi}
  \fi
\fi

\hyperbaseurl{}
\newcommand\hr[1]{\href{#1.dvi}{dvi}, \href{#1.pdf}{pdf}}
\newcommand\h[1]{#1 (\hr{#1})}

\begin{document}

%\tracingcommands=1
\newlength{\hW} % heading box width
\newlength{\pW} % page number field width
\settowidth{\hW}{\bf\docname}
\settowidth{\pW}{Page \pageref{lastpage}\ of \pageref{lastpage}}
\ifdim \pW > \hW \setlength{\hW}{\pW} \fi
\makeatletter
\def\@biblabel#1{#1.}
\newcommand{\ps@twolines}{%
  \renewcommand{\@oddhead}{%
    \docdate\hfill\parbox[t]{\hW}{{\hfill\bf\docname}\newline
                          Page \thepage\ of \pageref{lastpage}}}%
\renewcommand{\@evenhead}{}%
\renewcommand{\@oddfoot}{}%
\renewcommand{\@evenfoot}{}%
}%
\makeatother
\pagestyle{twolines}

\newcommand{\TS}{T_\text{scat}}
\newcommand{\TSs}[1]{T_{\text{scat}_{#1}}}
\newcommand{\DB}{\Delta B}
\newcommand{\oDB}{\overline{\DB}}
\newcommand{\MT}{\mathcal{T}}
\newcommand{\hMT}{\mathcal{T}^s}
\newcommand{\IF}[1]{\,\mathcal{A}_n\!\left(#1\right)} % Interpolation Function

\vspace{-10pt}
\begin{tabbing}
\phantom{References: }\= \\
To: \>Bill, Nathaniel, Van\\
Subject: \>Radiance and derivatives incorporating $\TS$\\
From: \>Van Snyder\\
Reference: \>\h{wvs-027}, \h{wvs-066}, \h{wvs-080}
\end{tabbing}

\parindent 0pt \parskip 6pt
\vspace{-10pt}

When $\TS$ is taken into account, the factor $\DB_{ij} =
\frac12\left(B_{i+1,j} - B_{i-1,j}\right)$ that appears in
the equation giving radiance at frequency $\nu_j \in \{\nu_j, j = 1,
\dots, n_f\}$, \emph{viz.}

\begin{equation}\label{one}
I_j = \int \frac{\partial B_j(s)}{\partial s} \MT_j(s)\,\text{d}s
 \approx \sum_{i=1}^{n_p} \DB_{ij} \MT_{ij}
\end{equation}

is replaced by

\begin{equation}\begin{split}\label{two}
\oDB_{ij}
 =\,& \frac12\left(
      ( 1 - \omega_{0_{i+1,j}} ) B_{i+1,j} + \omega_{0_{i+1,j}} \TSs{i+1,c} -
      ( 1 - \omega_{0_{i-1,j}} ) B_{i-1,j} - \omega_{0_{i-1,j}} \TSs{i-1,c}
  \right) \\
 =\,& \frac12\left(
  \left[
      ( 1 - \omega_{0_{i+1,j}} ) B_{i+1,j} -
      ( 1 - \omega_{0_{i-1,j}} ) B_{i-1,j} \right] +
      \left[
      \omega_{0_{i+1,j}} \TSs{i+1,c} - \omega_{0_{i-1,j}} \TSs{i-1,c}
      \right]  \right) \\
 =\,& \DB_{ij}^g + \DB_{ij}^s\,,
\end{split}\end{equation}

where $c$ is a channel number,

\begin{equation}\label{three}
\begin{array}{rll}
\DB_{ij}^g \,= & \frac12\left(( 1 - \omega_{0_{i+1,j}} ) B_{i+1,j} -
                ( 1 - \omega_{0_{i-1,j}} ) B_{i-1,j}\right) &
                \text{``gas,''} \\
\DB_{ij}^s \,= & \frac12\left( \omega_{0_{i+1,j}} \TSs{i+1,c} -
                 \omega_{0_{i-1,j}} \TSs{i-1,c}\right) &
                 \text{``scattering,''} \\
\omega_{0_{ij}} \,= & \frac{\beta_{{c\_s}_i}}{\beta_{e_{ij}}}
 \text{, and}\\
\beta_{e_{ij}} \,= & \alpha_{\text{gas}_{ij}} + \beta_{{c\_s}_i} +
 \beta_{{c\_a}_i} = \alpha_{\text{gas}_{ij}} + \beta_{{c\_e}_i}\,.
\end{array}
\end{equation}

We write $\beta_{{c\_s}_i}$ and $\beta_{{c\_e}_i}$ instead of
$\beta_{{c\_s}_{ij}}$ and $\beta_{{c\_e}_{ij}}$ because $\beta_{c\_s}$
and $\beta_{c\_e}$ change very slowly with frequency but depend strongly
upon ice water content (IWC) and temperature, so we can simply evaluate
the former for IWC$_i$ and T$_i$ using the tables of their dependence
upon IWC and T for the frequency at which they are tabulated that is
nearest to $\nu_j$.

In order to carry out frequency averaging, radiances are needed at
frequencies $\{\nu_{nc}\}$ for which the filter response function
$\phi_c(\nu_n) = \phi_{nc}$ is tabulated.  These frequencies are not the
same as $\{\nu_j\}$ for which Equation (\ref{one}) is evaluated.  The
radiances $I_j$ developed in Equation (\ref{one}) (and their derivatives)
are fitted with a spline in frequency, and that spline is then evaluated
at the frequencies $\{\nu_{nc}\}$, giving a new set of radiances
$\!\IF{I_j}$ at those frequencies, where the function $\!\IF{\cdot}$
interpolates from $\{\nu_j\}$ to $\{\nu_{nc}\}$.

Assuming rectangular quadrature (Simpson's quadrature is actually used),
when frequency averaging is carried out, the radiance in channel $c$ is

\begin{equation}\label{four}
I_c = \sum_{n=1}^{n_c} \phi_{nc} \Delta \nu_{nc} \IF{I_j}
       = \sum_{n=1}^{n_c} \phi_{nc} \Delta \nu_{nc}
           \IF{\sum_{i=1}^{n_p} \DB_{ij}^g \MT_{ij}} +
         \sum_{n=1}^{n_c} \phi_{nc} \Delta \nu_{nc}
           \IF{\sum_{i=1}^{n_p} \DB_{ij}^s \MT_{ij}}
\end{equation}

In the final term, substitute the definition of $\DB_{ij}^s$ from
Equation (\ref{two}), observe that $\TSs{i\pm1,c}$ is a channel quantity
and therefore unaffected by $\!\IF{\cdot}$, and exchange the order
of summation, giving

\begin{equation}\begin{split}\label{five}
\sum_{n=1}^{n_c} \phi_{nc} \Delta \nu_{nc}
  \IF{\sum_{i=1}^{n_p} \DB_{ij}^s \MT_{ij}}=\,&
\frac12\sum_{i=1}^{n_p} \sum_{n=1}^{n_c}
          \phi_{nc} \Delta \nu_{nc}
          \IF{( \omega_{0_{i+1,j}} \TSs{i+1,c} 
            - \omega_{0_{i-1,j}} \TSs{i-1,c} ) \MT_{ij}} \\
=\,&
\frac12\sum_{i=1}^{n_p}
 \left( \TSs{i+1,c} \sum_{n=1}^{n_c}
 \phi_{nc} \Delta \nu_{nc} \IF{\omega_{0_{i+1,j}} \MT_{ij}} \right.
- \\
\,&
\phantom{\frac12\sum_{i=1}^{n_p}\left(\right.}
 \left. \TSs{i-1,c} \sum_{n=1}^{n_c}
 \phi_{nc} \Delta \nu_{nc} \IF{\omega_{0_{i-1,j}} \MT_{ij}} \right)\,.
\end{split}\end{equation}

Defining the frequency-averaged values

\begin{equation}\label{six}
\hMT_{i\pm1,c} =
\sum_{n=1}^{n_c} \phi_{nc} \Delta \nu_{nc} \IF{\omega_{0_{i\pm1,j}} \MT_{ij}}
\,,
\end{equation}

Equation (\ref{four}) becomes

\begin{equation}\label{seven}
I_c = \sum_{n=1}^{n_c} \phi_{nc} \Delta \nu_{nc}
       \IF{\sum_{i=1}^{n_p} \DB_{ij}^g \MT_{ij}} +
      \frac12\left(
      \sum_{i=1}^{n_p} \hMT_{i+1,c} \TSs{i+1,c} -
                       \hMT_{i-1,c} \TSs{i-1,c}
      \right)\,.
\end{equation}

That is, the ``gas'' terms ($\DB_{ij}^g \MT_{ij}$) are frequency averaged
\emph{after} path integration, while the $\omega_{0_{i\pm1,j}} \MT_{ij}$
factors in the ``scattering'' terms are frequency averaged \emph{before}
path integration to give $\hMT_{i\pm1,c}$, then multiplied by the
$\TSs{i\pm1,c}$ factors (which are channel averages), then their products
are integrated along the path.  This is necessary because path
integration is done (by the {\tt Frequency\_Loop} subroutine of the full
forward model) for $\{\nu_j\}$, some of which appear in more than one
channel.  Therefore, in computing a product $\omega_{0_{i\pm1,j}}
\MT_{ij} \TSs{i+1,c}$, we do not know what $c$ to use for a particular
$j$.

Derivatives are treated similarly.  From Equation (\ref{seven})

\begin{equation}\begin{split}\label{eight}
\frac{\partial I_c}{\partial x} =\,&
 \sum_{n=1}^{n_c} \phi_{nc} \Delta \nu_{nc}
  \IF{ \sum_{i=1}^{n_p}
   \frac{\partial \DB_{ij}^g}{\partial x} \MT_{ij}
                        + \DB_{ij}^g \frac{\partial \MT_{ij}}{\partial x}
      } \\
+\,&
  \frac12
  \sum_{i=1}^{n_p} \left( \frac{\partial \hMT_{i+1,c}}{\partial x}
                          \TSs{i+1,c} +
                          \hMT_{i+1,c}
                           \frac{\partial \TSs{i+1,c}}{\partial x}
                        - \frac{\partial \hMT_{i+1,c}}{\partial x}
                          \TSs{i-1,c} -
                          \hMT_{i-1,c}
                           \frac{\partial \TSs{i-1,c}}{\partial x}
                   \right)
\end{split}\end{equation}

where $x$ is either a mixing ratio of some specie, or temperature, at
some point in the atmosphere.

\begin{minipage}{4.75in}
Assume the $k^{\text{th}}$ mixing ratio is represented on the red grid
and temperature on the blue grid, and we wish to have both at $l$, a point
on the integration path.

\vskip 3pt
Assume linear interpolation coefficients $\eta^k_{ml}$ and $\eta^T_{\mu
l}$ to interpolate the $k^{\text{th}}$ mixing ratio from points $m$ and
temperature from points $\mu$ to $l$, respectively.  Then $f^k_l = \sum_m
\eta^k_{ml} f^k_m$ and $T_l = \sum_\mu \eta^T_{\mu l} T_\mu$;
$m$ and $\mu$ are simply indices that count the points in the grid, not
co\"{o}rdinates.
\end{minipage}
\hspace*{0.1in}
\parbox{1.5in}{
\includegraphics[scale=1]{wvs-095-eta}
}

The coefficients $\eta^k_{ml}$ and $\eta^T_{\mu l}$ are actually products
of coefficients that interpolate in $\zeta$ and $\phi$, e.g.,
$\eta^k_{ml} = \eta^k(\zeta_m,\zeta_l) \eta^k(\phi_m,\phi_l)$, where
$\eta^k(\zeta_m,\zeta_l) = (\zeta^k_{m+1}-\zeta_l) /
(\zeta^k_{m+1}-\zeta^k_m)$ if $\zeta^k_m < \zeta \leq \zeta^k_{m+1}$ and
$(\zeta_l-\zeta^k_{m-1}) / (\zeta^k_m-\zeta^k_{m-1})$ if $\zeta^k_{m-1} <
\zeta \leq \zeta^k_m$.  For each $k$ and $m$ at most four of
$\{\eta^k_{ml}\}$ are nonzero, and for each $\mu$ at most four of
$\{\eta^T_{\mu l}\}$ are nonzero.

$\MT_{ij} = \exp(-\delta_{ij}) =
\exp\left(-\int_0^{s_i} \alpha_j(s)\, \text{d} s \right) =
\exp(-\sum_{k=1}^{n_s} \delta^k_{ij})$, where

\begin{equation}\label{nine}
\delta^k_{ij}
= \int_0^{s_i} \beta^k_j(s)\, \text{d} s
\approx
  \sum_{l=1}^i f^k_l \beta^k_{lj}(T_l,\nu_j)
  \Delta s_l
= \sum_{l=1}^i \sum_m \eta^k_{ml} f^k_m
  \beta^k_{lj}\left(\sum_\mu \eta^T_{\mu l} T_\mu,\nu_j\right)
  \Delta s_l
\end{equation}

in which $i$ and $l$ index points on the integration path, and $m$ and
$\mu$ index points in the solution grids.

From Equation (\ref{nine}),

\begin{equation}\begin{split}\label{ten}
\frac{\partial \MT_{ij}}{\partial f^k_m} =\,&
 - \MT_{ij} \frac{\partial \delta^k_{ij}}{\partial f^k_l}
            \frac{\partial f^k_l}{\partial f^k_m} =
 - \MT_{ij} \sum_{l=1}^{n_p} \eta^k_{ml} \beta^k_{lj} \Delta s_l
 \text{ and }\\
\frac{\partial \MT_{ij}}{\partial T_\mu} =\,&
 - \MT_{ij} \frac{\partial \delta_{ij}}{\partial T_l}
            \frac{\partial T_l}{\partial T_\mu} =
 - \MT_{ij} \sum_l \eta^T_{\mu l}
    \frac{\partial \alpha_{lj}}{\partial T_l} \Delta s_l =
 - \MT_{ij} \sum_{k,l} f^k_l \eta^T_{\mu l}
    \frac{\partial \beta^k_{lj}}{\partial T_l} \Delta s_l
\,.
\end{split}\end{equation}

The full forward model operates by integrating along the path first,
producing $I_j$ and derivatives for frequency $\nu_j$.  After doing this
for all $\{\nu_j\}$, it interpolates $I_j$ to $\!\IF{I_j}$ and does the
frequency averaging, and similarly for derivatives.  That is, the order
of summations is as in Equation (\ref{four}).

From Equations (\ref{three}), (\ref{six}) and (\ref{ten}),

\begin{equation}\begin{split}\label{eleven}
\frac{\partial \hMT_{i\pm1,c}}{\partial f^k_m} =\,&
 \sum_{n=1}^{n_c} \phi_{nc} \Delta \nu_{nc}
  \IF{\MT_{ij}
   \sum_{l=1}^{n_p} \eta^k_{ml}
    \left[
        \frac{\partial \omega_{0_{i\pm1,j}}} {\partial f^k_l} -
         \Delta s_l \omega_{0_{i\pm1,j}}
          \beta^k_{lj}
    \right]}\text{ and}\\
\frac{\partial \hMT_{i\pm1,c}}{\partial T_\mu} =\,&
 \sum_{n=1}^{n_c} \phi_{nc} \Delta \nu_{nc}
  \IF{\MT_{ij}
   \sum_{l=1}^{n_p} \eta^T_{\mu l}
     \left[
        \frac{\partial \omega_{0_{i\pm1,j}}}{\partial T_l} -
         \Delta s_l \omega_{0_{i\pm1,j}}
           \sum_k f^k_l \frac{\partial \beta^k_{lj}}{\partial T_l}
      \right]} .
\end{split}\end{equation}

From Equation (\ref{three})

\begin{equation}\begin{split}\label{twelve}
\frac{\partial \omega_{0_{ij}}}{\partial f^k_l} =\,&
 \frac1{\beta_{e_{ij}}}   \frac{\partial \beta_{{c\_s}_i}}{\partial f^k_l} -
 \frac1{\beta_{e_{ij}}^2} \frac{\partial \beta_{e_{ij}}}{\partial f^k_l}
=
 \frac1{\beta_{e_{ij}}}   \frac{\partial \beta_{{c\_s}_i}}{\partial f^k_l} -
 \frac1{\beta_{e_{ij}}^2}
                   \left( \frac{\partial \beta_{{c\_e}_i}}{\partial f^k_l} +
                          \frac{\partial \alpha_{\text{gas}_{ij}}}{\partial f^k_l}
                   \right) \\
=\,&
 \frac1{\beta_{e_{lj}}}   \frac{\partial \beta_{{c\_s}_l}}{\partial f^k_l} -
 \frac1{\beta_{e_{lj}}^2}
                   \left( \frac{\partial \beta_{{c\_e}_l}}{\partial f^k_l} +
                          \beta^k_{lj}
                   \right) \text{ and }
 \frac{\partial \omega_{0_{ij}}}{\partial f^k_m} =
 \sum_{l=1}^{n_p} \eta^k_{ml} \frac{\partial \omega_{0_{ij}}}{\partial f^k_l}\,,
\\
\frac{\partial \omega_{0_{ij}}}{\partial T_l} =\,&
 \frac1{\beta_{e_{ij}}}   \frac{\partial \beta_{{c\_s}_i}}{\partial T_l} -
 \frac1{\beta_{e_{ij}}^2}
                   \left( \frac{\partial \beta_{{c\_e}_i}}{\partial T_l} -
                          \frac{\partial\alpha_{\text{gas}_{ij}}}{\partial T_l}
                   \right)\\
=\,&
 \frac1{\beta_{e_{lj}}}   \frac{\partial \beta_{{c\_s}_l}}{\partial T_l} -
 \frac1{\beta_{e_{lj}}^2}
                   \left( \frac{\partial \beta_{{c\_e}_l}}{\partial T_l} -
                          \frac{\partial\alpha_{\text{gas}_{lj}}}{\partial T_l}
                   \right) \text{ and }
 \frac{\partial \omega_{0_{ij}}}{\partial T_\mu} =
   \sum_{l=1}^{n_p} \eta^T_{\mu l}
    \frac{\partial \omega_{0_{ij}}}{\partial T_l}
 \,.
\end{split}\end{equation}

$\alpha_\text{gas}$ depends upon temperature and atmospheric constituent
mixing ratios along the path, and frequency.

$\beta_{c\_s}$ and $\beta_{c\_e}$ depend upon temperature and IWC along
the path, and frequency.

$B$ depends upon temperature along the path, and frequency.

$\TS$ depends upon temperature, IWC and whatever molecules are specified
when it is computed, throughout the atmosphere, and frequency.  Therefore,
the second term in Equation (\ref{eight}) will produce nonzeros in the
Jacobian for every element in the state vector upon which $\TS$ depends,
not just those near the path.

There are several possibilities to perform the frequency averaging as
described in Equation (\ref{eleven}).  One method is to create
three-dimensional arrays to store $\left[ \frac{\partial
\omega_{0_{i\pm1,j}}}{\partial f^k_l} - \omega_{0_{i\pm1,j}}
\frac{\partial \delta_{ij}}{\partial f^k_l} \right] \MT_{ij}$ and
similarly for temperature derivatives, having extents $n_p^2 \times n_s
\times n_f$ and $n_p^2 \times n_f$, and to provide data structures that
keep track of nonzero elements.  Assuming $n_p\approx 150,\, n_s \approx
20$, and $n_f \approx 500$, these arrays would have $\approx
225\times10^6$ and $\approx 10^7$ elements, so this is clearly
impractical.

The forward model already stores $\eta^k_{il}$ and $f^k_l$, and
$\beta^k_{lj}$, $\frac{\partial \beta^k_{lj}}{\partial T_l}$ and
$\MT_{ij}$ for each frequency.  From Equation (\ref{twelve}),
$\frac{\partial \omega_{0_{ij}}}{\partial f^k_l}$ is only nonzero when $l
= i$. From Equations (\ref{three}) and (\ref{twelve}), $\omega_{0_{ij}}$ and
$\frac{\partial \omega_{0_{ij}}}{\partial f^k_l}$ depend upon quantities
the forward model already computes, plus $\beta_{c\_s_i}$,
$\beta_{c\_e_i}$ and their derivatives with respect to temperature and
IWC.

From Equation (\ref{ten}), $\frac{\partial \delta_{ij}}{\partial
f^k_l} \MT_{ij}$ and $\frac{\partial \delta_{ij}}{\partial T_i} \MT_{ij}$
are outer products $\eta^k_{il} \beta^k_{lj} \MT_{ij}$ and $\eta^k_{il} f^k_l
\frac{\partial\beta^k_{lj}}{\partial T_l} \MT_{ij}$, which are all
lower-dimensional quantities.  Arrays to represent the factors of the
outer products for all frequencies would have tractable extents $n_p
\times n_s \times n_f \approx 1.5\times10^6$ and $n_p \times n_f \approx
75\times10^3$.

Changes to storage in the forward model would be:

\begin{itemize}

\item Provide new arrays to represent
  $\beta_{c\_s_i}$, $\beta_{c\_e_i}$, and their derivatives with respect
  to temperature and IWC, and

\item Expand the arrays that represent $\beta^k_{ij}$, $\frac{\partial
  \beta^k_{ij}}{\partial T_i}$, and $\MT_{ij}$ within {\tt
  Frequency\_Loop} by adding a frequency dimension, and allocate them
  outside that subroutine.

\end{itemize}

Equations (\ref{seven}) and (\ref{eight}) could then be evaluated in
stages:

\begin{enumerate}

\item Evaluate $\beta_{c\_s_i}$, $\beta_{c\_e_i}$, and their derivatives
  with respect to $T_i$ and IWC$_i$, independently of frequency,

\item Evaluate the first term in each equation using a small modification
  of the {\tt Frequency\_Loop} subroutine to include the $\omega_0$ terms
  in $\Delta B^s_{ij}$, saving quantities enumerated above necessary to
  compute $\frac{\partial \omega_{0_{ij}}}{\partial f^k_i}$,
  $\frac{\partial \omega_{0_{ij}}}{\partial T_i}$, $\frac{\partial
  \delta_{ij}}{\partial f^k_l}$ and $\frac{\partial \delta_{ij}}{\partial
  T_i}$,

\item Compute $\frac{\partial \omega_{0_{ij}}}{\partial f^k_i}$,
  $\frac{\partial \omega_{0_{ij}}}{\partial T_i}$, $\frac{\partial
  \delta_{ij}}{\partial f^k_l}$ and $\frac{\partial \delta_{ij}}{\partial
  T_i}$, and frequency average them separately at each point along the
  path, similarly to how {\tt Inc\_Rad\_Path} is handled for PFA (see
  wvs-027), and

\item Evaluate the second term in a second invocation of {\tt
  Frequency\_Loop}, modified to use the linear forward model to compute
  the $\TSs{ic}$, $\frac{\partial \TSs{ic}}{\partial f^k_i}$ and
  $\frac{\partial \TSs{ic}}{\partial T_i}$ terms in Equations
  (\ref{seven}) and (\ref{eight}).  This is also similar to how PFA
  computations are done.

\end{enumerate}

A second method is to compute coefficients to interpolate linearly from
$\{\nu_j\}$ to $\{\nu_{nc}\}$, combine those coefficients with
$\phi_{nc}$ and $\Delta \nu_{nc}$, and use them to frequency average
$\omega_{0_{i\pm1,j}} \MT_{i\pm1,j}$ as they are calculated during path
integration.

In linear interpolation form we can write

\begin{equation}
\IF{\omega_{0_{i\pm1,j}}\MT_{ij}} =
 \sum_{j=1}^{n_f} \eta_{nj} \omega_{0_{i\pm1,j}} \MT_{ij}
\end{equation}

where $n_f$ is the number of frequencies $\{\nu_j,\, j = 1, \cdots,
n_f\}$ at which Equation (\ref{one}) is evaluated and $\eta_{nj} =
\frac{\nu_{j+1}-\nu_n}{\nu_{j+1}-\nu_j}$ is the linear interpolating
coefficient from those frequencies to the filter function frequencies
$\{\nu_{nc},\, n = 1, \cdots, n_c\}$.  To avoid carrying these quantities
through the path integration (because $n_c << n_f$) we can frequency
average $\!\IF{\omega_{0_{i\pm1,j}}\MT_{ij}}$ to get, by exchanging the
order of summation,

\begin{equation}\begin{split}\label{fourteen}
\hMT_{i\pm1,c} =\,& \sum_{n=1}^{n_c} \phi_{nc} \Delta \nu_{nc}
 \IF{\omega_{0_{i\pm1,j}}\MT_{ij}}
 = \sum_{n=1}^{n_c} \phi_{nc} \Delta \nu_{nc}
    \sum_{j=1}^{n_f} \eta_{nj} \omega_{0_{i\pm1,j}} \MT_{ij} \\
 =\,& \sum_{j=1}^{n_f} \omega_{0_{i\pm1,j}} \MT_{ij}
    \sum_{n=1}^{n_c} \phi_{nc} \Delta \nu_{nc} \eta_{nj}
 = \sum_{j=1}^{n_f} h_{jc} \omega_{0_{i\pm1,j}} \MT_{ij}\,,
\end{split}\end{equation}

where

\begin{equation}
h_{jc} = \sum_{n=1}^{n_c} \phi_{nc} \Delta \nu_{nc} \eta_{nj}\,.
\end{equation}

Before the {\tt Frequency\_Loop} subroutine in the full forward model is
invoked, compute $h_{jc}$ from the pointing frequency grid and the filter
function's frequency grids for all channels.  During path integration in
{\tt Frequency\_Loop}, evaluate Equation (\ref{fourteen}) to produce the
channel-averaged quantity $\hMT_{i\pm1,c}$.  This method also applies to
evaluating Equation (\ref{eleven}).

Because of the use of linear interpolation between $\{\nu_j\}$ and
$\{\nu_{nc}\}$, this scheme is somewhat less accurate than the first
method.

A third method is to divide references to {\tt Frequency\_Loop} into sets
of frequencies that appear in each channel and compute the
frequency-averaged channel radiance for each channel after invocation of
{\tt Frequency\_Loop} for the subset of $\{\nu_j\}$ that appear within
that channel.  This would allow forming products $\omega_{0_{i\pm1,j}}
\MT_{ij} \TSs{i+1,c}$ during path integration, and would not require
frequency averaging during path integration.

Some frequencies appear in more than one channel.  Depending upon the
number of such frequencies, either the path integrations for those
frequencies can be repeated, or a data structure can be developed to
keep track of results for frequencies that appear in more than one
channel and reuse those results instead of repeating the path
integrations.

A fourth method is to interpolate $\TSs{i\pm1,c}$ and its derivatives
from adjacent channel centers to the element of $\{\nu_j\}$ for which
{\tt Frequency\_Loop} is invoked.  In principle this requires invoking
the linear forward model twice per path point and frequency, but if
storage permits it might be possible to invoke the linear forward model
for every path point for the two channels surrounding a subset of
$\{\nu_j\}$, and interpolate in those retained results.  The amount of
storage necessary for $\TSs{i\pm1,c}$ for two channels is $2 n_p$.  The
amount of storage necessary for $\frac{\partial \TSs{i\pm1,c}}{\partial
f^k_l}$ and $\frac{\partial \TSs{i\pm1,c}}{\partial T}$ is $2 n_p^2
(\hat{n}_s+1)$, where $\hat{n}_s$ is the number of species upon which
$\TSs{i\pm1,c}$ depends, i.e., the number of values of $k$.  Since $l \leq
i$ this could be halved at some expense to clarity of the software.

\label{lastpage}
\end{document}

% $Id$

% $Log$
% Revision 1.4  2010/06/04 23:17:05  vsnyder
% Correct date, change file name of graphic
%
% Revision 1.3  2010/06/04 23:14:19  vsnyder
% Sunstantial revision of radiative transfer, added fourth method
%
% Revision 1.2  2010/06/03 00:08:09  vsnyder
% Three ideas for combining channel quantities with single-frequency quantities
%
% Revision 1.1  2010/05/20 23:49:47  vsnyder
% Initial commit
%
