\documentclass[11pt]{article}
\usepackage{alltt}
\usepackage[fleqn]{amsmath}
\usepackage{floatflt}
\usepackage{graphicx}
\usepackage{longtable}

\textwidth 6.5in
\oddsidemargin -0.25in
%\evensidemargin -0.5in
\topmargin -0.5in
\textheight 9in

\newcommand{\docname}{wvs-095r13}
\newcommand{\docdate}{27 July 2011}

\ifx\pdfoutput\undefined
  \pdfoutput=0
  \usepackage[hypertex,plainpages,hyperindex=true]{hyperref}
  \hypersetup{%
    hypertexnames=false%
  }
  % Specify the driver for the color package
  \ExecuteOptions{dvips}
  %\ExecuteOptions{xdvi}
\else
  \ifnum\pdfoutput>0
    \usepackage[pdftex,plainpages,hyperindex=true,pdfpagelabels]{hyperref}
    \hypersetup{%
      hypertexnames=false,%
      colorlinks=true,%
      linktocpage=true,%
    }
    % Specify the driver for the color package
    \ExecuteOptions{pdftex}
  \else
    \usepackage[hypertex,plainpages,hyperindex=true]{hyperref}
    \hypersetup{%
      hypertexnames=false%
    }
    % Specify the driver for the color package
    \ExecuteOptions{dvips}
    %\ExecuteOptions{xdvi}
  \fi
\fi

\hyperbaseurl{}
\newcommand\hr[1]{\href{#1.dvi}{dvi}, \href{#1.pdf}{pdf}}
\newcommand\h[1]{#1 (\hr{#1})}

\begin{document}

%\tracingcommands=1
\newlength{\hW} % heading box width
\newlength{\pW} % page number field width
\settowidth{\hW}{\bf\docname}
\settowidth{\pW}{Page \pageref{lastpage}\ of \pageref{lastpage}}
\ifdim \pW > \hW \setlength{\hW}{\pW} \fi
\makeatletter
\def\@biblabel#1{#1.}
\newcommand{\ps@twolines}{%
  \renewcommand{\@oddhead}{%
    \docdate\hfill\parbox[t]{\hW}{{\hfill\bf\docname}\newline
                          Page \thepage\ of \pageref{lastpage}}}%
\renewcommand{\@evenhead}{}%
\renewcommand{\@oddfoot}{}%
\renewcommand{\@evenfoot}{}%
}%
\makeatother
\pagestyle{twolines}

\newcommand{\TS}{T_\text{scat}}
\newcommand{\TSs}[1]{T_{\text{scat}_{#1}}}
\newcommand{\DB}{\Delta B}
\newcommand{\oDB}{\overline{\DB}}
\newcommand{\MT}{\mathcal{T}}
\newcommand{\hMT}{\MT^s}
\newcommand{\IF}[1]{\,\mathcal{A}_n\!\left(#1\right)} % Interpolation Function

\vspace{-10pt}
\begin{tabbing}
\phantom{References: }\= \\
To: \>Bill, Nathaniel, Van\\
Subject: \>Radiance and derivatives incorporating $\TS$\\
From: \>Van Snyder\\
Reference: \>\h{wvs-027}, \h{wvs-066}, \h{wvs-080}, JPL D-18130
\end{tabbing}

\parindent 0pt \parskip 6pt
\vspace{-10pt}

\section{Analysis}

\subsection{General}

When $\TS$ is taken into account, the factor $\DB_{ij} =
\frac12\left(B_{i+1,j} - B_{i-1,j}\right)$ that appears in
Equation (10.1) in the 19 August 2004 ATBD, JPL D-18130 (ignoring the special
forms at the ends of the path for now), giving radiance at frequency $\nu_j
\in \{\nu_j | j = 1, \dots, n_f\}$, \emph{viz.}

\begin{equation}\label{one}
I_j(s) = \MT_{0j} ( I(s_0) - B(s_0) ) + B(s) -
 \int_{s_0}^{s} \frac{\partial B_j(\sigma)}{\partial \sigma}
  \MT_j(\sigma)\,\text{d}\sigma
   \approx \sum_{i=1}^{n_p} \DB_{ij} \MT_{ij}
\end{equation}

is replaced (except at the ends of the path, where special treatment is
needed) by

\begin{equation}\begin{split}\label{two}
\oDB_{ij}
 =\,& \frac12\left(
      ( 1 - \omega_{0_{i+1,j}} ) B_{i+1,j} + \omega_{0_{i+1,j}} \TSs{i+1,c} -
      ( 1 - \omega_{0_{i-1,j}} ) B_{i-1,j} - \omega_{0_{i-1,j}} \TSs{i-1,c}
  \right) \\
 =\,& \frac12\left(
  \left[
      ( 1 - \omega_{0_{i+1,j}} ) B_{i+1,j} -
      ( 1 - \omega_{0_{i-1,j}} ) B_{i-1,j} \right] +
      \left[
      \omega_{0_{i+1,j}} \TSs{i+1,c} - \omega_{0_{i-1,j}} \TSs{i-1,c}
      \right]  \right) \\
 =\,& \DB_{ij}^g + \DB_{ij}^s\,,
\end{split}\end{equation}

where (including special treatment at the ends of the path)

\begin{equation}\label{three}
\begin{array}{rll}
\DB_{1j}^g \,= & \frac12\left(( 1 - \omega_{0_{2j}} ) B_{2j} +
                 ( 1 - \omega_{0_{1j}} ) B_{1j}\right) & \\
\DB_{ij}^g \,= & \frac12\left(( 1 - \omega_{0_{i+1,j}} ) B_{i+1,j} -
                 ( 1 - \omega_{0_{i-1,j}} ) B_{i-1,j}\right),
                 i = 2 \dots n_p-1 &
                 \text{ ``gas''} \\
\DB_{n_pj}^g \,= & B_{\text{space}} -
                 \frac12\left(( 1 - \omega_{0_{n_p,j}} ) B_{n_pj} +
                 ( 1 - \omega_{0_{n_p-1,j}} ) B_{n_p-1,j}\right) & \\
&\\
\DB_{1j}^s \,= & \frac12\left( \omega_{0_{2j}} \TSs{2c} +
                 \omega_{0_{1j}} \TSs{1c}\right) & \\
\DB_{ij}^s \,= & \frac12\left( \omega_{0_{i+1,j}} \TSs{i+1,c} -
                 \omega_{0_{i-1,j}} \TSs{i-1,c}\right),
                 i = 2 \dots n_p-1 &
                 \text{``scattering''} \\
\DB_{n_pj}^s \,= & -\frac12\left( \omega_{0_{n_pj}} \TSs{n_pc} +
                 \omega_{0_{n_p-1,j}} \TSs{n_p-1,c}\right) & \\
&\\
\omega_{0_{ij}} \,= & \frac{\beta_{{c\_s}_i}}{\beta_{e_{ij}}}
 \text{, and}\\
\beta_{e_{ij}} \,= & \alpha_{\text{gas}_{ij}} + \beta_{{c\_s}_i} +
 \beta_{{c\_a}_i} = \alpha_{\text{gas}_{ij}} + \beta_{{c\_e}_i}\,.
\end{array}
\end{equation}

We consistently use indices in the following way

\begin{longtable}{|ll|ll|ll|ll|}
\hline
$i$, $l$ & path position & $n_p$ & path length &
$k$ & species & $\mu$ & point in temperature grid \\
$j, n$ & frequency & $n_c$ & filter table size &
$c$ & channel & $m$ & point in species grid \\
\hline
\end{longtable}

We write $\beta_{{c\_s}_i}$ and $\beta_{{c\_e}_i}$ instead of
$\beta_{{c\_s}_{ij}}$ and $\beta_{{c\_e}_{ij}}$ because $\beta_{c\_s}$
and $\beta_{c\_e}$ change very slowly with frequency but depend strongly
upon ice water content (IWC) and temperature, so we can simply evaluate
the former for IWC$_i$ and T$_i$ using the tables of their dependence
upon IWC and T for the frequency at which they are tabulated that is
nearest to $\nu_j$.

\subsection{If $\TS$ were a channel quantity}

The following was developed when I was laboring under the misconception
that $\TS$ is a frequency-averaged channel quantity.  It's
actually a single-frequency quantity.  It's retained to show how to
combine frequency-averaged channel quantities with line-by-line
quantities.  A similar strategy was used for PFA.

In order to carry out frequency averaging, radiances are needed at
frequencies $\{\nu_{nc}\}$ for which the filter response function
$\phi_c(\nu_n) = \phi_{nc}$ is tabulated.  These frequencies are not the
same as $\{\nu_j\}$ for which Equation (\ref{one}) is evaluated.  The
radiances $I_j$ developed in Equation (\ref{one}) (and their derivatives)
are fitted with a spline in frequency, and that spline is then evaluated
at the frequencies $\{\nu_{nc}\}$, giving a new set of radiances
$\!\IF{I_j}$ at those frequencies, where the function $\!\IF{\cdot}$
interpolates from $\{\nu_j\}$ to $\{\nu_{nc}\}$.

Assuming rectangular quadrature (Simpson's quadrature is actually used),
when frequency averaging is carried out over the $n_c$ frequencies in
channel $c$, the radiance in channel $c$ is
%
\begin{equation}\label{four}
I_c = \sum_{n=1}^{n_c} \phi_{nc} \Delta \nu_{nc} \IF{I_j}
       = \sum_{n=1}^{n_c} \phi_{nc} \Delta \nu_{nc}
           \IF{\sum_{i=1}^{n_p} \DB_{ij}^g \MT_{ij}} +
         \sum_{n=1}^{n_c} \phi_{nc} \Delta \nu_{nc}
           \IF{\sum_{i=1}^{n_p} \DB_{ij}^s \MT_{ij}}
\end{equation}
%
In the final term, substitute the definition of $\DB_{ij}^s$ from
Equation (\ref{three}), observe that $\TSs{i\pm1,c}$ is a channel quantity
and therefore unaffected by $\!\IF{\cdot}$, and exchange the order
of summation, giving

\begin{equation}\begin{split}\label{five}
&
 \sum_{n=1}^{n_c} \phi_{nc} \Delta \nu_{nc}
  \IF{\sum_{i=1}^{n_p} \DB_{ij}^s \MT_{ij}}=
 \sum_{i=1}^{n_p} \TSs{ic} \sum_{n=1}^{n_c} \phi_{nc} \nu_{nc}
  \IF{\omega_{0_{ij}} \Delta \MT_{ij}} \\
& \renewcommand{\arraystretch}{1.25}
 \text{ where } \ \Delta \MT_{ij} =
 \left\{ \begin{array}{ll}
  \frac12 ( \MT_{1j} - \MT_{2j} )         & i = 1 \\
  \frac12 ( \MT_{i-1,j} - \MT_{i+1,j} )   & i = 2 \dots n_p-1 \\
  \frac12 ( \MT_{n_p-1,j} - \MT_{n_p,j} ) & i = n_p
 \end{array} \right.
\end{split}\end{equation}

(see the appendix for development of $\Delta \MT_{ij}$).   Defining the
frequency-averaged values

\begin{equation}\label{six}
\hMT_{ic} =
\sum_{n=1}^{n_c} \phi_{nc} \Delta \nu_{nc}
 \IF{\omega_{0_{ij}} \Delta \MT_{ij}}
\,,
\end{equation}

Equation (\ref{four}) becomes $I_c = I_c^g + I_c^s$ where

\begin{equation}\begin{split}\label{seven}
 I_c^g =\,& \sum_{n=1}^{n_c} \phi_{nc} \Delta \nu_{nc}
           \IF{\sum_{i=1}^{n_p} \DB_{ij}^g \MT_{ij}}
\text{ and } \\
 I_c^s =\,& \sum_{i=1}^{n_p} \TSs{ic}
  \sum_{n=1}^{n_c} \phi_{nc} \Delta \nu_{nc}
   \IF{\omega_{0_{ij}} \Delta \MT_{ij}}
       = \sum_{i=1}^{n_p} \TSs{ic}\, \hMT_{ic}
\end{split}\end{equation}

That is, the ``gas'' terms ($\DB_{ij}^g \MT_{ij}$) are frequency averaged
\emph{after} path integration, while the $\omega_{0_{ij}} \Delta \MT_{ij}$
factors in the ``scattering'' terms are frequency averaged \emph{before}
path integration to give $\hMT_{ic}$, then multiplied by the
$\TSs{ic}$ factors (which are channel averages), then their products
are integrated along the path.  This is necessary because path
integration is done (by the {\tt One\_Frequency} subrou\-tine of the full
forward model) for $\{\nu_j\}$, some of which appear in more than one
channel.  Therefore, in computing a product $\omega_{0_{ij}}
\Delta \MT_{ij} \TSs{ic}$, we would not know what $c$ to use for a particular
$j$.

\subsection{Derivatives}

Derivatives are treated similarly.  As in the previous section, using
$\MT^s_{ic}$ was done when I thought $\TS$ was a frequency-averaged
channel quantity.  Below, $\MT^s_{ic}$ ought to be $\MT_{ij}$ and $I^s_c$
should have the frequency averaging.

From Equation (\ref{seven})
$\frac{\partial I_c}{\partial x} = \frac{\partial I^g_c}{\partial x} +
\frac{\partial I^s_c}{\partial x}$ where

\begin{equation}\begin{split}\label{eight}
\frac{\partial I^g_c}{\partial x} =\,&
 \sum_{n=1}^{n_c} \phi_{nc} \Delta \nu_{nc}
  \IF{ \sum_{i=1}^{n_p}
   \frac{\partial \DB_{ij}^g}{\partial x} \MT_{ij}
   + \DB_{ij}^g \frac{\partial \MT_{ij}}{\partial x}
      } \\
=\,&
 \sum_{n=1}^{n_c} \phi_{nc} \Delta \nu_{nc}
  \IF{ \sum_{i=1}^{n_p} \MT_{ij} \left(
   \frac{\partial \DB_{ij}^g}{\partial x}
 - \DB_{ij}^g \sum_{l=1}^i \frac{\partial \delta_{lj}}{\partial x}
   \right)   } \text{ and} \\
\frac{\partial I^s_c}{\partial x}
=\,&
  \frac12
  \sum_{i=1}^{n_p} \left( \frac{\partial \hMT_{ic}}{\partial x}
                          \TSs{ic} +
                          \hMT_{ic}
                           \frac{\partial \TSs{ic}}{\partial x}
                   \right)
\end{split}\end{equation}

where $x$ is either the mixing ratio $f^k_m$ of species $k$  at point $m$
in the atmosphere, or temperature $T_\mu$ at point $\mu$ in the
atmosphere.

Handling frequency averaging for $\TS$ correctly, the definition of
$\frac{\partial I^s_c}{\partial x}$ in Equation (\ref{eight}) ought to be
similar to the definition of $\frac{\partial I^g_c}{\partial x}$ in
Equation (\ref{eight}), \emph{viz.}

\begin{equation}\begin{split}
\frac{\partial I^s_c}{\partial x}
=\,&
  \sum_{n=1}^{n_c} \phi_{nc} \Delta \nu_{nc}
  \IF{ \sum_{i=1}^{n_p} \MT_{ij} \left(
   \frac{\partial \DB_{ij}^s}{\partial x}
 - \DB_{ij}^s \sum_{l=1}^i \frac{\partial \delta_{lj}}{\partial x}
   \right)   } \text{, and, more simply than before} \\
\frac{\partial I_c}{\partial x}
=\,&
  \sum_{n=1}^{n_c} \phi_{nc} \Delta \nu_{nc}
  \IF{ \sum_{i=1}^{n_p} \MT_{ij} \left(
   \frac{\partial \oDB_{ij}}{\partial x}
 - \oDB_{ij}  \sum_{l=1}^i \frac{\partial \delta_{lj}}{\partial x}
   \right)   }
\end{split}\end{equation}

Since $B = \frac{h\nu}k \left( \exp\left(
\frac{h\nu}{kT}\right)+1\right)^{-1}$, $\frac{\partial B_{ij}}{\partial
f^k_m} = 0$ and $\frac{\partial B_{ij}}{\partial T_\mu} =
\frac{B_{ij}}{T_\mu^2} \left( \frac{h\nu}k + B_{ij} \right)$, derivatives
of the terms of the form $\left( 1 - \omega_{0_{ij}} \right) B_{ij}$ in
the definition of $\DB_{ij}^g$ in Equation (\ref{three}) are

\begin{equation}\begin{split}
\frac{\partial \left( 1 - \omega_{0_{ij}} \right) B_{ij}}{\partial f^k_m}
=\,&
 - \frac{\partial \omega_{0_{ij}}}{\partial f^k_i}B_{ij} \eta^k_{im}
\text{ and}\\
\frac{\partial \left( 1 - \omega_{0_{ij}} \right) B_{ij}}{\partial T_\mu}
=\,&
 \left(
  ( 1-\omega_{0_{ij}} ) \frac{\partial B_{ij}}{\partial T_i} -
  \frac{\partial \omega_{0_{ij}}}{\partial T_i}B_{ij}
 \right) \eta^T_{\mu i} \\
=\,&
 \left(
  ( 1-\omega_{0_{ij}} )
   \frac1{T_i^2} \left( \frac{h\nu}k + B_{ij} \right) -
  \frac{\partial \omega_{0_{ij}}}{\partial T_i}
 \right) B_{ij} \eta^T_{\mu i} \,.
\end{split}\end{equation}

\begin{minipage}{4.75in}
Assume the $k^{\text{th}}$ mixing ratio is represented on the red grid
and temperature on the blue grid, and we wish to have both at $l$, a point
on the integration path.

\vskip 3pt
Assume linear interpolation coefficients $\eta^k_{ml}$ and $\eta^T_{\mu
l}$ to interpolate the $k^{\text{th}}$ mixing ratio from points $m$ and
temperature from points $\mu$ to $l$, respectively.  Then $f^k_l = \sum_m
\eta^k_{ml} f^k_m$ and $T_l = \sum_\mu \eta^T_{\mu l} T_\mu$; $m$ and
$\mu$ are indices in the state vector, which is a two-dimensional quantity.
\end{minipage}
\hspace*{0.1in}
\parbox{1.5in}{
\includegraphics[scale=1]{wvs-095-eta}
}

The coefficients $\eta^k_{ml}$ and $\eta^T_{\mu l}$ are actually products
of coefficients that interpolate in $\zeta$ and $\phi$, e.g.,
$\eta^k_{ml} = \eta^k(\zeta_m,\zeta_l) \eta^k(\phi_m,\phi_l)$, where
$\eta^k(\zeta_m,\zeta_l) = (\zeta^k_{m+1}-\zeta_l) /
(\zeta^k_{m+1}-\zeta^k_m)$ if $\zeta^k_m < \zeta \leq \zeta^k_{m+1}$ and
$(\zeta_l-\zeta^k_{m-1}) / (\zeta^k_m-\zeta^k_{m-1})$ if $\zeta^k_{m-1} <
\zeta \leq \zeta^k_m$.  For each $k$ and $m$ at most four of
$\{\eta^k_{ml}\}$ are nonzero, and for each $\mu$ at most four of
$\{\eta^T_{\mu l}\}$ are nonzero.

$\MT_{ij} = \exp(-\delta_{ij}) =
\exp\left(-\int_0^{s_i} \alpha_j(s)\, \text{d} s \right) =
\exp(-\sum_{k=1}^{n_s} \delta^k_{ij})$, where

\begin{equation}\label{ten}
\delta^k_{ij}
= \int_0^{s_i} f^k(s) \beta^k_j(s)\, \text{d} s
\approx
  \sum_{l=1}^i f^k_l \beta^k_{lj}(T_l,\nu_j)
  \Delta s_l
= \sum_{l=1}^i \sum_m \eta^k_{ml} f^k_m
  \beta^k_{lj}\left(\sum_\mu \eta^T_{\mu l} T_\mu,\nu_j\right)
  \Delta s_l \,.
\end{equation}

From Equation (\ref{three})

\begin{equation}\begin{split}\label{eleven}
\frac{\partial \omega_{0_{ij}}}{\partial f^k_m} =\,&
  \frac{\partial \omega_{0_{ij}}}{\partial f^k_i} \eta^k_{im}
  \text{ where } \eta^k_{im} = \frac{\partial f^k_i}{\partial f^k_m}
  \text{ and} \\
  \frac{\partial \omega_{0_{ij}}}{\partial f^k_i}
=\,&
  \frac1{\beta_{e_{ij}}} \left(
   \frac{\partial \beta{{c\_s}_i}}{\partial f^k_i} -
    \omega_{0_{ij}} \left (
     \frac{\partial \alpha_{ij}}{\partial f^k_i} +
     \frac{\partial \beta_{{c\_e}_i}}{\partial f^k_i}
    \right) \right)
\,.
\end{split}\end{equation}

The $k=$ IWC and $k \neq$ IWC cases simplify
differently:

\begin{equation}\begin{split}\label{twelve}
\frac{\partial \omega_{0_{ij}}}{\partial f^{\text{IWC}}_m}
=\,&
  \left[
  \omega_{0_{ij}}
  \left ( \frac1{\beta_{{c\_s}_i}} - \frac1{\beta_{e_{ij}}} \right )
   \frac{\partial \beta_{{c\_s}_i}}{\partial f^{\text{IWC}}_i}
  -\frac{\omega_{0_{ij}}}{\beta_{e_{ij}}}
    \frac{\partial \beta_{{c\_a}_i}}{\partial f^{\text{IWC}}_i}
  \right]
  \eta^{\text{IWC}}_{im} \\
=\,&
  \left[
  (1-\omega_{0_{ij}})
   \frac{\partial \beta_{{c\_s}_i}}{\partial f^{\text{IWC}}_i}
  -\omega_{0_{ij}}
    \frac{\partial \beta_{{c\_a}_i}}{\partial f^{\text{IWC}}_i}
  \right]
  \frac{\eta^{\text{IWC}}_{im}}{\beta_{e_{ij}}} \\
%
\frac{\partial \omega_{0_{ij}}}{\partial f^{k \neq \text{IWC}}_m}
=\,&
 - \frac{\omega_{0_{ij}}}{\beta_{e_{ij}}}
   \frac{\partial \alpha_{\text{gas}_{ij}}}
        {\partial f^{k \neq \text{IWC}}_i} \eta^k_{im} =
 - \frac{\omega_{0_{ij}}}{\beta_{e_{ij}}} \beta^k_{ij}
   \eta^k_{im}
    \text{ (see also \h{wvs-102}).}
\end{split}\end{equation}

\begin{equation}\begin{split}\label{thirteen}
\frac{\partial \omega_{0_{ij}}}{\partial T_\mu}
=\,&
  \frac{\partial \omega_{0_{ij}}}{\partial T_i} \eta^T_{\mu l}
 \text{ where }
  \eta^T_{\mu l} = \frac{\partial T_i}{\partial T_\mu} \text{ and}\\
\frac{\partial \omega_{0_{ij}}}{\partial T_i}
=\,&
  \frac{1 - \omega_{0_{ij}}}{\beta_{e_{ij}}}
   \frac{\partial \beta_{{c\_s}_i}}{\partial T_i} -
    \frac{\omega_{0_{ij}}}{\beta_{e_{ij}}}
                  \left( \frac{\partial \beta_{{c\_a}_i}}{\partial T_i} +
                         \frac{\partial\alpha_{\text{gas}_{ij}}}{\partial T_i}
                  \right) \text{ or} \\
\frac{\partial \omega_{0_{ij}}}{\partial T_\mu}
=\,&
 \left[
  \frac{1 - \omega_{0_{ij}}}{\beta_{e_{ij}}}
   \frac{\partial \beta_{{c\_s}_i}}{\partial T_i} -
    \frac{\omega_{0_{ij}}}{\beta_{e_{ij}}}
                  \left( \frac{\partial \beta_{{c\_a}_i}}{\partial T_i} +
                         \frac{\partial\alpha_{\text{gas}_{ij}}}{\partial T_i}
                  \right)
 \right] \eta^T_{\mu l} \\
\end{split}\end{equation}

From Equation (\ref{ten}),

\begin{equation}\begin{split}\label{fourteen}
\frac{\partial \MT_{ij}}{\partial f^k_m} =
 - \MT_{ij} \frac{\partial \delta^k_{ij}}{\partial f^k_m}
 \text{ where }
  \frac{\partial \delta^k_{ij}}{\partial f^k_m} =\,& 
    \sum_{l=1}^i \frac{\partial f^k_l}{\partial f^k_m}
                 \frac{\partial \alpha_{lj}}{\partial f^k_l} \Delta s_l =
    \sum_{l=1}^i \eta^k_{ml} \beta^k_{lj} \Delta s_l
 \text{ and }\\
\frac{\partial \MT_{ij}}{\partial T_\mu} =
 - \MT_{ij} \frac{\partial \delta_{ij}}{\partial T_\mu}
  \text{ where }
    \frac{\partial \delta_{ij}}{\partial T_\mu} =\,&
     \sum_{l=1}^i \frac{\partial T_l}{\partial T_\mu}
     \left(
     \frac{\partial \alpha_{lj}}{\partial T_l} \Delta s_l +
     \alpha_{lj} \frac{\partial \Delta s_l}{\partial T_l} \right) = \\
    \,&
    \sum_{l=1}^i \eta^T_{\mu l} \left(
      \Delta s_l
      \sum_{k=1}^{n_s} f^k_l \frac{\partial \beta^k_{lj}}{\partial T_l} +
      \alpha_{lj} \frac{\partial \Delta s_l}{\partial T_l} \right)
\,.
\end{split}\end{equation}

In order to compute the derivatives of $\hMT_{ic}$ defined in Equation
(\ref{six}), \emph{viz.} $\frac{\partial\hMT_{ic}}{\partial x}$, where
$x$ is either $f^k_m$ or $T_\mu$, use Equations (\ref{twelve}),
(\ref{thirteen}) and (\ref{fourteen}) to compute

\begin{equation}\begin{split}\label{fifteen}
2\frac{\partial}{\partial x} \left( \omega_{0_{ij}} \Delta \MT_{ij} \right)
=\,&
 2 \left(
 \frac{\partial \omega_{0_{ij}}}{\partial x} \Delta \MT_{ij} +
 \omega_{0_{ij}} \frac{\partial \Delta \MT_{ij}}{\partial x}
 \right) \\
=\,& \left\{ \begin{array}{ll}
 \frac{\partial \omega_{0_{1j}}}{\partial x} ( \MT_{1j} - \MT_{2j} )
 - \omega_{0_{1j}} \left( \MT_{1j} \frac{\partial \delta^k_{1j}}{\partial x} -
                          \MT_{2j} \frac{\partial \delta^k_{2j}}{\partial x}
                   \right) & i = 1 \\
 \frac{\partial \omega_{0_{ij}}}{\partial x} ( \MT_{i-1,j} - \MT_{i+1,j} )
 - \omega_{0_{ij}}
     \left( \MT_{i-1,j} \frac{\partial \delta^k_{i-1,j}}{\partial x} -
            \MT_{i+1,j} \frac{\partial \delta^k_{i+1,j}}{\partial x}
     \right) & i = 2 \dots n_p-1 \\
 \frac{\partial \omega_{0_{n_p j}}}{\partial x} ( \MT_{n_p-1, j} - \MT_{n_p j} )
 - \omega_{0_{n_p j}}
     \left( \MT_{n_p-1,j} \frac{\partial \delta^k_{n_p-1,j}}{\partial x} +
            \MT_{n_p j} \frac{\partial \delta^k_{n_p j}}{\partial x}
     \right) & i = n_p \\
 \end{array}
 \right. \\
=\,& \left\{ \begin{array}{ll}
 \MT_{1j} \left( \frac{\partial \omega_{0_{1j}}}{\partial x} -
                 \omega_{0_{1j}} \frac{\partial \delta^k_{1j}}{\partial x}
          \right) -
 \MT_{2j} \left( \frac{\partial \omega_{0_{1j}}}{\partial x} -
                 \omega_{0_{1j}} \frac{\partial \delta^k_{2j}}{\partial x}
          \right)  & i = 1 \\
 \MT_{i-1,j} \left( \frac{\partial \omega_{0_{ij}}}{\partial x} -
                 \omega_{0_{ij}} \frac{\partial \delta^k_{i-1,j}}{\partial x}
          \right) -
 \MT_{i+1,j} \left( \frac{\partial \omega_{0_{ij}}}{\partial x} -
                 \omega_{0_{ij}} \frac{\partial \delta^k_{i+1,j}}{\partial x}
          \right)  & i = 2 \dots n_p-1 \\
 \MT_{n_p-1,j} \left( \frac{\partial \omega_{0_{n_p j}}}{\partial x} -
                 \omega_{0_{n_p j}} \frac{\partial \delta^k_{n_p-1,j}}{\partial x}
          \right) -
 \MT_{n_p j} \left( \frac{\partial \omega_{0_{n_p j}}}{\partial x} -
                 \omega_{0_{n_p j}} \frac{\partial \delta^k_{n_p j}}{\partial x}
          \right)  & i = n_p\,. \\
 \end{array}
 \right. \\
\end{split}\end{equation}

The full forward model operates by integrating along the path first,
producing $I_j$ and derivatives for frequency $\nu_j$.  After doing this
for all $\{\nu_j\}$, it interpolates $I_j$ to $\!\IF{I_j}$ and does the
frequency averaging, and similarly for derivatives.  That is, the order
of summations is as in Equation (\ref{four}).

$\alpha_\text{gas}$ depends upon temperature and atmospheric constituent
mixing ratios along the path, and frequency.

$\beta_{c\_s}$ and $\beta_{c\_e}$ depend upon temperature and IWC along
the path, and frequency.

$B$ depends upon temperature along the path, and frequency.

$\TS$ depends upon temperature, IWC and whatever molecules are specified
when it is computed, throughout the atmosphere, and frequency. 
Therefore, the second term in the definition of $\frac{\partial
I^s_c}{\partial x}$ in Equation (\ref{eight}) will produce nonzeros in
the Jacobian for every element in the state vector upon which $\TS$
depends, not just those near the path.

\newpage
\section{Implementation alternatives}

This section is also obsolete, applying only to the case of
frequency-averaged $\TS$.

There are several possibilities to perform the frequency averaging of the
derivative in Equation (\ref{fifteen}).  One method is to create arrays
to store $\left[ \frac{\partial \omega_{0_{ij}}}{\partial f^k_l} -
\omega_{0_{ij}} \frac{\partial \delta_{i\pm1,j}}{\partial f^k_l} \right]
\MT_{i\pm1,j}$ and similarly for temperature derivatives, having extents
$n_p^2 \times n_s \times n_f$ and $n_p^2 \times n_f$, and to provide data
structures that keep track of nonzero elements.  Assuming $n_p\approx
150,\, n_s \approx 20$, and $n_f \approx 500$, these arrays would have
$\approx 2.25\times10^8$ and $\approx 10^7$ elements, so this is clearly
impractical.

\subsection{Preferred method}

The forward model already stores $\eta^k_{il}$ and $f^k_l$, and computes
$\beta^k_{lj}$, $\frac{\partial \beta^k_{lj}}{\partial T_l}$ and
$\MT_{ij}$ for each frequency.  From Equation (\ref{eleven}),
$\frac{\partial \omega_{0_{ij}}}{\partial f^k_l}$ is only nonzero when $l
= i$. From Equations (\ref{three}) and (\ref{eleven}), $\omega_{0_{ij}}$ and
$\frac{\partial \omega_{0_{ij}}}{\partial f^k_l}$ depend upon quantities
the forward model already computes, plus $\beta_{c\_s_i}$,
$\beta_{c\_e_i}$ and their derivatives with respect to temperature and
IWC.

From Equation (\ref{fourteen}), $\frac{\partial \delta_{ij}}{\partial
f^k_m} \MT_{ij}$ and $\frac{\partial \delta_{ij}}{\partial T_\mu}
\MT_{ij}$ are sums of outer products of $\eta^k_{ml}$, $\beta^k_{lj}$,
$\Delta s_l$, $\eta^T_{\mu l}$, $f^k_l$,
$\frac{\partial\beta^k_{lj}}{\partial T_l}$, and $\MT_{ij}$, which are
all lower-dimensional quantities.  Arrays to represent the factors of the
outer products for all frequencies would have tractable extents $n_p
\times n_s \times n_f \approx 1.5\times10^6$ and $n_p \times n_f \approx
75\times10^3$.

Changes to storage in the forward model would be:

\begin{itemize}

\item Provide new arrays to represent
  $\beta_{c\_s_i}$, $\beta_{c\_e_i}$, and their derivatives with respect
  to temperature and IWC, and

\item Expand the arrays that represent $\beta^k_{ij}$, $\frac{\partial
  \beta^k_{ij}}{\partial T_i}$, and $\MT_{ij}$ within {\tt
  One\_Frequency} by adding a frequency dimension, and allocate them
  outside that subroutine.

\end{itemize}

Equations (\ref{seven}) and (\ref{eight}) could then be evaluated in
stages:

\begin{enumerate}

\item Evaluate $\beta_{c\_s_i}$, $\beta_{c\_e_i}$, and their derivatives
  with respect to $T_i$ and IWC$_i$, independently of frequency,

\item Evaluate the first term in each equation using a small modification
  of the {\tt One\_Frequency} subroutine to include the $\omega_0$ terms
  in $\Delta B^s_{ij}$, saving quantities enumerated above necessary to
  compute $\frac{\partial \omega_{0_{ij}}}{\partial f^k_i}$,
  $\frac{\partial \omega_{0_{ij}}}{\partial T_i}$, $\frac{\partial
  \delta_{ij}}{\partial f^k_l}$ and $\frac{\partial \delta_{ij}}{\partial
  T_i}$,

\item Compute $\frac{\partial \omega_{0_{ij}}}{\partial f^k_i}$,
  $\frac{\partial \omega_{0_{ij}}}{\partial T_i}$, $\frac{\partial
  \delta_{ij}}{\partial f^k_l}$ and $\frac{\partial \delta_{ij}}{\partial
  T_i}$, and frequency average them separately at each point along the
  path, similarly to how {\tt Inc\_Rad\_Path} is handled for PFA (see
  wvs-027), and

\item Evaluate the second term in a second invocation of {\tt
  One\_Frequency}, modified to use the linear forward model to compute
  the $\TSs{ic}$, $\frac{\partial \TSs{ic}}{\partial f^k_i}$ and
  $\frac{\partial \TSs{ic}}{\partial T_i}$ terms in Equations
  (\ref{seven}) and (\ref{eight}).  This is also similar to how PFA
  computations are done.

\end{enumerate}

\subsection{Other methods}

A second method is to compute coefficients to interpolate linearly from
$\{\nu_j\}$ to $\{\nu_{nc}\}$, combine those coefficients with
$\phi_{nc}$ and $\Delta \nu_{nc}$, and use them to frequency average
$\omega_{0_{i\pm1,j}} \MT_{i\pm1,j}$ as they are calculated during path
integration.

In linear interpolation form we can write

\begin{equation}
\IF{\omega_{0_{i\pm1,j}}\MT_{ij}} =
 \sum_{j=1}^{n_f} \eta_{nj} \omega_{0_{i\pm1,j}} \MT_{ij}
\end{equation}

where $n_f$ is the number of frequencies $\{\nu_j,\, j = 1, \cdots,
n_f\}$ at which Equation (\ref{one}) is evaluated and $\eta_{nj} =
\frac{\nu_{j+1}-\nu_n}{\nu_{j+1}-\nu_j}$ is the linear interpolating
coefficient from those frequencies to the filter function frequencies
$\{\nu_{nc},\, n = 1, \cdots, n_c\}$.  To avoid carrying these quantities
through the path integration (because $n_c << n_f$) we can frequency
average $\!\IF{\omega_{0_{i\pm1,j}}\MT_{ij}}$ to get, by exchanging the
order of summation,

\begin{equation}\begin{split}\label{seventeen}
\hMT_{ic} =\,& \sum_{n=1}^{n_c} \phi_{nc} \Delta \nu_{nc}
 \IF{\omega_{0_{ij}}\Delta \MT_{ij}}
 = \sum_{n=1}^{n_c} \phi_{nc} \Delta \nu_{nc}
    \sum_{j=1}^{n_f} \eta_{nj} \omega_{0_{ij}} \Delta \MT_{ij} \\
 =\,& \sum_{j=1}^{n_f} \omega_{0_{ij}} \Delta \MT_{ij}
    \sum_{n=1}^{n_c} \phi_{nc} \Delta \nu_{nc} \eta_{nj}
 = \sum_{j=1}^{n_f} h_{jc} \omega_{0_{ij}} \Delta \MT_{ij}\,,
\end{split}\end{equation}

where

\begin{equation}
h_{jc} = \sum_{n=1}^{n_c} \phi_{nc} \Delta \nu_{nc} \eta_{nj}\,.
\end{equation}

Before the {\tt One\_Frequency} subroutine in the full forward model is
invoked, compute $h_{jc}$ from the pointing frequency grid and the filter
function's frequency grids for all channels.  During path integration in
{\tt One\_Frequency}, evaluate Equation (\ref{seventeen}) to produce the
channel-averaged quantity $\hMT_{i\pm1,c}$.  This method also applies to
evaluating Equation (\ref{fifteen}).

Because of the use of linear interpolation between $\{\nu_j\}$ and
$\{\nu_{nc}\}$, this scheme is somewhat less accurate than the first
method.

A third method is to divide references to {\tt One\_Frequency} into sets
of frequencies that appear in each channel and compute the
frequency-averaged channel radiance for each channel after invocation of
{\tt One\_Frequency} for the subset of $\{\nu_j\}$ that appear within
that channel.  This would allow forming products $\omega_{0_{i\pm1,j}}
\MT_{ij} \TSs{i+1,c}$ during path integration, and would not require
frequency averaging during path integration.

Some frequencies appear in more than one channel.  Depending upon the
number of such frequencies, either the path integrations for those
frequencies can be repeated, or a data structure can be developed to
keep track of results for frequencies that appear in more than one
channel and reuse those results instead of repeating the path
integrations.

A fourth method is to interpolate $\TSs{i\pm1,c}$ and its derivatives
from adjacent channel centers to the element of $\{\nu_j\}$ for which
{\tt One\_Frequency} is invoked.  In principle this requires invoking
the linear forward model twice per path point and frequency, but if
storage permits it might be possible to invoke the linear forward model
for every path point for the two channels surrounding a subset of
$\{\nu_j\}$, and interpolate in those retained results.  The amount of
storage necessary for $\TSs{i\pm1,c}$ for two channels is $2 n_p$.  The
amount of storage necessary for $\frac{\partial \TSs{i\pm1,c}}{\partial
f^k_l}$ and $\frac{\partial \TSs{i\pm1,c}}{\partial T}$ is $2 n_p^2
(\hat{n}_s+1)$, where $\hat{n}_s$ is the number of species upon which
$\TSs{i\pm1,c}$ depends, i.e., the number of values of $k$.  Since $l \leq
i$ this could be halved at some expense to clarity of the software.

\newpage
\section*{Appendix -- development of $\Delta \hMT_{ic}$}

This is largely irrelevant, given that it was only needed because
frequency averaging for $\TS$ was handled incorrectly.

Using the definition of $\Delta B^s_{ij}$ from Equation (\ref{three}) and
expanding $\DB_{ij}^s \MT_{ij}$ in the second term of Equation
(\ref{four})

\begin{equation}\begin{split}
2 \sum_{n=1}^{n_c} & \phi_{nc} \Delta \nu_{nc}
           \IF{\sum_{i=1}^{n_p} \DB_{ij}^s \MT_{ij}} = \\
  & \sum_{n=1}^{n_c} \phi_{nc} \Delta \nu_{nc}
           \IF{(\omega_{0_{1j}} \TSs{1c} +
             \omega_{0_{2j}} \TSs{2c}) \MT_{1j}} \\
+ & \sum_{n=1}^{n_c} \sum_{i=2}^{n_p-1} \phi_{nc} \Delta \nu_{nc}
     \IF{(\omega_{0_{i+1,j}} \TSs{i+1,c} -
          \omega_{0_{i-1,j}} \TSs{i-1,c}) \MT_{ij}} \\
- & \sum_{n=1}^{n_c} \phi_{nc} \Delta \nu_{nc}
      \IF{(\omega_{0_{n_p-1,j}} \TSs{n_p-1,c} +
           \omega_{0_{n_pj}} \TSs{n_pc}) \MT_{n_pj}} \,.
\end{split}\end{equation}

Using the fact that $\TSs{ij}$ is a channel quantity, and can therefore be
factored out of the frequency interpolating operator, gives

\begin{equation}\begin{split}
2 \sum_{n=1}^{n_c} & \phi_{nc} \Delta \nu_{nc}
           \IF{\sum_{i=1}^{n_p} \DB_{ij}^s \MT_{ij}} = \\
  & \TSs{1c} \sum_{n=1}^{n_c} \phi_{nc} \Delta \nu_{nc}
     \IF{\omega_{0_{1j}} \MT_{1j}}
+   \TSs{2c} \sum_{n=1}^{n_c} \phi_{nc} \Delta \nu_{nc}
     \IF{\omega_{0_{2j}} \MT_{1j}} \\
+ & \sum_{i=2}^{n_p-1} \TSs{i+1,c}
     \sum_{n=1}^{n_c} \phi_{nc} \Delta \nu_{nc}
      \IF{\omega_{0_{i+1,j}} \MT_{ij}}
-   \sum_{i=2}^{n_p-1} \TSs{i-1,c}
     \sum_{n=1}^{n_c} \phi_{nc} \Delta \nu_{nc}
      \IF{\omega_{0_{i-1,j}} \MT_{ij}} \\
- & \TSs{n_p-1,c}
     \sum_{n=1}^{n_c} \phi_{nc} \Delta \nu_{nc}
      \IF{\omega_{0_{n_p-1,j}} \MT_{n_p j}}
-   \TSs{n_p c}
     \sum_{n=1}^{n_c} \phi_{nc} \Delta \nu_{nc}
      \IF{\omega_{0_{n_p j}} \MT_{n_p j}} \\
\end{split}\end{equation}

To simplify the notation, let $S_{kl} = \sum_n \phi_{nc} \Delta \nu_{nc}
\IF{\omega_{0_{kj}} \MT_{lj}}$, giving

\begin{equation}
  \TSs{1c} S_{11} + \TSs{2c} S_{21}
+ \sum_{i=2}^{n_p-1} \TSs{i+1,c} S_{i+1,i} -
  \sum_{i=2}^{n_p-1} \TSs{i-1,c} S_{i-1,i}
- \TSs{n_p-1,c} S_{n_p-1,n_p} - \TSs{n_p c} S_{n_p n_p}
\end{equation}

Shift the indices of summation to get $\TS$ with the same subscripts in
the sums, giving

\begin{equation}
  \TSs{1c} S_{11} + \TSs{2c} S_{21}
+ \sum_{i=3}^{n_p} \TSs{i,c} S_{i,i-1} -
  \sum_{i=1}^{n_p-2} \TSs{i,c} S_{i,i+1}
- \TSs{n_p-1,c} S_{n_p-1,n_p} - \TSs{n_p c} S_{n_p n_p}
\end{equation}

Move some terms out of the sums to put the summations over the same
ranges, giving

\begin{equation}\begin{split}
  & \TSs{1c} S_{11} + \TSs{2c} S_{21}
-   \TSs{1c} S_{12} - \TSs{2c} S_{23} \\
+ & \sum_{i=3}^{n_p-2} \TSs{ic} S_{i,i-1} -
    \sum_{i=3}^{n_p-2} \TSs{ic} S_{i,i+1} \\
+ & \TSs{n_p-1,c} S_{n_p-1,n_p-2} + \TSs{nc} S_{n_p,n_p-1}
-   \TSs{n_p-1,c} S_{n_p-1,n_p} - \TSs{n_p c} S_{n_p n_p}
\end{split}\end{equation}

Collect factors of $\TS$ with the same subscripts, giving

\begin{equation}\begin{split}
  & \TSs{1c} ( S_{11} - S_{12} )
+   \TSs{2c} ( S_{21} - S_{23} ) \\
+ & \sum_{i=3}^{n_p-2} \TSs{ic} ( S_{i,i-1} - S_{i,i+1} ) \\
+ & \TSs{n_p-1,c} ( S_{n_p-1,n_p-2} - S_{n_p-1,n_p} )
+   \TSs{n_p c} ( S_{n_p,n_p-1} - S_{n_p n_p} )
\end{split}\end{equation}

Observe that the second and penultimate terms fit the pattern of the
summation, giving

\begin{equation}
  \TSs{1c} ( S_{11} - S_{12} )
+ \sum_{i=2}^{n_p-1} \TSs{ic} ( S_{i,i-1} - S_{i,i+1} )
+ \TSs{n_p c} ( S_{n_p,n_p-1} - S_{n_p n_p} )
\end{equation}

Substituting the definition of $S_{kl}$ and observing that $\IF{\cdot}$
is a linear operator gives Equation (\ref{five}):

\begin{equation}
\sum_{i=1}^{n_p} \TSs{ic} \sum_{n=1}^{n_c} \phi_{nc} \Delta \nu_{nc}
 \IF{\omega_{0_{ij}} \Delta \MT_{ij}}
 \renewcommand{\arraystretch}{1.25}
 \text{ where } \ \Delta \MT_{ij} =
 \left\{ \begin{array}{ll}
  \frac12 ( \MT_{1j} - \MT_{2j} )         & i = 1 \\
  \frac12 ( \MT_{i-1,j} - \MT_{i+1,j} )   & i = 2 \dots n_p-1 \\
  \frac12 ( \MT_{n_p-1,j} - \MT_{n_p,j} ) & i = n_p\,.
 \end{array} \right.
\end{equation}

\label{lastpage}
\end{document}

% $Id$

% $Log$
% Revision 1.14  2011/07/26 23:46:55  vsnyder
% Repair Equation (13) -- IWC_a and IWC_s are just IWC
%
% Revision 1.13  2011/02/05 01:31:04  vsnyder
% Correct omega_0 derivatives
%
% Revision 1.12  2010/12/18 03:50:07  vsnyder
% Insert caveats about TScat being a single-frequency quantity, not a
% channel-averaged quantity
%
% Revision 1.11  2010/08/27 23:52:09  vsnyder
% Use n_p where n had been wrongly used
%
% Revision 1.10  2010/08/06 02:01:08  vsnyder
% Fix some indexing errors, add def of dB/dT
%
% Revision 1.9  2010/07/30 01:33:38  vsnyder
% Correct some sign errors, include a nonzero derivative in Equation 10
%
% Revision 1.8  2010/07/27 22:53:10  vsnyder
% Correct a dangling equation reference
%
% Revision 1.7  2010/07/26 20:57:13  vsnyder
% Substantial revision and reorganization
%
% Revision 1.6  2010/06/11 23:58:36  vsnyder
% Correct some subscripts in Equation 10
%
% Revision 1.5  2010/06/08 23:35:56  vsnyder
% Corrected Equations 10-11
%
% Revision 1.4  2010/06/04 23:17:05  vsnyder
% Correct date, change file name of graphic
%
% Revision 1.3  2010/06/04 23:14:19  vsnyder
% Sunstantial revision of radiative transfer, added fourth method
%
% Revision 1.2  2010/06/03 00:08:09  vsnyder
% Three ideas for combining channel quantities with single-frequency quantities
%
% Revision 1.1  2010/05/20 23:49:47  vsnyder
% Initial commit
%
