\documentclass[11pt]{article}
\usepackage{alltt}
\usepackage[fleqn]{amsmath}

\textwidth 6.5in
\oddsidemargin -0.25in
%\evensidemargin -0.5in
\topmargin -0.25in
\textheight 9in

\newcommand{\docname}{wvs-095r2}
\newcommand{\docdate}{2 June 2010}

\ifx\pdfoutput\undefined
  \pdfoutput=0
  \usepackage[hypertex,plainpages,hyperindex=true]{hyperref}
  \hypersetup{%
    hypertexnames=false%
  }
  % Specify the driver for the color package
  \ExecuteOptions{dvips}
  %\ExecuteOptions{xdvi}
\else
  \ifnum\pdfoutput>0
    \usepackage[pdftex,plainpages,hyperindex=true,pdfpagelabels]{hyperref}
    \hypersetup{%
      hypertexnames=false,%
      colorlinks=true,%
      linktocpage=true,%
    }
    % Specify the driver for the color package
    \ExecuteOptions{pdftex}
  \else
    \usepackage[hypertex,plainpages,hyperindex=true]{hyperref}
    \hypersetup{%
      hypertexnames=false%
    }
    % Specify the driver for the color package
    \ExecuteOptions{dvips}
    %\ExecuteOptions{xdvi}
  \fi
\fi

\hyperbaseurl{}
\newcommand\hr[1]{\href{#1.dvi}{dvi}, \href{#1.pdf}{pdf}}
\newcommand\h[1]{#1 (\hr{#1})}

\begin{document}

%\tracingcommands=1
\newlength{\hW} % heading box width
\newlength{\pW} % page number field width
\settowidth{\hW}{\bf\docname}
\settowidth{\pW}{Page \pageref{lastpage}\ of \pageref{lastpage}}
\ifdim \pW > \hW \setlength{\hW}{\pW} \fi
\makeatletter
\def\@biblabel#1{#1.}
\newcommand{\ps@twolines}{%
  \renewcommand{\@oddhead}{%
    \docdate\hfill\parbox[t]{\hW}{{\hfill\bf\docname}\newline
                          Page \thepage\ of \pageref{lastpage}}}%
\renewcommand{\@evenhead}{}%
\renewcommand{\@oddfoot}{}%
\renewcommand{\@evenfoot}{}%
}%
\makeatother
\pagestyle{twolines}

\newcommand{\TS}{T_\text{scat}}
\newcommand{\TSs}[1]{T_{\text{scat}_{#1}}}
\newcommand{\DB}{\Delta B}
\newcommand{\oDB}{\overline{\DB}}
\newcommand{\MT}{\mathcal{T}}
\newcommand{\hMT}{\mathcal{T}^s}
\newcommand{\IF}[1]{\,\mathcal{A}_n\!\left(#1\right)} % Interpolation Function

\vspace{-10pt}
\begin{tabbing}
\phantom{References: }\= \\
To: \>Bill, Nathaniel, Van\\
Subject: \>Radiance and derivatives incorporating $\TS$\\
From: \>Van Snyder\\
Reference: \>\h{wvs-066}, \h{wvs-080}
\end{tabbing}

\parindent 0pt \parskip 6pt
\vspace{-10pt}

When $\TS$ is taken into account, the factor $\DB_{ij}$ that appears in
the equation giving radiance at frequency $\nu_j$, \emph{viz.}

\begin{equation}\label{one}
I_j = \int \frac{\partial B_j}{\partial s} \MT_j\,\text{d}s
 \approx \sum_{i=1}^{n_p} \DB_{ij} \MT_{ij}\,,
\end{equation}

where $2 \DB_{ij} = B_{i+1,j} - B_{i-1,j}$, is replaced by

\begin{equation}\begin{split}\label{two}
2 \oDB_{ij}
 =\,& ( 1 - \omega_{0_{i+1,j}} ) B_{i+1,j} + \omega_{0_{i+1,j}} \TSs{i+1,c} -
      ( 1 - \omega_{0_{i-1,j}} ) B_{i-1,j} - \omega_{0_{i-1,j}} \TSs{i-1,c} \\
 =\,& \left[
      ( 1 - \omega_{0_{i+1,j}} ) B_{i+1,j} -
      ( 1 - \omega_{0_{i-1,j}} ) B_{i-1,j} \right] +
      \left[
      \omega_{0_{i+1,j}} \TSs{i+1,c} - \omega_{0_{i-1,j}} \TSs{i-1,c}
      \right] \\
 =\,& 2 \DB_{ij}^g + 2 \DB_{ij}^s\,,
\end{split}\end{equation}

where $c$ is a channel number,

\begin{equation}\label{three}
\begin{array}{rll}
2\DB_{ij}^g \,= & ( 1 - \omega_{0_{i+1,j}} ) B_{i+1,j} -
                ( 1 - \omega_{0_{i-1,j}} ) B_{i-1,j} &
                \text{``gas,''} \\
2\DB_{ij}^s \,= & \omega_{0_{i+1,j}} \TSs{i+1,c} -
                 \omega_{0_{i-1,j}} \TSs{i-1,c} &
                 \text{``scattering,''} \\
\omega_{0_{ij}} \,= & \frac{\beta_{{c\_s}_i}}{\beta_{e_{ij}}}
 \text{, and}\\
\beta_{e_{ij}} \,= & \alpha_{\text{gas}_{ij}} + \beta_{{c\_s}_i} +
 \beta_{{c\_a}_i} = \alpha_{\text{gas}_{ij}} + \beta_{{c\_e}_i}\,.
\end{array}
\end{equation}

We write $\beta_{{c\_s}_i}$ and $\beta_{{c\_e}_i}$ instead of
$\beta_{{c\_s}_{ij}}$ and $\beta_{{c\_e}_{ij}}$ because $\beta_{c\_s}$
and $\beta_{c\_e}$ change very slowly with frequency but depend strongly
upon ice water content (IWC) and temperature, so we can simply evaluate
the former for IWC$_i$ and T$_i$ using the tables of their dependence
upon IWC and T for the frequency at which they are tabulated that is
nearest to $\nu_j$.

In order to carry out frequency averaging, radiances are needed at
frequencies $\{\nu_{nc}\}$ for which the filter response function
$\phi_c(\nu_n) = \phi_{nc}$ is tabulated.  These frequencies are not the
same as $\{\nu_j\}$ for which Equation (\ref{one}) is evaluated.  The
radiances $I_j$ developed in Equation (\ref{one}) (and their derivatives)
are fitted with a spline in frequency, and that spline is then evaluated
at the frequencies $\{\nu_{nc}\}$, giving a new set of radiances
$\!\IF{I_j}$ at those frequencies, where the function $\!\IF{\cdot}$
interpolates from $\{\nu_j\}$ to $\{\nu_{nc}\}$.

Assuming rectangular quadrature (Simpson's quadrature is actually used),
when frequency averaging is carried out, the radiance in channel $c$ is

\begin{equation}\label{four}
I_c = \sum_{n=1}^{n_c} \phi_{nc} \Delta \nu_{nc} \IF{I_j}
       = \sum_{n=1}^{n_c} \phi_{nc} \Delta \nu_{nc}
           \IF{\sum_{i=1}^{n_p} \DB_{ij}^g \MT_{ij}} +
         \sum_{n=1}^{n_c} \phi_{nc} \Delta \nu_{nc}
           \IF{\sum_{i=1}^{n_p} \DB_{ij}^s \MT_{ij}}
\end{equation}

In the final term, substitute the definition of $\DB_{ij}^s$ from
Equation (\ref{two}), observe that $\TSs{i\pm1,c}$ is a channel quantity
and therefore unaffected by $\!\IF{\cdot}$, and exchange the order
of summation, giving

\begin{equation}\begin{split}\label{five}
\sum_{n=1}^{n_c} \phi_{nc} \Delta \nu_{nc}
  \IF{\sum_{i=1}^{n_p} \DB_{ij}^s \MT_{ij}}=\,&
\sum_{i=1}^{n_p} \sum_{n=1}^{n_c}
          \phi_{nc} \Delta \nu_{nc}
          \IF{( \omega_{0_{i+1,j}} \TSs{i+1,c} 
            - \omega_{0_{i-1,j}} \TSs{i-1,c} ) \MT_{ij}} \\
=\,&
\sum_{i=1}^{n_p} \TSs{i+1,c} \sum_{n=1}^{n_c}
 \phi_{nc} \Delta \nu_{nc} \IF{\omega_{0_{i+1,j}} \MT_{ij}}
- \\
\,&
\sum_{i=1}^{n_p} \TSs{i-1,c} \sum_{n=1}^{n_c}
 \phi_{nc} \Delta \nu_{nc} \IF{\omega_{0_{i-1,j}} \MT_{ij}}\,.
\end{split}\end{equation}

Defining the frequency-averaged values

\begin{equation}\label{w0hat}
\hMT_{i\pm1,c} =
\sum_{n=1}^{n_c} \phi_{nc} \Delta \nu_{nc} \IF{\omega_{0_{i\pm1,j}} \MT_{ij}}
\,,
\end{equation}

Equation (\ref{four}) becomes

\begin{equation}\label{seven}
I_c = \sum_{n=1}^{n_c} \phi_{nc} \Delta \nu_{nc}
       \IF{\sum_{i=1}^{n_p} \DB_{ij}^g \MT_{ij}} +
      \sum_{i=1}^{n_p} (\hMT_{i+1,c} \TSs{i+1,c} -
                        \hMT_{i-1,c} \TSs{i-1,c})\,.
\end{equation}

That is, the ``gas'' terms ($\DB_{ij}^g \MT_{ij}$) are frequency averaged
\emph{after} path integration, while the $\omega_{0_{i\pm1,j}} \MT_{ij}$
factors in the ``scattering'' terms are frequency averaged \emph{before}
path integration to give $\hMT_{i\pm1,c}$, then multiplied by the
$\TSs{i\pm1,c}$ factors (which are channel averages), then their products
are integrated along the path.  This is necessary because path
integration is done (by the {\tt Frequency\_Loop} subroutine of the full
forward model) for $\{\nu_j\}$, some of which appear in more than one
channel.  Therefore, in computing a product ${\omega_0}_{i\pm1,j}
\MT_{ij} \TSs{i+1,c}$, we do not know what $c$ to use for a particular
$j$.

Derivatives are treated similarly.  From Equation (\ref{seven})

\begin{equation}\begin{split}\label{eight}
\frac{\partial I_c}{\partial x} =\,&
 \sum_{n=1}^{n_c} \phi_{nc} \Delta \nu_{nc}
  \IF{ \sum_{i=1}^{n_p}
   \frac{\partial \DB_{ij}^g}{\partial x} \MT_{ij}
                        + \DB_{ij}^g \frac{\partial \MT_{ij}}{\partial x}
      } \\
+\,&
  \sum_{i=1}^{n_p} \left( \frac{\partial \hMT_{i+1,c}}{\partial x}
                          \TSs{i+1,c} +
                          \hMT_{i+1,c}
                           \frac{\partial \TSs{i+1,c}}{\partial x}
                        - \frac{\partial \hMT_{i+1,c}}{\partial x}
                          \TSs{i-1,c} -
                          \hMT_{i-1,c}
                           \frac{\partial \TSs{i-1,c}}{\partial x}
                   \right)
\end{split}\end{equation}

where $x$ is either a mixing ratio of some specie, or temperature, at
some point in the atmosphere.

From Equation (\ref{w0hat})

\begin{equation}\label{nine}
\frac{\partial \hMT_{i\pm1,c}}{\partial x} =
 \sum_{n=1}^{n_c} \phi_{nc} \Delta \nu_{nc}
  \IF{\frac{\partial \omega_{0_{i\pm1,j}}}{\partial x} \MT_{ij} +
         \omega_{0_{i\pm1,j}} \frac{\partial \MT_{ij}}{\partial x}}.
\end{equation}

From Equation (\ref{three})

\begin{equation}
\frac{\partial \omega_0}{\partial x} =
 \frac1{\beta_e^2} \left( \frac{\partial \beta_{c\_s}}{\partial x} -
                          \frac{\partial \beta_e}{\partial x} \right)
=
 \frac1{\beta_e^2} \left( \frac{\partial \beta_{c\_s}}{\partial x} -
                          \frac{\partial \beta_{c\_e}}{\partial x} -
                          \frac{\partial \alpha_\text{gas}}{\partial x}
                   \right) \,.
\end{equation}

$\alpha_\text{gas}$ depends upon temperature and atmospheric constituent
mixing ratios along the path, and frequency.

$\beta_{c\_s}$ and $\beta_{c\_e}$ depend upon temperature and IWC along
the path, and frequency.

$B$ depends upon temperature along the path, and frequency.

$\TS$ depends upon temperature, IWC and whatever molecules are specified
when it is computed, throughout the atmosphere, and frequency.  Therefore,
the second term in Equation (\ref{eight}) will produce nonzeros in the
Jacobian for every element in the state vector upon which $\TS$ depends,
not just those near the path.

In discretized form, $\MT_{ij} = \exp(-\delta_{ij})$, where $\delta_{ij}
= \sum_{l=1}^j \sum_{k=1}^{n_s} \eta^k_l f^k_l \beta^k_{lj}$, in which
$\eta^k_l$ is an interpolating coefficient from the solution grid to the
integration path, $f^k_l$ is the volume mixing ratio for the
$k^{\text{th}}$ species at the $l^{\text{th}}$ point on the path, and
$\beta^k_{lj}$ (which depends upon frequency and temperature) is the
absorption cross section of that species at that point.

From this,

\begin{equation}
\frac{\partial \MT_{ij}}{\partial f^k_l} =
 - \MT_{ij} \frac{\partial \delta_{ij}}{\partial f^k_l} =
 - \MT_{ij} \eta^k_l \beta^k_{lj}
 \text{ and }
\frac{\partial \MT_{ij}}{\partial T_l} =
 - \MT_{ij} \frac{\partial \delta_{ij}}{\partial T_l} =
 - \MT_{ij} \eta^k_l f^k_l \frac{\partial \beta^k_{lj}}{\partial T_l}
\end{equation}

The full forward model operates by integrating along the path first,
producing $I_j$ and derivatives for frequency $\nu_j$.  After doing this
for all $\{\nu_j\}$, it interpolates $I_j$ to $\!\IF{I_j}$ and does the
frequency averaging.  That is, the order of summations is as in Equation
(\ref{four}).

Along the way the forward model calculates an array {\tt d\_delta\_df} $=
\frac{\partial \delta_{ij}}{\partial f^k_l} = \eta^k_l \beta^k_{lj}$ as
an intermediate result for each frequency.  This is a two-dimensional
array having extents path length $\times$ state vector size. 
Measurements of program performance showed that a statement {\tt
d\_delta\_df = 0} was consuming a significant fraction of the cost of
executing the forward model.  Therefore a data structure was developed to
keep track of the nonzero elements.

In order to perform the frequency averaging as described in Equation
(\ref{nine}) one of two things will be necessary.  One method is to
expand {\tt d\_delta\_df} to a three-dimensional array having extents
path length $\times$ state vector size $\times$ frequencies, and to
expand the data structure that keeps track of nonzero elements
similarly.  Then interpolate {\tt d\_delta\_df} to the filter function
frequencies giving $\!\IF{\frac{\partial \delta_{ij}}{\partial f^k_l}} =
\IF{\eta^k_l \beta^k_{lj} }$, then frequency average those values giving
$\frac{\partial \delta_{ic}}{\partial f^k_l}$.

A second method is to compute coefficients to interpolate linearly from
$\{\nu_j\}$ to $\{\nu_{nc}\}$, combine those coefficients with
$\phi_{nc}$ and $\Delta \nu_{nc}$, and use them to frequency average
${\omega_0}_{i\pm1,j} \MT_{i\pm1,j}$ as they are calculated during path
integration.

In linear interpolation form we can write

\begin{equation}
\IF{\omega_{0_{i\pm1,j}}\MT_{ij}} =
 \sum_{j=1}^{n_f} \eta_{nj} \omega_{0_{i\pm1,j}} \MT_{ij}
\end{equation}

where $n_f$ is the number of frequencies $\{\nu_j,\, j = 1, \cdots,
n_f\}$ at which Equation (\ref{one}) is evaluated and $\eta_{nj} =
\frac{\nu_{j+1}-\nu_n}{\nu_{j+1}-\nu_j}$ is the linear interpolating
coefficient from those frequencies to the filter function frequencies
$\{\nu_{nc},\, n = 1, \cdots, n_c\}$.  To avoid carrying these quantities
through the path integration (because $n_c >> n_f$) we can frequency
average $\!\IF{\omega_{0_{i\pm1,j}}\MT_{ij}}$ to get, by exchanging the
order of summation,

\begin{equation}\begin{split}\label{thirteen}
\hMT_{i\pm1,c} =\,& \sum_{n=1}^{n_c} \phi_{nc} \Delta \nu_{nc}
 \IF{\omega_{0_{i\pm1,j}}\MT_{ij}}
 = \sum_{n=1}^{n_c} \phi_{nc} \Delta \nu_{nc}
    \sum_{j=1}^{n_f} \eta_{nj} \omega_{0_{i\pm1,j}} \MT_{ij} \\
 =\,& \sum_{j=1}^{n_f} \omega_{0_{i\pm1,j}} \MT_{ij}
    \sum_{n=1}^{n_c} \phi_{nc} \Delta \nu_{nc} \eta_{nj}
 = \sum_{j=1}^{n_f} h_{jc} \omega_{0_{i\pm1,j}} \MT_{ij}\,,
\end{split}\end{equation}

where

\begin{equation}
h_{jc} = \sum_{n=1}^{n_c} \phi_{nc} \Delta \nu_{nc} \eta_{nj}\,.
\end{equation}

Before the {\tt Frequency\_Loop} subroutine in the full forward model is
invoked, compute $h_{jc}$ from the pointing frequency grid and the filter
function's frequency grids for all channels.  During path integration in
{\tt Frequency\_Loop}, evaluate Equation (\ref{thirteen}) to produce the
channel-averaged quantity $\hMT_{i\pm1,c}$.  This method also applies to
evaluating Equation (\ref{nine}).

Because of the use of linear interpolation between $\{\nu_j\}$ and
$\{\nu_{nc}\}$, this scheme is somewhat less accurate than the first
method.

A third method is to divide references to {\tt Frequency\_Loop} into sets
of frequencies that appear in each channel and compute the
frequency-averaged channel radiance for each channel after invocation of
{\tt Frequency\_Loop} for the subset of $\{\nu_j\}$ that appear within
that channel.  This would allow forming products ${\omega_0}_{i\pm1,j}
\MT_{ij} \TSs{i+1,c}$ during path integration, and would not require
frequency averaging during path integration.

Some frequencies appear in more than one channel.  Depending upon the
number of such frequencies, either the path integrations for those
frequencies can be repeated, or a data structure can be developed to
keep track of results for frequencies that appear in more than one
channel and reuse those results instead of repeating the path
integrations.

\label{lastpage}
\end{document}

% $Id$

% $Log$
% Revision 1.1  2010/05/20 23:49:47  vsnyder
% Initial commit
%
