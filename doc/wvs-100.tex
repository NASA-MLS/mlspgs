\documentclass[11pt]{article}
\usepackage{alltt}
\usepackage[fleqn]{amsmath}
\usepackage{floatflt}
\usepackage{graphicx}
\usepackage{longtable}

\textwidth 6.5in
\oddsidemargin -0.25in
%\evensidemargin -0.5in
\topmargin -0.5in
\textheight 9in

\newcommand{\docname}{wvs-100r2}
\newcommand{\docdate}{20 January 2011}

\ifx\pdfoutput\undefined
  \pdfoutput=0
  \usepackage[hypertex,plainpages,hyperindex=true]{hyperref}
  \hypersetup{%
    hypertexnames=false%
  }
  % Specify the driver for the color package
  \ExecuteOptions{dvips}
  %\ExecuteOptions{xdvi}
\else
  \ifnum\pdfoutput>0
    \usepackage[pdftex,plainpages,hyperindex=true,pdfpagelabels]{hyperref}
    \hypersetup{%
      hypertexnames=false,%
      colorlinks=true,%
      linktocpage=true,%
    }
    % Specify the driver for the color package
    \ExecuteOptions{pdftex}
  \else
    \usepackage[hypertex,plainpages,hyperindex=true]{hyperref}
    \hypersetup{%
      hypertexnames=false%
    }
    % Specify the driver for the color package
    \ExecuteOptions{dvips}
    %\ExecuteOptions{xdvi}
  \fi
\fi

\hyperbaseurl{}
\newcommand\hr[1]{\href{#1.dvi}{dvi}, \href{#1.pdf}{pdf}}
\newcommand\h[1]{#1 (\hr{#1})}

\begin{document}

%\tracingcommands=1
\newlength{\hW} % heading box width
\newlength{\pW} % page number field width
\settowidth{\hW}{\bf\docname}
\settowidth{\pW}{Page \pageref{lastpage}\ of \pageref{lastpage}}
\ifdim \pW > \hW \setlength{\hW}{\pW} \fi
\makeatletter
\def\@biblabel#1{#1.}
\newcommand{\ps@twolines}{%
  \renewcommand{\@oddhead}{%
    \docdate\hfill\parbox[t]{\hW}{{\hfill\bf\docname}\newline
                          Page \thepage\ of \pageref{lastpage}}}%
\renewcommand{\@evenhead}{}%
\renewcommand{\@oddfoot}{}%
\renewcommand{\@evenfoot}{}%
}%
\makeatother
\pagestyle{twolines}

\renewcommand{\d}{\text{d}}
\newcommand{\T}{\mathcal{T}}

\vspace{-10pt}
\begin{tabbing}
\phantom{References: }\= \\
To: \>Nathaniel, Bill, Paul, Igor\\
Subject: \>Reducing the cost of radiance derivatives\\
From: \>Van Snyder\\
Reference: \>\h{wvs-093}, \h{wvs-101}, JPL D-18130 \\
\end{tabbing}

\parindent 0pt \parskip 6pt
\vspace{-10pt}

\section{Summary}

Three representations of derivatives of the clear-sky non-scattering
radiative transfer equation with respect to volume mixing ratio on a grid
through which the path of integration passes

\begin{equation*}\boxed{
\begin{split}
\frac{\partial I(s_m)}{\partial f^k_{lm}}
=\,&
 \int_{s_0}^{s_m} \T(s,s_m) \beta^k(s) \mu^k_{lm}(s)
  ( B(s) - I(s) ) \,\d s \\
=\,&
 -\T(s_0,s_m) \left( I(s_0) - B(s_0) \right)
   \int_{s_0}^{s_m} \beta^k(\sigma) \mu^k_{lm}(\sigma) \,\d \sigma \,+ \\
\,&
 \int_{s_0}^{s_m} \T(s,s_m) \frac{\partial B(s)}{\partial s}
  \left( \int_s^{s_m} \beta^k(\sigma) \mu^k_{lm}(\sigma) \,\d \sigma
  \right) \,\d s \\
=\,& -I(s_0) \T(s_0,s_m)
     \int_{s_0}^{s_m} \beta^k(\sigma) \mu^k_{lm}(\sigma)\, \d \sigma \,+ \\
\,&
\int_{s_0}^{s_m}
 \T(s,s_m) B(s) \left( \beta^k(s) \mu^k_{lm}(s) -
 \alpha(s) \int_s^{s_m} \beta^k(\sigma) \mu^k_{lm}(\sigma)\, \d \sigma \right)
 \, \d s
\end{split}}
\end{equation*}

 are developed and compared.
 
\section{Radiative transfer equation}

The clear-sky non-scattering radiative transfer equation is

\begin{equation}\label{rad}
\frac{\d I(s)}{\d s} + \alpha(s) I(s) = \alpha(s) B(s)\,.
\end{equation}

The solution of Equation (\ref{rad}) can be written

\begin{equation}\label{first}
I(s_m) = I(s_0) \T(s_0,s_m) +
 \int_{s_0}^{s_m} \T(s,s_m) \alpha(s) B(s) \,
  \d s\,,
\end{equation}

where $s_m$ is the location of the instrument, $s_0$ is the location of the end
of the path away from the instrument, $B(s)$ is the Planck function $\frac{h
\nu}k \left( \exp\left(\frac{h \nu}{k T(s)}\right) -1\right)^{-1}$,

\begin{equation}\label{two}
\T(s,s_m) = \exp\left( - \int_s^{s_m} \alpha(\sigma)
 \, \d \sigma \right)
 \,,
\end{equation}

and $\alpha(\sigma)$ is the absorption cross section. The volume mixing
ratio for the $k^\text{th}$ species at $(\phi^k_l,\zeta^k_m)$ is
$f^k_{lm}$ where $\phi$ is orbit geodetic angle, $\zeta = -\log_{10}(P)$,
and $P$ is pressure in hPa.

\section{Derivatives of the solution of the radiative transfer equation}

The derivative of Equation (\ref{first}) with respect to $f^k_{lm}$ is

\begin{equation}\label{six}
\frac{\partial I(s_m)}{\partial f^k_{lm}}
= I(s_0) \frac{\partial \T(s_0,s_m)}{\partial f^k_{lm}} +
\int_{s_0}^{s_m}
 \T(s,s_m) \frac{\partial \alpha(s)}{\partial f^k_{lm}} B(s) +
 \frac{\partial \T(s,s_m)}{\partial f^k_{lm}} \alpha(s) B(s)
 \, \d s \,.
\end{equation}

Substituting

\begin{equation}\label{three}
\alpha(s) = \sum_{k=1}^{N_s} \beta^k(s) \mu^k_{lm}(s) f^k_{lm}\,,
\end{equation}

where $\mu^k_{lm}(s)$ is an interpolating coefficient from
$(\phi^k_l,\zeta^k_m)$ to $s$ and $\beta^k(s)$ is the absorption
coefficient for the $k^\text{th}$ species at $s$, into Equation
(\ref{two}) gives

\begin{equation}\label{tau}
\T(s,s_m) = \exp\left( - \int_s^{s_m} \sum_{k=1}^{N_s} \beta^k(\sigma) \mu^k_{lm}(\sigma) f^k_{lm}
 \, \d \sigma \right) \,.
\end{equation}

From Equations (\ref{two}), (\ref{three}) and (\ref{tau})

\begin{equation}\begin{split}\label{dT}
\frac{\partial \alpha(s)}{\partial f^k_{lm}}
=\,& \beta^k(s) \mu^k_{lm}(s)
\text{ and } \\
\frac{\partial \T(s,s_m)}{\partial f^k_{lm}}
=\,& -\T(s,s_m) \frac{\partial}{\partial f^k_{lm}}
                \int_s^{s_m} \alpha(\sigma) \,\d \sigma
=    -\T(s,s_m) \int_s^{s_m}
                \frac{\partial}{\partial f^k_{lm}} \alpha(\sigma) \,\d \sigma \\
=\,& -\T(s,s_m) \int_s^{s_m} \beta^k(\sigma) \mu^k_{lm}(\sigma) \d \sigma
\end{split}\end{equation}

Substituting Equation (\ref{dT}) into Equation (\ref{six}) we have

\begin{equation}\boxed{\begin{split}\label{dI_dfklm}
\frac{\partial I(s_m)}{\partial f^k_{lm}}
=\,& -I(s_0) \T(s_0,s_m)
     \int_{s_0}^{s_m} \beta^k(\sigma) \mu^k_{lm}(\sigma)\, \d \sigma \,+ \\
\,&
\int_{s_0}^{s_m}
 \T(s,s_m) B(s) \left( \beta^k(s) \mu^k_{lm}(s) -
 \alpha(s) \int_s^{s_m} \beta^k(\sigma) \mu^k_{lm}(\sigma)\, \d \sigma \right)
 \, \d s\,.
\end{split}}\end{equation}

Even assuming that $\T(s,s_m)$ has been saved during evaluation of
$I(s_m)$ using Equation (\ref{first}), this is a double integral because
of the inner integral $\int_s^{s_m} \beta^k(\sigma) \mu^k_{lm}(\sigma)\,
\d \sigma$, which is different for every different value of $\{k,l,m\}$.

\section{Solutions of derivatives of the radiative transfer equation}

Instead of computing the derivative of Equation (\ref{first}), compute
the derivative of Equation (\ref{rad}) with respect to mixing ratios
$f^k_{lm}$ on the grid giving\footnote{This derivation, and another one
arriving at the same result starting from Equation (\ref{first}), also
appear in \h{wvs-093}.}

\begin{equation}
\frac{\partial}{\partial f^k_{lm}} \frac{\d I(s)}{\d s} +
 \frac{\partial \alpha(s)}{\partial f^k_{lm}} I(s) +
 \alpha(s) \frac{\partial I(s)}{\partial f^k_{lm}} =
 \frac{\partial \alpha(s)}{\partial f^k_{lm}} B(s) \,.
\end{equation}

Exchanging the order of differentiation, using $\frac{\partial
\alpha(s)}{\partial f^k_{lm}} = \beta^k(s) \mu^k_{lm}(s)$, and rearranging
the equation to the usual form gives

\begin{equation}\begin{split}
\frac{\d}{\d s} \frac{\partial I(s)}{\partial f^k_{lm}} +
 \alpha(s) \frac{\partial I(s)}{\partial f^k_{lm}}
=\,&
 \frac{\partial \alpha(s)}{\partial f^k_{lm}} ( B(s) - I(s) ) \\
=\,&
 \beta^k(s) \mu^k_{lm}(s) ( B(s) - I(s) )\,.
\end{split}\end{equation}

Solve for $\frac{\partial I(s_m)}{\partial f^k_{lm}}$ using
$\frac{\partial I(s_0)}{\partial f^k_{lm}}=0$ to get

\newcounter{foo}
\setcounter{foo}{\value{equation}}

\begin{equation}\label{drad}\boxed{
\frac{\partial I(s_m)}{\partial f^k_{lm}} =
 \int_{s_0}^{s_m} \T(s,s_m) \beta^k(s) \mu^k_{lm}(s)
  ( B(s) - I(s) ) \,\d s
\,.}
\end{equation}

This is actually the same as Equation (\ref{dI_dfklm}). To see this,
substitute Equation (\ref{first}) for $I(s)$ in Equation (\ref{drad}) and
use $\T(s_0,s) \T(s,s_m) = \T(s_0,s_m)$ and $\T(\sigma,s) \T(s,s_m) =
\T(\sigma,s_m)$ to get

\begin{equation}\begin{split}
\frac{\partial I(s_m)}{\partial f^k_{lm}}
=\,& \dots\,
-\int_{s_0}^{s_m} \T(s,s_m) \beta^k(s) \mu^k_{lm}(s) I(s) \,\d s \\
=\,& \dots\,
-\int_{s_0}^{s_m} \T(s,s_m) \beta^k(s) \mu^k_{lm}(s)
\left[ I(s_0) \T(s_0,s) +
 \int_{s_0}^s \T(\sigma,s) \alpha(\sigma) B(\sigma) \,\d\sigma
\right] \\
=\,& \dots\,
-I(s_0) \T(s_0,s_m) \int_{s_0}^{s_m} \beta^k(s) \mu^k_{lm}(s) \,\d s \\
\,&\phantom{\dots\,}
- \int_{s_0}^{s_m} \T(s,s_m) \beta^k(s) \mu^k_{lm}(s)
    \int_{s_0}^s \T(\sigma,s) \alpha(\sigma) B(\sigma) \,\d\sigma
     \,\d s \\
=\,& \dots\,
-I(s_0) \T(s_0,s_m) \int_{s_0}^{s_m} \beta^k(s) \mu^k_{lm}(s) \,\d s \\
\,&\phantom{\dots\,}
- \int_{s_0}^{s_m} \beta^k(s) \mu^k_{lm}(s)
    \int_{s_0}^s \T(\sigma,s_m) \alpha(\sigma) B(\sigma) \,\d\sigma
     \,\d s
\,.
\end{split}\end{equation}

Exchanging the order of integration in the final integral above, Equation
(\ref{drad}) becomes

\begin{equation}\boxed{\begin{split}\label{dI_dfklm2}
\frac{\partial I(s_m)}{\partial f^k_{lm}}
=\,&
-I(s_0) \T(s_0,s_m) \int_{s_0}^{s_m} \beta^k(s) \mu^k_{lm}(s) \,\d s\, + \\
\,&
\int_{s_0}^{s_m} \T(s,s_m) B(s)
 \left( \beta^k(s) \mu^k_{lm}(s)
 - \alpha(s)
   \int_s^{s_m} \beta^k(\sigma) \mu^k_{lm}(\sigma) \,\d \sigma
 \right) \,\d s
\end{split}}
\end{equation}

which is the same as Equation (\ref{dI_dfklm}).

\section{Comparison of methods}

An important difference between Equation (\ref{drad})

\newcounter{eqsv}
\setcounter{eqsv}{\value{equation}}
\setcounter{equation}{\value{foo}}
\begin{equation}\boxed{
\frac{\partial I(s_m)}{\partial f^k_{lm}} =
 \int_{s_0}^{s_m} \T(s,s_m) \beta^k(s) \mu^k_{lm}(s)
  ( B(s) - I(s) ) \,\d s
\,.}
\end{equation}
\setcounter{equation}{\value{eqsv}}


and (\ref{dI_dfklm}) or (\ref{dI_dfklm2}) is that integrals of the form

\begin{equation}\label{fourteen}
\int_s^{s_m} \beta^k(\sigma) \mu^k_{lm}(\sigma) \,\d \sigma
\end{equation}

do not appear in Equation (\ref{drad}).  Assuming that $\T(s,s_m)$ and
$I(s)$ have been saved during evaluation of $I(s_m)$ using Equation
(\ref{first}), Equation (\ref{drad}) is a single integral, not a double
integral, and is therefore less costly to evaluate than Equation
(\ref{dI_dfklm}).  From \h{wvs-101}, $\mu^k_{lm}(\sigma)$ is nonzero for
$\sigma$ in only four regions adjacent to $(\phi_k,\zeta_m)$, and the
path passes through only two of those.  When trapezoidal quadrature is
used, each evaluation of Equation (\ref{fourteen}) requires six
multiplies and two adds.  When 3-point Gauss-Legendre quadrature is used,
an additional eight multiplies and four adds are required.

Assuming the values of $\beta^k(s) \mu^k_{lm}(s)$ and Equation
(\ref{fourteen}) are saved and reused, finishing the evaluation of
Equation (\ref{dI_dfklm2}) requires an additional twelve multiplies and
five adds (Gauss-Legendre quadrature is not used for the outer
integral).  When Gauss-Legendre quadrature is used for the inner
integrals, the total cost of Equation (\ref{dI_dfklm2}) is 28 multiplies
and 13 adds, or 41 FLOPS.  The cost to evaluate Equation (\ref{drad}) is
twelve multiplies and three adds, or 15 FLOPS, a reduction of almost a
factor of three.  In addition to the reduction in computation cost, the
complexity of the {\tt rad\_tran\_m} module would be substantially
reduced.

If $\T(s,s_m) \alpha(s) = \frac{\partial \T(s,s_m)}{\partial s}$ is
substituted in Equation (\ref{first}), and then the integral is integrated
by parts, Equation (\ref{first}) becomes

\begin{equation}\label{fifteen}
I(s_m) = \T(s_0,s_m) \left( I(s_0) - B(s_0) \right) + B(s_m) -
 \int_{s_0}^{s_m} \T(s,s_m) \frac{\partial B(s)}{\partial s} \,\d s\,.
\end{equation}

Taking derivatives of this representation and using Equation (\ref{dT}) we
have

\begin{equation}\begin{split}\label{sixteen}
\frac{\partial I(s_m)}{\partial f^k_{lm}}
=\,&
 -\T(s_0,s_m) \left( I(s_0) - B(s_0) \right)
   \int_{s_0}^{s_m} \beta^k(\sigma) \mu^k_{lm}(\sigma) \,\d \sigma \,+ \\
\,&
 \int_{s_0}^{s_m} \T(s,s_m) \frac{\partial B(s)}{\partial s}
  \left( \int_s^{s_m} \beta^k(\sigma) \mu^k_{lm}(\sigma) \,\d \sigma
  \right) \,\d s \,.
\end{split}\end{equation}

Although somewhat simpler than Equation (\ref{dI_dfklm2}), it still has
the inner integral.  Its cost is 26 multiplies and ten adds, or 36
FLOPS.  Equation (\ref{drad}) reduces the cost compared to Equation
(\ref{sixteen}) by a factor of 2.4, and has the additional advantage that
$\frac{\partial B(s)}{\partial s}$ is not needed.  When Equation
(\ref{sixteen}) is approximated by a quadrature, $\frac{\partial
B(s)}{\partial s} \d s = \d B(s)$ is approximated by $\Delta B$ using a
central difference, which can introduce substantial error.

\label{lastpage}
\end{document}

% $Id$

% $Log$
% Revision 1.3  2010/12/23 20:32:04  vsnyder
% Add a summary, analysis of FLOPS
%
% Revision 1.2  2010/12/18 03:54:28  vsnyder
% Insert CVS stuff
%
