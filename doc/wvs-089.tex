\documentclass[11pt]{article}
\usepackage[fleqn]{amsmath}\textwidth 6.5in
\oddsidemargin -0.25in
%\evensidemargin -0.5in
\topmargin -0.25in
\textheight 9in

\newcommand{\docname}{\bf wvs-089}
\newcommand{\docdate}{25 January 2010}

\begin{document}

%\tracingcommands=1
\newlength{\hW} % heading box width
\newlength{\pW} % page number field width
\settowidth{\hW}{\docname}
\settowidth{\pW}{Page \pageref{lastpage}\ of \pageref{lastpage}}
\ifdim \pW > \hW \setlength{\hW}{\pW} \fi
\makeatletter
\def\@biblabel#1{#1.}
\newcommand{\ps@twolines}{%
  \renewcommand{\@oddhead}{%
    \docdate\hfill\parbox[t]{\hW}{{\hfill\docname}\newline
                          Page \thepage\ of \pageref{lastpage}}}%
\renewcommand{\@evenhead}{}%
\renewcommand{\@oddfoot}{}%
\renewcommand{\@evenfoot}{}%
}%
\makeatother
\pagestyle{twolines}

\vspace{-10pt}
\begin{tabbing}
\phantom{References: }\= \\
To: \>Bill, Van\\
Subject: \>TScat derivatives w.r.t. T and IWC\\
From: \>Van Snyder\\
Reference: \>wvs-066, wvs-076
\end{tabbing}

\parindent 0pt \parskip 6pt
\vspace{-10pt}

Denote the reference orbit geodetic angle $\phi$ by $\phi_i$.

The scattering points in the set $\{S_{jk}\}$, in the orbit plane, are on a
grid specified by $\{\phi_j\} \times \{\zeta_k\}$, where $\{\phi_j\}$ are orbit
geodetic angles.

The reference horizon is the line in the orbit plane from the scattering point
$S_{jk}$ and perpendicular to the line in the orbit plane from the center of
the Earth at $\phi_i$.  The angle $\chi$ is measured anticlockwise in the orbit
plane from that line.

The points in the set $\{A_{mn}\}$, also in the orbit plane, include points in
addition to $\{S_{jk}\}$, on a grid specified by $\{\phi_m\} \times
\{\zeta_n\}$.

$P_{jk}(\chi)$ is the Mie phase function, evaluated for IWC and T at $S_{jk}$;
$\chi = 0$ at the reference horizon.

The radiance arriving at $S_{jk}$ from direction $\chi$ in the orbit plane is
$I_{ijk}(\chi)$.

The radiance arriving at $S_{jk}$ from all directions in the orbit plane,
convolved with the Mie phase function, is

\begin{equation}
\overline{I}_{ijk} = \int_{-\pi}^\pi I_{ijk}(\chi) P_{jk}(\chi)\, \text{d} \chi
\,.
\end{equation}

The derivative of phase-convolved radiance at $S_{jk}$ with respect to some
quantity $x_{mn}$ at $A_{mn}$, where $x$ is either IWC or $T$, is


\begin{equation}\label{two}
\frac{\partial\overline{I}_{ijk}}{\partial x_{mn}} =
 \int_{-\pi}^\pi
  \left[ \frac{\partial I_{ijk}(\chi)}{\partial x_{mn}} P_{jk}(\chi) +
         I_{ijk}(\chi) \frac{\partial P_{jk}(\chi)}{\partial x_{mn}}
          \right] \, \text{d} \chi \,.
\end{equation}

If $S_{jk}$ is not at $A_{mn}$, then $\frac{\partial P_{jk}(\chi)}{\partial
x_{mn}} = 0$.  Therefore, when $S_{jk}$ is not at $A_{mn}$, Equation
(\ref{two}) can be simplified to

\begin{equation}
\frac{\partial\overline{I}_{ijk}}{\partial x_{mn}} =
 \int_{-\pi}^\pi
  \frac{\partial I_{ijk}(\chi)}{\partial x_{mn}} P_{jk}(\chi)\, \text{d} \chi \,.
\end{equation}

\label{lastpage}
\end{document}
% $Id$

% $Log$
