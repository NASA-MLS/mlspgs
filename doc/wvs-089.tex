\documentclass[11pt]{article}
\usepackage[fleqn]{amsmath}\textwidth 6.5in
\oddsidemargin -0.25in
%\evensidemargin -0.5in
\topmargin -0.25in
\textheight 9in

\newcommand{\docname}{\bf wvs-089r4}
\newcommand{\docdate}{28 April 2015}

\usepackage{graphicx}

\ifx\pdfoutput\undefined
  \pdfoutput=0
  \usepackage[hypertex,plainpages,hyperindex=true]{hyperref}
  \hypersetup{%
    hypertexnames=false%
  }
  % Specify the driver for the color package
  \ExecuteOptions{dvips}
  %\ExecuteOptions{xdvi}
\else
  \ifnum\pdfoutput>0
    \usepackage[pdftex,plainpages,hyperindex=true,pdfpagelabels]{hyperref}
    \hypersetup{%
      hypertexnames=false,%
      colorlinks=true,%
      linktocpage=true,%
    }
    % Specify the driver for the color package
    \ExecuteOptions{pdftex}
  \else
    \usepackage[hypertex,plainpages,hyperindex=true]{hyperref}
    \hypersetup{%
      hypertexnames=false%
    }
    % Specify the driver for the color package
    \ExecuteOptions{dvips}
    %\ExecuteOptions{xdvi}
  \fi
\fi

\hyperbaseurl{}
\newcommand\hr[1]{\href{#1.dvi}{dvi}, \href{#1.pdf}{pdf}}
\newcommand\h[1]{#1 (\hr{#1})}

\begin{document}

%\tracingcommands=1
\newlength{\hW} % heading box width
\newlength{\pW} % page number field width
\settowidth{\hW}{\docname}
\settowidth{\pW}{Page \pageref{lastpage}\ of \pageref{lastpage}}
\ifdim \pW > \hW \setlength{\hW}{\pW} \fi
\makeatletter
\def\@biblabel#1{#1.}
\newcommand{\ps@twolines}{%
  \renewcommand{\@oddhead}{%
    \docdate\hfill\parbox[t]{\hW}{{\hfill\docname}\newline
                          Page \thepage\ of \pageref{lastpage}}}%
\renewcommand{\@evenhead}{}%
\renewcommand{\@oddfoot}{}%
\renewcommand{\@evenfoot}{}%
}%
\makeatother
\pagestyle{twolines}

\newcommand{\TS}{T_\text{scat}}
\newcommand{\TSs}[1]{T_{\text{scat}_{#1}}}

\vspace{-10pt}
\begin{tabbing}
\phantom{References: }\= \\
To: \>Bill, Van\\
Subject: \>TScat derivatives w.r.t. T and IWC\\
From: \>Van Snyder\\
Reference: \>\h{wvs-066}, \h{wvs-076}
\end{tabbing}

\parindent 0pt \parskip 6pt
\vspace{-10pt}

Denote a reference orbit geodetic angle $\phi$ by $\phi_i$.  If the atmosphere
is not horizontally homogeneous along the orbit path, it might be desirable to
compute TScat tables for different values of $\phi$, for example, to accomodate
tables for different latitudes.

The scattering points in the set $\{S_{jk}\}$, in the orbit plane, are on a
grid specified by $\{\phi_j\} \times \{\zeta_k\}$, where $\{\phi_j\}$ are orbit
geodetic angles.

The reference horizon is the line in the orbit plane from the scattering point
$S_{jk}$ and perpendicular to the line in the orbit plane from the center of
the Earth at $\phi_i$.  The angle $\xi$ is measured anticlockwise in the orbit
plane from that line.

The points in the set $\{A_{mn}\}$, also in the orbit plane, include points in
addition to $\{S_{jk}\}$, on a grid specified by $\{\phi_m\} \times
\{\zeta_n\}$.

{\includegraphics[scale=0.85]{./wvs-089-pic}}

$P_{jk}(\xi)$ is the Mie phase function, evaluated for IWC and T at $S_{jk}$;
$\xi = 0$ at the reference horizon.

The radiance arriving at $S_{jk}$ from direction $\xi$ in the orbit plane is
$I_{ijk}(\xi)$.

The radiance arriving at $S_{jk}$ from all directions in the orbit plane,
convolved with the Mie phase function, is

\begin{equation}
\TSs{ijk} =
\overline{I}_{ijk} = \int_{-\pi}^\pi I_{ijk}(\xi) P_{jk}(\xi)\, \text{d} \xi
\,.
\end{equation}

The derivative of phase-convolved radiance at $S_{jk}$ with respect to some
quantity $x_{mn}$ at $A_{mn}$ is

\begin{equation}\label{two}
\frac{\partial\overline{I}_{ijk}}{\partial x_{mn}} =
 \int_{-\pi}^\pi
  \left[ \frac{\partial I_{ijk}(\xi)}{\partial x_{mn}} P_{jk}(\xi) +
         I_{ijk}(\xi) \frac{\partial P_{jk}(\xi)}{\partial x_{mn}}
          \right] \, \text{d} \xi \,.
\end{equation}

$I_{ijk}(\xi)$ depends upon whatever quantities are specified at $A_{mn}$
when $\TS$ is computed, i.e., throughout the entire atmosphere through
which rays scattered at $S_{jk}$ propagate; $P_{jk}(\xi)$ depends only
upon temperature and IWC at $S_{jk}$.

If $S_{jk}$ and $A_{mn}$ are the same point, both terms are necessary.
Otherwise $\frac{\partial P_{jk}(\xi)}{\partial x_{mn}} = 0$, and Equation
(\ref{two}) can be simplified to

\begin{equation}
\frac{\partial\overline{I}_{ijk}}{\partial x_{mn}} =
 \int_{-\pi}^\pi
  \frac{\partial I_{ijk}(\xi)}{\partial x_{mn}} P_{jk}(\xi)\, \text{d} \xi \,.
\end{equation}

In comments within the {\tt TScatSupport} module, terms of the form
$\frac{\partial}{\partial f^k_i}$ and $\frac{\partial}{\partial f^k_{lm}}$
appear.  The superscript $k$ refers to a species.  The subscript $i$
refers to a position on the path (black circles in the figure).  The
subscripts $lm$ refer to positions in the solution field ($A_{mn}$ in the
figure).

\label{lastpage}
\end{document}
% $Id$

% $Log$
% Revision 1.5  2015/04/28 23:39:03  vsnyder
% Explain subscripts in comments in TScatSupport
%
% Revision 1.4  2010/05/28 23:09:11  vsnyder
% Remove graphics extension so latex or pdflatex can select it
%
% Revision 1.3  2010/05/25 03:00:29  vsnyder
% Added description of dependence of derivatives upon atmospheric properties
%
% Revision 1.2  2010/05/25 02:20:45  vsnyder
% Added the picture
%
% Revision 1.1  2010/01/26 02:08:29  vsnyder
% Initial commit
%
