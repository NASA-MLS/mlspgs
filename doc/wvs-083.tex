\documentclass[11pt]{article}
\usepackage[fleqn]{amsmath}\textwidth 6.5in
\oddsidemargin -0.25in
%\evensidemargin -0.5in
\topmargin -0.25in
\textheight 9.0in

\newcommand{\docname}{\bf wvs-083r1}
\newcommand{\docdate}{18 December 2012}

\usepackage{graphicx}

\begin{document}

%\tracingcommands=1
\newlength{\hW} % heading box width
\newlength{\pW} % page number field width
\settowidth{\hW}{\docname}
\settowidth{\pW}{Page \pageref{lastpage}\ of \pageref{lastpage}}
\ifdim \pW > \hW \setlength{\hW}{\pW} \fi
\makeatletter
\def\@biblabel#1{#1.}
\newcommand{\ps@twolines}{%
  \renewcommand{\@oddhead}{%
    \docdate\hfill\parbox[u]{\hW}{{\hfill\docname}\newline
                          Page \thepage\ of \pageref{lastpage}}}%
\renewcommand{\@evenhead}{}%
\renewcommand{\@oddfoot}{}%
\renewcommand{\@evenfoot}{}%
}%
\makeatother
\pagestyle{twolines}

\vspace{-10pt}
\begin{tabbing}
\phantom{References: }\= \\
To: \>Van\\
Subject: \>2-d linear interpolation\\
From: \>Van Snyder\\
\end{tabbing}

\parindent 0pt \parskip 10pt
\vspace{-20pt}

Given data at the 11, 12, 21 and 22 points with coordinates of the $ij$ point
$= (x_i,y_j)$, interpolate using linear interpolation to $z(x,y)$.

{\includegraphics[scale=1.25]{./wvs-083-grid}}

\par
 
Let $\xi_1 = \frac{x_2-x}{x_2-x_1}$, $\xi_2 = \frac{x-x_1}{x_2-x_1} =
1-\xi_1$,
$\eta_1 = \frac{y_2-y}{y_2-y_1}$ and $\eta_2 = \frac{y-y_1}{y_2-y_1} =
1-\eta_1$,
$\xi = [\xi_1, \xi_2]$, $\eta = [\eta_1, \eta_2]$ and
$Z = \left| \begin{array}{cc} z_{11} & z_{12} \\ z_{21} & z_{22}\\
            \end{array} \right|$.

The interpolation can be done either by interpolating in $x$ to $z_c =
\xi_1 z_{11} + \xi_2 z_{12}$ and $z_d = \xi_1 z_{21} + \xi_2 z_{22}$
and then in $y$ to $z = \eta_1 z_c + \eta_2 z_d$, or interpolating in
$y$ to $z_a = \eta_1 z_{11} + \eta_2 z_{21}$ and $z_b = \eta_1 z_{12} +
\eta_2 z_{22}$ and then in $x$ to $z = \xi_1 z_a + \xi_2 z_b$.  The two
orders of interpolation are equivalent.  When expanded, both orders give

\begin{equation*}
z(x,y) = \xi_1 \eta_1 z_{11} + \xi_2 \eta_1 z_{12} +
\xi_1 \eta_2 z_{21} + \xi_2 \eta_2 z_{22},
\end{equation*}

which can be written in matrix-vector notation as

\begin{equation*}
z(x,y) = \eta^T Z \xi.
\end{equation*}

\label{lastpage}
\end{document}

% $Id$

% $Log$
% Revision 1.1  2010/05/28 23:06:35  vsnyder
% Initial commit
%
