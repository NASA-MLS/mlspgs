\documentclass[11pt]{article}
\usepackage{alltt}
\usepackage[fleqn]{amsmath}
\usepackage{floatflt}
\usepackage{graphicx}
\usepackage{longtable}
\usepackage[strings]{underscore}

\textwidth 6.5in
\oddsidemargin -0.25in
%\evensidemargin -0.5in
\topmargin -0.5in
\textheight 9in

\newcommand{\docname}{wvs-122}
\newcommand{\docdate}{8 April 2015}

\ifx\pdfoutput\undefined
  \pdfoutput=0
  \usepackage[hypertex,plainpages,hyperindex=true]{hyperref}
  \hypersetup{%
    hypertexnames=false%
  }
  % Specify the driver for the color package
  \ExecuteOptions{dvips}
  %\ExecuteOptions{xdvi}
\else
  \ifnum\pdfoutput>0
    \usepackage[pdftex,plainpages,hyperindex=true,pdfpagelabels]{hyperref}
    \hypersetup{%
      hypertexnames=false,%
      colorlinks=true,%
      linktocpage=true,%
    }
    % Specify the driver for the color package
    \ExecuteOptions{pdftex}
  \else
    \usepackage[hypertex,plainpages,hyperindex=true]{hyperref}
    \hypersetup{%
      hypertexnames=false%
    }
    % Specify the driver for the color package
    \ExecuteOptions{dvips}
    %\ExecuteOptions{xdvi}
  \fi
\fi

\hyperbaseurl{}
\newcommand\hr[1]{\href{#1.dvi}{dvi}, \href{#1.pdf}{pdf}}
\newcommand\h[1]{#1 (\hr{#1})}

\begin{document}

%\tracingcommands=1
\newlength{\hW} % heading box width
\newlength{\pW} % page number field width
\settowidth{\hW}{\bf\docname}
\settowidth{\pW}{Page \pageref{lastpage}\ of \pageref{lastpage}}
\ifdim \pW > \hW \setlength{\hW}{\pW} \fi
\makeatletter
\def\@biblabel#1{#1.}
\newcommand{\ps@twolines}{%
  \renewcommand{\@oddhead}{%
    \docdate\hfill\parbox[t]{\hW}{{\hfill\bf\docname}\newline
                          Page \thepage\ of \pageref{lastpage}}}%
\renewcommand{\@evenhead}{}%
\renewcommand{\@oddfoot}{}%
\renewcommand{\@evenfoot}{}%
}%
\makeatother
\pagestyle{twolines}

\newcommand{\TS}{T_\text{scat}}
\newcommand{\TSs}[1]{T_{\text{scat}_{#1}}}
\newcommand{\DB}{\Delta B}
\newcommand{\oDB}{\overline{\DB}}
\newcommand{\MT}{\mathcal{T}}
\newcommand{\hMT}{\MT^s}
\newcommand{\IF}[1]{\,\mathcal{A}_n\!\left(#1\right)} % Interpolation Function

\vspace{-10pt}
\begin{tabbing}
\phantom{References: }\= \\
To: \>Bill, Nathaniel, Paul, Michael, Van\\
Subject: \>Computing cross-orbit viewing geometry\\
From: \>Van Snyder\\
Reference: \> \h{wvs-121}
\end{tabbing}

\parindent 0pt \parskip 6pt
\vspace{-10pt}

Let $\mathbf{S}_m$ be the spacecraft position vector, and $\mathbf{T}_m$
be the tangent point position vector, for MIF $m$ of an arbitrary MAF. 
Then a vector that is normal to the plane defined by those vectors is

\begin{equation}
\mathbf{n} = \mathbf{S}_m \times \mathbf{T}_m \,.
\end{equation}

To calculate a position in the cross-orbit view from $\mathbf{S}_m$ to
$\mathbf{T}_m$, rotate $\mathbf{T}_m$ about $\mathbf{n}$ by an amount
specified by an angle $r$ taken from an {\tt xGrid} coordinate for the
basis of the quantity to be described (for example, the magnetic field),
using the method described in \h{wvs-121}.  Denote an unit vector in that
direction by $\hat{\mathbf{T}}_{mr}$.  Then the vector to the rotated
position, along the line of sight from $\mathbf{S}_m$ to $\mathbf{T}_m$,
is

\begin{equation}
\mathbf{T}_{mr} = |\mathbf{T}_m| \sec r \,.
\end{equation}

The line from $\mathbf{S}_m$ to $\mathbf{T}_m$ passes through the points
$\mathbf{T}_{mr}$ for all values of $r$.  The set of those lines for all
MIFs in a MAF define a ruled surface in which viewing takes place; this
surface would be a plane if the scan rate were constant.

If a quantity, such as magnetic field, is desired to be evaluated in that
surface, at specified altitudes, the geolocations of the desired points,
denoted $\mathbf{G}_{jr}$ for altitude $j$ could be calculated by
interpolating latitude and longitude in the subset $\{\mathbf{T}_{mr}\}$
for each $r$, from $\{|\mathbf{T}_{mr}|\}$ to the set of desired
altitudes $\{A_j\}$.

Interpolating in latitude and longitude is probably sufficiently accurate
near the equator, but might become significantly inaccurate near the most
nearly polar positions of the orbit.  Extrapolating in latitude and
longitude is probably always problematic.  Therefore, the vector
$\mathbf{G}_{jr}$ ought to be computed by interpolating the three
geocentric Cartesian coordinates (in ECR) of $\{\mathbf{T}_{mr}\}$ from
$\{|\mathbf{T}_{mr}|\}$ to $A_j$.

To carry out interpolation, the desired geolocations need to be expressed
in geocentric Cartesian coordinates (in ECR).  If the desired heights are
specified as geodetic heights instead of geocentric altitudes, these need
to be converted to geocentric altitudes.  This is done by calculating the
geocentric altitude of the Earth's surface, and then adding the geodetic
height.  This isn't quite right, as the geodetic height is measured along
the normal to the Earth's surface at the specified geodetic latitude,
while the geocentric altitude is measured along a line from the center of
the Earth, but the error is probably small.  If the resulting
co{\"o}rdinates are used to calculate a magnetic field, the error in the
magnetic field is almost surely small.

For a particular value of $r$, the set of points $\mathbf{T}_{mr}$ would
be on a straight line if the scan rate were constant.  If we extrapolate
linearly from the two shortest values of $|\mathbf{T}_{mr}|$ to altitudes
$A_j$ that are below the range of values of $|\mathbf{T}_{mr}|$, and from
the two longest values of $|\mathbf{T}_{mr}|$ to altitudes $A_j$ that are
above the range of values of $|\mathbf{T}_{mr}|$, this amounts to an
assumption that the scan rate would be constant outside the ranges of
altitudes $\{A_j\}$, if the tangent point altitudes were to encompass the
altitudes $\{A_j \cos r \}$.  In practice this has proven to be extreme,
as the scan rate for the first or last MIF, as indicated by the difference
between tangent heights, might be significantly different from the local
average scan rate.  This results in significant extrapolation error. 
Therefore, extrapolation is performed using the average slope of
components of $\mathbf{T}_{mr}$, computed from the top and bottom tangent
altitudes of the longest subsequence of MIFs in the MAF at monotonic
tangent altitudes. Denote the tangent altitudes with $h$, and the top and
bottom with superscript $t$ and $b$, respectively.  Then the difference
quotient used for extrapolation is

\begin{equation}
\frac{\mathbf{T}^t_{mr} - \mathbf{T}^b_{mr}}
     {h^t - h^b} \,.
\end{equation}

Alternatives are:

\begin{itemize}
\item Use a least-squares fits to a straight line.
\item Average the individual difference quotients between consecutive MIFs.
\item If the scan rate is nonuniform, use least-squares fits to two
straight lines and solve for where the break ought to be (this is a
nonlinear problem).
\item If the scan rate is nonuniform, use a least-squares fit to a
parabola and extrapolate with straight lines using the slopes of the
parabola at the ends of the range of the longest subsequence of MIFs in
the MAF at monotonic tangent altitudes.
\end{itemize}

\label{lastpage}
\end{document}

% $Id$

% $Log$
