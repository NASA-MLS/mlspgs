\documentclass[11pt]{article}
\usepackage[fleqn]{amsmath}\textwidth 6.5in
\oddsidemargin -0.25in
%\evensidemargin -0.5in
\topmargin -0.25in
\textheight 9in

\newcommand{\docname}{\bf wvs-069}
\newcommand{\docdate}{17 January 2008}

\begin{document}

%\tracingcommands=1
\newlength{\hW} % heading box width
\newlength{\pW} % page number field width
\settowidth{\hW}{\docname}
\settowidth{\pW}{Page \pageref{lastpage}\ of \pageref{lastpage}}
\ifdim \pW > \hW \setlength{\hW}{\pW} \fi
\makeatletter
\def\@biblabel#1{#1.}
\newcommand{\ps@twolines}{%
  \renewcommand{\@oddhead}{%
    \docdate\hfill\parbox[t]{\hW}{{\hfill\docname}\newline
                          Page \thepage\ of \pageref{lastpage}}}%
\renewcommand{\@evenhead}{}%
\renewcommand{\@oddfoot}{}%
\renewcommand{\@evenfoot}{}%
}%
\makeatother
\pagestyle{twolines}

\vspace{-10pt}
\begin{tabbing}
\phantom{References: }\= \\
To: \>Van\\
Subject: \>Inverting the radiative transfer equation\\
From: \>Van Snyder\\
%Reference:
\end{tabbing}

\parindent 0pt \parskip 6pt
\vspace{-10pt}

\renewcommand{\d}{\text{d}}

\section{Introduction}

The Microwave Limb Sounder instrument on the EOS Aura Spacecraft measures
microwave thermal emission from the limb of the atmosphere in five spectral
bands.  The measured emission is modeled by
the radiative transfer equation
\begin{equation}\label{one}
I(\nu) = \int_0^S \frac{\partial B}{\partial s}
 \exp\left( -\int_0^s \alpha(\sigma) \d\sigma \right) \d s
 \approx \sum_{i=1}^N \Delta B_i \mathcal{T}_i,
\end{equation}
%
where the instrument is located at $0$, $S$ is outside the atmosphere at the
end of the limb ray opposite the instrument (essentially $+\infty$), and in the
discrete approximation, summation runs along the line of sight. $I(\nu)$ is
microwave intensity at frequency $\nu$,
%
\begin{equation}\begin{split}\label{two}
&\Delta B_1 = (B_1 + B_2) / 2 \\
&\Delta B_i = (B_{i+1} - B_{i-1})/2 \\
&\Delta B_N = I_\infty(\nu) - (B_{N-1} + B_N)/2\,,
\end{split}\end{equation}
%
\begin{equation}
B_i = \frac{h\nu}{k\left(\exp\left( \frac{h\nu}{k T(s_i)} \right) -1 \right)}\,,
\end{equation}
%
$T$ is temperature, $I_\infty(\nu)$ is space radiance, the transmittance
$\mathcal{T}_i$ is given by
%
\begin{equation}\label{four}
\mathcal{T}_i = \exp\left( - \sum_{j=2}^i \Delta\delta_{j \rightarrow j-1} \, \right)\,,
\end{equation}
%
the incremental optical depth $\Delta\delta_{j \rightarrow j-1}$ is given by
%
\begin{equation}
\Delta\delta_{j \rightarrow j-1} =
 \int_{s_j}^{s_{j-1}} \alpha(s) \d s \text{, where}
\end{equation}
%
\begin{equation}\label{six}
\alpha(s) =
 \sum_{k=1}^{NS} \mu_k(s) \sum_{l=1}^{L_k} \beta_{kl}(P(s),T(s),\nu)\,,
\end{equation}
%
k is the chemical species index, $\mu_k$ is the $k^{\text{th}}$ chemical
species volume mixing ratio, $\nu$ is frequency, $\beta_{kl}$ is the
spectroscopic absorption cross-section function for the $k^{\text{th}}$
chemical species and its $l^{\text{th}}$ spectroscopic line, and $s$ is
position along the line of sight.

$\beta$ is a complicated function of temperature, pressure and frequency,
multiplied by the Fadeeva function $w$ (the generalization of erf to complex
argument) with its argument being a complicated function of temperature,
pressure and frequency for each spectral line within the frequency band to
which the instrument is sensitive.  In addition to this complication, pressure
and temperature are related by hydrostatic equilibrium, and path length depends
upon the index of refraction, which in turn depends upon temperature and
pressure.

Given $I(\nu)$ along several lines of sight intersecting different sets of
parcels of atmosphere $\{s_i\}$, we wish to solve for $T(s_i)$ and
$\mu_k(s_i)$. This is presently done using the discrete form of Equation
(\ref{one}) and a Newton method.

%\section{Alternative 1}
\section{Alternative}

Although both $T$ and $\mu_k$ appear nonlinearly in Equation (\ref{one}),
$\mathcal{T}_i$ appears linearly.  This suggests an algorithm that alternately
solves for linear and nonlinear parameters, as in e.g.\ the NIPALS algorithm of
Wold and Lyttkens:

\begin{enumerate}

\item Solve Equation (\ref{one}) for $\mathcal{T}_i$ for several lines of sight
  through each parcel of atmosphere and several frequencies.  This is a linear
  problem.

\item Within each parcel, using the logarithm of $\mathcal{T}_i$, and assuming
  linear interpolation from a standard grid to the points of intersections of
  the several lines of sight that pass through the parcel, solve Equations
  (\ref{four}-\ref{six}) for $\mu_k$ on the standard grid.  This is a linear
  problem.\label{mu}

\item Inserting the values of $\mu_k$, and again assuming linear interpolation
  from a standard grid to the points of intersections of the several lines of
  sight that pass through the parcel, solve Equation (\ref{one}) for $T$ on the
  standard grid.  This is a nonlinear problem.

\item Iterate until convergence.

\end{enumerate}

One significant disadvantage is that this is not a least-squares solution for
$\mu_k$, even if a least-squares solution is used in step \ref{mu}.  Since this
is not formulated as a Variable Projection method, there is some question
whether the iteration will converge, and if it does, whether it will converge
any faster or any better than a full nonlinear treatment.

\label{lastpage}
\end{document}

\section{Alternative 2}

Write Equation (\ref{one}) in the form
%
\begin{equation}
I_m(\nu) \approx \sum_{i=m}^N \Delta B_i \mathcal{T}_i =
\Delta B_m \mathcal{T}_m + I_{m+1}(\nu)
\end{equation}
%
where $I_1(\nu)$ is the measurement.  Putting $I_{m+1}(\nu)$ onto the left, and
taking logarithms, we have
%
\begin{equation}
\log( I_m(\nu) - I_{m+1}(\nu) ) = \log \Delta B_m + \log \mathcal{T}_m =
\log \Delta B_m - \int_{s_m}^{s_{m+1}} \alpha(s) \d s.
\end{equation}
%
Upon interpolating $\mu_k$ from a standard grid to the line of sight path, and
approximating the integral with a quadrature, the $\mu_k$ now appear
linearly~--- but $I_m(\nu)$ and $I_{m+1}(\nu)$ do not.  Can we solve this
problem by starting from the measurement $I_1(\nu)$, solving for $\mu_k$, $T$
and $I_2(\nu)$ in the first parcel, then proceeding by induction?

\label{lastpage}
\end{document}

In differential form, we have
\begin{equation}\label{seven}
\frac{\d I(\nu,s)}{\d s} = I\, \alpha(s) + B,\, I(\nu,0) = I_0(\nu)\,.
\end{equation}
%
The derivative of $B$ in Equation (\ref{one}) and the strange form of $\Delta
B_1$ and $\Delta B_N$ in Equation (\ref{two}) arise from integrating the
solution of Equation (\ref{seven}) by parts.

Although both $T$ and $\mu_k$ appear nonlinearly in Equation (\ref{one}), in
Equation (\ref{seven}), however, $\mu_k$ appears linearly.  As when using
Equation (\ref{one}), a Newton method would be used to invert the radiative
transfer.  This requires partial derivatives, which can be gotten by
integrating variational equations, \emph{viz}.
%
\begin{equation}\begin{split}
\frac{\d}{\d s} \frac{\partial I}{\partial \mu_k} =\,&
 I \frac{\partial \alpha}{\partial \mu_k} +
 \alpha \frac{\partial I}{\partial \mu_k} =
 I \sum_{l=1}^{L_k} \beta_{kl} + \alpha \frac{\partial I}{\partial \mu_k} =
 \left( I + \mu_k \frac{\partial I}{\partial \mu_k} \right)
 \sum_{l=1}^{L_k} \beta_{kl} \\
\frac{\d}{\d s} \frac{\partial I}{\partial T} =\,&
 I \frac{\partial \alpha}{\partial T} + \alpha \frac{\partial I}{\partial T} +
 \frac{\partial B}{\partial T} =
 \sum_{k=1}^{\text{NS}} \mu_k \sum_{l=1}^{L_k}
 \left( I \frac{\partial \beta_{kl}}{\partial T} +
       \beta_{kl} \frac{\partial I}{\partial T} \right) +
 \frac{\partial B}{\partial T} \,. \\
\end{split}\end{equation}

\label{lastpage}
\end{document}
% $Id$
