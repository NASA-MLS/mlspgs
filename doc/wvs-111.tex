\documentclass[11pt]{article}
\usepackage{alltt}
\usepackage[fleqn]{amsmath}
\usepackage{floatflt}
\usepackage{graphicx}
\usepackage{longtable}
\usepackage[strings]{underscore}

\textwidth 6.5in
\oddsidemargin -0.25in
%\evensidemargin -0.5in
\topmargin -0.5in
\textheight 9.1in

\newcommand{\docname}{wvs-111r1}
\newcommand{\docdate}{5 November 2018}

\ifx\pdfoutput\undefined
  \pdfoutput=0
  \usepackage[hypertex,plainpages,hyperindex=true]{hyperref}
  \hypersetup{%
    hypertexnames=false%
  }
  % Specify the driver for the color package
  \ExecuteOptions{dvips}
  %\ExecuteOptions{xdvi}
\else
  \ifnum\pdfoutput>0
    \usepackage[pdftex,plainpages,hyperindex=true,pdfpagelabels]{hyperref}
    \hypersetup{%
      hypertexnames=false,%
      colorlinks=true,%
      linktocpage=true,%
    }
    % Specify the driver for the color package
    \ExecuteOptions{pdftex}
  \else
    \usepackage[hypertex,plainpages,hyperindex=true]{hyperref}
    \hypersetup{%
      hypertexnames=false%
    }
    % Specify the driver for the color package
    \ExecuteOptions{dvips}
    %\ExecuteOptions{xdvi}
  \fi
\fi

\hyperbaseurl{}
\newcommand\hr[1]{\href{#1.dvi}{dvi}, \href{#1.pdf}{pdf}}
\newcommand\h[1]{#1 (\hr{#1})}

\begin{document}

%\tracingcommands=1
\newlength{\hW} % heading box width
\newlength{\pW} % page number field width
\settowidth{\hW}{\bf\docname}
\settowidth{\pW}{Page \pageref{lastpage}\ of \pageref{lastpage}}
\ifdim \pW > \hW \setlength{\hW}{\pW} \fi
\makeatletter
\def\@biblabel#1{#1.}
\newcommand{\ps@twolines}{%
  \renewcommand{\@oddhead}{%
    \docdate\hfill\parbox[t]{\hW}{{\hfill\bf\docname}\newline
                          Page \thepage\ of \pageref{lastpage}}}%
\renewcommand{\@evenhead}{}%
\renewcommand{\@oddfoot}{}%
\renewcommand{\@evenfoot}{}%
}%
\makeatother
\pagestyle{twolines}

\newcommand{\TS}{T_\text{scat}}
\newcommand{\TSs}[1]{T_{\text{scat}_{#1}}}
\newcommand{\DB}{\Delta B}
\newcommand{\oDB}{\overline{\DB}}
\newcommand{\MT}{\mathcal{T}}
\newcommand{\hMT}{\MT^s}
\newcommand{\IF}[1]{\,\mathcal{A}_n\!\left(#1\right)} % Interpolation Function

\vspace{-10pt}
\begin{tabbing}
\phantom{References: }\= \\
To: \>Bill, Nathaniel, Paul, Alyn, Van\\
Subject: \>Effect of minimum value of $\lambda$ on convergence\\
From: \>Van Snyder\\
Reference: \> \h{wvs-035}, \h{wvs-110}
\end{tabbing}

\parindent 0pt \parskip 6pt
\vspace{-10pt}

Some phases of some retrievals have significant convergence problems. 
Below are shown plots of the logarithm of the residual (black), and the
minimum norm expected for a linear model (red), for a retrieval of chunk
56 of 2005d037, using three profiles per chunk with no overlap.  The
retrieval was stopped after CorePlusR3; subsequent phases converged
rapidly and gracefully.

This plot shows the result if no minimum is put on the Levenberg-Marquardt
stabilization parameter, $\lambda$:

\includegraphics[scale=0.6,angle=270]{wvs-111-MinLambda=0}

\newpage

At Fred Krogh's suggestion, a facility was added to specify a minimum
value for $\lambda$ on the {\tt retrieve} specification in the {\tt l2cf},
using a {\tt lambdaMin} field.  This plot shows the result if the minimum
value of $\lambda$ in InitUTH is set to 0.25, and the minimum value of
$\lambda$ in CorePlusR3 is set to 0.5.  An experiment with a minimum value
of 0.1 did not significantly improve CorePlusR3.

\includegraphics[scale=0.6,angle=270]{wvs-111-MinLambda-not=0}

In both cases, the values of {\tt maxJ} were 8 for InitPTAN, 10 for
InitR2, 10 for FinalPTAN, 10 for InitRHI, 30 for InitUTH, and 20 for
CorePlusR3.  The value for {\tt ftolerance} was 0.1 in all phases.

There are several possible causes for the failure to converge gracefully
in the first case.  One is that the constraints are not tight enough,
allowing the state vector to enter regions where the forward model is not
physically realistic.  Another is that the problem is strongly nonlinear. 
In this case, ignoring Hessian information that would be used in a full
Newton iteration (as described in \h{wvs-110} might produce undesirable
Newton moves.  Another possibility is that the gradient moves do not have
a useful length.  In 2006, Herb suggested a strategy to calculate a
gradient move length.  This is described in \h{wvs-035}.

Another is that the Jacobian is ill conditioned.  Forcing a minimum value
for $\lambda$ improves the condition.

An alternative to cope with ill conditioning is to factor the problem
using Householder transformations, instead of forming normal equations,
followed by Cholesky factorization.  The latter method, which is presently
used, squares the condition number.  The former method requires about
twice as much work as the latter method.  It is not presently supported by
{\tt MatrixModule_0}, {\tt MatrixModule_1}, or {\tt RetrievalModule}.

\label{lastpage}
\end{document}

% $Id$

% $Log$
% Revision 1.1  2012/08/31 23:37:31  vsnyder
% Original commit
%
