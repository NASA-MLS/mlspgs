\documentclass[11pt]{article}
\usepackage[fleqn]{amsmath}\textwidth 6.5in
\oddsidemargin -0.25in
%\evensidemargin -0.5in
\topmargin -0.25in
\textheight 9in

\newcommand{\docname}{\bf wvs-107r2}
\newcommand{\docdate}{14 December 2011}

\usepackage{graphicx}
\usepackage{longtable}
\usepackage{multirow}
\usepackage{rotating}

\ifx\pdfoutput\undefined
  \pdfoutput=0
  \usepackage[hypertex,plainpages,hyperindex=true]{hyperref}
  \hypersetup{%
    hypertexnames=false%
  }
  % Specify the driver for the color package
  \ExecuteOptions{dvips}
  %\ExecuteOptions{xdvi}
\else
  \ifnum\pdfoutput>0
    \usepackage[pdftex,plainpages,hyperindex=true,pdfpagelabels]{hyperref}
    \hypersetup{%
      hypertexnames=false,%
      colorlinks=true,%
      linktocpage=true,%
    }
    % Specify the driver for the color package
    \ExecuteOptions{pdftex}
  \else
    \usepackage[hypertex,plainpages,hyperindex=true]{hyperref}
    \hypersetup{%
      hypertexnames=false%
    }
    % Specify the driver for the color package
    \ExecuteOptions{dvips}
    %\ExecuteOptions{xdvi}
  \fi
\fi

\hyperbaseurl{}
\newcommand\hr[1]{\href{#1.dvi}{dvi}, \href{#1.pdf}{pdf}}
\newcommand\h[1]{#1 (\hr{#1})}

\begin{document}

%\tracingcommands=1
\newlength{\hW} % heading box width
\newlength{\pW} % page number field width
\settowidth{\hW}{\docname}
\settowidth{\pW}{Page \pageref{lastpage}\ of \pageref{lastpage}}
\ifdim \pW > \hW \setlength{\hW}{\pW} \fi
\makeatletter
\def\@biblabel#1{#1.}
\newcommand{\ps@twolines}{%
  \renewcommand{\@oddhead}{%
    \docdate\hfill\parbox[t]{\hW}{{\hfill\docname}\newline
                          Page \thepage\ of \pageref{lastpage}}}%
\renewcommand{\@evenhead}{}%
\renewcommand{\@oddfoot}{}%
\renewcommand{\@evenfoot}{}%
}%
\makeatother
\pagestyle{twolines}

\newcommand{\TS}{T_\text{scat}}
\newcommand{\TSs}[1]{T_{\text{scat}_{#1}}}

\vspace{-10pt}
\begin{tabbing}
\phantom{References: }\= \\
To: \>Bill, Van\\
Subject: \>MIF Extinction\\
From: \>Van Snyder\\
% Reference: \>
\end{tabbing}

\parindent 0pt \parskip 6pt
\vspace{-10pt}

To solve for MIF extinction in MLSL2, several changes are made in the
interface, {\tt ForwardModelWrap\-per}, between the {\tt Retrieve} and {\tt
ForwardModel} modules.

MIF extinction is a minor-frame quantity with a vertical grid the same as
$P_\text{tan}$.  Extinction in the forward model is a geophysical
quantity with a vertical grid specified by the {\tt L2CF}.

When the retriever calls the forward model, MIF extinction is mapped to
forward-model extinction by interpolating vertically in MIF extinction
from $\{\zeta_{\text{tan}_i} = - \log_{10} P_{\text{tan}_i}\,|\, i = 1,
\dots, N_m\}$, where $N_m$ is the number of MIFs per MAF, to the
geophysical $\zeta_g$ grid specified for forward-model extinction, and
then replicating the result for the $N_\phi$ solution profiles specified
in the {\tt L2CF}.

Within a retriever state vector MIF extinction block for a single band,
the element $E^R_{ci,n}$ has associated channel $(c)$ and MAF $(n)$
numbers, and coordinates $(\phi_\text{tan}, \zeta_\text{tan})_i$.  $E^R$
is not a coherent quantity.  The first subscript of the {\tt VALUES}
component of the vector quantity is $c + (i-1) N_C$, where $N_C$ is the
number of channels, and the second is $n$.

Within a forward-model state vector extinction block, the element
$E^F_{cg,j}$ has an associated channel $(c)$ number and geophysical
$(\phi_j,\,\zeta_g)$ coordinates, the values and numbers of which are
specified in the {\tt L2CF}.  $E^F$ is a coherent quantity.  The first
subscript of the {\tt VALUES} component of the vector quantity is $c +
(g-1) N_C$, where $N_C$ is the number of channels, and the second is $j$.

In the interface between the retriever and forward model, blocks of the
retriever state vector that are not MIF extinction blocks are moved to
the forward model state vector.  Let $P_r$ denote the lowest retrieved
pressure and $\zeta_r = -\log_{10} P_r$.  Blocks that are MIF extinction
blocks are mapped to the forward model state vector using
%
\renewcommand{\arraystretch}{2}
\begin{equation}\label{one}
E^F_{cg,j} = \left\{
\begin{array}{llll}
 \overline{E}^R_{c,n}(\zeta_g)
  & \zeta_g \geq \zeta_r & j = 1, \dots, N_\phi & g = 1, \dots, N_\zeta \\
 \overline{E}^R_{c,n}(\zeta_r) \times 10^{-2(\zeta_g - \zeta_r)}
  & \zeta_g < \zeta_r    & j = 1, \dots, N_\phi & g = 1, \dots, N_\zeta \\
\end{array} \right.
\end{equation}
%
where $\overline{E}^R_{c,n}(\zeta)$ indicates interpolation from
$\{E^R_{ci,n}\}$, in $\{\zeta_i\}$ only, to $\zeta$, $N_\phi$ is the
number of $\phi$ instances specified for $E^F$ in the {\tt L2CF},
$N_\zeta$ is the number of $\zeta$ ($-\log_{10}$ pressure) levels
specified in the {\tt L2CF} for the forward model's extinction state
vector block,  and
%
\begin{equation}
10^{-2(\zeta_g - \zeta_r)} = \left( \frac{P_g}{P_r} \right)^2 \,.
\end{equation}

Notice that $E^F_{cg,j} = E^F_{cg,1},\, j= 1, \dots, N_\phi$.

The layout of the extinction blocks of the Jacobian used by the retriever
and those of the Jacobian computed by the forward model are shown in the
following table for a single band, wherein $N_M$ is the number of MAFs. 
This layout is replicated vertically, i.e., another set of rows, for each
band.

\newpage
\begin{longtable}{c|ccccc|p{0.25in}|ccccc|}
\multicolumn{1}{r}{}
& \multicolumn{5}{c}{Retriever Jacobian blocks} & \multicolumn{1}{c}{} &
  \multicolumn{5}{c}{Forward Model Jacobian blocks} \\[-10pt]
\multicolumn{1}{r}{}
& \multicolumn{5}{c}{$N_M$} & \multicolumn{1}{c}{} &
  \multicolumn{5}{c}{$N_\phi$} \\
  \cline{2-6}\cline{8-12}
% Rotating package doesn't render properly in xdvi
      & $K^R_{11}$ &0 &0 &  & 0     & & $K^F_{11}$    & $K^F_{12}$    & $K^F_{13}$    & $\cdots$ & $K^F_{1 N_\phi}$ \\ 
      &0 & $K^R_{22}$ &0 &  & 0     & & $K^F_{21}$    & $K^F_{22}$    & $K^F_{23}$    & $\cdots$ & $K^F_{2 N_\phi}$ \\ 
$N_M$ &0 &0 & $K^R_{33}$ &  & 0     & & $K^F_{31}$    & $K^F_{32}$    & $K^F_{33}$    & $\cdots$ & $K^F_{3 N_\phi}$ \\ 
      &  &  &  & $\cdots$   &       & & $\cdots$      & $\cdots$      & $\cdots$      & $\cdots$ & $\cdots$ \\    
      &0 &0 &0 &0 & $K^R_{N_M N_M}$ & & $K^F_{N_M 1}$ & $K^F_{N_M 2}$ & $K^F_{N_M 3}$ & $\cdots$ & $K^F_{ N_M N_\phi}$ \\
  \cline{2-6}\cline{8-12}
\end{longtable}

For retriever block $K^R_{nn}$, $n$ is the index of the MAF. Each
$K^R_{nn}$ block has shape $(N_C \times N_m,1)$.  Within each $K^R_{nn}$
block, third and fourth subscripts $(ci)$ conspire to denote the row
number (first subscript of the {\tt VALUES} component).  The subscript
$c$ is the channel number and $i$ is the MIF $(\phi_\text{tan},
\zeta_\text{tan})_i$ number.  The row number of an element is $c +
n_c(i-1)$.  The column number (second subscript) is always 1.

For forward model block $K^F_{nj}$, $n$ is the index of the MAF, while
$j$ is the index of solution coordinate $\phi_j$.  Each $K^F_{nj}$ block
has shape $(N_C \times N_m,N_\zeta)$.  Within each $K^F_{ij}$ block,
there are three further subscripts $(cig)$.  The first two, $c$ and $i$
conspire to denote the row number (first subscript) as above, while the
third $(g)$ denotes the column number (second subscript, corresponding
to $\zeta_g$) of an element of the {\tt VALUES} component.

After the forward model returns, blocks of $K^F$ are mapped to blocks of
$K^R$ using

\begin{equation}
\begin{array}{lll}
K^R_{nn,ci} = \sum_{g=1}^{N_\zeta}
 \left( \sum_{j=1}^{N_\phi} K^F_{nj,cig} \right)
 10^{-2(\zeta_g - \zeta_i)} & i = 1, \dots, N_m & c = 1, \dots, N_C\,. \\
\end{array}
\end{equation}

Forward-model radiance blocks $I^F_{ci,n}$ and retriever radiance blocks
$I^R_{ci,n}$ are organized as for $E^R_{ci,n}$.  Forward-model radiance
blocks $I^F_{ci,n}$ are mapped to retriever radiance blocks $I^R_{ci,n}$
using

\begin{equation}
\begin{array}{lll}
I^R_{ci,n} = I^F_{ci,n} + \sum_{g=1}^{N_\zeta}
 \left( \sum_{j=1}^{N_\phi} K^F_{nj,cig} \right)
 \left( E^R_{ci,n} 10^{-2(\zeta_g - \zeta_i)} -
  E^F_{cg,1} \right) & i = 1, \dots, N_m & c = 1, \dots, N_C \,, \\
\end{array}
\end{equation}

where $E^R$ is MIF extinction from the previous iteration of the
retriever, and $E^F$ is extinction input to the forward model, as
calculated by Equation (\ref{one}).  Remember that $E^F_{cg,j} =
E^F_{cg,1},\, j = 1, \dots, N_\phi$.

\label{lastpage}
\end{document}
% $Id$

% $Log$
% Revision 1.2  2011/12/13 23:39:55  vsnyder
% Include number of signals in row count for Jacobian layout
%
% Revision 1.1  2011/12/13 01:42:21  vsnyder
% Initial commit
%
