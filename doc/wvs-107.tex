\documentclass[11pt]{article}
\usepackage[fleqn]{amsmath}\textwidth 6.5in
\oddsidemargin -0.25in
%\evensidemargin -0.5in
\topmargin -0.25in
\textheight 9in

\newcommand{\docname}{\bf wvs-107r1}
\newcommand{\docdate}{13 December 2011}

\usepackage{graphicx}
\usepackage{longtable}
\usepackage{multirow}
\usepackage{rotating}

\ifx\pdfoutput\undefined
  \pdfoutput=0
  \usepackage[hypertex,plainpages,hyperindex=true]{hyperref}
  \hypersetup{%
    hypertexnames=false%
  }
  % Specify the driver for the color package
  \ExecuteOptions{dvips}
  %\ExecuteOptions{xdvi}
\else
  \ifnum\pdfoutput>0
    \usepackage[pdftex,plainpages,hyperindex=true,pdfpagelabels]{hyperref}
    \hypersetup{%
      hypertexnames=false,%
      colorlinks=true,%
      linktocpage=true,%
    }
    % Specify the driver for the color package
    \ExecuteOptions{pdftex}
  \else
    \usepackage[hypertex,plainpages,hyperindex=true]{hyperref}
    \hypersetup{%
      hypertexnames=false%
    }
    % Specify the driver for the color package
    \ExecuteOptions{dvips}
    %\ExecuteOptions{xdvi}
  \fi
\fi

\hyperbaseurl{}
\newcommand\hr[1]{\href{#1.dvi}{dvi}, \href{#1.pdf}{pdf}}
\newcommand\h[1]{#1 (\hr{#1})}

\begin{document}

%\tracingcommands=1
\newlength{\hW} % heading box width
\newlength{\pW} % page number field width
\settowidth{\hW}{\docname}
\settowidth{\pW}{Page \pageref{lastpage}\ of \pageref{lastpage}}
\ifdim \pW > \hW \setlength{\hW}{\pW} \fi
\makeatletter
\def\@biblabel#1{#1.}
\newcommand{\ps@twolines}{%
  \renewcommand{\@oddhead}{%
    \docdate\hfill\parbox[t]{\hW}{{\hfill\docname}\newline
                          Page \thepage\ of \pageref{lastpage}}}%
\renewcommand{\@evenhead}{}%
\renewcommand{\@oddfoot}{}%
\renewcommand{\@evenfoot}{}%
}%
\makeatother
\pagestyle{twolines}

\newcommand{\TS}{T_\text{scat}}
\newcommand{\TSs}[1]{T_{\text{scat}_{#1}}}

\vspace{-10pt}
\begin{tabbing}
\phantom{References: }\= \\
To: \>Bill, Van\\
Subject: \>MIF Extinction\\
From: \>Van Snyder\\
% Reference: \>
\end{tabbing}

\parindent 0pt \parskip 6pt
\vspace{-10pt}

To solve for MIF extinction in MLSL2, several changes are made in the
interface, {\tt ForwardModelWrap\-per}, between the {\tt Retrieve} and {\tt
ForwardModel} modules.

MIF extinction is a minor-frame quantity, one per L1 MAF, with a vertical
grid the same as $P_\text{tan}$.  Extinction in the forward model is a
geophysical quantity, one per solution profile, with a vertical grid
specified by the {\tt L2CF}.

When the retriever calls the forward model, MIF extinction is mapped to
forward-model extinction by interpolating vertically in MIF extinction
from $\{\zeta_{\text{tan}_i} = - \log_{10} P_{\text{tan}_i}\,|\, i = 1,
\dots, N_M\}$, where $N_M$ is the number of MIFs per MAF, to the
geophysical $\zeta_g$ grid specified for forward-model extinction, and
then replicating the result for the $N_\phi$ solution profiles within the
$\phi$ window for the invocation of the forward model.

Within a retriever state vector MIF extinction block, the element $E^R_i$
has associated coordinates $(\phi_\text{tan}, \zeta_\text{tan})_i$.

Within a forward-model state vector extinction block, the element
$E^F_{jg}$ has associated geophysical $(\phi_j,\,\zeta_g)$ coordinates,
the values and numbers of which are specified in the configuration.

In the interface between the retriever and forward model, blocks of the
retriever state vector that are not MIF extinction blocks are moved to
the forward model state vector.  Let $P_r$ denote the lowest retrieved
pressure and $\zeta_r = -\log_{10} P_r$.  Blocks that are MIF extinction
blocks are mapped to the forward model state vector using

\renewcommand{\arraystretch}{2}
\begin{equation}\label{one}
E^F_{jg} = \left\{
\begin{array}{llll}
 \overline{E}^R(\zeta_g)
  & \zeta_g \geq \zeta_r & j = 1, \dots, N_\phi & g = 1, \dots, N_\zeta \\
 \overline{E}^R(\zeta_r) \times 10^{-2(\zeta_g - \zeta_r)}
  & \zeta_g < \zeta_r    & j = 1, \dots, N_\phi & g = 1, \dots, N_\zeta \\
\end{array} \right.
\end{equation}
%
where $\overline{E}^R(\zeta)$ indicates interpolation from $\{E^R_i\}$ to
$\zeta$, $N_\phi$ is the width of the forward model's $\phi$ window,
$N_\zeta$ is the number of $\zeta$ ($-\log_{10}$ pressure) levels
specified in the configuration for the forward model's extinction state
vector block,  and

\begin{equation}
10^{-2(\zeta_g - \zeta_r)} = \left( \frac{P_g}{P_r} \right)^2 \,.
\end{equation}

Notice that $E^F_{jg} = E^F_{1g},\, j= 1, \dots, N_\phi$.  This is done
because the forward model wants to have the same $\phi$-basis for all
quantities.

The layout of the extinction blocks of the Jacobian used by the retriever
and those of the Jacobian computed by the forward model are shown in the
following table, wherein $N_S$ is the number of signals being processed.

\newpage
\begin{longtable}{c|ccccc|p{0.25in}|ccccc|}
\multicolumn{1}{r}{}
& \multicolumn{5}{c}{Retriever Jacobian blocks} & \multicolumn{1}{c}{} &
  \multicolumn{5}{c}{Forward Model Jacobian blocks} \\[-10pt]
\multicolumn{1}{r}{}
& \multicolumn{5}{c}{$N_M$} & \multicolumn{1}{c}{} &
  \multicolumn{5}{c}{$N_\phi$} \\
  \cline{2-6}\cline{8-12}
% Rotating package doesn't render properly in xdvi
\ifnum\pdfoutput>0
\multirow{5}{*}{
\begin{rotate}{90}\hspace*{-0.35in}$N_M \times N_S$\end{rotate}}
\fi
      & $K^R_{11}$ &0 &0 &  & 0     & & $K^F_{11}$    & $K^F_{12}$    & $K^F_{13}$    & $\cdots$ & $K^F_{1 N_\phi}$ \\ 
\ifnum\pdfoutput<1 $N_M$ \fi
      &0 & $K^R_{22}$ &0 &  & 0     & & $K^F_{21}$    & $K^F_{22}$    & $K^F_{23}$    & $\cdots$ & $K^F_{2 N_\phi}$ \\ 
\ifnum\pdfoutput<1 $\times$ \fi
      &0 &0 & $K^R_{33}$ &  & 0     & & $K^F_{31}$    & $K^F_{32}$    & $K^F_{33}$    & $\cdots$ & $K^F_{3 N_\phi}$ \\ 
\ifnum\pdfoutput<1 $N_S$ \fi
      &  &  &  & $\cdots$   &       & & $\cdots$      & $\cdots$      & $\cdots$      & $\cdots$ & $\cdots$ \\    
      &0 &0 &0 &0 & $K^R_{N_M N_M}$ & & $K^F_{N_M 1}$ & $K^F_{N_M 2}$ & $K^F_{N_M 3}$ & $\cdots$ & $K^F_{ N_M N_\phi}$ \\
  \cline{2-6}\cline{8-12}
\end{longtable}

For retriever block $K^R_{ii}$, $i$ is the index of $(\phi_\text{tan},
\zeta_\text{tan})_i$.  Each $K^R_{ii}$ block has shape $(N_C,1)$, where
$N_C$ is the number of channels.  Within each $K^R_{ii}$ block a third
subscript $(c)$ denotes the row (channel) number of an element.

For forward model block $K^F_{ij}$, $i$ is the index of
$\zeta_{\text{tan}_i}$, while $j$ is the index of solution coordinate
$\phi_j$.  Each $K^F_{ij}$ block has shape $(N_C,N_\zeta)$.  Within each
$K^F_{ij}$ block, third and fourth subscripts $(cg)$ denote the row
(channel) number and column ($\zeta_g$) number of an element.

After the forward model returns, blocks of $K^F$ are mapped to blocks of
$K^R$ using

\begin{equation}
\begin{array}{lll}
K^R_{ii,c} = \sum_{g=1}^{N_\zeta}
 \left( \sum_{j=1}^{N_\phi} K^F_{ij,cg} \right)
 10^{-2(\zeta_g - \zeta_i)} & i = 1, \dots, N_M & c = 1, \dots, N_C\,, \\
\end{array}
\end{equation}
%
and forward-model radiances $I^F$ are mapped to retriever radiances $I^R$
using

\begin{equation}
\begin{array}{lll}
I^R_{i,c} = I^F_{i,c} + \sum_{g=1}^{N_\zeta}
 \left( \sum_{j=1}^{N_\phi} K^F_{ij,cg} \right)
 \left( E^R_{\zeta_i} 10^{-2(\zeta_g - \zeta_i)} -
  E^F_{1g} \right) & i = 1, \dots, N_M & c = 1, \dots, N_C \,, \\
\end{array}
\end{equation}

where $E^R$ is MIF extinction from the previous iteration of the
retriever, and $E^F$ is extinction input to the forward model, as
calculated by Equation (\ref{one}).  Remember that $E^F_{jg} = E^F_{1g},\,
j = 1, \dots, N_\phi$.

\label{lastpage}
\end{document}
% $Id$

% $Log$
% Revision 1.1  2011/12/13 01:42:21  vsnyder
% Initial commit
%
