\documentclass[11pt]{article}
\usepackage[fleqn]{amsmath}
\usepackage{longtable}

\textwidth 6.5in
\oddsidemargin 0in
%\evensidemargin -0.5in
\topmargin 0in
\textheight 9in

\newcommand{\docname}{\bf wvs-107r13}
\newcommand{\docdate}{24 July 2013}

\usepackage{graphicx}
\usepackage{longtable}
\usepackage{multirow}
\usepackage{rotating}

\ifx\pdfoutput\undefined
  \pdfoutput=0
  \usepackage[hypertex,plainpages,hyperindex=true]{hyperref}
  \hypersetup{%
    hypertexnames=false%
  }
  % Specify the driver for the color package
  \ExecuteOptions{dvips}
  %\ExecuteOptions{xdvi}
\else
  \ifnum\pdfoutput>0
    \usepackage[pdftex,plainpages,hyperindex=true,pdfpagelabels]{hyperref}
    \hypersetup{%
      hypertexnames=false,%
      colorlinks=true,%
      linktocpage=true,%
    }
    % Specify the driver for the color package
    \ExecuteOptions{pdftex}
  \else
    \usepackage[hypertex,plainpages,hyperindex=true]{hyperref}
    \hypersetup{%
      hypertexnames=false%
    }
    % Specify the driver for the color package
    \ExecuteOptions{dvips}
    %\ExecuteOptions{xdvi}
  \fi
\fi

\hyperbaseurl{}
\newcommand\hr[1]{\href{#1.dvi}{dvi}, \href{#1.pdf}{pdf}}
\newcommand\h[1]{#1 (\hr{#1})}

\begin{document}

%\tracingcommands=1
\newlength{\hW} % heading box width
\newlength{\pW} % page number field width
\settowidth{\hW}{\docname}
\settowidth{\pW}{Page \pageref{lastpage}\ of \pageref{lastpage}}
\ifdim \pW > \hW \setlength{\hW}{\pW} \fi
\makeatletter
\def\@biblabel#1{#1.}
\newcommand{\ps@twolines}{%
  \renewcommand{\@oddhead}{%
    \docdate\hfill\parbox[t]{\hW}{{\hfill\docname}\newline
                          Page \thepage\ of \pageref{lastpage}}}%
\renewcommand{\@evenhead}{}%
\renewcommand{\@oddfoot}{}%
\renewcommand{\@evenfoot}{}%
}%
\makeatother
\pagestyle{twolines}

\newcommand{\TS}{T_\text{scat}}
\newcommand{\TSs}[1]{T_{\text{scat}_{#1}}}

\vspace{-10pt}
\begin{tabbing}
\phantom{References: }\= \\
To: \>Bill, Van\\
Subject: \>MIF Extinction\\
From: \>Van Snyder\\
% Reference: \>
\end{tabbing}

\parindent 0pt \parskip 6pt
\vspace{-10pt}

\section{The problem}

To solve for MIF extinction in MLSL2, several changes are made in the
interface, {\tt ForwardModel\-Wrap\-per}, between the {\tt Retrieve} and
{\tt ForwardModel} modules.

MIF extinction is a minor-frame quantity with a vertical grid the same as
$P_\text{tan}$.  Extinction in the forward model is a geophysical
quantity with a vertical grid specified by the {\tt L2CF}.

When the retriever calls the forward model, MIF extinction is mapped to
forward-model extinction by interpolating vertically in MIF extinction
from $\{\zeta_{\text{tan}_i} = - \log_{10} P_{\text{tan}_i}\,|\, i = 1,
\dots, N_m\}$, where $N_m$ is the number of MIFs per MAF, to the
geophysical $\zeta_g$ grid specified for forward-model extinction, and
then replicating the result for the $N_\phi$ solution profiles specified
in the {\tt L2CF}.

Within a retriever state vector MIF extinction block for a single band,
the element $E^R_{i,n}$ has an associated MAF $(n)$ number, and
coordinates $(\phi_\text{tan}, \zeta_\text{tan})_i$.  $E^R$ is not a
coherent quantity.

Within a forward-model state vector extinction block, the element
$E^F_{g,j}$ has associated geophysical $(\phi_j,\,\zeta_g)$ coordinates,
the values and numbers of which are specified in the {\tt L2CF}.  $E^F$
is a coherent quantity.

In the interface between the retriever and forward model, blocks of the
retriever state vector that are not MIF extinction blocks are moved to
the forward model state vector.  Let $P_r$ denote the lowest retrieved
pressure and $\zeta_r = -\log_{10} P_r$.  Blocks that are MIF extinction
blocks are mapped to the forward model state vector using
%
\renewcommand{\arraystretch}{2}
\begin{equation}\label{one}
E^F_{g,j} = \left\{
\begin{array}{llll}
 \overline{E}^R_{n}(\zeta_g)
  & \zeta_g \geq \zeta_r & j = 1, \dots, N_\phi & g = 1, \dots, N_\zeta \\
 \overline{E}^R_{n}(\zeta_r) \times 10^{-2(\zeta_g - \zeta_r)}
  & \zeta_g < \zeta_r    & j = 1, \dots, N_\phi & g = 1, \dots, N_\zeta \\
\end{array} \right.
\end{equation}
%
where $\overline{E}^R_{n}(\zeta)$ indicates interpolation from
$\{E^R_{i,n}\}$, in $\{\zeta_i\}$ only, to $\zeta$, $N_\zeta$ is the
number of $\zeta$ ($-\log_{10}$ pressure) levels specified in the {\tt
L2CF} for the forward model's extinction state vector block,  and
%
\begin{equation}
10^{-2(\zeta_g - \zeta_r)} = \left( \frac{P_g}{P_r} \right)^2 \,.
\end{equation}

Notice that $E^F_{g,j} = E^F_{g,1},\, j= 1, \dots, N_\phi$, where
$N_\phi$ is the number of $\phi$ instances specified for $E^F$ in the
{\tt L2CF}.

The layout of the extinction blocks of the Jacobian used by the retriever
and those of the Jacobian computed by the forward model are shown in the
following table for a single band, wherein $N_M$ is the number of MAFs. 
This layout is replicated vertically, i.e., another set of rows, for each
band.

\newpage
\begin{longtable}{c|ccccc|p{0.25in}|ccccc|}
\multicolumn{1}{r}{}
& \multicolumn{5}{c}{Retriever Jacobian blocks} & \multicolumn{1}{c}{} &
  \multicolumn{5}{c}{Forward Model Jacobian blocks} \\[-10pt]
\multicolumn{1}{r}{}
& \multicolumn{5}{c}{$N_M$} & \multicolumn{1}{c}{} &
  \multicolumn{5}{c}{$N_\phi$} \\
  \cline{2-6}\cline{8-12}
% Rotating package doesn't render properly in xdvi
      & $K^R_{11}$ &0 &0 &  & 0     & & $K^F_{11}$    & $K^F_{12}$    & $K^F_{13}$    & $\cdots$ & $K^F_{1 N_\phi}$ \\ 
      &0 & $K^R_{22}$ &0 &  & 0     & & $K^F_{21}$    & $K^F_{22}$    & $K^F_{23}$    & $\cdots$ & $K^F_{2 N_\phi}$ \\ 
$N_M$ &0 &0 & $K^R_{33}$ &  & 0     & & $K^F_{31}$    & $K^F_{32}$    & $K^F_{33}$    & $\cdots$ & $K^F_{3 N_\phi}$ \\ 
      &  &  &  & $\cdots$   &       & & $\cdots$      & $\cdots$      & $\cdots$      & $\cdots$ & $\cdots$ \\    
      &0 &0 &0 &0 & $K^R_{N_M N_M}$ & & $K^F_{N_M 1}$ & $K^F_{N_M 2}$ & $K^F_{N_M 3}$ & $\cdots$ & $K^F_{ N_M N_\phi}$ \\
  \cline{2-6}\cline{8-12}
\end{longtable}

For retriever block $K^R_{nn}$, $n$ is the index of the MAF. Each
$K^R_{nn}$ block has shape $(N_C \times N_m,N_m)$, where $N_C$ is the
number of channels.  Within each $K^R_{nn}$ block, third and fourth
subscripts $(ci)$ conspire to denote the row number (first subscript of
the {\tt VALUES} component).  The subscript $c$ is the channel number and
$i$ is the MIF $(\phi_\text{tan}, \zeta_\text{tan})_i$ number.  The row
number of an element is $c + N_c(i-1)$.  The column number within the
block (second subscript of the {\tt VALUES} component) is given by a fifth
subscript here, always $i$.

For forward model block $K^F_{nj}$, $n$ is the index of the MAF, while $j$
is the index of profile solution coordinate $\phi_j$.  Each $K^F_{nj}$
block has shape $(N_C \times N_m,N_\zeta)$.  Within each $K^F_{ij}$ block,
there are three further subscripts $(cig)$.  The first two, $c$ and $i$
conspire to denote the row number (first subscript) as above, while the
third $(g)$ denotes the column number (second subscript, corresponding to
$\zeta_g$) of an element of the {\tt VALUES} component.

After the forward model returns, blocks of $K^F$ are mapped to blocks of
$K^R$ using

\begin{equation}\label{three}
\begin{array}{lll}
K^R_{nn,cii} = \sum_{g=1}^{N_\zeta}
 \left( \sum_{j=1}^{N_\phi} K^F_{nj,cig} \right)
 10^{-2(\zeta_g - \zeta_i)} & i = 1, \dots, N_m & c = 1, \dots, N_C\,. \\
\end{array}
\end{equation}

Within forward-model radiance blocks $I^F_{ci,n}$ and retriever radiance
blocks $I^R_{ci,n}$, $c$ and $i$ are combined using $c + N_C (i-1)$ to
form the first subscript of the {\tt VALUES} component.  Forward-model
radiance blocks $I^F_{ci,n}$ are mapped to retriever radiance blocks
$I^R_{ci,n}$ using

\begin{equation}\label{four}
\begin{array}{lll}
I^R_{ci,n} = I^F_{ci,n} + \sum_{g=1}^{N_\zeta}
 \left( \sum_{j=1}^{N_\phi} K^F_{nj,cig} \right)
 \left( E^R_{i,n} 10^{-2(\zeta_g - \zeta_i)} -
  E^F_{g,1} \right) & i = 1, \dots, N_m & c = 1, \dots, N_C \,, \\
\end{array}
\end{equation}

where $E^R$ is MIF extinction from the previous iteration of the
retriever, and $E^F$ is extinction input to the forward model, as
calculated by Equation (\ref{one}).  Remember that $E^F_{g,j} =
E^F_{g,1},\, j = 1, \dots, N_\phi$.

\section{Changes in mlsl2 and l2cf}

Two new quantity types, MIFExtinction and MIFExtinctionv2, are introduced.
These are not molecules, so it is impossible to tell the forward model to
use them.

Quantities of type vmr, with their molecules being either extinction or
extinctionv2, and quantities of type either MIFExtinction or
MIFExtinctionv2, are specified by the l2cf as quantities in the state
vector template.

A switch named {\tt TransformMIFExtinction} is provided in the {\tt
ForwardModel} statement in the l2cf to request transformations between
extinction and MIF extinction.

If the switch is set, before calling the forward model, the forward model
wrapper destroys any extinction molecules it finds in the state vector
that correspond to MIF extinction quantities, then re-creates them by
transformation from corresponding MIF extinction quantities using
Equation (\ref{one}).  If it finds MIF extinction quantities without
corresponding extinction molecules, it announces an error.

If the switch isn't set, the forward model wrapper does not destroy
extinction molecules, acting on the assumption that the retriever is
retrieving extinction, not MIF extinction.

Upon return from the forward model, the forward model wrapper destroys
MIF extinction quantities in the state vector, and associated columns of
the Jacobian.  If the switch is set it re-creates MIF extinction
quantities by transforming from corresponding extinction molecules using
Equations (\ref{four}) and (\ref{three}), and destroys extinction
molecules, and associated columns of the Jacobian, that correspond to MIF
extinction quantities.  It does not destroy extinction molecules and
associated columns of the Jacobian if they do not correspond to MIF
extinction quantities.  This allows to retrieve extinction and MIF
extinction simultaneously, provided they do not correspond.

If the switch isn't set, the forward model wrapper does not re-create MIF
extinction quantities, and associated columns of the Jacobian.  This
prevents the retriever from retrieving MIF extinction.

The correspondence between an extinction molecule and a MIF extinction
quantity is established if they are both $\nu^2$ quantities, or neither
one is, and they are for the same radiometer.

An alternative to having the switch is to determine whether the state
vector has any MIF extinction quantities.  This requires one to make more
substantial changes in the l2cf to select whether to retrieve extinction
or MIF extinction, i.e., one must change the state vector template.

\section{Notation}

\vskip -10pt
\renewcommand{\arraystretch}{1}
\begin{longtable}{c|l||c|l}
Symbol   & Usage              & Symbol    & Usage \\
\hline
         &                    &           & \\[-11pt]
$E^F$    & FWM Extinction     & $E^R$     & Retriever (MIF) extinction\\
$I^F$    & FWM radiance       & $I^R$     & Retriever radiance \\
$K^F$    & FWM Jacobian block & $K^R$     & Retriever Jacobian block \\
$N_c$    & Number of channels & $N_m$     & Number of MIFs per MAF\\
$N_\phi$ & Number of profiles & $N_\zeta$ & Pressure levels per profile \\
$c$      & Channel index      & $g$       & FWM vertical index \\
$i$      & MIF index          & $j$       & Profile index       \\
$n$      & MAF number         & $r$       & Lowest retrieved pressure index \\
\end{longtable}

\section{L2CF changes}

\vspace*{-8pt}
One must be careful when constructing the l2cf to retrieve MIF extinction.

\vspace*{-5pt}

\begin{itemize}\parskip 3pt

\item There must be both extinction and MIF extinction quantities in the
  retrieval state vector for bands in which MIF extinction is being
  retrieved, because otherwise the forward model and MIF extinction
  transformation have nowhere to record the Jacobians of radiance with
  respect to extinction and MIF extinction, respectively.  This is because
  the Jacobian column space is defined by reference to the retrieval state
  vector.  This might eventually be handled automatically.

\item Profile extinction for the band in which MIF extinction is being
  retrieved needs to be a one-profile quantity, even (especially) in 2-D
  retrievals.  Use an {\tt hGrid} something like \\
  {\tt hGridSingle: hGrid, type=explicit, geodAngle=[0 degrees]}\\
  for extinction quantities.

\item It's important to use a rapidly-decaying profile for MIF extinction
  apriori, e.g.,

{\tt\small\begin{verbatim}
    Fill, quantity=apriori.totalMIFExtinctionr3,method=profile, $
      profileValues = [ 1000mb: 0.01, 681mb:0.01, 464mb : 0.008, 316mb : 0.004, $
      215mb : 0.001, 147mb : 0.0005, 100mb : 0.0003, 68mb:0.0002,46mb: 0.0001,  $
      32mb:7.0e-5,22mb:4.0e-5,15mb:9.3e-6, 10mb:2.7e-6, 1mb:1.0e-8, $
      0.1mb:1.5e-10,0.01mb:2.0e-12,0.001mb:3.3e-14, 0.0001mb:3.0e-16, $
      0.00001mb:3.0e-18], ptanQuantity=state.ptanGHz
\end{verbatim}}

\item Apriori precision for extinction needs to be huge, e.g.,\\[3pt]
  {\tt Fill,
  quantity=aprioriPrecision.totalextinctionv2r3,method=explicit, \$ \\
     /spread, explicitValues=1.0e6}\\[3pt]
  to prevent the Newton method from moving in the extinction direction. 
  The Jacobian of radiance with respect to extinction is deleted by the
  MIF extinction transformation, but the extinction apriori needs to be
  present to prevent the overall Jacobian from being singular. Apriori
  precision for MIF extinction should be what apriori precision for
  extinction would be if MIF extinction were not being retrieved, e.g.,

{\tt\small\begin{verbatim}
    Fill, quantity=aprioriprecision.totalMIFExtinctionV2R3,method=profile, $
      profileValues = [ 1000mb: 1.0E-7, 681mb:0.0005, 464mb : 0.001, 316mb : 0.001, $
      215mb : 0.0006, 147mb : 0.0002, 100mb : 0.0001, 68mb:0.0001, 46mb: 7.0e-5, $
      32mb:4.0e-5, 22mb:2.0e-5, 15mb:5.0e-6, 10mb:2.0e-6, 0.00001mb:1.0e-6], $
      ptanQuantity=state.ptanGHz
\end{verbatim}}

\item There must be a lowest retrieved pressure quantity in the ``extra''
  vector, e.g.,\\
  {\tt lowestRetrievedPressure: Quantity, type=lowestRetrievedPressure \\
     Fill, quantity=state.lowestRetrievedPressure, method=explicit, \$\\
       explicitValues= -2.6 zeta }\\
  It isn't needed in the retrieval state vector, or apriori vector.

\item Mask bits need to be set in MIF extinction (and indeed in any
  retrieved minor-frame quantity) so as not to try to retrieve where there
  are no radiances.  There is a {\tt radianceQuantity} field of the subset
  command from which to get masks, since one cannot predict where L1 will
  mark bad radiances.  The quantity to be masked must be a minor-frame
  quantity.  The only other fields allowed are {\tt additional}, {\tt
  channel}, {\tt ignore} (which reverses the sense of {\tt channel}, and
  {\tt instances}.  If the radiance quantity channels are masked first,
  the {\tt channels} field isn't necessary.  If no channel field appears,
  all are selected (or ignored).  The selected channels should be
  consistent with the signal used for retrieval.

\item Tikhonov regularization needs to be turned off for MIF extinction,
  and for extinction in bands where MIF extinction is being retrieved.

\item The {\tt TransformMIFExtinction} switch needs to be set in the
forward model configuration.

\end{itemize}

\label{lastpage}
\end{document}
% $Id$

% $Log$
% Revision 1.12  2012/05/10 23:18:15  vsnyder
% Add requirement to set TransformMIFExtinction switch
%
% Revision 1.11  2012/05/01 22:34:35  vsnyder
% Describe subset-using-radiances command
%
% Revision 1.10  2012/04/20 23:46:00  vsnyder
% Add l2cf considerations
%
% Revision 1.9  2012/04/18 22:52:45  vsnyder
% Add table of notations
%
% Revision 1.8  2012/02/24 02:22:01  vsnyder
% Correct rows -> columns
%
% Revision 1.7  2012/02/24 00:45:59  vsnyder
% Explain strategy in mlsl2 and l2cf
%
% Revision 1.6  2012/02/02 02:01:56  vsnyder
% Correct K^R_{nn} indexing description
%
% Revision 1.5  2012/01/25 00:02:52  vsnyder
% Revise indexing
%
% Revision 1.4  2011/12/21 21:07:35  vsnyder
% Correct some typos
%
% Revision 1.3  2011/12/15 01:54:53  vsnyder
% Revise indexing
%
% Revision 1.2  2011/12/13 23:39:55  vsnyder
% Include number of signals in row count for Jacobian layout
%
% Revision 1.1  2011/12/13 01:42:21  vsnyder
% Initial commit
%
