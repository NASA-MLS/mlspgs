\makeatletter\let\ifGm@compatii\relax\makeatother
\documentclass[landscape]{beamer}
\usepackage[nofancy,notoday]{rcsinfo}
%\usepackage{pstcol,pst-plot,pst-char,pst-node}
\usepackage{color}
\usepackage{amsmath}
%\usepackage{moreverb}
%\usepackage{multicol}
\usepackage{graphicx}
\usepackage{floatflt}
%
\renewcommand{\b}{\mathbf}
\newcommand{\T}{^{T}}
\newcommand{\cs}[1]{$_{\text{#1}}$}
\newcommand{\inv}{^{\mathrm -1}}
\newcommand{\cp}[1]{$^{\text{#1}}$}
\newcommand{\degsym}{\ensuremath{^\circ}}
\newcommand{\newframe}[2][]{\begin{frame}\frametitle{\hfill #2 \hfil}}

%
% -----------------------------------------------------------------------------
%
\title{Accurate and efficient computation of Mie parameters}
\subtitle{wvs-071}
\author{Van Snyder}
\date{22 February 2008}
\titlegraphic{\includegraphics[width=1.0in]{eos_mls_logo_onpink}}

\begin{document}
\sloppy
%
% -----------------------------------------------------------------------------
%
\begin{frame}
 \titlepage
\end{frame}
%
% -----------------------------------------------------------------------------
%
\newframe[Quantities]{Quantities to be computed}
The quantities to be computed are
%
\begin{equation}\begin{split}\label{def}
\beta_{c\_e} =\,&\, 2 \pi \int_0^\infty r^2\, n(r)\, \xi_e(r)\, \text{d} r \\
\beta_{c\_s} =\,&\, 2 \pi \int_0^\infty r^2\, n(r)\, \xi_s(r)\, \text{d} r \\
P(\theta) = \,&\,  \frac\pi{\beta_{c\_s}} \int_0^\infty
 r^2\, n(r)\, \xi_s(r)
  \frac{p_0(\theta,r)}
       {\frac12 \int_0^\pi p_0(\theta^\prime,r) \sin\theta^\prime\,
        \text{d} \theta^\prime} \,\text{d} r
\end{split}\end{equation}
%
and their derivatives with respect to temperature and ice water content (IWC).
The quantity $n(r)$ is the McFarquhar and Heymsfield 1997 (MH97) particle-size
distribution function
%
\begin{equation}
n(r) = N_1\, r \exp(-2 \alpha r) +
      \frac{N_2}r\, \exp\left(-\frac{\gamma^2}2 \right) \text{ where }
\gamma = \frac{\log(2r) - \mu}{\sigma}\,.
\end{equation}
%
$N_1$ and $\alpha$ depend upon IWC, and $N_2$, $\mu$ and $\sigma$ depend upon
temperature and IWC, but not on $\theta$ or $r$.
%
\end{frame}
%
% -----------------------------------------------------------------------------
%
\newframe[Efficiencies]{Mie efficiencies}
The Mie efficiencies $\xi_e$ and $\xi_s$ are given by
%
\begin{equation}\begin{split}
\xi_e(r) =\,& \frac2{\chi^2} \sum_{n=1}^\infty ( 2 n + 1 ) ( \Re a_n + \Re b_n ) \\
\xi_s(r) =\,& \frac2{\chi^2} \sum_{n=1}^\infty ( 2 n + 1 ) ( |a_n|^2 + |b_n|^2 ) \\
\end{split}\end{equation}
%
where $a_n$ and $b_n$ are given by
%
\begin{equation}\begin{split}\label{anbn}
a_n = \,& \frac{\hat a_n \Re W_n - \Re W_{n-1}}
               {\hat a_n W_n - W_{n-1}} \text{ where }
 \hat a_n = \frac{A_n}m + \frac{n}\chi\, \text{, and}\\
b_n = \,& \frac{\hat b_n \Re W_n - \Re W_{n-1}}
               {\hat b_n W_n - W_{n-1}} \text{ where }
 \hat b_n = m A_n + \frac{n}\chi\,,
                \\
\end{split}\end{equation}
%
wherein $\chi = \frac{2 \pi r}\lambda$, $m$ is the complex index of refraction,
which depends upon temperature, $r$ is the particle radius, and $\lambda$ is
the wavelength of radiation.
%
\end{frame}
%
% -----------------------------------------------------------------------------
%
\newframe[Hankel]{The Bessel and Hankel functions}
%
\small
The quantities $W_n$ and $A_n$ that appear in the definitions of $a_n$ and
$b_n$ in Equation (\ref{anbn}) are defined by Ulaby, Moore and Fung using the
recurrences
%
\begin{equation}\begin{split}
W_n =\,& \left( \frac{2n-1}\chi\right) W_{n-1} - W_{n-2} \\
W_0 =\,& \sin\chi + i \cos\chi\,,\, W_1 = \cos\chi - i \sin\chi\,, \text{ and}\\
A_n =\,& -\frac{n}{m\chi} + \left( \frac{n}{m\chi} - A_{n-1} \right)^{-1}
\,,\, A_0 = \cot m\chi\,. \\
\end{split}\end{equation}
%
The solutions of these recurrences, along with their initial conditions, are
\begin{equation}\label{An}
W_n = \chi h^{(2)}_n(\chi) = \chi \left( j_n(\chi) - i y_n(\chi) \right)
 \text{ and }
A_n = -\frac{n}{m\chi} + \frac{j_{n-1}(m\chi)}{j_n(m\chi)}
\end{equation}
%
where $h^{(2)}_n$ is the spherical Hankel function of the second kind, and
$j_n$ and $y_n$ are spherical Bessel functions of the first and second kind,
respectively.  The spherical Hankel and Bessel functions are related to
cylindrical Hankel and Bessel functions of half-integer order.
%
\end{frame}
%
% -----------------------------------------------------------------------------
%
\newframe{The Bessel and Hankel functions (cont.)}

\small
The recurrence for $W_n$ is a second-order recurrence.  As for second-order
differential equations, it has two fundamental solutions, any linear
combination of which is also a solution.  Even if one starts with the first
fundamental solution, using finite arithmetic causes some of the second one to
be added in, so the result becomes dominated by the solution that increases in
the direction of recurrence.  For $n>\chi$, $j_n(\chi)$ decreases and
$y_n(\chi)$ increases as $n$ increases.  Thus recurrence for $W_n$ in the
forward direction will eventually compute $y_n(\chi) - i\, y_n(\chi)$, while
recurrence in the backward direction will eventually compute $j_n(\chi) - i\,
j_n(\chi)$.  That is, the recurrence is unstable in both directions for
$n>\chi$; fortunately the real and imaginary parts can be computed separately
by recurrence in opposite directions.

The recurrence for $A_n$ can be derived from two recurrences of the same form
as the one for $W_n$.  Therefore, recurrence in the forward direction computes
$-\frac{n}{m\chi} + \frac{y_{n-1}(m\chi)}{y_n(m\chi)}$.  Backward recurrences
can be evaluated by a Miller algorithm, i.e., by recurring forward until
overflow, followed by backward recurrence from arbitrary values, say zero and
one, to the given boundary conditions, followed by normalization.

\end{frame}
%
% -----------------------------------------------------------------------------
%
\newframe[Phase]{The integrated phase function}
%
The quantity $p_0(\theta,r)$ that appears in Equation (\ref{def}) is defined by
%
\begin{equation}\begin{split}
p_0(\theta,r) =\,&
\left| \sum_{n=1}^\infty \frac{2n+1}{n(n+1)}
 \left(a_n \frac{\text{d} P_n^1}{\text{d}\theta} +
       b_n \frac{P_n^1}{\sin\theta}\right) \right|^2 + \\
\,&
\left| \sum_{n=1}^\infty \frac{2n+1}{n(n+1)}
 \left(a_n \frac{P_n^1}{\sin\theta} +
       b_n \frac{\text{d} P_n^1}{\text{d}\theta}\right) \right|^2 \\
\end{split}\end{equation}
%
where $P^1_n$ is the associated Legendre function of the first kind and order 1
of $\cos\theta$. By multiplying out the squares, $p_0(\theta,r)$ can be written
as a sum of products of functions of $\theta$ and $r$.  Integrating
term-by-term on $\theta$ shows that
%
\begin{equation}\label{C}
\frac12\, \int_0^\pi p_0(\theta^\prime,r) \sin\theta^\prime\,
 \text{d} \theta^\prime\ =
 \frac{\chi^2}2\, \xi_s(r)\,,
\end{equation}
%
so the definition of $P(\theta)$ in Equation (\ref{def}) simplifies to
%
\begin{equation}
P(\theta) = \frac{\lambda^2}{2 \pi \beta_{c\_s}}
 \int_0^\infty n(r) \, p_0(\theta,r) \, \text{d} r\,.
\end{equation}
%
\end{frame}
%
% -----------------------------------------------------------------------------
%
\newframe[Legendre]{The associated Legendre function}
%
The function  $P_n^1$ satisfies the recurrence
%
\begin{equation}\begin{split}\label{Legendre}
P_n^1(\cos\theta) =\,&
  \left(\frac{2n-1}{n-1}\right) \cos\theta\, P_{n-1}^1(\cos\theta) -
  \left(\frac{n}{n-1}\right) P_{n-2}^1(\cos\theta)\,,\\
P_0^1(\cos\theta) =\,& 0\,,\, P_1^1(\cos\theta) = \sin\theta\,,
\end{split}\end{equation}
%
which is stable in the forward direction.  Its derivative is related
to the function by the relation
\begin{equation}
\sin\theta \frac{\text{d} P_n^1(\cos\theta)}{\text{d}\theta} =
 n \cos\theta\, P_n^1(\cos\theta) - (n+1) P_{n-1}^1(\cos\theta)
\end{equation}
%
which is computationally more stable than differentiating Equation
(\ref{Legendre}).
%
\end{frame}
%
% -----------------------------------------------------------------------------
%
\newframe[Quadrature]{Quadratures}
%
The integrals in Equation (\ref{def}) and their derivatives with respect to
temperature and IWC are evaluated by an automatic adaptive quadrature package
from the Math77 library, known as {\tt DINT}.

This package uses quadrature formulae published in Mathematics of Computation
in 1968 by T.~N.~L.~Patterson, who recently retired from the Queens University
of Belfast.  The first formula in the family is the three-point Gauss formula. 
The second is the seven-point Kronrod formula, which is the optimal extension
of the three-point Gauss formula.  Each additional formula after Kronrod's is
an optimal extension of the previous one.  These additional extensions, and the
methods to calculate them, are Patterson's contribution.

The $k^{\text{th}}$ formula has $n = 2^k-1$ points, and algebraic order
$\frac{3n+1}2$, that is, it integrates polynomials up to degree $\frac{3n+1}2$
exactly, assuming infinite-precision arithmetic.  The package {\tt DINT} uses
formulae with up to 255 points.
%
\end{frame}
%
% -----------------------------------------------------------------------------
%
\newframe{Quadratures (cont.)}

The term ``automatic adaptive'' means that the quadrature package DINT
automatically selects the stepsize and formula to use.

It starts by applying a 15-point formula.  It gets error estimates by comparing
three and seven, then seven and fifteen point formulae.  If the error estimate
is too big (or it's working on the initial interval) it forms forward and
backward differences across the interval.  The differences ought to decrease if
there is no untoward behavior.  If the differences indicate a problem, DINT
applies a transformation to remove a singularity (if the problem is at an end
of the interval) or subdivide (and maybe apply a transformation) at the point
of the problem otherwise.

If the error estimate is large but the differences don't indicate any obvious
behavior, either a higher-order formula is used, or the interval size is
reduced.  This continues until the entire interval is covered.

For multidimensional problems, one-dimensional rules are composed.
%
\end{frame}
%
% -----------------------------------------------------------------------------
%
\newframe{Quadratures (cont.)}

Before discovering the simplification of the integral in Equation (\ref{C}),
the integrals defining $P(\theta)$ and its derivatives were evaluated as double
integrals (which no package other than {\tt DINT} is able to tackle in the form
given in Equation (\ref{def})).  Up to six million inner integrand values were
required for 12-digit answers.  After the simplification, the most expensive
integrals require roughly 4000 integrand values.
%
\end{frame}
%
% -----------------------------------------------------------------------------
%
\newframe[Results]{Plots of $P(\theta)$}
%
The figures below show polar plots of $P(\theta)$ at ten temperatures between
$-100^\circ$C and $-15^\circ$C, ten values of IWC on a logarithmic scale from
$10^{-5}$ to 1 gm/m$^3$, and ten frequencies within each band.

{\parskip 0pt\hfill\includegraphics[scale=0.4,angle=90]{wvs-071-plot}\hfill}

%
\end{frame}
%
% -----------------------------------------------------------------------------
%
\newframe[Derivatives]{Derivatives}
%
It is straight-forward to compute derivatives of $\beta_{c\_e}$,
$\beta_{c\_s}$, and $P(\theta)$, defined by Equation (\ref{def}), in terms of
derivatives of $n(r)$, $\xi_e(r)$, $\xi_s(r)$, $m$ and $p_0(\theta,r)$.  The
derivatives of $n(r)$ and $m$ are tedious but straight-forward to compute.  The
derivatives of $\xi_e(r)$, $\xi_s(r)$, and $p_0(\theta,r)$ are straight-forward
to compute from derivatives of $A_n$, $a_n$ and $b_n$.  The derivatives of
$n(r)$, $m$, $\xi_e(r)$, $\xi_s(r)$, and $p_0(\theta,r)$ are written in detail
in my memo {\tt wvs-066}.  $A_n$, $a_n$ and $b_n$ do not depend upon IWC, and
their only dependence upon temperature comes via the dependence of $m$ upon
temperature, but their derivatives with respect to $m$ are nontrivial. 
Starting from the definitions of $A_n$ in Equation (\ref{An}), and of $a_n$ and
$b_n$ in Equation (\ref{anbn}), after much tedious algebra, and a
simplification achieved by using the Wronskian relation (a.k.a.\ Casorati
Determinant)
%
\begin{equation}
j_n(\chi) y_{n-1}(\chi) - j_{n-1}(\chi) y_n(\chi) = \frac1{\chi^2}\,
\end{equation}
%
we have \dots
%
\end{frame}
%
% -----------------------------------------------------------------------------
%
\newframe{Derivatives (cont.)}
%
\begin{equation}\begin{split}
\frac{\partial A_n}{\partial m} =\,&
  \frac1{m^2} \left( \frac{n}\chi + 2 n \hat b_n - \chi \hat b_n^2 \right)
   - \chi = \frac{n(n+1)}{m^2\chi} - \chi(A_n^2+1) \\
\frac{\partial a_n}{\partial m} = \,&
 \frac{i}{\chi^2}\,
 \frac1{\left(\hat a_n h^{(2)}_n(\chi) - h^{(2)}_{n-1}(\chi)\right)^2}
       \, \frac{\partial \hat a_n}{\partial m}\, \text{, where}\\
\frac{\partial \hat a_n}{\partial m} =\,&
  \frac1m \frac{\partial A_n}{\partial m} - \frac1{m^2} A_n\, \text{, and}\\
\frac{\partial b_n}{\partial m} = \,&
 \frac{i}{\chi^2}\,
 \frac1{\left(\hat b_n h^{(2)}_n(\chi) - h^{(2)}_{n-1}(\chi)\right)^2}
       \, \frac{\partial \hat b_n}{\partial m}\,, \text{ where} \\
\frac{\partial \hat b_n}{\partial m} =\,&
  m \frac{\partial A_n}{\partial m} + A_n\,.
\end{split}\end{equation}
%
An obvious application of the chain rule gives derivatives with respect to
temperature.
\end{frame}
%
% ----------------------------------------------------------------------------
%
\end{document}

%%% Local Variables: 
%%% mode: latex
%%% TeX-master: t
%%% End: 

% $Log$
% Revision 1.3  2019/05/23 00:33:21  vsnyder
% Convert to beamer class
%
% Revision 1.2  2016/02/05 00:29:40  vsnyder
% A few trivial corrections
%
% Revision 1.1  2008/06/11 20:14:55  vsnyder
% Initial commit
%
% Revision 1.1  2008/06/11 20:14:55  vsnyder
% Initial commit
%
% Revision 1.7  2001/06/16 00:57:56  vsnyder
% More stuff about least-squares, some clean-up
%
% $Id$
