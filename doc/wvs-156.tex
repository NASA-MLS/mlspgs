\documentclass[11pt]{article}
\usepackage{alltt}
\usepackage[fleqn]{amsmath}
\usepackage{floatflt}
\usepackage{graphicx}
\usepackage{longtable}
\usepackage[strings]{underscore}

\textwidth 6.5in
\oddsidemargin -0.25in
%\evensidemargin -0.5in
\topmargin -0.5in
\textheight 9.1in

\newcommand{\docname}{wvs-156}
\newcommand{\docdate}{2 October 2019}

\ifx\pdfoutput\undefined
  \pdfoutput=0
  \usepackage[hypertex,plainpages,hyperindex=true]{hyperref}
  \hypersetup{%
    hypertexnames=false%
  }
  % Specify the driver for the color package
  \ExecuteOptions{dvips}
  %\ExecuteOptions{xdvi}
\else
  \ifnum\pdfoutput>0
   
\usepackage[pdftex,plainpages,hyperindex=true,pdfpagelabels]{hyperref}
    \hypersetup{%
      hypertexnames=false,%
      colorlinks=true,%
      linktocpage=true,%
    }
    % Specify the driver for the color package
    \ExecuteOptions{pdftex}
  \else
    \usepackage[hypertex,plainpages,hyperindex=true]{hyperref}
    \hypersetup{%
      hypertexnames=false%
    }
    % Specify the driver for the color package
    \ExecuteOptions{dvips}
    %\ExecuteOptions{xdvi}
  \fi
\fi

\hyperbaseurl{}
\newcommand\hr[1]{\href{#1.dvi}{dvi}, \href{#1.pdf}{pdf}}
\newcommand\h[1]{#1 (\hr{#1})}

\renewcommand\d{\text{d}}
\newcommand\T{\mathcal{T}}
\newcommand{\M}{\mathcal{M}}

\begin{document}

%\tracingcommands=1
\newlength{\hW} % heading box width
\newlength{\pW} % page number field width
\settowidth{\hW}{\bf\docname}
\settowidth{\pW}{Page \pageref{lastpage}\ of \pageref{lastpage}}
\ifdim \pW > \hW \setlength{\hW}{\pW} \fi
\makeatletter
\def\@biblabel#1{#1.}
\newcommand{\ps@twolines}{%
  \renewcommand{\@oddhead}{%
    \docdate\hfill\parbox[t]{\hW}{{\hfill\bf\docname}\newline
                          Page \thepage\ of \pageref{lastpage}}}%
\renewcommand{\@evenhead}{}%
\renewcommand{\@oddfoot}{}%
\renewcommand{\@evenfoot}{}%
}%
\makeatother
\pagestyle{twolines}

\vspace{-10pt}
\begin{tabbing}
\phantom{References: }\= \\
To: \>Bill Read, Nathaniel Livesey\\
Subject: \>Calculating incremental optical depth\\
From: \>Van Snyder\\
%Reference: \\
\end{tabbing}

\parindent 0pt \parskip 4pt

In the solution of the non-scattering radiative transfer equation

\begin{equation}\label{diff eq}
\frac{\partial I(s)}{\partial s} + \alpha(s) I(s) = \alpha(s) B(s)\,,
\end{equation}

\emph{viz.}

\begin{equation}\label{first}
I[\alpha(s),s_\M] = I(s_0) \T(s_0,s_\M) +
 \int_{s_0}^{s_\M} \T(s,s_\M) \alpha(s) B(s) \,
  \d s\,,
\end{equation}

the inner integral

\begin{equation}\label{two}
\T(s,s_\M) = \exp\left( - \int_s^{s_\M} \alpha(\sigma)
 \, \d \sigma \right)
\end{equation}

appears. In the full forward model, the argument of the exponential
function in Equation (\ref{two}), \emph{viz.}

\begin{equation}\label{three}
\delta(s,s_\M) = \int_s^{s_\M} \alpha(\sigma) \, \d \sigma
\end{equation}

is called \emph{optical depth}.  (In section 10.3 of the ATBD, it is
called the \emph{opacity integral}.)  It is evaluated separately for each
layer between constant-$\zeta$ surfaces:

\begin{equation}\label{four}
\Delta
\delta_{i \rightarrow i-1} = \int_{s_i}^{s_{i-1}} \alpha(\sigma) \,
 \d \sigma \,.
\end{equation}

In the full forward model, this is called the \emph{incremental optical
depth}, with the variable name {\tt incoptdepth}.  The full forward model
operates with $\zeta$ rather than $s$ as the variable of integration.
Therefore, Equation (\ref{four}) becomes

\begin{equation}\label{five}
\Delta
\delta_{i \rightarrow i-1} = \int_{\zeta_i}^{\zeta_{i-1}} \alpha(\zeta) \,
 \frac{\d s}{\d h} \frac{\d h}{\d \zeta}\, \d \zeta \,.
\end{equation}

The integral in Equation (\ref{four}) is first approximated by a
rectangular quadrature, with $s$ as the independent variable:

\begin{equation}\label{six}
\Delta
\delta_{j \rightarrow j-1} \approx \alpha_j\, \Delta s_{j \rightarrow j-1}
\text{, or }
\text{\tt incoptdepth(j) = alpha_path_c(j) * del_s(j)}
\end{equation}

where {\tt del_s(j)} is $\Delta s_{j \rightarrow j-1}$.

An initial approximation of the optical depth is the indefinite sum

\begin{equation*}
\delta_j = \sum_{i=1}^j \Delta \delta_{i \rightarrow i-1} \,.
\end{equation*}

An estimate of the error in using this approximation to evaluate Equation
(\ref{two}) is based upon the derivative

\begin{equation}\label{seven}
\frac{\d\, (-\exp(\delta_j))}{\d s} =
 -\exp(-\delta_j)\, \frac{\d\, \delta_j}{\d s} \approx
 -\exp(-\delta_j)\, ( \delta_{j+1} - \delta{_j-1} )\,.
\end{equation}

During this calculation, the value of $j$ where $-\delta_j < - \ln
\Omega$, where $\Omega$ is the largest representable floating-point
number, is saved, because $\exp(-\delta_j)$ will underflow. Where the
negative of the derivative in Equation (\ref{seven}) is larger than a
specified tolerance, a logical variable {\tt do_gl} is set {\tt .true.} to
indicate that the approximation in Equation (\ref{six}) should be replaced
by a three-point Gauss-Legendre quadrature.

Where {\tt do_gl} is {\tt .false.}, the approximation in Equation
(\ref{six}) is replaced by a trapezoidal approximation. This is done
inconsistently, inefficiently, and inaccurately:

{\tt\small
\begin{equation}\begin{split}\label{nine}
\Delta \delta_{i \rightarrow i-1} := \,&
 \Delta \delta_{i \rightarrow i-1} + \frac12 ( \alpha_{j-1} - \alpha_j )
 \frac{\d s}{\d h} \frac{\d h}{\d \zeta}\, \d \zeta{\text, or} \\
\text{incoptdepth(j)} \,& \text{= incoptdepth(j) + \& } \\
 \,& \text{\& ( alpha_path_c(j-1) - alpha_path_c(j) ) * dsdz_c(j-1) *
 del_zeta(j) }\\
\end{split}\end{equation}
}

where {\tt dsdz\_c} = $\frac12\,\frac{\d s}{\d h} \frac{\d h}{\d \zeta}$
and {\tt del\_zeta} = $\delta\, \zeta$.

This is inconsistent because Equation (\ref{six}) was evaluated with $s$
as the independent variable, not with $\zeta$ as the independent variable.

It is inefficient because it is a correction from a rectangular to a
trapezoidal approximation. It would be more efficient to write it directly
as a trapezoidal approximation, \emph{viz.}

{\tt\small
incoptdepth(j) = ( alpha_path_c(j-1) + alpha_path_c(j) ) * dsdz_c(j-1) *
 del_zeta(j)
}

It is inaccurate because computing this as a correction is less accurate
than computing the trapezoidal estimate directly.  More importantly,
{\tt dsdz_c(j-1) * del_zeta(j)} is a rectangular estimate of

\begin{equation*}
\Delta\,s_{j \rightarrow j-1} = \int_{\zeta_j}^{\zeta{_j-1}}
 \frac{\d s}{\d h} \frac{\d h}{\d \zeta} \, \d \zeta \,,
\end{equation*}

which is very inaccurate near the tangent point because $\frac{\d s}{\d
h}$ has a square-root singularity at the tangent point. In any case, the
forward model already has $\Delta\,s_{j \rightarrow j-1}$ as the variable
{\tt del\_s}, so there is no point to evaluate Equation (\ref{five}) using
Equation(\ref{nine}), i.e., using $\zeta$ as the independent variable,
instead of evaluating Equation (\ref{four}), i.e., using $s$ as the
independent variable.

The reason this has not been changed is that the retrieved results appear
to be more physically realistic using Equation (\ref{five}). The reason
for this remains a mystery.

\vspace{-1pt} % Somehow, this makes page numbering more likely correct.
\label{lastpage}
\end{document}
