\documentclass[11pt]{article}
\usepackage[fleqn]{amsmath}\textwidth 6.25in

\oddsidemargin -0.25in
%\evensidemargin -0.5in
\topmargin -0.5in
\textheight 9.00in

\newcommand{\docname}{\bf wvs-023r2}
\newcommand{\docdate}{7 September 2005}

\begin{document}

%\tracingcommands=1
\newlength{\hW} % heading box width
\newlength{\pW} % page number field width
\settowidth{\hW}{\docname}
\settowidth{\pW}{Page \pageref{lastpage}\ of \pageref{lastpage}}
\ifdim \pW > \hW \setlength{\hW}{\pW} \fi
\makeatletter
\def\@biblabel#1{#1.}
\newcommand{\ps@twolines}{%
  \renewcommand{\@oddhead}{%
    \docdate\hfill\parbox[t]{\hW}{{\docname}\newline
                          Page \thepage\ of \pageref{lastpage}}}%
\renewcommand{\@evenhead}{}%
\renewcommand{\@oddfoot}{}%
\renewcommand{\@evenfoot}{}%
}%
\makeatother
\pagestyle{twolines}

\vspace{-10pt}
\begin{tabbing}
\phantom{References: }\= \\
To: \>Bill, Dong, Michael, Nathaniel\\
Subject: \>Spectral parameter offsets and derivatives\\
From: \>Van Snyder\\
\end{tabbing}

\parindent 0pt \parskip 3pt
\vspace{-20pt}

The ability to specify or solve for spectral parameter offsets from the
catalog has been added to the full forward model in MLSL2.  The
parameters that can be offset are line center, line width, and line width
temperature dependence.

To specify that the full forward model is to include the offsets in its
calculation for a particular molecule, it is necessary so to specify in
the {\tt ForwardModel} specification in the L2CF.  This is done by using
three fields, named {\tt LineCenter}, {\tt LineWidth}, and {\tt
LineWidth\_TDep}.  The value for each of these fields is a list of
molecules.  Each molecule in these lists must be in the list in the {\tt
Molecules} field, and not PFA.  These specify the molecules for which the
parameters are to be added to their catalog values.  The values are not
molecule group names.  These fields can only be specified if the value of
the {\tt Type} field is {\tt Full}.

It is necessary to have state vector quantities defined for each spectral
parameter and molecule for which the offsets are to be used.  This is done
with a {\tt Quantity} specification, having a {\tt Type} field value that
is the same as the field name in the {\tt ForwardModel} specification, e.g.

{\tt\begin{verbatim}
  lineCenter_O2_V1: Quantity, type=lineCenter, molecule=O2_V1, $
    hGrid=hGridStandard, vGrid=vGridStandard
\end{verbatim}}

Once the quantity is established, it is necessary to mention it in the
definition of the vector template for the vectors specified in either the
{\tt FwdModelIn} field of a {\tt SIDS} specification or the {\tt State}
field of a {\tt Retrieve} specification, or the vector template for the
vectors specified in the {\tt FwdModelExtra} field of a {\tt SIDS} or
{\tt Retrieve} specification.

These quantities spring into existence filled with zeros, which would
probably be appropriate as the initial value for a retrieval.  Use
something like the following if you want to make this explicit:

{\tt\begin{verbatim}
  Fill, quantity=state.LineCenter_O2_V1, method=explicit, $
    explicitValues=0Ghz, /spread
\end{verbatim}}

For a SIDS run they might be filled with other values for experimental
purposes.

For line center and line width, the units of the fill value must be
frequency.  For line width temperature dependence, the units must be
dimensionless.

If a {\tt Jacobian} matrix is specified in a {\tt  Retrieve} or {\tt
SIDS} specification that uses the full forward model, derivatives of
radiance with respect to these quantities are computed, which allows to
solve for them in the retrieval case if they are in the {\tt State}
vector.

\label{lastpage}
\end{document}
% $Id$
