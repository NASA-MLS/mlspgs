\documentclass[11pt]{article}
\usepackage{alltt}
\usepackage[fleqn]{amsmath}
\usepackage{floatflt}
\usepackage{graphicx}
\usepackage{longtable}
\usepackage[strings]{underscore}

\textwidth 6.5in
\oddsidemargin -0.25in
%\evensidemargin -0.5in
\topmargin -0.5in
\textheight 9in

\newcommand{\docname}{wvs-131}
\newcommand{\docdate}{5 February 2016}

\ifx\pdfoutput\undefined
  \pdfoutput=0
  \usepackage[hypertex,plainpages,hyperindex=true]{hyperref}
  \hypersetup{%
    hypertexnames=false%
  }
  % Specify the driver for the color package
  \ExecuteOptions{dvips}
  %\ExecuteOptions{xdvi}
\else
  \ifnum\pdfoutput>0
    \usepackage[pdftex,plainpages,hyperindex=true,pdfpagelabels]{hyperref}
    \hypersetup{%
      hypertexnames=false,%
      colorlinks=true,%
      linktocpage=true,%
    }
    % Specify the driver for the color package
    \ExecuteOptions{pdftex}
  \else
    \usepackage[hypertex,plainpages,hyperindex=true]{hyperref}
    \hypersetup{%
      hypertexnames=false%
    }
    % Specify the driver for the color package
    \ExecuteOptions{dvips}
    %\ExecuteOptions{xdvi}
  \fi
\fi

\hyperbaseurl{}
\newcommand\hr[1]{\href{#1.dvi}{dvi}, \href{#1.pdf}{pdf}}
\newcommand\h[1]{#1 (\hr{#1})}

\begin{document}

%\tracingcommands=1
\newlength{\hW} % heading box width
\newlength{\pW} % page number field width
\settowidth{\hW}{\bf\docname}
\settowidth{\pW}{Page \pageref{lastpage}\ of \pageref{lastpage}}
\ifdim \pW > \hW \setlength{\hW}{\pW} \fi
\makeatletter
\def\@biblabel#1{#1.}
\newcommand{\ps@twolines}{%
  \renewcommand{\@oddhead}{%
    \docdate\hfill\parbox[t]{\hW}{{\hfill\bf\docname}\newline
                          Page \thepage\ of \pageref{lastpage}}}%
\renewcommand{\@evenhead}{}%
\renewcommand{\@oddfoot}{}%
\renewcommand{\@evenfoot}{}%
}%
\makeatother
\pagestyle{twolines}

\newcommand{\TS}{T_\text{scat}}
\newcommand{\TSs}[1]{T_{\text{scat}_{#1}}}
\newcommand{\DB}{\Delta B}
\newcommand{\oDB}{\overline{\DB}}
\newcommand{\MT}{\mathcal{T}}
\newcommand{\hMT}{\MT^s}
\newcommand{\IF}[1]{\,\mathcal{A}_n\!\left(#1\right)} % Interpolation Function

\vspace{-10pt}
\begin{tabbing}
\phantom{References: }\= \\
To: \>Van\\
Subject: \>Intersection of a line with an ellipsoid\\
From: \>Van Snyder\\
Reference: \> wvs-030 \\
\end{tabbing}

\parindent 0pt \parskip 6pt
\vspace{-20pt}

Using vector notation, an ellipsoid can be expressed as the set of points
$\mathbf{p}$ such that
%
\begin{equation}\label{one}
(\mathbf{p}\, \mathbf{M})^T (\mathbf{p}\, \mathbf{M}) = 1\,,
(\mathbf{p} - \mathbf{p}_0) \cdot \mathbf{n} = 0\,,
\end{equation}
%
where $p$ is a point on the surface of the ellipse,
%
\begin{equation}
\mathbf{M} = \left[
 \begin{array}{ccc}
  \frac1a & 0 & 0 \\
  0 & \frac1b & 0 \\
  0 & 0 & \frac1c \\
 \end{array}
\right]
\end{equation}
%
A line can be expressed as the set of points
%
\begin{equation}\label{two}
\mathbf{p} = d \, \mathbf{U} + \mathbf{U}_0 \,,
\end{equation}
%
where $\mathbf{U}_0$ is a point on the line and $\mathbf{U}$ is a vector
along the line.  Substituting Equation (\ref{two}) into Equation
(\ref{one}) gives
%
\begin{equation}
(\mathbf{U}\, \mathbf{M})^T (\mathbf{U}\, \mathbf{M})\, t^2 +
(\mathbf{C} \mathbf{M})^T (\mathbf{U}\, \mathbf{M})\, t +
(\mathbf{C} \mathbf{M})^T (\mathbf{C} \mathbf{M}) = 1
\end{equation}

A method to determine the points on a line and ellipsoid that are nearest
to each other is described in wvs-030.

\label{lastpage}
\vspace*{-0.1in} % Somehow, this causes lastpage to be defined
\end{document}

% $Id$

% $Log$
% Revision 1.1  2016/02/05 03:36:20  vsnyder
% Initial commit
