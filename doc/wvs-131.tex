\documentclass[11pt]{article}
\usepackage{alltt}
\usepackage[fleqn]{amsmath}
\usepackage{floatflt}
\usepackage{graphicx}
\usepackage{longtable}
\usepackage[strings]{underscore}
\usepackage{xr-hyper}
\externaldocument[146-]{\logdir/wvs-146}

\textwidth 6.5in
\oddsidemargin -0.25in
%\evensidemargin -0.5in
\topmargin -0.5in
\textheight 9in

\newcommand{\docname}{wvs-131r6}
\newcommand{\docdate}{13 April 2020}

\ifx\pdfoutput\undefined
  \pdfoutput=0
  \usepackage[hypertex,plainpages,hyperindex=true]{hyperref}
  \hypersetup{%
    hypertexnames=false%
  }
  % Specify the driver for the color package
  \ExecuteOptions{dvips}
  %\ExecuteOptions{xdvi}
\else
  \ifnum\pdfoutput>0
    \usepackage[pdftex,plainpages,hyperindex=true,pdfpagelabels]{hyperref}
    \hypersetup{%
      hypertexnames=false,%
      colorlinks=true,%
      linktocpage=true,%
    }
    % Specify the driver for the color package
    \ExecuteOptions{pdftex}
  \else
    \usepackage[hypertex,plainpages,hyperindex=true]{hyperref}
    \hypersetup{%
      hypertexnames=false%
    }
    % Specify the driver for the color package
    \ExecuteOptions{dvips}
    %\ExecuteOptions{xdvi}
  \fi
\fi

\hyperbaseurl{}
\newcommand\hr[1]{\href{#1.dvi}{dvi}, \href{#1.pdf}{pdf}}
\newcommand\h[1]{#1 (\hr{#1})}

\begin{document}

%\tracingcommands=1
\newlength{\hW} % heading box width
\newlength{\pW} % page number field width
\settowidth{\hW}{\bf\docname}
\settowidth{\pW}{Page \pageref{lastpage}\ of \pageref{lastpage}}
\ifdim \pW > \hW \setlength{\hW}{\pW} \fi
\makeatletter
\def\@biblabel#1{#1.}
\newcommand{\ps@twolines}{%
  \renewcommand{\@oddhead}{%
    \docdate\hfill\parbox[t]{\hW}{{\hfill\bf\docname}\newline
                          Page \thepage\ of \pageref{lastpage}}}%
\renewcommand{\@evenhead}{}%
\renewcommand{\@oddfoot}{}%
\renewcommand{\@evenfoot}{}%
}%
\makeatother
\pagestyle{twolines}

\newcommand{\TS}{T_\text{scat}}
\newcommand{\TSs}[1]{T_{\text{scat}_{#1}}}
\newcommand{\DB}{\Delta B}
\newcommand{\oDB}{\overline{\DB}}
\newcommand{\MT}{\mathcal{T}}
\newcommand{\hMT}{\MT^s}
\newcommand{\IF}[1]{\,\mathcal{A}_n\!\left(#1\right)} % Interpolation Function

\vspace{-10pt}
\begin{tabbing}
\phantom{References: }\= \\
To: \>Van\\
Subject: \>Intersection of a line with an ellipsoid\\
From: \>Van Snyder\\
Reference: \> wvs-030 \hr{wvs-030} \\
\end{tabbing}

\parindent 0pt \parskip 6pt
\vspace{-20pt}

\section{Ellipsoid}

Using vector notation, an ellipsoid with center at $\mathbf{p}_0$ and axes
aligned with the co\"ordinate directions can be expressed as the set of points
$\mathbf{p}$ such that
%
\begin{equation}\label{one}
(\mathbf{M} \, ( \mathbf{p} - \mathbf{p}_0))^T
(\mathbf{M} \, ( \mathbf{p} - \mathbf{p}_0)) = 1
\end{equation}
%
where $\mathbf{p}$ is a point on the surface of the ellipsoid,
%
\begin{equation}\label{M}
\mathbf{M} = \left[
 \begin{array}{ccc}
  \frac1a & 0 & 0 \\
  0 & \frac1b & 0 \\
  0 & 0 & \frac1c \\
 \end{array}
\right]
\end{equation}
%
and $a$, $b$, and $c$ are the semi-axes.

A line can be expressed as the set of points
%
\begin{equation}\label{two}
\mathbf{p} = \mathbf{C} + s \, \mathbf{U} \,,
\end{equation}
%
where $\mathbf{C}$ is a point on the line and $\mathbf{U}$ is a vector
along the line.  Substituting Equation (\ref{two}) into Equation
(\ref{one}) gives
%
\begin{equation}\label{four}
(\mathbf{M}\, \mathbf{U})^T \cdot (\mathbf{M}\, \mathbf{U})\, s^2 +
2 (\mathbf{M} ( \mathbf{C} - \mathbf{p}_0) )^T \cdot
(\mathbf{M}\, \mathbf{U})\, s +
(\mathbf{M} ( \mathbf{C} - \mathbf{p}_0) )^T \cdot
(\mathbf{M} ( \mathbf{C} - \mathbf{p}_0) ) = 1
\end{equation}

Let $\mathbf{V} = \mathbf{C} - \mathbf{p}_0$.  Then Equation (\ref{four})
becomes
%
\begin{equation}\label{five}
(\mathbf{M}\, \mathbf{U})^T \cdot (\mathbf{M}\, \mathbf{U})\, s^2 +
2 (\mathbf{M} \mathbf{V} )^T \cdot (\mathbf{M}\, \mathbf{U})\, s +
(\mathbf{M} \mathbf{V} )^T \cdot (\mathbf{M} \mathbf{V} ) = 1
\end{equation}

which will give zero, one, or two solutions for $s$ depending upon whether
the line does not intersect the ellipsoid, is tangent to it, or intersects
it twice.

A method to determine the points on a line and ellipsoid that are nearest
to each other is described in wvs-030 \hr{wvs-030}.

For the case of a sphere with radius $r$, Equation (\ref{five}) becomes
%
\begin{equation}\label{sphere}
\mathbf{U} \cdot \mathbf{U} \, s^2 +
2 \mathbf{V} \cdot \mathbf{U} \, s +
\mathbf{V} \cdot \mathbf{V} = r^2 \,.
\end{equation}

%=========================================================================
\section{When the ellipsoid is an oblate spheroid}

Instead of Equation (\ref{M}), write

\begin{equation}
\mathbf{M} = \left[ \begin{array}{lll}
 1 & 0 & 0 \\ 0 & 1 & 0 \\ 0 & 0 & \frac1E \\
 \end{array} \right]\,,
\end{equation}

where $E = \sqrt{1-e^2}$ and $e^2 = 1-\frac{c^2}{a^2}$ is the square of
the eccentricity.  Write the equation of an ellipsoid with its center at
$\mathbf{p}_0 = (0,0,0)$ as

\begin{equation}
( \mathbf{M} \mathbf{p} )^T \cdot ( \mathbf{M} \mathbf{p} ) = a^2\,.
\end{equation}

Equation (\ref{five}) then becomes

\begin{equation}\label{five-a}
(\mathbf{M}\, \mathbf{U})^T \cdot (\mathbf{M}\, \mathbf{U})\, s^2 +
2 (\mathbf{M} \mathbf{C} )^T \cdot (\mathbf{M}\, \mathbf{U})\, s +
(\mathbf{M} \mathbf{C} )^T \cdot (\mathbf{M} \mathbf{C} ) = a^2
\end{equation}

Equation (\ref{five-a}) has one solution for $s$, i.e., the line is
tangent to the ellipsoid, if and only if its discriminant is zero, i.e.,

\begin{equation}
\left( 2 (\mathbf{M} \mathbf{C} )^T \cdot (\mathbf{M}\, \mathbf{U}) \right)^2
- 4 (\mathbf{M}\, \mathbf{U})^T \cdot (\mathbf{M}\, \mathbf{U})\, 
\left[ (\mathbf{M} \mathbf{C} )^T \cdot (\mathbf{M} \mathbf{C} ) -a^2 \right]
 = 0 \,.
\end{equation}

Let $\mathbf{N} = \mathbf{C} \times \mathbf{U}$.  Then the discriminant is
zero for

\begin{equation}\label{discriminant-solution}
a^2 = \frac{ \left( \mathbf{M}^{-1} \mathbf{N} \right)^T \cdot
             \left( \mathbf{M}^{-1} \mathbf{N} \right) }
           { E^2 \left( \mathbf{M}\mathbf{U} \right )^T \cdot
                 \left( \mathbf{M}\mathbf{U} \right ) }\,,
\end{equation}

but this result is not needed for further discussion.

If $a$ is replaced by $a'$ in Equation (\ref{five-a}), then Equation
(\ref{five-a}) describes an ellipsoid with the same eccentricity, but with
semi-major axis $a'$.  If the discriminant is zero with $a'$ substituted
for $a$ in Equation (\ref{five-a}), then the line in Equation (\ref{two})
is tangent to the ellipsoid with semi-major axis $a'$ at the single
solution $s$ of Equation (\ref{five-a}), \emph{viz}.

\begin{equation}\label{s-solution}
s = - \frac{ (\mathbf{M} \mathbf{C} )^T \cdot (\mathbf{M}\, \mathbf{U}) }
           { (\mathbf{M}\, \mathbf{U})^T \cdot (\mathbf{M}\, \mathbf{U}) }
           \,.
\end{equation}

If $a' < a$, the line intersects the ellipsoid in two places. If $a' = a$,
the line is tangent to the ellipsoid.  If $a' > a$, the line does not
intersect the ellipsoid.

To calculate the point, on a line that does not intersect the ellipsoid
($a' > a$), that is nearest to the ellipsoid, see \h{wvs-030}.

\label{lastpage}
\vspace*{-0.1in} % Somehow, this causes lastpage to be defined
\end{document}

% $Id$

% $Log$
% Revision 1.10  2020/04/21 00:05:49  pwagner
% Should work better with make
%
% Revision 1.9  2020/04/14 00:55:33  vsnyder
% Forgot to update version number and date
%
% Revision 1.8  2020/04/14 00:52:38  vsnyder
% Remove calculation of nearest approach because it was wrong. Refer to wvs-030
%
% Revision 1.7  2020/03/28 04:09:22  vsnyder
% Add calculation of nearest approach
%
% Revision 1.6  2017/10/13 19:03:39  vsnyder
% Explain that the line will intersect 0, 1, or 2 times
%
% Revision 1.5  2016/03/19 01:39:34  vsnyder
% Correct first-order term for s
%
% Revision 1.4  2016/03/17 18:58:28  vsnyder
% Write matrix-vector products correctly, replace t with s
%
% Revision 1.3  2016/02/24 01:32:10  vsnyder
% Forgot to change the date
%
% Revision 1.2  2016/02/24 01:31:16  vsnyder
% Include center of ellipsoid in equations, specify orientation
%
% Revision 1.1  2016/02/18 20:57:06  vsnyder
% Initial commit
