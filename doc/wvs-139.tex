\makeatletter\let\ifGm@compatii\relax\makeatother
\documentclass[landscape]{beamer}

\usepackage{amsmath}
\usepackage{graphicx}

% \ifx\pdfoutput\undefined
%   \pdfoutput=0
%   \usepackage{hyperref}
%   \hypersetup{%
%     hypertex,%
%     plainpages,%
%     hyperindex=true,%
%     hypertexnames=false%
%   }
%   % Specify the driver for the color package
%   \ExecuteOptions{dvips}
%   %\ExecuteOptions{xdvi}
% \else
%   \ifnum\pdfoutput>0
%     \usepackage{hyperref}
%     \hypersetup{%
%       pdftex,%
%       plainpages,%
%       hyperindex=true,%
%       pdfpagelabels,%
%       hypertexnames=false,%
%       colorlinks=true,%
%       linktocpage=true,%
%     }
%     % Specify the driver for the color package
%     \ExecuteOptions{pdftex}
%   \else
%     \usepackage{hyperref}
%     \hypersetup{%
%       hypertex,%
%       plainpages,%
%       hyperindex=true,%
%       hypertexnames=false%
%     }
%     % Specify the driver for the color package
%     \ExecuteOptions{dvips}
%     %\ExecuteOptions{xdvi}
%   \fi
% \fi
% 
% \hyperbaseurl{}
% \newcommand\hr[1]{\href{#1.dvi}{dvi}, \href{#1.pdf}{pdf}}
% \newcommand\h[1]{#1 (\hr{#1})}

\newcommand{\newframe}[2][]{\begin{frame}\frametitle{\hfill #2 \hfil}}
\renewcommand{\d}{\text{d}}

\title{Radiative Transfer Equation \\
in the Full Forward Model}
\subtitle{wvs-139r4}

\author{Van Snyder}

\date{20 May 2020}

\titlegraphic{\includegraphics[width=1.0in]{eos_mls_logo_onpink}}

\begin{document}

\begin{frame}
  \titlepage
\end{frame}

% -----------------------------------------------------------------------------

\newframe{Radiative Transfer Equation}

The clear-sky non-scattering radiative-transfer equation for one frequency
is

\begin{equation}\label{one}
\frac{\d I(s,\mathbf{x})}{\d s} + \alpha(s,\mathbf{x})\, I(s,\mathbf{x}) =
 \alpha(s,\mathbf{x})\, B(T(s)) \,,
\end{equation}

where $I$ is radiative intensity (more simply radiance), $\mathbf{x}$ is
the state vector, $s$ is path length, $\alpha(s,\mathbf{x})$ is the
absorption cross section,

\begin{equation*}
B(T(s)) = \frac{h \nu}k \left( \exp \left(\frac{h \nu}{k T(s)} \right)
 - 1 \right)^{-1}
\end{equation*}

is the Planck black-body radiation function in brightness-temperature
units, $\nu$ is frequency, $h$ is Planck's constant, $k$ is Boltzmann's
constant, and $T(s) \in \mathbf{x}(s)$ is temperature.

\end{frame}

% -----------------------------------------------------------------------------

\newframe{Radiative Transfer Equation}

The solution to Equation (\ref{one}) can be written
%
\begin{equation}\label{two}
I(s_m,\mathbf{x}) = I(s_0,\mathbf{x})\, \mathcal{T}(s_m) +
 \int_{s_0}^{s_m} \mathcal{T}(s) \,
  \alpha(s,\mathbf{x})\, B(T(s)) \, \d s \,,
\end{equation}
%
where $s_0$ is deep space, $s_m$ is the instrument
position,
%
\begin{equation*}\begin{split}
\mathcal{T}(s) = \,& \exp \left( - \int_{s_0}^s \alpha(\sigma,\mathbf{x}) \,
 \d \sigma \right), \\
\alpha(s,\mathbf{x}(s)) = \,& \sum_{k} f^k(s)\, \beta^k(s,T(s)) \,,
\end{split}\end{equation*}

$f^k(s) \in \mathbf{x}(s)$ is the mixing ratio of the $k^\text{th}$
chemical species, and $\beta^k(s,T(s))$ is the absorption coefficient for
that species.

\end{frame}

% -----------------------------------------------------------------------------

\newframe{Radiative Transfer Equation}

In the Full Forward Model, Equation (\ref{two}) is transformed by observing
that

\begin{equation*}
\frac{\d \mathcal{T}(s)}{\d s} = -\alpha(s)\, \mathcal{T}(s)
\end{equation*}

giving

\begin{equation*}
I(s_m,\mathbf{x}) = I(s_0,\mathbf{x})\, \mathcal{T}(s_m) -
 \int_{s_0}^{s_m} B(T(s)) \, \frac{\d \mathcal{T}(s)}{\d s} \, \d s
\end{equation*}

which is then integrated by parts giving

\begin{equation}\label{three}
I(s_m,\mathbf{x}) = \left( I(s_0,\mathbf{x}) - B(s_m) \right) \,
 \mathcal{T}(s_m) +
  \int_{s_0}^{s_m} \mathcal{T}(s) \frac{\d B(T(s))}{\d s} \, \d s \,.
\end{equation}

These transformations do not appear to be necessary; they eliminate one
multiply in the evaluation of the integrand in Equation (\ref{two}), which
is replaced by two subtractions, one of which approximates $\frac{\d
B(T(s))}{\d s}$ inaccurately.

\end{frame}

% -----------------------------------------------------------------------------

\newframe{Derivatives of the Radiative Transfer Equation (1)}

To solve for $\mathbf{x}$ using a Newton method (see wvs-050), derivatives
of $I(s_m,\mathbf{x})$ with respect to $\mathbf{x}$ are needed.  These can
be gotten by differentiating Equation (\ref{one}) with respect to elements
of $\mathbf{x}(s)$, which after re-arranging becomes

\begin{equation}\label{four}
\frac{\d I_{x^\mu}(s)}{\d s} + \alpha(s,\mathbf{x}(s))\,
 I_{x^\mu}(s) = R(s,\mathbf{x}(s)) \,,
\end{equation}

where

\begin{equation*}
I_{x^\mu}(s) = \frac{\partial I(s,\mathbf{x}(s))}{\partial x^\mu(s)} \,,
\end{equation*}

$x^\mu(s) \in \mathbf{x}(s)$ (either $f^k(s)$ or $T(s)$), and

\begin{equation}\label{R}
R(s,\mathbf{x}(s)) =
 \frac{\partial \alpha(s,\mathbf{x}(s))}{\partial x^\mu(s)}
 \left( B(T(s)) - I(s,\mathbf{x}) \right) +
 \alpha(s,\mathbf{x}(s)) \frac{\partial B(T(s))}{\partial x^\mu(s)} \,.
\end{equation}

\end{frame}

% -----------------------------------------------------------------------------

\newframe{Derivatives of the Radiative Transfer Equation (2)}

$B(T(s))$ depends upon frequency and temperature, but not any mixing ratio.
Its derivatives are
%
\begin{equation*}\begin{split}
\frac{\text{d} B(T(s))}{\text{d} T(s)} = \,&
 \frac{B(T(s))}{T(s)^2} \left( \frac{h\nu}k + B(T(s)) \right)\, \text{and}
 \\
\frac{\partial B(T(s))}{\partial f^k} = \,& 0\,.
\end{split}\end{equation*}
%
Remember that  $\alpha(s,\mathbf{x}(s)) = \sum_{k} f^k(s)\,
\beta^k(s,T(s))$, observe that $\beta^k(s,\mathbf{T}(s))$ usually does not
depend upon $f^k(s)$, and $f^k(s)$ does not depend upon temperature. 
Therefore
%
\begin{equation*}
\frac{\partial \alpha(s,\mathbf{x}(s))}{\partial f^k(s)} =
\beta^k(s,T(s)) \text{ and }
\frac{\partial \alpha(s,\mathbf{x}(s))}{\partial T(s)} =
\sum_{k} f^k(s) \frac{\partial \beta^k(s,T(s))}{\partial T(s)}
\end{equation*}
%
(see wvs-018 concerning temperature derivatives).

\end{frame}

% -----------------------------------------------------------------------------

\newframe{Derivatives of the Radiative Transfer Equation(3)}

Equation (\ref{four}) is of the same form as Equation (\ref{one}), with
$I(s,\mathbf{x})$ replaced by $I_{x^\mu}(s)$, and $\alpha(s,\mathbf{x})\,
B(T(s))$ in the right-hand side replaced by $R(s,\mathbf{x})$.  The
solution is therefore of the same form as Equation (\ref{two}), with
$\alpha(s,\mathbf{x})\, B(T(s))$ replaced by $R(s,\mathbf{x})$,
\emph{viz}.

\begin{equation}\begin{split}\label{five}
I_{x^\mu}(s_m) = \,&
 I_{x^\mu}(s_0) \,
  \mathcal{T}(s_m) +
  \int_{s_0}^{s_m} \mathcal{T}(s) \, R(s,\mathbf{x}(s)) \, \d s \\
= \,&
 \int_{s_0}^{s_m} \mathcal{T}(s) \, R(s,\mathbf{x}(s)) \, \d s \,, \\
\end{split}\end{equation}

the last equality because the derivative of radiance at the deep-space
starting point, $s_0$, with respect to any state-vector element $x^\mu$,
is zero.

\end{frame}

% -----------------------------------------------------------------------------

\newframe{Derivatives of the Radiative Transfer Equation (4)}

We wrote $\mathbf{x}(s)$ instead of $\mathbf{x}$ to emphasize that these
derivatives are with respect to state vector quantities \emph{on the
integration path}.  The Newton method needs derivatives with respect to
state vector elements \emph{in solution profiles}, which are calculated
using the chain rule, \emph{viz}.
%
\begin{equation*}
\frac{\partial \cdot}{\partial x^\mu_j} =
\frac{\partial \cdot}{\partial x^\mu(s)}
\frac{\partial x^\mu(s)}{\partial x^\mu_j} =
\frac{\partial \cdot}{\partial x^\mu(s)} \, \eta^\mu_j(s)\,,
\end{equation*}
%
where $x^\mu_j$ is the $j^\text{th}$ element of the $\mu^\text{th}$
solution profile state vector quantity (either $f^k$ or $T$), and
$\eta^\mu_j(s)$ is the interpolation coefficient from the geolocation of
$x^\mu_j$ to the integration path at $s$.

\vspace*{4pt}
Equations (\ref{three}) and (\ref{five}) are discretized, with steps at
$s_i$. Because we use bilinear interpolation from ($\phi \times \zeta$) to
$s_i$, the $i^\text{th}$ row of the matrix of elements $\eta^\mu_j(s_i)$
has only four nonzeroes for each $\mu$, and the number of nonzeroes in the
$j^\text{th}$ column for the $\mu^\text{th}$ species is $\leq 7$ because
$s_i$ is on a constant-$\zeta$ surface and quadrature (described below)
uses values on those surfaces, or at most three points each in
$(\zeta_{i-1},\zeta_i)$ and $(\zeta_i,\zeta_{i+1})$.

\end{frame}

% -----------------------------------------------------------------------------

\newframe{Derivatives in the Full Forward Model (1)}

The Full Forward Model evaluates $I_{f^\mu}(s_m)$ by differentiating
Equation (\ref{two}) w.r.t. $f^\mu_j$, \emph{viz}.

\begin{equation}\begin{split}\label{I_f}
\frac{\partial I(s_m)}{\partial f^\mu_j} = \,&
 -I(s_0)\, \mathcal{T}(s_m)
  \int_{s_0}^{s_m} \beta^\mu(\sigma)\, \eta^\mu_j \d \sigma + \\
&
 \int_{s_0}^{s_m} \mathcal{T}(s)\, B(s)
  \left( \beta^\mu(s)\, \eta^\mu_j(s) -
         \alpha(s) \int_s^{s_m} \beta^\mu(\sigma)\, \eta^\mu_j(\sigma)
          \d \sigma \right) \d s \,, \\
\end{split}\end{equation}

which leads to an expression different from Equation (\ref{five}), but
which is necessarily equivalent as can be shown (with some difficulty --
see
% \h{wvs-093} and \h{wvs-100}%
wvs-093 and wvs-100%
) by integrating by parts.

\vspace*{5pt}

The forward model also evaluates $I_{T_j}(s_m)$ by differentiating
Equation (\ref{two}) w.r.t.\ $T_j$, producing an expression even more
complicated than Equation (\ref{I_f}).

\end{frame}

% -----------------------------------------------------------------------------

\newframe{Derivatives in the Full Forward Model (2)}

The original advantage of differentiating Equation (\ref{two}) rather than
solving Equation(\ref{four}), giving Equation (\ref{five}), is that
$I(s,\mathbf{x})$ is not averaged over all frequencies within a channel
(as specified by the pointing-frequency grid), and the average then
retained at each $\zeta_i$.  Rather, only $I(s_m,\mathbf{x})$ is retained
for all frequencies, and then averaged (see wvs-063, ``Pointing loop'').

\vspace*{5pt}

(Remember that in Equation (\ref{R}), $R(s,\mathbf{x}(s))$ depends upon
$I(s,\mathbf{x})$.  To compute frequency-averaged derivatives using
Equation (\ref{five}) would require to retain the frequency-averaged
$I(s_i,\mathbf{x})$, not just the frequency-averaged $I(s_m,\mathbf{x})$.)

\vspace*{5pt}

To combine results for line-by-line (LBL) quantities with those for
pre-frequency-averaged (PFA) quantities (see wvs-027), however,
$I(s_i,\mathbf{x})$ for LBL quantities is averaged over those frequencies
and retained for each channel, which would allow the simpler expression in
Equation (\ref{five}) to be used.  Frequency averaging at all $s_i$ would
be only a tiny bit slower than only at $s_m$.

\end{frame}

% -----------------------------------------------------------------------------

\newframe{Transmissivity in the Full Forward Model}

The Full Forward Model evaluates $\mathcal{T}(s)$ as the exponential of a
sum of a sequence of integrals, with $\zeta = -\log_{10} P$ where $P$ is
pressure, as the independent variable:

\begin{equation}\label{six}
\mathcal{T}(\zeta_n) =
 \exp \left( - \sum_{i=1}^n \int_{\zeta_{i-1}}^{\zeta_i}
 \alpha(\zeta,\mathbf{x}) \,
 \frac{\d s}{\d h} \frac{\d h}{\d \zeta} \, \d \zeta \right)
\end{equation}

where $\zeta_i = \zeta(s_i) = -\log_{10} P(s_i)$.

\vspace*{5pt}

Each integral is approximated by a quadrature.  The integrand is evaluated
at $\zeta_{i-1}$ and $\zeta_i$.  An estimate of the error in a trapezoidal
approximation of the integral is formed.  If the error is not too large
(according to a parameter in the {\tt l2cf}), the trapezoidal estimate is
used.

\vspace*{5pt}

If the error is too large, the integrand is evaluated at the abscissae of
the three-point Gauss-Legendre quadrature formula.

\end{frame}

% -----------------------------------------------------------------------------

\newframe{Path length in the Full Forward Model (1)}

For limb viewing, without considering refraction directly,

\begin{equation*}
s = \sqrt{h^2 - h_t^2}\,,
\end{equation*}

where $h_t$ is the geocentric tangent point height, and $h$ is the height
at the point a distance $s$ from the tangent point.

\begin{centering}
\includegraphics[scale=0.5]{wvs-139-1}\\
\end{centering}

\end{frame}

% -----------------------------------------------------------------------------

\newframe{Path length in the Full Forward Model (1)}

Thereby

\begin{equation}\label{sqrt}
\frac{\d s}{\d h} = \frac{h}{\sqrt{h^2 - h_t^2}} \,.
\end{equation}

Therefore, the integrand in Equation (\ref{six}) has a square-root
singularity at the tangent point.  Using either a Taylor series or a
binomial expansion, one sees that a square-root singularity cannot be
represented by a finite polynomial.

\vspace*{5pt}
In the neighborhood of the singularity, $\frac{\d s}{\d h}$ cannot be
accurately approximated by a polynomial of any finite order, and in
particular not by a polynomial that the quadrature formula would integrate
exactly.

\vspace*{5pt}
The exponential function is also not a finite polynomial, but because it
does not have a singularity, it can be approximately well, over a short
interval, by a polynomial.

\end{frame}

% -----------------------------------------------------------------------------

\newframe{Full Forward Model -- Singularity at the Tangent}

Using a rectangular quadrature for the integral in Equation (\ref{six}),

\begin{equation*}\begin{split}
\int_{\zeta_{i-1}}^{\zeta_i} \alpha(\zeta,\mathbf{x})
 \frac{\d s}{\d h} \frac{\d h}{\d \zeta} \, \d \zeta \approx \,&
\alpha(\zeta_{i-1},\mathbf{x}) \int_{\zeta_{i-1}}^{\zeta_i} 
 \frac{\d s}{\d h} \frac{\d h}{\d \zeta} \, \d \zeta \\
 \equiv Q_{i-1} = \,&
 \alpha(\zeta_{i-1},\mathbf{x})\, \delta s_{i-1 \rightarrow i} \,, \\
\end{split}\end{equation*}

where $\delta s_{i-1 \rightarrow i} = s_i - s_{i-1}$, that integral can be
written

\begin{equation*}
\int_{\zeta_{i-1}}^{\zeta_i} \alpha(\zeta,\mathbf{x})
 \frac{\d s}{\d h} \frac{\d h}{\d \zeta} \, \d \zeta =
 Q_{i-1} + \int_{\zeta_{i-1}}^{\zeta_i}
  \left ( \alpha(\zeta,\mathbf{x}) - \alpha(\zeta_{i-1},\mathbf{x}) \right)
 \frac{\d s}{\d h} \frac{\d h}{\d \zeta} \, \d \zeta \,,
\end{equation*}

which cancels the singularity at the tangent point (but not in other
integrals), provided $\left ( \alpha(\zeta,\mathbf{x}) -
\alpha(\zeta_{i-1},\mathbf{x}) \right) \rightarrow 0$ as $\zeta
\rightarrow \zeta_{i-1}$ faster than $\sqrt{h^2-h_t^2} \rightarrow 0$ as
$h \rightarrow h_t$.

\end{frame}

% -----------------------------------------------------------------------------

\newframe{Full Forward Model -- Radiative Transfer Integral (1)}

If the error estimate for the integral is small enough, a trapezoidal
approximation is constructed using either
%
\begin{equation*}
\int_{s_{i-1}}^{s_i} \alpha(\zeta,\mathbf{x})\, \d s \approx
  \frac12 \left ( \alpha(\zeta_{i-1},\mathbf{x}) +
                          \alpha(\zeta_i,\mathbf{x}) \right)
                          \, \delta s_{i-1 \rightarrow i}
\end{equation*}

or
%
\begin{equation*}
\int_{s_{i-1}}^{s_i} \alpha(\zeta,\mathbf{x})\, \d s \approx
  \frac12 \left ( \alpha(\zeta_{i-1},\mathbf{x}) +
                          \alpha(\zeta_i,\mathbf{x}) \right)
                          \, \frac{\d s_{i-1}}{\d h}
                             \frac{\d h_{i-1}}{\d\zeta}
                             ( \zeta_i - \zeta_{i-1} )
\end{equation*}

depending upon the setting of {\tt trapezoid} in the forward model
configuration in the {\tt l2cf} to {\tt correct} or {\tt wrong},
respectively, wherein
%
\begin{equation*}
\frac{\d s_{i-1}}{\d h} \frac{\d h_{i-1}}{\d\zeta}
         ( \zeta_i - \zeta_{i-1} ) \approx \int_{\zeta_{i-1}}^{\zeta_i}
 \frac{\d s}{\d h} \frac{\d h}{\d\zeta}\, \d \zeta
\end{equation*}

is a rectangular quadrature approximation to $\delta s_{i-1 \rightarrow
i}$.

\end{frame}

% -----------------------------------------------------------------------------

\newframe{Full Forward Model -- Radiative Transfer Integral (2)}

If the error estimate for the integral is not small enough, a 3-point
Gauss-Legendre approximation is constructed using
%
\begin{equation}\begin{split}\label{nine}
\int_{\zeta_{i-1}}^{\zeta_i} & \alpha(\zeta,\mathbf{x})
 \frac{\d s}{\d h} \frac{d h}{\d \zeta} \, \d \zeta \approx
Q_{i-1} + \\
 \,&
( \zeta_i - \zeta_{i-1} ) \,
 \sum_{n=1}^3 w_{in} \left ( \alpha(\zeta_{in},\mathbf{x}) -
 \alpha(\zeta_{i-1},\mathbf{x}) \right)
\end{split}\end{equation}

where $w_{in} = \omega_n \frac{\d s(\zeta_{in})}{\d h} \frac{\d
h(\zeta_{in})}{\d \zeta}$, $\omega_n$ is the weight for the $n^\text{th}$
Gauss-Legendre point, and $\zeta_{in}$ is the abscissa for that point
within the range $\zeta_{i-1} \rightarrow \zeta_i$.

\end{frame}

% -----------------------------------------------------------------------------

\newframe{Full Forward Model -- Radiative Transfer Integral (3)}

Although this strategy provides acceptable accuracy, it creates tremendous
mischief throughout the rest of the forward model by attempting to
optimize the calculation.  It has led to a complicated system of flags and
indices, especially to calculate derivatives.
\vspace*{0.1in}

An alternative strategy is to compute $\alpha(\zeta_{in},\mathbf{x})$ from
first principles if the error estimate is not small enough, and use
interpolation to $\zeta_{in}$ from $\alpha(\zeta_{i-1},\mathbf{x})$ and
$\alpha(\zeta_i,\mathbf{x})$ if it is small enough.  Then always use the
approximation in Equation (\ref{nine}) rather than trying to save a tiny
bit of computation by choosing whether to use the trapezoidal quadrature
or the Gauss-Legendre quadrature. \vspace*{0.1in}

This might well be faster by replacing a branch within the path loop with
a subtract, a multiply to interpolate $\alpha$ to $\zeta_{in}$, and
multiplying by $w_{in}$.  It would definitely be simpler, especially in
the case of computing derivatives, by eliminating numerous flags and
indices.  Notice that $w_{in}$ is calculated once for each path, then used
for every frequency, and for both the radiance and every derivative
evaluated at each frequency.

\end{frame}

% -----------------------------------------------------------------------------

\end{document}

% $Id$

% $Log$
% Revision 1.8  2019/09/04 00:22:00  vsnyder
% Corrected derivative w.r.t. state-vector elements on page 8, added page 9,
% embellished former page 9 (now page 10), improved discussion of singularity
% on page 12 (now page 13), improved alternative integration strategy
% description on page 15 (now page 16).
%
% Revision 1.7  2019/08/29 01:26:39  vsnyder
% Clarify stuff, correct inconsistent notations
%
% Revision 1.6  2019/08/27 01:22:42  vsnyder
% Minor improvements and typo corrections
%
% Revision 1.5  2017/10/31 18:16:44  vsnyder
% Add a graphic, some cannonball polishing
%
% Revision 1.4  2017/10/16 17:30:22  pwagner
% This modification was needed; dont know why
%
% Revision 1.3  2017/06/08 18:39:12  vsnyder
% Remove hyperlink stuff because Paul can't build with it included
%
% Revision 1.2  2017/06/08 01:27:55  vsnyder
% Update after presentation
%
