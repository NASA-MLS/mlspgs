\makeatletter\let\ifGm@compatii\relax\makeatother
\documentclass[landscape]{beamer}

\usepackage{amsmath}
\usepackage{graphicx}

% \ifx\pdfoutput\undefined
%   \pdfoutput=0
%   \usepackage{hyperref}
%   \hypersetup{%
%     hypertex,%
%     plainpages,%
%     hyperindex=true,%
%     hypertexnames=false%
%   }
%   % Specify the driver for the color package
%   \ExecuteOptions{dvips}
%   %\ExecuteOptions{xdvi}
% \else
%   \ifnum\pdfoutput>0
%     \usepackage{hyperref}
%     \hypersetup{%
%       pdftex,%
%       plainpages,%
%       hyperindex=true,%
%       pdfpagelabels,%
%       hypertexnames=false,%
%       colorlinks=true,%
%       linktocpage=true,%
%     }
%     % Specify the driver for the color package
%     \ExecuteOptions{pdftex}
%   \else
%     \usepackage{hyperref}
%     \hypersetup{%
%       hypertex,%
%       plainpages,%
%       hyperindex=true,%
%       hypertexnames=false%
%     }
%     % Specify the driver for the color package
%     \ExecuteOptions{dvips}
%     %\ExecuteOptions{xdvi}
%   \fi
% \fi
% 
% \hyperbaseurl{}
% \newcommand\hr[1]{\href{#1.dvi}{dvi}, \href{#1.pdf}{pdf}}
% \newcommand\h[1]{#1 (\hr{#1})}

\newcommand{\newframe}[2][]{\begin{frame}\frametitle{\hfill #2 \hfil}}
\renewcommand{\d}{\text{d}}

\title{Radiative Transfer Equation \\
in the Full Forward Model}

\author{Van Snyder}

\date{7 June 2017}

\begin{document}

\begin{frame}
  \titlepage
\end{frame}

% -----------------------------------------------------------------------------

\newframe{Radiative Transfer Equation}

The clear-sky non-scattering radiative-transfer equation for one frequency
is

\begin{equation}\label{one}
\frac{\d I(s,\mathbf{x})}{\d s} + \alpha(s,\mathbf{x})\, I(s,\mathbf{x}) =
 \alpha(s,\mathbf{x})\, B(T(s))
\end{equation}

where $\mathbf{x}$ is the state vector.  The solution can be written

\begin{equation}\label{two}
I(s_m,\mathbf{x}) = I(s_0,\mathbf{x})\, \mathcal{T}(s_m) +
 \int_{s_0}^{s_m} \mathcal{T}(s) \,
  \alpha(s,\mathbf{x})\, B(T(s)) \, \d s
\end{equation}

where
%
\begin{equation*}\begin{split}
B(T(s)) = \,& \frac{h \nu}k \left( \exp \left(\frac{h \nu}{k T(s)} \right)
 - 1 \right)^{-1} \\
\mathcal{T}(s) = \,& \exp \left( - \int_{s_0}^s \alpha(\sigma,\mathbf{x}) \,
 \d \sigma \right) \,,
\end{split}\end{equation*}

$s_0$ is deep space, and $s_m$ is the instrument position.

\end{frame}

% -----------------------------------------------------------------------------

\newframe{Radiative Transfer Equation}

In the Full Forward Model, Equation (\ref{two}) is transformed by observing
that

\begin{equation*}
\frac{\d \mathcal{T}(s)}{\d s} = -\alpha(s)\, \mathcal{T}(s)
\end{equation*}

giving

\begin{equation*}
I(s_m,\mathbf{x}) = I(s_0,\mathbf{x})\, \mathcal{T}(s_m) -
 \int_{s_0}^{s_m} B(T(s)) \, \frac{\d \mathcal{T}(s)}{\d s} \, \d s
\end{equation*}

which is then integrated by parts giving

\begin{equation*}
I(s_m,\mathbf{x}) = \left( I(s_0,\mathbf{x}) - B(s_m) \right) \,
 \mathcal{T}(s_m) +
  \int_{s_0}^{s_m} \mathcal{T}(s) \frac{\d B(T(s))}{\d s} \, \d s \,.
\end{equation*}

These transformations do not appear to be necessary; they eliminate one
multiply in the evaluation of the integrand in Equation (\ref{two}), which
is replaced by a subtract to approximate $\frac{\d B(T(s))}{\d s}$.

\end{frame}

% -----------------------------------------------------------------------------

\newframe{Derivatives of the Radiative Transfer Equation}

To solve for $\mathbf{x}$ using a Newton method, derivatives of
$I(s_m,\mathbf{x})$ with respect to $\mathbf{x}$ are needed.  These can be
gotten by differentiating Equation (\ref{one}) with respect to elements of
$\mathbf{x}(s)$:

\begin{equation}\label{three}
\frac{\d I^\prime_k(s)}{\d s} + \alpha(s,\mathbf{x}(s))\, I^\prime_k(s) =
 R(s,\mathbf{x}(s))
\end{equation}

where

\begin{equation*}
I^\prime_k(s) = \frac{\partial I(s,\mathbf{x}(s))}{\partial x_k(s)}
\end{equation*}

and

\begin{equation*}
R(s,\mathbf{x}(s)) =
 \frac{\partial \alpha(s,\mathbf{x}(s))}{\partial x_k(s)}
 \left( B(T(s)) - I(s,\mathbf{x}) \right) +
 \alpha(s,\mathbf{x}(s)) \frac{\partial B(T(s))}{\partial x_k(s)} \,.
\end{equation*}

\end{frame}

% -----------------------------------------------------------------------------

\newframe{Derivatives of the Radiative Transfer Equation}

$B(T(s))$ depends upon frequency and temperature, but not any mixing ratio.
It satisfies the differential equation

\begin{equation*}
\frac{\text{d} B(T(s))}{\text{d} T(s)} =
 \frac{B(T(s))}{T(s)^2} \left( \frac{h\nu}k + B(T(s)) \right)\,.
\end{equation*}

\begin{equation*}
\frac{\partial B(T(s))}{\partial x^k(s)} = 0 \text{ if }
 x^k(s) \text{ is not } T(s)\,.
\end{equation*}

\begin{equation*}
\alpha(s,\mathbf{x}(s)) = \sum_{k\neq T} f^k(s)\, \beta^k(s,T(s)) \,.
\end{equation*}

$\beta^k(s,\mathbf{x}^k(s))$ usually does not depend upon the mixing ratio
$f^k(s) \in \mathbf{x}(s)$ is the mixing ratio of the $k^\text{th}$
species, and mixing ratios do not depend upon temperature.  Therefore

\begin{equation*}
\frac{\partial \alpha(s,\mathbf{x}(s))}{\partial f^k(s)} =
\beta^k(s,T(s)) \text{ and }
\frac{\partial \alpha(s,\mathbf{x}(s))}{\partial T(s)} =
\sum_{k \neq T} f^k(s) \frac{\partial \beta^k(s,T(s))}{\partial T(s)}
\end{equation*}

\end{frame}

% -----------------------------------------------------------------------------

\newframe{Derivatives of the Radiative Transfer Equation}

Equation (\ref{three}) is of the same form as Equation (\ref{one}), with
$I(s,\mathbf{x})$ replaced by $I^\prime_k(s) = \frac{\partial
I(s_m,\mathbf{x})}{\partial x_k}$ and $\alpha(s,\mathbf{x})\, B(T(s))$
replaced by $R(s,\mathbf{x})$.  The solution is therefore of the same form
as Equation (\ref{two}), with $\alpha(s,\mathbf{x})\, B(T(s))$ replaced by
$R(s,\mathbf{x})$, \emph{viz}.

\begin{equation}\begin{split}\label{four}
\frac{\partial I(s_m,\mathbf{x}(s))}{\partial x_k(s)} = \,&
 \left.\frac{\partial I(s,\mathbf{x}(s))}{\partial x_k(s)} \right|_{s=s_0} \,
  \mathcal{T}(s_m) +
  \int_{s_0}^{s_m} \mathcal{T}(s) \, R(s,\mathbf{x}(s)) \, \d s \\
= \,&
 \int_{s_0}^{s_m} \mathcal{T}(s) \, R(s,\mathbf{x}(s)) \, \d s \\
\end{split}\end{equation}

because the derivative of radiance at the deep-space starting point,
$s_0$, with respect to any state-vector element, is zero.

\end{frame}

% -----------------------------------------------------------------------------

\newframe{Derivatives of the Radiative Transfer Equation}

We wrote $\mathbf{x}(s)$ instead of $\mathbf{x}$ to emphasize that these
derivatives are with respect to state vector quantities on the
integration path.  We calculate derivatives with respect to state vector
elements in solution profiles, which are needed for the Newton method,
using the chain rule

\begin{equation*}
\frac{\partial \cdot}{\partial x^k_i} =
\frac{\partial \cdot}{\partial \mathbf{x}^k(s)}
\frac{\partial \mathbf{x}^k(s)}{\partial x^k_i} =
\frac{\partial \cdot}{\partial \mathbf{x}^k(s)} \, \eta^k_i(s)
\end{equation*}

where $x^k_i$ is the $i^\text{th}$ element of the $k^\text{th}$ solution
profile state vector quantity and $\eta^k_i(s)$ is the interpolation
coefficient from its geolocation to the integration path.

\end{frame}

% -----------------------------------------------------------------------------

\newframe{Full Forward Model}

The Full Forward Model evaluates
$\frac{\partial I(s_m,\mathbf{x})}{\partial x^k}$ by differentiating
Equation (\ref{two}), which leads to an expression different from Equation
(\ref{four}), but which is necessarily equivalent as can be shown (with
some difficulty -- see
% \h{wvs-100}%
wvs-100%
) by integrating by parts.
\vspace*{0.1in}

The original advantage of this approach is that $I(s,\mathbf{x})$ is not
retained for each frequency; rather only $I(s_m,\mathbf{x})$ is retained.  For
PFA models, however,  $I(s,\mathbf{x})$ is retained, which would allow the
simpler expression in Equation (\ref{four}) to be used.

\end{frame}

% -----------------------------------------------------------------------------

\newframe{Full Forward Model}

The Full Forward Model evaluates $\mathcal{T}(s)$ as the exponential of a
sequence of integrals, with $\zeta = -\log_{10} P$ where $P$ is pressure, as the
independent variable:

\begin{equation}\label{five}
\mathcal{T}(\zeta_n) =
 \exp \left( - \sum_{i=1}^n \int_{\zeta_{i-1}}^{\zeta_i}
 \alpha(\zeta,\mathbf{x}) \,
 \frac{\d s}{\d h} \frac{\d h}{\d \zeta} \, \d \zeta \right)
\end{equation}

where $\zeta_0 = \zeta_{s_0}$ and $\zeta_n = \zeta_{s_n}$.
\vspace*{0.1in}

Each integral is approximated by a quadrature.  The integrand is evaluated
at $\zeta_{i-1}$ and $\zeta_i$.  An estimate of the error in a trapezoidal
approximation of the integral is formed.  If the error is not too large,
the trapezoidal estimate is used. \vspace*{0.1in}

If the error is too large, the integrand is evaluated at the abscissae of
the three-point Gauss-Legendre quadrature formula.

\end{frame}

% -----------------------------------------------------------------------------

\newframe{Full Forward Model}

For limb viewing, without considering refraction directly

\begin{equation*}
s = \sqrt{h^2 - h_t^2}
\end{equation*}

where $h_t$ is the geocentric tangent point height, and $h$ is the height
at the point a disance $s$ from the tangent point.

\includegraphics{wvs-139-1}

\end{frame}

% -----------------------------------------------------------------------------

\newframe{Full Forward Model}

Thereby

\begin{equation*}
\frac{\d s}{\d h} = \frac{h}{\sqrt{h^2 - h_t^2}} \,.
\end{equation*}

Therefore, the integrand in Equation (\ref{five}) has a square-root
singularity at the tangent point.  A square-root singularity in an
integrand exceeds the algebraic order of a quadrature formula, thereby
reducing its accuracy, so it should be removed.

\end{frame}

% -----------------------------------------------------------------------------

\newframe{Full Forward Model}

Using a rectangular quadrature for the integral in Equation (\ref{five})

\begin{equation*}\begin{split}
\int_{\zeta_{i-1}}^{\zeta_i} \alpha(\zeta,\mathbf{x})
 \frac{\d s}{\d h} \frac{\d h}{\d \zeta} \, \d \zeta \approx \,&
\alpha(\zeta_{i-1},\mathbf{x}) \int_{\zeta_{i-1}}^{\zeta_i} 
 \frac{\d s}{\d h} \frac{\d h}{\d \zeta} \, \d \zeta \\
 \equiv Q_{i-1} = \,&
 \alpha(\zeta_{i-1},\mathbf{x}) ( s_i - s_{i-1} ) \\
\end{split}\end{equation*}

that integral can be written

\begin{equation*}
\int_{\zeta_{i-1}}^{\zeta_i} \alpha(\zeta,\mathbf{x})
 \frac{\d s}{\d h} \frac{\d h}{\d \zeta} \, \d \zeta =
 Q_{i-1} + \int_{\zeta_{i-1}}^{\zeta_i}
  \left ( \alpha(\zeta,\mathbf{x}) - \alpha(\zeta_{i-1},\mathbf{x}) \right)
 \frac{\d s}{\d h} \frac{\d h}{\d \zeta} \, \d \zeta
\end{equation*}

which cancels the singularity provided $\left ( \alpha(\zeta,\mathbf{x}) -
\alpha(\zeta_{i-1},\mathbf{x}) \right) \rightarrow 0$ as $\zeta
\rightarrow \zeta_{i-1}$ faster than $\sqrt{h^2-h_t^2} \rightarrow 0$ as
$h \rightarrow h_t$.

\end{frame}

% -----------------------------------------------------------------------------

\newframe{Full Forward Model}

If the error estimate for the integral is small enough, a trapezoidal
approximation is constructed using

\begin{equation*}\begin{split}
\int_{s_{i-1}}^{s_i} \alpha(\zeta,\mathbf{x})\, \d s \approx \,&
Q_{i-1} + \frac12 \left ( \alpha(\zeta_i,\mathbf{x}) -
                          \alpha(\zeta_{i-1},\mathbf{x}) \right)
                          \, ( s_i - s_{i-1} ) \\
= \,& \frac12 \left ( \alpha(\zeta_{i-1},\mathbf{x}) +
                          \alpha(\zeta_i,\mathbf{x}) \right)
                          \, ( s_i - s_{i-1} ) \\
\end{split}\end{equation*}

If the error estimate for the integral is not small enough, a 3-point
Gauss-Legendre approximation is constructed using
%
\begin{equation}\begin{split}\label{six}
\int_{\zeta_{i-1}}^{\zeta_i} & \alpha(\zeta,\mathbf{x})
 \frac{\d s}{\d h} \frac{d h}{\d \zeta} \, \d \zeta \approx
Q_{i-1} + \\
 \,&
( \zeta_i - \zeta_{i-1} ) \,
 \sum_{n=1,3} w_{in} \left ( \alpha(\zeta_{in},\mathbf{x}) -
 \alpha(\zeta_{i-1},\mathbf{x}) \right)
\end{split}\end{equation}

where $w_{in} = \omega_n \frac{\d s(\zeta_{in})}{\d h} \frac{\d
h(\zeta_{in})}{\d \zeta}$, $\omega_n$ is the weight for the $n^\text{th}$
Gauss-Legendre point, and $\zeta_{in}$ is the abscissa for that point
within the range $\zeta_{i-1} \rightarrow \zeta_i$.

\end{frame}

% -----------------------------------------------------------------------------

\newframe{Full Forward Model}

Although this strategy provides acceptable accuracy, it creates tremendous
mischief throughout the rest of the forward model by attempting to
optimize the calculation.  It has led to a complicated system of flags and
indices, especially to calculate derivatives.
\vspace*{0.1in}

An alternative strategy is to compute $\alpha(\zeta_{in},\mathbf{x})$ from
first principles if the error estimate is not small enough, and use
interpolation to $\zeta_{in}$ from $\alpha(\zeta_{i-1},\mathbf{x})$ and
$\alpha(\zeta_i,\mathbf{x})$ if it is small enough.  Then always use the
approximation in Equation (\ref{six}) rather than trying to save a tiny
bit of computation by choosing whether to use the trapezoidal quadrature
or the Gauss-Legendre quadrature. \vspace*{0.1in}

This might well be faster by replacing a branch within the path loop with
a subtract, a multiply to interpolate $\alpha$ to $\zeta_{in}$, and
multiplying by $w_{in}$.  It would definitely be simpler, especially in
the case of computing derivatives, by eliminating numerous flags and
indices.  Notice that $w_{in}$ is calculated once for each path, then used
for every frequency, and both the radiance and every derivative evaluated
at each frequency.

\end{frame}

\end{document}

% $Id$

% $Log$
% Revision 1.4  2017/10/16 17:30:22  pwagner
% This modification was needed; dont know why
%
% Revision 1.3  2017/06/08 18:39:12  vsnyder
% Remove hyperlink stuff because Paul can't build with it included
%
% Revision 1.2  2017/06/08 01:27:55  vsnyder
% Update after presentation
%
