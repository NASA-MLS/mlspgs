\documentclass[11pt]{article}
\usepackage[fleqn]{amsmath}\textwidth 6.5in
\oddsidemargin -0.25in
%\evensidemargin -0.5in
\topmargin -0.25in
\textheight 9in

\newcommand{\docname}{\bf wvs-058}
\newcommand{\docdate}{17 September 2007}

\begin{document}

%\tracingcommands=1
\newlength{\hW} % heading box width
\newlength{\pW} % page number field width
\settowidth{\hW}{\docname}
\settowidth{\pW}{Page \pageref{lastpage}\ of \pageref{lastpage}}
\ifdim \pW > \hW \setlength{\hW}{\pW} \fi
\makeatletter
\def\@biblabel#1{#1.}
\newcommand{\ps@twolines}{%
  \renewcommand{\@oddhead}{%
    \docdate\hfill\parbox[t]{\hW}{{\hfill\docname}\newline
                          Page \thepage\ of \pageref{lastpage}}}%
\renewcommand{\@evenhead}{}%
\renewcommand{\@oddfoot}{}%
\renewcommand{\@evenfoot}{}%
}%
\makeatother
\pagestyle{twolines}

\vspace{-10pt}
\begin{tabbing}
\phantom{References: }\= \\
To: \>Dong, Bill\\
Subject: \>Mie solution in terms of Bessel functions\\
From: \>Van Snyder\\
\end{tabbing}

\parindent 0pt \parskip 6pt
\vspace{-10pt}

The quantities in section 4.4.2.1 \emph{Mie efficiencies} in JPL D-19299,
EOS MLS Algorithm Theoretical Basis for Cloud Measurements, use $j$ for a
subscript.  To avoid confusion with the spherical Bessel function of the
first kind, $n$ is used here instead.

The recurrence and initial conditions given for $W_n$, \emph{viz.}
%
\begin{equation}\begin{split}\label{three}
W_n =\,& \left(\frac{2n-1}{\chi}\right) W_{n-1} - W_{n-2}\\
W_0 =\,& \sin\chi + i \cos\chi = -i e^{-i \chi}\\
W_{-1} =\,& \cos\chi - i \sin\chi = e^{-i \chi}
\end{split}\end{equation}
%
define a Hankel function of the second kind of half-integer order, \emph{viz.}
%
\begin{equation}\label{four}
W_n = \sqrt{\frac{\pi \chi}2}
        \left( J_{n+\frac12} (\chi) - i Y_{n+\frac12}(\chi) \right) =
      j_n(\chi) - i\, y_n(\chi) =
      \sqrt{\frac{\pi \chi}2}\, H^{(2)}_{n+\frac12}(\chi) =
      \chi\, h^{(2)}_n (\chi)\,.
\end{equation}
%
The subroutine {\tt MieCoeff} in the module {\tt MieTheory} evaluates
$W_n$ using this recurrence in the forward direction.

Every second-order recurrence has two solutions.  Typically, one grows in
magnitude as the index increases, and the other shrinks.  When the
recurrence is evaluated in finite arithmetic, a combination of its two
solutions is necessarily computed.  When it is evaluated in the forward
direction, the solution that grows as the index increases comes to
dominate the calculation.  When it is evaluated in the reverse direction,
the solution that grows as the index decreases comes to dominate the
solution.

The real part of $h^{(2)}_n(\chi)$ decreases as $n$ increases, and the
imaginary part increases.  Thus the recurrence in Equation (\ref{three})
is stable in the forward direction for the real part of the solution, and
stable in the reverse direction for the imaginary part of the solution,
only if $n \ll (\chi+1)/2$.  For $n > (\chi+1)/2$, forward recurrence
eventually computes $y_n(\chi)+i y_n(\chi)$, and backward recurrence
eventually computes $j_n(\chi)+i j_n(\chi)$, no matter what initial
conditions are given:  The real and imaginary parts need to be computed
separately, the real part by backward recurrence, and the imaginary part
by forward recurrence.

High-quality library routines ought to be used instead of the recurrence
in 4.4.2.1 to compute $W_n$.

The recurrence and initial condition given for $A_n$, \emph{viz.}
%
\begin{equation}\begin{split}\label{five}
A_n =\,& - \frac{n}{m \chi} + \left( \frac{n}{m \chi} - A_{n-1} \right) ^{-1} \\
A_0 =\,& \text{cot} (m \chi)
\end{split}\end{equation}
%
define an expression involving Bessel functions, \emph{viz.}
%
\begin{equation}
A_n = - \frac{n}{m \chi} + \frac{J_{n-\frac12}(m \chi)}{J_{n+\frac12}(m \chi)}
    = - \frac{n}{m \chi} + \frac{j_{n-1}(m \chi)}{j_n(m \chi)}.
\end{equation}
%
The recurrence for $A_n$ is probably stable, since it is a first-order
recurrence.  There is the possibility, however, that $\frac{n}{m \chi} -
A_{n-1} = 0$ for some values of $n$, $m$ and $\chi$.  Alternatively $A_n$
can be written as the continued fraction
%
\begin{equation}
A_n = -\left(a_{n,0}-\frac1{a_{n,1}-}\, \frac1{a_{n,2}-} \dots\right)
\end{equation}
where $a_{n,0} = \frac{n}{m \chi}$ and $a_{n,i} = \frac{2(n-i)-1}{m
\chi}$.
%
By taking the continued fraction only $k$ levels deep it can be written
as $A_{n,k} \approx P_{n,k}/Q_{n,k}$ where $P_{n,k}$ and $Q_{n,k}$ are
the \emph{k$^{th}$ convergents of $A_n$}.  The convergents can be
calculated from
%
\begin{equation}\begin{array}{lll}
P_{n,-1}=1, & P_{n,0}=a_{n,0}, & P_{n,k} = a_{n,k} P_{n,k-1} - P_{n,k-2} \\
Q_{n,-1}=0, & Q_{n,0}=1,      & Q_{n,k} = a_{n,k} Q_{n,k-1} - Q_{n,k-2}
\end{array}\end{equation}
%
which allows evaluating the continued fraction for $A_n$ in the
``forward'' direction until convergence is achieved.

The quantities $a_n$ and $b_n$, \emph{viz.}
%
\begin{equation}
a_n = \frac{(A_n/m + n/\chi) \Re(W_n) - \Re(W_{n-1})}
           {(A_n/m + n/\chi) W_n - W_{n-1}}
\end{equation}
\begin{equation}
b_n= \frac{(m A_n + n/\chi) \Re(W_n) - \Re(W_{n-1})}
           {(m A_n + n/\chi) W_n - W_{n-1}}
\end{equation}
%
are expressions involving Bessel functions. Their reciprocals in terms of
spherical Bessel functions of the first and second kind are
%
\begin{equation}\begin{split}
\frac1{a_n} =\,& 1 - i \frac{\left(n(1-m^2) j_n(m \chi) -
                                   m \chi j_{n-1} (m \chi) \right) y_n(\chi) +
                              m^2 \chi j_n(m \chi) y_{n-1}(\chi)}
                            {\left(n(1-m^2) j_n (m \chi) +
                                   m \chi j_{n-1}(m \chi) \right) j_n(\chi) +
                              m^2 \chi j_n(m \chi) j_{n-1}(\chi)}
\\
\frac1{b_n} =\,& 1 - i \frac{m\, y_n(\chi) j_{n-1}(m \chi) -
                             y_{n-1}(\chi) j_n(m \chi)}
                            {m\, j_n(\chi) j_{n-1}(m \chi) -
                             j_{n-1}(\chi) j_n(m \chi)}
\end{split}\end{equation}
%
Assuming Equation (\ref{five}) is stable, and writing $A_n$ instead of
its equivalent in terms of Bessel functions, the reciprocals of $a_n$ and
$b_n$ can be written succinctly in terms of Bessel functions with real
arguments, i.e., that don't involve $m$, which is the complex index of
refraction.
%
\begin{equation}\begin{split}
\frac1a_n =\,& 1 + i \frac{(A_n/m + n/\chi) y_n(\chi) - y_{n-1}(\chi)}
                    {(A_n/m + n/\chi) j_n(\chi) - j_{n-1}(\chi)}
          = 1 + i \frac{(A_n/m + n/\chi) Y_{n+\frac12}(\chi) - Y_{n-\frac12}(\chi)}
                    {(A_n/m + n/\chi) J_{n+\frac12}(\chi) - J_{n-\frac12}(\chi)} \\
\frac1b_n =\,& 1 + i \frac{(A_n m + n/\chi) y_n(\chi) - y_{n-1}(\chi)}
                    {(A_n m + n/\chi) j_n(\chi) - j_{n-1}(\chi)}
          =1 + i \frac{(A_n m + n/\chi) Y_{n+\frac12}(\chi) - Y_{n-\frac12}(\chi)}
                    {(A_n m + n/\chi) J_{n+\frac12}(\chi) - J_{n-\frac12}(\chi)}
\,,
\end{split}\end{equation}
%
where factors of $\sqrt{\frac{\pi\chi}2}$ have been canceled in the latter forms.

This doesn't separate the real and imaginary parts, since $A_n$ and $m$
are complex.

Another approach to evaluating $W_n$ is to define $\Omega_n =
\frac{W_n}{W_{n-1}}$, and divide Equation (\ref{three}) by $W_{n-1}$
giving
%
\begin{equation}
\Omega_n = \frac{2n-1}\chi - \frac1{\Omega_{n-1}} ,\,\,
\Omega_0 = i
\end{equation}
%
For positive real $\chi$, the real and imaginary parts of $\Omega_n$ are
both positive and finite.  For fixed $n$ as $\chi \rightarrow \infty$,
$\Omega_n \sim n/\chi + i$.  $W_n$ can then be computed using $W_n =
\Omega_n W_{n-1}$.  The first four values of $\Omega_n$ are
%
\begin{equation}\begin{array}{c}
 i \\
 \\
 \frac1\chi+i \\
 \\
 2\frac{\chi^2+\frac32}{\chi(\chi^2+1)} + i \frac{\chi^2}{\chi^2+1} \\
 \\
 3\frac{\chi^4+4\chi^2+15}{\chi(\chi^4+3\chi^2+9)} +
   i \frac{\chi^4}{\chi^4+3\chi^2+9}\,.
\end{array}\end{equation}
%
Similarly, defining $\omega_n = \frac{W_{n-1}}{W_n}$ we have
\begin{equation}
\omega_n = \frac\chi{2 n - 1 - \chi \omega_{n-1}},\,\, \omega_0 = -i\,.
\end{equation}
%
For positive real $\chi$, the real and imaginary parts of $\omega_n$ are
both positive and finite.  For fixed $n$ as $\chi \rightarrow \infty$,
$\omega_n \sim n/\chi - i$.  $W_n$ can then be computed using $W_n =
\frac{\omega_n}{W_{n-1}}$.  The first four values of $\omega_n$ are
%
\begin{equation}\begin{array}{c}
 -i \\
 \\
 \frac\chi{\chi^2+1} - i \frac{\chi^2}{\chi^2+1}\\
 \\
 2 \chi \frac{\chi^2+\frac32}{\chi^4+3\chi^2+9} -
  i \frac{\chi^4}{\chi^4+3\chi^2+9}\\
 \\
 3 \chi \frac{\chi^4+4\chi^2+15}{\chi^6+6\chi^4+45=chi^2+225} -
  i \frac{\chi^6}{\chi^6+6\chi^4+45\chi^2+225}\,.
\end{array}\end{equation}

\label{lastpage}
\end{document}
% $Id$
