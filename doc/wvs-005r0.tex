\documentclass[11pt]{article}
\usepackage[fleqn]{amsmath}\textwidth 6.25in
\oddsidemargin -0.25in
%\evensidemargin -0.5in
\topmargin -0.5in
\textheight 9.00in

\begin{document}

%\tracingcommands=1
\newlength{\hW} % heading box width
%\settowidth{\hW}{\bf wvs-005}
\settowidth{\hW}{Page \pageref{lastpage}\ of \pageref{lastpage}}
\makeatletter
\def\@biblabel#1{#1.}
\newcommand{\ps@twolines}{%
  \renewcommand{\@oddhead}{%
    1 May 2000\hfill\parbox[t]{\hW}{{\bf wvs-005}\newline
                          Page \thepage\ of \pageref{lastpage}}}%
\renewcommand{\@evenhead}{}%
\renewcommand{\@oddfoot}{}%
\renewcommand{\@evenfoot}{}%
}%
\makeatother
\pagestyle{twolines}

\vspace{-10pt}
\begin{tabbing}
\phantom{References: }\= \\
To: \>Dave, Nathaniel, Joe\\
Subject: \>What would Fred Krogh do for the Level 2 Program?\\
From: \>Van Snyder\\
\end{tabbing}

\parindent 0pt \parskip 3pt
\vspace{-20pt}

The mathematical problem immanent in the level 2 retrieval is a nonlinear
least-squares problem.

Fred Krogh has an unique combination of knowledge, skills and talents in
computational mathematics and mathematical software.  In particular, he
has written two software packages to solve nonlinear least-squares
problems.  One of these packages is used by the trajectory planning
group; it routinely and efficiently solves problems that no other package
is able to solve at all.

Fred also has a deep understanding of the relation between linear
algebra, software to solve problems in linear algebra, and the solution
of nonlinear least-squares problems.

Fred has already given me substantial advice on methods and software to
solve nonlinear least-squares problems, and software and algorithms to
solve the linear problems that arise in solving nonlinear least-squares
problems.  In particular, he has given me software to solve nonlinear
least-squares problems, that already has many of the features we will
need in the level 2 program.  I would find his advice to be useful in the
development of the level 2 program.

I propose to institute a consulting contract with Fred, for 17 days of
work (20 hours per month for the remainder of the fiscal year), at
\$47.50 per hour.  This would cost \$6460.00.  I expect that exploiting
his knowledge could well save several work months of effort, and result
in software of superior reliability, accuracy and performance.

Fred is a JPL retiree, and already has a consulting agreement with JPL,
under which he provides services to three groups.  I understand that the
paperwork to add funds and work authorization to an existing contract is
substantially simpler than creating a new consulting agreement.

\label{lastpage}
\end{document}
% $Id$
