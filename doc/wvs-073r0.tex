\documentclass[11pt]{article}
\usepackage[fleqn]{amsmath}\textwidth 6.5in
\oddsidemargin -0.25in
%\evensidemargin -0.5in
\topmargin -0.25in
\textheight 9.25in

\newcommand{\docname}{\bf wvs-073}
\newcommand{\docdate}{25 February 2008}

\begin{document}

%\tracingcommands=1
\newlength{\hW} % heading box width
\newlength{\pW} % page number field width
\settowidth{\hW}{\docname}
\settowidth{\pW}{Page \pageref{lastpage}\ of \pageref{lastpage}}
\ifdim \pW > \hW \setlength{\hW}{\pW} \fi
\makeatletter
\def\@biblabel#1{#1.}
\newcommand{\ps@twolines}{%
  \renewcommand{\@oddhead}{%
    \docdate\hfill\parbox[t]{\hW}{{\hfill\docname}\newline
                          Page \thepage\ of \pageref{lastpage}}}%
\renewcommand{\@evenhead}{}%
\renewcommand{\@oddfoot}{}%
\renewcommand{\@evenfoot}{}%
}%
\makeatother
\pagestyle{twolines}

\vspace{-10pt}
\begin{tabbing}
\phantom{References: }\= \\
To: \>Nathaniel, Alyn, Bill, David, Mike, Paul\\
Subject: \>Summary of ``faster retrieval'' meeting\\
From: \>Van Snyder\\
%Reference: \>wvs-058
\end{tabbing}

\parindent 0pt \parskip 6pt
\vspace{-10pt}

The slides for my presentation of variable projection are in {\tt
$\sim$vsnyder/mls/doc/wvs-072.pdf}.  I think the approach described therein,
and the approach Bill discussed, are very similar.  See also {\tt
http://www-sccm.stanford.edu/pub/sccm/sccm-02-07.ps.gz} (also in
{\tt $\sim$vsnyder/mls/doc}).

In addition to these and the topics on Nathaniel's slides, we should not forget
that we also discussed

\begin{itemize}
\item Tomographic methods to decouple ray tracing, allowing to solve for
temperature and mixing ratio separately in each parcel.  There is a chance that
these are very closely related to the variable-projection and extinction-based
methods, with the additional possibility that decoupling can be done with an
FFT-based approach, which would have $O(n \log n)$ complexity instead of
$O(n^3)$ complexity.

\item New computing platforms, especially IBM's Cell-BE (Broadband Engine)
architecture.  IBM has Fortran and C compilers, and system-development kits,
for this processor.  IBM plans to produce blade-based clusters using this
processor.  The Sony Playstation 3 is built using this processor.  Mercury
Computer Systems, Inc.~(http://mc.com/microsites/cell/) offers systems
(including workstations, and 1U and blade-based clusters), PCI boards, and
software.  It may be worthwhile to investigate using graphics chips for some
steps as well.

\item VLSI neural nets, especially analog VLSI (for speed).

\end{itemize}


\label{lastpage}
\end{document}
% $Id$
