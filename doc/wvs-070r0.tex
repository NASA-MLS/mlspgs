\documentclass[11pt]{article}
\usepackage[fleqn]{amsmath}\textwidth 6.5in
\oddsidemargin -0.25in
%\evensidemargin -0.5in
\topmargin -0.25in
\textheight 9in

\newcommand{\docname}{\bf wvs-070}
\newcommand{\docdate}{1 February 2008}

\begin{document}

%\tracingcommands=1
\newlength{\hW} % heading box width
\newlength{\pW} % page number field width
\settowidth{\hW}{\docname}
\settowidth{\pW}{Page \pageref{lastpage}\ of \pageref{lastpage}}
\ifdim \pW > \hW \setlength{\hW}{\pW} \fi
\makeatletter
\def\@biblabel#1{#1.}
\newcommand{\ps@twolines}{%
  \renewcommand{\@oddhead}{%
    \docdate\hfill\parbox[t]{\hW}{{\hfill\docname}\newline
                          Page \thepage\ of \pageref{lastpage}}}%
\renewcommand{\@evenhead}{}%
\renewcommand{\@oddfoot}{}%
\renewcommand{\@evenfoot}{}%
}%
\makeatother
\pagestyle{twolines}

\vspace{-10pt}
\begin{tabbing}
\phantom{References: }\= \\
To: \>Dong, Van\\
Subject: \>Reducing Mie phase function integrated
 scattering from double to single integrals \\
From: \>Van Snyder\\
Reference: \>Cloud ATBD JPL D-19299 (4 June 2004), wvs-068
\end{tabbing}

\parindent 0pt \parskip 6pt
\vspace{-10pt}

\renewcommand{\d}{\text{d}}

The phase function is

\begin{equation}\label{one}
p(\theta,r) = \frac{p_0(\theta,r)}{C(r)}
\end{equation}

where

\begin{equation}\begin{split}\label{two}
p_0(\theta,r) =\,& |S_1(\theta,r)|^2 + |S_2(\theta,r)|^2 =
  \Re (S_1(\theta,r))^2 + \Im (S_1(\theta,r))^2 +
  \Re (S_2(\theta,r))^2 + \Im (S_2(\theta,r))^2 \\
S_1 =\,& \sum_{j=1}^\infty \frac{2j+1}{j(j+1)} \left(
 a_j(r,T) \frac{\d P_j^1(\cos\theta)}{\d \theta} +
 b_j(r,T) \frac{P_j^1(\cos\theta)}{\sin\theta} \right) \\
S_2 =\,& \sum_{j=1}^\infty \frac{2j+1}{j(j+1)} \left(
 a_j(r,T) \frac{P_j^1(\cos\theta)}{\sin\theta} +
 b_j(r,T) \frac{\d P_j^1(\cos\theta)}{\d \theta} \right) \\
C(r) =\,& \frac12 \int_0^\pi p_0(\theta,r) \sin\theta \, \d \theta \,. \\
\end{split}\end{equation}

Each of the terms in $p_0(\theta,r)$ can be written in the form

\begin{equation}\label{three}
\left(\sum_{j=1}^\infty u_j(r) v_j(\theta) + w_j(r) x_j(\theta) \right)^2
\end{equation}

where $u_j(r)$ is one of $\Re a_j$ or $\Im a_j$, $w_j(r)$ is one of $\Re b_j$
or $\Im b_j$, $v_j(\theta) = \frac{2j+1}{j(j+1)}
\frac{P^1_j(\cos\theta)}{\sin\theta}$ and $x_j(\theta) = \frac{2j+1}{j(j+1)}
\frac{\d P^1_j(\cos \theta)}{\d\theta}$.

Equation (\ref{three}) can in turn be written

\begin{equation}\begin{split}\label{four}
\left(\sum_{j=1}^\infty u_j(r) v_j(\theta) + w_j(r) x_j(\theta) \right)^2 = \,&
\left(\sum_{j=1}^\infty u_j(r) v_j(\theta) + w_j(r) x_j(\theta) \right)
\left(\sum_{i=1}^\infty u_i(r) v_i(\theta) + w_i(r) x_i(\theta) \right) \\
= \,&
\sum_{i=1}^\infty \sum_{j=1}^\infty
 ( u_i(r) v_i(\theta) + w_i(r) x_i(\theta) )
 ( u_j(r) v_j(\theta) + w_j(r) x_j(\theta) ) \\
= \,&
\sum_{i=1}^\infty \sum_{j=1}^\infty
 u_i(r) u_j(r) v_i(\theta) v_j(\theta) +
 u_i(r) w_j(r) v_i(\theta) x_j(\theta) + & \\
\,&
 w_i(r) u_j(r) x_i(\theta) v_j(\theta) +
 w_i(r) w_j(r) v_i(\theta) x_j(\theta) \,.
\end{split}\end{equation}

Each of the terms in Equation (\ref{four}) can in turn be written in the form
$f_{ij}(r) g_{ij}(\theta)$ where $f_{ij}(r)$ is one of $u_i(r) u_j(r)$,
$u_i(r) w_j(r)$ or $w_i(r) w_j(r)$, and $g_{ij}(\theta)$ is one of
$v_i(\theta) v_j(\theta)$, $v_i(\theta) x_j(\theta)$ or $x_i(\theta)
x_j(\theta)$.

Thus the normalization function $C(r)$ can be written

\begin{equation}
C(r) = \frac12 \int_0^\pi p_0(\theta,r)\,\d\theta = \sum \left(
 \sum_{i=1}^\infty\sum_{j=1}^\infty f_{ij}(r) G_{ij} \right) \text{ where }
 G_{ij} = \frac12 \int_0^\pi g_{ij}(\theta) \sin\theta\, \d\theta
\end{equation}

and the outer sum is taken over the possible forms of $f_{ij}(r)$ and
$G_{ij}$.

The quantities $G_{ij}$ can be evaluated in advance and tabulated.  It is not
necessary to evaluate $f_{ij}(r)$ or $G_{ij}$ for values of $i$ and $j$ that
are larger than the value of $j$ for which $a_j$ and $b_j$ are underflowed.  At
240 GHz, this is less than 12.  Up to 640 GHz, this is about 25, and is
probably not much more than 50 for 2.5 THz.

Thus, using this form for $C(r)$, although messy, the integral

\begin{equation}
P(\theta) = \frac\pi{\beta_{c\_s}} \int_0^\infty
 r^2 n(r) \xi_s(r) p(\theta,r) \, \d r =
 \frac\pi{\beta_{c\_s}} \int_0^\infty
 r^2 n(r) \xi_s(r)
  \frac{p_0(\theta,r)}{C(r)} \,\d r
\end{equation}

and its derivative w.r.t.\ IWC are no longer double integrals ($p_0(\theta,r)$
does not depend upon IWC).

The derivative

\begin{equation}
\frac{\partial C(r)}{\partial T} =
 \sum \left( \sum_{i=1}^\infty \sum_{j=1}^\infty
 \frac{\partial f_{ij}(r)}{\partial T} G_{ij} \right)
 \text{, where }
 \frac{\partial f_{ij}(r)}{\partial T} \text{ is of the form }
 y_i \frac{\partial z_j}{\partial T} +
 \frac{\partial y_i}{\partial T} z_j
\end{equation}

and $y$ and $z$ are one of $u$ and $w$, respectively, is messy but
straightforward.  Thus the derivative

\begin{equation}
\frac{\partial p(\theta,r)}{\partial T}  =
 \frac1{C(r)} \left( \frac{\partial p_0(\theta,r)}{\partial T} -
  p(\theta,r) \frac{\partial C(r)}{\partial T} \right)
\end{equation}

is again messy but straightforward, and can be written in terms of the
pre-tabulated integrals $G_{ij}$.  Thus, the integral

\begin{equation}\begin{split}
\frac{\partial P(\theta)}{\partial T} =\,&
 \frac{\pi}{\beta_{c\_s}} \int_0^\infty r^2\, n(r)\, \xi_s(r)\, p(\theta,r)\,
 \left( \frac1{n(r)}\, \frac{\partial n(r)}{\partial T} +
        \frac1{\xi_s(r)}\, \frac{\partial \xi_s(r)}{\partial T} +
        \frac1{p(\theta,r)}\, \frac{\partial p(\theta,r)}{\partial T}
 \right) \, \d r \\
 \,&
 -\frac{P(\theta)}{\beta_{c\_s}}\, \frac{\partial \beta_{c\_s}}{\partial T} \\
\end{split}\end{equation}

is also no longer a double integral.

\label{lastpage}
\end{document}
% $Id$
