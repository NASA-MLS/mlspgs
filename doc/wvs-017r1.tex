\documentclass[12pt]{article}

\usepackage[fleqn]{amsmath}\textwidth 6.25in
\oddsidemargin -0.25in
\evensidemargin -0.25in
\topmargin -0.5in
\textheight 9.00in

\begin{document}

%\tracingcommands=1
\newlength{\hW} % heading box width
\settowidth{\hW}{\bf wvs-017r1}
%\settowidth{\hW}{Page \pageref{lastpage}\ of \pageref{lastpage}}
\makeatletter
\def\@biblabel#1{#1.}
\newcommand{\ps@twolines}{%
  \renewcommand{\@oddhead}{%
    17 June 2004\hfill\parbox[t]{\hW}{{\bf wvs-017r1}\newline
                          Page \thepage\ of \pageref{lastpage}}}%
\renewcommand{\@evenhead}{}%
\renewcommand{\@oddfoot}{}%
\renewcommand{\@evenfoot}{}%
}%
\makeatother
\pagestyle{twolines}

\vspace{-10pt}
\begin{tabbing}
\phantom{References: }\= \\
To: \>Bill\\
Subject: \>Pre-frequency-averaging\\
From: \>Van Snyder\\
\end{tabbing}

\parindent 0pt \parskip 3pt
\vspace{-20pt}

This note expands on and analyzes some aspects of pre-frequency-averaging.

Consider the frequency-averaged radiance from one channel

\begin{equation}\label{first}\begin{split}
I =& \int \phi(\nu) \sum_{i=1}^{2N_p} \Delta B_{in} \tau_{in}^s \tau_{in}^w \; d\nu
\approx\ \sum_{n=1}^{N_f} \phi_n \Delta \nu_n \sum_{i=1}^{2N_p} \Delta B_{in}
 \tau_{in}^s \tau_{in}^w \\
 &\text{ where }
 \tau_{in}^s =\; \exp\left(-\sum_{j=2}^i \sum_{k \in \text{strong}}
 \delta_{jkn}\right),\;
 \tau_{in}^w =\; \exp\left(-\sum_{j=2}^i \sum_{k \in \text{weak}}
 \delta_{jkn}\right) \;,
\end{split}\end{equation}

$n$ indexes frequencies within the channel, $i$ and $j$ index path positions,
$k$ indexes spectral lines, $\phi_n$ is the filter function for the
$n^\text{th}$ frequency in the channel, and $\tau_{in}^s$ and $\tau_{in}^w$ are
factors of the transmission due to lines that have strong and weak variation
within the channel, respectively.  Let $\phi_n \Delta\nu_n = \overline{\phi_n}$
and expand $\tau_{in}^w$ using two terms of a Taylor series, giving

\begin{equation}\label{taylor}
I \approx\; \sum_{n=1}^{N_f} \overline{\phi_n} \sum_{i=1}^{2N_p} \Delta B_{in}
 \tau_{in}^s \tau_{in}^w = 
 \sum_{n=1}^{N_f} \overline{\phi_n} \sum_{i=1}^{2N_p} \Delta B_{in}
 \tau_{in}^s
 \left(1-\sum_{j=2}^i
  \sum_{k \in \text{weak}} \delta_{jkn}\right) \;.
\end{equation}

Since the differential Planck radiation function $\Delta B_{in}$ changes very
slowly with frequency within one channel, we can replace it with a channel
average $\overline{\Delta B_i}$, which can be adequately approximated by
evaluating $\Delta B_{in}$ at the channel center frequency.  Replacing $\Delta
B_{in}$ with $\overline{\Delta B_i}$and expanding Equation (\ref{taylor}) we
have

\begin{equation}\label{split}
I \approx\;
  \sum_{n=1}^{N_f} \overline{\phi_n} \sum_{i=1}^{2N_p} \overline{\Delta B_i}
    \tau_{in}^s -
  \sum_{n=1}^{N_f} \overline{\phi_n} \sum_{i=1}^{2N_p} \overline{\Delta B_i}
    \tau_{in}^s \sum_{j=2}^i \sum_{k \in \text{weak}} \delta_{jkn} \;.
\end{equation}

Exchanging the order of summation over $i$ and $n$ and replacing
$\sum_{n=1}^{N_f} \overline{\phi_n} \tau_{in}^s$ by $\overline{\tau_i^s}$,
we have

\begin{equation}\label{exchange}
I \approx\; \sum_{i=1}^{2N_p} \overline{\Delta B_i} \left( \overline{\tau_i^s} -
  \sum_{n=1}^{N_f}
  \overline{\phi_n} \tau_{in}^s \sum_{j=2}^i
  \sum_{k \in \text{weak}} \delta_{jkn} \right)
\end{equation}

Define the set \{weak\} such that $\delta_{jkn}$ changes slowly as a function
of $n$ so long as $k \in$ \{weak\}, i.e., it represents a weak spectral line or
a weak tail of a line that has its center far from the channel.  We can
therefore replace $\delta_{jkn}$ for $k \in$ \{weak\} by some kind of average
w.r.t. $n$, \emph{viz}. $\overline{\delta_{jk}^w}$.  Letting
$\overline{\delta_j^w} = \sum_{k\in\text{weak}} \overline{\delta_{jk}^w}$ we
have

\begin{equation}
\sum_{n=1}^{N_f}
  \overline{\phi_n} \tau_{in}^s \sum_{j=2}^i
  \sum_{k \in \text{weak}} \delta_{jkn} \approx
\left(\sum_{n=1}^{N_f}
  \overline{\phi_n} \tau_{in}^s\right) \left(\sum_{j=2}^i
  \overline{\delta_j^w} \right) =
\overline{\tau_i^s} \sum_{j=2}^i \overline{\delta_j^w}
\end{equation}

Use this result in Equation (\ref{exchange}), giving

\begin{equation}\label{final}
I \approx\; \sum_{i=1}^{2N_p} \overline{\Delta B_i} \left( \overline{\tau_i^s}
  - \sum_{n=1}^{N_f} \overline{\phi_n} \tau_{in}^s 
  \sum_{j=2}^i \overline{\delta_j^w} \right) =
  \sum_{i=1}^{2N_p} \overline{\Delta B_i}\; \overline{\tau_i^s}
   \left( 1 - \sum_{j=2}^i \overline{\delta_j^w} \right) \;.
\end{equation}

The quantities $\overline{\delta_j^w}$ can be interpolated from tables computed
off line separately for each channel, species, pressure and temperature.  The
quantities $\overline{\tau_i^s}$ are computed on line, from the quantities
$\tau_{in}^s$, $\phi_n$ and $\Delta \nu_n$.

Returning to Equation (\ref{first}) and using the same argument about the
small variation of $\delta_{jkn}$ for $k \in$ \{weak\},

\begin{equation}\label{exp1}\begin{split}
 I \approx\;& \sum_{i=1}^{2N_p} \overline{\Delta B_i}
    \sum_{n=1}^{N_f} \overline{\phi_n} \tau_{in}^s \tau_{in}^w
    \approx \sum_{i=1}^{2N_p} \overline{\Delta B_i}\; \overline{\tau_{i}^w}
    \sum_{n=1}^{N_f} \overline{\phi_n} \tau_{in}^s =
    \sum_{i=1}^{2N_p} \overline{\Delta B_i} \; \overline{\tau_i^s} \;
     \overline{\tau_i^w}\;,\\
 & \text{ where }
 \tau_{in}^w =\;
   \exp\left(-\sum_{j=2}^i \sum_{k \in \text{weak}} \delta_{jkn} \right)
 \approx \overline{\tau_{i}^w} =
  \exp\left(-\sum_{j=2}^i \overline{\delta_j^w} \right)\;.
\end{split}\end{equation}

Equation (\ref{exp1}) illustrates that the argument concerning weak lines also
applies without a Taylor series expansion for the exponential function. 
Replacing $\overline{\tau_{i}^w}$ by two terms of a Taylor series gives
Equation (\ref{final}).

\label{lastpage}
\end{document}
% $Id$
