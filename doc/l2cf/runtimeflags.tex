\section{Runtime Flags}
\label{sec:Runtime Flags}

The original purpose of mlsl2's Runtime Flags was to hold Boolean values so that
mlsl2 could respond to a data condition appropriately by switching between
different lines of the l2cf. 

What has evolved is more flexible. The Runtime Flags
can hold strings or numerical data, too. They can be used in control structures and
to assign values to data internal to mlsl2.

There are still places in the code and in the wiki pages where we refer to them
as Runtime Booleans. The evolution hasn't always been smooth and it's not yet
complete.

\citet{MLSWikiBooleans}


\subsection{\snippet{Boolean}}
The declaration of a Runtime Flag is one of the areas in which the older purpose
can still be seen clearly.

\lstinputlisting{declaration-1.l2cf}
\subsection{\snippet{Logical formula}}
The \snippet{formula} field is optional. If present it can refer to a
previously-defined flag like this
\lstinputlisting{declaration-2.l2cf}
In most cases mlsl2 will automatically attempt to evaluate any names it finds in
\snippet{formula} until it can't find any more. The \snippet{formula} must contain only
previously-defined \snippet{Boolean}, \emph{true}, \emph{false}, the logical
operators \emph{and, or, not}, and the logical relations 
 \[
 ==     equals
 \]
 \[
 /=   does\ not\ equal
 \]
 \[
 <    is\ less\ than
 \]
 \[
 >    is\ greater\ than     
 \]

\subsection{\snippet{Literal}}
To prevent mlsl2 from interpreting \snippet{formula} as a logical use
the \snippet{/literal} flag. 
\lstinputlisting{declaration-4.l2cf}

If you combine the \snippet{/literal} and \snippet{/reevaluate} flags you can
achieve unique results, as in the next example which places the name of the
current l2cf file in the \snippet{l2cf} Runtime Flag.
\lstinputlisting{declaration-3.l2cf}

\subsection{\snippet{Reevaluate}}
You can change the value stored in a Runtime Flags using this command which
looks a lot like a declaration 
\lstinputlisting{reevaluate.l2cf}

\subsection{\snippet{Arrays}}
You can store arrays in Runtime Flags. 

\lstinputlisting{declaration-5.l2cf}

Then you can use the arrays, assign new
values to them, and even resize them.
\lstinputlisting{reevaluate-2.l2cf}

\subsection{\snippet{Skip}}
The simplest \emph{Control structure} is the \snippet{Skip} command. If given
without any fields, it results in mlsl2 skipping unconditionally to the end of
the current section.

A more useful application is to supply the \snippet{Boolean} field in which case
it skips to the end of the current sction if the \snippet{Boolean} evaluates to
\emph{true}.
\lstinputlisting{skip-1.l2cf}
Similar is the effect of the \snippet{formula} field which causes mlsl2 to skip
to the end of the current section if \snippet{formula} evaluates to
\emph{true}.

\snippet{Skip} is limited to act only within the current section unless you
include the \snippet{/NextChunk} flag which skips over all the remaining l2cf
lines and sections to the next chunk.


\subsection{\snippet{Repeat}}
The \snippet{Repeat} command loops over the remaining statements in the current
section. It can appear bare, in which case only a subsequent \snippet{Skip} command can
exit the endless loop as in this example combining the two commands
\lstinputlisting{skip-2.l2cf}

You can include the \snippet{formula} field on the same line as \snippet{Repeat}
as this example shows
\lstinputlisting{repeat-1.l2cf}
The final way \snippet{Repeat} can appear is with explicit values given in a
\snippet{values} field as the following example shows
\lstinputlisting{repeat-2.l2cf}

\subsection{\snippet{Select .. Case .. End Select}}
These span sections. Similar to Fortran's select case structure, they provide a clearer layout for a multiple branch decision.

While they may span sections, each of these special commands must appear within a section.

\snippet{Select} begins the structure. It can take one of the following example forms 

\lstinputlisting{select-1.l2cf}
or

\lstinputlisting{select-2.l2cf}
After \snippet{Select}, all the subsequent l2cf lines will be skipped until a
\snippet{Select} statement appears that matches the string in example (1) or the
value of name in example (2). 

The \snippet{Case} statement supplies a possible match for the string being
sought. Assume the string set by the \snippet{Select} statement is "faro". Then if the
\snippet{Case} statement supplies "faro", all the subsequent l2cf lines will be
executed, until either another \snippet{Case} statement or a concluding
\snippet{EndSelect} statement appears. If the \snippet{Case} statement supplies a
different string, all the subsequent l2cf lines will be skipped, until either
another \snippet{Case} statement or a concluding \snippet{EndSelect} statement
appears.

The \snippet{Case} statement can take one of two example forms

\lstinputlisting{case-1.l2cf}
or

\lstinputlisting{case-2.l2cf}
which should be clear from the explanation supplied about the two usages of 
\snippet{Select}.


There is a special value for string: if string=="default" it will match any
pattern we are currently seeking. In other words, it is a \emph{wild card}. 

%%% Local Variables: 
%%% mode: latex
%%% TeX-master: "cfm"
%%% End: 
