\documentclass[11pt]{article}

\usepackage{alltt}
\usepackage[fleqn]{amsmath}
\usepackage{floatflt}
\usepackage{graphicx}
\usepackage{longtable}
\usepackage[strings]{underscore}

\textwidth 6.5in
\oddsidemargin -0.25in
%\evensidemargin -0.5in
\topmargin -0.5in
\textheight 9in

\newcommand{\docname}{wvs-118r1}
\newcommand{\docdate}{9 April 2014}

\ifx\pdfoutput\undefined
  \pdfoutput=0
  \usepackage[hypertex,plainpages,hyperindex=true]{hyperref}
  \hypersetup{%
    hypertexnames=false%
  }
  % Specify the driver for the color package
  \ExecuteOptions{dvips}
  %\ExecuteOptions{xdvi}
\else
  \ifnum\pdfoutput>0
    \usepackage[pdftex,plainpages,hyperindex=true,pdfpagelabels]{hyperref}
    \hypersetup{%
      hypertexnames=false,%
      colorlinks=true,%
      linktocpage=true,%
    }
    % Specify the driver for the color package
    \ExecuteOptions{pdftex}
  \else
    \usepackage[hypertex,plainpages,hyperindex=true]{hyperref}
    \hypersetup{%
      hypertexnames=false%
    }
    % Specify the driver for the color package
    \ExecuteOptions{dvips}
    %\ExecuteOptions{xdvi}
  \fi
\fi

\hyperbaseurl{}
\newcommand\hr[1]{\href{#1.dvi}{dvi}, \href{#1.pdf}{pdf}}
\newcommand\h[1]{#1 (\hr{#1})}

\begin{document}

%\tracingcommands=1
\newlength{\hW} % heading box width
\newlength{\pW} % page number field width
\settowidth{\hW}{\bf\docname}
\settowidth{\pW}{Page \pageref{lastpage}\ of \pageref{lastpage}}
\ifdim \pW > \hW \setlength{\hW}{\pW} \fi
\makeatletter
\def\@biblabel#1{#1.}
\newcommand{\ps@twolines}{%
  \renewcommand{\@oddhead}{%
    \docdate\hfill\parbox[t]{\hW}{{\hfill\bf\docname}\newline
                          Page \thepage\ of \pageref{lastpage}}}%
\renewcommand{\@evenhead}{}%
\renewcommand{\@oddfoot}{}%
\renewcommand{\@evenfoot}{}%
}%
\makeatother
\pagestyle{twolines}

\renewcommand{\d}{\text{d}}
\newcommand{\T}{\mathcal{T}}

\vspace{-10pt}
\begin{tabbing}
\phantom{References: }\= \\
To: \>Nathaniel, Paul, Bill, Alyn, Mike, Vince\\
Subject: \>Grammar for MLS configuration files\\
From: \>Van Snyder\\
Reference: \>\h{wvs-004} \\
\end{tabbing}

\vspace*{-15pt}

\parindent 0pt
\parskip 6pt
\newenvironment{enum}
 {\renewcommand{\theenumi}{\arabic{enumi}}%
  \renewcommand{\labelenumi}{\theenumi}%
  \renewcommand{\theenumii}{\alph{enumii}}%
  \renewcommand{\labelenumii}{\theenumii}%
  \renewcommand{\theenumiii}{\roman{enumiii}}%
  \renewcommand{\labelenumiii}{\theenumiii}%
  \setlength{\leftmargini}{0.375in}%
  \setlength{\leftmarginii}{0.375in}%
  \setlength{\leftmarginiii}{0.375in}%
  \setlength{\leftmarginiv}{0.375in}%
  \setlength{\partopsep}{0pt}
  \setlength{\parskip}{0pt}
  \begin{enumerate}%
  \setlength{\topsep}{0pt}
  \setlength{\partopsep}{0pt}
  \setlength{\parskip}{0pt}
  \setlength{\itemsep}{2pt}
  \setlength{\itemindent}{0pt}
  \setlength{\labelsep}{0pt}
  \setlength{\listparindent}{0pt}
  \setlength{\labelwidth}{0.25in}
  \renewcommand{\makelabel}[1]{\parbox[t]{\labelwidth}{##1}\hfill}
 }%
 {\end{enumerate}%
  \vspace*{-\topsep}%                     Kludge!
 }

\h{wvs-004} describes the syntax and meaning of grammars.
\vspace*{-5pt}
\begin{tabbing}
SOG \= means ``Start Of Grammar.''  The parser skips this automatically. \\
EOG \> means ``End Of Grammar.''  This is generated by end-of-file. \\
EOS \> means ``End Of Statement.''\\
\end{tabbing}

\vspace*{-15pt}

\begin{enum}
\input l2cf.tex.h
\end{enum}

\label{lastpage}
\end{document}

% $Id$

% $Log$
% Revision 1.1  2014/02/28 22:00:18  vsnyder
% Initial commit
%
