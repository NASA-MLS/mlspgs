\documentclass[11pt]{article}
\usepackage[fleqn]{amsmath}

\textwidth 6.5in
\oddsidemargin -0.25in
%\evensidemargin -0.5in
\topmargin -0.25in
\textheight 9.0in

\newcommand{\docname}{\bf wvs-093r3}
\newcommand{\docdate}{23 November 2010}

\ifx\pdfoutput\undefined
  \pdfoutput=0
  \usepackage[hypertex,plainpages,hyperindex=true]{hyperref}
  \hypersetup{%
    hypertexnames=false%
  }
  % Specify the driver for the color package
  \ExecuteOptions{dvips}
  %\ExecuteOptions{xdvi}
\else
  \ifnum\pdfoutput>0
    \usepackage[pdftex,plainpages,hyperindex=true,pdfpagelabels]{hyperref}
    \hypersetup{%
      hypertexnames=false,%
      colorlinks=true,%
      linktocpage=true,%
    }
    % Specify the driver for the color package
    \ExecuteOptions{pdftex}
  \else
    \usepackage[hypertex,plainpages,hyperindex=true]{hyperref}
    \hypersetup{%
      hypertexnames=false%
    }
    % Specify the driver for the color package
    \ExecuteOptions{dvips}
    %\ExecuteOptions{xdvi}
  \fi
\fi

\hyperbaseurl{}
\ifx\dvidir\undefined
  \newcommand\hr[1]{\href{#1.dvi}{dvi} \href{#1.pdf}{pdf}}
\else
  \newcommand\hr[1]{\href{\dvidir/#1.dvi}{dvi} \href{\pdfdir/#1.pdf}{pdf}}
\fi
\newcommand\h[1]{#1 \hr{#1}}

\begin{document}

%\tracingcommands=1
\newlength{\hW} % heading box width
\newlength{\pW} % page number field width
\settowidth{\hW}{\docname}
\settowidth{\pW}{Page \pageref{lastpage}\ of \pageref{lastpage}}
\ifdim \pW > \hW \setlength{\hW}{\pW} \fi
\makeatletter
\def\@biblabel#1{#1.}
\newcommand{\ps@twolines}{%
  \renewcommand{\@oddhead}{%
    \docdate\hfill\parbox[u]{\hW}{{\hfill\docname}\newline
                          Page \thepage\ of \pageref{lastpage}}}%
\renewcommand{\@evenhead}{}%
\renewcommand{\@oddfoot}{}%
\renewcommand{\@evenfoot}{}%
}%
\makeatother
\pagestyle{twolines}

\renewcommand{\d}{\text{d}}

\vspace{-10pt}
\begin{tabbing}
\phantom{References: }\= \\
To: \>Bill, Nathaniel, Igor\\
Subject: \>Computing derivatives in the MLS Full Forward Model\\
From: \>Van Snyder\\
Reference: \> \h{iy-007}\\
\end{tabbing}

\parindent 0pt \parskip 5pt
\vspace{-20pt}

\section{Introduction}

This note shows that the MLS Full Forward Model could compute the Jacobian
matrix necessary for retrieval more efficiently than it presently does.

Thanks are due to the seminar \emph{Sensitivity Analysis of Mathematical Models}
on 29 March -- 1 April by Eugene Ustinov.

\section{Mathematical generalities}

Consider a function $F(x_1, \dots, x_n)$.  From the first term of its Taylor
series, its variation due to variation $\delta x_j$ of its arguments is $\delta
F = \sum_j a_j \delta x_j$.  From this it is obvious that the sensitivities of
$F$ due to variations in its arguments are the partial derivatives

\begin{equation}
\frac{\partial F}{\partial x_j} = a_j\,.
\end{equation}

If we carry this over to a functional $F[X(\xi)]$ of a continuum $X(\xi)$
instead of a function of the finite set $\{x_1,\dots,x_n\}$, \emph{viz.}

\begin{equation}
F[X(\xi)] = \int_{\xi_0}^{\xi_1} a(\xi) X(\xi) \,\d \xi
\end{equation}

then its variation due to variation $\delta X(\xi)$ of its argument function
(also known as ``continuous parameter'') $X(\xi)$ is

\begin{equation}\label{VAR}
\delta F[X(\xi)] = \int_{\xi_0}^{\xi_1} a(\xi) \,\delta X(\xi) \,\d \xi
\end{equation}

and its sensitivity due to variation $\delta X(\xi)$ of its argument function
$X(\xi)$ is

\begin{equation}\label{SENS}
\frac{\delta F[X(\xi)]}{\delta X(\xi)} = a(\xi)
\end{equation}

(replace the integral in Equation (\ref{VAR}) by a quadrature, or see
pages 27-29 of {\bf Calculus of Variations} by I. M. Gelfand and S. V.
Fomin).

%=========================================================================

\section{Radiance}

The radiance is the solution of the non-scattering radiative transfer equation

\begin{equation}\label{diff eq}
\frac{\partial I(s)}{\partial s} + \alpha(s) I(s) = \alpha(s) B(s)\,.
\end{equation}

Using notation similar to iy-007 and following the arguments therein, the
solution is usually written as

\begin{equation}\label{first}
I[\alpha(s),s_m] = I(s_0) \mathcal{T}(s_0) +
 \int_{s_0}^{s_m} \mathcal{T}(s) \alpha(s) B(s) \,
  \d s\,,
\end{equation}

where $s_m$ is the location of the instrument, $s_0$ is the location of
the end of the path away from the instrument, $B(s)$ is the Planck
function $\frac{h \nu}k \left( \exp\left(\frac{h \nu}{k T(s)}\right)
-1\right)^{-1}$,

\begin{equation}\label{two}
\mathcal{T}(s) = \exp\left( - \int_s^{s_m} \alpha(\sigma)
 \, \d \sigma \right) \,,
\end{equation}

$\alpha(\sigma) = \sum_k \beta^k(s) f^k(s) = \sum_k \beta^k(s) \sum_l
f^k_{lm} \mu_{lm}(s)$ is the total absorption cross section for all
mole\-cules, $f^k(\sigma)$ is the volume mixing ratio of the
$k^\text{th}$ species at $\sigma$, $f^k_{lm}$ is the volume mixing ratio
of the $k^\text{th}$ species at the $(l,m)$ representation basis point,
$\mu_{lm}(s) = (\xi_l(s) + \xi_{l-1}(s))(\eta_m(s) + \eta_{m-1}(s))$,
$\xi_l(s) = (\phi_{l+1}-\phi(s))/(\phi_{l+1}-\phi_l)$ for $\phi_l \leq
\phi(s) \leq \phi_{l+1}$ and zero otherwise and $\eta_m(s) =
(\phi_{m+1}-\phi(s))/(\phi_{m+1}-\phi_m)$ for $\phi_m \leq \phi(s) \leq
\phi_{m+1}$ and zero otherwise are interpolation coefficients from the
representation basis to the integration path, and $\beta^k(\sigma)$ is
the absorption cross section for the $k^\text{th}$ species at $\sigma$. 
From Equation (\ref{two})

\begin{equation}
\frac{\partial \mathcal{T}(s)}{\partial s} = \alpha(s) \mathcal{T}(s).
\end{equation}

Substituting this into Equation (\ref{first}) gives

\begin{equation}\label{RAD}
I[\alpha(s),s_m] = \mathcal{T}(s_0) I(s_0) +
 \int_{s_0}^{s_m} B(s) \frac{\partial \mathcal{T}(s)}{\partial s}
  \, \d s \,.
\end{equation}

In the MLS full forward model, Equation (\ref{RAD}) is
integrated by parts to give

\begin{equation}\begin{split}\label{I}
I[\alpha(s),s_m] =\,& \mathcal{T}(s_0) (I(s_0)-B(s_0)) + B(s_m) -
 \int_{s_0}^{s_m} \mathcal{T}(s) \, \frac{\partial B(s)}{\partial s} \,
  \d s \\
=\,& \mathcal{T}(s_0) (I(s_0)-B(s_0)) + B(s_m) -
 \int_{B(s_0)}^{B(s_m)} \mathcal{T}(s) \, \d B(s)
\end{split}\end{equation}

(remember $\mathcal{T}(s_m)=1$). Henceforth, $I[\alpha(\sigma),s_m]$ will
be abbreviated to $I$.

%=========================================================================

\section{Sensitivity of radiance to variation of $\alpha(\sigma)$}

Since $B(\sigma)$ doesn't depend upon $\alpha(\sigma)$, the variation of
$I$ as a result of variation of $\alpha(\sigma)$ is

\begin{equation}\label{I var}
\delta_\alpha I =
 \left( I(s_0)-B(s_0) \right) \delta_\alpha \mathcal{T}(s_0) -
  \int_{s_0}^{s_m} \delta_\alpha \mathcal{T}(\sigma) \,
   \frac{\partial B(\sigma)}{\partial\sigma} \, \d \sigma\,.
\end{equation}

To compute $\delta_\alpha \mathcal{T}(\sigma)$, replace $\alpha(\sigma)$
in Equation (\ref{two}) by $\alpha(\sigma^\prime) +
\delta\alpha(\sigma^\prime)$ and compute the difference from Equation
(\ref{two}):

\begin{equation}\begin{split}\label{tau vary}
\delta_\alpha \mathcal{T}(\sigma)
=\,& \exp\left( - \int_\sigma^{s_m} (\alpha(\sigma^\prime) +
                                       \delta\alpha(\sigma^\prime) )
         \, \d \sigma^\prime \right) -
     \exp\left( - \int_\sigma^{s_m} \alpha(\sigma^\prime)
         \, \d \sigma^\prime \right) \\
=\,& \exp\left( - \int_\sigma^{s_m} \alpha(\sigma^\prime)
         \, \d \sigma^\prime \right)
     \exp\left( - \int_\sigma^{s_m} \delta\alpha(\sigma^\prime)
         \, \d \sigma^\prime \right) -
     \exp\left( - \int_\sigma^{s_m} \alpha(\sigma^\prime)
         \, \d \sigma^\prime \right) \\
=\,& \exp\left( - \int_\sigma^{s_m} \alpha(\sigma^\prime)
         \, \d \sigma^\prime \right)
     \left(
       \exp\left( - \int_\sigma^{s_m} \delta\alpha(\sigma^\prime)
         \, \d \sigma^\prime \right) -1
     \right) \\
=\,& \exp\left( - \int_\sigma^{s_m} \alpha(\sigma^\prime)
         \, \d \sigma^\prime \right)
     \left(1 - \int_\sigma^{s_m} \delta\alpha(\sigma^\prime) +
           O\left( \left( \int_\sigma^{s_m} \delta\alpha(\sigma^\prime)
                           \, \d \sigma^\prime
            \right)^2 \right) - 1 \right)\,.
\end{split}\end{equation}

Therefore, in the limit for small perturbation $\delta
\alpha(\sigma^\prime)$

\begin{equation}\label{tau var}
\delta_\alpha \mathcal{T}(\sigma) =
 \exp \left( - \int_\sigma^{s_m} \alpha(\sigma^\prime) \,\d \sigma^\prime \right)
 \cdot
 \left( - \int_\sigma^{s_m} \delta \alpha(\sigma^\prime) \,\d \sigma^\prime \right)
=
 -\mathcal{T}(\sigma) \int_\sigma^{s_m} \delta \alpha(\sigma^\prime) \,\d \sigma^\prime\,.
\end{equation}

Substituting Equation (\ref{tau var}) into the integral in Equation
(\ref{I var}) gives (exchanging the order of integration in the last step)

\begin{equation}\begin{split}\label{var1}
- \int_{s_0}^{s_m} \delta_\alpha \mathcal{T}(\sigma) \,
 \frac{\partial B(\sigma)}{\partial\sigma} \, \d \sigma
=\,&
 \int_{s_0}^{s_m} \d \sigma \,\mathcal{T}(\sigma)\,
    \frac{\partial B(\sigma)}{\partial\sigma} 
    \int_\sigma^{s_m} \d \sigma^\prime\, \delta
    \alpha(\sigma^\prime) \\
=\,&
 \int_{s_0}^{s_m} \d \sigma\, \delta \alpha(\sigma)
   \int_{s_0}^\sigma \d \sigma^\prime \, \mathcal{T}(\sigma^\prime) \,
    \frac{\partial B(\sigma^\prime)}{\partial\sigma^\prime}\,.
\end{split}\end{equation}

Substituting Equation (\ref{tau var}) into the first term for $\delta_\alpha
I$ in Equation (\ref{I var}) gives

\begin{equation}\label{var2}
\left( I(s_0)-B(s_0) \right) \delta_\alpha \mathcal{T}(s_0)
= -\mathcal{T}(s_0) \left( I(s_0)-B(s_0) \right)
   \int_{s_0}^{s_m} \delta \alpha(\sigma)\, \d \sigma\,.
\end{equation}

Collecting Equations (\ref{var1}) and (\ref{var2}) and factoring, the
resulting variation in $I$ due to variation in $\alpha(\sigma)$ is

\begin{equation}
\delta_\alpha I = \int_{s_0}^{s_m} \d \sigma \,\delta \alpha(\sigma)
 \left\{ \int_{s_0}^\sigma \mathcal{T}(\sigma^\prime) \,
    \frac{\partial B(\sigma^\prime)}{\partial\sigma^\prime} \,
     \d \sigma^\prime
 - \mathcal{T}(s_0) \left( I(s_0)-B(s_0) \right) \right\}\,.
\end{equation}

According to Equations (\ref{VAR}) and (\ref{SENS}), the sensitivity of $I$ to
variation $\delta \alpha(\sigma)$ in $\alpha(\sigma)$ is the factor between \{
and \}, \emph{viz.}

\begin{equation}\label{I var alpha}
\frac{\delta I}{\delta \alpha(\sigma)} =
 \int_{s_0}^\sigma \mathcal{T}(\sigma^\prime) \,
    \frac{\partial B(\sigma^\prime)}{\partial\sigma^\prime} \,
     \d \sigma^\prime
 - \mathcal{T}(s_0) \left( I(s_0)-B(s_0) \right)\,.
\end{equation}

From this and $\frac{\partial \alpha(\sigma)}{\partial f^k_{lm}}
= \beta^k(\sigma) \mu_{lm}(\sigma)$, we can compute the sensitivity of
$I$ to variation of $f^k_{lm}$ as

\begin{equation}
\frac{\delta I}{\delta f^k_{lm}} =
 \frac{\delta I}{\delta \alpha(\sigma)}
  \frac{\partial \alpha(\sigma)}{\partial f^k_{lm}} =
 \frac{\delta I}{\delta \alpha(\sigma)} \beta^k(\sigma) \mu_{lm}(\sigma) =
 \left( B(\sigma) - I(\sigma) \right) \beta^k(\sigma) \mu_{lm}(\sigma) \,.
\end{equation}

The last result follows because the integrands in Equation (\ref{I var
alpha}) and Equation (\ref{I}) are the same.

If we evaluate the indefinite sum

\begin{equation}\label{sum}
S_j = \sum_{i=1}^j
 \frac12\left(\mathcal{T}(\sigma_i)+\mathcal{T}(\sigma_{i-1})\right)
  \Delta B(\sigma_i) -
   \mathcal{T}(s_0)\left(I(s_0)-B(s_0)\right)
\approx \frac{\delta I}{\delta \alpha(\sigma_j)}
\end{equation}

with $\sigma_0 = s_0$ and $s_0 < \sigma_1 < \sigma_2 \dots < \sigma_{N_p}
\leq s_m$, then from Equation (\ref{I}) we have

\begin{equation}\begin{split}\label{var sum}
I(s_m) =\,& 
 B(s_m) - \frac{\delta I}{\delta \alpha(s_m)} 
 \approx B(s_m) - S_{N_p} \text{ and}\\
\frac{\delta I(s_m)}{\delta f^k_{lm}}
 \approx\,& \beta^k(\sigma_j) \mu_{lm}(\sigma_j) S_j \\
\end{split}\end{equation}

That is, we can evaluate both $I$ and its sensitivity to variation in the
discrete parameters $f^k_{lm}$ by evaluating one indefinite sum, and
multiplying each of the $N_p$ elements by $N_s$ molecular absorption
cross sections and $N_r$ interpolation coefficients, where $N_r$ is the
number of points in the representation basis.  The additional cost to
compute the sensitivities is $N_p \times N_s \times N_r$ additional
multiplies.

In the present scheme, we compute the sensitivities by evaluating $N_p
\times N_s \times N_r$ additional quadrature steps, each requiring $2 N_p
+ 11 G + N_s N_p^2$ FLOPS, where $0 \leq G \leq N_p$ is the number of
panels requiring Gauss-Legendre quadrature, because we compute the
derivative of Equation (\ref{first}) with respect to $f^k(\sigma)$, and
leave $\frac{\partial\alpha(\sigma)}{\partial f^k(\sigma)} =
\beta^k(\sigma)$ inside the integral:

\begin{equation}
\frac{\partial I(s_m)}{\partial f^k(\sigma)} =
 \int_{s_0}^\sigma \frac{\partial \mathcal{T}(s)}{\partial f^k(s)}
  \frac{\partial B(s)}{\partial s} \, \d s
\end{equation}

Thus the approach described here should be less time consuming by a
factor of $O(N_s \times N_p^2)$, and require less memory than the current
method.

Evaluating the Hessian tensor should similarly be possible without integrating
variational equations.

%=========================================================================

\section{Sensitivity of radiance to variation of temperature}

Because both $\mathcal{T}$ and $B$ depend upon temperature

\begin{equation}\begin{split}\label{T vary}
\delta_T I =\,& -\delta_T B(s_0)) \mathcal{T}(s_0)
 + (I(s_0) - B(s_0)) \delta_T \mathcal{T}(s_0) + \delta_T B(s_m) \\
& - \int_{s_0}^{s_m} \delta_T
  \mathcal{T}(\sigma) B^\prime(\sigma) \, \d \sigma
 - \int_{s_0}^{s_m}
  \mathcal{T}(\sigma) \delta_T B^\prime(\sigma)
  \, \d \sigma\,,
\end{split}\end{equation}

where $B^\prime(\sigma) = \frac{\partial B(\sigma)}{\partial \sigma}$.
Following a development similar to Equations (\ref{tau vary}--\ref{tau
var}),

\begin{equation}
\int_{s_0}^{s_m} \delta_T
  \mathcal{T}(\sigma) B^\prime(\sigma) \d \sigma = 
  \int_{s_0}^{s_m} \d \sigma\,
   \delta T(\sigma) \frac{\partial \alpha(\sigma)}{\partial T(\sigma)}
   \int_{s_0}^\sigma \d \sigma^\prime\,
    \mathcal{T}(\sigma^\prime) B^\prime(\sigma^\prime)
\end{equation}

while

\begin{equation}
\int_{s_0}^{s_m} \mathcal{T}(\sigma) \delta_T B(\sigma)\, \d \sigma =
 \int_{s_0}^{s_m} \d \sigma\, \delta T(\sigma) \mathcal{T}(\sigma)
  \frac{\partial B^\prime(\sigma)}{\partial T(\sigma)}\,.
\end{equation}

Therefore, again using Equations (\ref{VAR}--\ref{SENS}),

\begin{equation}\begin{split}\label{dIdT}
\frac{\delta I(s_m)}{\delta T(\sigma)} =\,&
 -\frac{\delta B(s_0)}{\delta T(\sigma)} \mathcal{T}(s_0)
  + (I(s_0) - B(s_0)) \frac{\delta \mathcal{T}(s_0)}{\delta T(\sigma)}
  + \frac{\delta B(s_m)}{\delta T(\sigma)}\\
&
  - \frac{\partial \alpha(\sigma)}{\partial T(\sigma)}
   \int_{s_0}^\sigma \d \sigma^\prime\,
    \mathcal{T}(\sigma^\prime) B^\prime(\sigma^\prime)
  - \mathcal{T}(\sigma) \frac{\partial B^\prime(\sigma)}{\partial T(\sigma)} \\
=\,&
 (I(s_0) - B(s_0)) \frac{\delta \mathcal{T}(s_0)}{\delta \alpha(\sigma)}
  \frac{\partial \alpha(\sigma)}{\partial T(\sigma)}
  - \frac{\partial \alpha(\sigma)}{\partial T(\sigma)}
   \int_{s_0}^\sigma \d \sigma^\prime\,
    \mathcal{T}(\sigma^\prime) B^\prime(\sigma^\prime)
  - \mathcal{T}(\sigma) \frac{\partial B^\prime(\sigma)}{\partial T(\sigma)} \\
=\,&
  -(I(s_0) - B(s_0)) \mathcal{T}(s_0)
  \frac{\partial \alpha(\sigma)}{\partial T(\sigma)}
  - \frac{\partial \alpha(\sigma)}{\partial T(\sigma)}
   \int_{s_0}^\sigma \d \sigma^\prime\,
    \mathcal{T}(\sigma^\prime) B^\prime(\sigma^\prime)
  - \mathcal{T}(\sigma) \frac{\partial B^\prime(\sigma)}{\partial T(\sigma)} \\
=\,&
  \frac{\delta I}{\delta \alpha(\sigma)}
  \frac{\partial \alpha(\sigma)}{\partial T(\sigma)}
  - \mathcal{T}(\sigma) \frac{\partial B^\prime(\sigma)}{\partial T(\sigma)},
\end{split}\end{equation}

where Equation (\ref{I var alpha}) has been used in the last step to
represent the integral in Equation (\ref{dIdT}).

To integrate Equation (\ref{T vary}) in $\zeta$ instead of $\sigma$
coordinates, replace $B^\prime(\sigma) = \frac{\partial B(\sigma)}{\partial
\sigma} \frac{\d \sigma}{\d h} \frac{\partial h}{\partial \zeta}$, from which
Equation (\ref{dIdT}) becomes

\begin{equation}
\frac{\delta I(s_m)}{\delta T(\sigma)} =
  \frac{\delta I}{\delta \alpha(\sigma)}
  \frac{\partial \alpha(\sigma)}{\partial T(\sigma)}
- \mathcal{T}(\sigma) \frac{\d \sigma}{\d h}
   \left( \frac{\partial^2 B(\sigma)}{\partial \sigma \partial T(\sigma)}
    \frac{\partial h}{\partial \zeta} +
   \frac{\partial B(\sigma)}{\partial \sigma}
    \frac{\partial^2 h}{\partial \zeta \partial T(\sigma)}
   \right)
   \,.
\end{equation}

%Thus we can compute both $\frac{\partial I}{\partial f_i^k}$
%and $\frac{\partial I}{\partial T_i}$ by evaluating Equation (\ref{var
%sum}) only once, as an indefinite sum for $j = 1, \dots, N_p$.

\label{lastpage}
\end{document}

% $Id$

% $Log$
% Revision 1.3  2010/11/23 21:48:42  vsnyder
% Incorporate interpolating factors, re-work temperature derivatives
%
% Revision 1.2  2010/05/20 23:47:56  vsnyder
% Add Id line
%
% Revision 1.1  2010/04/09 02:23:24  vsnyder
% Initial commit -- yes, at r1, not r0
%
