\documentclass[11pt]{article}
\usepackage[fleqn]{amsmath}

\textwidth 6.5in
\oddsidemargin -0.25in
%\evensidemargin -0.5in
\topmargin -0.25in
\textheight 9.0in

\newcommand{\docname}{\bf wvs-093r5}
\newcommand{\docdate}{2 December 2010}

\ifx\pdfoutput\undefined
  \pdfoutput=0
  \usepackage[hypertex,plainpages,hyperindex=true]{hyperref}
  \hypersetup{%
    hypertexnames=false%
  }
  % Specify the driver for the color package
  \ExecuteOptions{dvips}
  %\ExecuteOptions{xdvi}
\else
  \ifnum\pdfoutput>0
    \usepackage[pdftex,plainpages,hyperindex=true,pdfpagelabels]{hyperref}
    \hypersetup{%
      hypertexnames=false,%
      colorlinks=true,%
      linktocpage=true,%
    }
    % Specify the driver for the color package
    \ExecuteOptions{pdftex}
  \else
    \usepackage[hypertex,plainpages,hyperindex=true]{hyperref}
    \hypersetup{%
      hypertexnames=false%
    }
    % Specify the driver for the color package
    \ExecuteOptions{dvips}
    %\ExecuteOptions{xdvi}
  \fi
\fi

\hyperbaseurl{}
\ifx\dvidir\undefined
  \newcommand\hr[1]{\href{#1.dvi}{dvi} \href{#1.pdf}{pdf}}
\else
  \newcommand\hr[1]{\href{\dvidir/#1.dvi}{dvi} \href{\pdfdir/#1.pdf}{pdf}}
\fi
\newcommand\h[1]{#1 \hr{#1}}

\begin{document}

%\tracingcommands=1
\newlength{\hW} % heading box width
\newlength{\pW} % page number field width
\settowidth{\hW}{\docname}
\settowidth{\pW}{Page \pageref{lastpage}\ of \pageref{lastpage}}
\ifdim \pW > \hW \setlength{\hW}{\pW} \fi
\makeatletter
\def\@biblabel#1{#1.}
\newcommand{\ps@twolines}{%
  \renewcommand{\@oddhead}{%
    \docdate\hfill\parbox[u]{\hW}{{\hfill\docname}\newline
                          Page \thepage\ of \pageref{lastpage}}}%
\renewcommand{\@evenhead}{}%
\renewcommand{\@oddfoot}{}%
\renewcommand{\@evenfoot}{}%
}%
\makeatother
\pagestyle{twolines}

\renewcommand{\d}{\text{d}}
\newcommand{\T}{\mathcal{T}}

\vspace{-10pt}
\begin{tabbing}
\phantom{References: }\= \\
To: \>Bill, Nathaniel, Igor\\
Subject: \>Computing derivatives in the MLS Full Forward Model\\
From: \>Van Snyder\\
Reference: \> \h{iy-007}\\
\end{tabbing}

\parindent 0pt \parskip 5pt
\vspace{-20pt}

\section{Introduction}

This note shows that the MLS Full Forward Model could compute the Jacobian
matrix necessary for retrieval more efficiently than it presently does.

Thanks are due to the seminar \emph{Sensitivity Analysis of Mathematical Models}
on 29 March -- 1 April by Eugene Ustinov.

\section{Mathematical generalities}

Consider a function $F(x_1, \dots, x_n)$ with Maclaurin series $c +
\sum_j a_j x_j + \sum_{jk} b_{jk} x_j x_k \dots$.  Its variation due to
variation $\delta x_j$ of its arguments is $\delta F = \sum_j a_j \delta
x_j + O(\delta x_j^2)$.  From this, in the limit $\delta x_j \rightarrow
0$, it is obvious that the sensitivities of $F$ due to variations in its
arguments are the partial derivatives

\begin{equation}
\frac{\partial F}{\partial x_j} = a_j\,.
\end{equation}

If we carry this over to a functional $F[X(\xi)]$ of a continuum $X(\xi)$
instead of a function of the finite set $\{x_1,\dots,x_n\}$, replacing
the summations in the Maclaurin series with integrals,
\emph{viz.}

\begin{equation}
F[X(\xi)] = c + \int_{\xi_0}^{\xi_1} a(\xi) X(\xi) \,\d \xi +
 \int_{\xi_0}^{\xi_1} \int_{\xi_0}^{\xi_1} b(\xi,\xi^\prime) X(\xi)
 X(\xi^\prime) \d \xi \d \xi^\prime + \dots
\end{equation}

then its variation due to variation $\delta X(\xi)$ of its argument
function (also known as ``continuous parameter'') $X(\xi)$, for small
variation $\delta X(\xi)$, is

\begin{equation}\label{VAR}
\delta F[X(\xi)] = \int_{\xi_0}^{\xi_1} a(\xi) \,\delta X(\xi) \,\d \xi
  + O(\delta X(\xi)^2)
 = \lim_{\Delta \xi_i \rightarrow 0}
  \sum_i a(\xi_i) \delta X(\xi_i) \Delta \xi_i + O(\delta X(\xi)^2)
\end{equation}

and its sensitivity due to variation $\delta X(\xi)$ of its argument
function $X(\xi)$ is

\begin{equation}\label{SENS}
\frac{\delta F[X(\xi)]}{\delta X(\xi)}
 \equiv \lim_{\Delta \xi \rightarrow 0 \atop \delta X(\xi) \rightarrow 0}
  \frac{\partial F[X(\xi)]}{\Delta \xi \partial X(\xi)}
 = \frac{a(\xi) \delta X(\xi) \Delta \xi}{\delta X(\xi) \Delta \xi}
 = a(\xi)
\end{equation}

(also see pages 27-29 of {\bf Calculus of Variations} by I. M. Gelfand
and S. V. Fomin).

%=========================================================================

\section{Radiance}

The radiance is the solution of the non-scattering radiative transfer
equation

\begin{equation}\label{diff eq}
\frac{\partial I(s)}{\partial s} + \alpha(s) I(s) = \alpha(s) B(s)\,.
\end{equation}

Using notation similar to iy-007 and following the arguments therein, the
solution is usually written as

\begin{equation}\label{first}
I[\alpha(s),s_m] = I(s_0) \T(s_0,s_m) +
 \int_{s_0}^{s_m} \T(s,s_m) \alpha(s) B(s) \,
  \d s\,,
\end{equation}

where $s_m$ is the location of the instrument, $s_0$ is the location of
the end of the path away from the instrument, $B(s)$ is the Planck
function $\frac{h \nu}k \left( \exp\left(\frac{h \nu}{k T(s)}\right)
-1\right)^{-1}$,

\begin{equation}\label{two}
\T(s,s_m) = \exp\left( - \int_s^{s_m} \alpha(\sigma)
 \, \d \sigma \right) \,,
\end{equation}

$\alpha(\sigma) = \sum_k \beta^k(s) f^k(s) = \sum_k \beta^k(s) \sum_l
f^k_{lm} \mu_{lm}(s)$ is the total absorption cross section for all
mole\-cules, $f^k(\sigma)$ is the volume mixing ratio of the
$k^\text{th}$ species at $\sigma$, $f^k_{lm}$ is the volume mixing ratio
of the $k^\text{th}$ species at the $(l,m)$ representation basis point,
$\mu_{lm}(s) = (\xi_l(s) + \xi_{l-1}(s))(\eta_m(s) + \eta_{m-1}(s))$,
$\xi_l(s) = (\phi_{l+1}-\phi(s))/(\phi_{l+1}-\phi_l)$ for $\phi_l \leq
\phi(s) \leq \phi_{l+1}$ and zero otherwise and $\eta_m(s) =
(\zeta_{m+1}-\zeta(s))/(\zeta_{m+1}-\zeta_m)$ for $\zeta_m \leq \zeta(s)
\leq \zeta_{m+1}$ and zero otherwise are interpolation coefficients from
the representation basis to the integration path, and $\beta^k(\sigma)$
is the absorption cross section for the $k^\text{th}$ species at
$\sigma$.  From Equation (\ref{two})

\begin{equation}
\frac{\partial \T(s,s_m)}{\partial s} = \alpha(s) \T(s,s_m).
\end{equation}

Substituting this into Equation (\ref{first}) gives

\begin{equation}\label{RAD}
I[\alpha(s),s_m] = \T(s_0,s_m) I(s_0) +
 \int_{s_0}^{s_m} B(s) \frac{\partial \T(s,s_m)}{\partial s}
  \, \d s \,.
\end{equation}

In the MLS full forward model, Equation (\ref{RAD}) is
integrated by parts to give

\begin{equation}\begin{split}\label{I}
I[\alpha(s),s_m] =\,& \T(s_0,s_m) (I(s_0)-B(s_0)) + B(s_m) -
 \int_{s_0}^{s_m} \T(s,s_m) \, B^\prime(s) \,
  \d s \\
=\,& \T(s_0,s_m) (I(s_0)-B(s_0)) + B(s_m) -
 \int_{B(s_0)}^{B(s_m)} \T(s,s_m) \, \d B(s)\,,
\end{split}\end{equation}

where $B^\prime(s) = \frac{\partial B(s)}{\partial s}$ (remember
$\T(s_m,s_m)=1$). Henceforth, $I[\alpha(\sigma),s_m]$ will be abbreviated
to $I$.

%=========================================================================

\section{Sensitivity of radiance to variation of $\alpha(\sigma)$}

Since $B(\sigma)$ doesn't depend upon $\alpha(\sigma)$, the variation of
$I$ as a result of variation of $\alpha(\sigma)$ is

\begin{equation}\label{I var}
\delta_\alpha I =
 \left( I(s_0)-B(s_0) \right) \delta_\alpha \T(s_0,s_m) -
  \int_{s_0}^{s_m} \delta_\alpha \T(\sigma,s_m) \,
   B^\prime(\sigma) \, \d \sigma\,.
\end{equation}

To compute $\delta_\alpha \T(\sigma,s_m)$, replace $\alpha(\sigma)$
in Equation (\ref{two}) by $\alpha(\sigma^\prime) +
\delta\alpha(\sigma^\prime)$ and compute the difference from Equation
(\ref{two}):

\begin{equation}\begin{split}\label{tau vary}
\delta_\alpha \T(\sigma,s_m)
=\,& \exp\left( - \int_\sigma^{s_m} (\alpha(\sigma^\prime) +
                                       \delta\alpha(\sigma^\prime) )
         \, \d \sigma^\prime \right) -
     \exp\left( - \int_\sigma^{s_m} \alpha(\sigma^\prime)
         \, \d \sigma^\prime \right) \\
=\,& \exp\left( - \int_\sigma^{s_m} \alpha(\sigma^\prime)
         \, \d \sigma^\prime \right)
     \exp\left( - \int_\sigma^{s_m} \delta\alpha(\sigma^\prime)
         \, \d \sigma^\prime \right) -
     \exp\left( - \int_\sigma^{s_m} \alpha(\sigma^\prime)
         \, \d \sigma^\prime \right) \\
=\,& \exp\left( - \int_\sigma^{s_m} \alpha(\sigma^\prime)
         \, \d \sigma^\prime \right)
     \left(
       \exp\left( - \int_\sigma^{s_m} \delta\alpha(\sigma^\prime)
         \, \d \sigma^\prime \right) -1
     \right) \\
=\,& \exp\left( - \int_\sigma^{s_m} \alpha(\sigma^\prime)
         \, \d \sigma^\prime \right)
     \left(1 - \int_\sigma^{s_m} \delta\alpha(\sigma^\prime)
                \, \d \sigma^\prime +
           O\left( \left( \int_\sigma^{s_m} \delta\alpha(\sigma^\prime)
                           \, \d \sigma^\prime
            \right)^2 \right) - 1 \right)\,.
\end{split}\end{equation}

Therefore, in the limit for small perturbation $\delta
\alpha(\sigma^\prime)$

\begin{equation}\label{tau var}
\delta_\alpha \T(\sigma,s_m) =
 \exp \left( - \int_\sigma^{s_m} \alpha(\sigma^\prime) \,\d \sigma^\prime \right)
 \cdot
 \left( - \int_\sigma^{s_m} \delta \alpha(\sigma^\prime) \,\d \sigma^\prime \right)
=
 -\T(\sigma,s_m) \int_\sigma^{s_m} \delta \alpha(\sigma^\prime) \,\d \sigma^\prime\,.
\end{equation}

Substituting Equation (\ref{tau var}) into the integral in Equation
(\ref{I var}) gives (exchanging the order of integration in the last step)

\begin{equation}\begin{split}\label{var1}
- \int_{s_0}^{s_m} \delta_\alpha \T(\sigma,s_m) \,
 B^\prime(\sigma) \, \d \sigma
=\,&
 \int_{s_0}^{s_m} \d \sigma \,\T(\sigma,s_m)\,
    B^\prime(\sigma) 
    \int_\sigma^{s_m} \d \sigma^\prime\, \delta
    \alpha(\sigma^\prime) \\
=\,&
 \int_{s_0}^{s_m} \d \sigma\, \delta \alpha(\sigma)
   \int_{s_0}^\sigma \d \sigma^\prime \, \T(\sigma^\prime,s_m) \,
    B^\prime(\sigma^\prime)\,.
\end{split}\end{equation}

Substituting Equation (\ref{tau var}) into the first term for $\delta_\alpha
I$ in Equation (\ref{I var}) gives

\begin{equation}\label{var2}
\left( I(s_0)-B(s_0) \right) \delta_\alpha \T(s_0,s_m)
= -\T(s_0,s_m) \left( I(s_0)-B(s_0) \right)
   \int_{s_0}^{s_m} \delta \alpha(\sigma)\, \d \sigma\,.
\end{equation}

Collecting Equations (\ref{var1}) and (\ref{var2}) and factoring, the
resulting variation in $I$ due to variation in $\alpha(\sigma)$ is

\begin{equation}\begin{split}
\delta_\alpha I
=\,& \int_{s_0}^{s_m} \d \sigma \,\delta \alpha(\sigma)
 \left\{ \int_{s_0}^\sigma \T(\sigma^\prime,s_m) \,
    B^\prime(\sigma^\prime) \,
     \d \sigma^\prime
 - \T(s_0,s_m) \left( I(s_0)-B(s_0) \right) \right\} \\
=\,& \int_{s_0}^{s_m} \d \sigma \,\delta \alpha(\sigma)
 \left\{ \T(\sigma,s_m) \left[ \int_{s_0}^\sigma \T(\sigma^\prime,\sigma) \,
    B^\prime(\sigma^\prime) \,
     \d \sigma^\prime
 - \T(s_0,\sigma) \left( I(s_0)-B(s_0) \right) \right] \right\}\,.
\end{split}\end{equation}

According to Equations (\ref{VAR}) and (\ref{SENS}), the sensitivity of
$I$ to variation $\delta \alpha(\sigma)$ in $\alpha(\sigma)$ is the
factor between \{ and \}, \emph{viz.}


\begin{equation}\label{I var alpha}
\frac{\delta I}{\delta \alpha(\sigma)}
=
 \T(\sigma,s_m) \left[ \int_{s_0}^\sigma \T(\sigma^\prime,s_m) \,
    B^\prime(\sigma^\prime) \,
     \d \sigma^\prime
 - \T(s_0,\sigma) \left( I(s_0)-B(s_0) \right) \right] \,.
\end{equation}

Using Equation (\ref{I}) in Equation (\ref{I var alpha}) we have

\begin{equation}\label{var I}\boxed{
\frac{\delta I}{\delta \alpha(\sigma)} =
 \T(\sigma,s_m) \left( B(\sigma) - I(\sigma) \right)}\,.
\end{equation}

From this and $\frac{\partial \alpha(\sigma)}{\partial f^k_{lm}} =
\beta^k(\sigma) \mu_{lm}(\sigma)$, we can compute the sensitivity of $I$
to variation of $f^k_{lm}$ as

\begin{equation}
\frac{\delta I}{\delta f^k_{lm}} =
 \frac{\delta I}{\delta \alpha(\sigma)}
  \frac{\partial \alpha(\sigma)}{\partial f^k_{lm}} =
 \frac{\delta I}{\delta \alpha(\sigma)} \beta^k(\sigma) \mu_{lm}(\sigma) =
 \T(\sigma,s_m)
  \left( B(\sigma) - I(\sigma) \right) \beta^k(\sigma) \mu_{lm}(\sigma) \,.
\end{equation}

If we evaluate the indefinite sum

\begin{equation}\label{sum}
S_j = \sum_{i=1}^j
 \frac12\left(\T(\sigma_i,s_m)+\T(\sigma_{i-1},s_m)\right)
  \Delta B(\sigma_i) -
   \T(s_0,s_m)\left(I(s_0)-B(s_0)\right)
\approx \frac1{\T(\sigma,s_m)} \frac{\delta I}{\delta \alpha(\sigma_j)}
\end{equation}

with $\sigma_0 = s_0$ and $s_0 < \sigma_1 < \sigma_2 \dots < \sigma_{N_p}
\leq s_m$, then from Equation (\ref{I}) we have

\begin{equation}\begin{split}\label{var sum}
I(s_m) =\,& 
 B(s_m) - \frac1{\T(\sigma,s_m)} \frac{\delta I}{\delta \alpha(s_m)} 
 \approx B(s_m) - S_{N_p} \text{ and}\\
\frac{\delta I(s_m)}{\delta f^k_{lm}}
 \approx\,& \T(\sigma,s_m) \beta^k(\sigma_j) \mu_{lm}(\sigma_j) S_j \\
\end{split}\end{equation}

That is, we can evaluate both $I$ and its sensitivity to variation in the
discrete parameters $f^k_{lm}$ by evaluating one indefinite sum, and
multiplying each of the $N_p$ elements by $N_s$ molecular absorption
cross sections and $N_r$ interpolation coefficients, where $N_r$ is the
number of points in the representation basis.  The additional cost to
compute the sensitivities is $N_p \times N_s \times N_r$ additional
multiplies.

In the present scheme, we compute the sensitivities by evaluating $N_p
\times N_s \times N_r$ additional quadrature steps, each requiring $2 N_p
+ 11 G + N_s N_p^2$ FLOPS, where $0 \leq G \leq N_p$ is the number of
panels requiring Gauss-Legendre quadrature, because we compute the
derivative of Equation (\ref{first}) with respect to $f^k(\sigma)$, and
leave $\frac{\partial\alpha(\sigma)}{\partial f^k(\sigma)} =
\beta^k(\sigma)$ inside the integral:

\begin{equation}
\frac{\partial I(s_m)}{\partial f^k(\sigma)} =
 \int_{s_0}^\sigma \frac{\partial \T(s,s_m)}{\partial f^k(s)}
  B^\prime(s) \, \d s
\end{equation}

Thus the approach described here should be less time consuming by a
factor of $O(N_s \times N_p^2)$, and require less memory than the current
method.

Evaluating the Hessian tensor should similarly be possible without integrating
variational equations.

%=========================================================================

\section{Alternative derivation of sensitivity to $\alpha(\sigma)$}

Instead of computing the variation of Equation (\ref{first}) with respect
to variation of $\alpha(\sigma)$, we can compute the derivative of
Equation (\ref{diff eq}) with respect to variation of $\alpha(\sigma)$
and solve the resulting differential equation:

\begin{equation}\label{four}
\frac{\partial}{\partial \alpha(\sigma)} \frac{\d}{\d s} I(s) +
 \frac{\partial \alpha(s)}{\partial \alpha(\sigma)} I(s) +
  \alpha(s) \frac{\partial I(s)}{\partial \alpha(\sigma)}
= \frac{\partial \alpha(s)}{\partial \alpha(\sigma)} B(s)\,.
\end{equation}

Replacing $\frac{\partial \alpha(s)}{\partial \alpha(\sigma)}$ by
the Dirac $\delta$ function $\delta(s-\sigma)$ and exchanging the order of
differentiation, we have

\begin{equation}\begin{split}\label{five}
\frac{\d}{\d s} \frac{\partial I(s)}{\partial \alpha(\sigma)} +
 \delta(s-\sigma) I(s) +
  \alpha(s) \frac{\partial I(s)}{\partial \alpha(\sigma)}
=\,& \delta(s-\sigma) B(s) \text{ or}\\
\frac{\d}{\d s} \frac{\partial I(s)}{\partial \alpha(\sigma)} +
 \alpha(s) \frac{\partial I(s)}{\partial \alpha(\sigma)}
=\,&
 \delta(s-\sigma) (B(s)-I(s))\,.
\end{split}\end{equation}

Using the initial condition $\frac{\partial I(s_0)}{\partial \alpha(\sigma)}=0$
the solution for Equation (\ref{five}) can be written

\begin{equation}\begin{split}
\frac{\partial I(s)}{\partial \alpha(\sigma)}
=\,& \int_{s_0}^s \delta(s^\prime-\sigma)
  (B(s^\prime)-I(s^\prime)) \T(s^\prime,s)\, \d s^\prime \\
  \,&\\
=\,& (B(\sigma)-I(\sigma)) \T(\sigma,s)\,,
\end{split}\end{equation}

which is the same result as Equation (\ref{var I}).

%=========================================================================

\section{Sensitivity of radiance to variation of temperature}

Because both $\T$ and $B$ depend upon temperature

\begin{equation}\begin{split}\label{T vary}
\delta_T I =\,& -\delta_T B(s_0)) \T(s_0,s_m)
 + (I(s_0) - B(s_0)) \delta_T \T(s_0,s_m) + \delta_T B(s_m) \\
& - \int_{s_0}^{s_m} \delta_T
  \T(\sigma,s_m) B^\prime(\sigma) \, \d \sigma
 - \int_{s_0}^{s_m}
  \T(\sigma,s_m) \delta_T B^\prime(\sigma)
  \, \d \sigma\,.
\end{split}\end{equation}

Following a development similar to Equations (\ref{tau vary}--\ref{tau
var}),

\begin{equation}\begin{split}
\int_{s_0}^{s_m} \delta_T
  \T(\sigma,s_m) B^\prime(\sigma) \d \sigma
= \,&
  \int_{s_0}^{s_m} \d \sigma\,
   \delta T(\sigma) \frac{\partial \alpha(\sigma)}{\partial T(\sigma)}
   \int_{s_0}^\sigma \d \sigma^\prime\,
    \T(\sigma^\prime,s_m) B^\prime(\sigma^\prime) \\
=\,& \int_{s_0}^{s_m} \d \sigma\,
   \delta T(\sigma) \frac{\partial \alpha(\sigma)}{\partial T(\sigma)}
   \T(\sigma,s_m) \int_{s_0}^\sigma \d \sigma^\prime\,
    \T(\sigma^\prime,\sigma) B^\prime(\sigma^\prime)
\end{split}\end{equation}

while

\begin{equation}
\int_{s_0}^{s_m} \T(\sigma,s_m) \delta_T B^\prime(\sigma)\, \d \sigma =
 \int_{s_0}^{s_m} \d \sigma\, \delta T(\sigma) \T(\sigma,s_m)
  \frac{\partial B^\prime(\sigma)}{\partial T(\sigma)}\,.
\end{equation}

Therefore, again using Equations (\ref{VAR}--\ref{SENS}),

\begin{equation}\begin{split}\label{dIdT B'}
\frac{\delta I(s_m)}{\delta T(\sigma)} =\,&
 -\frac{\delta B(s_0)}{\delta T(\sigma)} \T(s_0,s_m)
  + (I(s_0) - B(s_0)) \frac{\delta \T(s_0,s_m)}{\delta T(\sigma)}
  + \frac{\delta B(s_m)}{\delta T(\sigma)}\\
&
  - \frac{\partial \alpha(\sigma)}{\partial T(\sigma)}
    \T(\sigma,s_m)
    \int_{s_0}^\sigma \d \sigma^\prime\,
     \T(\sigma^\prime,\sigma) B^\prime(\sigma^\prime)
  - \T(\sigma,s_m) \frac{\partial B^\prime(\sigma)}{\partial T(\sigma)} \\
=\,&
 (I(s_0) - B(s_0)) \frac{\delta \T(s_0,s_m)}{\delta \alpha(\sigma)}
  \frac{\partial \alpha(\sigma)}{\partial T(\sigma)}
  - \frac{\partial \alpha(\sigma)}{\partial T(\sigma)}
    \T(\sigma,s_m)
    \int_{s_0}^\sigma \d \sigma^\prime\,
     \T(\sigma^\prime,\sigma) B^\prime(\sigma^\prime)
  - \T(\sigma,s_m) \frac{\partial B^\prime(\sigma)}{\partial T(\sigma)} \\
=\,&
  -(I(s_0) - B(s_0)) \T(s_0,s_m)
  \frac{\partial \alpha(\sigma)}{\partial T(\sigma)}
  - \frac{\partial \alpha(\sigma)}{\partial T(\sigma)}
    \T(\sigma,s_m)
    \int_{s_0}^\sigma \d \sigma^\prime\,
     \T(\sigma^\prime,\sigma) B^\prime(\sigma^\prime)
  - \T(\sigma,s_m) \frac{\partial B^\prime(\sigma)}{\partial T(\sigma)} \\
=\,&
  \T(\sigma,s_m) \left( B(s_m) - I(\sigma) \right)
  \frac{\partial \alpha(\sigma)}{\partial T(\sigma)}
  - \T(\sigma,s_m) \frac{\partial B^\prime(\sigma)}{\partial T(\sigma)}
 = \frac{\partial \alpha(\sigma)}{\partial T(\sigma)}
   \frac{\delta I(s_m)}{\delta \alpha(\sigma)}
   - \T(\sigma,s_m) \frac{\partial B^\prime(\sigma)}{\partial T(\sigma)} \,,
\end{split}\end{equation}

where Equation (\ref{I var alpha}) has been used in the last step to
represent the integral in Equation (\ref{dIdT B'}).

The factor $\frac{\partial B^\prime(\sigma)}{\partial T(\sigma)}$ in the
final expression is troublesome.  If we start instead with Equation
(\ref{first}) we arrive at the alternative result

\begin{equation}\label{dIdT}
\frac{\delta I(s_m)}{\delta T(\sigma)} =
 \frac{\partial \alpha(\sigma)}{\partial T(\sigma)}
   \frac{\delta I(s_m)}{\delta \alpha(\sigma)} +
   \alpha(\sigma) \T(\sigma,s_m)
    \frac{\partial B(\sigma)}{\partial T(\sigma)}\,.
\end{equation}

$B(\sigma)$ satisfies the differential equation $\frac{\partial
B(\sigma)}{\partial T(\sigma)} = \frac{B(\sigma)}{T(\sigma)^2} \left(
\frac{h \nu}k + B(\sigma) \right)$.  Inserting this result into Equation
(\ref{dIdT}) we have

\begin{equation}\label{var T}\boxed{
\frac{\delta I(s_m)}{\delta T(\sigma)} =
 \frac{\partial \alpha(\sigma)}{\partial T(\sigma)}
   \frac{\delta I(s_m)}{\delta \alpha(\sigma)} +
   \alpha(\sigma) \T(\sigma,s_m)
    \frac{B(\sigma)}{T(\sigma)^2} \left( \frac{h \nu}k + B(\sigma) \right)
}\,.
\end{equation}

%=========================================================================

\section{Alternative derivation of sensitivity to temperature}

Instead of computing the variation of Equation (\ref{first}) with respect
to variation of $T(\sigma)$, we can compute the derivative of
Equation (\ref{diff eq}) with respect to variation of $T(\sigma)$
and solve the resulting differential equation:

\begin{equation}\begin{split}\label{T four}
\frac{\partial}{\partial T(\sigma)} \frac{\d}{\d s} I(s) +
 \frac{\partial \alpha(s)}{\partial T(\sigma)} I(s) +
  \alpha(s) \frac{\partial I(s)}{\partial T(\sigma)}
=\,& \frac{\partial \alpha(s)}{\partial T(\sigma)} B(s)
  +  \alpha(s) \frac{\partial B(s)}{\partial T(\sigma)} \text{ or}\\
\frac{\partial}{\partial T(\sigma)} \frac{\d}{\d s} I(s) +
 \frac{\partial \alpha(s)}{\partial \alpha(\sigma)}
 \frac{\partial \alpha(\sigma)}{\partial T(\sigma)} I(s) +
  \alpha(s) \frac{\partial I(s)}{\partial T(\sigma)}
=\,& \frac{\partial \alpha(s)}{\partial \alpha(\sigma)}
     \frac{\partial \alpha(\sigma)}{\partial T(\sigma)} B(s)
  +  \alpha(s)
      \frac{\partial B(s)}{\partial B(\sigma)}
      \frac{\partial B(\sigma)}{\partial T(\sigma)}\,.
\end{split}\end{equation}

Replacing $\frac{\partial \alpha(s)}{\partial \alpha(\sigma)}$ and
$\frac{\partial B(s)}{\partial B(\sigma)}$ by the Dirac $\delta$
function $\delta(s-\sigma)$ and exchanging the order of differentiation,
we have

\begin{equation}\begin{split}\label{T five}
\frac{\d}{\d s} \frac{\partial I(s)}{\partial T(\sigma)} +
 \delta(s-\sigma) \frac{\partial \alpha(\sigma)}{\partial T(\sigma)} I(s) +
  \alpha(s) \frac{\partial I(s)}{\partial T(\sigma)}
=\,& \delta(s-\sigma) \left( \frac{\partial \alpha(\sigma)}{\partial T(\sigma)}
     B(s) + \alpha(s) \frac{\partial B(\sigma)}{\partial T(\sigma)} \right
     ) \text{ or}\\
\frac{\d}{\d s} \frac{\partial I(s)}{\partial \alpha(\sigma)} +
 \alpha(s) \frac{\partial I(s)}{\partial \alpha(\sigma)}
=\,&
 \delta(s-\sigma)
  \left( \frac{\partial \alpha(\sigma)}{\partial T(\sigma)}
    \left(B(s)-I(s)\right) +
    \alpha(s) \frac{\partial B(\sigma)}{\partial T(\sigma)}
    \right )\,.
\end{split}\end{equation}

Using the initial condition $\frac{\partial I(s_0)}{\partial \alpha(\sigma)}=0$
the solution for Equation (\ref{T five}) can be written

\begin{equation}\begin{split}
\frac{\partial I(s)}{\partial T(\sigma)}
=\,& \int_{s_0}^s \delta(s^\prime-\sigma)
  \left ( \frac{\partial \alpha(\sigma)}{\partial T(\sigma)}
    \left(B(s)-I(s)\right) +
    \alpha(s) \frac{\partial B(\sigma)}{\partial T(\sigma)}
    \right ) \T(s^\prime,s) \,\d s^\prime \\
  \,&\\
=\,& \left (
    \frac{\partial \alpha(\sigma)}{\partial T(\sigma)}
    \left(B(s)-I(s)\right) +
     \alpha(s) \frac{\partial B(\sigma)}{\partial T(\sigma)}
     \right ) \T(\sigma,s)
 = \frac{\partial \alpha(\sigma)}{\partial T(\sigma)}
   \frac{\delta I(s)}{\delta \alpha(\sigma)} +
   \alpha(\sigma) \T(\sigma,s)
    \frac{\partial B(\sigma)}{\partial T(\sigma)} \,,
\end{split}\end{equation}

which is the same result as Equation (\ref{var T}).

\label{lastpage}
\end{document}

% $Id$

% $Log$
% Revision 1.4  2010/11/23 22:53:30  vsnyder
% Repair a typo in the final equation
%
% Revision 1.3  2010/11/23 21:48:42  vsnyder
% Incorporate interpolating factors, re-work temperature derivatives
%
% Revision 1.2  2010/05/20 23:47:56  vsnyder
% Add Id line
%
% Revision 1.1  2010/04/09 02:23:24  vsnyder
% Initial commit -- yes, at r1, not r0
%
