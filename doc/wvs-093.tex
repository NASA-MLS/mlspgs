\documentclass[11pt]{article}
\usepackage[fleqn]{amsmath}

\textwidth 6.5in
\oddsidemargin -0.25in
%\evensidemargin -0.5in
\topmargin -0.45in
\textheight 9.6in

\newcommand{\docname}{\bf wvs-093r12}
\newcommand{\docdate}{14 March 2013}

\ifx\pdfoutput\undefined
  \pdfoutput=0
  \usepackage[hypertex,plainpages,hyperindex=true]{hyperref}
  \hypersetup{%
    hypertexnames=false%
  }
  % Specify the driver for the color package
  \ExecuteOptions{dvips}
  %\ExecuteOptions{xdvi}
\else
  \ifnum\pdfoutput>0
    \usepackage[pdftex,plainpages,hyperindex=true,pdfpagelabels]{hyperref}
    \hypersetup{%
      hypertexnames=false,%
      colorlinks=true,%
      linktocpage=true,%
    }
    % Specify the driver for the color package
    \ExecuteOptions{pdftex}
  \else
    \usepackage[hypertex,plainpages,hyperindex=true]{hyperref}
    \hypersetup{%
      hypertexnames=false%
    }
    % Specify the driver for the color package
    \ExecuteOptions{dvips}
    %\ExecuteOptions{xdvi}
  \fi
\fi

\hyperbaseurl{}
\ifx\dvidir\undefined
  \newcommand\hr[1]{\href{#1.dvi}{dvi} \href{#1.pdf}{pdf}}
\else
  \newcommand\hr[1]{\href{\dvidir/#1.dvi}{dvi} \href{\pdfdir/#1.pdf}{pdf}}
\fi
\newcommand\h[1]{#1 \hr{#1}}

\begin{document}

%\tracingcommands=1
\newlength{\hW} % heading box width
\newlength{\pW} % page number field width
\settowidth{\hW}{\docname}
\settowidth{\pW}{Page \pageref{lastpage}\ of \pageref{lastpage}}
\ifdim \pW > \hW \setlength{\hW}{\pW} \fi
\makeatletter
\def\@biblabel#1{#1.}
\newcommand{\ps@twolines}{%
  \renewcommand{\@oddhead}{%
    \docdate\hfill\parbox[u]{\hW}{{\hfill\docname}\newline
                          Page \thepage\ of \pageref{lastpage}}}%
\renewcommand{\@evenhead}{}%
\renewcommand{\@oddfoot}{}%
\renewcommand{\@evenfoot}{}%
}%
\makeatother
\pagestyle{twolines}

\renewcommand{\d}{\text{d}}
\newcommand{\T}{\mathcal{T}}
\newcommand{\M}{\mathcal{M}}

\vspace{-10pt}
\begin{tabbing}
\phantom{References: }\= \\
To: \>Bill, Nathaniel, Igor\\
Subject: \>Computing derivatives in the MLS Full Forward Model\\
From: \>Van Snyder\\
References: \> \h{iy-007}, \h{wvs-100}, \h{wvs-101}\\
\end{tabbing}

\parindent 0pt \parskip 5pt
\vspace{-20pt}

\section{Introduction}

This note shows that the MLS Full Forward Model could compute the Jacobian
matrix necessary for retrieval more efficiently than it presently does.

Thanks are due to the seminar \emph{Sensitivity Analysis of Mathematical Models}
on 29 March -- 1 April 2010 by Eugene Ustinov.

\section{Mathematical generalities}

Consider a function $F(x_1, \dots, x_n)$ with Maclaurin series $c +
\sum_j a_j x_j + \sum_{jk} b_{jk} x_j x_k \dots$.  Its variation due to
variation $\delta x_j$ of its arguments is $\delta F = \sum_j a_j \delta
x_j + O(\delta x_j^2)$.  From this, in the limit $\delta x_j \rightarrow
0$, it is obvious that the sensitivities of $F$ due to variations in its
arguments are the partial derivatives

\begin{equation}
\frac{\partial F}{\partial x_j} = a_j\,.
\end{equation}

If we carry this over to a functional $F[X(\xi)]$ of a continuum $X(\xi)$
instead of a function of the finite set $\{x_1,\dots,x_n\}$, replacing
the summations in the Maclaurin series with integrals,
\emph{viz.}

\begin{equation}
F[X(\xi)] = c + \int_{\xi_0}^{\xi_1} a(\xi) X(\xi) \,\d \xi +
 \int_{\xi_0}^{\xi_1} \int_{\xi_0}^{\xi_1} b(\xi,\xi^\prime) X(\xi)
 X(\xi^\prime) \d \xi \d \xi^\prime + \dots
\end{equation}

then its variation due to variation $\delta X(\xi)$ of its argument
function (also known as ``continuous parameter'') $X(\xi)$, for small
variation $\delta X(\xi)$, is

\begin{equation}\label{VAR}
\delta F[X(\xi)] = \int_{\xi_0}^{\xi_1} a(\xi) \,\delta X(\xi) \,\d \xi
  + O(\delta X(\xi)^2)
 = \lim_{\Delta \xi_i \rightarrow 0}
  \sum_i a(\xi_i) \delta X(\xi_i) \Delta \xi_i + O(\delta X(\xi)^2)
\end{equation}

and its sensitivity due to variation $\delta X(\xi)$ of its argument
function $X(\xi)$ is

\begin{equation}\label{SENS}
\frac{\delta F[X(\xi)]}{\delta X(\xi)}
 \equiv \lim_{\Delta \xi \rightarrow 0 \atop \delta X(\xi) \rightarrow 0}
  \frac{\partial F[X(\xi)]}{\Delta \xi \partial X(\xi)}
 = \frac{a(\xi) \delta X(\xi) \Delta \xi}{\delta X(\xi) \Delta \xi}
 = a(\xi)
\end{equation}

(also see pages 27-29 of {\bf Calculus of Variations} by I. M. Gelfand
and S. V. Fomin).

%=========================================================================

\section{Radiance}

The radiance is the solution of the non-scattering radiative transfer
equation

\begin{equation}\label{diff eq}
\frac{\partial I(s)}{\partial s} + \alpha(s) I(s) = \alpha(s) B(s)\,.
\end{equation}

Using notation similar to iy-007 and following the arguments therein, the
solution is usually written as

\begin{equation}\label{first}
I[\alpha(s),s_\M] = I(s_0) \T(s_0,s_\M) +
 \int_{s_0}^{s_\M} \T(s,s_\M) \alpha(s) B(s) \,
  \d s\,,
\end{equation}

(meaning $I$ is explicitly a functional of $\alpha(s)$ and a function of
$s_\M$) where $s_\M$ is the location of the instrument, $s_0$ is the
location of the end of the path away from the instrument, $B(s)$ is the
Planck function $\frac{h \nu}k \left( \exp\left(\frac{h \nu}{k
T(s)}\right) -1\right)^{-1}$,

\begin{equation}\label{two}
\T(s,s_\M) = \exp\left( - \int_s^{s_\M} \alpha(\sigma)
 \, \d \sigma \right) \,,
\end{equation}

$\alpha(\sigma) = \sum_k \beta^k(\sigma) f^k(\sigma) = \sum_k
\beta^k(\sigma) f^k_{lm} \mu^k_{lm}(\sigma)$ is the volume absorption
coefficient, $f^k(\sigma)$ is the volume mixing ratio of the $k^\text{th}$
species at $\sigma$, $f^k_{lm}$ is the volume mixing ratio of the
$k^\text{th}$ species at the $(\phi^k_l,\zeta^k_m)$ representation basis
point for the $k^\text{th}$ species, $\mu^k_{lm}(\sigma)$ is the
interpolation coefficient from the $(\phi^k_l,\zeta^k_m)$ representation
basis point for the $k^\text{th}$ species to the point $\sigma$ on the
integration path (see \h{wvs-101}), and $\beta^k(\sigma)$ is the
absorption cross section for the $k^\text{th}$ species at $\sigma$.

From Equation (\ref{two})

\begin{equation}
\frac{\partial \T(s,s_\M)}{\partial s} = \alpha(s) \T(s,s_\M).
\end{equation}

Substituting this into Equation (\ref{first}) gives

\begin{equation}\label{RAD}
I[\alpha(s),s_\M] = \T(s_0,s_\M) I(s_0) +
 \int_{s_0}^{s_\M} B(s) \frac{\partial \T(s,s_\M)}{\partial s}
  \, \d s \,.
\end{equation}

In the MLS full forward model, Equation (\ref{RAD}) is
integrated by parts to give

\begin{equation}\begin{split}\label{I}
I[\alpha(s),s_\M] =\,& \T(s_0,s_\M) (I(s_0)-B(s_0)) + B(s_\M) -
 \int_{s_0}^{s_\M} \T(s,s_\M) \, B^\prime(s) \,
  \d s \\
=\,& \T(s_0,s_\M) (I(s_0)-B(s_0)) + B(s_\M) -
 \int_{B(s_0)}^{B(s_\M)} \T(s,s_\M) \, \d B(s)\,,
\end{split}\end{equation}

where $B^\prime(s) = \frac{\partial B(s)}{\partial s}$ (remember
$\T(s_\M,s_\M)=1$). Henceforth, $I[\alpha(\sigma),s_\M]$ will be abbreviated
to $I(s_\M)$.

%=========================================================================

\section{Sensitivity of radiance to variation of $\alpha(\sigma)$}

Since $B(\sigma)$ does not depend upon $\alpha(\sigma)$, the variation of
$I(s_\M)$ as a result of variation of $\alpha(\sigma)$ is

\begin{equation}\label{I var}
\delta_\alpha I(s_\M) =
 \left( I(s_0)-B(s_0) \right) \delta_\alpha \T(s_0,s_\M) -
  \int_{s_0}^{s_\M} \left(\delta_\alpha \T(\sigma,s_\M) \right)\,
   B^\prime(\sigma) \, \d \sigma\,.
\end{equation}

To compute $\delta_\alpha \T(\sigma,s_\M)$, replace $\alpha(\sigma)$
in Equation (\ref{two}) by $\alpha(\hat\sigma) +
\delta\alpha(\hat\sigma)$ and compute the difference from Equation
(\ref{two}):

\begin{equation}\begin{split}\label{tau vary}
\delta_\alpha \T(\sigma,s_\M)
=\,& \exp\left( - \int_\sigma^{s_\M} (\alpha(\hat\sigma) +
                                       \delta\alpha(\hat\sigma) )
         \, \d \hat\sigma \right) -
     \exp\left( - \int_\sigma^{s_\M} \alpha(\hat\sigma)
         \, \d \hat\sigma \right) \\
=\,& \exp\left( - \int_\sigma^{s_\M} \alpha(\hat\sigma)
         \, \d \hat\sigma \right)
     \exp\left( - \int_\sigma^{s_\M} \delta\alpha(\hat\sigma)
         \, \d \hat\sigma \right) -
     \exp\left( - \int_\sigma^{s_\M} \alpha(\hat\sigma)
         \, \d \hat\sigma \right) \\
=\,& \exp\left( - \int_\sigma^{s_\M} \alpha(\hat\sigma)
         \, \d \hat\sigma \right)
     \left(
       \exp\left( - \int_\sigma^{s_\M} \delta\alpha(\hat\sigma)
         \, \d \hat\sigma \right) -1
     \right) \\
=\,& \exp\left( - \int_\sigma^{s_\M} \alpha(\hat\sigma)
         \, \d \hat\sigma \right)
     \left(1 - \int_\sigma^{s_\M} \delta\alpha(\hat\sigma)
                \, \d \hat\sigma +
           O\left( \left( \int_\sigma^{s_\M} \delta\alpha(\hat\sigma)
                           \, \d \hat\sigma
            \right)^2 \right) - 1 \right)\,.
\end{split}\end{equation}

Therefore, in the limit for small perturbation $\delta
\alpha(\hat\sigma)$

\begin{equation}\label{tau var}
\delta_\alpha \T(\sigma,s_\M) =
 \exp \left( - \int_\sigma^{s_\M} \alpha(\hat\sigma) \,\d \hat\sigma \right)
 \cdot
 \left( - \int_\sigma^{s_\M} \delta \alpha(\hat\sigma) \,\d \hat\sigma \right)
=
 -\T(\sigma,s_\M) \int_\sigma^{s_\M} \delta \alpha(\hat\sigma) \,\d \hat\sigma\,.
\end{equation}

Substituting Equation (\ref{tau var}) into the integral in Equation
(\ref{I var}) gives (exchanging the order of integration in the last step)

\begin{equation}\begin{split}\label{var1}
- \int_{s_0}^{s_\M} \delta_\alpha \T(\sigma,s_\M) \,
 B^\prime(\sigma) \, \d \sigma
=\,&
 \int_{s_0}^{s_\M} \d \sigma \,\T(\sigma,s_\M)\,
    B^\prime(\sigma) 
    \int_\sigma^{s_\M} \d \hat\sigma\, \delta
    \alpha(\hat\sigma) \\
=\,&
 \int_{s_0}^{s_\M} \d \sigma\, \delta \alpha(\sigma)
   \int_{s_0}^\sigma \d \hat\sigma \, \T(\hat\sigma,s_\M) \,
    B^\prime(\hat\sigma)\,.
\end{split}\end{equation}

Substituting Equation (\ref{tau var}) into the first term for $\delta_\alpha
I(s_\M)$ in Equation (\ref{I var}) gives

\begin{equation}\label{var2}
\left( I(s_0)-B(s_0) \right) \delta_\alpha \T(s_0,s_\M)
= -\T(s_0,s_\M) \left( I(s_0)-B(s_0) \right)
   \int_{s_0}^{s_\M} \delta \alpha(\sigma)\, \d \sigma\,.
\end{equation}

Collecting Equations (\ref{var1}) and (\ref{var2}) and factoring, the
resulting variation in $I(s_\M)$ due to variation in $\alpha(\sigma)$ is

\begin{equation}\begin{split}
\delta_\alpha I(s_\M)
=\,& \int_{s_0}^{s_\M} \d \sigma \,\delta \alpha(\sigma)
 \left\{ \int_{s_0}^\sigma \T(\hat\sigma,s_\M) \,
    B^\prime(\hat\sigma) \,
     \d \hat\sigma
 - \T(s_0,s_\M) \left( I(s_0)-B(s_0) \right) \right\} \\
=\,& \int_{s_0}^{s_\M} \d \sigma \,\delta \alpha(\sigma)
 \left\{ \T(\sigma,s_\M) \left[ \int_{s_0}^\sigma \T(\hat\sigma,\sigma) \,
    B^\prime(\hat\sigma) \,
     \d \hat\sigma
 - \T(s_0,\sigma) \left( I(s_0)-B(s_0) \right) \right] \right\}\,.
\end{split}\end{equation}

According to Equations (\ref{VAR}) and (\ref{SENS}), the sensitivity of
$I(s_\M)$ to variation $\delta \alpha(\sigma)$ in $\alpha(\sigma)$ is the
factor between \{ and \}, \emph{viz.}


\begin{equation}\label{I var alpha}
\frac{\delta I(s_\M)}{\delta \alpha(\sigma)}
=
 \T(\sigma,s_\M) \left[ \int_{s_0}^\sigma \T(\hat\sigma,s_\M) \,
    B^\prime(\hat\sigma) \,
     \d \hat\sigma
 - \T(s_0,\sigma) \left( I(s_0)-B(s_0) \right) \right] \,.
\end{equation}

Using Equation (\ref{I}) in Equation (\ref{I var alpha}) we have

\begin{equation}\label{var I}\boxed{
\frac{\delta I(s_\M)}{\delta \alpha(\sigma)} =
 \T(\sigma,s_\M) \left( B(\sigma) - I(\sigma) \right)}\,.
\end{equation}

%=========================================================================

\section{Alternative derivation of sensitivity to $\alpha(\sigma)$}

Instead of computing the variation of Equation (\ref{first}) with respect
to variation of $\alpha(\sigma)$, we can compute the derivative of
Equation (\ref{diff eq}) with respect to variation of $\alpha(\sigma)$
and solve the resulting differential equation:

\begin{equation}\label{four}
\frac{\partial}{\partial \alpha(\sigma)} \frac{\d}{\d s} I(s) +
 \frac{\partial \alpha(s)}{\partial \alpha(\sigma)} I(s) +
  \alpha(s) \frac{\partial I(s)}{\partial \alpha(\sigma)}
= \frac{\partial \alpha(s)}{\partial \alpha(\sigma)} B(s)\,.
\end{equation}

Replacing $\frac{\partial \alpha(s)}{\partial \alpha(\sigma)}$ by
the Dirac $\delta$ function $\delta(s-\sigma)$ and exchanging the order of
differentiation, we have

\begin{equation}\begin{split}\label{five}
\frac{\d}{\d s} \frac{\partial I(s)}{\partial \alpha(\sigma)} +
 \delta(s-\sigma) I(s) +
  \alpha(s) \frac{\partial I(s)}{\partial \alpha(\sigma)}
=\,& \delta(s-\sigma) B(s) \text{ or}\\
\frac{\d}{\d s} \frac{\partial I(s)}{\partial \alpha(\sigma)} +
 \alpha(s) \frac{\partial I(s)}{\partial \alpha(\sigma)}
=\,&
 \delta(s-\sigma) (B(s)-I(s))\,.
\end{split}\end{equation}

Using the initial condition $\frac{\partial I(s_0)}{\partial \alpha(\sigma)}=0$
the solution for Equation (\ref{five}) can be written

\begin{equation}\begin{split}
\frac{\partial I(s)}{\partial \alpha(\sigma)}
=\,& \int_{s_0}^s \delta(\hat{s}-\sigma)
  (B(\hat{s})-I(\hat{s})) \T(\hat{s},s)\, \d \hat{s} \\
  \,&\\
=\,& (B(\sigma)-I(\sigma)) \T(\sigma,s)\,,
\end{split}\end{equation}

which is (reassuringly) the same result as Equation (\ref{var I}).

%=========================================================================

\section{Sensitivity to mixing ratio, $f^k_{lm}$, on the grid}

Remember that Equation (\ref{var I}) is the variation of the
\emph{functional} $I[\alpha(\sigma),s_\M]$ with respect to variation of
the \emph{function} $\alpha(\sigma)$.


Using bilinear interpolation of mixing ratios $f^k_{lm}$ from a
$(\phi^k_l,\phi^k_{l+1}) \times (\zeta^k_m,\zeta^k_{m+1})$ grid cell for
the $k^\text{th}$ species to path points within the cell,
%
\begin{equation}\begin{split}\label{three}
\alpha(s)
\,&= \sum_k \beta^k(s) f^k(s)
 = \sum_k \beta^k(s) \sum_{l,m}
     \left[ \begin{array}{ll} \xi^k_l(s) &
                               1-\xi^k_l(s) \end{array} \right]
      \left[ \begin{array}{ll} f^k_{\ell,m}   & f^k_{\ell,m+1} \\
                               f^k_{\ell+1,m} & f^k_{\ell+1,m+1}
             \end{array} \right]
      \left[ \begin{array}{l} \eta^k_m(s) \\ 1-\eta^k_m(s)
      \end{array} \right] \\
=\,& \sum_k \beta^k(s) \sum_{l,m} \mu^k_{lm}(s) f^k_{lm} \\
\end{split}\end{equation}
\vspace*{-0.35in}
\begin{equation}\begin{split}
\text{ where }
\xi^k_l(s)
=\,& \frac{\phi^k_{l+1}-\phi(s)}{\phi^k_{l+1}-\phi^k_l}
 \text{ for } \phi^k_l \leq \phi(s) \leq \phi^k_{l+1}
 \text{ and zero elsewhere, and } \\
\eta^k_m(s)
=\,& \frac{\zeta^k_{m+1}-\zeta(s)}{\zeta^k_{m+1}-\zeta^k_m}
 \text{ for } \zeta^k_m \leq \zeta(s) \leq \zeta^k_{m+1}
 \text{ and zero elsewhere.} \\
\end{split}\end{equation}

and where by shifting indices in Equation (\ref{three})
%
\begin{equation}\begin{split}\label{mu}
\mu^k_{lm}(s) =
\left\{
\begin{array}{lll}
\xi_{i,\ell}(s) \eta_{i,m}(s)
  & \phi_l \leq \phi(s) \leq \phi_{l+1}
  & \zeta_m \leq \zeta(s) \leq \zeta_{m+1} \\
(1-\xi_{i,\ell-1}(s)) \eta_{i,m}(s)
  & \phi_{l-1} \leq \phi(s) \leq \phi_l
  & \zeta_m \leq \zeta(s) \leq \zeta_{m+1} \\
\xi_{i,\ell}(s) (1-\eta_{i,m-1}(s))
  & \phi_l \leq \phi(s) \leq \phi_{l+1}
  & \zeta_{m-1} \leq \zeta(s) \leq \zeta_m \\
(1-\xi_{i,\ell-1}(s))  (1-\eta_{i,m-1}(s))
  & \phi_{l-1} \leq \phi(s) \leq \phi_l
  & \zeta_{m-1} \leq \zeta(s) \leq \zeta_m \\
0 & \text{elsewhere} \\
\end{array}
\right.
\end{split}\end{equation}

Thus, using $\frac{\partial \alpha(s)}{\partial f^k_{lm}} = \beta^k(s)
\mu^k_{lm}(s)$,  the sensitivity of $I(s_\M)$ to variation of the discrete
parameter $f^k_{lm}$, that is, a \emph{partial derivative}, is
%
\begin{equation}
\boxed{
\begin{split}\label{dI_dfklm}
\frac{\partial I(s_\M)}{\partial f^k_{lm}}
=\,&
 \int_{s_0}^{s_\M}
 \frac{\delta I(s_\M)}{\delta \alpha(\sigma)}
  \frac{\partial \alpha(\sigma)}{\partial f^k(\sigma)}
  \frac{\partial f^k(\sigma)}{\partial f^k_{lm}} \,\d \sigma =
 \int_{s_0}^{s_\M}
 \frac{\delta I(s_\M)}{\delta \alpha(\sigma)} \beta^k(\sigma)
  \mu^k_{lm}(\sigma) \,\d \sigma\\
=\,&
 \int_{s_0}^{s_\M}
 \T(\sigma,s_\M)
  \left( B(\sigma) - I(\sigma) \right) \beta^k(\sigma) \mu^k_{lm}(\sigma)
  \,\d \sigma \,.
\end{split}}
\end{equation}

See \h{wvs-100} for a derivation of the relationship of this result to
the form used in the full forward model.  From Equation (\ref{mu}),
$\mu^k_{lm}(\sigma)$ is nonzero for only a small range of $\sigma$.

Replacing integrals by quadratures, if we evaluate the indefinite product
and indefinite sum
%
\begin{equation}\begin{split}\label{sum}
P_i
=\,& \prod_{j=i}^{N_p-1} \exp \left(
  - \sum_k w_k \alpha(\sigma_{jk}) \Delta \sigma_j
  \right)
 = P_{i+1} \exp \left(- \sum_k w_k \alpha(\sigma_{jk}) \Delta \sigma_i
                 \right)
 \approx \T(\sigma_i,s_\M)
  \text{ and} \\
S_j
=\,& P_0 \left(I(s_0)-B(s_0)\right) + B(s_\M)
 - \sum_{i=1}^j  P_i \Delta B(\sigma_i)
 = S_{j-1} - P_j \Delta B(\sigma_j)
 \approx I(\sigma_j)
\end{split}\end{equation}
%
where $s_0 = \sigma_0 < \sigma_1 < \sigma_2 \dots < \sigma_{N_p}
\leq s_\M$, and $w_k$ and $\sigma_j \leq \sigma_{jk} \leq \sigma_{j+1}$
are quadrature weights and abscissae, then from Equations (\ref{I}) and
(\ref{var I}) we have
%
\begin{equation}\boxed{\boxed{
\begin{split}\label{var sum}
I(s_\M) \approx\,& S_{N_p} \text{ and}\\
\frac{\delta I(s_\M)}{\delta f^k_{lm}}
 \approx\,&
  \sum_j
  P_j \beta^k(\sigma_j) \mu^k_{lm}(\sigma_j)
    (S_j - B(\sigma_j) ) \Delta \sigma_j
 = \sum_j
  P_j \beta^k(\zeta_j) \mu^k_{lm}(\zeta_j)
    (S_j - B(\zeta_j) ) \frac{\d \sigma_j}{\d h_j}
                         \frac{\d h_j}{\d \zeta_j} \Delta \zeta_j\\
\end{split}}}
\end{equation}

where $h$ is height, $\zeta = -\log_{10}(p)$, and $p$ is pressure.  The
second formulation allows for the path to depend upon temperature, since
height and pressure are related by hydrostatic equilibrium, which depends
upon temperature.

Since we use bilinear interpolation, $\mu^k_{lm}(\sigma_j)$ is nonzero
only for two values of $j$. Therefore, we can evaluate both $I(s_\M)$ and
its sensitivity to variation in the discrete parameters $f^k_{lm}$ by
evaluating one indefinite sum, and multiplying each of the $N_p$ elements
by $2 N_s$ molecular absorption cross sections, $2 N_r$ interpolation
coefficients, and two values of $\Delta \sigma$ , where $N_r$ is the
number of points in the representation basis.  The additional cost to
compute the sensitivities is $6 N_p \times N_s \times N_r$ multiplies and
two adds.

In the present scheme, we compute the sensitivities by evaluating $N_p
\times N_s \times N_r$ additional quadrature steps, each requiring $2 N_p
+ 11 G + N_s N_p^2$ FLOPS, where $0 \leq G \leq N_p$ is the number of
panels requiring three-point Gauss-Legendre quadrature instead of
trapezoidal quadrature, because we compute the derivative of Equation
(\ref{first}) with respect to $f^k(\sigma)$, and leave
$\frac{\partial\alpha(\sigma)}{\partial f^k_{lm}} = \beta^k(\sigma)$
inside the integral:

\begin{equation}\begin{split}
\frac{\partial I(s_\M)}{\partial f^k_{lm}}
=\,&
 \int_{s_0}^{s_\M} \frac{\delta \T(s,s_\M)}{\delta \alpha(s)}
  \frac{\delta \alpha(s)}{\delta f^k(s)}
  \frac{\partial f^k(s)}{\partial f^k_{lm}}
  B^\prime(s) \, \d s
=\,&
 -\int_{s_0}^{s_\M} \T(s,s_\M) \mu^k_{lm}(s) \beta^k(s)
  B^\prime(s) \, \d s
  \,,
\end{split}\end{equation}

in which the integral defining $\T(s,s_\M)$ is recomputed.  Thus the
approach described here should be less time consuming by a factor of
$O(N_s \times N_p^2)$, and require less memory than the current method.

Evaluating the Hessian tensor should similarly be possible at lower cost.

%=========================================================================

\section{Alternative derivation of sensitivity to $f^k_{lm}$}

Start with the clear-sky non-scattering radiative transfer equation:

\begin{equation}\label{D1}
\frac{\d I(s)}{\d s} + \alpha(s) I(s) = \alpha(s) B(s)\,.
\end{equation}

Taking derivatives of Equation (\ref{D1}) with respect to mixing ratios
$f^k_{lm}$ on the grid gives

\begin{equation}
\frac{\partial}{\partial f^k_{lm}} \frac{\d I(s)}{\d s} +
 \frac{\partial \alpha(s)}{\partial f^k_{lm}} I(s) +
 \alpha(s) \frac{\partial I(s)}{\partial f^k_{lm}} =
 \frac{\partial \alpha(s)}{\partial f^k_{lm}} B(s) \,.
\end{equation}

Exchanging the order of differentiation, using $\frac{\partial
\alpha(s)}{\partial f^k_{lm}} = \beta^k(s) \mu^k_{lm}(s)$, and rearranging
the equation to the usual form gives

\begin{equation}\begin{split}
\frac{\d}{\d s} \frac{\partial I(s)}{\partial f^k_{lm}} +
 \alpha(s) \frac{\partial I(s)}{\partial f^k_{lm}}
=\,&
 \frac{\partial \alpha(s)}{\partial f^k_{lm}} ( B(s) - I(s) ) \\
=\,&
 \beta^k(s) \mu^k_{lm}(s) ( B(s) - I(s) )\,.
\end{split}\end{equation}

Solve for $\frac{\partial I(s_\M)}{\partial f^k_{lm}}$ using
$\frac{\partial I(s_0)}{\partial f^k_{lm}}=0$ to get

\begin{equation}\boxed{
\frac{\partial I(s_\M)}{\partial f^k_{lm}} =
 \int_{s_0}^{s_\M} \T(s,s_\M) \beta^k(s) \mu^k_{lm}(s)
  ( B(s) - I(s) )\, \d s
}
\end{equation}

which is (reassuringly) the same as Equation (\ref{dI_dfklm}).

%=========================================================================

\section{Sensitivity of radiance to variation of temperature}

Because both $\T$ and $B$ depend upon temperature

\begin{equation}\begin{split}\label{T vary}
\delta_T I(s_\M) =\,& -\delta_T B(s_0)) \T(s_0,s_\M)
 + (I(s_0) - B(s_0)) \delta_T \T(s_0,s_\M) + \delta_T B(s_\M) \\
& - \int_{s_0}^{s_\M} \delta_T
  \T(\sigma,s_\M) B^\prime(\sigma) \, \d \sigma
 - \int_{s_0}^{s_\M}
  \T(\sigma,s_\M) \delta_T B^\prime(\sigma)
  \, \d \sigma\,.
\end{split}\end{equation}

Following a development similar to Equations (\ref{tau vary}--\ref{tau
var}),

\begin{equation}\begin{split}
\int_{s_0}^{s_\M} \delta_T
  \T(\sigma,s_\M) B^\prime(\sigma) \d \sigma
= \,&
  \int_{s_0}^{s_\M} \d \sigma\,
   \delta T(\sigma) \frac{\partial \alpha(\sigma)}{\partial T(\sigma)}
   \int_{s_0}^\sigma \d \hat\sigma\,
    \T(\hat\sigma,s_\M) B^\prime(\hat\sigma) \\
=\,& \int_{s_0}^{s_\M} \d \sigma\,
   \delta T(\sigma) \frac{\partial \alpha(\sigma)}{\partial T(\sigma)}
   \T(\sigma,s_\M) \int_{s_0}^\sigma \d \hat\sigma\,
    \T(\hat\sigma,\sigma) B^\prime(\hat\sigma)
\end{split}\end{equation}

while

\begin{equation}
\int_{s_0}^{s_\M} \T(\sigma,s_\M) \delta_T B^\prime(\sigma)\, \d \sigma =
 \int_{s_0}^{s_\M} \d \sigma\, \delta T(\sigma) \T(\sigma,s_\M)
  \frac{\partial B^\prime(\sigma)}{\partial T(\sigma)}\,.
\end{equation}

Therefore, again using Equations (\ref{VAR}--\ref{SENS}),

\begin{equation}\begin{split}\label{dIdT B'}
\frac{\delta I(s_\M)}{\delta T(\sigma)} =\,&
 -\frac{\delta B(s_0)}{\delta T(\sigma)} \T(s_0,s_\M)
  + (I(s_0) - B(s_0)) \frac{\delta \T(s_0,s_\M)}{\delta T(\sigma)}
  + \frac{\delta B(s_\M)}{\delta T(\sigma)}\\
&
  - \frac{\partial \alpha(\sigma)}{\partial T(\sigma)}
    \T(\sigma,s_\M)
    \int_{s_0}^\sigma \d \hat\sigma\,
     \T(\hat\sigma,\sigma) B^\prime(\hat\sigma)
  - \T(\sigma,s_\M) \frac{\partial B^\prime(\sigma)}{\partial T(\sigma)} \\
=\,&
 (I(s_0) - B(s_0)) \frac{\delta \T(s_0,s_\M)}{\delta \alpha(\sigma)}
  \frac{\partial \alpha(\sigma)}{\partial T(\sigma)}
  - \frac{\partial \alpha(\sigma)}{\partial T(\sigma)}
    \T(\sigma,s_\M)
    \int_{s_0}^\sigma \d \hat\sigma\,
     \T(\hat\sigma,\sigma) B^\prime(\hat\sigma)
  - \T(\sigma,s_\M) \frac{\partial B^\prime(\sigma)}{\partial T(\sigma)} \\
=\,&
  -(I(s_0) - B(s_0)) \T(s_0,s_\M)
  \frac{\partial \alpha(\sigma)}{\partial T(\sigma)}
  - \frac{\partial \alpha(\sigma)}{\partial T(\sigma)}
    \T(\sigma,s_\M)
    \int_{s_0}^\sigma \d \hat\sigma\,
     \T(\hat\sigma,\sigma) B^\prime(\hat\sigma)
  - \T(\sigma,s_\M) \frac{\partial B^\prime(\sigma)}{\partial T(\sigma)} \\
=\,&
  \T(\sigma,s_\M) \left( B(s_\M) - I(\sigma) \right)
  \frac{\partial \alpha(\sigma)}{\partial T(\sigma)}
  - \T(\sigma,s_\M) \frac{\partial B^\prime(\sigma)}{\partial T(\sigma)}
 = \frac{\partial \alpha(\sigma)}{\partial T(\sigma)}
   \frac{\delta I(s_\M)}{\delta \alpha(\sigma)}
   - \T(\sigma,s_\M) \frac{\partial B^\prime(\sigma)}{\partial T(\sigma)} \,,
\end{split}\end{equation}

where Equation (\ref{I var alpha}) has been used in the last step to
represent the integral in Equation (\ref{dIdT B'}).

The factor $\frac{\partial B^\prime(\sigma)}{\partial T(\sigma)}$ in the
final expression is troublesome.  If we start instead with Equation
(\ref{first}) we arrive at the alternative result

\begin{equation}\label{dIdT}
\frac{\delta I(s_\M)}{\delta T(\sigma)} =
 \frac{\delta I(s_\M)}{\delta \alpha(\sigma)}
   \frac{\partial \alpha(\sigma)}{\partial T(\sigma)} +
   \alpha(\sigma) \T(\sigma,s_\M)
    \frac{\partial B(\sigma)}{\partial T(\sigma)}\,.
\end{equation}

$B(\sigma)$ satisfies the differential equation $\frac{\partial
B(\sigma)}{\partial T(\sigma)} = \frac{B(\sigma)}{T(\sigma)^2} \left(
\frac{h \nu}k + B(\sigma) \right)$.  Inserting this result into Equation
(\ref{dIdT}) we have

\begin{equation}\label{var T}
\boxed{\begin{split}
\frac{\delta I(s_\M)}{\delta T(\sigma)}
=\,&
 \frac{\delta I(s_\M)}{\delta \alpha(\sigma)}
   \frac{\partial \alpha(\sigma)}{\partial T(\sigma)} +
   \alpha(\sigma) \T(\sigma,s_\M)
    \frac{B(\sigma)}{T(\sigma)^2} \left( \frac{h \nu}k + B(\sigma) \right)
\\
 =\,&
  \T(\sigma,s_\M)
   \left[ \left( B(\sigma) - I(\sigma) \right)
    \frac{\partial \alpha(\sigma)}{\partial T(\sigma)} +
    \alpha(\sigma)
    \frac{B(\sigma)}{T(\sigma)^2} \left( \frac{h \nu}k + B(\sigma) \right)
   \right]
\end{split}}
\end{equation}

%=========================================================================

\section{Alternative derivation of sensitivity to temperature}

Instead of computing the variation of Equation (\ref{first}) with respect
to variation of $T(\sigma)$, we can compute the derivative of
Equation (\ref{diff eq}) with respect to variation of $T(\sigma)$
and solve the resulting differential equation:

\begin{equation}\begin{split}\label{T four}
\frac{\partial}{\partial T(\sigma)} \frac{\d}{\d s} I(s) +
 \frac{\partial \alpha(s)}{\partial T(\sigma)} I(s) +
  \alpha(s) \frac{\partial I(s)}{\partial T(\sigma)}
=\,& \frac{\partial \alpha(s)}{\partial T(\sigma)} B(s)
  +  \alpha(s) \frac{\partial B(s)}{\partial T(\sigma)} \text{ or}\\
\frac{\partial}{\partial T(\sigma)} \frac{\d}{\d s} I(s) +
 \frac{\partial \alpha(s)}{\partial \alpha(\sigma)}
 \frac{\partial \alpha(\sigma)}{\partial T(\sigma)} I(s) +
  \alpha(s) \frac{\partial I(s)}{\partial T(\sigma)}
=\,& \frac{\partial \alpha(s)}{\partial \alpha(\sigma)}
     \frac{\partial \alpha(\sigma)}{\partial T(\sigma)} B(s)
  +  \alpha(s)
      \frac{\partial B(s)}{\partial B(\sigma)}
      \frac{\partial B(\sigma)}{\partial T(\sigma)}\,.
\end{split}\end{equation}

Replacing $\frac{\partial \alpha(s)}{\partial \alpha(\sigma)}$ and
$\frac{\partial B(s)}{\partial B(\sigma)}$ by the Dirac $\delta$
function $\delta(s-\sigma)$ and exchanging the order of differentiation,
we have

\begin{equation}\begin{split}\label{T five}
\frac{\d}{\d s} \frac{\partial I(s)}{\partial T(\sigma)} +
 \delta(s-\sigma) \frac{\partial \alpha(\sigma)}{\partial T(\sigma)} I(s) +
  \alpha(s) \frac{\partial I(s)}{\partial T(\sigma)}
=\,& \delta(s-\sigma) \left( \frac{\partial \alpha(\sigma)}{\partial T(\sigma)}
     B(s) + \alpha(s) \frac{\partial B(\sigma)}{\partial T(\sigma)} \right
     ) \text{ or}\\
\frac{\d}{\d s} \frac{\partial I(s)}{\partial \alpha(\sigma)} +
 \alpha(s) \frac{\partial I(s)}{\partial \alpha(\sigma)}
=\,&
 \delta(s-\sigma)
  \left( \frac{\partial \alpha(\sigma)}{\partial T(\sigma)}
    \left(B(s)-I(s)\right) +
    \alpha(s) \frac{\partial B(\sigma)}{\partial T(\sigma)}
    \right )\,.
\end{split}\end{equation}

Using the initial condition $\frac{\partial I(s_0)}{\partial \alpha(\sigma)}=0$
the solution for Equation (\ref{T five}) can be written

\begin{equation}\begin{split}\label{T9}
\frac{\partial I(s)}{\partial T(\sigma)}
=\,& \int_{s_0}^s \delta(\hat{s}-\sigma)
  \left ( \frac{\partial \alpha(\sigma)}{\partial T(\sigma)}
    \left(B(s)-I(s)\right) +
    \alpha(s) \frac{\partial B(\sigma)}{\partial T(\sigma)}
    \right ) \T(\hat{s},s) \,\d \hat{s} \\
  \,&\\
=\,& \left (
    \frac{\partial \alpha(\sigma)}{\partial T(\sigma)}
    \left(B(s)-I(s)\right) +
     \alpha(s) \frac{\partial B(\sigma)}{\partial T(\sigma)}
     \right ) \T(\sigma,s)
 = \frac{\delta I(s)}{\delta \alpha(\sigma)}
   \frac{\partial \alpha(\sigma)}{\partial T(\sigma)} +
   \alpha(\sigma) \T(\sigma,s)
    \frac{\partial B(\sigma)}{\partial T(\sigma)} \,,
\end{split}\end{equation}

which is (reassuringly) the same result as Equation (\ref{dIdT}).

%=========================================================================

\section{Sensitivity to temperature, $T_{lm}$, on the grid}

Compute the derivative of Equation (\ref{diff eq}) with respect to
$T_{lm}$ to get

\begin{equation}
\frac{\partial}{\partial T_{lm}} \frac{\d}{\d s} I(s) +
 \frac{\partial \alpha(s)}{\partial T_{lm}} I(s) +
  \alpha(s) \frac{\partial I(s)}{\partial T_{lm}}
=\frac{\partial \alpha(s)}{\partial T_{lm}} B(s)
  +  \alpha(s) \frac{\partial B(s)}{\partial T_{lm}} \,.
\end{equation}

Exchanging the order of integration and collecting terms involving
$\frac{\partial I(s)}{\partial T_{lm}}$ on the left gives

\begin{equation}
\frac{\d}{\d s}\frac{\partial I(s)}{\partial T_{lm}} +
  \alpha(s) \frac{\partial I(s)}{\partial T_{lm}}
=\frac{\partial \alpha(s)}{\partial T_{lm}} \left( B(s) - I(s) \right)
  +  \alpha(s) \frac{\partial B(s)}{\partial T_{lm}} \,.
\end{equation}

Replacing $\frac{\partial \alpha(s)}{\partial T_{lm}} = \frac{\partial
\alpha(s)}{\partial T(s)} \frac{\partial T(s)}{\partial T_{lm}} =
\frac{\partial \alpha(s)}{\partial T(s)} \mu^T_{lm}(s)$ and
$\frac{\partial B(s)}{\partial T_{lm}} = \frac{\partial
B(s)}{\partial T(s)} \frac{\partial T(s)}{\partial T_{lm}} =
\frac{\partial B(s)}{\partial T(s)} \mu^T_{lm}(s)$, where $\mu^T_{lm}(s)$
is the interpolation coefficient from $(\phi_l^T,\zeta_m^T)$ to $s$, gives

\begin{equation}
\frac{\d}{\d s}\frac{\partial I(s)}{\partial T_{lm}} +
  \alpha(s) \frac{\partial I(s)}{\partial T_{lm}}
= \mu^T_{lm}(s) \left[
 \frac{\partial \alpha(s)}{\partial T(s)} \left( B(s) - I(s) \right)
  +  \alpha(s) \frac{\partial B(s)}{\partial T(s)} \right] \,,
\end{equation}

for which the solution, using $\frac{\partial I(s_0)}{\partial
T_{lm}} = 0$, is

\begin{equation}\label{T10}\boxed{
\frac{\partial I(s)}{\partial T_{lm}} =
 \int_{s_0}^s \T(\hat{s},s_\M)\, \mu^T_{lm}(\hat{s})
  \left[
 \frac{\partial \alpha(\hat{s})}{\partial T(\hat{s})} \left( B(\hat{s}) - I(\hat{s}) \right)
  +  \alpha(\hat{s}) \frac{\partial B(\hat{s})}{\partial T(\hat{s})} \right]
 \,\d \hat{s} \,.
}\end{equation}

This is the same as the integral in Equation (\ref{T9}), with
$\delta(\hat{s}-\sigma)$ replaced by $\frac{\partial
T(\hat{s})}{\partial T_{lm}} = \mu^T_{lm}(\hat{s})$.

Taking the derivative of Equation (\ref{I}) with respect to $T_{lm}$ gives

\begin{equation}\begin{split}\label{Teq}
\frac{\partial I(s)}{\partial T_{lm}}
=\,&
 -\T(s_0,s) \left(I(s_0) -B(s_0) \right)
  \int_{s_0}^s \frac{\partial \alpha(\hat{s})}{\partial T(\hat{s})}
   \mu^T_{lm}(\hat{s}) \,\d \hat{s} \\
 +
\,&
 \int_{s_0}^s \mu^T_{lm}(\hat{s}) \left[ B^\prime(\hat{s})
  \int_s^s  \frac{\partial \alpha(\sigma)}{\partial T(\sigma)}
   \,\d \sigma -
   \T(\hat{s},s) \frac{\partial B^\prime(\hat{s})}{\partial T(\hat{s})}
   \right] \,\d \hat{s}\,.
\end{split}\end{equation}

This is equivalent to Equation (\ref{T10}), as can be seen (after some
work) by substituting Equation (\ref{I}) into Equation (\ref{T10}).

Equations (\ref{T10}) and (\ref{Teq}) are both complicated by the fact
that the path depends upon temperature everywhere below the path.  When
integrated in $\zeta = -\log_{10}(p)$ coordinates, Equation (\ref{T10})
becomes

\begin{equation}\label{T half}
\frac{\partial I(s)}{\partial T_{lm}} =
 \int_{\zeta_0}^{\zeta_\M} \T(\zeta,\zeta_\M)\, \mu^T_{lm}(\zeta)
  \left[
 \frac{\partial \alpha(\zeta)}{\partial T(\zeta)}
     \left( B(\zeta) - I(\zeta) \right)
  +  \alpha(\zeta) \frac{\partial B(\zeta)}{\partial T(\zeta)} \right]
 \, \frac{\d s}{\d h} \frac{\d h}{\d \zeta}
 \,\d \zeta \,,
\end{equation}

where $\frac{\d h}{\d \zeta}$ depends upon temperature below the path
through hydrostatic equilibrium.

Equation (\ref{T half}) is not sufficient to capture all of the dependence
upon temperature below the path.  If the integral in Equation (\ref{I}) is
written in $\zeta$ coordinates we have

\begin{equation}
\int_{\zeta_0}^{\zeta_\M} \T(\zeta,\zeta_\M) B^\prime(\zeta)
 \, \frac{\d s}{\d h} \frac{\d h}{\d \zeta}
 \,\d \zeta \,.
\end{equation}

Taking the derivative with respect to $T_{lm}$ gives

\begin{equation}
\int_{\zeta_0}^{\zeta_\M}
 \frac{\partial \T(\zeta,\zeta_\M) B^\prime(\zeta)}
      {\partial T_{lm}}
 \, \frac{\d s}{\d h} \frac{\d h}{\d \zeta}
 \,\d \zeta \,+
\int_{\zeta_0}^{\zeta_\M}
 \T(\zeta,\zeta_\M) B^\prime(\zeta) \,
 \frac{\partial}{\partial T_{lm}} \left[
  \frac{\d s}{\d h} \frac{\d h}{\d \zeta} \right]
 \,\d \zeta \,.
\end{equation}

The first integral, together with the derivative of the boundary terms in
Equation (\ref{I}), are equivalent to Equation (\ref{T half}).  The
complete expression for the derivative is therefore

\begin{equation}\boxed{\begin{split}
\frac{\partial I(s)}{\partial T_{lm}}
=\,&
 \int_{\zeta_0}^{\zeta_\M} \T(\zeta,\zeta_\M)\, \mu^T_{lm}(\zeta)
  \left(
 \frac{\partial \alpha(\zeta)}{\partial T(\zeta)}
  \left( B(\zeta) - I(\zeta) \right)
  +  \alpha(\zeta) \frac{\partial B(\zeta)}{\partial T(\zeta)} \right)
 \, \frac{\d s}{\d h} \frac{\d h}{\d \zeta}
 \,\d \zeta \\
\,& + \int_{\zeta_0}^{\zeta_\M}
 \T(\zeta,\zeta_\M) B^\prime(\zeta) \,
   \frac{\partial}{\partial T_{lm}} \left[
  \frac{\d s}{\d h} \frac{\d h}{\d \zeta} \right]
 \,\d \zeta \\
=\,&
  \int_{\zeta_0}^{\zeta_\M} \T(\zeta,\zeta_\M)\, \left[ \mu^T_{lm}(\zeta)
  \left(
 \frac{\partial \alpha(\zeta)}{\partial T(\zeta)}
  \left( B(\zeta) - I(\zeta) \right)
  +  \alpha(\zeta) \frac{\partial B(\zeta)}{\partial T(\zeta)} \right)
  -\alpha(\zeta) B(\zeta) \right]
 \, \frac{\d s}{\d h} \frac{\d h}{\d \zeta}
 \,\d \zeta \\
\,& + \int_{\zeta_0}^{\zeta_\M}
 \T(\zeta,\zeta_\M) B(\zeta) \,
 \frac{\partial}{\partial T_{lm}} \left[
  \frac{\d s}{\d h} \frac{\d h}{\d \zeta} \right]
 \,\d \zeta \\
=\,&
  \int_{\zeta_0}^{\zeta_\M} \T(\zeta,\zeta_\M)\, \left[ \mu^T_{lm}(\zeta)
  \left(
 \frac{\partial \alpha(\zeta)}{\partial T(\zeta)}
  \left( B(\zeta) - I(\zeta) \right)
  +  \alpha(\zeta) \frac{\partial B(\zeta)}{\partial T(\zeta)} \right)
  -\alpha(\zeta) B(\zeta) \right]
 \, \frac{\d s}{\d h} \frac{\d h}{\d \zeta}
 \,\d \zeta \\
\,& + \int_{\zeta_0}^{\zeta_\M}
 \T(\zeta,\zeta_\M) B(\zeta)\mu^T_l(\zeta) \mu^T_m(\phi(\zeta)) \,
   \frac{\partial}{\partial T} \left[
  \frac{\d s}{\d h} \frac{\d h}{\d \zeta} \right]
 \,\d \zeta
 \,.
\end{split}}
\end{equation}

The first integral accounts for the dependence of $\alpha(\zeta)$ and
$B(\zeta)$ on the path upon $T_{lm}$, for which $\mu^T_{lm}(\zeta)$ is
only nonzero where $(\phi^T_l,\zeta^T_m)$ is adjacent to the path.  The
second accounts for the dependence of the path upon $T_{lm}$, for which
$\mu^T_l(\zeta)$ and $\mu^T_m(\phi(\zeta))$ are nonzero where
$(\phi^T_l,\zeta^T_m)$ are adjacent to or below the path.

From the 19 August 2004 ATBD,

\begin{equation}
\frac{\d s}{\d h} = \frac{h(T)}{\sqrt{h(T)^2-H_t(T)^2}}
\text{ and }
\frac{\d h}{\d \zeta} = \frac{h(T)^2 k\, T \ln 10}{g_0 R^2_0 \mathcal{M}} \,,
\end{equation}

where $t$ denotes the tangent point.  From this,

\begin{equation}
\frac{\partial}{\partial T}
 \left[ \frac{\d s}{\d h} \frac{\d h}{\d \zeta} \right] =
 \left(
 \frac{2 h(T)^2 - 3 H_t(T)^2}{\left(h(T)^2-H_t(T)^2\right)^\frac32}
  \frac{\d h(T)}{\d T} +
 \frac{h(T) H_t(T)}{\left(h(T)^2-H_t(T)^2\right)^\frac32}
  \frac{\d H_t(T)}{\d T} +
 \frac1T \frac{\d s}{\d h}
 \right)  \frac{\d h}{\d \zeta} \,.
\end{equation}

From the definition of $B(\zeta)$ given after Equation (\ref{first}),

\begin{equation}
\frac{\text{d} B}{\text{d} T} =
\frac{B}{T^2} \left( \frac{h\nu}k + B \right)\,.
\end{equation}

\label{lastpage}
\end{document}

% $Id$

% $Log$
% Revision 1.12  2013/03/14 01:49:35  vsnyder
% Correct some typos, reduce notation confusion
%
% Revision 1.11  2012/08/24 22:06:53  vsnyder
% Cannonball polishing
%
% Revision 1.10  2011/02/05 01:29:11  vsnyder
% More temperature derivatives on the grid
%
% Revision 1.9  2010/12/30 00:51:43  vsnyder
% Temperature derivative on the grid
%
% Revision 1.8  2010/12/23 20:33:48  vsnyder
% Simplify discussion of interpolation coefficients by reference to wvs-101
%
% Revision 1.7  2010/12/18 03:53:07  vsnyder
% Correct typo in Equation (30)
%
% Revision 1.6  2010/12/14 23:57:07  vsnyder
% Maybe it's correct now
%
% Revision 1.5  2010/12/04 02:24:40  vsnyder
% Correct VMR derivative, add temperature derivative
%
% Revision 1.4  2010/11/23 22:53:30  vsnyder
% Repair a typo in the final equation
%
% Revision 1.3  2010/11/23 21:48:42  vsnyder
% Incorporate interpolating factors, re-work temperature derivatives
%
% Revision 1.2  2010/05/20 23:47:56  vsnyder
% Add Id line
%
% Revision 1.1  2010/04/09 02:23:24  vsnyder
% Initial commit -- yes, at r1, not r0
%
