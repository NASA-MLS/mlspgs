\documentclass[11pt]{article}
\usepackage[fleqn]{amsmath}

\textwidth 6.5in
\oddsidemargin -0.25in
%\evensidemargin -0.5in
\topmargin -0.25in
\textheight 9.0in

\newcommand{\docname}{\bf wvs-093r1}
\newcommand{\docdate}{6 April 2010}

\begin{document}

%\tracingcommands=1
\newlength{\hW} % heading box width
\newlength{\pW} % page number field width
\settowidth{\hW}{\docname}
\settowidth{\pW}{Page \pageref{lastpage}\ of \pageref{lastpage}}
\ifdim \pW > \hW \setlength{\hW}{\pW} \fi
\makeatletter
\def\@biblabel#1{#1.}
\newcommand{\ps@twolines}{%
  \renewcommand{\@oddhead}{%
    \docdate\hfill\parbox[u]{\hW}{{\hfill\docname}\newline
                          Page \thepage\ of \pageref{lastpage}}}%
\renewcommand{\@evenhead}{}%
\renewcommand{\@oddfoot}{}%
\renewcommand{\@evenfoot}{}%
}%
\makeatother
\pagestyle{twolines}

\vspace{-10pt}
\begin{tabbing}
\phantom{References: }\= \\
To: \>Bill, Nathaniel, Igor\\
Subject: \>Computing derivatives in the MLS Full Forward Model\\
From: \>Van Snyder\\
\end{tabbing}

\parindent 0pt \parskip 5pt
\vspace{-20pt}

\section{Introduction}

This note shows that the MLS Full Forward Model could compute the Jacobian
matrix necessary for retrieval more efficiently than it presently does.

Thanks are due to the seminar \emph{Sensitivity Analysis of Mathematical Models}
on 29 March -- 1 April by Eugene Ustinov.

\section{Mathematical generalities}

Consider a function $F(x_1, \dots, x_n)$.  From the first term of its Taylor
series, its variation due to variation $\delta x_j$ of its arguments is $\delta
F = \sum_j a_j \delta x_j$.  From this it is obvious that the sensitivities of
$F$ due to variations in its arguments are the partial derivatives

\begin{equation}
\frac{\partial F}{\partial x_j} = a_j\,.
\end{equation}

If we carry this over to a functional $F[X(\xi)]$ of a continuum $X(\xi)$
instead a function of the finite set $\{x_1,\dots,x_n\}$, \emph{viz.}

\begin{equation}
F[X(\xi)] = \int_{\xi_0}^{\xi_1} a(\xi) X(\xi) \,\text{d} \xi
\end{equation}

then its variation due to variation $\delta X(\xi)$ of its argument function
(also known as ``continuous parameter'') $X(\xi)$ is

\begin{equation}\label{VAR}
\delta F[X(\xi)] = \int_{\xi_0}^{\xi_1} a(\xi) \,\delta X(\xi) \,\text{d} \xi
\end{equation}

and its sensitivity due to variation $\delta X(\xi)$ of its argument function
$X(\xi)$ is

\begin{equation}\label{SENS}
\frac{\delta F[X(\xi)]}{\delta X(\xi)} = a(\xi)
\end{equation}

(replace the integral in Equation (\ref{VAR}) by a quadrature, or see
pages 27-29 of {\bf Calculus of Variations} by I. M. Gelfand and S. V.
Fomin).

\section{Radiance}

The radiance is the solution of the radiative transfer equation.  It is usually
written in the form

\begin{equation}\label{RAD}
I[\alpha(\zeta)] = -\int_0^{\zeta_0} B(\zeta) \,\text{d}\tau(\zeta) +
 \tau(\zeta_0) \Upsilon B_s
\end{equation}

as a functional of the function $\alpha(\zeta)$, which can be transformed by
integrating by parts to

\begin{equation}\begin{split}\label{I}
I[\alpha(\zeta)] =\,& B(0) + \int_0^{\zeta_0} \tau(\zeta) \,\text{d} B(\zeta) +
 \tau(\zeta_0) \left[ \Upsilon B_s - B(\zeta_0) \right] \\
=\,& B(0) + \int_0^{\zeta_0} \tau(\zeta) \,\frac{\text{d} B(\zeta)}{\text{d} s}
 \frac{\text{d} s}{\text{d} h} \frac{\text{d} h}{\text{d} \zeta}
 \,\text{d} \zeta +
 \tau(\zeta_0) \left[ \Upsilon B_s - B(\zeta_0) \right]
\end{split}\end{equation}

where $\frac{\text{d} s}{\text{d} h}$ and $\frac{\text{d} h}{\text{d} \zeta}$
appear in Equation (10.11) on page 45 of the 19 August 2004 ATBD,

\begin{equation}
\tau(\zeta) = \exp \left( -\int_0^\zeta \alpha(\zeta^\prime) \,\text{d} \zeta^\prime \right)\,,
\end{equation}

$B(\zeta)$ is the atmospheric Planck function, $\Upsilon$ is the surface
emissivity, and $B_s$ is the surface Planck function.  (This is the form for
nadir sounding; limb sounding has space radiance instead of surface radiance.) 
Henceforth, $I[\alpha(\zeta)]$ will be abbreviated to $I$.

\section{Sensitivity of radiance to variation of $\alpha(\zeta)$}

Since $B$ doesn't depend upon $\alpha(\zeta)$, the variation of $I$ as a result
of variation of $\alpha(\zeta)$ is

\begin{equation}\label{I var}
\delta_\alpha I = \int_0^{\zeta_0} \delta_\alpha \tau(\zeta) \,\text{d} B(\zeta) +
 \delta_\alpha \tau(\zeta_0) \left[ \Upsilon B_s - B(\zeta_0) \right]
\end{equation}

where

\begin{equation}\label{tau var}
\delta_\alpha \tau(\zeta) =
 \exp \left( - \int_0^\zeta \alpha(\zeta^\prime) \,\text{d} \zeta^\prime \right)
 \cdot
 \left( - \int_0^\zeta \delta \alpha(\zeta^\prime) \,\text{d} \zeta^\prime \right)
=
 -\tau(\zeta) \int_0^\zeta \delta \alpha(\zeta^\prime) \,\text{d} \zeta^\prime\,.
\end{equation}

Substituting Equation (\ref{tau var}) into the first term for $\delta_\alpha I$
in Equation (\ref{I var}) gives (exchanging the order of integration in the
last step)

\begin{equation}\label{var1}
\int_0^{\zeta_0} \delta_\alpha \tau(\zeta) \,\text{d} B(\zeta)
=
 - \int_0^{\zeta_0}\text{d} B(\zeta)\, \tau(\zeta)
    \int_0^\zeta \text{d} \zeta^\prime\, \delta \alpha(\zeta^\prime)
=
- \int_0^{\zeta_0} \text{d} \zeta\, \delta \alpha(\zeta)
   \int_\zeta^{\zeta_0} \text{d} B(\zeta^\prime)\, \tau(\zeta^\prime)\,.
\end{equation}

Substituting Equation (\ref{tau var}) into the second term for $\delta_\alpha
I$ in Equation (\ref{I var}) gives

\begin{equation}\label{var2}
\delta_\alpha \tau(\zeta_0) \left[ \Upsilon B_s - B(\zeta_0) \right]
= -\tau(\zeta_0) \left[ \Upsilon B_s - B(\zeta_0) \right]
   \int_0^{\zeta_0}\text{d} \zeta\, \delta \alpha(\zeta)\,.
\end{equation}

Collecting Equations (\ref{var1}) and (\ref{var2}) and factoring, the
resulting variation in $I$ due to variation in $\alpha(\zeta)$ is

\begin{equation}
\delta_\alpha I = -\int_0^{\zeta_0} \text{d} \zeta \,\delta \alpha(\zeta)
 \left\{ \int_\zeta^{\zeta_0} \tau(\zeta^\prime) \,\text{d} B(\zeta^\prime)
 + \tau(\zeta_0) \left[ \Upsilon B_s - B(\zeta_0) \right] \right\}\,.
\end{equation}

According to Equations (\ref{VAR}) and (\ref{SENS}), the sensitivity of $I$ to
variation $\delta \alpha(\zeta)$ in $\alpha(\zeta)$ is the factor between \{
and \}, \emph{viz.}

\begin{equation}\label{I var alpha}
\frac{\delta I}{\delta \alpha(\zeta)} = -
 \left\{ \int_\zeta^{\zeta_0} \tau(\zeta^\prime) \,\text{d} B(\zeta^\prime) +
  \tau(\zeta_0) \left[ \Upsilon B_s - B(\zeta_0) \right] \right\}\,.
\end{equation}

From this and

\begin{equation}
\alpha(\zeta) = \sum_{k=1}^{N_s} f^k(\zeta) \beta^k(\zeta)
\text{, from which }
\frac{\partial \alpha(\zeta)}{\partial f^k(\zeta)} = \beta^k(\zeta)\,,
\end{equation}

where $N_s$ is the number of species, $f^k(\zeta)$ is the volume mixing ratio
of the $k^\text{th}$ species at $\zeta$, and $\beta^k(\zeta)$ is the absorption
cross section for the $k^\text{th}$ species at $\zeta$, we can compute the
sensitivity of $I$ to variation of $f^k(\zeta)$ as

\begin{equation}
\frac{\delta I}{\delta f^k(\zeta)} =
 \frac{\delta I}{\delta \alpha(\zeta)} \frac{\partial \alpha(\zeta)}{\partial f^k(\zeta)} =
 \frac{\delta I}{\delta \alpha(\zeta)} \beta^k(\zeta)\,.
\end{equation}

Notice that the integrand in Equation (\ref{I var alpha}) is the same as the
integrand in Equation (\ref{I}).  In fact,

\begin{equation}
I = B(0) - \frac{\delta I}{\delta \alpha(0)}\,.
\end{equation}

This means that while we approximate the integral in Equation (\ref{I}) as

\begin{equation}\label{sum}
\int_0^{\zeta_0} \tau(\zeta) \,\text{d} B(\zeta) \approx
 \sum_{i=1}^{N_p} \tau(\zeta_i) \Delta B(\zeta_i)
\end{equation}

we can at the same time approximate the integrals in Equation (\ref{I var
alpha}) for $\zeta \in \{\zeta_j | \, j = 1, \dots, N_p\}$ as

\begin{equation}\label{var sum}
\int_{\zeta_j}^{\zeta_0} \tau(\zeta) \,\text{d} B(\zeta) \approx
 \sum_{i=j}^{N_p} \tau(\zeta_i) \Delta B(\zeta_i)\,,\, j = 1, \dots, N_p\,.
\end{equation}

That is, we can evaluate both $I$ and its sensitivity to variation in the
discrete parameters $\alpha(\zeta_j)$ by evaluating one indefinite sum.  It
might seem that we need to evaluate $N_p+1$ sums, but the sums in Equation
(\ref{var sum}) can be organized as a single indefinite sum taken in the order
$N_p$, $N_p-1$, \dots, 1 if we save the terms $\tau(\zeta_i) \Delta B(\zeta_i)$
or evaluate them in that order; the last of the sums in Equation (\ref{var
sum}) is the sum in Equation (\ref{sum}).  We can therefore evaluate both $I$
and its sensitivity to variations in the mixing ratios of $N_s$ species at
$N_p$ points by evaluating one indefinite sum, followed by $N_p \times N_s$
multiplies by the $N_s$ absorption cross sections $\{\beta^k(\zeta_j)|\,
k=1,\dots,N_s,\, j=1,\dots,N_p\}$ at the $N_p$ points on the integration path.

In the present scheme, we evaluate $N_s$ indefinite sums because we leave
$\frac{\partial\alpha(\zeta)}{\partial f^k(\zeta)}$ inside the integral. Thus
the approach described here should be less time consuming and require less
memory than evaluating $N_s$ sums as the current method does.

Evaluating the Hessian tensor should similarly be possible without integrating
variational equations.

\section{Sensitivity of radiance to variation of temperature}

Rewrite Equation (\ref{RAD}) in the form

\begin{equation}
I[G(\zeta)] = -\int_0^{\zeta_0} G(\zeta)
 \frac{\text{d} s(\zeta)}{\text{d} h}
 \,\text{d} \zeta +
 \tau(\zeta_0) \Upsilon B_s
 \text{ where }
 G = B(\zeta) \frac{\text{d} h}{\text{d} \zeta}
  \frac{\text{d} \tau(\zeta)}{\text{d} s}
\end{equation}

because all of the factors of $G(\zeta)$ depend upon $T$, but $\frac{\text{d}
s(\zeta)}{\text{d} h}$ does not.  Then

\begin{equation}
\delta_G I = -\int_0^{\zeta_0} \delta G(\zeta)
 \frac{\text{d} s(\zeta)}{\text{d} h}
 \,\text{d} \zeta
\end{equation}

from which by Equation (\ref{SENS})

\begin{equation}
\frac{\delta I}{\delta G(\zeta)} =
 - \frac{\text{d} s(\zeta)}{\text{d} h}\,.
\end{equation}

Then

\begin{equation}
\frac{\delta I}{\delta T(\zeta)} =
 \frac{\delta I}{\delta \alpha(\zeta)} \frac{\partial \alpha(z)}{\partial T} +
 \frac{\delta I}{\delta G(\zeta)} \frac{\partial G(\zeta)}{\partial T} +
 \frac{\partial}{\partial T} \left[ \tau(\zeta_0) \Upsilon B_s \right]
\end{equation}

where $\frac{\delta I}{\delta \alpha(\zeta)}$ is computed using Equation
(\ref{I var alpha}).  Thus we can compute both $\frac{\partial I}{\partial
f_i^k}$ and $\frac{\partial I}{\partial T_i}$ by evaluating Equation (\ref{var
sum}) only once, as an indefinite sum for $j = 1, \dots, N_p$.

\label{lastpage}
\end{document}

% $Log$
