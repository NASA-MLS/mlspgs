\documentclass[11pt]{article}
\usepackage{alltt}
\usepackage[fleqn]{amsmath}
\usepackage{floatflt}
\usepackage{graphicx}
\usepackage{longtable}
\usepackage[strings]{underscore}

\textwidth 6.5in
\oddsidemargin -0.25in
%\evensidemargin -0.5in
\topmargin -0.5in
\textheight 9in

\newcommand{\docname}{wvs-148r2}
\newcommand{\docdate}{27 November 2017}

\ifx\pdfoutput\undefined
  \pdfoutput=0
  \usepackage[hypertex,plainpages,hyperindex=true]{hyperref}
  \hypersetup{%
    hypertexnames=false%
  }
  % Specify the driver for the color package
  \ExecuteOptions{dvips}
  %\ExecuteOptions{xdvi}
\else
  \ifnum\pdfoutput>0
    \usepackage[pdftex,plainpages,hyperindex=true,pdfpagelabels]{hyperref}
    \hypersetup{%
      hypertexnames=false,%
      colorlinks=true,%
      linktocpage=true,%
    }
    % Specify the driver for the color package
    \ExecuteOptions{pdftex}
  \else
    \usepackage[hypertex,plainpages,hyperindex=true]{hyperref}
    \hypersetup{%
      hypertexnames=false%
    }
    % Specify the driver for the color package
    \ExecuteOptions{dvips}
    %\ExecuteOptions{xdvi}
  \fi
\fi

\hyperbaseurl{}
\newcommand\hr[1]{\href{#1.dvi}{dvi}, \href{#1.pdf}{pdf}}
\newcommand\h[1]{#1 (\hr{#1})}

\begin{document}

%\tracingcommands=1
\newlength{\hW} % heading box width
\newlength{\pW} % page number field width
\settowidth{\hW}{\bf\docname}
\settowidth{\pW}{Page \pageref{lastpage}\ of \pageref{lastpage}}
\ifdim \pW > \hW \setlength{\hW}{\pW} \fi
\makeatletter
\def\@biblabel#1{#1.}
\newcommand{\ps@twolines}{%
  \renewcommand{\@oddhead}{%
    \docdate\hfill\parbox[t]{\hW}{{\hfill\bf\docname}\newline
                          Page \thepage\ of \pageref{lastpage}}}%
\renewcommand{\@evenhead}{}%
\renewcommand{\@oddfoot}{}%
\renewcommand{\@evenfoot}{}%
}%
\makeatother
\pagestyle{twolines}

\vspace{-10pt}
\begin{tabbing}
\phantom{References: }\= \\
To: \>Nathaniel, Bill\\
Subject: \>Alternative to state-vector representation\\
From: \>Van Snyder\\
% Reference: \> \\
\end{tabbing}

\parindent 0pt \parskip 6pt
\vspace{-20pt}

Both Fred Krogh and Eugene Ustinov have suggested that if the state vector
were represented by an expansion in smooth functions, the cost of
retrieval might be reduced, and the size of representation would almost
certainly be reduced.

An example of a basis for this representation is to use spherical
harmonics in horizontal directions, and Chebyshev polynomials in $\zeta =
-\log_{10}\,P$.  If there is a substantial ``kink'' in the vertical
direction at the tropopause and stratopause that is not well represented
by Chebyshev polynomials, cubic splines in $\zeta$ could be used, with
solution for the locations of the knots below and above the tropopause and
stratopause.  Solving for coefficients in the latter case is more
complicated, but there is the potential for fewer vertical expansion
coefficients.

Assuming Chebyshev polynomials in $\zeta$ for vertical representation, the
form of representation of a geophysical product on the path of integration
would be

\begin{equation}\begin{split}
f^k(s) = \,& \sum_{ij\ell}
              \left( \eta^{ck}_{ij\ell} Y^{ic}_j ( \theta(s), \phi(s) ) +
                     \eta^{sk}_{ij\ell} Y^{is}_j ( \theta(s), \phi(s) )
              \right) C_\ell(\zeta(s)) \text{ and} \\
 \,& \\
T(s) = \,& \sum_{ij\ell}
              \left( \eta^{cT}_{ij\ell} Y^{ic}_j ( \theta(s), \phi(s) ) +
                     \eta^{sT}_{ij\ell} Y^{is}_j ( \theta(s), \phi(s) )
              \right) C_\ell(\zeta(s)) \\
\end{split}\end{equation}

wherein the superscipts $c$ and $s$ indicate the real (cosine) and
imaginary (sine) parts of the spherical harmonic, $k$ is a species index,
$T$ is temperature, $\theta$ is geocentric colatitude, and $\phi$ is
longitude.  The argument $s$ is path length, and the superscript $s$ is
not path length.  Unlike the representation currently used, the expansion
coefficients do not depend upon the path position.

At the expense of some additional complication in the computation of basis
functions, Lam\'e functions (ellipsoidal harmonics) could be used.  To
represent the gravitational potential, this has resulted in a 16-fold
reduction in the number of coefficients.  For MLS, it would almost
certainly result in fewer expansion coefficients than the use of spherical
harmonics.

Spherical or ellipsoidal harmonics are useful only for entire-Earth
representations.  A local representation might use a product of three
Chebychev polynomials in the SMLS case -- one for each horizontal
direction, and one for the vertical direction, or two Chebychev
polynomials in the EMLS case -- one for the horizontal direction (along
the orbit track) and one for the vertical direction.

\begin{equation}\begin{array}{ll}
f^k(s) = \sum_{ij\ell} \eta^k_{ij\ell}\, C_i(\phi(s))\, C_j(\theta(s))\,
C_\ell(\zeta(s)) & \phi \text{ is longitude, } \theta \text{ is latitude,
}\zeta\\
\text{or} \\[5pt]
f^k(s) = \sum_{i\ell} \eta^k_{i\ell}\, C_i(\phi(s))\,
C_\ell(\zeta(s)) & \phi \text{ is orbit angle, } \zeta\\
\end{array}\end{equation}

where $C_i$ etc.\ are Chebyshev polynomials.  Chebyshev polynomials are
attractive because an error estimate can be calculated from the first
neglected term (or the last used one).

For any smooth representation, one would solve for the coefficients of the
expansions instead of the values of the geophysical products.  Since the
basis is smooth, the radiative transfer equation and its variational
equations could be integrated by an adaptive ODE solver or two-dimensional
quadrature (because the solution of the radiative-transfer equation is an
integral of the exponential function of an integral), instead of being
integrated on a fixed grid dictated by the intersections of the line of
sight with surfaces of constant $\zeta$, resulting in fewer integrand
evaluations.  Costs would also be reduced because there would be fewer
coefficients of a smooth representation than interpolating coefficients
for an explicit state-vector representation.  Assuming the same expansion
orders and the same integration steps for all products, the basis
functions would only need to be evaluated once for all products at each
point along the path of integration.

Lines of sight could be chosen to correspond to MIFs, with the set of MIFs
selected for the most information using EOF.

Derivatives of radiance with respect to expansion coefficients would be
computed using the chain rule:

\begin{equation} \begin{split}
\frac{\partial I(s)}{\partial \eta^{ck}_{ij\ell}} = \,&
\frac{\partial I(s)}{\partial f^k(s)}
 \frac{\partial f^k(s)}{\partial \eta^{ck}_{ij\ell}} =
\frac{\partial I(s)}{\partial f^k(s)}\,
 Y^{ic}_j ( \theta(s), \phi(s))\, C_\ell(\zeta(s)) \,\,\, \text{ and}\\
\frac{\partial I(s)}{\partial \eta^{sk}_{ij\ell}} = \,&
\frac{\partial I(s)}{\partial f^k(s)}
 \frac{\partial f^k(s)}{\partial \eta^{sk}_{ij\ell}} =
\frac{\partial I(s)}{\partial f^k(s)}\,
 Y^{ic}_j ( \theta(s), \phi(s))\, C_\ell(\zeta(s))\\
\text{or} \,& \\
\frac{\partial I(s)}{\partial \eta^k_{ij\ell}} = \,&
\frac{\partial I(s)}{\partial f^k(s)}
 \frac{\partial f^k(s)}{\partial \eta^k_{ij\ell}} =
\frac{\partial I(s)}{\partial f^k(s)}\,
 C_i(\phi(s))\, C_j(\theta(s))\, C_\ell(\zeta(s)) \\
\text{or} \,& \\
\frac{\partial I(s)}{\partial \eta^k_{i\ell}} = \,&
\frac{\partial I(s)}{\partial f^k(s)}
 \frac{\partial f^k(s)}{\partial \eta^k_{i\ell}} =
\frac{\partial I(s)}{\partial f^k(s)}\,
 C_i(\phi(s))\, C_\ell(\zeta(s)) \\
\end{split}\end{equation}

and similarly for temperature coefficients.  The $Y^{ic}_j ( \theta(s),
\phi(s))\, C_\ell(\zeta(s))$ or $C_i(\phi(s))\, C_j(\theta(s))\,
C_\ell(\zeta(s))$ or $C_i(\phi(s))\, C_\ell(\zeta(s))$ factor is the same
for every species and temperature.

Once the expansion coefficients are in hand, values of the geophysical
products could be computed on any specified grid by evaluating the basis
functions for specified values of $(\phi, \theta, \zeta)$ or $(\phi,
\zeta)$.

Because the integration path and the geometry of the final representation
are independent, the ``metrics'' problem in the present forward model
would be eliminated.

In the present forward model, the effect of refraction is approximated by
``stretching'' the path by increasing the difference of the $\phi$
co\"ordinate away from the tangent point.  Using a smooth representation,
the effect of refraction could be accomodated by representing the
geometric path using a smooth function instead of a straight line.

\label{lastpage}
\vspace*{-0.1in} % Somehow, this causes lastpage to be defined
\end{document}

% $Id$

% $Log$
% Revision 1.2  2017/11/28 01:33:32  vsnyder
% Repair blunder in Equation 3
%
% Revision 1.1  2017/11/28 01:29:12  vsnyder
% Initial commit
%
