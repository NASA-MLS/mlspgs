\fillMethod{Profile}

\index{Interpolation}

\begin{cfFragment}
Fill, quantity=<vector>.<quantity>, method=profile, 
   profileValues=[ <height> : <value>, ... ]
   {, instances=[ <instances>, ... ] }
   {, logSpace=<true|false> }
   {, dontMask=<true|false> }
\end{cfFragment}
\cfField{logSpace}
\cfField{profileValues}
\cfField{dontMask}

This fill method allows the user to simply fill a quantity with an
interpolated version of a coarse vertical profile.  It is probably most
useful for quantities such as \emph{a priori} precisions and smoothing
weights.  Consider a simple example:

\begin{cfFragment}
Fill, quantity=aprioriPrecision.temperature, method=profile,
  profileValues=[ 1000mb:5K, 220mb:10K, 68mb:20K ]
\end{cfFragment}

This will fill all the instances of the given quantity with a linear
vertical interpolation (in log pressure) between the values at the tie
points indicated in the \cf+profileValues+ argument.  If the user wishes
to apply the fill only to selected instances of the quantity, the
\cf+instances+ argument may be supplied.  The interpolation is performed
in log space for those quantities whose \cf+logBasis+ flag is set on
template creation, and linear space otherwise.  To overwrite that
behavior set the \cf+logSpace+ flag appropriately.

\fillMaskComment

%%% Local Variables: 
%%% mode: latex
%%% TeX-master: "l2ug"
%%% End: 
