\fillMethod{L2GP}

\index{L2GP!filling from}
\index{L2PC!intelligent}

\begin{cfFragment}
Fill, quantity=<vector>.<quantity>, method=l2gp,
   sourceL2GP=<l2gp>
   [, interpolate=<true|false>]
   [, profile=<profile number>]
\end{cfFragment}
\index{interpolate}
\index{profile}

This method fills a vector quantity with the values of an l2gp,
typically read in an earlier `Read a priori' section of the l2cf.  This
method can only be used to fill coherent, stacked quantities on pressure
surfaces.

The program checks the l2gp data to find which profiles correspond to
those to fill in the vector and raises an error if it cannot find a
match for all the profiles.  This can give problems on occasions,
particularly at the start and end of the input dataset (typically day
boundaries).  Setting the \cf+forbidOverspill+ option on the definition
of the hGrid for the quantity may alleviate such problems.  On rare
occasions, one might want to fill every profile in the vector with data
from a specific profile in the l2gp file (this is the case when setting
up `intelligent' l2pc files for example).  To do this, one can supply
the \cf+profile+ argument.

The program also checks that the vertical coordinates of the quantity
match those in the l2pc file, and raises an error if they do not.  To
bypass this check and to have the program vertically interpolate the
data (in log pressure) from the l2gp surfaces to those in the vector,
set the \cf+interpolate+ flag.  Note that for those quantities who's
\cf+logBasis+ flag is set on creation the interpolation is made in log
space.




%%% Local Variables: 
%%% mode: latex
%%% TeX-master: "l2ug"
%%% End: 
