\fillMethod{L1B}

\index{Radiances!reading}

\begin{cfFragment}
Fill, quantity=<vector>.<quantity>, method=l1b 
   [, isPrecision=<true|false>]
   [, precisionQuantity=<vector>.<quantity>]
\end{cfFragment}
\cfField{isPrecision}
\cfField{precisionQuantity}

This fill method can only be applied to a limited set of vector
quantities.  The quantity is filled with data read directly from the
appropriate field in the input L1B data set.

Special options can apply when reading radiance and radiance quantities.
The \cf+isPrecision+ flag should be set to read the precision for a
radiance quantity instead of the radiance itself from the L1B files.
When reading the radiances (\ie \cf+isPrecision+ not set), one can supply
the \cf+precisionQuantity+ to the fill command.  In this case, the
supplied precision quantity (presumably filled earlier) is inspected,
and where it is negative, the corresponding radiance quantity is masked
with the `linAlg' flag.  It is by this mechanism that the level~2
software is told not to use radiances who's corresponding errors have
been set negative by level~1.  Naturally the quantity to fill and the
precision quantity must describe the same signal and sideband.

%%% Local Variables: 
%%% mode: latex
%%% TeX-master: "l2ug"
%%% End: 
