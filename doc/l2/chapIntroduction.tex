\chapter{Introduction}
\rcsInfo $Id$

The EOS Microwave Limb Sounder (MLS) is one of four instruments on the
EOS Aura platform, to be launched in early 2004.  EOS MLS observes the
thermal microwave emission from the Earth's limb in several spectral
regions around 118, 190, 240, 640 and 2520\,GHz.  More details of EOS
MLS are given in \citet{EOSMLSOverview}.  This document describes the
`Level~2' data processing software, known as \cf+mlsl2+.

\section{Overview of this document}

This chapter of the document describes the role of the software in the
MLS data processing, and introduces the various data types and
operations the software deals with.

The software can be run in two modes. One mode is aimed at production
processing where the EOS Science Data Processing (SDP) toolkit is used
to manage many tasks such as input and output.  The other mode is
intended for `general users' who do not want to deal with the additional
complexity that comes from using the toolkit.  The details of both of
these modes are given in Chapter~\ref{chap:Running}, which also includes
a discussion of issues such as command line arguments to the level~2
software.  The software can be run in a variety of `parallel processing'
modes.  These are also detailed in Chapter~\ref{chap:Running}.

Arguably the most complex and powerful aspect of the software is the
`Level~2 Configuration File' (L2CF).  This is in effect a high level
language that is used to describe the operations of the software.  The
description of this forms the main bulk of this document in
Chapters~\ref{chap:L2CFIntro} and~\ref{chap:L2CFRef}.

Typical runs of the software involve many hundreds of L2CF instructions,
a lot of which are somewhat verbose and repetitive.  In order to shield
the typical user from the complexity and tedium of such complicated L2CF
files, we use the GNU macro preprocessor \cf+m4+ to automate the
creation of `complete' L2CFs from smaller building blocks, allowing
typical users to work on short (a hundred or so line) L2CF files.  The
details of our use of \cf+m4+ for this task are given in
Chapter~\ref{chap:m4}.

The software can be run in tandem with a separate suite of IDL programs
that in many ways act as a `graphical debugger'.  Details of this
process, known as \emph{snooping}, are given in
Chapter~\ref{chap:Snoop}.

\section{The role of the level~2 software}

The EOS MLS data, in common with other satellite observations are
divided into distinct `levels'.  Level~0 data describe raw instrument
telemetry.  Level~1 data\footnote{Note that unlike some instruments, EOS
  MLS does not have distinct Level 1A and 1B data} describe geolocated
and calibrated observations of radiance. Level~2 data are retrieved
geophysical parameters at locations corresponding to the measurement
footprint of the instrument. Level~3 data are retrieved geophysical
parameters on some fixed horizontal grid.

The level~2 data are generated by the level~2 software (the subject of
this document) which takes the calibrated radiances from the level~1
data and applies the `Retrieval theory' methods \citep[for
example]{Rodgers76} to obtain vertical profiles of geophysical
parameters.  Full details of the theoretical aspects of the MLS level~2
software are given in \citet{EOSMLSL2ATBD,EOSMLSFWMATBD}.

Comment on the flexibility of the software

\section{Data flows to and from the level~2 software}

\section{Breakdown of the level~2 software}



%%% Local Variables: 
%%% mode: latex
%%% TeX-master: "l2ug"
%%% End: 
