\cfCommand{Merge}{MergeGrids}
\index{Gridded data!merging}
\index{Gridded data!climatology}
\index{Gridded data!operational}
\index{Gridded data!missing points in}

\begin{cfFragment}
<newGrid>: Merge, climatology=<grid>, operational=<grid>, 
   height=<pressure>, scale=<length>
\end{cfFragment}
\cfField{climatology}
\cfField{operational}
\cfField{height}
\cfField{scale}

The \cf+Merge+ command is used to combine gridded data from a climatology
dataset with that from operational datasets (\eg NCEP or GMAO/DAO).
Typically the climatology dataset is something like a series of zonal
means for `typical' months, while the operational data are on some much
finer temporal and horizontal grid.  However, there are no strict
requirements, other than that both datasets must be on the use the same
latitude coordinate (latitude vs.\ equivalent latitude) and use pressure
as their vertical coordinate.

The resulting grid takes its horizontal coordinates from the operational
grid, and its vertical coordinates from the climatology dataset.  The
result is majority operational data below the pressure given in the
\cf+height+ field, and mostly climatology above.  The `smoothing' of
this transition is controlled by the \cf+scale+ length.  For points
where the operational data are marked bad, the climatological data are
returned (with no attempt made to `smooth' transitions).

If the operational data are completely missing, the result is simply a
copy of the climatology dataset.  The climatology data may not contain
any missing data points.

%%% Local Variables: 
%%% mode: latex
%%% TeX-master: "l2ug"
%%% End: 
