\fillMethod{Vector}

\index{Interpolation}
\index{Quantities!Copying / filling from}

\begin{cfFragment}
Fill, quantity=<vector>.<quantity>, method=vector, 
   sourceQuantity=<vector>.<quantity>, 
   {, interpolate=<true|false>}
   {, force=<true|false>}
   {, dontMask=<true|false>}
\end{cfFragment}
\cfField{sourceQuantity}
\cfField{interpolate}
\cfField{force}
\cfField{dontMask}

This fill method in it's simplest form allows the user to fill a vector
quantity with the values of another, `similar' vector quantity.  How
`similar' the quantities are required to be depends on the user
requirements.  By default, the software insists that the quantities
share the same template.  If, however, the user sets the
\cf+interpolate+ flag, then the quantities can have different vertical
coordinates, provided that they share horizontal and frequency
coordinates, and describe the same physical quantity (molecule / band
etc.).  If the user sets the \cf+force+ option, the quantities must only
share the same horizontal and frequency coordinate system.

Currently the software only supports vertical interpolation of
single channel quantities (even when \cf+force+ is set).  This may be
changed at some later date if such functionality is required.

\fillMaskComment

%%% Local Variables: 
%%% mode: latex
%%% TeX-master: "l2ug"
%%% End: 
