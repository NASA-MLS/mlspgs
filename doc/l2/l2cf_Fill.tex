\cfCommand{Fill}{Fill}

\begin{cfFragment}
Fill, quantity=<vector>.<quantity>, method=<fill method>, <parameters>
\end{cfFragment}

Fill is one of the most complex commands in the level~2 configuration
file.  It is used to fill the values of one quantity within a vector.
The data placed in the quantity can come from a variety of sources.
These are distinguished by the `fill method'.

Some but not all of the fill methods obey the `fill' mask flag.  Those
that do, typically also take the \cf+dontMask+ option, which indicates
that the fill mask flag should be ignored.

The rest of this section breaks down the various fill methods in detail.

% Each fill method is described in alphabetical order.  If you add more
% methods please do so in a separate file as done below, and in the
% right place.

\newcommand{\fillMaskComment}{This command does obeys the \cf+fill+
  mask, so the fill can be applied to a subset of the quantity (\eg a
  restricted height range) by issuing a \cf+Subset+ command first.  The
  \cf+dontMask+ flag instructs the program to ignore the mask.}

\input{l2cf_Fill_BinMax.tex}
\input{l2cf_Fill_BinMin.tex}
\fillMethod{Gridded}

\index{Gridded data!filling from}

\begin{cfFragment}
Fill, quantity=<vector>.<quantity>, method=gridded,
   sourceGrid=<grid>
   {, allowMissing=<true|false> }
\end{cfFragment}
\cfField{sourceGrid}
\cfField{allowMissing}

This fill method fills a vector quantity by interpolating a gridded
dataset in all the appropriate directions (temporal, horizontal,
vertical, solar time / zenith).  The grid would have been defined by a
\cf+gridded+ command in the \cf+readApriori+ section, or by a \cf+merge+
instruction in the \cf+mergeGrids+ section.  The vertical coordinate
types for the grid and the quantity have to match.  No other checking is
done to ensure that the fill is appropriate (\eg that you're not trying
to fill a temperature quantity with H\cs2O data).

By default the program will raise an error if any of the input gridded
data values used in the interpolation are missing or bad.  If the user
wishes to allow missing values in the resulting vector quantity (not
recommended), then set the \cf+allowMissing+ option.


%%% Local Variables: 
%%% mode: latex
%%% TeX-master: "l2ug"
%%% End: 

\fillMethod{L1B}

\index{Radiances!reading}

\begin{cfFragment}
Fill, quantity=<vector>.<quantity>, method=l1b 
   [, isPrecision=<true|false>]
   [, precisionQuantity=<vector>.<quantity>]
\end{cfFragment}
\cfField{isPrecision}
\cfField{precisionQuantity}

This fill method can only be applied to a limited set of vector
quantities.  The quantity is filled with data read directly from the
appropriate field in the input L1B data set.

Special options can apply when reading radiance and radiance quantities.
The \cf+isPrecision+ flag should be set to read the precision for a
radiance quantity instead of the radiance itself from the L1B files.
When reading the radiances (\ie \cf+isPrecision+ not set), one can supply
the \cf+precisionQuantity+ to the fill command.  In this case, the
supplied precision quantity (presumably filled earlier) is inspected,
and where it is negative, the corresponding radiance quantity is masked
with the `linAlg' flag.  It is by this mechanism that the level~2
software is told not to use radiances who's corresponding errors have
been set negative by level~1.  Naturally the quantity to fill and the
precision quantity must describe the same signal and sideband.

%%% Local Variables: 
%%% mode: latex
%%% TeX-master: "l2ug"
%%% End: 

\fillMethod{L2GP}

\index{L2GP!filling from}
\index{L2PC!intelligent}

\begin{cfFragment}
Fill, quantity=<vector>.<quantity>, method=l2gp,
   sourceL2GP=<l2gp>
   [, interpolate=<true|false>]
   [, profile=<profile number>]
\end{cfFragment}
\index{interpolate}
\index{profile}

This method fills a vector quantity with the values of an l2gp,
typically read in an earlier `Read a priori' section of the l2cf.  This
method can only be used to fill coherent, stacked quantities on pressure
surfaces.

The program checks the l2gp data to find which profiles correspond to
those to fill in the vector and raises an error if it cannot find a
match for all the profiles.  This can give problems on occasions,
particularly at the start and end of the input dataset (typically day
boundaries).  Setting the \cf+forbidOverspill+ option on the definition
of the hGrid for the quantity may alleviate such problems.  On rare
occasions, one might want to fill every profile in the vector with data
from a specific profile in the l2gp file (this is the case when setting
up `intelligent' l2pc files for example).  To do this, one can supply
the \cf+profile+ argument.

The program also checks that the vertical coordinates of the quantity
match those in the l2pc file, and raises an error if they do not.  To
bypass this check and to have the program vertically interpolate the
data (in log pressure) from the l2gp surfaces to those in the vector,
set the \cf+interpolate+ flag.  Note that for those quantities who's
\cf+logBasis+ flag is set on creation the interpolation is made in log
space.




%%% Local Variables: 
%%% mode: latex
%%% TeX-master: "l2ug"
%%% End: 

\fillMethod{Profile}

\index{Interpolation}

\begin{cfFragment}
Fill, quantity=<vector>.<quantity>, method=profile, 
   profileValues=[ <height> : <value>, ... ]
   {, instances=[ <instances>, ... ] }
   {, logSpace=<true|false> }
   {, dontMask=<true|false> }
\end{cfFragment}
\cfField{logSpace}
\cfField{profileValues}
\cfField{dontMask}

This fill method allows the user to simply fill a quantity with an
interpolated version of a coarse vertical profile.  It is probably most
useful for quantities such as \emph{a priori} precisions and smoothing
weights.  Consider a simple example:

\begin{cfFragment}
Fill, quantity=aprioriPrecision.temperature, method=profile,
  profileValues=[ 1000mb:5K, 220mb:10K, 68mb:20K ]
\end{cfFragment}

This will fill all the instances of the given quantity with a linear
vertical interpolation (in log pressure) between the values at the tie
points indicated in the \cf+profileValues+ argument.  If the user wishes
to apply the fill only to selected instances of the quantity, the
\cf+instances+ argument may be supplied.  The interpolation is performed
in log space for those quantities whose \cf+logBasis+ flag is set on
template creation, and linear space otherwise.  To overwrite that
behavior set the \cf+logSpace+ flag appropriately.

\fillMaskComment

%%% Local Variables: 
%%% mode: latex
%%% TeX-master: "l2ug"
%%% End: 

\fillMethod{ScaleOverlaps}

\index{Overlaps!Scaling}

\begin{cfFragment}
Fill, quantity=<vector>.<quantity>, method=scaleOverlaps,
   multiplier= [ <numeric value> , ... ]
   {, dontMask=<true|false> }
\end{cfFragment}
\cfField{multiplier}
\cfField{dontMask}

This fill method is somewhat unusual in that it modifies a quantities
values rather than overwriting them.  It is probably best described by
example:

\begin{cfFragment}
Fill, quantity=aprioriPrecision.temperature, method=scaleOverlaps, 
   multiplier=[0.1, 0.4, 0.8]
\end{cfFragment}


This will `tighten up' the \emph{a priori} precision at the edges of the
chunk, assuming that there are overlaps.  So if the \emph{a priori}
precision at 100\,hPa was 10\,K by default, this instruction would
change it to 1\,K, 4\,K, 8\,K, 10\,K, 10\,K, \ldots\ 10\,K, 8\,K, 4\,K,
1\,K.

Note that it will only do this in the overlap regions of the quantity.
In cases where there are no overlaps at one or other end (\eg the start
and end of the day for some configurations), no scaling will occur at
those ends.  If more multipliers are given than there are overlapped
profiles, then it ignores the later multipliers.  For example, if there
were only two profiles of lower overlap and one of upper in the above
example, the result would be 1\,K, 4\,K, 10\,K, \ldots\ 10\,K, 1\,K.

\fillMaskComment

%%% Local Variables: 
%%% mode: latex
%%% TeX-master: "l2ug"
%%% End: 

\fillMethod{Vector}

\index{Interpolation}
\index{Quantities!Copying / filling from}

\begin{cfFragment}
Fill, quantity=<vector>.<quantity>, method=vector, 
   sourceQuantity=<vector>.<quantity>, 
   {, interpolate=<true|false>}
   {, force=<true|false>}
   {, dontMask=<true|false>}
\end{cfFragment}
\cfField{sourceQuantity}
\cfField{interpolate}
\cfField{force}
\cfField{dontMask}

This fill method in it's simplest form allows the user to fill a vector
quantity with the values of another, `similar' vector quantity.  How
`similar' the quantities are required to be depends on the user
requirements.  By default, the software insists that the quantities
share the same template.  If, however, the user sets the
\cf+interpolate+ flag, then the quantities can have different vertical
coordinates, provided that they share horizontal and frequency
coordinates, and describe the same physical quantity (molecule / band
etc.).  If the user sets the \cf+force+ option, the quantities must only
share the same horizontal and frequency coordinate system.

Currently the software only supports vertical interpolation of
single channel quantities (even when \cf+force+ is set).  This may be
changed at some later date if such functionality is required.

\fillMaskComment

%%% Local Variables: 
%%% mode: latex
%%% TeX-master: "l2ug"
%%% End: 


%%% Local Variables: 
%%% mode: latex
%%% TeX-master: "l2ug"
%%% End: 
