\cfCommand{Fill}

Fill is one of the most complex commands in the level~2 configuration
file.  It is used to fill the values of one quantity within a vector.
The data placed in the quantity can come from a variety of sources.
These are distinguished by the `fill method'.  The basic syntax of the
fill command is:

\begin{cfFragment}
Fill, quantity=<vector>.<quantity>, method=<fill method>, <parameters>
\end{cfFragment}

Some but not all of the fill methods obey the `fill' mask flag.  Those
that do, typically also take the \cf+dontMask+ option, which indicates
that the fill mask flag should be ignored.

The rest of this section breaks down the various fill methods in detail.

% Each fill method is described in alphabetical order.  If you add more
% methods please do so in a separate file as done below, and in the
% right place.

\fillMethod{L1B}

\index{Radiances!reading}

\begin{cfFragment}
Fill, quantity=<vector>.<quantity>, method=l1b 
   [, isPrecision=<true|false>]
   [, precisionQuantity=<vector>.<quantity>]
\end{cfFragment}
\cfField{isPrecision}
\cfField{precisionQuantity}

This fill method can only be applied to a limited set of vector
quantities.  The quantity is filled with data read directly from the
appropriate field in the input L1B data set.

Special options can apply when reading radiance and radiance quantities.
The \cf+isPrecision+ flag should be set to read the precision for a
radiance quantity instead of the radiance itself from the L1B files.
When reading the radiances (\ie \cf+isPrecision+ not set), one can supply
the \cf+precisionQuantity+ to the fill command.  In this case, the
supplied precision quantity (presumably filled earlier) is inspected,
and where it is negative, the corresponding radiance quantity is masked
with the `linAlg' flag.  It is by this mechanism that the level~2
software is told not to use radiances who's corresponding errors have
been set negative by level~1.  Naturally the quantity to fill and the
precision quantity must describe the same signal and sideband.

%%% Local Variables: 
%%% mode: latex
%%% TeX-master: "l2ug"
%%% End: 

\fillMethod{L2GP}

\index{L2GP!filling from}
\index{L2PC!intelligent}

\begin{cfFragment}
Fill, quantity=<vector>.<quantity>, method=l2gp,
   sourceL2GP=<l2gp>
   [, interpolate=<true|false>]
   [, profile=<profile number>]
\end{cfFragment}
\index{interpolate}
\index{profile}

This method fills a vector quantity with the values of an l2gp,
typically read in an earlier `Read a priori' section of the l2cf.  This
method can only be used to fill coherent, stacked quantities on pressure
surfaces.

The program checks the l2gp data to find which profiles correspond to
those to fill in the vector and raises an error if it cannot find a
match for all the profiles.  This can give problems on occasions,
particularly at the start and end of the input dataset (typically day
boundaries).  Setting the \cf+forbidOverspill+ option on the definition
of the hGrid for the quantity may alleviate such problems.  On rare
occasions, one might want to fill every profile in the vector with data
from a specific profile in the l2gp file (this is the case when setting
up `intelligent' l2pc files for example).  To do this, one can supply
the \cf+profile+ argument.

The program also checks that the vertical coordinates of the quantity
match those in the l2pc file, and raises an error if they do not.  To
bypass this check and to have the program vertically interpolate the
data (in log pressure) from the l2gp surfaces to those in the vector,
set the \cf+interpolate+ flag.  Note that for those quantities who's
\cf+logBasis+ flag is set on creation the interpolation is made in log
space.




%%% Local Variables: 
%%% mode: latex
%%% TeX-master: "l2ug"
%%% End: 


%%% Local Variables: 
%%% mode: latex
%%% TeX-master: "l2ug"
%%% End: 
