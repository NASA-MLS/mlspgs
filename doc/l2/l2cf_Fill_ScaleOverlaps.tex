\fillMethod{ScaleOverlaps}

\index{Overlaps!Scaling}

\begin{cfFragment}
Fill, quantity=<vector>.<quantity>, method=scaleOverlaps,
   multiplier= [ <numeric value> , ... ]
   {, dontMask=<true|false> }
\end{cfFragment}
\cfField{multiplier}
\cfField{dontMask}

This fill method is somewhat unusual in that it modifies a quantities
values rather than overwriting them.  It is probably best described by
example:

\begin{cfFragment}
Fill, quantity=aprioriPrecision.temperature, method=scaleOverlaps, 
   multiplier=[0.1, 0.4, 0.8]
\end{cfFragment}


This will `tighten up' the \emph{a priori} precision at the edges of the
chunk, assuming that there are overlaps.  So if the \emph{a priori}
precision at 100\,hPa was 10\,K by default, this instruction would
change it to 1\,K, 4\,K, 8\,K, 10\,K, 10\,K, \ldots\ 10\,K, 8\,K, 4\,K,
1\,K.

Note that it will only do this in the overlap regions of the quantity.
In cases where there are no overlaps at one or other end (\eg the start
and end of the day for some configurations), no scaling will occur at
those ends.  If more multipliers are given than there are overlapped
profiles, then it ignores the later multipliers.  For example, if there
were only two profiles of lower overlap and one of upper in the above
example, the result would be 1\,K, 4\,K, 10\,K, \ldots\ 10\,K, 1\,K.

\fillMaskComment

%%% Local Variables: 
%%% mode: latex
%%% TeX-master: "l2ug"
%%% End: 
