\section{Reference}
\label{sec:Reference}

The ICFM software is made of two parts: the server, which is an
executable, and the client, which is an IDL library. The two
communicate with each other via PVM. The users of ICFM have to 
start PVM on the machine where the server and the client run,
then start the server, then use procedures provided in the IDL 
library to invoke the forward model.

How to run the server will be shown in Appendix~\ref{app:Example}.
This section will focus on the IDL library pertaining to ICFM.
Although the example will involve the use of a few AtGod
procedures, it is only an illustration of ICFM's compatibility
to AtGod library, and those AtGod procedures won't be
discuss here.

\subsection{Data types}
\label{sec:DataTypes}

\subsubsection{\snippet{Vector}}
Vector is an anonymous data type that contains a list of quantities.
To access a quantity inside the vector, the users use the dot 
notation: \snippet{[vector].[quantity name]}. Vector should be created and 
manipulated via the procedures provided in Section~\ref{sec:Procedures}
only.

\subsubsection{\snippet{Quantity}}
This is another anonymous data type.  It lays out a collection of
profiles (`instances') for a given quantity (temperature, composition,
tangent pressure, radiance etc.).  It is similar to CFM's 
QuantityTemplate\_T type. For more information, please refer to 
Section~\ref{sec:QuantityTemplates} in \citet{NguyenEtal10}. However,
unlike QuantityTemplate\_T, Quantity can contains both the quantity's
value and its mask. Not every Quantity object is the same, because
there are fields that some quantities have and others don't.
Consequently, users should only create quantities via CreateQuantity
procedure described below. 

\subsection{Procedures and Functions}
\label{sec:Procedures}

This subsection describes the procedures/functions the users will
need to invoke in order to create the input data for and to invoke
the forward model.  They are presented in alphabetical order.  
The example given in Appendix~\ref{app:Example} illustrates their 
invocation and logical flow.

\subsubsection{\snippet{AddQuantity2Vector}}
\lstinputlisting{addquantity2vector-interface.pro}

\subsubsection{\snippet{Core2Qty}}
This procedure is to help convert AtGod relaxed data type into
the stricter Quantity data type.

\lstinputlisting{core2qty-interface.pro}

\subsubsection{\snippet{CreateQuantity}}
\lstinputlisting{createquantity-interface.pro}

\subsubsection{\snippet{CreateVector}}
\lstinputlisting{createvector-interface.pro}

\subsubsection{\snippet{ForwardModel}}
\lstinputlisting{forwardmodel-interface.pro}

\subsubsection{\snippet{ICFM\_Cleanup}}
\lstinputlisting{icfm-cleanup-interface.pro}

\subsubsection{\snippet{ICFM\_SendQuantity}}
\lstinputlisting{icfm-sendquantity-interface.pro}

\subsubsection{\snippet{ICFM\_SendVector}}
\lstinputlisting{icfm-sendvector-interface.pro}

\subsubsection{\snippet{ICFM\_Setup}}
\lstinputlisting{icfm-setup-interface.pro}

%%% Local Variables: 
%%% mode: latex
%%% TeX-master: "icfm"
%%% End: 
