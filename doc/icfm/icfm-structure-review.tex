\section{Review of overall MLSCFM structure}
\label{sec:Review}

This section introduces the various data entities and functionalities
associated with the MLSCFM software. More details on specific derived
type definitions and function / subroutine invocation are given in
Section~\ref{sec:Review}.

\subsection{MLS radiance measurements and related parameters}

The calibrated MLS radiance measurements are stored, along with radiance
geolocation information, in the MLS Level~1B (L1 hereafter) files.  The
L1 files of relevance to MLSCFM applications are the radiance files and
the orbit/attitude file.  These files typically have a daily granularity
running from midnight to midnight UT.

\subsubsection{Major frames and minor frames}

The principal divisions within the MLS L1 files are known as \emph{major
  frames} (MAFs) and \emph{minor frames} (MIFs).  A major frame is one
complete vertical scan of the MLS GHz and THz antennas (including limb
scanning, calibration and retrace activities).  The antennae move
continuously during a MAF and radiances are integrated at 1/6\,s
intervals known as minor frames (MIFs).  During a standard scan, limb
views are taken for the first 120 MIFs (114 for the THz antenna).  The
remaining $\sim$26 MIFs in the MAF are spent performing calibration and
antenna retraces.  The on board software attempts to lock the MLS scans
to the Aura orbit, such that scans are made at essentially the same
latitudes each orbit.  In order to accomplish this, occasional `leap
MIFs' must be added in selected MAFs.

\subsubsection{Bands and channels}
The other divisions of relevance for L1 data are band and channel.  MLS
makes measurements in 34 spectral bands, each of which has from 4 to 129
channels.  The bands are described using the MLS `signal designation'
nomenclature
\begin{quote}
\snippet{[Radiometer name].[Band].[Switch].[Spectrometer]}
\end{quote}
Examples are \snippet{R1A:118.B1F:PT.S0.FB25-1}, or
\snippet{R3:240.B7F:O3.S0.FB25-7}.  The complete list of MLS bands is
given in the \snippet{MLS-Aura_L2Cal-Signals_\ldots} file supplied with
the software.  For most bands, the number of channels is quoted in the
\snippet{Spectrometer} clause (\snippet{FB25}, \snippet{MB11} or
\snippet{WF4}) and are numbered 1\ldots$n$.  The exception are the
Digital Autocorrelator Spectrometers (DACs) which have 129 channels
numbered 0\ldots128.

\subsubsection{MLS L1 files}

Three radiances files are produced for each day of MLS observations.
The L1BRADG file contains the radiances for the majority of the GHz
bands.  The L1BRADD file contains the radiances for the four MLS digital
autocorrelator spectrometers, also measuring GHz spectral regions.  The
L1BRADT file contains radiances measured by the MLS THz bands.

L1BRAD files are stored in HDF5 (not HDF-EOS) and are flat files
containing many HDF arrays.  The most important of these are radiances
which are named according to the signal designation nomenclature
described above.  The arrays are dimensioned channel, MIF, MAF with
channel the most rapidly changing index.  Precisions are stored for each
radiance, in a separate SD with the name `\snippet{<signal> precision}'
(note the intervening space).  Radiances that the MLS Level~1 software
deems should not be used are so indicated by having their precisions set
negative.  The additional quantity `\snippet{<signal> Baseline}' values
(dimensioned channel, MAF) should be added (for each MIF per
channel/MAF) before radiances are used.  Similarly, the
`\snippet{<signal> Baseline precision}' should be added in quadrature to
the precisions.  (The baseline `AC' and `DC' terms are diagnostic and
may be ignored).  All numbers refer to brightness temperature in
Kelvins.

The L1BOA (orbit/attitude) file contains information on spacecraft
location and the locations of the GHz and THz tangent points.  All
angles are degrees and all dimensions are meters.  Time is the EOS
standard TAI93 time (number of seconds, including leap seconds, since
midnight UT on January 1st, 1993.  Quantities are mostly dimensioned
MIF, MAF, with vectors dimensioned 3, MIF, MAF.

\subsubsection{A note on \snippet{GeodAngle}}

An important quantity for MLS L1 (and L2) data is the so-called `Orbit
Geodetic Angle' (referred to in the software as \snippet{GeodAngle} or
\snippet{Phi}).  This is used as a horizontal coordinate for most
quantities, and is defined as the geodetic great circle angle along the
Aura orbit track, increasing with time from the first ascending equator
crossing of the day (defined as 0\degsym).  Unusually for an angle, this
number accumulates beyond 360\degsym.  So, the first complete orbit in
the day is 0\degsym\,--\,360\degsym, the second is
360\degsym\,--\,720\degsym\ etc.

\subsection{Constructing the MLS state vector}

The MLS state vector consists of a heterogeneous collection of
quantities, some geophysical, some instrument related.  As with the
measurement vector (which mainly consists of MLS L1 radiances), these
are contained in Fortran~90 derived types consisting of `quantities' and
`vectors' as detailed in the following subsection.

The principal geophysical quantities in the state vector are consecutive
vertical profiles of atmospheric temperature and composition on a
regular grid in geodetic angle and pressure (actually
$-\log_{10}\left(p/\text{hPa}\right)$).  Other important state vector
quantities are the `tangent pressures' for the MLS GHz and THz limb
views, which are dimensioned MIF, MAF (only the GHz is likely to be of
interest in most CFM applications).

\subsection{Quantities and Vectors in MLSCFM}
\label{sec:QuantityTemplates}

The heterogeneous nature of the MLS state and measurement vectors
dictates that a flexible system be used for their description.  In the
MLSCFM software vectors consist of `templates' and `values'.  Separating
the vector template from its values allows for efficient storage of
related vectors (e.g., the state, the initial guess, and the \emph{a
  priori}, or the measurements and their forward model estimates).
Separating the template from the values also aids more rigid error
checking, such as only allowing to vectors to be added if they share a
template.  Vector templates are, in turn, composed of one or more
`quantity templates'.  While quantities do not, strictly speaking, exist
in isolation from vectors, the term `quantity' is often loosely used in
the software to refer to either a specific quantity within a vector,
and/or its template.

Vector quantities typically describe a single- or multi-valued physical
quantity (temperature profiles, abundances for one species, radiances
for one band).  They can have up to three dimensions, named `channels',
`surfaces' and 'instances' (channels is the fastest changing index,
instances the slowest).  Instance is a time-like or horizontal
dimension, and typically refers to either profiles (for geophysical
quantities) or major frames (for instrument quantities such as measured
radiances).  Similarly, `surfaces' can refer to pressure levels for
geophysical quantities, or minor frames for instrument quantities.  The
channels dimension is typically only used for radiance quantities.  The
actual Fortran arrays storing vector quantity values are dimensioned
\snippet{( channels * surfaces, instances )}.  While increasing the
complexity of access for multi-channel quantities, this simplifies the
storage of matrices in the software, as discussed below.

`Coherent' quantities are those where the surfaces are the same for
every instance.  Geophysical quantities, being reported on a fixed
pressure grid, are coherent, while radiance quantities where each MAF
involves views of different altitudes, are incoherent.

`Stacked' quantities are those where the different surfaces in each
instance share the same horizontal geolocation (latitude, longitude,
time etc.).  Again, geophysical quantities are typically stacked, with
the quantity describing vertical profiles, while radiance and tangent
pressure quantities are unstacked, as each minor frame within the scan
has a slightly different geolocation.

As stated above, geophysical quantities are generally both coherent and
stacked, while `minor frame' (i.e., more instrument related) quantities
are always incoherent and unstacked.  Two `minor frame' vector
quantities describing parameters associated with the same MLS module
(GHz, THz or spacecraft) can, by definition, be assumed to have the same
geolocation.  This important assertion is relied upon throughout the
forward model software.  Conversely, for most geophysical quantities,
geolocation information does not need to be consistent from quantity to
quantity.  The vertical (or horizontal) grids for temperature need not
relate to those for composition, nor do the grids for all molecules need
to be identical.

In addition to geolocation information, quantity templates typically
contain additional information to describe the quantity.  This includes
the molecule for an abundance-related quantity, the signal, radiometer
and module for a radiance-related quantity etc. etc.

\subsection{Matrices in MLSCFM}
\label{sec:Matrices}

Matrices in the CFM software are defined by the vectors that describe
their rows and columns.  For example, the Jacobian matrix relates a
state vector to a measurement vector, while a covariance matrix
describes the covariance of a single vector. The matrices are stored in
a blocked manner, with subblocks relating one instance of one quantity
in the rows vector with an instance of another quantities in the columns
vector.  The tomographic nature of the MLS retrievals
\citep{LiveseyEtal06} means that many of the blocks are all zero and can
be omitted for efficiency of storage and computation.  Similarly, many
of the individual blocks are themselves sparse, and can be stored as
such in the software.

The MLSCFM software contains two main modules for dealing with matrices.
\snippet{MatrixModule_0} contains the software for storing and
manipulating individual blocks, while \snippet{MatrixModule_1} contains
the software that describes matrices as a collection of
\snippet{Matrix_0} blocks, and the algorithms for operating upon these
`\snippet{Matrix_1}' entities.

\subsection{Defining coordinate systems}

As discussed above, vector quantities describe geophysical, measurement,
or other variables on up to three dimensions.  These three dimensions
are known as \snippet{vGrid} (for vertical coordinates), \snippet{hGrid}
(for horizontal coordinates) and \snippet{fGrid} (for frequency /
channel descriptions).

These grids can be specified prior to the construction of quantity
templates.  However, they are not essential to the creation of vector
quantities.  For example, minor frame quantities take their geolocation
information (both horizontal and vertical) directly from the L1B
information.  In addition, radiance quantities take their
frequency/channel information directly from the MLS signals database.

\subsubsection{Vertical coordinates -- \snippet{vGrid}}

MLS vector quantities can be described on a variety of vertical
coordinates, including pressure (and more commonly
$-\log_{10}\left[\text{Pressure / hPa}\right]$ (known as \snippet{zeta}
in the software) and altitude (in meters).  Typically \snippet{vGrids}
are used only to define coherent vector quantities (see discussion
above).  Thus far, the only incoherent quantities in MLSL2 are minor
frame quantities.  These have their vertical coordinates defined by MLS L1B
tangent point altitude information (in meters).

\subsubsection{Horizontal coordinates -- \snippet{hGrid}}

The definition of horizontal coordinates differs from that of vertical
and frequency coordinates, in that the coordinates change from scene to
scene, and run to run, depending on the timespan of MLS data under
consideration.  There are several types of \snippet{hGrids} in the L2
software, but for the purposes of the CFM, only `regular' grids are of
interest.

These are created for a given `chunk' (i.e., range of MAFs from the L1
file), and distribute profiles with a requested geodetic angle spacing
along the orbit.  The absolute offset of this even spacing can also be
specified.  Typically this is set to zero degrees meaning that this even
spacing would place a profile exactly on the equator (were the MAFs in
the supplied chunk to encompass the equator).


\subsubsection{Frequency coordinates -- \snippet{fGrid}}

As discussed above, few quantities have more than one channel.  The most
significant group of multi-channeled quantities is radiances, these take
their frequency information directly from MLS signals database.  Some
state vector quantities are used to describe spectrally flat, or slowly
varying, offsets to observed radiance or atmospheric emission/absorption
(`baseline' and `extinction', respectively).  If a (typically slow)
frequency variation is required for these, then this is described as a
linear interpolation between tie points specified in a \snippet{fGrid}.

\subsection{Sources of information for MLS state vector components}

The MLS forward models require a wide variety of vector quantities as
input.  Typically, two state vectors are supplied, one
(\snippet{ForwardModelIn}) containing the quantities of interest (i.e.,
those quantities for which a solution is sought for some or all
elements), and the other (\snippet{ForwardModelExtra}) containing other
(e.g., calibration) quantities, also needed by the forward model.  The
majority of these lesser quantities will be defined and filled for the user by
subroutines supplied as part of the MLSCFM software, and need little
discussion here.

However, an important exception is the tangent pressure information, for
which there are a variety of approaches.  Firstly, the value retrieved
as part of the MLS processing can be read from the L2 `DGM' file.  This
is useful for initial `O--F' studies, but is not feasible when radiance
assimilation is to be performed in real time.  In addition the
production processing's use of GMAO temperature as an \emph{a priori}
introduces potential biases in this approach.

An alternative approach is to estimate the tangent pressure given
knowledge of atmospheric temperature and geopotential height, Aura
orientation and MLS pointing.  However, the pointing knowledge is
imperfect and can also introduce biases.

The best approach is likely to solve for tangent pressure for each minor
frame as part of any radiance assimilation activity.  Whether this
proves to be practical is the subject of further investigations.
(Aside, one approach may be to take GMAO GPH as `truth' and retrieve a
-- hopefully slowly varying -- Aura attitude offset term).

\subsection{Defining forward models}

In addition to defining the state and measurement vectors (and possibly
defining Jacobian matrices to connect them), the CFM software also
requires individual forward models to be defined.  There are several
different types of forward models that can be defined.  In the context
of the CFM, we anticipate that the `Full', `Linearized', `Baseline' and
`Scan2D' models will be of interest.

\subsubsection{The `full' forward model}

The `full' forward model is the cornerstone of the MLS forward models.
This model is described in \citet{ReadEtal06} and \citet{SchwartzEtal06}
and is a full 2-dimensional ray tracing line-by-line forward model.  One
full forward model call computes radiances for one or more channels
within a given radiometer.  As the forward model computes for the full
spectral range covered by all the channels requested, in cases where
radiances from distinct spectral regions (e.g., groups of channels above
and below, but not at, the line center) are to be computed, more
efficiency is gained if separate full forward models are defined, each
covering only adjacent channels.  Full forward models may be `scalar' or
`polarized', the latter being needed for accurate modeling of the
118\,GHz O\cs2 line in the mesosphere.

In addition to computing radiances, full forward models can also be
requested to compute Jacobians with respect some state vector quantities
(down to the element by element level).  Various approximations (e.g.,
the degree to which full Gauss-Legendre integration is performed, and
the so called `pre frequency averaging' assumption) can be requested,
depending on the desired efficiency and accuracy.

\subsubsection{The `linearized' forward model}

The `linearized' forward model simply computes radiances based on a
linearized form of the full forward model, to whit:
\begin{equation}
\b{y} = \b{y}_0 + \b{K} \left ( \b{x} - \b{x}_0\right)
\end{equation}
Where $\b{x}$ is the state vector, and $\b{y}$ is a vector of radiances
(on a fixed tangent pressure grid).  $\b{y}_0$ and $\b{K}_0$ are
pre-computed radiances and jacobians, respectively, corresponding to a
pre-defined linearization point $\b{x}_0$ in state space.  Following
this Taylor series expansion, the radiances are interpolated to the
actual tangent pressures defined in the supplied state vector.  The
precomputed radiances and Jacobians for pre-selected states are stored
in `Level~2 Processing Coefficient' (L2PC) files, and are generated by
off-line runs of the MLSL2 software invoking the full forward models
described above.  The linearization points are typically defined as a
function of latitude (in $\sim$15\degsym-wide bins) and calendar month.

The linearized model produces results much faster than all but the
simplest full forward model runs.  However, being linear, its accuracy
degrades rapidly as atmospheric optical depth increases (i.e., close to
line centers and/or lower in the atmosphere).

\subsubsection{The `baseline' forward model}

Most MLS geophysical products derive from observations of spectral
contrast (channel to channel changes in observed radiances).  Typically
the `baseline' upon which these spectra sit is of less interest (and is
typically more sensitive to errors in spectroscopy or instrument
calibration).  The MLS data processing algorithms can retrieve a
`baseline' offset term along with composition and temperature profiles
etc.  This term is simply a spectrally-flat (or slowly varying with
frequency) offset added to each forward model calculation.  This can be
invoked by adding the \snippet{do_baseline} flag to a forward model, or,
if desired, a separate `baseline' forward model can be manually invoked
to add these terms.

\subsubsection{The `scan' forward model}

Given knowledge, from the MLS state vector, of the atmospheric
temperature profile on pressure surfaces, the geopotential height at one
pressure surface, and MLS tangent point pressures for each minor frame,
it is possible, through hydrostatic balance (and refraction
calculations, for which a water vapor profile is also needed for
accuracy), to compute the expected tangent point altitudes for the MLS
radiance observations.  Comparing these to the tangent point altitudes
derived from the MLS antenna position encoder and the Aura attitude
control system provides an additional constraint on the atmospheric
temperature profiles, and enables retrieval of geopotential height
(subject to inaccuracies in MLS/Aura pointing).

The `scan residual' model invokes these calculations as described in
\citet{ReadEtal06}.  For a variety of reasons, this model is formulated
to compute the `scan residual', being the difference between the
tangent point altitudes inferred from hydrostatic balance and those
inferred from Aura/MLS pointing.  The `measurement' of this quantity is
assumed to be zero with a precision reflective of the precision of the
MLS scan encoder ($\sim$30\,m).

\subsection{Invoking forward models}

Once forward models are defined, and state and measurement vectors are
defined and filled, forward models can be invoked.  As described above,
two state vectors can be supplied -- \snippet{ForwardModelIn} defining
the quantities for which solutions are sought, and
\snippet{ForwardModelExtra} defining other needed quantities.  From the
forward model's perspective, the only difference between these two is
that Jacobians, if requested, are only computed for quantities in the
first vector.

The forward models, particularly the full forward model, require a
diverse range of quantities to be supplied through the state vector(s).
These are detailed in Table~\ref{tab:StateVectorNeeds}.

\begin{table}
\caption{State vector quantities needed for different forward model
  types.  Cells with `?' indicate that the quantity may be required for
  some configurations.}
\label{tab:StateVectorNeeds}
\begin{center}
\begin{tabular}{lccc}
\toprule
Quantity & Full & Linearized & Scan \\
\midrule
Temperature & Y & Y & Y \\
Reference GPH & Y & Y & Y \\
Tangent pressure & Y & Y & Y \\
Composition & Y & Y & Y \\
Space radiance & Y \\
Earth reflectivity & Y \\
Spacecraft altitude & Y \\
Orbital inclination & Y \\
Line of sight velocity & Y \\
Spacecraft altitude & Y \\
Field of view offsets & Y \\
Sideband fractions & Y & ? \\
Magnetic field & ? \\
\bottomrule
\end{tabular}
\end{center}
\end{table}

\subsection{A note on the `mockup'}

The mockup code supplied at the end of this document is simply meant to
serve as an example of the kind of code that can use the forward model.
It is \emph{not} intended to be turned into some kind of subroutine to
be called by external software.  To do so would necessitate rereading
all the calibration files, recreating the vector templates etc.  This is
largely unnecessary work (though we note that the vector templates do
need to be recreated each time a new `scene' is considered).  Once the
calibration files are read and vectors defined and filled, forward
models can be called as many times, with as many different
configurations as are desired.

%%% Local Variables: 
%%% mode: latex
%%% TeX-master: "cfm"
%%% End: 
