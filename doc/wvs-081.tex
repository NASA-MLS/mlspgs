\documentclass[11pt]{article}
\usepackage[fleqn]{amsmath}\textwidth 6.5in
\oddsidemargin -0.25in
%\evensidemargin -0.5in
\topmargin -0.25in
\textheight 9.0in

\newcommand{\docname}{\bf wvs-081}
\newcommand{\docdate}{24 June 2008}

\begin{document}

%\tracingcommands=1
\newlength{\hW} % heading box width
\newlength{\pW} % page number field width
\settowidth{\hW}{\docname}
\settowidth{\pW}{Page \pageref{lastpage}\ of \pageref{lastpage}}
\ifdim \pW > \hW \setlength{\hW}{\pW} \fi
\makeatletter
\def\@biblabel#1{#1.}
\newcommand{\ps@twolines}{%
  \renewcommand{\@oddhead}{%
    \docdate\hfill\parbox[u]{\hW}{{\hfill\docname}\newline
                          Page \thepage\ of \pageref{lastpage}}}%
\renewcommand{\@evenhead}{}%
\renewcommand{\@oddfoot}{}%
\renewcommand{\@evenfoot}{}%
}%
\makeatother
\pagestyle{twolines}

\vspace{-10pt}
\begin{tabbing}
\phantom{References: }\= \\
To: \>Nathaniel, Bill, Alyn\\
Subject: \>Variable projection applied to retrieval\\
From: \>Van Snyder\\
\end{tabbing}

\parindent 0pt \parskip 10pt
\vspace{-20pt}

The solution of the radiative transfer equation, in quadrature form, after
integrating by parts, is
%
\begin{equation}\label{one}
I = \sum_i \Delta B_i (T_i)\, \mathcal{T}_i (T_i, f_i^k)\,,
\end{equation}
%
where $I$ is radiance, $i$ is the index of a point on the ray, $B_i$ is the
Planck function, $T_i$ is temperature, $f_i^k$ is the mixing ratio of the
$k^{\text{th}}$ species, and $\mathcal{T}_i$ is given by (again in quadrature
form):
%
\begin{equation}\label{two}
 \mathcal{T}_i (T_i, f_i^k) =
   \exp \left( - \sum_{j=2}^i \Delta \delta_{j \rightarrow j-1}
                                           ( T_j, f_j^k) \right )\,,
\end{equation}
%
where $j$ is the index of a point on the ray and $\Delta \delta_{j \rightarrow
j-1}$ is given by
%
\begin{equation}\label{three}
 \Delta \delta_{j \rightarrow j-1} ( T_j, f_j^k) = \sum_k f_j^k \beta ( T_j )\,,
\end{equation}
%
where $\beta$ is a nonlinear function of $T$.

It is desired to solve for $T_i$ and $f_i^k$.  The solution method consists of
the following steps.

\begin{enumerate}

\item Given an initial guess for $T_i$, evaluate $\Delta B_i (T_i)$.

\item \label{second}Given values for $I$ at several frequencies and several
  viewing angles, and $\Delta B_i (T_i)$, solve Equation (\ref{one}) for
  $\mathcal{T}_i (T_i, f_i^k)$.  This is a linear problem.

\item \label{third}Take the logarithm of Equation (\ref{two}) and solve for
  $\Delta \delta_{j \rightarrow j-1} ( T_j, f_j^k)$.  This is a linear problem.

\item Solve Equation (\ref{three}) for $f_j^k$ and $T_j$.  This is a
  \emph{separable} nonlinear problem in which $f_j^k$ appears linearly and
  $T_j$ appears nonlinearly.  Separable problems are best solved by using the
  \emph{Variable Projection}\footnote{G. Golub and V. Pereyra, ``Separable
  nonlinear least squares: the variable projection method and its
  applications,'' Technical Report SCCM-02-07, Department of Computing Science,
  Stanford University,  Stanford, CA 94305, USA, July 2002. {\tt
  http://www-sccm.stanford.edu/pub/sccm/sccm02-07.ps.gz}.} method, which keeps
  the linear variables $(f_j^k)$ on a solution consistent with the nonlinear
  variables $(T_j)$ at every iteration, thereby converging much more quickly
  and reliably.

\item Using the newly-calculated values of $T_i$, evaluate $\Delta B_i (T_i)$
  and then Equation (\ref{one}).

\item If the new value of $I$ is not sufficiently close to the measured value,
  repeat from step \ref{second}.

\end{enumerate}

Because of the logarithm in step \ref{third}, the statistics of the solution are
log-normal instead of normal.  The covariance matrix of the solution has to be
interpreted in this context.

I have not given any thought to antenna convolution, frequency averaging, or
sideband folding.  At least the first two of these are linear, so incorporating
them shouldn't be difficult.

\label{lastpage} \end{document}
% $Id$
