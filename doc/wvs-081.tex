\documentclass[11pt]{article}
\usepackage[fleqn]{amsmath}\textwidth 6.5in
\oddsidemargin -0.25in
%\evensidemargin -0.5in
\topmargin -0.25in
\textheight 9.0in

\newcommand{\docname}{\bf wvs-081r2}
\newcommand{\docdate}{27 June 2008}

\begin{document}

%\tracingcommands=1
\newlength{\hW} % heading box width
\newlength{\pW} % page number field width
\settowidth{\hW}{\docname}
\settowidth{\pW}{Page \pageref{lastpage}\ of \pageref{lastpage}}
\ifdim \pW > \hW \setlength{\hW}{\pW} \fi
\makeatletter
\def\@biblabel#1{#1.}
\newcommand{\ps@twolines}{%
  \renewcommand{\@oddhead}{%
    \docdate\hfill\parbox[u]{\hW}{{\hfill\docname}\newline
                          Page \thepage\ of \pageref{lastpage}}}%
\renewcommand{\@evenhead}{}%
\renewcommand{\@oddfoot}{}%
\renewcommand{\@evenfoot}{}%
}%
\makeatother
\pagestyle{twolines}

\vspace{-10pt}
\begin{tabbing}
\phantom{References: }\= \\
To: \>Nathaniel, Bill, Alyn\\
Subject: \>Variable projection applied to retrieval\\
From: \>Van Snyder\\
\end{tabbing}

\parindent 0pt \parskip 10pt
\vspace{-20pt}

The solution of the radiative transfer equation, in quadrature form, after
integrating by parts, is
%
\begin{equation}\label{one}
I \approx \sum_i \frac{\partial B_i(T_i)}{\partial s_i}
      \mathcal{T}_i (T_i, f_i^k)\, \Delta s_i \approx
    \sum_i \Delta B_i (T_i)\, \mathcal{T}_i (T_i, f_i^k)\,,
\end{equation}
%
where $I$ is radiance, $i$ is the index of a point on the ray, $B_i$ is the
Planck function, $T_i$ is temperature, $f_i^k$ is the mixing ratio of the
$k^{\text{th}}$ species, and $\mathcal{T}_i$ is given by (again in quadrature
form):
%
\begin{equation}\label{two}
 \mathcal{T}_i (T_i, f_i^k) \approx
   \exp \left( - \sum_{j=2}^i \Delta \delta_{j \rightarrow j-1}
                                           ( T_j, f_j^k)\, \Delta s_j
        \right )\,,
\end{equation}
%
where $j$ is the index of a point on the ray and $\Delta \delta_{j \rightarrow
j-1}$ is given by
%
\begin{equation}\label{three}
 \Delta \delta_{j \rightarrow j-1} ( T_j, f_j^k) = \sum_k f_j^k \beta ( T_j )\,,
\end{equation}
%
where $\beta$ is a nonlinear function of $T$.

It is desired to solve for $T$ and $f^k$ on a grid, through which Equation
(\ref{one}) is integrated along a very large number of paths.  We use linear
interpolation from the grid to the integration path. The solution method
consists of the following steps.

\begin{enumerate}

\item Given an initial guess for $T_i$, evaluate $\Delta B_i (T_i)$.

\item \label{second}Given measured values $\hat I$ for $I$ at several
  frequencies and several viewing angles, and $\Delta B_i (T_i)$, solve
  Equation (\ref{one}) for $\mathcal{T}_i (T_i, f_i^k)$.  This is a linear
  problem.

\item \label{third}Take the logarithm of Equation (\ref{two}) and solve for
  $\Delta \delta_{j \rightarrow j-1} ( T_j, f_j^k)$.  This is a linear problem.

\item\label{fourth} Solve Equation (\ref{three}) for $f_j^k$ and $T_j$.  This
  is a \emph{separable} nonlinear problem in which $f_j^k$ appears linearly and
  $T_j$ appears nonlinearly.  Separable problems are best solved by using the
  \emph{Variable Projection}\footnote{G. Golub and V. Pereyra, ``Separable
  nonlinear least squares: the variable projection method and its
  applications,'' Technical Report SCCM-02-07, Department of Computing Science,
  Stanford University,  Stanford, CA 94305, USA, July 2002. {\tt
  http://www-sccm.stanford.edu/pub/sccm/sccm02-07.ps.gz}.} method, which keeps
  the linear variables $(f_j^k)$ on a solution consistent with the nonlinear
  variables $(T_j)$ at every iteration, thereby converging much more quickly
  and reliably.

\item\label{fifth} Using the newly-calculated values of $T_i$, evaluate
  $\Delta B_i (T_i)$ and then Equation (\ref{one}).

\item If $| \hat I - I|$ is too large, repeat from step \ref{second}.

\end{enumerate}

Because of the logarithm in step \ref{third}, the statistics of the solution are
log-normal instead of normal.  The covariance matrix of the solution has to be
interpreted in this context.

I have not given any thought to antenna convolution, frequency averaging, or
sideband folding.  At least the first two of these are linear, so incorporating
them shouldn't be difficult.

An alternative that might converge more rapidly is to replace step \ref{fifth}
above with the following steps.

\begin{enumerate}
  \makeatletter
  \renewcommand{\theenumi}{\ref{fifth}\@alph\c@enumi}
  \makeatother

\item\label{fifth.a} Evaluate $\Delta B_i(T_i)$ and $\frac{\partial \Delta
  B_i(T_i)}{\partial T_i}$ using the new values of $T_i$ from step \ref{fourth}.

\item Use $\frac{\partial \delta_{j \rightarrow j-1}}{\partial T_j}$, which are
  necessarily calculated in step \ref{fourth}, together with $\Delta B_i$ and
  $\frac{\partial \Delta B_i(T_i)}{\partial T_i}$ evaluated in step
  \ref{fifth.a}, to calculate $\frac{\partial I}{\partial T_i}$ and a new value
  of $I$.

\item\label{fifth.c} Solve $\frac{\partial I}{\partial T_i}\, \delta T_i \sim
  \hat I - I$ for $\delta T_i$.

\item Replace $T_i$ from step \ref{fourth} by $T_i + \delta T_i$.

\item Evaluate $\Delta B_i(T_i)$.

\item\label{fifth.f} Use $\delta T_i$ from step \ref{fifth.c} to re-evaluate
  $\beta(T_j) \leftarrow \beta(T_j) + \frac{\partial \beta(T_j)}{\partial
  T_j}\, \delta T_j$ (evaluating $\beta(T_j)$ is very expensive). 
  $\frac{\partial \beta(T_j)}{\partial T_j}$ are necessarily calculated in step
  \ref{fourth}.

\item\label{fifth.g} Evaluate $I$ using values of $f_j^k$ from step
  \ref{fourth} and $\beta(T_i)$ from step \ref{fifth.f}.

\end{enumerate}

In order to use the first-order approximation in step \ref{fifth.f}, the values
of $\beta(T_j)$ and $\frac{\partial\beta(T_j)}{\partial T_j}$ would have to be
interpolated from the path back to the grid, and then the updated values
interpolated back to the path.  Alternatively, they could be evaluated on the
grid instead of the path, and then interpolated to the path.  Either way, the
memory requirements might be prohibitive.

Although the instrument has only $\sim$1750 channels, we might need to evaluate
$\beta(T_j)$ and $\frac{\partial\beta(T_j)}{\partial T_j}$ at as many as 600,000
frequencies, and then average these computations, weighted with the filter
shape for each channel, so as to model the shapes of the spectral lines
accurately.  This leads to a requirement for about 17 gigawords of storage.

If step \ref{fifth.f} is too expensive, use $\mathcal{T}_i$ from step
\ref{second} in step \ref{fifth.g}.

\label{lastpage}
\end{document}

\newpage

The radiative transfer equation in differential form is
%
\begin{equation}
\frac{\text{d}\,I}{\text{d}s} = \delta I + B
\end{equation}
%
In integral form, after integrating by parts, the solution is
%
\begin{equation}\label{five}
I = \int_0^\infty \frac{\partial B}{\partial s}\,
 \exp \left( - \int_0^s \delta(\sigma)\, \text{d} \sigma \right)
  \text{d} s
\end{equation}
%
where $\delta(\sigma) = \sum_k f^k(\sigma)\, \beta^k(\sigma)$.  $B$ and
$\beta$ also depend upon temperature and frequency.  Temperature is a function
of $s$, but frequency is not.

Equation (\ref{one}) is a quadrature form of Equation (\ref{five}).

\label{lastpage}
\end{document}
% $Id$
