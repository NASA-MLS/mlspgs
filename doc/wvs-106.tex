\documentclass[11pt]{article}
\usepackage[fleqn]{amsmath}

\textwidth 6.5in
\oddsidemargin -0.25in
%\evensidemargin -0.5in
\topmargin -0.25in
\textheight 9.0in

\newcommand{\docname}{\bf wvs-106}
\newcommand{\docdate}{25 July 2011}

\ifx\pdfoutput\undefined
  \pdfoutput=0
  \usepackage[hypertex,plainpages,hyperindex=true]{hyperref}
  \hypersetup{%
    hypertexnames=false%
  }
  % Specify the driver for the color package
  \ExecuteOptions{dvips}
  %\ExecuteOptions{xdvi}
\else
  \ifnum\pdfoutput>0
    \usepackage[pdftex,plainpages,hyperindex=true,pdfpagelabels]{hyperref}
    \hypersetup{%
      hypertexnames=false,%
      colorlinks=true,%
      linktocpage=true,%
    }
    % Specify the driver for the color package
    \ExecuteOptions{pdftex}
  \else
    \usepackage[hypertex,plainpages,hyperindex=true]{hyperref}
    \hypersetup{%
      hypertexnames=false%
    }
    % Specify the driver for the color package
    \ExecuteOptions{dvips}
    %\ExecuteOptions{xdvi}
  \fi
\fi

\hyperbaseurl{}
\ifx\dvidir\undefined
  \newcommand\hr[1]{\href{#1.dvi}{dvi} \href{#1.pdf}{pdf}}
\else
  \newcommand\hr[1]{\href{\dvidir/#1.dvi}{dvi} \href{\pdfdir/#1.pdf}{pdf}}
\fi
\newcommand\h[1]{#1 \hr{#1}}

\begin{document}

%\tracingcommands=1
\newlength{\hW} % heading box width
\newlength{\pW} % page number field width
\settowidth{\hW}{\docname}
\settowidth{\pW}{Page \pageref{lastpage}\ of \pageref{lastpage}}
\ifdim \pW > \hW \setlength{\hW}{\pW} \fi
\makeatletter
\def\@biblabel#1{#1.}
\newcommand{\ps@twolines}{%
  \renewcommand{\@oddhead}{%
    \docdate\hfill\parbox[u]{\hW}{{\hfill\docname}\newline
                          Page \thepage\ of \pageref{lastpage}}}%
\renewcommand{\@evenhead}{}%
\renewcommand{\@oddfoot}{}%
\renewcommand{\@evenfoot}{}%
}%
\makeatother
\pagestyle{twolines}

\newcommand{\T}{\mathcal{T}}

\vspace{-10pt}
\begin{tabbing}
\phantom{References: }\= \\
To: \>Van, Igor, Nathaniel, Bill, Paul\\
Subject: \>Alternative formulation for derivatives\\
From: \>Van Snyder\\
Reference: \>\h{wvs-100}, \h{wvs-102}\\
\end{tabbing}

\parindent 0pt \parskip 5pt
\vspace{-20pt}

Start with the clear-sky non-scattering radiative-transfer equation

\renewcommand{\d}{\text{d}}
\begin{equation}
\frac{\d I(s)}{\d s} + \alpha(s) I(s) = \alpha(s) B(s) \,.
\end{equation}

Take the derivative w.r.t. $f^k_{lm}$ to get

\begin{equation}
\frac{\partial}{\partial f^k_{lm}} \frac{\d I(s)}{\d s} +
 \frac{\partial}{\partial f^k_{lm}} \alpha(s) I(s) =
  \frac{\partial}{\partial f^k_{lm}} \alpha(s) B(s) \,.
\end{equation}

Exchange the order of differentiation, observe $\frac{\partial
B(s)}{\partial f^k_{lm}} = 0$, and collect terms involving $\frac{\partial
I(s)}{\partial f^k_{lm}}$ on the left to get

\begin{equation}\label{three}
\frac{\d}{\d s} \frac{\partial I(s)}{\partial f^k_{lm}} +
 \alpha(s) \frac{\partial I(s)}{\partial f^k_{lm}} =
  \frac{\partial \alpha(s)}{\partial f^k_{lm}} ( B(s) - I(s) )
\end{equation}

Assuming $\frac{\partial I(s_0)}{\partial f^k_{lm}} = 0$, the solution
for $\frac{\partial I(s_r)}{\partial f^k_{lm}}$ can be written

\begin{equation}\label{four}
\frac{\partial I(s_r)}{\partial f^k_{lm}} =
 \int_{s_0}^{s_r} \mathcal{T}(s,s_r)
  \frac{\partial \alpha(s)}{\partial f^k_{lm}}
   ( B(s) - I(s) ) \,\d s
\text{ where }
 \mathcal{T}(s_1,s_2) = \exp \left( -\int_{s_1}^{s_2} \alpha(\sigma)
 \d \sigma \right) \,.
\end{equation}

Differentiate Equation (\ref{three}) w.r.t.
$f^{\kappa}_{\lambda\mu}$ and re-arrange similarly to get


\begin{equation}
\frac{\d}{\d s} \frac{\partial^2 I(s)}
                     {\partial f^k_{lm} \partial f^{\kappa}_{\lambda\mu}} +
 \alpha(s) \frac{\partial^2 I(s)}
                {\partial f^k_{lm} \partial f^{\kappa}_{\lambda\mu}} =
  \frac{\partial^2 \alpha(s)}
       {\partial f^k_{lm} \partial f^{\kappa}_{\lambda\mu}} (B(s)-I(s)) -
   \frac{\partial \alpha(s)}{\partial f^k_{lm}}
    \frac{\partial I(s)}{\partial f^{\kappa}_{\lambda\mu}}
   -
   \frac{\partial \alpha(s)}{\partial f^{\kappa}_{\lambda\mu}}
    \frac{\partial I(s)}{\partial f^k_{lm}}
   \,.
\end{equation}

Assuming $\frac{\partial^2 I(s_0)}{\partial f^k_{lm} \partial
f^{\kappa}_{\lambda\mu}} = 0$, the solution for $\frac{\partial^2
I(s_r)}{\partial f^k_{lm} \partial f^{\kappa}_{\lambda\mu}}$ can be
written

\begin{equation}\label{six}
\frac{\partial^2 I(s_r)}
     {\partial f^k_{lm} \partial f^{\kappa}_{\lambda\mu}} =
 \int_{s_0}^{s_r} \mathcal{T}(s,s_r) \left[
  \frac{\partial^2 \alpha(s)}
       {\partial f^k_{lm} \partial f^{\kappa}_{\lambda\mu}}
   ( B(s) - I(s) ) -
   \frac{\partial \alpha(s)}{\partial f^k_{lm}}
    \frac{\partial I(s)}{\partial f^{\kappa}_{\lambda\mu}}
   -
   \frac{\partial \alpha(s)}{\partial f^{\kappa}_{\lambda\mu}}
    \frac{\partial I(s)}{\partial f^k_{lm}}
   \right] \d s
  \,.
\end{equation}

See \h{wvs-102} for definitions of $\frac{\partial \alpha(s)}{\partial
f^k(s)}$ and $\frac{\partial^2 \alpha(s)}{\partial f^k(s) \partial
f^\kappa(s)}$.

Notice that there are no explicit inner integrals in this formulation. 
We need $\mathcal{T}(s,s_r)$ and $I(s)$ to compute $\frac{\partial
I(s_r)}{\partial f^k_{lm}}$, which are integrals, but we get them while
computing $I(s_r)$.  We need $\frac{\partial I(s)}{\partial f^k_{lm}}$ to
compute $\frac{\partial^2 I(s_r)}{\partial f^k_{lm} \partial
f^{\kappa}_{\lambda\mu}}$, but we get this while computing $\frac{\partial
I(s_r)}{\partial f^k_{lm}}$.  We could either save $\frac{\partial
\alpha(s)}{\partial f(s)}$ while computing $\frac{\partial
I(s_r)}{\partial f^k_{lm}}$, or compute it anew while computing
$\frac{\partial^2 I(s_r)}{\partial f^k_{lm} \partial
f^{\kappa}_{\lambda\mu}}$.

\label{lastpage}
\end{document}

% $Id$

% $Log$

