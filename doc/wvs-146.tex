\documentclass[11pt]{article}
\usepackage{alltt}
\usepackage[fleqn]{amsmath}
\usepackage{floatflt}
\usepackage{graphicx}
\usepackage{longtable}
\usepackage[strings]{underscore}

\textwidth 6.5in
\oddsidemargin -0.25in
%\evensidemargin -0.5in
\topmargin -0.5in
\textheight 9in

\newcommand{\docname}{wvs-146}
\newcommand{\docdate}{12 October 2017}

\ifx\pdfoutput\undefined
  \pdfoutput=0
  \usepackage[hypertex,plainpages,hyperindex=true]{hyperref}
  \hypersetup{%
    hypertexnames=false%
  }
  % Specify the driver for the color package
  \ExecuteOptions{dvips}
  %\ExecuteOptions{xdvi}
\else
  \ifnum\pdfoutput>0
   
\usepackage[pdftex,plainpages,hyperindex=true,pdfpagelabels]{hyperref}
    \hypersetup{%
      hypertexnames=false,%
      colorlinks=true,%
      linktocpage=true,%
    }
    % Specify the driver for the color package
    \ExecuteOptions{pdftex}
  \else
    \usepackage[hypertex,plainpages,hyperindex=true]{hyperref}
    \hypersetup{%
      hypertexnames=false%
    }
    % Specify the driver for the color package
    \ExecuteOptions{dvips}
    %\ExecuteOptions{xdvi}
  \fi
\fi

\hyperbaseurl{}
\newcommand\hr[1]{\href{#1.dvi}{dvi}, \href{#1.pdf}{pdf}}
\newcommand\h[1]{#1 (\hr{#1})}

\begin{document}

%\tracingcommands=1
\newlength{\hW} % heading box width
\newlength{\pW} % page number field width
\settowidth{\hW}{\bf\docname}
\settowidth{\pW}{Page \pageref{lastpage}\ of \pageref{lastpage}}
\ifdim \pW > \hW \setlength{\hW}{\pW} \fi
\makeatletter
\def\@biblabel#1{#1.}
\newcommand{\ps@twolines}{%
  \renewcommand{\@oddhead}{%
    \docdate\hfill\parbox[t]{\hW}{{\hfill\bf\docname}\newline
                          Page \thepage\ of \pageref{lastpage}}}%
\renewcommand{\@evenhead}{}%
\renewcommand{\@oddfoot}{}%
\renewcommand{\@evenfoot}{}%
}%
\makeatother
\pagestyle{twolines}

\vspace{-10pt}
\begin{tabbing}
\phantom{References: }\= \\
To: \>Van\\
Subject: \>Radii of curvature of an oblate spheroid\\
From: \>Van Snyder\\
%Reference: \\
\end{tabbing}

\parindent 0pt \parskip 6pt
\vspace{-20pt}

\section{Definitions}

Let $a$ be the equatorial radius, and $c$ the polar radius of the Earth,
an oblate spheroid.

Let $\lambda$ be geocentric latitude.

If a surface is cut by a plane that includes its normal at a point, there
is a curve in that plane.  Assuming the surface is smooth, the curvature
of the curve at the point of the normal is defined.

Among all the planes that include the normal, the curvature of the curve
in one of the planes is maximum, and the curvature in another one is
minimum.  These curvatures are the \emph{principal
curvatures}\footnote{Harry Lass, {\bf Vector and Tensor Analysis},
McGraw-Hill (1950) pp 71--78.}, denoted $\kappa_1$ and $\kappa_2$,
respectively, which are roots of

\begin{equation}
(E G - F^2)\, \kappa^2 - ( e G + g E - 2 f F )\, \kappa + ( e g - f^2 ) = 0
\end{equation}

where $E$, $F$, and $G$ are coefficients of the first fundamental form for
an oblate spheroid:

\begin{equation}\begin{split}
E = \,& a^2 \sin^2 \lambda \\
F = \,& 0 \\
G = \,& a^2 \cos^2 \lambda + c^2 \sin^2 \lambda
\end{split}\end{equation}

and $e$, $f$, and $g$ are coefficients of the second fundamental form:

\begin{equation}\begin{split}
g = \,& \frac{a c}
             {\sqrt{G}} \\
f = \,& 0 \\
e = \,& g \sin^2 \lambda\\
\end{split}\end{equation}

\section{Gaussian curvature of oblate spheroid}

The Gaussian curvature $K$ is the product of the principal curvatures:

\begin{equation}
K = \kappa_1 \kappa_2
= \frac{ e g - f^2 }{ E G -F^2}
= \frac{ e g }{ E G }
= \frac{c^2}{G^2}
\end{equation}

and the radius of Gaussian curvature $\sqrt{1/K} = \frac{G}c$ is the
geometric mean of the principal radii of curvature.

\section{Mean curvature of oblate spheroid}

The mean curvature $H$ is the average of the principal curvatures:

\begin{equation}
H = \frac12(\kappa_1 + \kappa_2) = \frac{e G + g E - 2 f F}{2\,(E G -F^2)}
= \frac{e G + g E}{2\, E\, G}
= \frac{c(a^2+G)}{2\,a \,G^{3/2}}
\end{equation}

and the radius of mean curvature is $1/H$.  The average of the principal
radii of curvature is

\begin{equation}
\frac12 \left( \frac1{\kappa_1} + \frac1{\kappa_2} \right)
= \frac{H}{K}
= \frac{(a^2+G) \sqrt{G}}{2\,a\,c}
\neq \frac2{\kappa_1 + \kappa_2}\,.
\end{equation}

\section{Curvature of a sphere}

For a sphere, $a = c = r$, $G = r^2$, $K = 1/r^2$, and $H = 1/r$, as
exected.

\label{lastpage}
\vspace*{-0.1in} % Somehow, this causes lastpage to be defined
\end{document}
