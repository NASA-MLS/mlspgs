\documentclass[11pt]{article}
\usepackage{alltt}
\usepackage[fleqn]{amsmath}
\usepackage{floatflt}
\usepackage{graphicx}
\usepackage{longtable}
\usepackage[strings]{underscore}

\textwidth 6.5in
\oddsidemargin -0.25in
%\evensidemargin -0.5in
\topmargin -0.5in
\textheight 9in

\newcommand{\docname}{wvs-146r1}
\newcommand{\docdate}{11 November 2019}

\ifx\pdfoutput\undefined
  \pdfoutput=0
  \usepackage[hypertex,plainpages,hyperindex=true]{hyperref}
  \hypersetup{%
    hypertexnames=false%
  }
  % Specify the driver for the color package
  \ExecuteOptions{dvips}
  %\ExecuteOptions{xdvi}
\else
  \ifnum\pdfoutput>0
   
\usepackage[pdftex,plainpages,hyperindex=true,pdfpagelabels]{hyperref}
    \hypersetup{%
      hypertexnames=false,%
      colorlinks=true,%
      linktocpage=true,%
    }
    % Specify the driver for the color package
    \ExecuteOptions{pdftex}
  \else
    \usepackage[hypertex,plainpages,hyperindex=true]{hyperref}
    \hypersetup{%
      hypertexnames=false%
    }
    % Specify the driver for the color package
    \ExecuteOptions{dvips}
    %\ExecuteOptions{xdvi}
  \fi
\fi

\hyperbaseurl{}
\newcommand\hr[1]{\href{#1.dvi}{dvi}, \href{#1.pdf}{pdf}}
\newcommand\h[1]{#1 (\hr{#1})}

\begin{document}

%\tracingcommands=1
\newlength{\hW} % heading box width
\newlength{\pW} % page number field width
\settowidth{\hW}{\bf\docname}
\settowidth{\pW}{Page \pageref{lastpage}\ of \pageref{lastpage}}
\ifdim \pW > \hW \setlength{\hW}{\pW} \fi
\makeatletter
\def\@biblabel#1{#1.}
\newcommand{\ps@twolines}{%
  \renewcommand{\@oddhead}{%
    \docdate\hfill\parbox[t]{\hW}{{\hfill\bf\docname}\newline
                          Page \thepage\ of \pageref{lastpage}}}%
\renewcommand{\@evenhead}{}%
\renewcommand{\@oddfoot}{}%
\renewcommand{\@evenfoot}{}%
}%
\makeatother
\pagestyle{twolines}

\vspace{-10pt}
\begin{tabbing}
\phantom{References: }\= \\
To: \>Van\\
Subject: \>Radii of curvature of an oblate spheroid\\
From: \>Van Snyder\\
Reference: \>Harry Lass, {\bf Vector and Tensor Analysis}, McGraw-Hill
           (1950), pp 71-78\\
\end{tabbing}

\renewcommand{\d}{\text{d}}

\parindent 0pt \parskip 6pt
\vspace{-20pt}

\section{Definitions}

Let $a$ be the equatorial radius, and $c$ the polar radius of the Earth,
an oblate spheroid, and let
%
\begin{equation}
\mathbf{r}(\lambda,\theta) = \left[ \begin{array}{l}
x(\lambda,\theta) =  a \cos \theta \cos \lambda \\
y(\lambda,\theta) = a \cos \theta \sin \lambda \\
z(\lambda,\theta) = c \sin \theta \\
\end{array}\right] \,,
\end{equation}

be the vector to a point on its surface, where $\lambda$ is longitude and
$\theta$ is geocentric latitude.

Then

\begin{equation}
\begin{array}{lll}
\frac{\partial \mathbf{r}}{\partial \lambda} =
\left[ \begin{array}{l}
-a \cos \theta \sin \lambda \\
a \cos \theta \cos \lambda \\
0
\end{array} \right] &
\frac{\partial \mathbf{r}}{\partial \theta} =
\left[ \begin{array}{l}
-a \sin \theta \cos \lambda \\
-a \sin \theta \sin \lambda \\
c\, \cos \theta \\
\end{array}
\right] & \\
\frac{\partial^2 \mathbf{r}}{\partial \lambda^2} =
\left[ \begin{array}{l}
-a \cos \theta \cos \lambda \\
-a \cos \theta  \sin \lambda\\
0
\end{array} \right] &
\frac{\partial^2 \mathbf{r}}{\partial \lambda\, \partial \theta} =
\left[ \begin{array}{l}
a \sin \theta \sin \lambda \\
-a \sin \theta  \sin \lambda\\
0 \\
\end{array} \right ] &
\frac{\partial^2 \mathbf{r}}{\partial \theta^2} =
\left[ \begin{array}{l}
-a \cos \theta \cos \lambda \\
-a \cos \theta \sin \lambda \\
-c\, \sin \theta
\end{array} \right] \,. \\
\end{array}
\end{equation}

If a surface is cut by a plane that includes its normal at a point, the
intersection of the surface and that plane is a curve in that plane. 
Assuming the surface is smooth, the curvature of the curve at the point of
the normal is defined.

Among all the planes that include the normal, the curvature of the curve
in one of the planes is maximum, and the curvature in another one is
minimum.  These curvatures are the \emph{principal curvatures}, denoted
$\kappa_1$ and $\kappa_2$, respectively, which are roots of

\begin{equation}
(E G - F^2)\, \kappa^2 - ( e G + g E - 2 f F )\, \kappa + ( e g - f^2 ) = 0
\,,
\end{equation}

(Equation (125) from {\bf Vector and Tensor Analysis}) where $E$, $F$, and
$G$ are coefficients of the first fundamental form for a surface:

\begin{equation}
\text{d} s^2 = E\, \text{d} \lambda^2 +
               2 F\, \text{d} \lambda\, \text{d} \theta +
               G \, \text{d} \theta^2
\end{equation}

where, for an oblate spheroid,

\begin{equation}\begin{split}
E = \,& \frac{\partial \mathbf{r}}{\partial \lambda} \cdot
        \frac{\partial \mathbf{r}}{\partial \lambda}
  = a^2 \cos^2 \theta \\
F = \,& \frac{\partial \mathbf{r}}{\partial \lambda} \cdot
        \frac{\partial \mathbf{r}}{\partial \theta} = 0 \\
G = \,& \frac{\partial \mathbf{r}}{\partial \theta} \cdot
        \frac{\partial \mathbf{r}}{\partial \theta}
  = a^2 \sin^2 \theta + c^2 \cos^2 \theta
\end{split}\end{equation}

and $e$, $f$, and $g$ are coefficients of the second fundamental form:

\begin{equation}
2\, D = e\, \text{d} \lambda^2 +
        2 f\, \text{d} \lambda\, \text{d} \theta +
        g \, \text{d} \theta^2
\end{equation}

where

\begin{equation}\begin{split}
g = \,& \mathbf{n} \cdot
        \frac{\partial^2 \mathbf{r}}{\partial \theta^2}
  = \frac{a c}{\sqrt{G}} \\
f = \,& \mathbf{n} \cdot
        \frac{\partial^2 \mathbf{r}}{\partial \lambda \partial \theta}
  = 0 \\
e = \,& \mathbf{n} \cdot
        \frac{\partial^2 \mathbf{r}}{\partial \lambda^2}
  = g \cos^2 \theta \,,\\
\end{split}\end{equation}

$\mathbf{n}$ is the unit normal to the surface at $(\lambda,\theta)$,
$\d {\bf r} = {\bf r}(\lambda + \d \lambda, \theta + \d \theta) - {\bf
r}(\lambda, \theta)$ and $D = \d {\bf r}\cdot {\bf n}$ is the distance
from the point at ${\bf r} + \d \mathbf{r}$ to the tangent plane at
$\bf{r}(\lambda,\theta)$. The vectors $\frac{\partial \mathbf{r}}{\partial
\lambda}$ and $\frac{\partial \mathbf{r}}{\partial \theta}$ are tangent to
the surface. Therefore, a vector normal to the surface at
$\bf{r}(\lambda,\theta)$ is

\begin{equation}
\mathbf{N} = \frac{\partial \mathbf{r}}{\partial \lambda} \times
             \frac{\partial \mathbf{r}}{\partial \theta}\,,
           =
           \left[ \begin{array}{l}
             a\, c\, \cos \lambda \cos^2 \theta \\
             a\, c\, \sin \lambda \cos^2 \theta \\
             a^2\, \cos \lambda \sin \theta
            \end{array} \right],\,
\mathbf{n} = \frac{\mathbf{N}}{|\mathbf{N}|}
= \frac{1}{\sqrt{G}} \left[ \begin{array}{l}
  c \cos \theta \cos \lambda \\
  c \cos \theta \sin \lambda \\
  a \sin \theta \\
  \end{array} \right],
\text{ and }
| \mathbf{N} | = a \cos\theta \sqrt{G}
\,.
\end{equation}

\section{Principal curvatures}

The principal curvatures are

\begin{equation}\begin{split}
\kappa_1 =\,& \frac{a c}{G^\frac32}
 \, \text{ and} \\
\kappa_2 =\,& \frac{c}{a \sqrt{G}} \\
\end{split}\end{equation}

The maximum and minimum radii of curvature are $R_1 = \frac1{\kappa_1}$
and $R_2 = \frac1{\kappa_2}$.

\section{Gaussian curvature of oblate spheroid}

The Gaussian curvature $K$ is the product of the principal curvatures:

\begin{equation}
K = \kappa_1 \kappa_2
= \frac{ e g - f^2 }{ E G -F^2}
= \frac{ e g }{ E G }
= \frac{c^2}{G^2}
\end{equation}

and the radius of Gaussian curvature $\sqrt{1/K} = \frac{G}c$ is the
geometric mean of the principal radii of curvature.

\section{Mean curvature of oblate spheroid}

The mean curvature $H$ is the average of the principal curvatures:

\begin{equation}
H = \frac12(\kappa_1 + \kappa_2) = \frac{e G + g E - 2 f F}{2\,(E G -F^2)}
= \frac{e G + g E}{2\, E\, G}
= \frac{c(a^2+G)}{2\,a \,G^{3/2}}
\end{equation}

and the radius of mean curvature is $1/H$.  The average of the principal
radii of curvature is

\begin{equation}
\frac12 \left( \frac1{\kappa_1} + \frac1{\kappa_2} \right)
= \frac{H}{K}
= \frac{(a^2+G) \sqrt{G}}{2\,a\,c}
\neq \frac2{\kappa_1 + \kappa_2}\,.
\end{equation}

\section{Radius in prime vertical and meridional radius of curvature}

The radius $R_N$ in the prime vertical is usually denoted $N$ for the
distance from the surface to the polar axis in the direction normal to the
surface.  It's not a radius of curvature.  Using $\phi$ for geodetic
latitude ($c^2 \tan \phi = a^2 \tan \theta$),

\begin{equation}
R_N =
N = \frac{a^2}{\sqrt{a^2 \cos^2 \phi + c^2 \sin^2 \phi}}
  = a^2 \sqrt{\frac{a^4 \cos^2 \theta + c^4 sin^2 \theta}
                   {a^6 \cos^2 \theta + c^6 sin^2 \theta}}
\end{equation}

The radius of curvature $R_M$ in the meridian is the radius of a circle
that is tangent to the Earth surface and has the same radius of curvature
as the ellipsoid in the meridional direction.  Using $\phi$ for geodetic
latitude and eccentricity $e = 1 - \left(\frac{c}{a}\right)^2$,

\begin{equation}
R_M = \frac{a(1-e^2)}{(1 - e^2 \sin^2 \phi)^{\frac32}}
    = \frac{(ac)^2}{(a^2 \cos^2 \phi + c^2 \sin^2 \phi)^{\frac32}}
    = (a c)^2 \left( \frac{a^4 \cos^2 \theta + c^4 \sin^2 \theta}
                           {a^6 \cos^2 \theta + c^6 \sin^2 \theta}
               \right)^{\frac32}
\end{equation}

\section{Curvature of a sphere}

For a sphere, $a = c = r$, $G = r^2$, $K = 1/r^2$, $H = 1/r$, $\kappa_1 =
1/r$, $\kappa_2 = 1/r$, and $N = r$, as expected.

\label{lastpage}
\vspace*{-0.1in} % Somehow, this causes lastpage to be defined
\end{document}

% $Id$

% $Log$
% Revision 1.2  2017/10/13 19:10:51  vsnyder
% Add CVS stuff
%
