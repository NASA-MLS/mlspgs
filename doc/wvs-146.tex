% Making the graphics:
% tgif -print -color -eps wvs-146-1.obj; epstopdf wvs-146-1.eps
% tgif -print -color -eps wvs-146-2.obj; epstopdf wvs-146-2.eps
% tgif -print -color -eps wvs-146-3.obj; epstopdf wvs-146-3.eps
% tgif -print -color -jpeg wvs-146-1.obj
% tgif -print -color -jpeg wvs-146-2.obj
% tgif -print -color -jpeg wvs-146-3.obj

\documentclass[11pt]{article}
\usepackage{alltt}
\usepackage[fleqn]{empheq} % 
\usepackage{amsmath}
\usepackage{floatflt}
\usepackage{graphicx}
\usepackage{longtable}
\usepackage[strings]{underscore}

\textwidth 6.5in
\oddsidemargin -0.25in
%\evensidemargin -0.5in
\topmargin -0.5in
\textheight 9in

\newcommand{\docname}{wvs-146r2}
\newcommand{\docdate}{5 January 2020}

\ifx\pdfoutput\undefined
  \pdfoutput=0
  \usepackage[hypertex,plainpages,hyperindex=true]{hyperref}
  \hypersetup{%
    hypertexnames=false%
  }
  % Specify the driver for the color package
  \ExecuteOptions{dvips}
  %\ExecuteOptions{xdvi}
\else
  \ifnum\pdfoutput>0
   
\usepackage[pdftex,plainpages,hyperindex=true,pdfpagelabels]{hyperref}
    \hypersetup{%
      hypertexnames=false,%
      colorlinks=true,%
      linktocpage=true,%
    }
    % Specify the driver for the color package
    \ExecuteOptions{pdftex}
  \else
    \usepackage[hypertex,plainpages,hyperindex=true]{hyperref}
    \hypersetup{%
      hypertexnames=false%
    }
    % Specify the driver for the color package
    \ExecuteOptions{dvips}
    %\ExecuteOptions{xdvi}
  \fi
\fi

\hyperbaseurl{}
\newcommand\hr[1]{\href{#1.dvi}{dvi}, \href{#1.pdf}{pdf}}
\newcommand\h[1]{#1 (\hr{#1})}

\begin{document}

%\tracingcommands=1
\newlength{\hW} % heading box width
\newlength{\pW} % page number field width
\settowidth{\hW}{\bf\docname}
\settowidth{\pW}{Page \pageref{lastpage}\ of \pageref{lastpage}}
\ifdim \pW > \hW \setlength{\hW}{\pW} \fi
\makeatletter
\def\@biblabel#1{#1.}
\newcommand{\ps@twolines}{%
  \renewcommand{\@oddhead}{%
    \docdate\hfill\parbox[t]{\hW}{{\hfill\bf\docname}\newline
                          Page \thepage\ of \pageref{lastpage}}}%
\renewcommand{\@evenhead}{}%
\renewcommand{\@oddfoot}{}%
\renewcommand{\@evenfoot}{}%
}%
\makeatother
\pagestyle{twolines}

\vspace{-10pt}
\begin{tabbing}
\phantom{References: }\= \\
To: \>Van\\
Subject: \>Radii of curvature of an oblate spheroid\\
From: \>Van Snyder\\
Reference: \>Harry Lass, {\bf Vector and Tensor Analysis}, McGraw-Hill
             (1950), pp 71-78\\
           \>R.~E.~Deaton, \emph{The Normal Section
             Curve of an Ellipsoid}, November 2009,\\
           \>{\tt http://www.mygeodesy.id.au/documents/NormalSection.pdf}\\
           \>M. Ligas, \emph{Various parameterizations of ``latitude''
                equation -- Cartesian to geodetic} \\
           \>\emph{coordinates transformation}, {\bf Journal of Geodetic
                Science 3}, 2 (2013) pp 87-94. \\ \>B. R. Bowring,
           \emph{Notes on the curvature in the prime vertical section}, \\
           \>{\bf Survey Review 29}:226 (October 1987) pp 195-196
\end{tabbing}

\renewcommand{\d}{\text{d}}

\parindent 0pt \parskip 6pt
\vspace{-20pt}

%=========================================================================
\section{Definitions}

Let $a$ be the equatorial radius, and $c$ the polar radius of the Earth,
an oblate spheroid
%
\begin{equation}
\Sigma = \frac{x^2}{a^2} + \frac{y^2}{a^2} + \frac{z^2}{c^2} = 1
\text{ or }
x^2 +y^2 + \frac{z^2}{1-e^2} = a^2
\end{equation}
where eccentricity $e = 1-\frac{c^2}{a^2}$.

\begin{centering}
\includegraphics[scale=0.75]{wvs-146-3}\\[5pt]\label{latitudes}
{\bf Latitude definitions} $\theta$: Reduced latitude;
$\gamma$: Geocentric latitude; $\phi$: Geodetic latitude\\
\end{centering}

%=========================================================================
\section{Curvatures in terms of reduced latitude $\theta$}
Let
%
\begin{equation}\begin{split}\label{first}
\mathbf{r}(\lambda,\theta) = \,& \left[ \begin{array}{l}
x(\lambda,\theta) \\
y(\lambda,\theta) \\
z(\lambda,\theta) \\
\end{array}\right] =
\left[ \begin{array}{l}
a \cos \theta \cos \lambda \\
a \cos \theta \sin \lambda \\
c \sin \theta \\
\end{array}\right]
=a \left[ \begin{array}{l}
 \cos \theta \cos \lambda \\
 \cos \theta \sin \lambda \\
 \sqrt{1-e^2}\, \sin \theta \\
 \end{array}\right] \text{ and } \\[5pt]
\rho(\theta) = \,&
 \sqrt{x(\lambda,\theta)^2 + y(\lambda,\theta)^2} = a \cos \theta
\end{split}\end{equation}

be the vector to a point on its surface, where $\lambda$ is longitude and
$\theta$ is the reduced latitude.\footnote{Called the \emph{eccentric
anomaly} in astronomy.} Then

\begin{equation}
\begin{array}{lll}
\frac{\partial \mathbf{r}}{\partial \lambda} =
\left[ \begin{array}{l}
-a \cos \theta \sin \lambda \\
a \cos \theta \cos \lambda \\
0
\end{array} \right] &
\frac{\partial \mathbf{r}}{\partial \theta} =
\left[ \begin{array}{l}
-a \sin \theta \cos \lambda \\
-a \sin \theta \sin \lambda \\
c\, \cos \theta \\
\end{array}
\right] & \\
 & & \\
\frac{\partial^2 \mathbf{r}}{\partial \lambda^2} =
\left[ \begin{array}{l}
-a \cos \theta \cos \lambda \\
-a \cos \theta  \sin \lambda\\
0
\end{array} \right] &
\frac{\partial^2 \mathbf{r}}{\partial \lambda\, \partial \theta} =
\left[ \begin{array}{l}
a \sin \theta \sin \lambda \\
-a \sin \theta \cos \lambda\\
0 \\[10pt]
\end{array} \right ] &
\frac{\partial^2 \mathbf{r}}{\partial \theta^2} =
\left[ \begin{array}{l}
-a \cos \theta \cos \lambda \\
-a \cos \theta \sin \lambda \\
-c\, \sin \theta
\end{array} \right] \,. \\
\end{array}
\end{equation}

If a surface is cut by a plane that includes its normal at a point, the
intersection of the surface and that plane is a curve in that plane. 
Assuming the surface is smooth, the curvature of the curve at the point of
the normal is defined.

Among all the planes that include the normal, the curvature of the curve
in one of the planes is maximum, and the curvature in another one is
minimum.  These curvatures are the \emph{principal curvatures}, denoted
$\kappa_1$ and $\kappa_2$, respectively, which are roots of

\begin{equation}\label{third}
(E G - F^2)\, \kappa^2 - ( e G + g E - 2 f F )\, \kappa + ( e g - f^2 ) = 0
\,,
\end{equation}

(Equation (125) from {\bf Vector and Tensor Analysis}) where $E$, $F$, and
$G$ are coefficients of the first fundamental form for a surface:

\begin{equation}
\text{d} s^2 = E\, \text{d} \lambda^2 +
               2 F\, \text{d} \lambda\, \text{d} \theta +
               G \, \text{d} \theta^2 \,.
\end{equation}

For an oblate spheroid,

\begin{equation}\begin{split}
E = \,& \frac{\partial \mathbf{r}}{\partial \lambda} \cdot
        \frac{\partial \mathbf{r}}{\partial \lambda}
  = a^2 \cos^2 \theta \\
F = \,& \frac{\partial \mathbf{r}}{\partial \lambda} \cdot
        \frac{\partial \mathbf{r}}{\partial \theta} = 0 \\
G = \,& \frac{\partial \mathbf{r}}{\partial \theta} \cdot
        \frac{\partial \mathbf{r}}{\partial \theta}
  = a^2 \sin^2 \theta + c^2 \cos^2 \theta = a^2 ( 1 - e^2 \cos^2 \theta )
\,.
\end{split}\end{equation}

$e$, $f$, and $g$ are coefficients of the second fundamental form:

\begin{equation}
2\, d = e\, \text{d} \lambda^2 +
        2 f\, \text{d} \lambda\, \text{d} \theta +
        g \, \text{d} \theta^2
\end{equation}

where $d = \d {\bf r}\cdot {\bf n}$ is the distance from the point at
${\bf r} + \d \mathbf{r}$ to the tangent plane at
$\bf{r}(\lambda,\theta)$, $\d {\bf r} = {\bf r}(\lambda + \d \lambda,
\theta + \d \theta) - {\bf r}(\lambda, \theta)$, and $\mathbf{n}$ is the
unit normal to the surface at $(\lambda,\theta)$.  The vectors
$\frac{\partial \mathbf{r}}{\partial \lambda}$ and $\frac{\partial
\mathbf{r}}{\partial \theta}$ are tangent to the surface.  Therefore, a
vector normal to the surface at $\bf{r}(\lambda,\theta)$ is

\begin{equation}\begin{split}
\mathbf{N} =\,& \frac{\partial \mathbf{r}}{\partial \lambda} \times
             \frac{\partial \mathbf{r}}{\partial \theta}\,
           =
           \left[ \begin{array}{l}
             a\, c\, \cos \lambda \cos^2 \theta \\
             a\, c\, \sin \lambda \cos^2 \theta \\
             a^2\, \sin \theta \cos \theta
            \end{array} \right]
 = a^2 c\, \nabla \Sigma,\,
\mathbf{n} = \frac{\mathbf{N}}{|\mathbf{N}|}
= \frac{\nabla\Sigma}{|\nabla\Sigma|}
= \frac{1}{\sqrt{G}} \left[ \begin{array}{l}
  c \cos \theta \cos \lambda \\
  c \cos \theta \sin \lambda \\
  a \sin \theta \\
  \end{array} \right],\, \\
| \mathbf{N} | =\,& a \cos\theta \sqrt{G}
\, \text{, and } \\
\end{split}\end{equation}
\vspace*{-0.3in}
\begin{equation}\begin{split}
g = \,& \mathbf{n} \cdot
        \frac{\partial^2 \mathbf{r}}{\partial \theta^2}
  = \frac{a\, c}{\sqrt{G}} \\
f = \,& \mathbf{n} \cdot
        \frac{\partial^2 \mathbf{r}}{\partial \lambda \partial \theta}
  = 0 \\
e = \,& \mathbf{n} \cdot
        \frac{\partial^2 \mathbf{r}}{\partial \lambda^2}
  = g \cos^2 \theta = \frac{a\, c \cos^2 \theta}{\sqrt{G}} \,.\\
\end{split}\end{equation}

% If the normal vector is computed in Cartesian co\"ordinates
% 
% {\renewcommand{\arraystretch}{1.1}
% \begin{equation}
% \mathbf{N}^\prime = \frac12 \nabla\left( \frac{x^2}{a^2} + \frac{y^2}{a^2} +
%                                   \frac{z^2}{c^2} \right) =
%  \left[\begin{array}{c} \frac{x}{a^2} \\ \frac{y}{a^2} \\ \frac{z}{c^2} \\
%        \end{array} \right] =
%  \left[\begin{array}{c} \frac1a \cos\theta \cos\lambda \\
%                         \frac1a \cos\theta \sin\lambda \\
%                         \frac1c \sin\theta \\
%        \end{array} \right]
% \text{ and } | \mathbf{N}^\prime | = \frac{\sqrt{G}}{a c},
% \end{equation}
% }
% 
% then $\mathbf{n}^\prime = \frac{\mathbf{N}^\prime}{|\mathbf{N}^\prime|} =
% \mathbf{n}$, and
% the coefficients of the second normal form are unchanged, as expected.

%=========================================================================
\subsection{Principal curvatures}

With $F = f = 0$, the roots of Equation (\ref{third}) (the principal
curvatures), are

\begin{equation}\begin{split}\label{principal-reduced}
\kappa_1 =\,& \frac{g}G = \frac{a\, c}{G^\frac32}
         = a^2 \frac{\sqrt{1-e^2}}{G^\frac32} \\[2pt]
\kappa_2 =\,& \frac{e}E = \frac{c}{a \sqrt{G}}
         = \sqrt{\frac{1-e^2}{G}} = \frac{G}{a^2}\,\kappa_1\\[2pt]
\kappa_1 - \kappa_2
         = \,& \frac{c( a^2 - c^2)}{a}\, \frac{\cos^2 \theta }
                                              {G^{\frac32}}
         = e^2 \cos^2 \theta\, \kappa_1 > 0 \,.\\
\end{split}\end{equation}

The minimum and maximum radii of curvature are $R_1 = \frac1{\kappa_1}$
and $R_2 = \frac1{\kappa_2}$.

\subsection{Gaussian curvature of oblate spheroid}

The Gaussian curvature $K$ is the product of the principal curvatures:

\begin{equation}
K = \kappa_1 \kappa_2
= \frac{ e g - f^2 }{ E G -F^2}
= \frac{ e g }{ E G }
= \frac{c^2}{G^2}
\end{equation}

and the radius of Gaussian curvature $\sqrt{1/K} = \frac{G}c$ is the
geometric mean of the principal radii of curvature.

\subsection{Mean curvature of oblate spheroid}

The mean curvature $H$ is the average of the principal curvatures:

\begin{equation}
H = \frac12(\kappa_1 + \kappa_2) = \frac{e G + g E - 2 f F}{2\,(E G -F^2)}
= \frac{e G + g E}{2\, E\, G}
= \frac{c(a^2+G)}{2\,a \,G^{3/2}}
\end{equation}

and the radius of mean curvature is $1/H$.  The average of the principal
radii of curvature is

\begin{equation}
\frac12 \left( \frac1{\kappa_1} + \frac1{\kappa_2} \right)
= \frac12 \left( R_1 + R_2 \right)
= \frac{H}{K}
= \frac{(a^2+G) \sqrt{G}}{2\,a\,c}
\neq \frac2{\kappa_1 + \kappa_2}\,.
\end{equation}

%=========================================================================
\section{Development of curvatures using geodetic latitude $\phi$}\label{Geod}

Let $D = a^2 \cos^2 \phi + c^2 \sin^2 \phi = a^2 ( 1 - e^2 \sin^2 \phi )$
and $N(\phi) = \frac{a^2}{\sqrt{D}}$.  Then the parametric equations of an
oblate spheroid using longitude and geodetic latitude derived in Section
\ref{Rel},\footnote{See also Ligas} are

\begin{equation}\begin{split}
\mathbf{r}(\lambda,\phi) = \,&
\left[ \begin{array}{l}
x(\lambda,\phi) \\ y(\lambda,\phi) \\ z(\lambda,\phi) \\ \end{array} \right] =
\frac1{\sqrt{D}}
\left[ \begin{array}{l}
 a^2\, \cos \phi \cos \lambda \\
 a^2\, \cos \phi \sin \lambda \\
 c^2\, \sin \phi \\
\end{array} \right]
= N(\phi) \left[ \begin{array}{l}
 \cos \phi \cos \lambda \\
 \cos \phi \sin \lambda \\
 (1-e^2) \sin \phi \\
\end{array} \right]
 \text{ and } \\
\rho(\phi) = \,& \sqrt{x(\lambda,\phi)^2 + y(\lambda,\phi)^2} = 
 \frac{a^2}{\sqrt{D}} \cos \phi = N(\phi)\,\cos\phi\,. \\
\end{split}\end{equation}

The derivatives are

\begin{equation}\begin{array}{ll}
\frac{\partial \mathbf{r}}{\partial \lambda} =
 \frac{a^2}{\sqrt{D}}
 \left[ \begin{array}{l}
  -\cos \phi \sin \lambda \\
  \cos \phi \cos \lambda \\
  0 \\
 \end{array} \right]
&
\frac{\partial \mathbf{r}}{\partial \phi} =
 \frac{a^2 c^2}{D^{\frac32}}
 \left[ \begin{array}{l}
  -\sin \phi \cos \lambda \\
  -\sin \phi \sin \lambda\\
  \cos \phi\\
 \end{array} \right]\\[3pt]
\\
\frac{\partial^2 \mathbf{r}}{\partial \lambda^2} =
 \frac{a^2}{\sqrt{D}}
 \left[ \begin{array}{l}
  -\cos \phi \cos \lambda \\
  -\cos \phi \sin \lambda \\
  0 \\
 \end{array} \right]
&
\frac{\partial^2 \mathbf{r}}{\partial \lambda \partial \phi} =
 \frac{a^2 c^2}{D^{\frac32}}
 \left[ \begin{array}{l}
 \sin \phi \sin \lambda \\
 -\sin \phi \cos \lambda \\
 0 \\
 \end{array} \right] \\
&\\[-5pt]&
\frac{\partial \mathbf{r}}{\partial \phi^2} =
 \frac{a^2 c^2}{D^{\frac52}}
 \left[ \begin{array}{l}
 \cos \phi \cos \lambda ( 2 ( c^2 - a^2 ) \sin^2 \phi - a^2 ) \\
 \cos \phi \sin \lambda ( 2 ( c^2 - a^2 ) \sin^2 \phi - a^2 ) \\
 \sin \phi ( 2 ( a^2 - c^2 ) \cos^2 \phi - c^2 ) \\
 \end{array} \right] \,. \\
\end{array}
\end{equation}

The normal vector is

\begin{equation}
\mathbf{N} = \frac{\partial \mathbf{r}}{\partial \lambda} \times
             \frac{\partial \mathbf{r}}{\partial \theta}\,
           =
           \frac{a^4 c^2}{D^2}
           \left[ \begin{array}{l}
             \cos^2 \phi \cos \lambda\\
             \cos^2 \phi \sin \lambda\\
             \cos \phi \sin \phi\\
           \end{array} \right],\,
\mathbf{n} = \frac{\mathbf{N}}{|\mathbf{N}|}
= \left[ \begin{array}{l}
  \cos \phi \cos \lambda \\
  \cos \phi \sin \lambda \\
  \sin \phi \\
  \end{array} \right],
\text{ and }
| \mathbf{N} | = \frac{a^4 c^2 \cos\phi}{D^2}
\,.
\end{equation}

The coefficients of the first fundamental form are
%
\begin{equation}\begin{split}
E = \,& \frac{\partial \mathbf{r}}{\partial \lambda} \cdot
        \frac{\partial \mathbf{r}}{\partial \lambda}
  = \frac{a^4 \cos^2 \phi}{D} \\
F = \,& \frac{\partial \mathbf{r}}{\partial \lambda} \cdot
        \frac{\partial \mathbf{r}}{\partial \phi} = 0 \\
G = \,& \frac{\partial \mathbf{r}}{\partial \phi} \cdot
        \frac{\partial \mathbf{r}}{\partial \phi}
  = \frac{a^4 c^4}{D^3} \,. \\
\end{split}\end{equation}

The coefficients of the second fundamental form are
%
\begin{equation}\begin{split}
e = \,& \mathbf{n} \cdot
        \frac{\partial^2 \mathbf{r}}{\partial \lambda^2}
  = \frac{a^2 \cos^2 \phi}{\sqrt{D}} = \sqrt{E} \cos \phi \\
f = \,& \mathbf{n} \cdot
        \frac{\partial^2 \mathbf{r}}{\partial \lambda \partial \phi}
  = 0 \\
g = \,& \mathbf{n} \cdot
        \frac{\partial^2 \mathbf{r}}{\partial \phi^2}
  = \frac{a^2 c^2}{D^{\frac32}} = \sqrt{G} \,. \\
\end{split}\end{equation}

The principal curvatures are
%
\begin{equation}\begin{split}\label{principal-geod}
\kappa_1 =\,& \frac{g}G = \frac{D^\frac32}{a^2 c^2}
         = \frac1g = \frac1{\sqrt{G}}
 \, \text{ and} \\[5pt]
\kappa_2 =\,& \frac{e}E = \frac{\sqrt{D}}{a^2} \,, \\
\end{split}\end{equation}

which, upon substituting $a \tan \theta = c \tan \phi$ become the same as
Equations (\ref{principal-reduced}).  The Gaussian curvature is
%
\begin{equation}
K = \kappa_1 \kappa_2 = \frac{D^2}{a^4 c^2} \,.
\end{equation}

The radius of the Gaussian curvature = $\sqrt{1/K} = \frac{a^2 c}D$.

The mean curvature $H$ is the average of the principal curvatures
%
\begin{equation}
H = \frac12 ( \kappa_1 + \kappa_2 ) = \frac{\sqrt{D} ( D+c^2 )}
                                           {2 a^2 c^2} \,.
\end{equation}

The average of the principal radii of curvature is
%
\begin{equation}
\frac12 \left( \frac1{\kappa_1} + \frac1{\kappa_2} \right) =
\frac12 ( R_1 + R_2 ) = \frac{a^2 ( D + c^2 )}{2 D^{\frac32}} \,.
\end{equation}

%=========================================================================
\section{Development of curvatures using geocentric latitude $\gamma$}

Using $\gamma$ for geocentric latitude,

\begin{equation}
r(\lambda,\gamma) = \left[ \begin{array}{l}
x(\lambda,\gamma) =  R(\gamma) \cos \gamma \cos \lambda \\
y(\lambda,\gamma) =  R(\gamma) \cos \gamma \sin \lambda \\
z(\lambda,\gamma) =  R(\gamma) \sin \gamma
\end{array} \right] \text{ and } \rho(\gamma) =
\sqrt{x(\lambda,\gamma)^2 + y(\lambda,\gamma)^2} = R(\gamma) \cos \gamma\,.
\end{equation}

Substitute $r(\lambda,\gamma)$ into the equation for an oblate spheroid
\begin{equation}\begin{split}\label{R}
\frac{x^2}{a^2} + \frac{y^2}{a^2} + \frac{z^2}{c^2} = \,&
R(\gamma)^2 \left( \frac{\cos^2 \gamma \cos^2 \lambda}{a^2} +
           \frac{\cos^2 \gamma \sin^2 \lambda}{a^2} +
           \frac{\sin^2 \gamma}{c^2} \right) \\
        = \,& R(\gamma)^2
           \left( \frac{\cos^2 \gamma}{a^2} +
                  \frac{\sin^2 \gamma}{c^2} \right) =
        R(\gamma)^2\, \frac{1-e^2 \cos^2 \gamma}{a^2 (1-e^2)}= 1 \\
\end{split}\end{equation}

from which
%
\begin{equation}
R(\gamma) = \frac{ac}{\sqrt{a^2 \sin^2 \gamma + c^2 \cos^2 \gamma}}
 = a \sqrt{\frac{1-e^2}{1-e^2 \cos^2 \gamma}} \,.
\end{equation}

The derivatives are
%
\begin{equation}
\begin{array}{lll}
\frac{\partial \mathbf{r}}{\partial \lambda} =
R(\gamma) \left[ \begin{array}{l}
- \cos \gamma \sin \lambda \\
  \cos \gamma \cos \lambda \\
  0
\end{array} \right] &
\frac{\partial \mathbf{r}}{\partial \gamma} =
\frac{R(\gamma)^3}{c^2} \left[ \begin{array}{l}
- \sin \gamma \cos \lambda \\
- \sin \gamma \sin \lambda \\
  \frac{c^2}{a^2} \cos \gamma \\
\end{array}
\right] & \\
 & & \\
\frac{\partial^2 \mathbf{r}}{\partial \lambda^2} =
R(\gamma) \left[ \begin{array}{l}
- \cos \gamma \cos \lambda \\
- \cos \gamma  \sin \lambda\\
0
\end{array} \right] &
\frac{\partial^2 \mathbf{r}}{\partial \lambda\, \partial \gamma} =
\frac{R(\gamma)^3}{c^2} \left[ \begin{array}{l}
 \sin \gamma \sin \lambda \\
-\sin \gamma \cos \lambda\\
0 \\
\end{array} \right ] \\
& \\[-3pt] &
\frac{\partial^2 \mathbf{r}}{\partial \gamma^2} =
2\,\frac{R(\gamma)^5}{a^2 c^2}
\left[ \begin{array}{l}
 \cos \gamma \cos \lambda \left( \frac{a^2}{R(\gamma)^2} - \frac32 \right) \\
 \cos \gamma \sin \lambda \left( \frac{a^2}{R(\gamma)^2} - \frac32 \right) \\
 \sin \gamma \left( \frac{c^2}{R(\gamma)^2} -\frac32 \right) \\
\end{array} \right] \,. \\
\end{array}
\end{equation}

Let $\Delta = a^4 \sin^2 \gamma + c^4 \cos^2 \gamma
= a^4 \left( \sin^2\gamma + (1-e^2)^2 \cos^2\gamma \right)$. Then the normal
vector is
%
\begin{equation}\begin{split}
\mathbf{N} = \,& \frac{\partial \mathbf{r}}{\partial \lambda} \times
                 \frac{\partial \mathbf{r}}{\partial \theta}\,
           =
           \frac{R(\gamma)^4 \cos \gamma}{a^2 c^2}
           \left[ \begin{array}{l}
             c^2 \cos \gamma \cos \lambda\\
             c^2 \cos \gamma \sin \lambda\\
             a^2 \sin \gamma\\
           \end{array} \right],\,
\mathbf{n} = \frac{\mathbf{N}}{|\mathbf{N}|}
= \frac1{\sqrt{\Delta}}
  \left[ \begin{array}{l}
  c^2 \cos \gamma \cos \lambda \\
  c^2 \cos \gamma \sin \lambda \\
  a^2 \sin \gamma \\
  \end{array} \right] \\
| \mathbf{N} | = \,&
 \frac{R(\gamma)^4 \cos \gamma\, \sqrt{\Delta}}
      {a^2 c^2}
\,.
\end{split}\end{equation}

The coefficients of the first fundamental form are
%
\begin{equation}\begin{split}
E = \,& \frac{\partial \mathbf{r}}{\partial \lambda} \cdot
        \frac{\partial \mathbf{r}}{\partial \lambda}
  = R(\gamma)^2 \cos^2 \gamma \\
F = \,& \frac{\partial \mathbf{r}}{\partial \lambda} \cdot
        \frac{\partial \mathbf{r}}{\partial \gamma} =  0 \\
G = \,& \frac{\partial \mathbf{r}}{\partial \gamma} \cdot
        \frac{\partial \mathbf{r}}{\partial \gamma}
  = \frac{R(\gamma)^6 \Delta}{a^4 c^4} \\
\end{split}\end{equation}

The coefficients of the second fundamental form are
%
\begin{equation}\begin{split}
e = \,& \mathbf{n} \cdot
        \frac{\partial^2 \mathbf{r}}{\partial \lambda^2}
  = \frac{c^2 R(\gamma)\, \cos^2 \gamma}{\sqrt{\Delta}} \\
f = \,& \mathbf{n} \cdot
        \frac{\partial^2 \mathbf{r}}{\partial \lambda \partial \gamma}
  = 0 \\
g = \,& \mathbf{n} \cdot
        \frac{\partial^2 \mathbf{r}}{\partial \gamma^2}
  = \frac{R(\gamma)^3}{\sqrt{\Delta}} \,. \\
\end{split}\end{equation}

The principal curvatures are
%
\begin{equation}\begin{split}\label{principal-geoc}
\kappa_1 =\,& \frac{g}G = \frac{a^4 c^4}{R(\gamma)^3 \Delta^{\frac32}}
 \, \text{ and} \\[5pt]
\kappa_2 =\,& \frac{e}E = \frac{c^2}{R(\gamma) \, \sqrt{\Delta}} \,, \\
\end{split}\end{equation}

%=========================================================================
\section{Relationship of latitude definitions}
%
Reduced latitude $\theta$, geocentric latitude $\gamma$, and geodetic
latitude $\phi$ are related by
%
\begin{empheq}[box=\fbox]{equation}\label{tangents}
\begin{split}
\frac{z(\lambda,\theta)}{\rho(\theta)} = \,&
\frac{c}{a}\, \frac{\sin \theta}{\cos \theta} =
\frac{c}{a} \tan \theta = \sqrt{1-e^2}\, \tan \theta\\
\frac{z(\lambda,\gamma)}{\rho(\gamma)} = \,&
\frac{\sin \gamma}{\cos \gamma} =
\tan{\gamma}\\
\frac{z(\lambda,\phi)}{\rho(\phi)} = \,&
\frac{c^2}{a^2} \frac{\sin \phi}{\cos \phi} =
\frac{c^2}{a^2} \tan \phi = (1-e^2)\, \tan \phi\\
\end{split}
\end{empheq}

or
%
\begin{empheq}[box=\fbox]{equation}
\begin{split}
\tan \gamma = \,& \frac{c}{a} \tan \theta = \frac{c^2}{a^2} \tan \phi \\
\tan \gamma = \,& \sqrt{1-e^2}\, \tan \theta = (1-e^2)\, \tan \phi \,.
\end{split}
\end{empheq}

%=========================================================================
\section{Prime vertical and meridional radii of curvature}\label{Prime}

The prime vertical is the east-west plane that includes the zenith. The
radius of curvature $R_N$ in the prime vertical is frequently denoted $N$.
It is the distance from the surface to the polar axis in the direction
normal to the surface.\footnote{B. R. Bowring, \emph{Notes on the curvature
in the prime vertical section}, {\bf Survey Review 29}, 226 (October 1987)
pp 195-196, reproduced in the appendix.}  Using $c \tan \phi = a \tan
\theta$ and $c^2 \tan \phi = a^2 \tan \gamma$ from Equations
(\ref{tangents}), the results derived in Section \ref{geometric} and
Equations (\ref{principal-reduced}), (\ref{principal-geod}), and
(\ref{principal-geoc}) are
%
\makeatletter
\newlength{\negph@wd}
\DeclareRobustCommand{\negphantom}[1]{%
  \ifmmode
    \mathpalette\negph@math{#1}%
  \else
    \negph@do{#1}%
  \fi
}
\newcommand{\negph@math}[2]{\negph@do{$\m@th#1#2$}}
\newcommand{\negph@do}[1]{%
  \settowidth{\negph@wd}{#1}%
  \hspace*{-\negph@wd}%
}
\makeatother
%
\begin{equation}\begin{split}\label{RN}
R_N\,& =  \frac{a^2}{\sqrt{D}}
       \negphantom{\frac{a^2}{\sqrt{D}}}
       \phantom{\frac{R(\gamma) \sqrt{\Delta}}{c^2}}
       =  \frac{a^2}{\sqrt{a^2 \cos^2 \phi + c^2 \sin^2 \phi}}
       = \frac{a}{\sqrt{1-e^2 \sin^2 \phi}} \\
   \,& =  \frac{a\sqrt{G}}{c}
       \negphantom{\frac{a\sqrt{G}}{c}}
       \phantom{\frac{R(\gamma) \sqrt{\Delta}}{c^2}}
       =  \frac{a}{c} \sqrt{a^2 \sin^2 \theta + c^2 \cos^2 \theta}
       = a \sqrt{\frac{1 - e^2 \cos^2 \theta}{1-e^2}} \\
   \,& =  \frac{R(\gamma) \sqrt{\Delta}}{c^2}
       =  \frac{a}{c}\, \sqrt{\frac{a^4 \sin^2 \gamma + c^4 \cos^2 \gamma}
                                   {a^2 \sin^2 \gamma + c^2 \cos^2 \gamma}}
       = a \sqrt{\frac{\sin^2\gamma + (1-e^2)^2 \cos^2 \gamma}
                      {(1-e^2)(1 - e^2 \cos^2 \gamma)}} \\
   \,& =  R_2 \, \text{, the maximum radius of curvature.} \\
\end{split}\end{equation}

The radius of curvature $R_M$ in the meridian is the radius of a circle in
the plane of a meridian that is tangent to the Earth surface, and has the
same radius of curvature as the ellipsoid in the meridional direction. 
Using $c \tan \phi = a \tan \theta$ and $c^2 \tan \phi = a^2 \tan \gamma$
from Equations (\ref{tangents}), the results derived in  Equations
(\ref{principal-reduced}), (\ref{principal-geod}), and
(\ref{principal-geoc}) are
%
\begin{equation}\begin{split}
R_M\,& = \frac{1-e^2}{a^2} R_N^3 \\
   \,& = \frac{(a c)^2}{D^{\frac32}}
       \negphantom{\frac{(a c)^2}{D^{\frac32}}}
       \phantom{\frac{R(\gamma)^3 \Delta^{\frac32}}{a^4 c^4}}
       = \frac{(ac)^2}{(a^2 \cos^2 \phi + c^2 \sin^2 \phi)^{\frac32}}
       = \frac{a(1-e^2)}{(1 - e^2 \sin^2 \phi )^\frac32} \\
   \,& = \frac{G^{\frac32}}{ac}
       \negphantom{\frac{G^{\frac32}}{ac}}
       \phantom{\frac{R(\gamma)^3 \Delta^{\frac32}}{a^4 c^4}}
       = \frac1{ac} \left( a^2 \sin^2 \theta + c^2 \cos^2 \theta \right)
             ^{\frac32}
       = \frac{a}{\sqrt{1-e^2}} ( 1 - e^2 \cos^2 \theta )^\frac32 \\
   \,& = \frac{R(\gamma)^3 \Delta^{\frac32}}{a^4 c^4}
       = \frac1{a\,c}
           \left( \frac{a^4 \sin^2 \gamma + c^4 \cos^2 \gamma}
                       {a^2 \sin^2 \gamma + c^2 \cos^2 \gamma} \right)^\frac32
       = \frac{a}{\sqrt{1-e^2}}
           \left( \frac{\sin^2\gamma + (1-e^2)^2 \cos^2 \gamma}
                       {1 - e^2 \cos^2 \gamma} \right)^\frac32 \\
   \,& = R_1 \,\text{, the minimum radius of curvature.}
\end{split}\end{equation}

%=========================================================================
\section{Curvature of a sphere}

For a sphere, $a = c = r$, $G = r^2$, $K = 1/r^2$, $H = 1/r$, $\kappa_1 =
\kappa_2 = 1/r$, and $R_N = R_M = r$.

%=========================================================================
\section{Geometric computation of prime vertical radius of curvature}
\label{geometric}

A line in the normal direction from the surface of an oblate spheroid

\begin{equation}\label{one}
\Sigma = \frac{x^2+y^2}{a^2} + \frac{z^2}{c^2}
  = \frac{\rho^2}{a^2} + \frac{z^2}{c^2} = 1 \,.
\end{equation}

will intersect the polar axis. The gradient to the ellipse at a point $P =
(x,y,z)$ is
%
\begin{equation}
\vec{g} = \nabla \Sigma = \nabla \left( \frac{x^2 + y^2}{a^2} +
                                   \frac{z^2}{c^2} \right) =
[ \Sigma_x, \Sigma_y, \Sigma_z ]^T =
\left[ \frac{2 \, x}{a^2}, \frac{2 \, y}{a^2}, \frac{2\, z}{c^2} \right]^T
\text{ and } |\vec{g}| = 2 \sqrt{ \frac{\rho^2}{a^4} + \frac{z^2}{c^4}} \,.
\end{equation}

\begin{centering}
\includegraphics{wvs-146-1}\\[5pt]\label{fig}
\end{centering}

Therefore, the slope of a line $\mathcal{L}$, normal to the ellipse at $P$
and intersecting the minor axis at $Q = ( 0, -\zeta )$ is
%
\begin{equation}\label{four}
%m = 
\tan \phi = \frac{\Sigma_z}{\Sigma_{\rho}} = \frac{a^2}{c^2}\, \frac{z}{\rho}
= \frac1{1-e^2} \frac{z}{\rho}
\, \text{, where }
\Sigma_\rho = \sqrt{\Sigma_x^2 + \Sigma_y^2} = \frac{2\, \rho}{a^2}\,.
\end{equation}

% From the equation for a line, $z = m\,x + \zeta$, we have $\zeta = z -
% m\,x = z \left( 1 - \frac{a^2}{c^2} \right)$ and $z - \zeta = m\,x =
% \frac{a^2}{c^2} z$.
% 
% The vector $\vec{R}$ in the direction from $Q$ to $P$ (i.e., parallel to the
% gradient at $P$) with a length equal to the distance from $Q$ to $P$ is
% %
% \begin{equation}\label{five}
% \vec{R} =
% \left[ \begin{array}{c} x \\ z \end{array} \right] -
% \left[ \begin{array}{c} 0 \\ \zeta \end{array} \right] =
% \left[ \begin{array}{c} x \\ \frac{a^2}{c^2}\,z \end{array} \right]
% \text{ and }
% | \vec{R} | = \sqrt{x^2 + \frac{a^4}{c^4} \,z^2}
% \end{equation}
% 
% Substituting \Sigmaquation (\ref{first}) with $\lambda=0$ into \Sigmaquation
% (\ref{five}) we have
% %
% \begin{equation}
% \vec{R} =
% \left[ \begin{array}{c} a \cos \theta \\ \frac{a^2}{c} \sin \theta
% \end{array} \right] \text{ and }
% | \vec{R} | = \frac{a}{c} \sqrt{a^2 \sin^2 \theta + c^2 \cos^2 \theta}
%  = \frac{a \, \sqrt{G}}c = \frac1{\kappa_2} = R_2 \,.
% \end{equation}
% 
% In an alternative derivation
%
The line $\mathcal{L}$ from $P$ to $Q$ can be represented by the point $P$
and some distance $s$ along the normal vector $\vec{g}$ in the direction
from $P$ to $Q$, \emph{viz.}
%
\renewcommand{\arraystretch}{1.4}
\begin{equation}\label{L}
\mathcal{L} = -\vec{R} =
\vec{P} + s \vec{g} = \vec{Q} \,\,\text{ or}
\left[ \begin{array}{c} \rho \\ z \end{array} \right] +
s \left[ \begin{array}{c} \frac{2 \rho}{a^2} \\ \frac{2 z}{c^2}
 \end{array} \right] =
\left[ \begin{array}{c} 0 \\ -\zeta \end{array} \right] \,,
\end{equation}

from which, solving for $s$ and $\zeta$ gives
%
\begin{equation}\begin{split}\label{zeta}
s = \,& -\frac{a^2}2 \\
| \mathcal{L} | = | \vec{R} | = \,& s | \vec{g} | =
\frac{a^2}{\sqrt{a^2 \cos^2 \phi + c^2 \sin^2 \phi}} =
\frac{a}{\sqrt{1-e^2 \sin^2 \phi}} =
\frac{a^2}{\sqrt{D}} = \frac1{\kappa_2} = R_2 \text{ and} \\
\zeta = \,& z \left( 1 - \frac{a^2}{c^2} \right)
      = -z\,\frac{e^2}{1-e^2}
      = -R_2\, e^2 \sin \phi
\end{split}\end{equation}

% \begin{equation}
% \vec{R} =
% \left[ \begin{array}{c} a \cos \theta \\ \frac{a^2}{c} \sin \theta
% \end{array} \right] \text{ and }
% | \vec{R} | = \frac{a}{c} \sqrt{a^2 \sin^2 \theta + c^2 \cos^2 \theta}
%  = \frac{a \, \sqrt{G}}c = \frac1{\kappa_2} = R_2 \,.
% \end{equation}

% which is the same result as Equation (\ref{five}).

% Similarly, solving for $s$ and $\xi$ in
% 
% \renewcommand{\arraystretch}{1.4}
% \begin{equation}
% \vec{P} + s \vec{g} = \vec{T} \,\,\text{ or}
% \left[ \begin{array}{c} x \\ y \end{array} \right] +
% s \left[ \begin{array}{c} \frac{2x}{a^2} \\ \frac{2y}{b^2}
%  \end{array} \right] =
% \left[ \begin{array}{c} \xi \\ 0 \end{array} \right] \,,
% \end{equation}
% 
% gives
% %
% \begin{equation}\begin{split}
% s = \,& -\frac{c^2}2 \\
% \xi = \,& x \left( 1 - \frac{c^2}{a^2} \right) \text{ and }\\
% | \vec{PT} | = \,& s | \vec{g} | = \frac{c \sqrt{G}}{a}
% = \frac{c^2}{a^2} | \vec{R} | = \frac{c^2}{a^2} R_2\,, \\
% \end{split}\end{equation}
% 
% which is the normal distance from $\mathbf{P}$ to the equatorial plane
% (which might not be interesting).

%=========================================================================
\section{Parametric represention using geodetic latitude}\label{Rel}

Solving for $\rho$ in terms of $z$, and vice-versa, in Equation (\ref{four})
and substituting into Equation (\ref{one}) gives parametric equations in
terms of the geodetic angle $\phi$:
%
\begin{equation}\begin{split}\label{nine}
\rho = \,& z\, \frac{a^2}{c^2}\, \frac{\cos \phi}{\sin \phi}
\Rightarrow \frac{z^2 a^4 \cos^2 \phi}{a^2 c^4 \sin^2 \phi} +
 \frac{z^2}{c^2} = 1
\Rightarrow z^2 \left( a^2 \cos^2 \phi + c^2 \sin^2 \phi \right)
 = c^4 \sin^2 \phi\\
z = \,& \rho\, \frac{c^2}{a^2}\, \frac{\sin \phi}{\cos \phi}
\Rightarrow \frac{\rho^2}{a^2} +
 \frac{\rho^2 c^4 \sin^2 \phi}{a^4 c^2 \cos^2 \phi} = 1
\Rightarrow \rho^2 \left( a^2 \cos^2 \phi + c^2 \sin^2 \phi \right)
 = a^4 \cos^2 \phi \\
 \text{from } \,& \text{which} \\
%
\rho = \,& \frac{a^2 \cos \phi}{\sqrt{a^2 \cos^2 \phi + c^2 \sin^2 \phi}}
     = \frac{a^2 \cos \phi}{\sqrt{D}} = N(\phi) \cos \phi
     = \frac{a \cos \phi}{\sqrt{1-e^2\sin^2\phi}} \\
z = \,& \frac{c^2 \sin \phi}{\sqrt{a^2 \cos^2 \phi + c^2 \sin^2 \phi}}
     = \frac{c^2 \sin \phi}{\sqrt{D}} = \frac{c^2}{a^2} N(\phi) \sin \phi
     = \frac{a ( 1-e^2 ) \sin \phi}{\sqrt{1-e^2\sin^2\phi}} \,. \\
 \text{or, } \,& \text{for an ellipsoid} \\
x = \,& \frac{a^2 \cos \phi \cos \lambda}
             {\sqrt{a^2 \cos^2 \phi + c^2 \sin^2 \phi}}
     = \frac{a^2 \cos \phi \cos \lambda}{\sqrt{D}}
     = N(\phi) \cos \phi \cos \lambda = \rho \cos \lambda
     = \frac{a \cos \phi \cos \lambda}{\sqrt{1-e^2\sin^2\phi}} \\
y = \,& \frac{a^2 \cos \phi \sin \lambda}
             {\sqrt{a^2 \cos^2 \phi + c^2 \sin^2 \phi}}
     = \frac{a^2 \cos \phi \sin \lambda}{\sqrt{D}}
     = N(\phi) \cos \phi \sin \lambda = \rho \sin \lambda
     = \frac{a \cos \phi \sin \lambda}{\sqrt{1-e^2\sin^2\phi}} \\
z = \,& \frac{c^2 \sin \phi}{\sqrt{a^2 \cos^2 \phi + c^2 \sin^2 \phi}}
     = \frac{c^2 \sin \phi}{\sqrt{D}} = \frac{c^2}{a^2} N(\phi) \sin \phi
     = (1-e^2) N(\phi) \sin \phi
     = \frac{a ( 1-e^2 ) \sin \phi}{\sqrt{1-e^2\sin^2\phi}} \,. \\
\end{split}\end{equation}

%=========================================================================
\section{Ellipse of rotated equatorial plane}

If a plane initially in the equatorial plane is rotated through an angle
$\beta$ about a line $y = x \tan \lambda$, its semi-major axis is $a$, and
its semi-minor axis is given by Equation (\ref{R}):
%
\begin{equation}
b = R(\beta) = \frac{a\,c}{\sqrt{a^2 \sin^2 \beta + c^2 \cos^2 \beta}}
  = a\, \sqrt{\frac{1-e^2}{1-e^2 \cos^2 \beta}} \,.
\end{equation}

If a local two-dimensional co\"ordinate system is established with axes
%
\begin{equation}\begin{split}
\xi = \,& [ \cos \lambda, \sin \lambda, 0 ]^T \\
\eta = \,& [ \cos \gamma \cos\lambda, \cos \gamma \sin \lambda,
              \sin \gamma ]^T \\
\end{split}\end{equation}

the equation for the ellipse in that plane is
%
\begin{equation}\label{E2}
\frac{\xi^2}{a^2} + \frac{\eta^2}{b^2} = 1 \,.
\end{equation}

The parametric equations using the central angle are

\begin{equation}\begin{split}
\xi = \,& R(\gamma) \cos \gamma \\
\eta = \,& R(\gamma) \sin \gamma \,. \\
\end{split}\end{equation}

Substituting in Equation (\ref{E2}) gives

\begin{equation}
R(\gamma)^2 \left( \frac{\cos^2 \gamma}{a^2} +
                       \frac{\sin^2 \gamma}{b^2} \right) = 1
 = R(\gamma)^2 \, \frac{1-e^{\prime 2} \cos^2 \beta}
                       {a^2 ( 1-e^{\prime 2} )} \,,
\end{equation}

where $e^{\prime 2} = 1 - \frac{b^2}{a^2}$ is the eccentricity of the
rotated ellipse, from which

\begin{equation}
R(\gamma) = \frac{a\,b}
                     {\sqrt{a^2 \sin^2 \gamma + b^2 \cos^2 \gamma}}
 = a \sqrt{ \frac{1-e^{\prime 2}}{1-e^{\prime 2} \cos^2 \beta}} \,.
\end{equation}

The curvature is

\begin{equation}\begin{split}
\kappa(\phi) = \,& \frac{(a^2 \cos^2 \phi + b^2 \sin^2 \phi)^\frac32}
                        {(a\,b)^2}
             = \frac{(1 - e^{\prime 2} \sin^2 \phi)^\frac32}{a(1-e^{\prime 2})} \\
\kappa(\theta) = \,&
  \frac{a\, b}
       {(a^2 \sin^2 \theta + b^2 \cos^2 \theta)^\frac32}
               = \frac{\sqrt{1-e^{\prime 2}}}{a(1-e^{\prime 2} \cos^2 \theta)^\frac32} \\
\kappa(\gamma) = \,& a\,b\,
                     \left( \frac{ a^2 \sin^2 \gamma + b^2 \cos^2 \gamma }
                                 { a^4 \sin^2 \gamma + b^4 \cos^4 \gamma }
                     \right)^\frac32
               = \frac{\sqrt{1-e^{\prime 2} }}a
                     \left( \frac{ 1 - e^{\prime 2} \cos^2 \gamma }
                                 { 1 + ( e^4 - 2 e^{\prime 2} ) \cos^2 \gamma }
                     \right)^\frac32 \,, \\
\end{split}\end{equation}

which is the same as the meridional curvature of an oblate spheroid with
eccentricity $e'$.

%=========================================================================
%\newpage
\section{Ellipse in a normal section}\label{Normal-Section}

From Equations (40) and (43) in Deaton, the equation of the ellipse in a
normal section at azimuth $\alpha$, in a local co\"ordinate system
with its origin at $(\lambda,\phi)$, is
%
\begin{equation}\label{Normal}
\xi^{\prime2} + \eta^{\prime2} + ( g\,\xi' + h \, \eta')^2
 = (g^2+1) \xi^{\prime2} + 2 \xi' \eta' g h + (h^2+1)\eta^{\prime2}
 = -2 N \xi' \text{ where}
\end{equation}
\vspace*{-0.2in}
\begin{equation}
g^2 = \frac{e^2}{1-e^2} \sin^2 \phi \text{ and } h = g \cos \alpha\,.
\end{equation}

Substituting $\xi' = x \cos \omega + y \sin \omega$ and
$\eta' = -x \sin \omega + y \cos \omega$, the coefficient of the $xy$
term, which we wish to remove, is $(h^2-g^2) \sin \omega \cos \omega +
(\cos^2 \omega - \sin^2 \omega)\, g h$.

Substituting $\omega = \tan^{-1} \frac{h}g = \tan^{-1} \cos\, \alpha$,
Equation (\ref{Normal}) becomes
%
\begin{equation}\label{Normal-rot}
(g^2+h^2+1) x^2 + y^2 = 2 N \frac{h y - g x}{\sqrt{g^2+h^2}}\,.
\end{equation}

Substituting $x = \xi - \xi_c$ and $y = \eta - \eta_c$, and solving for
$\xi_c$ and $\eta_c$ to make the coefficients of the $\xi$ and $\eta$
terms zero, Equation (\ref{Normal-rot}) becomes
%
\begin{equation}\label{Normal-rot-cent}
(g^2+h^2+1) \xi^2 + \eta^2 = N^2\,\frac{h^2+1}{g^2+h^2+1}\,,
\end{equation}

from which the semi-axes are
%
\begin{equation}\begin{split}
a^{\prime2} = \,& N^2\,\frac{h^2+1}{g^2+h^2+1}
= a^2 \frac{1-e^2+e^2\sin^2\phi\cos^2\alpha}
           {(1-e^2\sin^2\phi)(1-e^2\cos^2\phi+e^2\sin^2\phi\cos^2\alpha)}
           \, \text{ and}\\
b^{\prime2} = \,& N^2\,\frac{h^2+1}{(g^2+h^2+1)^2}
= a^2 \frac{1-e^2+e^2\sin^2\phi\cos^2\alpha}
           {(1-e^2\sin^2\phi)(1-e^2\cos^2\phi+e^2\sin^2\phi\cos^2\alpha)^2}
           \\
\end{split}\end{equation}

and the eccentricity is
%
\begin{equation}
\epsilon^2 = 1-\frac{b^{\prime2}}{a^{\prime2}} = \frac{g^2+h^2}{g^2+h^2+1}
= \frac{e^2\sin^2\phi ( 1 + \cos^2\alpha )}
       {1-e^2\cos^2\phi+e^2\sin^2\phi\cos^2\alpha} \,.
\end{equation}

The radius of curvature is
%
\begin{equation}
R_M = a^\prime\, \frac{1-\epsilon^2}{(1-\epsilon^2 \sin^2 \psi)^\frac32} \,,
\end{equation} 
where $\psi$ is the geodetic angle with respect to the semi-major axis of
the section.

For the prime vertical section, $\alpha=\frac{\pi}2$, and therefore $h=0$.
Equation (\ref{Normal-rot-cent}) becomes
%
\begin{equation}
(g^2+1)\, \xi^2 + \eta^{\prime2} = \frac{N^2}{g^2+1}
\end{equation}

from which the semi-axes are
%
\begin{equation}\begin{split}\label{fifty-two}
& a^{\prime2} = \frac{N^2}{g^2+1}
             = a^2 \frac{1-e^2}
                    {(1-e^2 \sin^2 \phi)(1-e^2 \cos^2 \phi)}
 \text{ and} \\
& b^{\prime2} = \frac{N^2}{(g^2+1)^2}
             = a^2 \frac{(1-e^2)^2}
                    {(1-e^2 \sin^2 \phi)(1-e^2 \cos^2 \phi)^2} \,,\\
\end{split}\end{equation}

and the eccentricity is
%
\begin{equation}
\epsilon^2 = \frac{g^2}{g^2+1}
           = \frac{e^2 \sin^2 \phi}{1-e^2 \cos^2 \phi} \,.
\end{equation}

For the prime vertical section, the center is at $[ x^\prime, y^\prime,
z^\prime]^T$, where $x^\prime = (1-\delta)x = \frac{b^\prime \,e^2}{1-e^2}
\sin^2\phi \cos \phi \cos \lambda$, $y^\prime = (1-\delta)y$ =
$\frac{b^\prime \,e^2}{1-e^2} \sin^2\phi \cos \phi \sin \lambda$,
$z^\prime = z - \delta(z+\zeta) = -b^\prime \,e^2 \sin \phi \cos^2 \phi$,
$\delta = \frac{b^\prime}{N} = \frac1{1+g^2} = \sqrt{\frac{1-e^2}{1-e^2
\cos^2 \phi}}$, and $\zeta$ is given in Equation (\ref{zeta}).

Let $\vec{P}$, $\vec{g}$ and $\mathcal{L}$ be as in Equation (\ref{L}). 
Then using the results from \h{wvs-131}, the point where the line
$\mathcal{L}$ intersects the other side of the Earth is

{\large\renewcommand{\arraystretch}{1.4}
\begin{equation}
\vec{v} =
\left[ \begin{array}{l}
-\frac{e^4 \cos^2 \theta - 2 e^2 + 1}{1 + (e^4-2 e^2) \cos^2 \theta}\, x \\
-\frac{e^4 \cos^2 \theta - 2 e^2 + 1}{1 + (e^4-2 e^2) \cos^2 \theta}\, y \\
-\frac{1 - e^4 \cos^2 \theta}{1 + (e^4-2 e^2) \cos^2 \theta}\, z \\
\end{array} \right] \text{ and }
|\vec{v} - \vec{P}|
 = 2\,a\, \frac{\sqrt{1-e^2} ( 1 - e^2 \cos^2 \theta )^\frac32}
          {1+(e^4 - 2 e^2)\cos^2 \theta}
\end{equation}}
%
which, upon substituting $a \tan \theta = c \tan \phi$ becomes
%
\begin{equation}
\vec{v} = \left[ \begin{array}{l}
-\frac{e^2\cos^2\phi -2 e^2 + 1}{1-e^2 \cos^2 \phi} \,x \\
-\frac{e^2\cos^2\phi -2 e^2 + 1}{1-e^2 \cos^2 \phi} \,y \\
-\frac{1 + e^2 \cos^2 \phi}{1-e^2 \cos^2 \phi} \,z \\
\end{array} \right] \text{ and }
| \vec{v} - \vec{P}| =
2\, a \frac{1-e^2}{(1-e^2 \cos^2 \phi)\sqrt{1-e^2 \sin^2 \phi}}
 = 2\, b^\prime \,.
\end{equation}

This shows that the semi-minor axis of the prime vertical section is
normal to the surface.  Therefore, for the prime vertical section,
the radius of curvature at an angle $\psi$ from the semi-minor axis is
%
\begin{equation}
R^\oplus_{\text{eq}} = a^\prime\,
  \frac{1-\epsilon^2}{(1-\epsilon^2 \cos^2 \psi)^\frac32}\,.
\end{equation}

For the prime vertical section, $\psi=0$ and $R^\oplus_{\text{eq}} = N$. 
At azimuth $\alpha$, $\psi=\omega=\tan^{-1} \cos\,\alpha$, and
%
\begin{equation}\label{Req}
R^\oplus_{\text{eq}} = a^\prime\,
   \frac{1-\epsilon^2}{(1-\epsilon^2 \cos^2 \psi)^\frac32}
 =  a^\prime\, (1-\epsilon^2) \left(
   \frac{1+\cos^2 \alpha}{1+\cos^2 \alpha-\epsilon^2} \right)^\frac32 \,.
\end{equation}

%=========================================================================
\section{Azimuth of a normal section}

In his Equation (11), Deaton shows that the ``East-North-Up'' or ``ENU''
co\"ordinate system with its origin at $(x_1,y_1,z_1)$ is related to
geocentric Cartesian ECR co\"ordinates $(x,y,z)$ by
%
\begin{equation}
\left[ \begin{array}{c} U \\ E \\ N \\ \end{array} \right] =
\mathbf{R}_{\phi\lambda}
\left[ \begin{array}{c} x-x_1 \\ y-y_1 \\ z-z_1 \end{array} \right]
\end{equation}

where $\mathbf{R}_{\phi\lambda}$ is the product of two rotations
%
\begin{equation}\label{Rotations}
\mathbf{R}_{\phi\lambda} = \mathbf{R}_\phi \mathbf{R}_\lambda =
\left[ \begin{array}{ccc} \cos \phi_1 & 0 & \sin \phi_1 \\
                               0      & 1 &      0      \\
                         -\sin \phi_1 & 0 & \cos \phi_1 \\
\end{array} \right] \,
\left[ \begin{array}{ccc} \cos \lambda_1 & \sin \lambda_1 & 0 \\
                         -\sin \lambda_1 & \cos \lambda_1 & 0 \\
                               0         &      0         & 1 \\
\end{array} \right] \,.
\end{equation}

The first rotation, $\mathbf{R}_\lambda$, a positive right-handed rotation
about the $z$ axis that takes the $(x,y,z)$ axes to $(x',y',z')$. The $z'$
axis is coincident with the $z$ axis and the $x'-y'$ plane is the
equatorial plane.

The second rotation, $\mathbf{R}_\phi$, is a rotation about the $y'$ axis
by $\phi_1$ that takes the $(x',y',z')$ axes to the $(x'',y'',z'')$ axes. 
The $x''$ axis is parallel to the $U$ axis, the $y''$ axis is parallel to
the $E$ axis, and the $z''$ axis is parallel to the $N$ axis.

Performing the multiplication in Equation (\ref{Rotations}) gives
%
\begin{equation}
\mathbf{R}_{\phi\lambda}
\left[ \begin{array}{ccc}
  \cos \phi_1 \cos \lambda_1 &  \cos \phi_1 \sin \lambda_1 & \sin \phi_1 \\
 -\sin \lambda_1             & \cos \lambda_1              & 0 \\
 -\sin \phi_1 \cos \lambda_1 & -\sin \phi_1 \sin \lambda_1 & \cos \phi_1 \\
\end{array} \right] \,.
\end{equation}

Re-ordering the rows of $\mathbf{R}_{\phi\lambda}$ gives the
transformation in the more usual $(E,N,U)$ form
%
\begin{equation}
\left[ \begin{array}{c} E \\ N \\ U \\ \end{array} \right] =
\mathbf{R}
\left[ \begin{array}{c} x-x_1 \\ y-y_1 \\ z-z_1 \end{array} \right]
\text{ where }
\mathbf{R}
\left[ \begin{array}{ccc}
 -\sin \lambda_1             & \cos \lambda_1              & 0 \\
 -\sin \phi_1 \cos \lambda_1 & -\sin \phi_1 \sin \lambda_1 & \cos \phi_1 \\
  \cos \phi_1 \cos \lambda_1 &  \cos \phi_1 \sin \lambda_1 & \sin \phi_1 \\
\end{array} \right] \,.
\end{equation}

The co\"oordinate differences $(\Delta E = E_2-E_1, \Delta N = N_2-N_1,
\Delta U = U_2 - U_1$) in the local geodetic horizon plane are given by
%
\begin{equation}\label{Delta-RNU}
\left[ \begin{array}{c}
 \Delta E \\ \Delta N \\ \Delta U \\ \end{array} \right] =
\mathbf{R}
\left[ \begin{array}{c}
 \Delta x \\ \Delta y \\ \Delta z \end{array} \right] \,.
\end{equation}

From Equation (18) in Deaton (the ratio of the first two rows of Equation
(\ref{Delta-RNU})), the azimuth of a normal section plane from
$(\lambda_1,\phi_1)$ to $(\lambda_2,\phi_2)$ is
%
\begin{equation}
\tan \alpha_{12} = \frac{\Delta E}{\Delta N}
 = \frac{\Delta x \sin \lambda_1 - \Delta y \cos \lambda_1}
        {  \sin \phi_1 ( \Delta x \cos \lambda_1 +
                         \Delta y \sin \lambda_1 )
         - \Delta z \cos \phi_1} \,.
\end{equation}

Replacing geocentric Cartesian ECR co\"ordinates with geodetic co\"ordinates,
%
\begin{equation}
\tan \alpha_{12}
 = \frac{N_2 \cos \phi_2 \sin(\lambda_1-\lambda_2)}
        {N_2 ( \cos \phi_2 \sin \phi_1 \cos(\lambda_1-\lambda_2) -
           (1-e^2) \cos \phi_1 \sin \phi_2 )
         - N_1 e^2 \sin \phi_1 \cos \phi_1} \,,
\end{equation}

where
%
\begin{equation}
N_1 = \frac{a}{\sqrt{1-e^2 \sin^2 \phi_1}} \text{ and }
N_2 = \frac{a}{\sqrt{1-e^2 \sin^2 \phi_2}} \,.
\end{equation}

%=========================================================================
\newpage
\section*{Appendix: Bowring's proof of prime vertical radius of curvature}

From \emph{Notes on the curvature in the prime vertical section}, {\bf
Survey Review 29}, 226 (October 1987) pp 195-196.

Let $P$ be a point on the reference ellipsoid and let $A$ and $B$ be two
points on the prime vertical section through $P$ such that $A$ and $B$ are
symmetrically placed about $P$ (see the figure below).  Let the normals to
the ellipsoid at $A$ and $B$ intersect at $G$ on the minor axis.  (This is
so because by symmetry about the meridian through $A$ the normal at $A$
intersects the minor axis, and similarly for the normal at $B$. 
Furthermore the points at $A$ and $B$ are at the same latitude and
therefore the normals intersect the minor axis at the same point).

Let the plane $ABG$ intersect the surface in the line $AP'B$.  Then AG is
perpendicular to the line $AP'B$ at $A$ (because $AG$ is perpendicular to
all surface lines through $A$), and similarly $BG$ is perpendicular to the
line at $B$.  Let now $A$ and $B$ approach $P$ (on the prime vertical
section) \emph{simultaneously} and \emph{symmetrically}.  Then $AP'B
\rightarrow APB$, and at all times the normals at the ends intersect on
the minor axis.  This is the geometrical evidence that the centre of
curvature lies on the minor axis, and therefore the centre of curvature
$C$ of the prime vertical section through $P$ lies on the intersection of
the normal at $P$ (to the ellipsoid) with the minor axis, and the radius
of curvature is equal to the length $CP$.

\begin{centering}
\includegraphics[scale=1.0]{wvs-146-2}\\
\end{centering}


\label{lastpage}
\vspace*{-0.1in} % Somehow, this causes lastpage to be defined
\end{document}

% $Id$

% $Log$
% Revision 1.4  2020/01/06 21:20:17  vsnyder
% Substantial revision and additions
%
% Revision 1.3  2019/11/13 01:33:03  vsnyder
% Added a lot more material
%
% Revision 1.2  2017/10/13 19:10:51  vsnyder
% Add CVS stuff
%
