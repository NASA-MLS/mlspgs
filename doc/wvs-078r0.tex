\documentclass[11pt]{article}
\usepackage[fleqn]{amsmath}\textwidth 6.5in
\oddsidemargin -0.25in
%\evensidemargin -0.5in
\topmargin -0.25in
\textheight 9.0in

\newcommand{\docname}{\bf wvs-078}
\newcommand{\docdate}{16 April 2008}

\begin{document}

%\tracingcommands=1
\newlength{\hW} % heading box width
\newlength{\pW} % page number field width
\settowidth{\hW}{\docname}
\settowidth{\pW}{Page \pageref{lastpage}\ of \pageref{lastpage}}
\ifdim \pW > \hW \setlength{\hW}{\pW} \fi
\makeatletter
\def\@biblabel#1{#1.}
\newcommand{\ps@twolines}{%
  \renewcommand{\@oddhead}{%
    \docdate\hfill\parbox[u]{\hW}{{\hfill\docname}\newline
                          Page \thepage\ of \pageref{lastpage}}}%
\renewcommand{\@evenhead}{}%
\renewcommand{\@oddfoot}{}%
\renewcommand{\@evenfoot}{}%
}%
\makeatother
\pagestyle{twolines}

\vspace{-10pt}
\begin{tabbing}
\phantom{References: }\= \\
To: \>Dave\\
Subject: \>Tasks for cloud forward model\\
From: \>Van Snyder\\
\end{tabbing}

\parindent 0pt \parskip 10pt
\vspace{-20pt}

The tasks needed to incorporate clouds into the full forward model
according to the strategy Bill has laid out are

\begin{enumerate}

\item\label{one} Compute tables of $\beta_{c\_s}$, $\beta_{c\_e}$ and
$P(\theta)$ for several values of temperature, ice water content and
$\theta$.  Software to produce these tables is completed and tables have
been computed.

\item\label{two} Modify the full forward model to compute radiative
transfer to a specific point in the atmosphere, along a specific path. 
This is also useful for observer-in-atmosphere studies.

\item Using the tables computed in step \ref{one} and a sample
atmosphere, compute radiative transfer to specified scattering points at
specified angles using the method made available by step \ref{two}.

At each of these points, convolve the radiances and derivatives from the
several directions with $P(\theta)$ and its derivatives with respect to
temperature and ice water content, giving $T_{\text{scat}}$ and its
derivatives, both with respect to state vector quantities throughout the
atmosphere, and with respect to temperature and ice water content at the
scattering points.

Store these results in L2PC-like files for later use in steps \ref{four}
and \ref{five}.

\item\label{four} Create a linear evaluator for $T_{\text{scat}}$, along
the lines of the quasi-linear forward model.

\item\label{five} Use the linear evaluator developed in step \ref{four}
during full forward model calculations to evaluate $T_{\text{scat}}$ and
its derivatives at points along the gas-phase radiative transfer line of
sight.

\end{enumerate}

\label{lastpage}
\end{document}
% $Id$
