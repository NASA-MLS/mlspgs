\documentclass[11pt]{article}

\usepackage{alltt}
\usepackage[fleqn]{amsmath}
\usepackage{floatflt}
\usepackage{graphicx}
\usepackage{longtable}
\usepackage[strings]{underscore}

\newcommand{\docname}{wvs-121}
\newcommand{\docdate}{30 Oct 2014}

\textwidth 6.5in
\oddsidemargin -0.25in
%\evensidemargin -0.5in
\topmargin -0.5in
\textheight 9in

\parindent 0pt
\parskip 6pt

\ifx\pdfoutput\undefined
  \pdfoutput=0
  \usepackage[hypertex,plainpages,hyperindex=true]{hyperref}
  \hypersetup{%
    hypertexnames=false%
  }
  % Specify the driver for the color package
  \ExecuteOptions{dvips}
  %\ExecuteOptions{xdvi}
\else
  \ifnum\pdfoutput>0
    \usepackage[pdftex,plainpages,hyperindex=true,pdfpagelabels]{hyperref}
    \hypersetup{%
      hypertexnames=false,%
      colorlinks=true,%
      linktocpage=true,%
    }
    % Specify the driver for the color package
    \ExecuteOptions{pdftex}
  \else
    \usepackage[hypertex,plainpages,hyperindex=true]{hyperref}
    \hypersetup{%
      hypertexnames=false%
    }
    % Specify the driver for the color package
    \ExecuteOptions{dvips}
    %\ExecuteOptions{xdvi}
  \fi
\fi

\hyperbaseurl{}
\newcommand\hr[1]{\href{#1.dvi}{dvi}, \href{#1.pdf}{pdf}}
\newcommand\h[1]{#1 (\hr{#1})}

\begin{document}

%\tracingcommands=1
\newlength{\hW} % heading box width
\newlength{\pW} % page number field width
\settowidth{\hW}{\bf\docname}
\settowidth{\pW}{Page \pageref{lastpage}\ of \pageref{lastpage}}
\ifdim \pW > \hW \setlength{\hW}{\pW} \fi
\makeatletter
\def\@biblabel#1{#1.}
\newcommand{\ps@twolines}{%
  \renewcommand{\@oddhead}{%
    \docdate\hfill\parbox[t]{\hW}{{\hfill\bf\docname}\newline
                          Page \thepage\ of \pageref{lastpage}}}%
\renewcommand{\@evenhead}{}%
\renewcommand{\@oddfoot}{}%
\renewcommand{\@evenfoot}{}%
}%
\makeatother
\pagestyle{twolines}

\renewcommand{\d}{\text{d}}
\newcommand{\T}{\mathcal{T}}

\vspace{-10pt}
\begin{tabbing}
\phantom{References: }\= \\
To: \>Nathaniel, Paul, Bill, Mike\\
Subject: \>Three-dimensional rotation about a specified axis\\
From: \>Van Snyder\\
Reference: \>\h{wvs-120} \\
\end{tabbing}

\vspace*{-30pt}

\section*{Application to computing magnetic field}

In the case that the magnetic field is needed in a direction that is not
in the orbit plane, the angular positions of the points in the magnetic
field are specified on a great circle that includes PTan and the
spacecraft position.  To calculate those positions, rotate the unit vector
in the PTan direction about the normal to the plane containing PTan and
the spacecraft position by the the angles $\phi$ specified in the HGrid
for the magnetic field.

There are (at least) two ways to calculate these positions.

First, let $\hat{S}$ and $\hat{P}$ be unit vectors in the directions of
the spacecraft and PTan, respectively.  Then the unit vector
$\hat{\mathbf{n}}$ that is normal to the plane containing them, in the
right-hand sense, is $\hat{\mathbf{n}} = \hat{S} \times \hat{P}$.  Given
this vector, a positive rotation of PTan about $\hat{\mathbf{n}}$ rotates
PTan away from or toward the spacecraft, by an amount given by a value of
$\phi$ in the XGrid for the magnetic field, depending upon whether $\phi$
is positive or negative, respectively.

Second, let $\hat{\mathbf{n}}$ be an unit vector along the line of sight
from the spacraft to PTan.
\section*{Mathematical formalism for rotation}

Define the completely unsymmetric Levi-Civita symbol in three dimensions
as

\begin{equation*}
\epsilon_{ijk} = \left\{ \begin{array}{ll}
+1 & \text{ if } (i,j,k) \text{ is } (1,2,3),\, (2,3,1),\text{ or } (3,1,2) \\
-1 & \text{ if } (i,j,k) \text{ is } (3,2,1),\, (1,3,2),\text{ or } (2,1,3) \\
0  & \text{ if } i = j \text{ or } j = k \text{ or } i = k \\
\end{array} \right. \,.
\end{equation*}

To rotate a three-dimensional vector $V$ by an angle $\phi$ in a
right-handed sense about a unit vector $\hat{\mathbf{n}}$, multiply it on
the left by the matrix

\begin{equation*}
R_{ij} = \cos\phi\, \delta_{ij} + (1-\cos\phi)\, n_i n_j -
             \sum_k \sin\phi\, \epsilon_{ijk} n_k \,\,\,\text{or}
\end{equation*}

where $n_i$ are the coordinates of $\hat{\mathbf{n}}$, $\delta_{ij}$ is
the Kronecker tensor, $\epsilon_{ijk}$ is the Levi-Civita symbol.  There
is at most one nonzero in the summation, when $i$, $j$, and $k$ are
distinct.  This is called the \emph{Rodrigues formula}.  In expanded form,
$R$ is

\begin{equation*}
R = \left( \begin{array}{ccc}
\cos\phi + n_1^2(1-\cos\phi) &
  n_1 n_2 ( 1 - \cos\phi ) - n_3 \sin\phi &
    n_1 n_3 ( 1 - \cos\phi ) + n_2 \sin\phi \\
n_1 n_2 ( 1 - \cos\phi ) + n_3 \sin\phi &
  \cos\phi + n_2^2(1-\cos\phi) &
    n_2 n_3 ( 1 - \cos\phi ) - n_1 \sin\phi \\
n_1 n_3 ( 1 - \cos\phi ) - n_2 \sin\phi &
  n_2 n_3 ( 1 - \cos\phi ) + n_1 \sin\phi &
    \cos\phi + n_3^2(1-\cos\phi) \\
\end{array} \right)
\end{equation*}

This is derived in {\tt
http://scipp.ucsc.edu/$\sim$haber/ph216/rotation_12.pdf}.

\section*{Alternative method}

Compute geocentric latitude of the tangent point from its geodetic
latitude:

\begin{equation*}\begin{split}
\lambda_{tc} = \,& \sin^{-1} \left(
 \frac{\sin \lambda_{tg}}
      { \sqrt{\frac{a^4}{b^4} \cos^2 \lambda_{tg} + \sin^2 \lambda_{tg}} }
      \right) \\
\,&\\
 = \,& \tan^{-1} \left(
  \frac{\sin \lambda_{tg}}
       {f^2 \cos \lambda_{tg}} \right) \text{ where } f = \frac{a}b
      \\
\,&\\
 = \,& \sin^{-1} \left(
  \frac{\tan \lambda_{tg}}
       {\sqrt{f^2 + \tan^2 \lambda_{tg}}}
      \right) \\
\end{split}\end{equation*}

The second method is used in the function {\tt GeodToGeocLat} in the {\tt
Geometry} module.

Let $\hat{\mathbf{n}}$ be the unit vector in ECR coordinates along the
line of sight between the instrument and the tangent point, for the same
MIF.

Let $\phi$ be the increment along the line of sight at which the
geocentric latitude and longitude are desired.  Compute the longitude of
that point:

\begin{equation*}\begin{split}
\mu = \,& \tan^{-1} \left(
  \frac{\sin \mu_t \cos \lambda_{tc} + n_y \tan \phi}
       {\cos \mu_t \cos \lambda_{tc} + n_x \tan \phi}
  \right ) \\
\,& \\
 = \,& \tan^{-1} \left(
  \frac{\sin \mu_t \cos \lambda_{tc} \cos \phi + n_y \sin \phi}
       {\cos \mu_t \cos \lambda_{tc} \cos \phi + n_x \sin \phi}
  \right ) \,, \\
\end{split}\end{equation*}

the latter being better behaved for $\phi \approx \frac\pi2$.
Compute the geocentric latitude of that point:

\begin{equation*}\begin{split}
\lambda_c = \,& \sin^{-1} \left(
 \sin \lambda_{tc} + \frac{n_z \tan \phi}
                          {\sqrt{1 + \tan^2 \phi}} \right) \\
\,& \\
 = \,& \sin^{-1} \left( \sin \lambda_{tc} +
        n_z \sin \phi \text{ signum} \cos \phi \right)
 \\
\,& \\
 = \,& \sin^{-1} \left( \sin \lambda_{tc} + n_z \sin \phi \right)
 \text{ for } | \phi | \leq \frac\pi2
 \\
\end{split}\end{equation*}

\label{lastpage}
\end{document}

% $Id$

% $Log$
