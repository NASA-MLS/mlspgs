\documentclass[11pt]{article}

\usepackage{alltt}
\usepackage[fleqn]{amsmath}
\usepackage{floatflt}
\usepackage{graphicx}
\usepackage{longtable}
\usepackage[strings]{underscore}

\newcommand{\docname}{wvs-121r1}
\newcommand{\docdate}{14 Nov 2014}

\textwidth 6.5in
\oddsidemargin -0.25in
%\evensidemargin -0.5in
\topmargin -0.5in
\textheight 9in

\parindent 0pt
\parskip 6pt

\ifx\pdfoutput\undefined
  \pdfoutput=0
  \usepackage[hypertex,plainpages,hyperindex=true]{hyperref}
  \hypersetup{%
    hypertexnames=false%
  }
  % Specify the driver for the color package
  \ExecuteOptions{dvips}
  %\ExecuteOptions{xdvi}
\else
  \ifnum\pdfoutput>0
    \usepackage[pdftex,plainpages,hyperindex=true,pdfpagelabels]{hyperref}
    \hypersetup{%
      hypertexnames=false,%
      colorlinks=true,%
      linktocpage=true,%
    }
    % Specify the driver for the color package
    \ExecuteOptions{pdftex}
  \else
    \usepackage[hypertex,plainpages,hyperindex=true]{hyperref}
    \hypersetup{%
      hypertexnames=false%
    }
    % Specify the driver for the color package
    \ExecuteOptions{dvips}
    %\ExecuteOptions{xdvi}
  \fi
\fi

\hyperbaseurl{}
\newcommand\hr[1]{\href{#1.dvi}{dvi}, \href{#1.pdf}{pdf}}
\newcommand\h[1]{#1 (\hr{#1})}

\begin{document}

%\tracingcommands=1
\newlength{\hW} % heading box width
\newlength{\pW} % page number field width
\settowidth{\hW}{\bf\docname}
\settowidth{\pW}{Page \pageref{lastpage}\ of \pageref{lastpage}}
\ifdim \pW > \hW \setlength{\hW}{\pW} \fi
\makeatletter
\def\@biblabel#1{#1.}
\newcommand{\ps@twolines}{%
  \renewcommand{\@oddhead}{%
    \docdate\hfill\parbox[t]{\hW}{{\hfill\bf\docname}\newline
                          Page \thepage\ of \pageref{lastpage}}}%
\renewcommand{\@evenhead}{}%
\renewcommand{\@oddfoot}{}%
\renewcommand{\@evenfoot}{}%
}%
\makeatother
\pagestyle{twolines}

\renewcommand{\d}{\text{d}}
\newcommand{\T}{\mathcal{T}}

\vspace{-10pt}
\begin{tabbing}
\phantom{References: }\= \\
To: \>Nathaniel, Bill, Mike\\
Subject: \>Three-dimensional rotation about a specified axis\\
From: \>Van Snyder\\
Reference: \>\h{wvs-120} \\
\end{tabbing}

\vspace*{-30pt}

\section*{Application to computing magnetic field}

The angular positions of the points in the magnetic field are specified on
a great circle that includes the tangent point and spacecraft positions. 
To calculate those positions, rotate the tangent point position vector
about the normal to the plane containing the tangent point and spacecraft
positions by the the angles $\phi$ specified in the HGrid for the magnetic
field.

There are (at least) two ways to calculate these positions.

First, let $S$ and $P$ be position vectors for the spacecraft and tangent
point, respectively.  Then the vector $\eta = S \times P$ is normal to the
plane containing $S$ and $P$, in the right-hand sense, and $\hat{\eta}$ is
an unit vector parallel to $\eta$.  A rotation of $P$ about $\hat{\eta}$
by an angle $\phi$ rotates $P$ away from or toward $S$, depending upon
whether $\phi$ is positive or negative, respectively.

\section*{Mathematical formalism for rotation}

Define the completely antisymmetric Levi-Civita tensor in three dimensions
as
%
\begin{equation*}
\epsilon_{ijk} = \left\{ \begin{array}{ll}
+1 & \text{ if } (i,j,k) \text{ is } (1,2,3),\, (2,3,1),\text{ or } (3,1,2) \\
-1 & \text{ if } (i,j,k) \text{ is } (3,2,1),\, (1,3,2),\text{ or } (2,1,3) \\
0  & \text{ if } i = j \text{ or } j = k \text{ or } i = k \\
\end{array} \right. \,.
\end{equation*}
%
To rotate a three-dimensional vector $V$ by an angle $\phi$ about a unit
vector $\hat{\eta}$, multiply $V$ on the left by the matrix
%
\begin{equation*}
R_{ij} = \cos\phi\, \delta_{ij} + (1-\cos\phi)\, \hat{\eta}_i \hat{\eta}_j
          - \sum_{k=1}^3 \sin\phi\, \epsilon_{ijk} \hat{\eta}_k
\end{equation*}
%
where $\hat{\eta}_i$ are the coordinates of $\hat{\eta}$, $\delta_{ij}$
is the Kronecker delta, and $\epsilon_{ijk}$ is the Levi-Civita tensor. 
This is called the \emph{Rodrigues formula}.  There is at most one nonzero
in the summation, which occurs when $i$, $j$, and $k$ are distinct.  In
expanded form, $R$ is
%
\begin{equation*}
R = \left( \begin{array}{ccc}
\cos\phi + \hat{\eta}_1^2(1-\cos\phi) &
  \hat{\eta}_1 \hat{\eta}_2 ( 1 - \cos\phi ) - \hat{\eta}_3 \sin\phi &
    \hat{\eta}_1 \hat{\eta}_3 ( 1 - \cos\phi ) + \hat{\eta}_2 \sin\phi \\
\hat{\eta}_1 \hat{\eta}_2 ( 1 - \cos\phi ) + \hat{\eta}_3 \sin\phi &
  \cos\phi + \hat{\eta}_2^2(1-\cos\phi) &
    \hat{\eta}_2 \hat{\eta}_3 ( 1 - \cos\phi ) - \hat{\eta}_1 \sin\phi \\
\hat{\eta}_1 \hat{\eta}_3 ( 1 - \cos\phi ) - \hat{\eta}_2 \sin\phi &
  \hat{\eta}_2 \hat{\eta}_3 ( 1 - \cos\phi ) + \hat{\eta}_1 \sin\phi &
    \cos\phi + \hat{\eta}_3^2(1-\cos\phi) \\
\end{array} \right)
\end{equation*}
%
This is derived in, e.g., {\tt
http://scipp.ucsc.edu/$\sim$haber/ph216/rotation_12.pdf}.

(As a side note, the vector product $A \times B$ can be written using the
Levi-Civita tensor as $(A \times B)_k = \sum_{i,j} \epsilon_{ijk} A_i
B_j$.  This can be generalized to a product of $n-1$ $n$-dimensional
vectors using the order $n$ Levi-Civita tensor.)

\section*{Alternative method}

% Compute geocentric latitude of the tangent point from its geodetic
% latitude:
% 
% \begin{equation*}\begin{split}
% \lambda_{tc} = \,& \sin^{-1} \left(
%  \frac{\sin \lambda_{tg}}
%       { \sqrt{\frac{a^4}{b^4} \cos^2 \lambda_{tg} + \sin^2 \lambda_{tg}} }
%       \right) \\
% \,&\\
%  = \,& \tan^{-1} \left(
%   \frac{\sin \lambda_{tg}}
%        {f^2 \cos \lambda_{tg}} \right) \text{ where } f = \frac{a}b
%       \\
% \,&\\
%  = \,& \sin^{-1} \left(
%   \frac{\tan \lambda_{tg}}
%        {\sqrt{f^2 + \tan^2 \lambda_{tg}}}
%       \right) \\
% \end{split}\end{equation*}
% 
% The second method is used in the function {\tt GeodToGeocLat} in the {\tt
% Geometry} module.

Let $\hat{\mathbf{n}}$ be either the unit vector in ECR coordinates along
the line of sight between the instrument and the tangent point, the unit
tangent at the tangent point along the great circle containing the tangent
point and spacecraft positions, in the direction of the spacecraft, for
the same MIF, or the third row of the ECR-to-FOV matrix from the {\tt
L1BOA} file.

Let $\phi$ be the increment along the line of sight at which the
geocentric latitude and longitude are desired, $\lambda_{tc}$ be the
geocentric latitude of the tangent point, and $\mu_t$ be the longitude of
the tangent point.  Compute the geocentric latitude of that point:
%
\begin{equation*}\begin{split}
\lambda_c = \,& \sin^{-1} \left(
 \sin \lambda_{tc} + \frac{n_z \tan \phi}
                          {\sqrt{1 + \tan^2 \phi}} \right) \\
\,& \\
 = \,& \sin^{-1} \left( \sin \lambda_{tc} +
        n_z \sin \phi \text{ signum} \cos \phi \right)
 \\
\,& \\
 = \,& \sin^{-1} \left( \sin \lambda_{tc} + n_z \sin \phi \right)
 \text{ for } | \phi | \leq \frac\pi2
 \\
\end{split}\end{equation*}
%
The latter two being better behaved for $|\phi| \approx \frac\pi2$.  Compute
the longitude of that point:

\begin{equation*}\begin{split}
\mu = \,& \tan^{-1} \left(
  \frac{\sin \mu_t \cos \lambda_{tc} + n_y \tan \phi}
       {\cos \mu_t \cos \lambda_{tc} + n_x \tan \phi}
  \right ) \\
\,& \\
 = \,& \tan^{-1} \left(
  \frac{\sin \mu_t \cos \lambda_{tc} \cos \phi + n_y \sin \phi}
       {\cos \mu_t \cos \lambda_{tc} \cos \phi + n_x \sin \phi}
  \right ) \,, \\
\end{split}\end{equation*}

the latter being better behaved for $|\phi| \approx \frac\pi2$.

\section*{Test results}

I computed the result for rotation angles of -3.0, -1.5, 0.0, 1.5, and 3.0
degrees, using all methods, in double precision, for spacecraft and
tangent point positions taken from the first MIF of the first MAF in the
UARS {\tt L1BOA} file:

\begin{longtable}{lcccccc}
Module & GeodLat & GeocLat & Lon      & X       & Y       & Z \\
\hline
SC     & 13.3604 & 13.2741 & -170.867 & -0.9609 & -0.1545 & 0.2296 \\
GHz    & 26.2619 & 26.1095 & -152.039 & -0.7931 & -0.4210 & 0.4401 \\
\hline
\end{longtable}
\vspace*{-10pt}

As above, let $S$ be the spacecraft position vector and $P$ be the tangent
point position vector, and let $R$ be the result of the rotation
calculation.  In the test results table below, the {\tt Trig 1} method is
computed with $\hat{\mathbf{n}}$ the unit vector parallel to $(P - S)$.
The {\tt Trig 2} method is computed with $\hat{\mathbf{n}}$ the unit
vector parallel to $(S \times P) \times P$, i.e., tangent to the great
circle containing $S$ and $P$.  The {\tt Trig 3} method is computed with
$\hat{\mathbf{n}}$ the last row of the ECR-to-FOV matrix from the UARS
{\tt L1BOA} file.

The values of $\hat{\mathbf{n}}$ for the three methods are

\begin{longtable}{lccc}
Method & X & Y & Z \\
\hline
Trig 1 & 0.4429829 & -0.7036025 & 0.5556153 \\
Trig 2 & 0.6041459 & -0.6353586 & 0.4809648 \\
Trig 3 & 0.5929608 & -0.6871292 & 0.4198217 \\
\hline
\end{longtable}

\vspace*{-10pt}
The {\tt Plane} and {\tt Angle} columns are diagnostics defined by
%
\begin{equation*}\begin{split}
\text{\tt Plane} =\,&
 \cos^{-1} \left( \hat{\eta} \cdot
           \frac{R \times P}{|R \times P|} \right) \text{ and} \\
\,&\\[-7pt]
\text{\tt Angle} = \,&  \sin^{-1} | R \times P|
 \text{ signum} ( | P - S | - | R - S | ) - \phi
 \\
\end{split}\end{equation*}
%
\vspace*{-20pt}
\begin{itemize}

\item [{\tt Plane}] is the angle between $\eta$ and the normal to the $R -
      P$ plane.  This is zero if the rotation occurred in the $S - P$
      plane.

\item [{\tt Angle}] is the difference between the angle between $P$ and
      $R$, and $\phi$.  It should zero.

\end{itemize}
\vspace*{-10pt}

\newcommand{\C}[1]{\multicolumn{1}{c}{#1}}
\begin{longtable}{lrccrrrrr}
\multicolumn{9}{c}{\bfseries Test Results} \\
\hline
Method & \C{$\phi$} & GeocLat & Lon & \C{X} & \C{Y} & \C{Z} & \C{Plane} & \C{Angle} \\
\hline
Vector & -3.0 & 24.48 & -154.82 & -0.8237 & -0.3872 & 0.4143 &   0.00 &  0.0000 \\
Trig 1 &      & 24.27 & -154.80 & -0.8249 & -0.3882 & 0.4110 & 176.59 & -0.1032 \\
Trig 2 &      & 24.51 & -154.82 & -0.8234 & -0.3871 & 0.4149 & 179.39 &  0.0208 \\
Trig 3 &      & 24.72 & -154.96 & -0.8230 & -0.3845 & 0.4181 & 174.82 &  0.0152 \\
Vector & -1.5 & 25.30 & -153.44 & -0.8087 & -0.4042 & 0.4273 &   0.00 &  0.0000 \\
Trig 1 &      & 25.19 & -153.42 & -0.8093 & -0.4049 & 0.4255 & 176.12 & -0.0518 \\
Trig 2 &      & 25.31 & -153.44 & -0.8086 & -0.4042 & 0.4275 & 179.69 &  0.0052 \\
Trig 3 &      & 25.41 & -153.51 & -0.8084 & -0.4029 & 0.4291 & 175.14 &  0.0031 \\
All    &  0.0 & 26.11 & -152.04 & -0.7931 & -0.4210 & 0.4401 &   0.00 &  0.0000 \\
Vector &  1.5 & 26.91 & -150.62 & -0.7770 & -0.4375 & 0.4525 &   0.00 &  0.0000 \\
Trig 1 &      & 27.04 & -150.65 & -0.7764 & -0.4365 & 0.4546 &   4.82 &  0.0519 \\
Trig 2 &      & 26.92 & -150.62 & -0.7770 & -0.4375 & 0.4527 &   0.31 &  0.0051 \\
Trig 3 &      & 26.81 & -150.55 & -0.7772 & -0.4388 & 0.4511 &   4.21 &  0.0056 \\
Vector &  3.0 & 27.69 & -149.18 & -0.7604 & -0.4537 & 0.4647 &   0.00 &  0.0000 \\
Trig 1 &      & 27.98 & -149.26 & -0.7590 & -0.4514 & 0.4692 &   5.31 &  0.1034 \\
Trig 2 &      & 27.73 & -149.18 & -0.7602 & -0.4535 & 0.4653 &   0.63 &  0.0203 \\
Trig 3 &      & 27.52 & -149.05 & -0.7606 & -0.4561 & 0.4621 &   3.87 &  0.0196 \\
\hline
\end{longtable}
\vspace*{-10pt}

As a simple test, I put the spacecraft at (0.0 N, 0.0 E) and the tangent
point at (30.0 N, 0.0 E) geocentric, and $\phi = 3.0$.  The rotated
tangent point should be at (33.0 N, 0.0 E).

\begin{longtable}{lrccrrrrr}
\multicolumn{9}{c}{\bfseries Test Results} \\
\hline
Method & \C{$\phi$} & GeocLat & Lon & \C{X} & \C{Y} & \C{Z} & \C{Plane} & \C{Angle} \\
\hline
Vector &  3.0 & 33.00 &    0.00 & 0.8387 & 0.0000 & 0.5446 &   0.00 &  0.0000 \\
Trig 1 &      & 33.40 &    0.00 & 0.8348 & 0.0000 & 0.5506 &   0.00 &  0.4049 \\
Trig 2 &      & 33.05 &    0.00 & 0.8382 & 0.0000 & 0.5453 &   0.00 &  0.0468 \\
\hline
\end{longtable}

% The vector rotation method always rotates $P$ by the correct amount, and
% $R$ is always in the $S - P$ plane. The {\tt Trig 1} method rotates the
% tangent point by approximately the correct amount (within about 3.4\%),
% but it is out of the $S - P$ plane by as much as 5.3 degrees.  The {\tt
% Trig 2} method is more accurate by both measures.  The {\tt Trig 3} method
% rotates the tangent point by a slightly more accurate amount than either
% the {\tt Trig 1} or {\tt Trig 2} method, but the vector is as far out of
% the $S - P$ plane as the one the {\tt Trig 1} method computes.

\label{lastpage}
\end{document}

% $Id$

% $Log$
% Revision 1.1  2014/11/10 20:27:39  vsnyder
% Initial commit
%
