\documentclass[11pt]{article}

\usepackage{alltt}
\usepackage[fleqn]{amsmath}
\usepackage{floatflt}
\usepackage{graphicx}
\usepackage{longtable}
\usepackage[strings]{underscore}

\textwidth 6.5in
\oddsidemargin -0.25in
%\evensidemargin -0.5in
\topmargin -0.5in
\textheight 9in

\newcommand{\docname}{wvs-119}
\newcommand{\docdate}{7 May 2014}

\ifx\pdfoutput\undefined
  \pdfoutput=0
  \usepackage[hypertex,plainpages,hyperindex=true]{hyperref}
  \hypersetup{%
    hypertexnames=false%
  }
  % Specify the driver for the color package
  \ExecuteOptions{dvips}
  %\ExecuteOptions{xdvi}
\else
  \ifnum\pdfoutput>0
    \usepackage[pdftex,plainpages,hyperindex=true,pdfpagelabels]{hyperref}
    \hypersetup{%
      hypertexnames=false,%
      colorlinks=true,%
      linktocpage=true,%
    }
    % Specify the driver for the color package
    \ExecuteOptions{pdftex}
  \else
    \usepackage[hypertex,plainpages,hyperindex=true]{hyperref}
    \hypersetup{%
      hypertexnames=false%
    }
    % Specify the driver for the color package
    \ExecuteOptions{dvips}
    %\ExecuteOptions{xdvi}
  \fi
\fi

\hyperbaseurl{}
\newcommand\hr[1]{\href{#1.dvi}{dvi}, \href{#1.pdf}{pdf}}
\newcommand\h[1]{#1 (\hr{#1})}

\begin{document}

%\tracingcommands=1
\newlength{\hW} % heading box width
\newlength{\pW} % page number field width
\settowidth{\hW}{\bf\docname}
\settowidth{\pW}{Page \pageref{lastpage}\ of \pageref{lastpage}}
\ifdim \pW > \hW \setlength{\hW}{\pW} \fi
\makeatletter
\def\@biblabel#1{#1.}
\newcommand{\ps@twolines}{%
  \renewcommand{\@oddhead}{%
    \docdate\hfill\parbox[t]{\hW}{{\hfill\bf\docname}\newline
                          Page \thepage\ of \pageref{lastpage}}}%
\renewcommand{\@evenhead}{}%
\renewcommand{\@oddfoot}{}%
\renewcommand{\@evenfoot}{}%
}%
\makeatother
\pagestyle{twolines}

\renewcommand{\d}{\text{d}}
\newcommand{\T}{\mathcal{T}}

\vspace{-10pt}
\begin{tabbing}
\phantom{References: }\= \\
To: \>Nathaniel, Paul, Bill\\
Subject: \>Exploiting sparsity in Cholesky factorization\\
From: \>Van Snyder\\
Reference: \>\h{wvs-050} \\
\end{tabbing}

\vspace*{-15pt}

In \emph{A column approximate minimum degree ordering algorithm}, which
appeared in September 2004 {\bf ACM Transactions on Mathematical
Software}, the authors observe that permutation of the rows and columns of
a sparse symmetric matrix can have a profound impact on the number of
nonzeros in the Cholesky factors.

\parindent 0pt
\parskip 6pt

A companion of that paper, entitled \emph{Algorithm 836: COLAMD, A column
approximate minimum degree ordering algorithm}, describes C functions to
compute a ``good'' permutation.

In MLS Level 2, we solve the sparse least-squares problem
%
\begin{equation}
\mathbf{\tilde K \, \delta \tilde x} \simeq - \mathcal{F}
\end{equation}
%
for the Newton move $\mathbf{\delta \tilde x}$ by forming normal equations
%
\begin{equation}\label{two}
\mathbf{\tilde K ^T \tilde K \, \delta \tilde x} =
 - \mathbf{\tilde K ^T} \mathcal{F}
\end{equation}
%
followed by Cholesky factorization
%
\begin{equation}
\mathbf{\tilde K ^T \tilde K \, \delta \tilde x} =
\mathbf{U ^T U \, \delta \tilde x} =
 - \mathbf{\tilde K ^T} \mathcal{F}
\end{equation}
%
wherein $\mathbf{U}$ is upper triangular, and then solve for
$\mathbf{\delta \tilde x}$ using two back solves.  We take no care to
permute the rows and columns of $\mathbf{\tilde K ^T \tilde K}$.  The
factor $\mathbf{U}$ might therefore have more nonzeros than necessary to
compute the solution, and therefore might have required more time to
compute.

Abbreviate Equation (\ref{two}) to
%
\begin{equation}
\mathbf{A \, x} = \mathbf{b}
\end{equation}
%
where $\mathbf{A} = \mathbf{\tilde K ^T \tilde K}$, $\mathbf{x} = \mathbf{
\delta \tilde x}$, and $\mathbf{b} = - \mathbf{\tilde K ^T} \mathcal{F}$. 
Permute the rows and columns giving
%
\begin{equation}
\mathbf{P A P  \left(P^{-1}\, x \right)} = \mathbf{P b}
\phantom{space} \text{ or } \phantom{space}
\mathbf{P A P \, \hat x} = \mathbf{\hat b}
\end{equation}
%
where $\mathbf{P}$ is a permutation matrix, $\mathbf{\hat x} =
\mathbf{P^{-1} x}$ and $\mathbf{\hat b} = \mathbf{P b}$.  Cholesky factor
$\mathbf{P A P}$ giving
%
\begin{equation}
\mathbf{P A P \, \hat x} = \mathbf{\hat U ^T \hat U \, \hat x}
 = \mathbf{\hat b} \,.
\end{equation}
%
Solve for $\mathbf{\hat x}$ using two back solves, and then compute
$\mathbf{x} = \mathbf{ \delta \tilde x} = \mathbf{P \hat x}$.

The software that embodies Algorithm 836, as it applies to symmetric
matrices, is a C function entitled {\tt symamd}.  It does not examine or
manipulate the values of the matrix, or factor it.  Rather, it is given
the pattern of nonzeros, and returns a permutation.  In the case of MLS
Level 2, the pattern of nonzeros would be the pattern of non-absent matrix
blocks.  The Cholesky factoring and back solving procedures in {\tt
MatrixModule_1} could use the permutation to compute $\mathbf{\delta
\tilde x}$.

{\tt MatrixModule_0} provides a sparse representation of matrix blocks,
but the forward model does not exploit it.  Permuting rows and columns of
the diagonal blocks of $\mathbf{\tilde K ^T \tilde K}$ before Cholesky
factoring them could reduce computing time, but would require detecting
the sparsity pattern.  There is at present, however, no sparse Cholesky
factorization procedure in {\tt MatrixModule_0}.

\label{lastpage}
\end{document}

% $Id$

% $Log$
