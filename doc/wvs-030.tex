\documentclass[11pt]{article}

\usepackage[fleqn]{amsmath}

\textwidth 6.5in
\oddsidemargin -0.25in
%\evensidemargin -0.5in
\topmargin -0.5in
\textheight 9.00in

\newcommand{\docname}{\bf wvs-030r1}
\newcommand{\docdate}{1 Mar 2016}

\ifx\pdfoutput\undefined
  \pdfoutput=0
  \usepackage[hypertex,plainpages,hyperindex=true]{hyperref}
  \hypersetup{%
    hypertexnames=false%
  }
  % Specify the driver for the color package
  \ExecuteOptions{dvips}
  %\ExecuteOptions{xdvi}
\else
  \ifnum\pdfoutput>0
    \usepackage[pdftex,plainpages,hyperindex=true,pdfpagelabels]{hyperref}
    \hypersetup{%
      hypertexnames=false,%
      colorlinks=true,%
      linktocpage=true,%
    }
    % Specify the driver for the color package
    \ExecuteOptions{pdftex}
  \else
    \usepackage[hypertex,plainpages,hyperindex=true]{hyperref}
    \hypersetup{%
      hypertexnames=false%
    }
    % Specify the driver for the color package
    \ExecuteOptions{dvips}
    %\ExecuteOptions{xdvi}
  \fi
\fi

\hyperbaseurl{}
\newcommand\hr[1]{\href{#1.dvi}{dvi}, \href{#1.pdf}{pdf}}
\newcommand\h[1]{#1 (\hr{#1})}

\begin{document}

\renewcommand{\d}{\text{d}}

%\tracingcommands=1
\newlength{\hW} % heading box width
\newlength{\pW} % page number field width
\settowidth{\hW}{\docname}
\settowidth{\pW}{Page \pageref{lastpage}\ of \pageref{lastpage}}
\ifdim \pW > \hW \setlength{\hW}{\pW} \fi
\makeatletter
\def\@biblabel#1{#1.}
\newcommand{\ps@twolines}{%
  \renewcommand{\@oddhead}{%
    \docdate\hfill\parbox[u]{\hW}{{\docname}\newline
                          Page \thepage\ of \pageref{lastpage}}}%
\renewcommand{\@evenhead}{}%
\renewcommand{\@oddfoot}{}%
\renewcommand{\@evenfoot}{}%
}%
\makeatother
\pagestyle{twolines}

\vspace{-10pt}
\begin{tabbing}
\phantom{References: }\= \\
To: \>Nathaniel, Van\\
Subject: \>Points on line and surface of oblate spheroid that are nearest
to each other\\
From: \>Van Snyder\\
References: \>wvs-131 \hr{wvs-131}
\end{tabbing}

\parindent 0pt \parskip 3pt
\vspace{-20pt}

\section*{Co\"ordinate approach}

Define an oblate spheroid $\mathcal{S}(r)$
%
\begin{equation}\label{spheroid}
\frac{x^2}{a^2} + \frac{y^2}{a^2} + \frac{z^2}{c^2} = r^2
\end{equation}

and a line $\mathcal{L}$
%
\begin{equation}\begin{split}\label{line}
x =\,& x_0 + \alpha t \\
y =\,& y_0 + \beta  t \\
z =\,& z_0 + \gamma t \,.
\end{split}\end{equation}

There is a value of $r$ such that these intersect in one point, specified
by $t$.  At that point, the gradient of $\mathcal{S}(r)$ is orthogonal to
$\mathcal{L}$.  Substitute Equation (\ref{line}) into Equation
(\ref{spheroid}) giving
%
\begin{equation}\label{spheroid-t}
\frac{(x_0+\alpha t)^2}{a^2} + \frac{(y_0+\beta t)^2}{a^2} +
\frac{(z_0+\gamma t)^2}{c^2} = r^2 \,.
\end{equation}

Substitute Equation (\ref{line}) into the gradient  of Equation
(\ref{spheroid}), take the inner product with the vector parallel to
$\mathcal{L}$, \emph{viz.} $(\alpha,\beta,\gamma)$, and set the result to
zero, giving
%
\begin{equation}
\frac{\alpha(x_0+\alpha t)}{a^2} + \frac{\beta(y_0+\beta t)}{a^2} +
\frac{\gamma(z_0+\gamma t)}{c^2} = 0 \,.
\end{equation}

Solve for $t$ giving
%
\begin{equation}\label{t}
t_1 = -\frac{c^2 (\alpha x_0 + \beta y_0 ) + a^2 z_0 \gamma}
          {c^2 ( \alpha^2 + \beta^2) + a^2 \gamma^2} \,.
\end{equation}

Substituting this value into Equation (\ref{spheroid-t}) gives $r$, while
substituting it into Equation (\ref{line}) gives $(x_1,y_1,z_1)$, the
point where $\mathcal{L}$ intersects $\mathcal{S}(r)$.  If $r \leq 1$
then $\mathcal{L}$ intersects $\mathcal{S}(1)$.  If $r \approx 1$ then
$(x_1,y_1,z_1)$ is very close to the point on $\mathcal{L}$ that is
nearest to $\mathcal{S}(1)$, and $(x_1/r,y_1/r,z_1/r)$ is very close to
the point on $\mathcal{S}(1)$ that is closest to $\mathcal{L}$.

To find the point on $\mathcal{S}(1)$ that is nearest to $(x_1,y_1,z_1)$,
define the surface of $\mathcal{S}(1)$ parametrically in spherical
co\:ordinates:
%
\begin{equation}\label{spherical}\begin{split}
x =\,& a \cos \phi \cos \theta \\
y =\,& a \cos \phi \sin \theta \\
z =\,& c \sin \phi \,.
\end{split}\end{equation}

The square of the distance, $s^2$, from ($x_1,y_1,z_1$) to
$\mathcal{S}(1)$ as defined using Equation (\ref{spherical}) is
%
\begin{equation}\begin{split}
s^2 =\,& (x-x_1)^2 + (y-y_1)^2 + (z-z_1)^2\\
    =\,& (a\cos\theta\cos\phi-x_1)^2+(a\sin\theta\cos\phi-y_1)^2+(c\sin\phi-z_1)^2
\,.
\end{split}\end{equation}

The derivatives with respect to $\theta$ and $\phi$ are
\begin{equation}\begin{split}
-\frac1{2a}\frac{\partial s}{\partial \theta} =\,&
(x_1\sin\theta-y_1\cos\theta)\cos\phi \\
-\frac12\frac{\partial s}{\partial \phi} =\,& 
 a \sin\phi \left( a \cos\phi - y_1\sin\theta - x_1\cos\theta \right) -
 c \cos\phi \left( c \sin\phi - z_1 \right)
\,.
\end{split}\end{equation}

Setting both derivatives to zero and solving for $\theta$ and $\phi$ gives
%
\begin{equation}\label{on-ellipse}\begin{split}
\theta =\,& \tan^{-1} \frac{y_1}{x_1} \\
\phi   =\,& \tan^{-1} \frac{g}{1 - v}
\end{split}\end{equation}

where $g = \frac{c z_1}{a \rho_1}$, $h = \frac{e^2}{a \rho_1}$, $e^2 = a^2-c^2$,
$\rho_1^2 = x_1^2+y_1^2$ and
$ v^4 - 2v^3 + (1+g^2-h^2)v^2 + 2 h^2 v - h^2 = 0$.
As a sanity check, consider the case $a=c$, equivalently $h=0$.  We have
$v^4 - 2v^3 + (1+g^2)v^2 = 0$, or $v = \{0,0,1+ig,1-ig\}$, giving for the
real roots, as expected, $\phi = \tan^{-1}g =
\tan^{-1}\frac{z_1}{\rho_1}$.

Let $(x_2,y_2,z_2) = (a \cos\phi \cos\theta, a \cos\phi \sin\theta, c
\sin\phi)$.  The point on $\mathcal{L}$ nearest to $(x_2,y_2,z_2)$ is given by
%
\begin{equation}\label{on-line}
t_2 = \frac{(x_2-x_0)\alpha+(y_2-y_0)\beta+(z_2-z_0)\gamma}
      {\alpha^2+\beta^2+\gamma^2}
\,.
\end{equation}

If $|(x_2,y_2,z_2) - (x_1,y_1,z_1)| <
    |(x_1/r, y_1/r, z_1/r) - (x_1,y_1,z_1)|$,
Equations (\ref{line}), (\ref{on-ellipse}) and (\ref{on-line}) can be
iterated.

\section{Vector approach}

Define a point $\mathbf{p}(t)$ on a line $\mathcal{L}$
%
\begin{equation}\label{p}
\mathbf{p}(t) = \mathbf{C} + t \, \mathbf{U}
\end{equation}

where $\mathbf{C} = [ x_1, y_1, z_1 ]^T$ and
      $\mathbf{U} = [ x_2, y_2, z_2 ]^T$.

Define an ellipsoid $\mathcal{S}(r)$ with axes in co\"ordinate directions
and center at $\mathbf{p}_0$, and substitute Equation (\ref{p}), giving
%
\begin{equation}\label{E}
\left(\mathbf{M}( \mathbf{p}(t) - \mathbf{p}_0 )\right)^T
 \left(\mathbf{M}( \mathbf{p}(t) - \mathbf{p}_0 )\right) =
\left(\mathbf{M}(\mathbf{C} - \mathbf{p}_0 + t \mathbf{U})\right)^T 
\left(\mathbf{M}(\mathbf{C} - \mathbf{p}_0 + t \mathbf{U})\right) = r^2
\end{equation}

where
%
\begin{equation}
\mathbf{M} = \left[ \begin{array}{ccc} \frac1a & 0 & 0 \\
                                       0 & \frac1b & 0 \\
                                       0 & 0 & \frac1c \\
                    \end{array} \right]
\end{equation}

and $a$, $b$, and $c$ are the semi-minor axis lengths.  Let $\mathbf{D} =
\mathbf{C} - \mathbf{p}_0$.  Computing the gradient of Equation (\ref{E})
gives
%
\begin{equation}
\frac12\, \mathbf{G} =
 \mathbf{M}^T \mathbf{M} ( \mathbf{p}(t) - \mathbf{p}_0 ) =
 \mathbf{M}^T \mathbf{M} (\mathbf{D} + t\, \mathbf{U}) \,.
\end{equation}

There is a value of $r$ such that $\mathcal{L}$ and $\mathcal{S}(r)$
intersect in one point.  At this point, $\mathbf{G}$ is orthogonal to
$\mathbf{U}$, i.e.,
%
\begin{equation}
\frac12\, \mathbf{G} \cdot \mathbf{U} =
 \left(\mathbf{M} \mathbf{p}(t)\right)^T \mathbf{M}\mathbf{U} =
 \left(\mathbf{M} \mathbf{(\mathbf{D} +
  t \mathbf{U})}\right)^T \mathbf{M}\mathbf{U} =
 \left(\mathbf{M} \mathbf{D}\right)^T \mathbf{M}\mathbf{U} +
 t \left(\mathbf{M} \mathbf{U}\right)^T \mathbf{M}\mathbf{U}
  = 0 \,.
\end{equation}

Solving for $t$ we have
%
\begin{equation}\label{tv}
t =
-\frac{\left(\mathbf{M}\mathbf{D}\right)^T
       \left(\mathbf{M}\mathbf{U}\right)}
      {\left(\mathbf{M}\mathbf{U}\right)^T
       \left(\mathbf{M}\mathbf{U}\right)} =
-\frac{\frac{x_1 x_2}{a^2} + \frac{y_1 y_2}{b^2} + \frac{z_1 z_2}{c^2}}
      {\frac{x_2^2}{a^2} + \frac{y_2^2}{b^2} + \frac{z_2^2}{c^2}} \,.
\end{equation}

Substituting Equation (\ref{tv}) into Equation(\ref{E}) gives a value for
$r$, while substituting it into Equation (\ref{p}) gives the point
$\mathbf{p}(t)$ where $\mathcal{L}$ is nearest to $\mathcal{S}(r)$.  If
$r < 1$, $\mathcal{L}$ intersects $\mathcal{S}(r)$ at $\mathbf{p}(t)$;
otherwise $\mathbf{p}(t)$ is the point on $\mathcal{L}$ that is closest to
$\mathcal{S}(r)$.

Let $\mathbf{N} = \mathbf{p}(t) + u \, \mathbf{G}$ be a normal to
$\mathcal{S}(1)$ that intersects $\mathbf{p}(t)$.  Then the point on
$\mathcal{S}(1)$ that is nearest to $\mathbf{p}(t)$ is the point where
$\mathbf{N}$ intersects $\mathcal{S}(1)$.  This is described in wvs-131
\hr{wvs-131}.

\label{lastpage}
\end{document}
% $Id$
