\documentclass[11pt]{article}
\usepackage{alltt}
\usepackage[fleqn]{amsmath}
\usepackage{floatflt}
\usepackage{graphicx}
\usepackage{longtable}
\usepackage[strings]{underscore}

\textwidth 6.5in
\oddsidemargin -0.25in
%\evensidemargin -0.5in
\topmargin -0.5in
\textheight 9in

\newcommand{\docname}{wvs-138}
\newcommand{\docdate}{1 June 2017}

\ifx\pdfoutput\undefined
  \pdfoutput=0
  \usepackage[hypertex,plainpages,hyperindex=true]{hyperref}
  \hypersetup{%
    hypertexnames=false%
  }
  % Specify the driver for the color package
  \ExecuteOptions{dvips}
  %\ExecuteOptions{xdvi}
\else
  \ifnum\pdfoutput>0
    \usepackage[pdftex,plainpages,hyperindex=true,pdfpagelabels]{hyperref}
    \hypersetup{%
      hypertexnames=false,%
      colorlinks=true,%
      linktocpage=true,%
    }
    % Specify the driver for the color package
    \ExecuteOptions{pdftex}
  \else
    \usepackage[hypertex,plainpages,hyperindex=true]{hyperref}
    \hypersetup{%
      hypertexnames=false%
    }
    % Specify the driver for the color package
    \ExecuteOptions{dvips}
    %\ExecuteOptions{xdvi}
  \fi
\fi

\hyperbaseurl{}
\newcommand\hr[1]{\href{#1.dvi}{dvi}, \href{#1.pdf}{pdf}}
\newcommand\h[1]{#1 (\hr{#1})}

\begin{document}

%\tracingcommands=1
\newlength{\hW} % heading box width
\newlength{\pW} % page number field width
\settowidth{\hW}{\bf\docname}
\settowidth{\pW}{Page \pageref{lastpage}\ of \pageref{lastpage}}
\ifdim \pW > \hW \setlength{\hW}{\pW} \fi
\makeatletter
\def\@biblabel#1{#1.}
\newcommand{\ps@twolines}{%
  \renewcommand{\@oddhead}{%
    \docdate\hfill\parbox[t]{\hW}{{\hfill\bf\docname}\newline
                          Page \thepage\ of \pageref{lastpage}}}%
\renewcommand{\@evenhead}{}%
\renewcommand{\@oddfoot}{}%
\renewcommand{\@evenfoot}{}%
}%
\makeatother
\pagestyle{twolines}

\newcommand{\Z}{$\zeta$}

\vspace{-10pt}
\begin{tabbing}
\phantom{References: }\= \\
To: \>Van\\
Subject: \>Interpolating coefficients, index arrays, and flags in the
           forward model \\
From: \>Van Snyder\\
Reference: \>\h{wvs-090}, \h{wvs-093}\\
\end{tabbing}

\parindent 0pt \parskip 6pt
\vspace{-20pt}

The forward model integrates the radiative-transfer equation along several
lines of sight at several frequencies, and calculates derivatives of the
computed radiance with respect to state vector elements.  See \h{wvs-090}
and \h{wvs-093}.

The \emph{state vector} is a one-dimensional concatenation of all species
under consideration by the forward model.  Each element's value
corresponds to a four-tuple $[f, \zeta, \phi, S]$, where $f$ is frequency,
$\zeta$ is the negative base-10 logarithm of pressure, $\phi$ is orbit
geodetic angle, and $S$ is a species index.  Most species' mixing ratios
do not depend upon frequency.  The number of $\zeta$ indices could be
different for each species.  The number of $\phi$ indices is assumed to be
the same for all species.

The {\tt Grids_f} structure defined in {\tt load_sps_data_m} is a
representation of the state vector, including the frequencies and
geolocations for each element, taken from \emph{vector quantities} of the
{\tt Input} and {\tt Extra} \emph{vectors} provided to the forward model. 
For details of the distinction between \emph{vector quantities} and
\emph{vectors}, consult {\tt VectorsModule}.

There are several arrays of indices and flags in the forward model that
indicate where calculations are needed.

The \emph{coarse path} is the set of points where the line of sight
pierces a surface of constant pressure.  The tangent point is represented
twice so that we can multiply by Earth reflectivity if the ray is
reflected from the Earth's surface, or 1.0 if it's not.

The set of constant pressure surfaces that the forward model uses is the
union of the sets of pressure coordinates for all state vector elements.

The \emph{fine path} is the coarse path with Gauss-Legendre quadrature
abscissae inserted between every pair of coarse path points.  There is
space for these abscissae between the two coarse path points that
represent the tangent point, but they're never used.  We might someday
choose to use Gauss-Lobatto quadrature.  ``GL'' is used below for either
Gauss-Legendre quadrature or Gauss-Lobatto quadrature.

\begin{description}
\item[{\tt NG}] is the number of GL abscissae inserted between coarse path
  points.
\item[{\tt NGP1}] is {\tt NG + 1}
\item[{\tt Eta_ZP}] is an array of interpolating coefficients from the
  $\zeta$ and $\phi$ coordinates of the state vector onto the fine path,
  for all species including those that depend upon frequency.
\item[{\tt Do_Calc_ZP}] is an array of logical values indicating where
  {\tt Eta_ZP} is not zero.
\item[{\tt Eta_FZP}] is an array of interpolating coefficients from the
  $f$, $\zeta$ and $\phi$ coordinates of the state vector onto the fine
  path, also for all species. It is initially spread from {\tt Eta_ZP} for
  those quantities that depend upon frequency, and then multiplied by the
  interpolating coefficient from the frequency grid for the species to the
  pointing frequency grid for the line of sight.  {\tt Eta_ZP} and {\tt
  Eta_FZP} are separate because {\tt Eta_ZP} depends only upon the line of
  sight, and is therefore computed outside the frequency loop.  The
  frequency dependency of {\tt Eta_FZP} is computed within the frequency
  loop.
\item[{\tt Do_Calc_FZP}] is an array of logical values indicating where
  {\tt Eta_FZP} is not zero.
\item[{\tt Do_GL}] is a logical array indexed by the coarse path index
  that indicates it is necessary to use GL for sufficient accuracy. 
  Otherwise, trapezoidal quadrature is used.
\item[{\tt F_Inds}] is an array of integers that gives the fine path
  indices of all and only the GL abscissae.  This includes indices for GL
  abscissae that are not used because trapezoidal quadrature is good
  enough; it does not include the coarse path points.
\item[{\tt GL_Inds}] is an array of integers, a subset of {\tt F_Inds},
  that gives the fine path indices only of GL abscissae on panels of the
  coarse path for which GL is necessary.  This does not include the coarse
  path points, or fine path points on panels where GL is not necessary.
\end{description}

Within {\tt rad_tran_m} the {\tt Get_Do_Calc_Indexed} subroutine uses {\tt
GL_Inds} (which it confusingly calls {\tt F_Inds}) to calculate

\begin{description}
\item[{\tt Do_Calc}] is a logical array indexed by the coarse path index that
  indicates that {\tt Do_Calc_FZP} is true for at least one GL point in
  the panel.  {\tt Do_Calc} might be true even if neither of the
  elements at coarse-path positions in {\tt Do_Calc_FZP} are true.
\item[{\tt Inds}] are coarse path indices where {\tt Do_Calc} is true.
\item[{\tt N_Inds}] is the number of true elements in {\tt Do_Calc}, and
  the number of meaningful elements in {\tt Inds}.
\end{description}

\label{lastpage}
\vspace*{-0.1in} % Somehow, this causes lastpage to be defined
\end{document}

% $Id$

% $Log$
