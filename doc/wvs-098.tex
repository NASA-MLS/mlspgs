\documentclass[11pt]{article}
\usepackage{alltt}
\usepackage[fleqn]{amsmath}
\usepackage{floatflt}
\usepackage{graphicx}
\usepackage{longtable}

\textwidth 6.5in
\oddsidemargin -0.25in
%\evensidemargin -0.5in
\topmargin -0.5in
\textheight 9in

\newcommand{\docname}{wvs-098r2}
\newcommand{\docdate}{20 August 2012}

\ifx\pdfoutput\undefined
  \pdfoutput=0
  \usepackage[hypertex,plainpages,hyperindex=true]{hyperref}
  \hypersetup{%
    hypertexnames=false%
  }
  % Specify the driver for the color package
  \ExecuteOptions{dvips}
  %\ExecuteOptions{xdvi}
\else
  \ifnum\pdfoutput>0
    \usepackage[pdftex,plainpages,hyperindex=true,pdfpagelabels]{hyperref}
    \hypersetup{%
      hypertexnames=false,%
      colorlinks=true,%
      linktocpage=true,%
    }
    % Specify the driver for the color package
    \ExecuteOptions{pdftex}
  \else
    \usepackage[hypertex,plainpages,hyperindex=true]{hyperref}
    \hypersetup{%
      hypertexnames=false%
    }
    % Specify the driver for the color package
    \ExecuteOptions{dvips}
    %\ExecuteOptions{xdvi}
  \fi
\fi

\hyperbaseurl{}
\newcommand\hr[1]{\href{#1.dvi}{\textnormal{dvi}}, \href{#1.pdf}{\textnormal{pdf}}}
\newcommand\h[1]{#1 (\hr{#1})}
\newcommand\hb[1]{{\bf #1} (\hr{#1})}

\begin{document}

%\tracingcommands=1
\newlength{\hW} % heading box width
\newlength{\pW} % page number field width
\settowidth{\hW}{\bf\docname}
\settowidth{\pW}{Page \pageref{lastpage}\ of \pageref{lastpage}}
\ifdim \pW > \hW \setlength{\hW}{\pW} \fi
\makeatletter
\def\@biblabel#1{#1.}
\newcommand{\ps@twolines}{%
  \renewcommand{\@oddhead}{%
    \docdate\hfill\parbox[t]{\hW}{{\hfill\bf\docname}\newline
                          Page \thepage\ of \pageref{lastpage}}}%
\renewcommand{\@evenhead}{}%
\renewcommand{\@oddfoot}{}%
\renewcommand{\@evenfoot}{}%
}%
\makeatother
\pagestyle{twolines}

\vspace{-10pt}
\begin{tabbing}
\phantom{References: }\= \\
To: \>Bill, Igor, Nathaniel\\
Subject: \>Mixing-ratio derivative of radiance\\
From: \>Van Snyder\\
Reference: \>\h{iy-007}, \h{wvs-093}, \h{wvs-099}\\
\end{tabbing}

\parindent 0pt \parskip 6pt
\vspace{-10pt}

{\bf This is oversimplified.  See \h{wvs-093} and \h{wvs-099}.}

Using notation similar to iy-007, the solution of the radiative transfer
equation in the form used in the full forward model is

\begin{equation}\label{one}
I(s_m) = (I(s_0)-B(s_0)) \mathcal{T}(s_0) + B(s_m) +
 \int_{B(s_m)}^{B(s_0)} \mathcal{T}(s) \, \text{d} B
\end{equation}

where $s_m$ is the location of the instrument, $s_0$ is the location of
the end of the path away from the instrument, $B$ is the Planck function
$\frac{h \nu}k \left( \exp\left(\frac{h \nu}{k T}\right) -1\right)^{-1}$,
and

\begin{equation}\label{two}
\mathcal{T}(s) = \exp\left( - \int_s^{s_m} \sum_k \beta_k(x) f_k(x)
 \, \text{d} x \right)
\end{equation}

where $f_k(x)$ is the volume mixing ratio of the $k^\text{th}$ chemical
species and $\beta_k(x)$ is its absorption cross section.  Taking the
derivative of Equation (\ref{one}) with respect to $f_k(x)$ and observing
that $I(0)$ and $B(\cdot)$ do not depend upon $f_k(x)$ gives

\begin{equation}\label{three}
\frac{\partial I(s_m)}{\partial f_k(x)} =
 (I(s_0)-B(s_0)) \frac{\partial \mathcal{T}(s_0)}{\partial f_k(x)} +
  \int_{B(s_m)}^{B(s_0)} \frac{\partial\mathcal{T}(s)}{\partial f_k(x)}
   \, \text{d} B \,.
\end{equation}

Taking the derivative of Equation (\ref{two}) with respect to $f_k(x)$
gives

\begin{equation}
\frac{\partial \mathcal{T}(s)}{\partial f_k(x)} =
 -\beta_k(x) \mathcal{T}(s)\,.
\end{equation}

Substituting that derivative into Equation (\ref{three}) gives

\begin{equation}\label{five}
\frac{\partial I(s_m)}{\partial f_k(x)} =
 -(I(s_0)-B(s_0)) \beta_k(x) -
 \int_{B(s_m)}^{B(s_0)} \beta_k(x) \mathcal{T}(s) \, \text{d} B \,.
\end{equation}

$\beta_k(x)$ does not depend upon $s$.  Therefore it can be taken out of
the integral in Equation (\ref{five}), giving


\begin{equation}
\frac{\partial I(s_m)}{\partial f_k(x)} = -\beta_k(x) (I(s)-B(s_m)) \,.
\end{equation}

This shows that a variational equation such as Equation (\ref{three}) or
Equation (\ref{five}) does not need to be integrated to compute
derivatives of radiance with respect to mixing ratio.

\label{lastpage}
\end{document}

%$Id$

%$Log$
%Revision 1.2  2010/12/04 02:52:50  vsnyder
%Add 'oversimplified' caveat
%
%Revision 1.1  2010/11/13 01:34:11  vsnyder
%Initial commit
%
