\documentclass[11pt]{article}
\usepackage[fleqn]{amsmath}\textwidth 6.5in
\oddsidemargin -0.25in
%\evensidemargin -0.5in
\topmargin -0.25in
\textheight 9.0in

\newcommand{\docname}{\bf wvs-082}
\newcommand{\docdate}{15 January 2009}

\begin{document}

%\tracingcommands=1
\newlength{\hW} % heading box width
\newlength{\pW} % page number field width
\settowidth{\hW}{\docname}
\settowidth{\pW}{Page \pageref{lastpage}\ of \pageref{lastpage}}
\ifdim \pW > \hW \setlength{\hW}{\pW} \fi
\makeatletter
\def\@biblabel#1{#1.}
\newcommand{\ps@twolines}{%
  \renewcommand{\@oddhead}{%
    \docdate\hfill\parbox[u]{\hW}{{\hfill\docname}\newline
                          Page \thepage\ of \pageref{lastpage}}}%
\renewcommand{\@evenhead}{}%
\renewcommand{\@oddfoot}{}%
\renewcommand{\@evenfoot}{}%
}%
\makeatother
\pagestyle{twolines}

\vspace{-10pt}
\begin{tabbing}
\phantom{References: }\= \\
To: \>Nathaniel, Bill, Alyn, Michelle\\
Subject: \>Current status of scattering additions to forward model\\
From: \>Van Snyder\\
\end{tabbing}

\parindent 0pt \parskip 10pt
\vspace{-20pt}

I've been adding support for scattering to the forward model.  The problem is
in three stages.

\begin{enumerate}

\item Calculate Mie coefficients offline.  This is completed.  The program
is in {\tt mlspgs/Mie}.

\item\label{two} Use Mie coefficients to calculate $T_{\text{scat}}$,
again to be done offline.  This is partly completed.

\item Use $T_{\text{scat}}$ online, similarly to the way L2PC are used. 
This has not been started.

\end{enumerate}

Item \ref{two} above requires several steps.

\begin{enumerate}

\item Allow the radiative transfer calculations to be carried out on only
part of a path.  This appears to be completed, but needs more testing.

\item Compute a path at a specified phase angle from the horizon, in the
plane of an orbit with a specified inclination, that intersects a
specified scattering point in the orbital plane.  See {\tt
mlspgs/doc/wvs-074}.  Phase angles that result in paths that do not
reflect from the Earth's surface are specified by pressure surfaces. 
This appears to be working.  Phase angles that result in paths that
reflect from the Earth's surface are specified explicitly.  This works
for some angles, but for steeply reflecting rays, {\tt metrics} insists
on putting the path on the other side of the earth from the scattering
point.  The difficulty has been in trying to manipulate {\tt metrics} by
remote control, through several layers of procedure references.  I have
an IDL prototype in $\sim${\tt vsnyder/idl/scat\_gen.pro} that calculates
paths correctly, but the method it uses doesn't fit into the full forward
model.

\item Convolve the radiances arriving at the scattering point from the
several angles with the Mie phase function to product $T_{\text{scat}}$. 
This has not been begun, but is a straight-forward problem.

\end{enumerate}

I have not checked in any of the code developed for item \ref{two}, but I
have periodically checked that it does not produce results that disagree
with any of my reference runs.  At present, the forward model and metrics
codes in my directory are littered with PRINT statements.

\label{lastpage}
\end{document}

% $Id$

% $Log$
% Revision 1.1  2009/06/22 23:05:45  vsnyder
% Initial commit
%
