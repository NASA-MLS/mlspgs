\documentclass[12pt]{article}

\usepackage[fleqn]{amsmath}

\textwidth 6.5in
\oddsidemargin -0.25in
%\evensidemargin -0.5in
\topmargin -0.5in
\textheight 9.1in

\newcommand{\docname}{wvs-015r1}
\newcommand{\docdate}{28 January 2021}

\ifx\pdfoutput\undefined
  \pdfoutput=0
\fi
\ifnum\pdfoutput>0
  \usepackage[pdftex,plainpages,hyperindex=true,pdfpagelabels]{hyperref}
  \hypersetup{%
    hypertexnames=false,%
    colorlinks=true,%
    linktocpage=true,%
  }
  % Specify the driver for the color package
  \ExecuteOptions{pdftex}
\else
  \usepackage[hypertex,plainpages,hyperindex=true]{hyperref}
  \hypersetup{%
    hypertexnames=false%
  }
  % Specify the driver for the color package
  \ExecuteOptions{dvips}
  %\ExecuteOptions{xdvi}
\fi

\hyperbaseurl{}
\newcommand\hr[1]{\href{#1.dvi}{dvi}, \href{#1.pdf}{pdf}}
\newcommand\h[1]{#1 (\hr{#1})}

\begin{document}

\newlength{\hW} % heading box width
\newlength{\pW} % page number field width
\settowidth{\hW}{\bf\docname}
\settowidth{\pW}{Page \pageref{lastpage}\ of \pageref{lastpage}}
\ifdim \pW > \hW \setlength{\hW}{\pW} \fi
\makeatletter
\def\@biblabel#1{#1.}
\newcommand{\ps@twolines}{%
  \renewcommand{\@oddhead}{%
    \docdate\hfill\parbox[t]{\hW}{{\hfill\bf\docname}\newline
                          Page \thepage\ of \pageref{lastpage}}}%
\renewcommand{\@evenhead}{}%
\renewcommand{\@oddfoot}{}%
\renewcommand{\@evenfoot}{}%
}%
\makeatother
\pagestyle{twolines}

\vspace{-10pt}
\begin{tabbing}
\phantom{References: }\= \\
To: \>Michael, Bill, Paul, William Daffer\\
Subject: \>Exponential of a $2\times2$ matrix, and derivative thereof\\
From: \>Van Snyder\\
\end{tabbing}

\parindent 0pt \parskip 3pt
\vspace{-20pt}

This memo describes how to compute the exponential function of a $2 \times 2$
matrix, and the derivative thereof, without encountering problems as the
eigenvalues approach each other.

The original version of this note had some needless complication.

One can apply the Hamilton-Cayley theorem (``A matrix satisfies its
characteristic equation'') to the series representation of any function of a
matrix to develop a formula known as ``Sylvester's Identity.''  In the case of
a $2 \times 2$ matrix {\bf A}, having eigenvalues $\lambda_1$ and $\lambda_2$,
we have

\begin{equation}
F(\mathbf{A}) = \frac{F(\lambda_1)(\mathbf{A}-\lambda_2\mathbf{I})}
                        {\lambda_1-\lambda_2} +
                   \frac{F(\lambda_2)(\mathbf{A}-\lambda_1\mathbf{I})}
                        {\lambda_2-\lambda_1} \;.
\end{equation}

When written in this form, there is clearly a computational problem as
$\lambda_1 \rightarrow \lambda_2$.

In our case $F = \exp$.  Using
 $s = \frac12 ( \lambda_1 + \lambda_2 ) = \frac12 \text{tr}(\mathbf{A})$
and
 $d = \frac12 ( \lambda_1 - \lambda_2 ) = \sqrt { s^2 - \det(\mathbf{A})}$,
this can be rewritten as

\begin{equation}
\exp(\mathbf{A}) =
  e^s \left [ 
    \frac{\sinh d}d ( \mathbf{A} - s\, \mathbf{I} ) + \cosh d\, \mathbf{I}
      \right ] \;.
\end{equation}

Noticing that $\lim_{d\rightarrow 0} \frac{\sinh\,d}d = 1$, one can conclude
that there must be a way to compute $\frac{\sinh\,d}d$ without computational
difficulty, even as $d \rightarrow 0$. The easiest way is to use 1 if $|d| <
\sqrt{\rho}$, where $\rho$ is the unit round-off of the arithmetic in use, a
Taylor series

\begin{equation}\label{three}
\frac{\sinh\,d}d = \sum_{i=0}^\infty \frac{d^{2 i}}{(2 i+1)!}
=
1 + \frac{d^2}6 + \frac{d^4}{24} + \frac{d^6}{5040} + O(d^8)
\end{equation}

for larger but still small $|d|$, and the Fortran intrinsic SINH function
otherwise. For complex $d$ with $|d|<0.12$, this series converges to 16-digit
accuracy with seven or fewer terms.

Derivatives of $\exp(\mathbf{A})$ with respect to a parameter $P$, say
temperature, are needed. The derivative of a matrix is just the matrix of the
derivatives of its elements. For $\exp(\mathbf{A})$, one might write this as 

\renewcommand{\d}{\text{d}}
\begin{equation}\begin{split}
\frac{\d \exp(\mathbf{A})}{\d P}
 = \,& \exp(\mathbf{A}) \frac{\d \mathbf{A}}{\d P} \text{, or is it }
\frac{\d \exp(\mathbf{A})}{\d P}
 = \,& \frac{\d \mathbf{A}}{\d P} \exp(\mathbf{A}) \,?
\end{split}\end{equation}

In general

\begin{equation}
\exp(\mathbf{A}) \text{ and }
\frac{\d \mathbf{A}}{\d P}
\end{equation}

do not commute. Therefore, compute

\begin{equation}
\frac{\d \exp(\mathbf{A})}{\d P}
\end{equation}

by computing the derivatives of its elements. Denote $\frac{\d s}{\d P}$ by
$s^\prime$ and $\frac{\d d}{\d P}$ by $d^\prime$. Then

\begin{equation}
\frac{\d\exp(\mathbf{A})}{\d P} =
s^\prime \exp(\mathbf{A}) +
  e^s \left [ 
    d^\prime \frac{\d}{\d P} \left(\frac{\sinh d}d \right)
        ( \mathbf{A} - s\, \mathbf{I} )
    + \left( d^\prime\,  \sinh d - s^\prime\, \frac{\sinh d}d \, \right) \mathbf{I}
      \right ] \;.
\end{equation}

Using the usual method,

\begin{equation}
\frac{\d}{\d P} \frac{\sinh d}d = \frac{\cosh d}d - \frac{\sinh d}{d^2} \,,
\end{equation}

which is clearly problematic as $d \rightarrow 0$. But a Taylor series, the
derivative of Equation (\ref{three})

\begin{equation}
\frac{\d}{\d P} \frac{\sinh d}d
 = \sum_{i=1}^\infty \frac{2 i}{(2 i + 1)!} x^{2 i - 1}
\end{equation}

converges rapidly.

\label{lastpage}
\end{document}
