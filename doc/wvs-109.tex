\documentclass[11pt]{article}
\usepackage[fleqn]{amsmath}

\textwidth 6.5in
\oddsidemargin -0.25in
%\evensidemargin -0.5in
\topmargin -0.25in
\textheight 9.0in

\newcommand{\docname}{\bf wvs-109}
\newcommand{\docdate}{26 July 2012}

\ifx\pdfoutput\undefined
  \pdfoutput=0
  \usepackage[hypertex,plainpages,hyperindex=true]{hyperref}
  \hypersetup{%
    hypertexnames=false%
  }
  % Specify the driver for the color package
  \ExecuteOptions{dvips}
  %\ExecuteOptions{xdvi}
\else
  \ifnum\pdfoutput>0
    \usepackage[pdftex,plainpages,hyperindex=true,pdfpagelabels]{hyperref}
    \hypersetup{%
      hypertexnames=false,%
      colorlinks=true,%
      linktocpage=true,%
    }
    % Specify the driver for the color package
    \ExecuteOptions{pdftex}
  \else
    \usepackage[hypertex,plainpages,hyperindex=true]{hyperref}
    \hypersetup{%
      hypertexnames=false%
    }
    % Specify the driver for the color package
    \ExecuteOptions{dvips}
    %\ExecuteOptions{xdvi}
  \fi
\fi

\hyperbaseurl{}
\ifx\dvidir\undefined
  \newcommand\hr[1]{\href{#1.dvi}{dvi} \href{#1.pdf}{pdf}}
\else
  \newcommand\hr[1]{\href{\dvidir/#1.dvi}{dvi} \href{\pdfdir/#1.pdf}{pdf}}
\fi
\newcommand\h[1]{#1 \hr{#1}}

\begin{document}

%\tracingcommands=1
\newlength{\hW} % heading box width
\newlength{\pW} % page number field width
\settowidth{\hW}{\docname}
\settowidth{\pW}{Page \pageref{lastpage}\ of \pageref{lastpage}}
\ifdim \pW > \hW \setlength{\hW}{\pW} \fi
\makeatletter
\def\@biblabel#1{#1.}
\newcommand{\ps@twolines}{%
  \renewcommand{\@oddhead}{%
    \docdate\hfill\parbox[u]{\hW}{{\hfill\docname}\newline
                          Page \thepage\ of \pageref{lastpage}}}%
\renewcommand{\@evenhead}{}%
\renewcommand{\@oddfoot}{}%
\renewcommand{\@evenfoot}{}%
}%
\makeatother
\pagestyle{twolines}

\vspace{-10pt}
\begin{tabbing}
\phantom{References: }\= \\
To: \>Van, Nathaniel, Bill, Alyn, Paul\\
Subject: \>Might be doing Tikhonov wrong\\
From: \>Van Snyder\\
Reference: \>\h{wvs-050}\\
\end{tabbing}

\parindent 0pt \parskip 5pt
\vspace{-20pt}

We might be doing the Tikhonov regularization wrong, or in any case
suboptimally.

Our Newton iteration consists of solving a least squares problem for the
Newton move $\delta \mathbf{x} = \mathbf{x}_{n+1}-\mathbf{x}_n$:

\begin{equation}\label{one}
\left[ \begin{array}{c}
 \mathbf{G}_\epsilon^{-1} \mathbf{J} \\
 \mathbf{G}_a^{-1} \\
 \mathbf{W R} \\
 \lambda \mathbf{I} \end{array} \right]
\delta \mathbf{x} \simeq
\left[ \begin{array}{c}
 - \mathbf{G}_\epsilon^{-1} \mathbf{F}(\mathbf{x}_n) \\
 \mathbf{G}_a^{-1} ( \mathbf{x}_a - \mathbf{x}_n ) \\
 - \mathbf{W R}\, \mathbf{x}_n \\
 0 \end{array} \right]
\end{equation}

where $\mathbf{G}_\epsilon$ is the Cholesky factor of the measurement
covariance, $\mathbf{J}$ is the Jacobian of the forward model,
$\mathbf{F}(\mathbf{x}_n)$ is the forward model evaluated at the current
state $\mathbf{x}_n$, $\mathbf{G}_a$ is the Cholesky factor of the apriori
covariance, $\mathbf{x}_a$ is the apriori state, $\mathbf{W}$ provides the
weights for Tikhonov regularization (the {\tt hRegWeights} and {\tt
vRegWeights} quantities in the {\tt l2cf}, which were obtained
empirically), $\mathbf{R}$ is the Tikhonov regularization operator, in our
case a second difference operator,

\begin{equation}\label{two}
\mathbf{R} =
\left[ \begin{array}{cccccc}
1 & -2 & 1 & 0 & 0 & \dots \\
0 & 1 & -2 & 1 & 0 & \dots \\
0 & 0 & 1 & -2 & 1 & \dots \\
\dots & \dots & \dots & \dots & \dots & \dots \\
\dots & 0 & 0 & 1 & -2 & 1 \\
\end{array} \right]
\end{equation}

and $\lambda$ is the Levenberg-Marquardt stabilization parameter, which is
computed automatically.  There are two problems with this.

The third row of Equation (\ref{one}), when put back into the form of a
constraint on the new state, \emph{viz.} $\mathbf{W R}\, \mathbf{x}_{n+1}
\simeq 0$, says, in English, ``The second difference of the current state
ought to be zero, in the least-squares sense, with different intensity of
belief at different places.''

This should probably be ``The second difference of the new state ought to
be like the second difference of the apriori, in the least-squares sense,
with different intensity of belief at different places,'' i.e., $\mathbf{W
R}\, \mathbf{x}_{n+1} \simeq \mathbf{W R}\, \mathbf{x}_a$, or when put
into the form of a least-squares condition on the Newton move (the third
row in Equation (\ref{one})), $\mathbf{W R}\, \delta \mathbf{x} \simeq
\mathbf{W R}\, ( \mathbf{x}_a - \mathbf{x}_n )$.

The second problem is that we use the second difference as a proxy for the
second derivative.  A second difference is a proxy for a second derivative
if the step size is constant.  For us, the horizontal step size (the
solution profile spacing specified in the {\tt hGrid}) is constant, but
the vertical step size (the pressure levels specified by {\tt vGrid}) is
not constant.  We ought, therefore, to be using a second divided
difference.  Using $p$ to denote either $\phi$ or $\zeta$, and $i$ to
denote either a horizontal or vertical index, the three nonzeros in a row
of the Tikhonov regularization operator ought to be

\begin{equation}
\begin{array}{ccc}
\frac1{(p_{i+1}-p_i)(p_{i+1}-p_{i-1})} & 
-\frac1{(p_{i+1}-p_i)(p_i-p_{i-1})} &
\frac1{(p_{i+1}-p_{i-1})(p_i-p_{i-1})}
\end{array}
\end{equation}

In the case of constant spacing, this reduces to our current value, with
appropriate scaling which is absorbed into $\mathbf{W}$.

There is a possibility approximately to accomodate a change from a second
difference to a second divided difference automatically, or at least in a
simple systematic but perhaps manual way.  We could, for example, multiply
the appropriate row of the current $\mathbf{W}$ by $\frac12
(p_{i+1}-p_{i-1})^2$, which reduces to the current state where the step
size is fixed, and not too much different from the current state where the
step size does not change dramatically.

It is not likely possible automatically to accomodate a change from
desiring the second derivative to be zero to desiring the second
derivative to be like the second derivative of apriori. This would require
a new series of experiments to compute a new $\mathbf{W}$, which was a lot
of work for Alyn in the first place.

One advantage, especially of accomodating varying vertical step size, is
that we might reduce undesirable ``glitches'' in the vertical pattern of
solutions in the areas where vertical step size changes.

Ideally, we ought to reduce the weight for Tikhonov regularization at
altitudes where we know that dramatic changes occur as a function of
altitude, for example at the tropopause.  Accomodating the second
derivative of the current state to the second derivative of the apriori
has somewhat this effect, to the extent that the dramatic changes in the
current state occur where the dramatic changes occur in the apriori.  We
almost certainly already approximate this, using an assumption that the
actual position is not terribly different from some assumed average
position.  An iterative technique, involving solutions for temperature
(essentially solving for the vertical position of the tropopause and
stratopause on each profile), might be used to provide additional scaling
for $\mathbf{W}$ (or simply set rows to zero) for those species for which
dramatic variation in the vertical direction is known to occur where there
is dramatic variation of temperature in the vertical direction.

It is possible that accomodating the second derivative of the current
state to the second derivative of the apriori, perhaps weighted by the
apriori standard deviation (i.e., a weaker constraint where we have less
confidence in our knowledge of apriori), would eliminate the need for
$\mathbf{W}$, up to a scalar factor, which might be determined as
described by Mark Lucas in \emph{Strong robust generalized
cross-validation for choosing the regularization parameter}, which
appeared in {\bf Inverse Problems 24} (2008), or the simpler method
described by Tilman Steck in \emph{Methods for determining regularization
for atmospheric retrieval problems}, which appeared in {\bf Applied Optics
41}, 9 (20 March 2002).

\label{lastpage}
\end{document}

% $Id$
