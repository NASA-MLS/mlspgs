% To make the graphics
% tgif -print -epsi wvs-128-QTM-1.obj; epstopdf wvs-128-QTM-1.eps
% tgif -print -jpeg wvs-128-QTM-1.obj
% tgif -print -epsi wvs-128-QTM-2.obj; epstopdf wvs-128-QTM-2.eps
% tgif -print -jpeg wvs-128-QTM-2.obj
% tgif -print -epsi wvs-128-QTM-3.obj; epstopdf wvs-128-QTM-3.eps
% tgif -print -jpeg wvs-128-QTM-3.obj
% tgif -print -epsi Barycentric.obj; epstopdf Barycentric.eps
% tgif -print -jpeg Barycentric.obj

\makeatletter\let\ifGm@compatii\relax\makeatother
\documentclass[landscape]{beamer}
\usepackage[nofancy,notoday]{rcsinfo}
\usepackage{color}
\usepackage{amsmath}
\usepackage{moreverb}
\usepackage{multicol}
\usepackage{graphicx}
\usepackage{floatflt}
%
\renewcommand{\b}{\mathbf}
\renewcommand{\d}{\text{d}}
\newcommand{\T}{^{T}}
\newcommand{\cs}[1]{$_{\text{#1}}$}
\newcommand{\inv}{^{\mathrm -1}}
\newcommand{\cp}[1]{$^{\text{#1}}$}
\newcommand{\degsym}{\ensuremath{^\circ}}
\newcommand{\newframe}[2][]{\begin{frame}\frametitle{\hfill #2 \hfil}}

\parskip 2pt
%
% -----------------------------------------------------------------------------
%
\title{Summary of QTM}
\subtitle{wvs-155}
\author{Van Snyder}
\date{24 September 2019}
\titlegraphic{\includegraphics[width=1.0in]{eos_mls_logo_onpink}}

\begin{document}
\sloppy
%
% -----------------------------------------------------------------------------
%
\begin{frame}
 \titlepage
\end{frame}
%
% -----------------------------------------------------------------------------
%
\newframe{Geoffrey Dutton}

In {\bf A Hierarchical Coordinate System for Geoprocessing and
Cartography}, Geoffrey Dutton described the {\bf Quaternary Triangular
Mesh}, or {\bf QTM}.

\begin{itemize}

\item The ratio of the area of the smallest facet to the area of the
      largest facet is bounded by about 6/11.
      
\item In a lat-lon grid, this ratio goes to zero at the poles.

\item Start with an inscribed octahedron, projecting triangular facets to
      octants on the Earth surface.

\item Where a facet is too big, divide it into four smaller ones by
      connecting the midpoints of the edges.

\end{itemize}

\end{frame}
%
% -----------------------------------------------------------------------------
%
\newframe{Geoffrey Dutton}

\begin{centering}
\includegraphics[scale=0.5]{wvs-128-QTM-1}\\
From Dutton's book, showing uniform refinement to the
fourth level \\
\end{centering}

\end{frame}
%
% -----------------------------------------------------------------------------
%
\newframe{Construction}

\begin{itemize}

\item By construction, number of facets at level $\ell$, where $\ell = 1$ is the initial
      octahedron is
%
      \begin{equation*}
      f_\ell = 8 \times 4^{\ell-1} = 2^{2 \ell + 1}
      \end{equation*}

\item Smaller facets are formed by bisecting edges, so the number of
      vertices $v_\ell = v_{\ell-1} + e_{\ell-1}$.

\item From Euler's polyhedron formula, $e_\ell = f_\ell + v_\ell - 2$.

\item Solving these equations,
%
      \begin{equation*}\begin{split}
      v_\ell =\,& 2^{2\ell} + 2 \\
      e_\ell =\,& 3 \times 2^{2 \ell}
      \end{split}\end{equation*} 

\end{itemize}

\end{frame}
%
% -----------------------------------------------------------------------------
%
\newframe{Construction}

\begin{itemize}

\item Counting the poles and equator, vertices are on $2^{\ell+1}$
      equally-spaced parallels of latitude.\\[10pt]

\item Along the $k^\text{th}$ parallel of latitude from each pole,
      vertices are on $4k$ equally-spaced longitudes.

\end{itemize}

\end{frame}
%
% -----------------------------------------------------------------------------
%
\newframe{Facet identification}

Facet identifier, called \emph{QTM ID} or \emph{QID} consists of

\begin{itemize}

\itemsep 10pt

\item Octant number, followed by

\item a sequence of two-bit numbers that describe which sub-facet of a
      larger facet contains the smaller facet.

\item Two-bit numbers can be considered to be base-4 digits.

\item Too-large facets are divided into four smaller facets.

\item Hence: \emph{Quaternary
      Triangular Mesh}.

\end{itemize}

\end{frame}
%
% -----------------------------------------------------------------------------
%
\newframe{Resolution}

\begin{itemize}

\itemsep 10pt

\item Octant number is in the range $8 \cdots 15$, allowing to determine
      unambiguously the number of digits in the QID, and thereby the facet
      resolution.

\item Four-bit octant number and two-bit sub-facet number allows QIDs for
      thirteen levels of refinement to fit in a 31-bit integer.

\item At the $13^\text{th}$ level of refinement, height of a facet
      (north-south extent) is about 1.2 km.

\item Using a 63-bit integer would allow 29 levels of refinement.

\item At the $29^\text{th}$ level of refinement, height of a facet is
      about 4 cm.

\end{itemize}

\end{frame}
%
% -----------------------------------------------------------------------------
%
\newframe{ZOT projection and facet numbering}

Dutton developed a simple way, called the \emph{Zenithial OrthoTriangular}
projection, or ZOT projection, to identify facets within a QTM, and to
compute the facets within which a position is contained.

\begin{centering}
\includegraphics[scale=0.4,angle=270]{wvs-128-QTM-2}\\[5pt]
ZOT projections of refinements to levels 2 and 3, from Dutton's
book\\
\end{centering}

\end{frame}
%
% -----------------------------------------------------------------------------
%
\newframe{ZOT projection and facet numbering}

\begin{centering}
\includegraphics[scale=0.4,angle=270]{wvs-128-QTM-2}\\
\end{centering}

\begin{itemize}

\item Center is the north pole. Corners are the south pole.

\item Heavy lines are edges of the octahedron.

\item Outer boundary consists of southern-hemisphere meridians.

\item Halves of each outer boundary are aliased.

\end{itemize}

\end{frame}
%
% -----------------------------------------------------------------------------
%
\newframe{ZOT projection and facet numbering}

\begin{centering}
\includegraphics[scale=0.4,angle=270]{wvs-128-QTM-2}\\
\end{centering}

\begin{itemize}

\item Diagonal lines are parallels of latitide.

\item Diagonal heavy lines are the equator.

\item Meridians are straight lines from the poles to the equator.

\item Narrow lines are boundaries of level-2 refinement. Dashed lines are
      boundaries of level-3 refinement.

\end{itemize}

\end{frame}
%
% -----------------------------------------------------------------------------
%
\newframe{ZOT projection and facet numbering}

\begin{centering}
\includegraphics[scale=0.4,angle=270]{wvs-128-QTM-2}\\
\end{centering}

\begin{itemize}

\item Vertices are assigned \emph{basis numbers} in $1 \cdots 3$.

\item To start, poles are 1, $0^\circ$ and $180^\circ$ equatorial
      vertices are 2, $90^\circ$ and $270^\circ$ are 3.

\end{itemize}

\end{frame}
%
% -----------------------------------------------------------------------------
%
\newframe{ZOT projection and facet numbering -- subdivision}

\begin{centering}
\includegraphics[scale=0.35,angle=270]{wvs-128-QTM-2}\\
\end{centering}

\begin{itemize}

\item The basis number of the midpoint is $B_n = 6 - (B_a + B_b)$ where
      $B_a$ and $B_b$ are basis numbers of its end points.

\item QID of each initial facet is its octant number.

\item QID of the central facet is the QID of the parent facet, with
      zero appended.

\item QID of each other facet is the QID of the parent facet with the
      basis number of the parent vertex appended.

\end{itemize}

\end{frame}
%
% -----------------------------------------------------------------------------
%
\newframe{ZOT projection}

\begin{centering}
\includegraphics[scale=0.4,angle=270]{wvs-128-QTM-2}\\
\end{centering}

\begin{itemize}

\item Each facet is an isosceles right triangle.

\item Right angle is the one nearest to the pole; the vertex is called the
      \emph{pole node}.

\item One edge incident on the pole node is vertical; the other is
      horizontal.

\end{itemize}

\end{frame}
%
% -----------------------------------------------------------------------------
%
\newframe{ZOT co\"ordinates}

\begin{itemize}

\item ZOT coordinates are in the range $-1 \cdots 1$.

\item North pole is at $(0,0)$.

\item South pole is at $(\pm -1, \pm -1)$.

\item Mapping from longitude and latitude to ZOT coordinates uses the
      $\ell_1$ metric and is computed by the function {\tt Geo\_to\_ZOT}
      in the module {\tt QTM\_m}.

\end{itemize}

\end{frame}
%
% -----------------------------------------------------------------------------
%
\newframe{ZOT co\"ordinates}

Given longitude $\lambda$ and latitude $\theta$, compute initial ZOT
coordinates using
%
\begin{equation*}\begin{split}
\delta x =\,& 1 - \frac{| \theta |}{ 90 }\\
x = \,& \delta x \,
    \frac{ \lfloor \phi \rfloor \!\!\!\!\mod 90 + \phi - \lfloor \phi \rfloor }
    {90} \\
y = \,& \delta x - x \\
\end{split}\end{equation*}
%
\vspace*{-15pt}
%
\begin{itemize}

\item Then in the southern hemisphere, replace $x = 1 - y$ and $y = 1 - x$.

\item Then in even-numbered octants exchange $x$ and $y$.

\item Then in the top half of ZOT space, negate $y$, and in the left half,
      negate $x$.

\end{itemize}

No square roots or trigonometric functions are necessary.

(Maybe in {\tt Geo\_to\_ZOT} this should be done by an eight-way SELECT
CASE construct using the octant number.)

\end{frame}
%
% -----------------------------------------------------------------------------
%
\newframe{Computing QID for a facet}

To compute the QID of a facet containing a $(\lambda,\theta)$ point:

\begin{itemize}

\item Compute its $x$ and $y$ ZOT co\"ordinates.

\item Use its octant number as the initial QID.

\item While the level of identified facet does not have sufficient
      resolution, determine a sub-facet QID:

  \begin{itemize}
    \itemsep 4pt

    \item Compute the distances $\d x$ and $\d y$ in ZOT co\"ordinates
          from the pole node.

    \item If $|\d x| + |\d y| < s/2$, where $s$ is the length in ZOT
          co\"ordinates of a horizontal or vertical edge of the parent
          facet, append the basis number of the pole node to the QID of
          the parent facet,

    \item Otherwise if $|\d x| > s/2$ or $|\d y| > s/2$, append the basis
          number of the $x$ or $y$ node, respectively,

    \item Otherwise append zero.

  \end{itemize}

\end{itemize}

If $|\d x| > s/2$ and $|\d y| > s/2$, you made a mistake along the way and
you're looking in the wrong facet.

\end{frame}
%
% -----------------------------------------------------------------------------
%
\newframe{Computing QID for a sub-facet}

\begin{centering}
\includegraphics{wvs-128-QTM-3}\\
Calculation of sub-facet and QID, from Dutton's book\\
\end{centering}

\end{frame}
%
% -----------------------------------------------------------------------------
%
\newframe{QTM in MLS}

\begin{itemize}

\item Assume polar circumference is 40,000 km (the original definition of
      the meter).

\item At level $\ell$ of refinement, height of a facet, i.e., distance
      from its polar node to the opposite edge, is 20,000/$2^\ell$ km.

\item Same distance for every facet at the same level of refinement
      because diagonal edges are equally-spaced parallels of latitude.

\item With $\ell = 7$ the distance is 156.25 km, or $\approx 1.4^\circ$
      latitude.

\item QTM over entire Earth will not be used:

  \begin{itemize}

  \item $v_7 = 2^{14} + 2 = 16,386$ profiles.

  \item Solving for 20 quantities on 72 levels, Jacobian would have
        $16,386 \times 20 \times 72 = 23,585,840$ columns.

  \item QTM will be constructed within a specified polygon.

  \end{itemize}

\end{itemize}

\end{frame}
%
% -----------------------------------------------------------------------------
%
\newframe{Identifying vertices}

\begin{itemize}

\item Each facet is bounded by three vertices.

\item Each vertex that is not a vertex of the octahedron is a member of
      six facets.

\item Each vertex that is a vertex of the octahedron is a member of
      four facets.

\item Every vertex needs an unique sequential index number, which is used
      to calculate a column subscript in the Jacobian, and a subscript in
      the state vector.

\end{itemize}

\end{frame}
%
% -----------------------------------------------------------------------------
%
\newframe{Constructing a QTM within a polygon}

On the surface of the Earth, ``inside'' a polygon is ambiguous.

\begin{enumerate}

\item Specify a point that is within the polygon.

\item If a polygon has an edge that crosses a southern-hemisphere meridian
      of the original octahedron, add edges in the ZOT projection of the
      polygon along the crossed meridian to join the aliased points.

\item Construct the QTM top down, starting with level 1.

\item If a facet has a vertex within the polygon, i.e., a line from that
      vertex to the ``inside'' vertex of the polygon crosses an even
      number (including zero) of the edges of the polygon,\\[4pt]

      or the polygon has a vertex in the facet,\\[4pt]
      
      or an edge of the polygon intersects an edge of the facet,\\[4pt]

      and the level of refinement of that facet is not sufficient, refine
      that facet.

\end{enumerate}

\end{frame}
%
% -----------------------------------------------------------------------------
%
\newframe{Constructing a QTM within a polygon}

\begin{enumerate}
\setcounter{enumi}{4}

\item When a facet is constructed, add its vertices that are within the
      polygon to the set of vertices that are geolocations for the state
      vectors.\\[5pt]

      If the facet is at the specified refinement level, and it has any
      vertices within the polygon, add all its vertices to the set, so
      that it is possible to interpolate within it, even if some of its
      vertices are outside the polygon.\\[5pt]

      A hash table lookup, based upon the QID and basis number of each
      vertex, is used to avoid duplicates.\\[5pt]

      Each added vertex gets an unique serial number that is used to
      calculate a column subscript of the Jacobian, and a subscript of the
      state vector.\\[5pt]

\end{enumerate}

\end{frame}
%
% -----------------------------------------------------------------------------
%
\newframe{Constructing a QTM within a polygon}

\begin{enumerate}
\setcounter{enumi}{5}

\item As the QTM is constructed, construct a quadtree in which each vertex
      represents a facet.\\[8pt]
      
      A vertex of the quadtree contains the serial numbers of QTM vertices
      of the facet that it represents, that are at the finest level of
      refinement and inside the polygon, or zero for a vertex that is
      outside the polygon or not at the finest level of refinement.\\[8pt]

      Internal vertices of the quadtree represent facets that are not at
      the finest level of refinement, and therefore contain zeroes instead
      of QTM vertex serial numbers.

\end{enumerate}

\end{frame}
%
% -----------------------------------------------------------------------------
%
\newframe{Using a QTM}

When a value is needed at a $(\lambda,\theta)$ point

\begin{enumerate}

\item Determine its ZOT coordinates.

\item Determine the QID of the facet at the desired level of
refinement.

\item Use the digits of the QID to traverse the quadtree.\\[4pt]

      Cost is proportional to the level of refinement, not the size of the
      QTM.

\item If the facet containing the specified point has serial numbers for
      all its vertices, and the specified point is within the polygon,
      interpolate within that facet.

\end{enumerate}

\end{frame}
%
% -----------------------------------------------------------------------------
%
\newframe{Using a QTM}

\begin{enumerate}
\setcounter{enumi}{4}

\item If the facet containing the specified point does not have serial
      numbers for all its vertices, use the value on the polygon boundary
      that is nearest to that point. This is constant extrapolation.

  \begin{enumerate}

    \item Find the polygon vertex that is nearest to the specified point.

    \item Construct lines from the specified point that are perpendicular
          to the two edges incident on that polygon vertex.

    \item If either of those lines intersects an incident edge between the
          nearest vertex and an adjacent one, choose the intersection that is
          closer to the specified point (there might be two if the polygon
          is not convex).

    \item Otherwise choose the vertex nearest to the specified point.

    \item Interpolate within that facet to the chosen point.

  \end{enumerate}

\end{enumerate}

\end{frame}
%
% -----------------------------------------------------------------------------
%
\newframe{Using a QTM}

\parskip 8pt

ZOT polygons are not comformal to spherical polygons.

A point that is within (outwith) a ZOT polygon might be outwith (within) a
spherical polygon.

Assuming the specified polygon is defined by $(\lambda,\theta)$
co\"ordinates for its vertices, and its edges are great circles, rather
than worry about this, choose a slightly larger polygon that encloses the
interesting points (and maybe a few others).

\end{frame}
%
% -----------------------------------------------------------------------------
%
\newframe{Barycentric co\"ordinates}

\parskip 6pt

The unnormalized barycentric coordinates $\hat\lambda_{s_k}$ of a point
$(x,y)$ within a triangle having vertices $(x_{s_i},y_{s_i})$, where $i
\in \{1,2,3\}$ and $s_i$ is the serial number of vertex $i$ of the QTM
facet, are

\begin{equation*}
\hat\lambda_{s_k} = (y_{s_j}-y_{s_i})(x-x_{s_i}) -
                    (x_{s_j}-x_{s_i})(y-y_{s_i})
\end{equation*}

where $i$, $j$, $k$ are distinct and in $\{1,2,3\}$.

The signs depend upon the ordering of the vertices.

Normalized co\"ordinates $\lambda_{s_k}$ are obtained by dividing the
unnormalized co\"ordinates by their sum. Therefore, the ordering doesn't
matter.

If any normalized co\"ordinate is negative, the point $(x,y)$ is outside
the triangle.

\end{frame}
%
% -----------------------------------------------------------------------------
%
\newframe{Barycentric interpolation}

Assuming $(x,y)$ is within the facet, the value $f(x,y)$ interpolated from
$\{f(x_{s_k},y_{s_k})\}$ is

\begin{equation*}
f(x,y) = \sum_{k=1}^3 \lambda_{s_k} f(x_{s_k},y_{s_k})\,.
\end{equation*}

The system was introduced in 1827 by August Ferdinand M{\"o}bius.

Barycentric co\"ordinates can be computed using either longitude and
latitude, or ZOT co\"ordinates, which would result in slightly different
values for $\lambda_{s_k}$ and $f(x,y)$.

\end{frame}
%
% -----------------------------------------------------------------------------
%
\newframe{Barycentric interpolation}

The normalized barycentric interpolation coefficient $\lambda_a$ used to
interpolate from the point $a$ to the point $(x,y)$ is the same as the
ratio of the area of triangle $A$ to the area of triangle $abc$, and
similarly for the other vertices of the triangle.

\begin{centering}
\includegraphics[scale=0.6]{Barycentric}\\
\end{centering}

\end{frame}
%
% -----------------------------------------------------------------------------
%
\newframe{Three dimensions}

The state vector uses a stacked and coherent basis: The same QTM at every
level, and the same set of levels on every line perpendicular to a surface
QTM vertex.

To interpolate to a point $(x,y,z)$, compute the barycentric coordinates
$\lambda_{s_k}$ of $(x,y)$. Use linear interpolation to compute an
interpolation coefficient $\mu_h$ from $(z_h,z_{h+1})$ to $z$, assuming
$z_h \leq z \leq z_{h+1}$, and $\mu_{h+1} = 1 - \mu_h$.

This results in six interpolation weights $\eta_{s_k}j$ where $k
\in \{1,2,3\}$ and $j \in \{h,h+1\}$.  Then

\begin{equation*}
f(x,y,z) = \sum_{k=1}^3 \sum_{j=h}^{h+1} \lambda_{s_k}\mu_j
                                         f(x_{s_k},y_{s_k},z_j)
         = \sum_{k=1}^3 \sum_{j=h}^{h+1} \eta_{s_{k}j}
                                         f(x_{s_k},y_{s_k},z_j)
\end{equation*}

\end{frame}
%
% -----------------------------------------------------------------------------
%
\newframe{Three dimensions}

\dots\ and the derivative is

\begin{equation*}
\frac{\partial f(x,y,z)}{\partial f(x_{s_k},y_{s_k},z_j)} = \eta_{s_{k}j}
\,.
\end{equation*}

\end{frame}
%
% ----------------------------------------------------------------------------
%
\end{document}

% $Log$
