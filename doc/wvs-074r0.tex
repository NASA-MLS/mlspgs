\documentclass[11pt]{article}
\usepackage[fleqn]{amsmath}\textwidth 6.5in
\oddsidemargin -0.25in
%\evensidemargin -0.5in
\topmargin -0.25in
\textheight 9.0in

\newcommand{\docname}{\bf wvs-074}
\newcommand{\docdate}{27 March 2008}

\usepackage{graphicx}

\usepackage{floatflt}

\begin{document}

%\tracingcommands=1
\newlength{\hW} % heading box width
\newlength{\pW} % page number field width
\settowidth{\hW}{\docname}
\settowidth{\pW}{Page \pageref{lastpage}\ of \pageref{lastpage}}
\ifdim \pW > \hW \setlength{\hW}{\pW} \fi
\makeatletter
\def\@biblabel#1{#1.}
\newcommand{\ps@twolines}{%
  \renewcommand{\@oddhead}{%
    \docdate\hfill\parbox[t]{\hW}{{\hfill\docname}\newline
                          Page \thepage\ of \pageref{lastpage}}}%
\renewcommand{\@evenhead}{}%
\renewcommand{\@oddfoot}{}%
\renewcommand{\@evenfoot}{}%
}%
\makeatother
\pagestyle{twolines}

\vspace{-10pt}
\begin{tabbing}
\phantom{References: }\= \\
To: \>Bill\\
Subject: \>Tangent heights and angles for scattering calculations\\
From: \>Van Snyder\\
\end{tabbing}

\parindent 0pt \parskip 10pt
\vspace{-20pt}

{\includegraphics[width=474pt,height=563pt,clip]{./wvs-074-scatter.eps}}

\newpage

Assume a set of points are given, each one specified by its height and
angle from a reference line.

We wish to calculate the tangent heights and angles to produce radiative
transfer through each of the points P at specified coordinates
$(H,\phi)$, where $H$ is height measured from the center of the
equivalent circular earth and $\phi$ is the angle measured anti-clockwise
from the reference (red) line, at specified angles $-\pi \leq \xi \leq 0$,
i.e., below the horizon, where the horizon is defined to be perpendicular
to the reference line.  In the figure, radiation is considered to be
transferred from below the horizon if $\xi \geq \xi_c$ or $\xi \leq
\xi_c-\pi$, and above the horizon followed by reflection from the surface
otherwise.  The other segment of the path, i.e., the part above the
horizon, can be used to calculate radiative transfer from opposite
directions.

Ultimately, we wish to calculate the radiance scattered from the source
to an observer at the right on the horizon.

The critical angle $\xi_c$ is the angle at which a ray reaching
the point P from the source is just tangent to the earth's surface.

\begin{equation}
\xi_c = \phi - \cos^{-1} \frac{R}H
\end{equation}

Case 1: The given angle is $\xi = \xi_2 \geq \xi_c$ or $\xi_2 \leq -\xi_c
- \pi$

The tangent height is $Ht_2$ and the tangent angle is $\xi_2$, where

\begin{equation}
Ht_2 = H | \cos (\phi - \xi_2) |
\end{equation}

Case 2: The given angle is $\xi = \xi_3$ and $-\pi - \xi_c < \xi_3 < \xi_c$

The tangent height is $Ht_3$ and the tangent angle is $\xi_4$, where

\begin{equation}\begin{split}
Ht_3 =\,& H | \cos (\phi - \xi_3) | = R \cos \theta \\
\theta =\,& \cos^{-1} \frac{Ht_3}R =
            \cos^{-1} \left| \frac{\cos(\phi-\xi_3)}{\cos(\phi-\xi_c)}
                     \right| \\
\xi_4 =\,&\left\{ \begin{array}{ll}
  2 \theta + \xi_3 & \xi_3 > \phi-\frac\pi2 \\
  -2\theta - \xi_3 & \text{otherwise} \\
  \end{array} \right.
\end{split}\end{equation}

\label{lastpage}
\end{document}
% $Id$
