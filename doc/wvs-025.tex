\documentclass[11pt]{article}
\usepackage[fleqn]{amsmath}\textwidth 6.25in

\oddsidemargin -0.25in
%\evensidemargin -0.5in
\topmargin -0.5in
\textheight 9.00in

\newcommand{\docname}{\bf wvs-025r1}
\newcommand{\docdate}{31 March 2010}

\begin{document}

%\tracingcommands=1
\newlength{\hW} % heading box width
\newlength{\pW} % page number field width
\settowidth{\hW}{\docname}
\settowidth{\pW}{Page \pageref{lastpage}\ of \pageref{lastpage}}
\ifdim \pW > \hW \setlength{\hW}{\pW} \fi
\makeatletter
\def\@biblabel#1{#1.}
\newcommand{\ps@twolines}{%
  \renewcommand{\@oddhead}{%
    \docdate\hfill\parbox[t]{\hW}{{\docname}\newline
                          Page \thepage\ of \pageref{lastpage}}}%
\renewcommand{\@evenhead}{}%
\renewcommand{\@oddfoot}{}%
\renewcommand{\@evenfoot}{}%
}%
\makeatother
\pagestyle{twolines}

\vspace{-10pt}
\begin{tabbing}
\phantom{References: }\= \\
To: \>Bill, Nathaniel, Mike\\
Subject: \>New frequency averaging scheme for use with PFA\\
From: \>Van Snyder\\
Reference: \>wvs-024\\
\end{tabbing}

\parindent 0pt \parskip 3pt
\vspace{-20pt}

\section{Introduction}\label{intro}

Recall from wvs-024 the equations to combine PFA and line-by-line radiances
for a particular channel $c$

\begin{equation}\label{combined}
I_c \approx \sum_{n=1}^{N_f} \phi_{nc} \Delta \nu_{nc}
               \sum_{i=1}^{N_p} \Delta B_{ic} \tau^s_{in} \tau^w_{ic}
\end{equation}

and to combine PFA and line-by-line derivatives for that channel

\begin{equation}\label{final}
\frac{\partial I_c}{\partial x_k}
\approx \sum_{n=1}^{N_f} \phi_{nc} \Delta \nu_{nc} \sum_{i=1}^{N_p}
 \left(
  \frac{\partial \Delta B_{ic}}{\partial x_k} -
  \Delta B_{ic}
   \sum_{j=1}^i \left( \frac{\partial \delta^s_{jn}}{\partial x_k} +
                      \frac{\partial \delta^w_{jc}}{\partial x_k}
                \right)
  \right) \tau^s_{in} \tau^w_{ic}\,.
\end{equation}

The term $\frac{\partial \delta^s_{jn}}{\partial x_k}$ is a three-dimensional
array that could, in the worst case, require 6 GB of storage per species.  To
reduce this requirement, it is proposed to revise the scheme for frequency
averaging.

\section{Current frequency-averaging scheme}\label{current}

The scheme outlined in section \ref{intro} isn't precisely representative. The
forward model first calculates ``raw'' radiances and derivatives on a
pre-specified frequency grid, the \emph{pointing frequency} grid, with a
separate grid for each pointing:

\begin{equation}\begin{split}\label{raw}
R_n =\,& \sum_{i=1}^{N_p} \Delta B_{in} \tau^s_{in} \tau^w_{ic} \\
\frac{\partial R_n}{\partial x_k} =\,&
\sum_{i=1}^{N_p}
 \left(
  \frac{\partial \Delta B_{in}}{\partial x_k} -
  \Delta B_{ic}
   \sum_{j=1}^i \left( \frac{\partial \delta^s_{jn}}{\partial x_k} +
                      \frac{\partial \delta^w_{jc}}{\partial x_k}
                \right)
  \right) \tau^s_{in} \tau^w_{ic}\,.
\end{split}\end{equation}

Then (in the non-DACS case), for each channel, those radiances and derivatives
are interpolated to the frequency grid upon which the filter function for the
channel is recorded.  Finally, the interpolated radiances and derivatives are
multiplied by the filter function for the corresponding frequency in the filter
grid, and these products are integrated using the Newton 3/8 quadrature rule:

\begin{equation}\begin{split}\label{average}
I_c =\,& \sum_{n=1}^{N_f} w_n \phi_{nc} \Delta \nu_{nc} \hat R_n\\
\frac{\partial I_c}{\partial x_k} =\,&
 \sum_{n=1}^{N_f} w_n \phi_{nc} \Delta \nu_{nc}
  \frac{\partial \hat R_n}{\partial x_k}
\end{split}\end{equation}

where $w_n$ is a weight for the Newton 3/8 rule, the hat indicates quantities
interpolated from the pointing frequency grid to the filter function grid, and
$n$ now ranges over the filter grid instead of the pointing frequency grid.

The computed radiances are interpolated to the filter function grid, rather
than vice-versa, because it has a finer spacing than the pointing frequency
grid.

\section{Incremental frequency-averaging scheme}

The frequency averaging scheme described in section \ref{current} can be
applied incrementally during the frequency loop.

Rather than evaluating and saving the sums over $N_p$ in Equation (\ref{raw})
for every frequency in the pointing frequency grid, and then doing frequency
averaging later, evaluate and save them for some fixed number of frequencies,
say four.  Then interpolate polynomials, say cubics, to them, evaluate the
interpolating polynomials on the filter grid, and accumulate a little bit of
the sums in Equation (\ref{average}).  Then discard the first saved quantity,
evaluate it for the next point in the pointing frequency grid, and repeat.  For
cubic order, this has slightly better accuracy than the current scheme, and
needs far less storage.  The extent of the pointing-frequency dimension for all
quantities that have a pointing frequency dimension, including $\frac{\partial
\delta^s_{jn}}{\partial x_k}$, is reduced to, say, four. Variations on this
scheme are possible by changing the number of saved points on the pointing
frequency grid and therefore the order of interpolating polynomial.

A scheme based upon ideas similar to B-spline basis functions (but without the
continuity constraints) is possible:  The sum over $N_p$ is evaluated for one
point in the pointing frequency grid.  Then its value is assumed to be zero at
some (even) number, say two, of symmetrically-placed adjacent points in that
grid, and polynomials are interpolated through those values.  The interpolating
polynomials are then evaluated on the filter grid, multiplied by the filter
function, and a little bit of the sums in Equation (\ref{average}) is
accumulated.  When this is repeated for every point in the pointing frequency
grid, the effect is as if the sum over $N_p$ were evaluated at every point on
the pointing frequency grid, and the results fitted with piecewise polynomials
of the same order as is used for each point.  This scheme requires only one of
each of the quantities to be saved; the storage saving is even more than in the
previous case.  Variations on this scheme are possible by changing the number
of adjacent points at which the values are assumed to be zero, equivalently,
changing the order of the piecewise interpolating polynomial.  Piecewise linear
is essentially the same as the representation basis used for mixing ratios.

Both of these schemes require obvious and fairly simple special considerations
at the ends of the pointing frequency grid.  Neither one is compatible with the
current FFT-based method for DACS frequency averaging.

\label{lastpage}
\end{document}
% $Id$

% $Log$
