\documentclass[11pt]{article}
\usepackage[fleqn]{amsmath}\textwidth 6.5in
\oddsidemargin -0.25in
%\evensidemargin -0.5in
\topmargin -0.25in
\textheight 9.0in

\newcommand{\docname}{\bf wvs-077}
\newcommand{\docdate}{15 April 2008}

\begin{document}

%\tracingcommands=1
\newlength{\hW} % heading box width
\newlength{\pW} % page number field width
\settowidth{\hW}{\docname}
\settowidth{\pW}{Page \pageref{lastpage}\ of \pageref{lastpage}}
\ifdim \pW > \hW \setlength{\hW}{\pW} \fi
\makeatletter
\def\@biblabel#1{#1.}
\newcommand{\ps@twolines}{%
  \renewcommand{\@oddhead}{%
    \docdate\hfill\parbox[u]{\hW}{{\hfill\docname}\newline
                          Page \thepage\ of \pageref{lastpage}}}%
\renewcommand{\@evenhead}{}%
\renewcommand{\@oddfoot}{}%
\renewcommand{\@evenfoot}{}%
}%
\makeatother
\pagestyle{twolines}

\vspace{-10pt}
\begin{tabbing}
\phantom{References: }\= \\
To: \>Bill\\
Subject: \>Antenna convolution as a Stieltjes integral\\
From: \>Van Snyder\\
\end{tabbing}

\parindent 0pt \parskip 10pt
\vspace{-20pt}

In the antenna convolution
%
\begin{equation}\label{one}
\overline{I}(\chi_0) = \frac{\int_{-\frac\pi2}^\frac\pi2 I(\chi) G(\chi_0-\chi) \, d\chi}
                            {\int_{-\frac\pi2}^\frac\pi2 G(\chi_0-\chi) \, d\chi}
                       \,,
\end{equation}
%
where $I$ and $\chi$ are functions of $T$, the integrals can be expressed in
terms of Stieltjes integrals.  Lumping $I(\chi) G(\chi_0-\chi)$ into a single
function $f(\chi;T)$ we have for just the numerator
%
\begin{equation}\label{two}
\overline{f}(T) = \int_{a}^{b} f(\chi(u;T);T) \, d \chi(u;T) =
 \int_{\chi^{-1}(a;T)}^{\chi^{-1}(b;T)} f(\chi(u;T);T)
  \, \frac{\partial \chi(u;T)}{\partial u} \, d u\,,
\end{equation}
%
the latter being equivalent to the Stieltjes integral in the middle if $\chi$
is continuous and monotone.  Calculating the derivative with respect to $T$
gives
%
\begin{equation}\begin{split}\label{three}
\frac{\partial \overline{f}(T)}{\partial T}
=\,&
\int_{\chi^{-1}(a;T)}^{\chi^{-1}(b;T)}
 \left[\left(
  \frac{\partial f(\chi(u;T);T)}{\partial \chi(u;T)}
  \frac{\partial \chi(u;T)}{\partial T} +
  \frac{\partial f(\chi(u;T);T)}{\partial T} \right)
  \frac{\partial \chi(u;T)}{\partial u} \right. \\
%\,&\\
  &\left. \phantom{\int_{\chi^{-1}(a;T)}^{\chi^{-1}(b;T)} \left[\right.}
  + f(\chi(u;T);T) \frac{\partial^2 \chi(u;T)}{\partial u \, \partial T}
 \right] \, d u\\
%\,&\\
\,&+
 \frac{\partial \chi^{-1}(b;T)}{\partial T} f(\chi(\chi^{-1}(b,T);T))
  \left.\frac{\partial \chi(u;T)}{\partial u}\right|_{u=\chi^{-1}(b;T)} \\
%\,&\\
\,&-
 \frac{\partial \chi^{-1}(a;T)}{\partial T} f(\chi(\chi^{-1}(a,T);T))
  \left.\frac{\partial \chi(u;T)}{\partial u}\right|_{u=\chi^{-1}(a;T)} \\
%\,&\\
=\,&
\int_{\chi^{-1}(a;T)}^{\chi^{-1}(b;T)}
 \left[\left(
  \frac{\partial f(\chi(u;T);T)}{\partial \chi(u;T)}
  \frac{\partial \chi(u;T)}{\partial T} +
  \frac{\partial f(\chi(u;T);T)}{\partial T} \right)
  \frac{\partial \chi(u;T)}{\partial u} \right. \\
%\,&\\
  &\left. \phantom{\int_{\chi^{-1}(a;T)}^{\chi^{-1}(b;T)} \left[\right.}
  + f(\chi(u;T);T) \frac{\partial^2 \chi(u;T)}{\partial u \, \partial T}
 \right] \, d u\\
%\,&\\
\,&+
 \frac{\partial \chi^{-1}(b;T)}{\partial T} f(b;T))
  \left.\frac{\partial \chi(u;T)}{\partial u}\right|_{u=\chi^{-1}(b;T)} -
 \frac{\partial \chi^{-1}(a;T)}{\partial T} f(a;T))
  \left.\frac{\partial \chi(u;T)}{\partial u}\right|_{u=\chi^{-1}(a;T)} \,. \\
\end{split}\end{equation}
%
In the case of antenna convolution, we have $u = \mathcal{N}_t H_t / H_s$,
$\chi = \chi(u) = \sin^{-1} u$, $\frac{\partial \chi}{\partial u} =
\left(1-u^2\right)^{-1/2}$, $\chi^{-1}(u) = u$, $a = -\frac\pi2$, $b =
\frac\pi2$, $\chi^{-1}(a) = -1$, and $\chi^{-1}(b) = 1$.  Substituting into
Equation (\ref{two}) gives
%
\begin{equation}
\overline{f}(T) = \int_{a}^{b} f(\chi(u;T);T) \, d \chi(u;T) =
 \int_{\chi^{-1}(a;T)}^{\chi^{-1}(b;T)} f(\chi(u;T);T)
  \, \frac{d u}{\sqrt{1-u^2}}\,.
\end{equation}
%
If the limits $a$ and $b$ really are $-\pi/2$ and $\pi/2$, respectively, we have
%
\begin{equation}
\overline{f}(T) = \int_{-\pi/2}^{\pi/2} f(\chi(u;T);T) \, d \chi(u;T) =
 \int_{-1}^{1} f(\chi(u;T);T) \, \frac{d u}{\sqrt{1-u^2}}\,,
\end{equation}
%
giving an integrable singularity in the integrand at the limits.

Substituting into Equation (\ref{three}) gives
%
\begin{equation}\begin{split}\label{six}
\frac{\partial \overline{f}(T)}{\partial T}
=\,&
\int_{\chi^{-1}(a;T)}^{\chi^{-1}(b;T)}
 \left[\left(
  \frac{\partial f(\chi(u;T);T)}{\partial \chi(u;T)}
  \frac1{\sqrt{1-u^2}} \frac{\partial u}{\partial T} +
  \frac{\partial f(\chi(u;T);T)}{\partial T} \right)
  \frac1{\sqrt{1-u^2}} \right. \\
  &\phantom{\int_{\chi^{-1}(a;T)}^{\chi^{-1}(b;T)}\left[\right.} \left.
  + f(\chi(u;T);T) \frac1{(1-u^2)^{3/2}} \frac{\partial u}{\partial T}
 \right] \, d u\\
%\,&\\
\,&+
 \frac{\partial u}{\partial T} f(b;T))
  \left.\frac1{\sqrt{1-u^2}}\right|_{u=\chi^{-1}(b;T)} -
 \frac{\partial u}{\partial T} f(a;T))
  \left.\frac1{\sqrt{1-u^2}}\right|_{u=\chi^{-1}(a;T)} \\
%\,&\\
\,&=
\int_{\chi^{-1}(a;T)}^{\chi^{-1}(b;T)}
 \left[\left(
  \frac{\partial f(\chi(u;T);T)}{\partial \chi(u;T)} +
  \frac{f(\chi(u;T);T)}{\sqrt{1-u^2}} \right)
  \frac1{\sqrt{1-u^2}} \frac{\partial u}{\partial T} +
  \frac{\partial f(\chi(u;T);T)}{\partial T} \right]
  \frac{d u}{\sqrt{1-u^2}}\\
%\,&\\
\,&+
 \frac{\partial u}{\partial T} f(b;T))
  \left.\frac1{\sqrt{1-u^2}}\right|_{u=\chi^{-1}(b;T)} -
 \frac{\partial u}{\partial T} f(a;T))
  \left.\frac1{\sqrt{1-u^2}}\right|_{u=\chi^{-1}(a;T)} \,. \\
\end{split}\end{equation}
%
The situation here is worse than in Equation (\ref{two}) if $a=-\pi/2$ or
$b=\pi/2$, because the integrand is not integrable over the entire range, and
infinities occur in the last two integrated terms, which arise from derivatives
of the limits.

Similar treatment applies to the denominator of Equation (\ref{one}).

I don't think the antenna convolution code in the forward model includes the
last two terms in Equation (\ref{six}).

Similar considerations apply to the derivative with respect to H$_2$O mixing
ratio because $\mathcal{N}_t$ depends on it, but derivatives with respect to
other parameters do not include the last two integrated terms since the bounds
do not depend on any parameters other then $T$ and the H$_2$O mixing ratio.

\label{lastpage}
\end{document}
% $Id$
