\documentclass[11pt]{article}
\usepackage{alltt}
\usepackage[fleqn]{amsmath}
\usepackage{longtable}
\usepackage[strings]{underscore}

\textwidth 6.5in
\oddsidemargin -0.25in
%\evensidemargin -0.5in
\topmargin -0.5in
\textheight 9in

\newcommand{\docname}{wvs-161r1}
\newcommand{\docdate}{12 June 2020}

\ifx\pdfoutput\undefined
  \pdfoutput=0
\fi
\ifnum\pdfoutput>0
  \usepackage[pdftex,plainpages,hyperindex=true,pdfpagelabels]{hyperref}
  \hypersetup{%
    hypertexnames=false,%
    colorlinks=true,%
    linktocpage=true,%
  }
  % Specify the driver for the color package
  \ExecuteOptions{pdftex}
\else
  \usepackage[hypertex,plainpages,hyperindex=true]{hyperref}
  \hypersetup{%
    hypertexnames=false%
  }
  % Specify the driver for the color package
  \ExecuteOptions{dvips}
  %\ExecuteOptions{xdvi}
\fi

\hyperbaseurl{}
\newcommand\hr[1]{\href{#1.dvi}{dvi}, \href{#1.pdf}{pdf}}
\newcommand\h[1]{#1 (\hr{#1})}

\begin{document}

%\tracingcommands=1
\newlength{\hW} % heading box width
\newlength{\pW} % page number field width
\settowidth{\hW}{\bf\docname}
\settowidth{\pW}{Page \pageref{lastpage}\ of \pageref{lastpage}}
\ifdim \pW > \hW \setlength{\hW}{\pW} \fi
\makeatletter
\def\@biblabel#1{#1.}
\newcommand{\ps@twolines}{%
  \renewcommand{\@oddhead}{%
    \docdate\hfill\parbox[t]{\hW}{{\hfill\bf\docname}\newline
                          Page \thepage\ of \pageref{lastpage}}}%
\renewcommand{\@evenhead}{}%
\renewcommand{\@oddfoot}{}%
\renewcommand{\@evenfoot}{}%
}%
\makeatother
\pagestyle{twolines}

\vspace{-10pt}
\begin{tabbing}
\phantom{References: }\= \\
To: \>Van\\
Subject: \>Area of spherical triangle using vertex co\"ordinates\\
From: \>Van Snyder\\
%Reference: \> foo \\
\end{tabbing}

\parindent 0pt \parskip 6pt
\vspace{-20pt}

Why are we interested in the areas of spherical triangles?

Within a planar triangle whose vertices are
$(\mathbf{v}_1,\mathbf{v}_2,\mathbf{v}_3)$, the barycentric interpolation
coefficient $\eta_i$ from vertex $\mathbf{v}_i$ to a point $\mathbf{X}$ is
the ratio of the area of the triangle
$(\mathbf{v}_j,\mathbf{v}_k,\mathbf{X})$ to the area of the triangle
$(\mathbf{v}_1,\mathbf{v}_2,\mathbf{v}_3) \,\, (i \neq j \neq k)$.

There are (at least) three ways to define barycentric interpolation on a
sphere, using planar barycentric interpolation:

\begin{itemize}

\item Project $\mathbf{X}$ to the planar triangle whose vertices are
      $(\mathbf{v}_1,\mathbf{v}_2,\mathbf{v}_3)$ and use barycentric
      interpolation within that triangle to the projection of
      $\mathbf{X}$ to that plane.

\item Construct a plane tangent to the sphere and parallel to the plane
      defined by $(\mathbf{v}_1,\mathbf{v}_2,\mathbf{v}_3)$.  Project
      $\mathbf{X}$ and $(\mathbf{v}_1,\mathbf{v}_2,\mathbf{v}_3)$ onto
      that plane, and use barycentric interpolation within the projected
      triangle to the projected image of $\mathbf{X}$.

\item Construct a plane tangent to the sphere at $\mathbf{X}$. Project
      $(\mathbf{v}_1,\mathbf{v}_2,\mathbf{v}_3)$ into that plane, and use
      barycentric interpolation within the projected triangle.

\end{itemize}

In all of these methods, one loses either linear precision

\begin{equation*}
f(\mathbf{X}) = \sum_i \eta_i f(\mathbf{v}_i)
\end{equation*}

(assuming $f$ is linear), or unitarity

\begin{equation*}
\sum_i \eta_i = 1 \,.
\end{equation*}

To preserve both linear precision and unitarity, one must use the areas of
spherical triangles to compute barycentric interpolation coefficients.

The area of a spherical triangle whose vertices are
$(\mathbf{v}_1,\mathbf{v}_2,\mathbf{v}_3)$ and whose edges are arcs of
great circles is
%
\begin{equation}
A = R^2\, E
\end{equation}
%
where $E = \alpha + \beta + \gamma - \pi$ is the \emph{spherical excess},
and $\alpha$, $\beta$, and $\gamma$ are angles at the vertices.  For
example, for one octant, $A = \frac{4 \pi R^2}8$. With $\alpha = \beta
= \gamma = \frac\pi2$, $E = \frac\pi2$ and $A = \frac{\pi R^2}2$.

If one has vectors in Cartesian co\"ordinates from the center to the
vertices, it is easy to compute the arc lengths, in radians, as the
inverse cosine of the dot product of unit vectors to adjacent vertices.

If one has the arc lengths of edges, angles at the corners can be
calculated from
%
\begin{equation}\label{Alpha}
\tan \frac{\alpha}2 = \sqrt{ \frac{ \sin(s-b) \sin(s-c) }
                                  { \sin s \sin(s-a) }}
\end{equation}
%
where $a$ is the arc length of the side opposite angle $\alpha$, $b$ is
the arc length of the side opposite angle $\beta$, $c$ is the arc length
of the side opposite angle $\gamma$, $s = \frac12(a + b + c)$, and
similarly for the other vertices. Thereby, to calculate $E$ from three dot
products of vectors to adjacent vertices requires three inverse cosines,
four sines, seven multiplies, three divides, three square roots, three
inverse tangents, and three multiplies. Some spherical triangles are
poorly characterized by their edge arc lengths, so this might have some
numerical difficulty.

Using the formula for the sine of a sum of angles,
%
\begin{equation}
\sin(p+q+r) = -\sin p \sin q \sin r + \sin p \cos q \cos r +
               \cos p \sin q \cos r + \cos p \cos q \sin r \,,
\end{equation}
%
we see that Equation (\ref{Alpha}) can be written in terms of sines and
cosines of halves of edge arc lengths.  Assuming
$(\mathbf{v}_1,\mathbf{v}_2,\mathbf{v}_3)$ are unit vectors, the cosine of
the arc length is the dot product of the unit vectors from the center of
the sphere to the adjacent vertices, e.g., $a =
\mathbf{p}_1\cdot\mathbf{p}_2$.  Using the half-angle formulae
%
\begin{equation}\label{half}
\cos\frac\theta2 = \sqrt{\frac{1+\cos\theta}2} \text{ and }
\sin\frac\theta2 = \sqrt{\frac{1-\cos\theta}2}\,,
\end{equation}
%
the sines and cosines of half arc lengths can be written in terms of
square roots of dot products.  Therefore, Equation (\ref{Alpha}) can be
written in terms of square roots involving dot products.  It is not
necessary to compute inverse cosines of the dot products, and sines of the
sums of angles.

Writing $\cos\frac{a}2 = x$, $\cos\frac{b}2 = y$, $\cos\frac{c}2 = z$,
        $\sin\frac{a}2 = \xi$, $\sin\frac{b}2 = \eta$, and
        $\sin\frac{c}2 = \zeta$, we have
%
\begin{equation}\begin{array}{lrr}
\sin s = \sin(\frac{a+b+c}2) & =
 -\xi \eta \zeta + \xi y z + x \eta z + x y \zeta & 8 \text{ multiplies} \\
\sin(s-a) = \sin(\frac{-a+b+c}2) & =
  \xi \eta \zeta - \xi y z + x \eta z + x y \zeta & 0 \text{ multiplies} \\
\sin(s-b) = \sin(\frac{a-b+c}2) & =
  \xi \eta \zeta + \xi y z - x \eta z + x y \zeta & 0 \text{ multiplies} \\
\sin(s-c) = \sin(\frac{a+b-c}2) & =
  \xi \eta \zeta + \xi y z + x \eta z - x y \zeta & 0 \text{ multiplies} \\
\end{array}\end{equation}
%
The last three rows do not require new multiplies because the products in
$\sin(s)$ can be reused.

Thereby, once the sines of the arc lengths are obtained, each corner angle
can be calculated using one inverse tangent, three multiplies, one divide
and one square root.  Calculating $E$ requires three inverse tangents, 18
multiplies, three divides, and nine square roots.

A formula for the spherical excess, in terms of arc lengths of the sides,
is
%
\begin{equation}\label{Tangent}
\tan\left(\frac{E}4\right) = \sqrt{\tan\left(\frac{s}2\right)
                                   \tan\left(\frac{s-a}2\right)
                                   \tan\left(\frac{s-b}2\right)
                                   \tan\left(\frac{s-c}2\right)}\,.
\end{equation}

To calculate $E$ from three dot products of vectors to adjacent vertices
requires three inverse cosines, four tangents, four multiplies, four
divides, one square root, and one inverse tangent.

Using the half-angle formula for tangent, this can be written
%
\begin{equation}\label{Dot}
\tan\left(\frac{E}4\right) = \sqrt{
  \frac{(1 - \cos(s)) (1 - \cos(s-a)) (1 - \cos(s-b)) (1 - \cos(s-c))}
       {(1 + \cos(s)) (1 + \cos(s-a)) (1 + \cos(s-b)) (1 + \cos(s-c))}}
\end{equation}

Using the formula for the cosine of a sum of angles,
%
\begin{equation}
\cos(p+q+r) = \cos p \cos q \cos r - \cos p \sin q \sin r
            - \sin p \cos q \sin r - \sin p \sin q \cos r \,,
\end{equation}
%
we see that Equation (\ref{Dot}) can be written in terms of sines and
cosines of halves of the edge arc lengths. Therefore, Equation
(\ref{Dot}) can be written in terms of square roots involving dot
products. It is not necessary to compute inverse cosines of the dot
products, and tangents of the sums of angles.

Writing $\cos\frac{a}2 = x$, $\cos\frac{b}2 = y$, $\cos\frac{c}2 = z$,
        $\sin\frac{a}2 = \xi$, $\sin\frac{b}2 = \eta$, and
        $\sin\frac{c}2 = \zeta$, we have
%
\begin{equation}\begin{array}{lrr}
\cos s = \cos(\frac{a+b+c}2) & =
 x y z - x \eta \zeta - \xi y \zeta - \xi \eta z & 8 \text{ multiplies} \\
\cos(s-a) = \cos(\frac{-a+b+c}2) & =
 x y z + x \eta \zeta - \xi y \zeta - \xi \eta z & 0 \text{ multiplies} \\
\cos(s-b) = \cos(\frac{a-b+c}2) & =
 x y z - x \eta \zeta + \xi y \zeta - \xi \eta z & 0 \text{ multiplies} \\
\cos(s-c) = \cos(\frac{a+b-c}2) & =
 x y z - x \eta \zeta - \xi y \zeta + \xi \eta z & 0 \text{ multiplies} \\
\end{array}\end{equation}

The last three rows do not require new multiplies because the products in
$\cos(s)$ can be reused.

Thereby, $E$ can be calculated using four square roots, 14 multiplies,
one divide, and one arctangent.

In summary the operation counts (excluding dot products, vector norms,
addition and subtraction) to compute $E$ are:
%
\begin{longtable}{lcccc}
& Equation (\ref{Alpha}) & Equation (\ref{Alpha}) & Equation (\ref{Tangent}) & Equation (\ref{Dot}) \\
Using & Sines            & Dot products           & Tangents                 & Dot products \\
\hline
\\[-9pt]
$\cos^{-1}$  & 3         & 0                      & 3                        & 0 \\
$\sin$       & 4         & 0                      & 0                        & 0 \\
$\tan$       & 0         & 0                      & 4                        & 0 \\
$\tan^{-1}$  & 3         & 3                      & 1                        & 1 \\
sqrt         & 3         & 9                      & 1                        & 4 \\
multiply     & 7         & 18                     & 4                        & 15 \\
divide       & 3         & 3                      & 4                        & 1 \\[-10pt]
\end{longtable}
%
Some triangles are poorly characterized by their edge lengths. In such
cases these equations should not be used.

The angles $\alpha$, $\beta$, and $\gamma$ cannot be computed as dot
products involving $(\mathbf{v}_1,\mathbf{v}_2,\mathbf{v}_3)$. The
differences of the azimuths between the two edges incident on each vertex
must be computed.  This requires six azimuth computations because
$\alpha_{ij} \neq -\alpha_{ji}$, where $\alpha_{ij}$ is the azimuth from
$\mathbf{p}_i$ to $\mathbf{p}_j$ In spherical co\"ordinates
$(\lambda,\phi)$ where $\lambda$ is longitude and $\phi$ is latitude:
%
\begin{equation}\label{Azimuth}
\tan \alpha_{ij} =
  \frac{\sin(\lambda_j-\lambda_i)}
       {\cos\phi_i \tan\phi_j - \sin\phi_i \cos(\lambda_j-\lambda_i)} \,.
\end{equation}

Equation (\ref{Azimuth}) can be derived by two rotations,
$\mathbf{p}^{\prime\prime} = \mathbf{R}_2 \mathbf{R}_1 \mathbf{p}$, with
%
\begin{equation}
\begin{array}{ll}
\mathbf{R}_1 = \left[ \begin{array}{ccc}
 \cos\lambda_i & \sin \lambda_i & 0 \\
 -\sin \lambda_i & \cos\lambda_i & 0 \\
 0 & 0 & 1 \end{array} \right]\,, &
\mathbf{R}_2 = \left[ \begin{array}{ccc}
 \cos\phi_i & 0 & \sin\phi_i \\
 0 & 1 & 0 \\
 -\sin\phi_i & 0 & \cos\phi_i \\
\end{array} \right]\,. \end{array}
\end{equation}
%
$\mathbf{R}_1$ rotates about the $z$ axis by $-\lambda_i$.  This puts
$\mathbf{p}_i'$ on the prime meridian of the rotated co\"ordinate system.
$\mathbf{R}_2$ rotates about the new $y$ axis by $-\phi_i$.  This puts
$\mathbf{p}_i^{\prime\prime}$ on the equator of the co\"ordinate system of
the second rotation.  Then Equation (\ref{Azimuth}) is
%
\begin{equation}
\tan \alpha_{ij} =
\frac{\mathbf{p}^{\prime\prime}_{j_2}}{\mathbf{p}^{\prime\prime}_{j_3}}.
\end{equation}

Using $(x_i,y_i,z_i)$ as Cartesian co\"ordinates of $\mathbf{p}_i$,
$\tan\lambda_i = \frac{y_i}{x_i}$ and $\tan\phi_i =
\frac{z_i}{\sqrt{x_i^2+y_i^2}}$. Calculating spherical co\"ordinates from
Cartesian co\"ordinates requires two divides, one square root, and two
arctangents for each point. Using sines and cosines of latitude and
longitude expressed in Cartesian co\"ordinates and $\mathbf{r}_i = ( x_i,
y_i, 0 )$:
%
\begin{equation}\begin{split}\label{Spherical}
\sin\lambda_i = \,& \frac{y_i}{\sqrt{x_i^2+y_i^2}} =
\frac{y_i}{|\mathbf{r}_i|}
 \hspace{0.2in} \sin(\lambda_j-\lambda_i) =
     \frac{x_i y_j - x_j y_i}
          {\sqrt{x_i^2+y_i^2} \, \sqrt{x_j^2+y_j^2} } =
     \frac{( \mathbf{r}_i \times \mathbf{r}_j )_3}
          { |\mathbf{r}_i|\,|\mathbf{r}_j|} \\
\cos\lambda_i = \,& \frac{|x_i|}{\sqrt{x_i^2+y_i^2}} =
\frac{|x_i|}{|\mathbf{r}_i|}
 \hspace{0.2in} \cos(\lambda_j-\lambda_i) =
     \frac{x_i x_j + y_i y_j}
          {\sqrt{x_i^2+y_i^2} \, \sqrt{x_j^2+y_j^2} } =
     \frac{\mathbf{r}_i \cdot \mathbf{r}_j}
          { |\mathbf{r}_i|\,|\mathbf{r}_j|} \\
\sin\phi_i = \,& \frac{z_i}{\sqrt{x_i^2+y_i^2+z_i^2}} =
 \frac{z_i}{| \mathbf{p}_i |} \hspace{0.25in}
\cos\phi_i = \frac{\sqrt{x_i^2+y_i^2}}{\sqrt{x_i^2+y_i^2+z_i^2}} =
 \frac{|\mathbf{r}_i|}{| \mathbf{p}_i |} \,. \\
\end{split}\end{equation}
%
Thereby, $\alpha_{ij}$ can be computed from Equation (\ref{Azimuth}) in
terms of Cartesian co\"ordinates using four square roots, 11 multiplies,
six divides, and one arctangent.

% One could also compute
% %
% \begin{equation}
% \mathbf{n}_{ij} = \frac{\mathbf{N}_{ij}}{|\mathbf{N}_{ij}|} \text{ where }
% \mathbf{N}_{ij} = \mathbf{p}_i \times \mathbf{p}_j \,,
% \end{equation}
% 
% from which $\sin(\lambda_j-\lambda_i) = \mathbf{n}_{{ij}_1}$ and
% $\cos(\lambda_j-\lambda_i) = \mathbf{n}_{{ij}_2}$.  This requires six
% multiplies, one square root, and three divides instead of eight
% multiplies, two square roots, and two divides using Equation
% (\ref{Spherical}).

Equation (\ref{Azimuth}) can be expressed directly in Cartesian
co\"ordinates by substituting Equations (\ref{Spherical}):
%
% \begin{equation}\begin{split}
% \tan \alpha_{ij} = \,&
%   \frac{\sqrt{x_i^2 + y_i^2 + z_i^2} \,(x_j^2 + y_j^2 + z_j^2) \, (x_i y_j - x_j y_i)}
%        {(x_i^2 + y_i^2 + z_i^2) ( x_j^2 + y_j^2 ) z_j
%         - ( x_i x_j + y_i y_j ) (x_j^2 + y_j^2 + z_j^2) z_i} \\[5pt]
% =\,&
%   \frac{|\mathbf{p}_i| |\mathbf{p}_j|^2 ( \mathbf{r}_i \times \mathbf{r}_j )_3 }
%        {|\mathbf{p}_i|^2 |\mathbf{r}_j|^2 z_j -
%         \mathbf{r}_i \cdot \mathbf{r}_j |\mathbf{p}_j|^2 z_i} \,,
% \end{split}\end{equation}
% %
% which requires 16 multiplies, one square root, one divide, and one
% arctangent, to compute each azimuth.
%
\begin{equation}\begin{split}
\tan \alpha_{ij} = \,&
 \frac{\sqrt{x_i^2+y_i^2+z_i^2}\,(x_i y_j - x_j y_i)}
      {(x_i^2+y_i^2)z_j - ( x_i x_j + y_i y_j )z_i}
=
\frac{|\mathbf{p}_i| \, ( \mathbf{r}_i \times \mathbf{r}_j )_3 }
     {\mathbf{r}_i \cdot ( z_j \mathbf{r}_i - z_i \mathbf{r}_j)}
\end{split}\end{equation}
%
which requires 10 multiplies, one square root, one divide, and one
arctangent, to compute each azimuth.

\label{lastpage}
\vspace*{-0.1in} % Somehow, this causes lastpage to be defined
\end{document}

% $Id$
