\documentclass[11pt]{article}
\usepackage[fleqn]{amsmath}\textwidth 6.5in
\oddsidemargin -0.25in
%\evensidemargin -0.5in
\topmargin -0.25in
\textheight 9.0in

\newcommand{\docname}{\bf wvs-075}
\newcommand{\docdate}{2 April 2008}

\usepackage{graphicx}

\usepackage{floatflt}

\begin{document}

%\tracingcommands=1
\newlength{\hW} % heading box width
\newlength{\pW} % page number field width
\settowidth{\hW}{\docname}
\settowidth{\pW}{Page \pageref{lastpage}\ of \pageref{lastpage}}
\ifdim \pW > \hW \setlength{\hW}{\pW} \fi
\makeatletter
\def\@biblabel#1{#1.}
\newcommand{\ps@twolines}{%
  \renewcommand{\@oddhead}{%
    \docdate\hfill\parbox[t]{\hW}{{\hfill\docname}\newline
                          Page \thepage\ of \pageref{lastpage}}}%
\renewcommand{\@evenhead}{}%
\renewcommand{\@oddfoot}{}%
\renewcommand{\@evenfoot}{}%
}%
\makeatother
\pagestyle{twolines}

\vspace{-10pt}
\begin{tabbing}
\phantom{References: }\= \\
To: \>Wayne Enright\\
Subject: \>Title and abstract for ``Scientific Computation on High Performance
Workstations''\\
From: \>Van Snyder\\
\end{tabbing}

\parindent 0pt \parskip 10pt
\vspace{-20pt}

Title: ``Data analysis for the Microwave Limb Sounder instrument on the
EOS Aura Satellite''

Abstract:

A brief discussion of the goals of the Microwave Limb Sounder (MLS)
instrument on NASA's Earth Observing System (EOS) Aura satellite, which
was launched on 15 July 2004, is followed by a discussion of the
organization and mathematical methods of the software used for analysis
of data received from that instrument.

The MLS instrument measures microwave thermal emission from the
atmosphere by scanning the earth's limb, looking forward in the plane of
the orbit, 3500 times per day.  The limb tangent altitude varies from 8
to 80 km. Roughly 600 million measurements in five spectral bands are
returned every day.  From these, estimates of the concentration of
approximately 20 trace atmospheric constituents, and temperature, are
formed at 70 pressure levels on each of these scans -- altogether roughly
5 million results per day.  The program is organized to process
``chunks'' consisting of about 20 scans, with a five-scan overlap at both
ends of the chunk, giving 350 chunks per day.  Processing each chunk
requires about 15 hours on a 3.6 GHz Pentium Xeon workstation.  Although
one chunk can be processed on a workstation, processing all of the data
requires the attention of 350 such processors.

The software is an interpreter of a ``little language,'' as described in
April 2008 Software: Practice and Experience.  This confers substantial
benefits for organization, development, maintenance and operation of the
software.  Most importantly, it separates the responsibilities, and the
requirements for expertise, of the software engineers who develop and
maintain the software, from the scientists who configure the software.

Mathematically, the problem consists of tracing rays through the
atmosphere, roughly corresponding to limb views of the instrument.  The
radiative transfer equation, together with variational equations with
respect to the parameters of interested, are then integrated along these
rays.  A Newton method is then used to solve for the parameters of
interest.

\label{lastpage}
\end{document}
% $Id$
