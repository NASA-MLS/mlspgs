\documentclass[11pt]{article}
\textwidth 6.25in
\oddsidemargin -0.25in
%\evensidemargin -0.5in
\topmargin -0.5in
\textheight 9.25in

\begin{document}

%\tracingcommands=1
\makeatletter
\def\@biblabel#1{#1.}
\newcommand{\ps@twolines}{%
  \renewcommand{\@oddhead}{%
    11 September 2002\hfill Page \thepage\ of \pageref{lastpage}\hfill{\bf
    wvs-011r2}}%
\renewcommand{\@evenhead}{}%
\renewcommand{\@oddfoot}{}%
\renewcommand{\@evenfoot}{}%
}%
\makeatother
\pagestyle{twolines}

\vspace{-20pt}
\begin{tabbing}
\phantom{References: }\= \\
To: \>Dave, Nathaniel\\
Subject: \>More remarks on solving the linear system during retrieval\\
From: \>Van Snyder\\
\end{tabbing}

\parindent 0pt \parskip 3pt
\vspace{-20pt}

Consider the $m \times n$ least-squares problem $A {\bf x} \simeq {\bf
b}$.  Multiply this problem from the left by an orthogonal matrix $Q$
defined so as to give

$$
Q A {\bf x} = 
     \left [ \begin {array}{c} U \\ 0 \end {array} \right ]
     {\bf x} \simeq Q {\bf b} =
     \left [ \begin {array}{c}
     {\bf \hat b_1} \\ {\bf \hat b_2} \end {array} \right ]
$$

where $U$ is $n \times n$ and upper triangular, ${\bf \hat b_1}$ has $n$
elements, and ${\bf \hat b_2}$ has $m-n$ elements.

Let $\hat A = [ \begin{array}{cc} A & -{\bf b} \end{array} ]$.  Write a new
system of normal equations, {\em viz.}

$$
\hat A^T \hat A
 \left [ \begin{array}{c} {\bf x} \\ 1 \end{array} \right ] =
\left [ \begin{array}{cc}
 A^T \\
 -{\bf b}^T
 \end{array} \right ]
 \left [
 \begin{array}{cc}
 A & -{\bf b}
 \end{array} \right ]
 \left [ \begin{array}{c} {\bf x} \\ 1 \end{array} \right ] =
\left [ \begin{array}{cc}
 A^T A        & -A^T {\bf b} \\
 -{\bf b}^T A & {\bf b}^T {\bf b} \end{array} \right ]
 \left [ \begin{array}{c} {\bf x} \\ 1 \end{array} \right ] =
\left [ \begin{array}{c} {\bf 0}\\ r^2 \end{array} \right ]
$$

where the first $n$ rows of the last equation are the usual normal
equations $A^T A {\bf x} = A^T {\bf b}$, and the last row is

$$
-{\bf b}^T A {\bf x} + {\bf b^T b} =
{\bf b}^T ({\bf b} -A {\bf x}) =
{\bf b}^T Q^T Q ({\bf b} - A {\bf x}) =
\left [ \begin{array}{cc}
{\bf \hat b_1^T} & {\bf \hat b_2^T}
\end{array} \right ]
\left [
  \begin{array}{c}
    {\bf 0}\\
    {\bf \hat b_2}
  \end{array} \right ] = {\bf \hat b_2^T \hat b_2} = r^2
$$

where we have used

$$
  Q ({\bf b} - A {\bf x}) =
\left [ \begin{array}{c} {\bf \hat b_1} \\ {\bf \hat b_2} \end{array} \right ]
- \left [ \begin{array}{c} U \\ {0} \end{array} \right ] {\bf x} =
\left [ \begin{array}{c} {\bf \hat b_1} \\ {\bf \hat b_2} \end{array} \right ]
- \left [ \begin{array}{c} {\bf \hat b_1} \\ {0} \end{array} \right ] =
\left [ \begin{array}{c} {\bf 0} \\ {\bf \hat b_2} \end{array} \right ] .
$$

Therefore $r = ||{\bf \hat b_2}||$.  Since $||Q ({\bf b} - A {\bf x})|| =
||Q|| \, ||({\bf b} - A {\bf x})|| = ||({\bf b} - A {\bf x})|| = ||{\bf
\hat b_2}||$, $r$ is the norm of the part of ${\bf b}$ that is not in the
column space of $A$.

Now consider the Cholesky factor $\hat U$ of $\hat A^T \hat A$, i.e. $\hat U^T
\hat U = \hat A^T \hat A$.  Dissect
$
  \hat U = \left [
           \begin{array}{cc} \tilde U & -{\bf v} \\ {\bf 0}^T & w
           \end{array} \right ] .
$
Then

$$
\hat A^T \hat A
 \left [ \begin{array}{c} {\bf x} \\ 1 \end{array} \right ] =
\hat U^T \hat U
 \left [ \begin{array}{c} {\bf x} \\ 1 \end{array} \right ] =
\left [ \begin{array}{cc} \tilde U^T \tilde U & -\tilde U^T {\bf v} \\
                          -{\bf v}^T \tilde U & {\bf v}^T {\bf v} + w^2
                       \end{array} \right ]
 \left [ \begin{array}{c} {\bf x} \\ 1 \end{array} \right ] =
\left [ \begin{array}{c} {\bf 0}\\ r^2 \end{array} \right ] .
$$

The last equation above can be written as the two equations $\tilde U^T
\tilde U {\bf x} = \tilde U^T {\bf v}$ and $-{\bf v}^T \tilde U {\bf x} +
{\bf v}^T {\bf v} + w^2 = r^2$.  Solve the first of these to get ${\bf x}
= \tilde U^{-1} {\bf v}$.  Substitute this into the second to get $w^2 =
r^2$.  Therefore, by computing $\hat U$ instead of the Cholesky factor of
$A^T A$:

\begin{enumerate}
\item We can solve for ${\bf x}$ with a single back-solve (this isn't
 actually any less work), and
\item The nonlinear solver needs $r = w = \hat U_{n+1,n+1}$.  The obvious
 way to compute it, {\em viz.} $r^2 = {\bf b}^T({\bf b} - A {\bf x})$,
 requires keeping $A$ until after ${\bf x}$ is computed.
\end{enumerate}

\label{lastpage}
\end{document}
% $Id$
