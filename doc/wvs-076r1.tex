\documentclass[11pt]{article}
\usepackage[fleqn]{amsmath}\textwidth 6.5in
\oddsidemargin -0.25in
%\evensidemargin -0.5in
\topmargin -0.25in
\textheight 9.0in

\newcommand{\docname}{\bf wvs-076r1}
\newcommand{\docdate}{15 April 2008}

\usepackage{graphicx}

\usepackage{floatflt}

\begin{document}

%\tracingcommands=1
\newlength{\hW} % heading box width
\newlength{\pW} % page number field width
\settowidth{\hW}{\docname}
\settowidth{\pW}{Page \pageref{lastpage}\ of \pageref{lastpage}}
\ifdim \pW > \hW \setlength{\hW}{\pW} \fi
\makeatletter
\def\@biblabel#1{#1.}
\newcommand{\ps@twolines}{%
  \renewcommand{\@oddhead}{%
    \docdate\hfill\parbox[t]{\hW}{{\hfill\docname}\newline
                          Page \thepage\ of \pageref{lastpage}}}%
\renewcommand{\@evenhead}{}%
\renewcommand{\@oddfoot}{}%
\renewcommand{\@evenfoot}{}%
}%
\makeatother
\pagestyle{twolines}

\vspace{-10pt}
\begin{tabbing}
\phantom{References: }\= \\
To: \>Bill, Dave, Nathaniel\\
Subject: \>Organization of changes to Full Forward Model for TSCAT-computation
mode\\
From: \>Van Snyder\\
References: \>wvs-074\\
\end{tabbing}

\parindent 0pt \parskip 10pt
\vspace{-20pt}

\newcommand{\R}{\stackrel*{R^\oplus_{\text{eq}_i}}}
\begin{enumerate}

\item The orbit geodetic angle $\phi_i$ for each profile $\Pi_i$ is
gotten from the {\tt phiTan} quantity of the {\tt Forward\-Model\-In} or
{\tt Forward\-Model\-Extra} vector in the usual way.  {\tt phiTan} is a
MIF $\times$ MAF quantity; the $\phi$ values for all MIFs for the
hypothetical reference MAF should all be set to the same value, probably
using a {\tt forge} command.  We can assume $\phi_i = 0$.

\item The latitude and equivalent earth radius $\R$ for each $\Pi_i$ are
computed in the usual way.

\item $\Phi = \{\phi_m\}$ and $Z = \{\zeta_n\}$ for forward model
calculations are established in the usual way: $\Phi$ is the temperature
basis, which should be centered at $\phi_i$, and Z is the union of the
$\zeta$'s for all species in the configuration.

\item The height reference grid {\tt h\_GLgrid} is calculated from $Z$,
$\Phi$, and the temperatures using hydrostatic equilibrium in the usual
way.

\item The $(\phi_j,\zeta_k)$ co{\"o}rdinates of the scattering points $S
= \{S_{jk}\}$ at which computations are to be done are on a grid that is
the Cartesian product $\{\phi_j\} \times \{\zeta_k\}$, where $\{\phi_j\}$
and $\{\zeta_k\}$ are taken from {\tt hGrid} and {\tt vGrid} fields of
the output structure.  It is necessary that $\{\phi_j\} \subseteq \Phi$
and $\{\zeta_k\} \subseteq Z$, i.e., the scattering points horizontal and
vertical bases are subsets of the temperature bases, else the ray along
which radiative transfer takes place doesn't have a breakpoint at $S_{jk}$
and radiative transfer therefore can't stop at $S_{jk}$.

\item The height $H_{jk}$ of the scattering point $S_{jk}$ is calculated
by interpolation in {\tt h\_GLgrid}.

\item For each $\phi_i$ and $\phi_j$ the angle $\phi_{ij} = \phi_j -
\phi_i$ is the angle between the limb tangent point and the scattering
point, as seen from the center of the equivalent circular earth tangent to
the true earth at the tangent point.

\item The angles $X = \{\chi_l\}$ between the line of sight and the rays
scattered to the line of sight are taken from the tables of Mie phase
functions.

\item For each $\Pi_i$ and each scattering point $S_{jk}$ there is a
critical angle $\chi^c_{ijk} = \phi_{ij} - \cos^{-1}\left( \R /
H_{jk} \right)$.

The quantity $\R$ is not precisely the correct value; what is really
wanted is the radius of the equivalent circular earth at the point of
reflection of the ray from the earth's surface, which is not necessarily
the same as the radius to the point on the surface below the limb ray
tangent point.  For small values of $\phi_{ij}$, $\R$ should be
sufficiently close to the correct value.  We're ignoring topography
anyway, so errors in $\R$ are probably of second order.

\item The tangent height for each scattered ray $P_{ijkl}$ from space to
$S_{jk}$ at angle $\chi_l$ $H_{ijkl} = H_{jk} \cos( \phi_{ij} - \chi_l)$
but the tangent point geodetic angle $\phi_{ijkl}$ depends upon the
relation between $\chi_l$ and $\chi^c_{ijk}$:

\begin{equation*}
\phi_{ijkl} = \left\{
\begin{array}{ll}
\chi_l & \text{if }
         \chi_l \geq \chi^c_{ijk} \text{ or } \chi_l \leq -\chi^c_{ijk} -\pi \\
& \\
\left\{\begin{array}{ll}
 2 \theta + \chi_l & \text{if } \chi_l > \phi_{jk} - \pi/2 \\
 -2 \theta -\chi_l & \text{otherwise} \\
\end{array}\right.
&
\begin{array}{l}
\text{if } -\chi^c_{ijk} -\pi < \chi_l < \chi^c_{ijk} \\
\text{(earth intersecting ray)} \\
\end{array} \\

\end{array}\right.
\end{equation*}
%
where
%
\begin{equation*}
\theta = \cos^{-1} \left| \frac{\cos( \phi_{ij} - \chi_l )}
                               {\cos( \phi_{ij} - \chi^c_{ijk} )}
                   \right|
\end{equation*}

\item\label{xfer} Given a model atmosphere, for each hypothetical profile
$\Pi_i$, scattering angle $\chi_l$, and scattering point $S_{jk}$,
radiative transfer is calculated from space to $S_{jk}$ along the ray
$P_{ijkl}$.  In order for the radiative transfer to stop at $S_{jk}$,
$(\phi_j,\zeta_k)$ must be a break point of the line-of-sight
integration.  The radiative transfer calculation yields radiance
$I_{ijkl}$ and its derivatives with respect to species $s$ mixing ratio
(pretend temperature is a species mixing ratio) $D^s_{ijklp} =
\frac{\partial I_{ijkl}} {\partial f^s_{p}}$ for each point $p$ along
the ray $P_{ijkl}$.

\item\label{final} The basis sets of temperatures $T = \{T_m\}$ and ice
water content values IWC = \{IWC$_n$\} at which the Mie phase functions
are evaluated are the same at every scattering point $S_{jk}$.  They are
taken from the tables of Mie phase functions.  At each $S_{jk}$ and for
each $T_m$, IWC$_n$, and species $s$, the radiances $I_{ijkl}$ and
derivatives $D^s_{ijklp}$ calculated in step \ref{xfer} are convolved
over $X$ with the Mie phase function $P(\chi)$, giving arrays of
scattered radiance $\overline{I}_{ijkmn}$ and derivatives
$\overline{D}^s_{ijkmnp}$.
%
\begin{equation*}
\overline{I}_{ijkmn} = \frac
 {\int_{-\pi}^\pi I_{ijk}(\chi) P_{mn}(-\chi)\, d \chi}
 {\int_{-\pi}^\pi P_{mn}(-\chi)\, d \chi}
\approx \frac
 {\sum_l I_{ijkl} P_{mn}(-\chi_l)}
 {\sum_l P_{mn}(-\chi_l)}
\end{equation*}
%
$\frac{\partial \overline{I}_{ijk}}{\partial T_{jk}}$ is the convolution
of $\frac{\partial P(\chi)}{\partial T_{jk}}$ with $I_{ijkl}$

$\frac{\partial \overline{I}_{jk}}{\partial \text{IWC}_{jk}}$ is the
convolution of $\frac{\partial P(\chi)}{\partial \text{IWC}_{jk}}$ with
$I_{ijkl}$

The derivatives $\overline{D}^s_{ijkmnp}$ actually have to be put onto a
grid giving $\overline{D}^s_{ijkmnqr}$, but I don't know how to do that yet.

\end{enumerate}

In addition to needing the results calculated in step \ref{final} for
several latitudes, which are gotten by having several $\Pi_i$, they are
presumably also needed for several seasons.

\label{lastpage}
\end{document}
% $Id$
