\documentclass[11pt]{article}

\usepackage{alltt}
\usepackage[fleqn]{amsmath}
\usepackage{floatflt}
\usepackage{graphicx}
\usepackage{longtable}
\usepackage[strings]{underscore}

\newcommand{\docname}{wvs-120r1}
\newcommand{\docdate}{6 July 2020}

\textwidth 6.5in
\oddsidemargin -0.25in
%\evensidemargin -0.5in
\topmargin -0.5in
\textheight 9in

\parindent 0pt
\parskip 6pt

\ifx\pdfoutput\undefined
  \pdfoutput=0
  \usepackage[hypertex,plainpages,hyperindex=true]{hyperref}
  \hypersetup{%
    hypertexnames=false%
  }
  % Specify the driver for the color package
  \ExecuteOptions{dvips}
  %\ExecuteOptions{xdvi}
\else
  \ifnum\pdfoutput>0
    \usepackage[pdftex,plainpages,hyperindex=true,pdfpagelabels]{hyperref}
    \hypersetup{%
      hypertexnames=false,%
      colorlinks=true,%
      linktocpage=true,%
    }
    % Specify the driver for the color package
    \ExecuteOptions{pdftex}
  \else
    \usepackage[hypertex,plainpages,hyperindex=true]{hyperref}
    \hypersetup{%
      hypertexnames=false%
    }
    % Specify the driver for the color package
    \ExecuteOptions{dvips}
    %\ExecuteOptions{xdvi}
  \fi
\fi

\hyperbaseurl{}
\newcommand\hr[1]{\href{#1.dvi}{dvi}, \href{#1.pdf}{pdf}}
\newcommand\h[1]{#1 (\hr{#1})}

\begin{document}

%\tracingcommands=1
\newlength{\hW} % heading box width
\newlength{\pW} % page number field width
\settowidth{\hW}{\bf\docname}
\settowidth{\pW}{Page \pageref{lastpage}\ of \pageref{lastpage}}
\ifdim \pW > \hW \setlength{\hW}{\pW} \fi
\makeatletter
\def\@biblabel#1{#1.}
\newcommand{\ps@twolines}{%
  \renewcommand{\@oddhead}{%
    \docdate\hfill\parbox[t]{\hW}{{\hfill\bf\docname}\newline
                          Page \thepage\ of \pageref{lastpage}}}%
\renewcommand{\@evenhead}{}%
\renewcommand{\@oddfoot}{}%
\renewcommand{\@evenfoot}{}%
}%
\makeatother
\pagestyle{twolines}

\renewcommand{\d}{\text{d}}
\newcommand{\T}{\mathcal{T}}

\vspace{-10pt}
\begin{tabbing}
\phantom{References: }\= \\
To: \>Nathaniel, Paul, Bill, Michael\\
Subject: \>Magnetic fields\\
From: \>Van Snyder\\
%Reference: \>\h{wvs-050} \\
\end{tabbing}

\vspace*{-30pt}

\section*{Current method in MLSL2}

\subsection*{For AURA}

The magnetic field for AURA is calculated by a {\tt fill} statement in the
{\tt l2cf}. The coordinates of the magnetic field quantity are $\phi$ and
$\zeta$. The magnetic field quantity is required to have the same $\phi$
and $\zeta$ grids as GPH (which is in turn required to have the same grids
as temperature).  The value in the GPH quantity at each $\phi$ and $\zeta$
is then used as if it were geometric height for the corresponding element
of the magnetic field quantity.  Latitude and longitude for each $\phi$ of
the magnetic field quantity are taken from the ones for the GPH quantity
for the same $\phi$.  Latitude, longitude, and height are then converted
to Cartesian coordinates for use by the magnetic field model (IGRF), which
produces the magnetic field for that point in the ECR frame.

\subsection*{For UARS}

The magnetic field for UARS is calculated by the forward model wrapper,
which means it is calculated for each iteration.  It is calculated in a
plane determined by the geolocation of the reference MIF of PTan of the
current MAF, the geolocation of the same MIF of the spacecraft velocity
vector of the current MAF, and the center of the Earth.  The reference
MIF, and the HGrid and VGrid  of the magnetic field quantity, are
specified by the L2CF. The zero of HGrid is assumed to be at the reference
MIF of the PTan of the current MAF.  Latitude and longitude corresponding
to each $\phi$ are computed along the line (actually great circle) passing
through the latitude and longitude of the reference MIF of the PTan and
spacecraft position, with spacing specified by the HGrid.

The VGrid coordinate of the magnetic field is $\zeta$, and is required to
be the same as GPH.  GPH is assumed to have only one instance.  Its
latitude, longitude, and values are assumed to have been filled as if it
were a profile at the reference MIF for PTan.  As for AURA, the value of
GPH is used as if it were height.  The magnetic field is then calculated,
using latitude, longitude, and height, in the same way as for AURA. 
Whatever GPH is found in the state or extra vectors is used.  If the GPH
is the same on each iteration, the magnetic field is the same on each
iteration, i.e., the calculation is redundant.

\subsection*{In the forward model}

The magnetic field at each point on the path being integrated in the
forward model is obtained by interpolation based on $\phi$ and $\zeta$,
and then rotated from the ECR frame to the FOV frame.

\section*{First proposed revision}

For UARS, calculate GPH in the forward model wrapper before calculating
the magnetic field.  Thereby, the magnetic field position reflects the
currently-estimated state of the atmosphere, not the apriori state.

\section*{Longer-term revision}

Instead of $\phi$ and $\zeta$, use $\phi$ and $h$ for the magnetic field
quantity's coordinates.  In the forward model, this is a very minor
change, requiring to use $h$, instead of $\zeta$, to interpolate from the
magnetic field quantity to the path of integration.  The {\tt metrics}
calculation already provides $h$ for each point on the path of
integration.

For both AURA and UARS, calculate the magnetic field by a {\tt fill}
statement in the {\tt l2cf}, not in the wrapper for each iteration.  The
$\phi$ and $h$ coordinates would be specified by {\tt HGrid} and {\tt
VGrid} statements in the {\tt l2cf}.  For AURA, latitude and longitude
could be taken from any geophysical quantity with the same extent and
spacing of its HGrid.

For UARS, latitude and longitude for each point in the magnetic field grid
would be calculated using the latitude, longitude, and height of PTan and
the spacecraft using the same MIF in the current MAF for both, with the
MIF chosen such that the line between those two points passes closest to
the point in the magnetic field grid.  This would be based on the zero for
$\phi$ being at that MIF in PTan, and the other points being along the
geometric (not refracted) line of sight between PTan and the spacecraft
position, at that MIF, with the angular separation specified by the HGrid.

The IGRF model is a nontrivial model, entailing evaluation of a fairly
high order spherical harmonic expansion.  It is probably somewhat more
expensive than calculating $\beta$ for a single species, even in the
polarized case.  Therefore, calculating it in advance, instead of in each
iteration, could make a noticable difference in run times.  The expense of
evaluating a more elaborate magnetic field model, incorporating the solar
wind and diurnal effects, would certainly be greater than for the IGRF
model, making the difference in cost between prior calculation and on-line
calculation even greater.

\section*{For SMLS}

For SMLS, the horizontal grid is a QTM. A QTM-based grid for the magnetic
field will be necessary for SMLS. The vertical axis ought to be height,
not $\zeta$ or GPH, because geometric co\"ordinates used by the IGRF
model.

A QTM-based grid for the magnetic field would also work for UARS. The
scheme used for UARS is complicated. It might be useful to change to a
QTM-based magnetic field for UARS, and remove the complication caused by
the UARS scheme. The complication appears in the {\tt Grids_t} structure,
which has an additional ``cross'' coordinate.

\label{lastpage}
\end{document}

% $Id$

% $Log$
% Revision 1.1  2014/11/10 20:27:39  vsnyder
% Initial commit
%
