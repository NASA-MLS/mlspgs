\documentclass[11pt]{article}
\usepackage{alltt}
\usepackage[fleqn]{amsmath}
\usepackage{floatflt}
\usepackage{graphicx}
\usepackage{longtable}
\usepackage[strings]{underscore}

\textwidth 6.5in
\oddsidemargin -0.25in
%\evensidemargin -0.5in
\topmargin -0.5in
\textheight 9in

\newcommand{\docname}{wvs-135}
\newcommand{\docdate}{8 April 2016}

\ifx\pdfoutput\undefined
  \pdfoutput=0
  \usepackage[hypertex,plainpages,hyperindex=true]{hyperref}
  \hypersetup{%
    hypertexnames=false%
  }
  % Specify the driver for the color package
  \ExecuteOptions{dvips}
  %\ExecuteOptions{xdvi}
\else
  \ifnum\pdfoutput>0
    \usepackage[pdftex,plainpages,hyperindex=true,pdfpagelabels]{hyperref}
    \hypersetup{%
      hypertexnames=false,%
      colorlinks=true,%
      linktocpage=true,%
    }
    % Specify the driver for the color package
    \ExecuteOptions{pdftex}
  \else
    \usepackage[hypertex,plainpages,hyperindex=true]{hyperref}
    \hypersetup{%
      hypertexnames=false%
    }
    % Specify the driver for the color package
    \ExecuteOptions{dvips}
    %\ExecuteOptions{xdvi}
  \fi
\fi

\hyperbaseurl{}
\newcommand\hr[1]{\href{#1.dvi}{dvi}, \href{#1.pdf}{pdf}}
\newcommand\h[1]{#1 (\hr{#1})}

\begin{document}

%\tracingcommands=1
\newlength{\hW} % heading box width
\newlength{\pW} % page number field width
\settowidth{\hW}{\bf\docname}
\settowidth{\pW}{Page \pageref{lastpage}\ of \pageref{lastpage}}
\ifdim \pW > \hW \setlength{\hW}{\pW} \fi
\makeatletter
\def\@biblabel#1{#1.}
\newcommand{\ps@twolines}{%
  \renewcommand{\@oddhead}{%
    \docdate\hfill\parbox[t]{\hW}{{\hfill\bf\docname}\newline
                          Page \thepage\ of \pageref{lastpage}}}%
\renewcommand{\@evenhead}{}%
\renewcommand{\@oddfoot}{}%
\renewcommand{\@evenfoot}{}%
}%
\makeatother
\pagestyle{twolines}

\vspace{-10pt}
\begin{tabbing}
\phantom{References: }\= \\
To: \>Nathaniel\\
Subject: \>Outline of 3-dimensional forward model\\
From: \>Van Snyder\\
%Reference: \> \\
\end{tabbing}

\parindent 0pt \parskip 6pt
\vspace{-20pt}

\section{Forward model}

Regardless of whether the forward model is solving for the atmospheric
state on a one-, two-, or three-dimensional grid, the mathematical problem
still consists of integrating the radiative-transfer equation along a
one-dimensional path through the atmosphere.

To solve for the atmospheric state using a Gauss-Newton iteration, one
needs partial derivatives of the radiance computed at the instrument
position with respect to the atmospheric state at the points where
solutions are desired.  To integrate the radiative-transfer equation, and
its variational equations to compute partial derivatives, the atmospheric
state is needed on the path along which the radiative-transfer equation is
integrated.  This state is gotten by interpolating from the state at the
points where solutions are desired to the integration path.

The interpolation problem is different in detail depending upon the
dimensionality, and the details, of the representation of the atmospheric
state.  I.e., whether the atmospheric state is represented on a one-,
two-, or three-dimensional grid, and if more than one dimension, whether
the height direction is stacked, coherent, or both.  As an abstraction,
it's the same problem.  The only difference that an observer from outside
the interpolation abstraction might notice is that the number of
interpolating coefficients is different.

A forward model could, in principle, be constructed that will work with an
atmospheric state represented on any grid, of any dimensionality.  The
only differences would be in the representation of the grid, the
computation of the relationship between the grid and integration path, and
the method to interpolate from the grid to the path.

Let us assume that the process that creates the grid produces a serial
number for each point in the grid, which can be used to compute a column
subscript for the Jacobian.  Let us further assume that this same serial
number is used to index the set of interpolation coefficients.  With these
parts in hand, the remainder of the forward model should be insensitive to
the grid.

The present full forward model is intimately aware of the grids used to
represent the atmospheric state.  The principles and details are not
hidden inside abstractions.  Further, the present forward model represents
the atmospheric state differently from the state-vector representation
used throughout the remainder of the {\tt mlsl2} program, and represents
the Jacobian differently from the matrix representation used by the
retriever.

A revised forward model should treat the grids and the relationship of the
grids to the path as abstractions.  It should further use the state-vector
representation directly instead of copying it to an internal
representation, and it should produce the Jacobian in the form of the
matrix representation  used by the retriever.

The sequence of operations of the revised forward model can be the same as
for the present forward model:

\begin{enumerate}

\item Compute the sizes of everything; determine processing options.

\item Acquire any extra state, beyond the atmospheric state, e.g., the
      magnetic field.

\item Compute the relationship between temperature, pressure, and height,
      using the hydrostatic equilibrium assumption, and compute
      derivatives with respect to temperature if necessary.

\item Compute the paths upon which to integrate the radiative-transfer
      equation.

\item For each sideband

  \begin{enumerate}

    \item Determine the frequencies at which to evaluate radiances.

    \item\label{metrics} For each path, determine the relationship between
          the path and the grid upon which the atmospheric state is
          represented.  This step might produce some interpolating
          coefficients that will make interpolation from the state to the
          path more efficient.  For example, when using a
          three-dimensional grid, it might produce two-dimensional
          (horizontal-only) interpolating coefficients for points on the
          boundaries of grid elements, which can be combined with heights
          to produce three-dimensional interpolation coefficients more
          efficiently than producing three-dimensional interpolation
          coefficients from first principles during path integration.  The
          representation of whatever information is produced should be
          hidden in an abstraction that is used later for interpolation,
          but not examined or otherwise manipulated directly by the
          forward model proper.  In the present forward model, this is the
          responsibility of the {\tt metrics} module.  In the prototype
          three-dimensional forward model, this would be the
          responsibility of the {\tt Metrics_3D} module.  In an ideal
          world, this distinction should be managed by the process that
          creates the grid, and should otherwise be hidden from the
          forward model.

    \item On each path

    \begin{enumerate}

      \item Compute the relationship between the integration path and the
            grid upon which the atmospheric state is represented.  This
            should conspire with step \ref{metrics} to use an
            interpolation abstraction provided by the grid construction
            method and the path calculation method used in step
            \ref{metrics}, but the conspiracy should be hidden, so that
            the forward model is unaware of and unaffected by the details
            of representation of the state.

      \item For each frequency, integrate the radiative-transfer equation,
            and its derivatives, along that path.  In the present forward
            model, an adaptive method is used.  The major integration
            steps proceed from one intersection of the path with a surface
            of constant pressure to the next.  In the prototype
            three-dimensional forward model, the major steps are bounded
            by intersections of the path with faces of cells of the
            three-dimensional grid.  This detail, and the fact that it is
            different in two and three dimensions, should be hidden from
            the forward model.  An error estimate is formed.  If the error
            is not sufficiently small, the atmospheric state is evaluated
            at intermediate points, currently at abscissae of a
            three-point Gauss-Legendre quadrature rule.  Once the
            positions of necessary additional points are determined, in
            terms of the independent variable of integration (no matter
            whether it is $\zeta$ of $s$), the representation of the path
            and grid should be used by an interpolation abstraction to
            produce values of atmospheric state at these additional
            points.  The forward model should not be aware of the method
            by which this is accomplished.

      \item If necessary, compute the channel-average radiance and
            derivatives.  In the present forward model, this is called
            \emph{frequency averaging}.  Depending upon the characteristics
            of the instrument, the channel characteristics might be such
            that the model is evaluated at only one frequency in each
            channel.  In the present forward model, in order to combine
            PFA and non-PFA results, it is necessary to compute
            frequency-averaged results at every major point along the path
            (not including interposed Gauss-Legendre abscissae). 
            Depending upon the characteristics of the instrument, in
            particular if frequency averaging is not used, this
            consideration might not play a r\^ole.

      \end{enumerate}

    \item Convolve the radiances and derivatives along the several paths
          with the antenna pattern.

  \end{enumerate}

\end{enumerate}

The mechanisms of the conspiracies that hide details of the grid, and the
relationship of the grid to the integration path, should exploit
type-bound procedures, type extension, and polymorphism.  At present, both
the NAG (version 6) and Intel (version 16.0.2) compilers support all the
necessary features.  Intel version 14.0.0 does not, but methods to work
around its deficiencies have so far been found.

\section{Modules needing attention}

The intent is that a newer forward model, whether for two or three
dimensions, would use the vector and matrix types, and their underlying
geolocation abstractions, rather than having internal representations for
atmospheric state and Jacobians.  If we take care with abstractions, the
geolocations of the newer vector-related types, and the newer forward
model, ought to work for one-, two-, or three-dimensional problems. 
Therefore, an entirely new forward model module will be needed.  Much of
its design can be copied from the existing forward model module, and many
of its component parts, especially those in separate modules, can be used,
either as they are, or with modest changes, or at least their designs can
be used.

It is desirable to change the {\tt QuantityTemplates} module to use the
new {\tt Geolocation_m} and {\tt Geolocation_0} modules, as structures and
methods within those modules are used by the {\tt Metrics_3D_m} module. 
The alternative is to copy geolocation information from one representation
to another, a practice we are trying to replace with a redesigned forward
model.

This will induce ubiquitous changes elsewhere, to the extent other modules
use the geolocation components of vector quantities directly.  This use
should have been handled by abstractions in the first place, and should be
replaced by reference to abstractions.  This might well be the most
time-consuming step.

Changes might be necessary in {\tt HGrid} and {\tt VGrid} modules in the
{\tt l2} directory, and {\tt HGridsDatabase} and {\tt VGridsDatabase}
modules in the {\tt lib} directory.  Some changes have already been
introduced in the {\tt HGrid} and {\tt HGridsDatabase} modules to handle
QTM horizontal grids.

The {\tt VectorValue_t} type in the {\tt Vectors} module already includes
components for values with up to four dimensions.  There should be only
minor changes, if any, in the {\tt Vectors} module, although how it is
used by other modules might require change.  For example, with a QTM
horizontal grid, the geolocation of a three-dimensional atmospheric state
is represented by a one-dimensional set of horizontal geolocations, and a
two-dimensional set of heights, which seems backward.  The horizontal
geolocations of the QTM vertices are identified by a single serial number,
which is recorded in the data structure used to find a facet of the QTM
from a horizontal geolocation, not by two subscripts, because the
horizontal grid is not rectangular.  The {\tt Metrics_3D_m} module assumes
the grid is stacked but not necessarily coherent.  It determines
automatically whether it is coherent, by examining the extent of the
second dimension of the heights array.  The serial number of a QTM vertex
is used to index the second extent of the heights array if the grid is not
coherent.  The serial number and a height index, within the range of the
first extent of the heights array, are used to compute a subscript for a
value of a vector quantity, and a column subscript for a block of the
Jacobian.  In a two-dimensional problem, the horizontal serial number
would be what is currently viewed as a ``phi'' index.

Some low-level modules that represent geolocation using arrays, and
operate on them using methods from the {\tt Geometry} module, should
gradually be revised to use structures and methods from the {\tt
Geolocation_0} module.  This can proceed slowly and independently for the
most part, with little interaction with other modules.

\label{lastpage}
\vspace*{-0.1in} % Somehow, this causes lastpage to be defined
\end{document}

% $Id$

% $Log$
