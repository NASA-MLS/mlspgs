\documentclass[11pt]{article}
\usepackage[fleqn]{amsmath}\textwidth 6.5in
\oddsidemargin -0.25in
%\evensidemargin -0.5in
\topmargin -0.25in
\textheight 9.0in

\newcommand{\docname}{\bf wvs-079}
\newcommand{\docdate}{8 May 2008}

\begin{document}

%\tracingcommands=1
\newlength{\hW} % heading box width
\newlength{\pW} % page number field width
\settowidth{\hW}{\docname}
\settowidth{\pW}{Page \pageref{lastpage}\ of \pageref{lastpage}}
\ifdim \pW > \hW \setlength{\hW}{\pW} \fi
\makeatletter
\def\@biblabel#1{#1.}
\newcommand{\ps@twolines}{%
  \renewcommand{\@oddhead}{%
    \docdate\hfill\parbox[u]{\hW}{{\hfill\docname}\newline
                          Page \thepage\ of \pageref{lastpage}}}%
\renewcommand{\@evenhead}{}%
\renewcommand{\@oddfoot}{}%
\renewcommand{\@evenfoot}{}%
}%
\makeatother
\pagestyle{twolines}

\vspace{-10pt}
\begin{tabbing}
\phantom{References: }\= \\
To: \>Van\\
Subject: \>Logarithm of $2 \times 2$ matrix\\
From: \>Van Snyder\\
\end{tabbing}

\parindent 0pt \parskip 10pt
\vspace{-20pt}

According to {\bf Functions of Matrices: Theory and Computation} by
Nicholas J. Higham, a function of a matrix, in particular the logarithm of a
matrix, can be computed using its Schur decomposition

\begin{equation}
f(A) = Q\, f(T)\,\, Q^*
\end{equation}

where $Q$ is unitary and $T$ is upper triangular.  Applying this to a $2 \times
2$ matrix $T$, using Equation (11.27) from the above cited reference, we have

\begin{equation}
\log \left( \begin{array}{cc}
 t_{11} & t_{12} \\
 0      & t_{22}
 \end{array} \right) =
 \left\{ \begin{array}{ll}
  \left( \begin{array}{cc}
   \log t_{11} & f_{12} \\
   0           & \log t_{22}
   \end{array} \right)
   & t_{11} \neq t_{22} \\
   & \\
  \left( \begin{array}{cc}
   \log t_{11} & \frac{t_{12}}{t_{11}} \\
   0           & \log t_{11}
   \end{array} \right)
   & t_{11} = t_{12}
  \end{array}
  \right.
\end{equation}

where $z = \frac{t_{11}-t_{22}}{t_{11}+t_{22}}$ and

\begin{equation}
f_{12} = t_{12} \frac{2 \tanh^{-1} z + 2 \pi i [
 \mathcal{U} ( \log t_{22} - \log t_{11} ) +
 \mathcal{U} ( \log(1+z) - \log(1-z))]}{t_{11}-t_{12}} \,.
\end{equation}

$\mathcal{U}(z)$ is the \emph{unwinding number} for $z$, defined by Equation
(11.2), \emph{viz}.

\begin{equation}
\mathcal{U}(z) = \frac{z - \log(e^z)}{2 \pi i} =
\left \lceil \frac{\Im z - \pi}{2 \pi} \right \rceil\,,
\end{equation}

which is zero when $-\pi < \Im z \leq \pi$, and in particular when $z$ is real.

When $t_{11} \approx t_{22}$, eqv. $z \approx 0$, the series expansion

\begin{equation}
\frac{\tanh^{-1} z}z = \sum_{k=0}^\infty \frac{z^{2k}}{2k+1}
\end{equation}

should be used.

\label{lastpage}
\end{document}
% $Id$
