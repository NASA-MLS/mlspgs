% To make the graphics
% tgif -print -eps wvs-160-1.obj; epstopdf wvs-160-1.eps
% tgif -print -eps wvs-160-2.obj; epstopdf wvs-160-2.eps
% tgif -print -eps wvs-160-facets.obj; epstopdf wvs-160-facets.eps
% tgif -print -eps wvs-160-surf.obj; epstopdf wvs-160-surf.eps
% tgif -print -eps wvs-160-cone.obj; epstopdf wvs-160-cone.eps

\makeatletter\let\ifGm@compatii\relax\makeatother
\documentclass[landscape]{beamer}
\usepackage[nofancy,notoday]{rcsinfo}
\usepackage{color}
\usepackage{amsmath}
\usepackage{amssymb}
\usepackage{moreverb}
\usepackage{multicol}
\usepackage{graphicx}
\usepackage{floatflt}
%\usepackage[strings]{underscore}
\usepackage{xr-hyper}
\externaldocument[128-]{\logdir/wvs-128}
\externaldocument[131-]{\logdir/wvs-131}
\externaldocument[146-]{\logdir/wvs-146}

\usepackage{hyperref}
% \hypersetup{%
%   hypertexnames=false,%
%   colorlinks=true,%
%   linktocpage=true,%
% }
% Specify the driver for the color package
%\ExecuteOptions{pdftex}

%
\renewcommand{\b}{\mathbf}
\renewcommand{\d}{\text{d}}
\newcommand{\T}{^{T}}
\newcommand{\cs}[1]{$_{\text{#1}}$}
\newcommand{\inv}{^{\mathrm -1}}
\newcommand{\cp}[1]{$^{\text{#1}}$}
\newcommand{\degsym}{\ensuremath{^\circ}}
\newcommand{\newframe}[2][]{\begin{frame}\frametitle{\hfill #2 \hfil}}

\hyperbaseurl{}
\newcommand\hr[1]{\href{#1.pdf}{#1}}
\newcommand\h[1]{\hr{#1}}

\parskip 2pt
%
% -----------------------------------------------------------------------------
%
\title{3-D QTM Metrics}
\subtitle{wvs-160r1}
\author{Van Snyder}
\date{12 June 2020}
\titlegraphic{\includegraphics[width=1.0in]{eos_mls_logo_onpink}}

\begin{document}
\sloppy
%
% -----------------------------------------------------------------------------
%
\begin{frame}
 \titlepage
\end{frame}
%
% -----------------------------------------------------------------------------
%
\newframe{The problem to be solved}

Given

\begin{itemize}

\item A three-dimensional coherent and stacked (in pressure) QTM, i.e.,
  vertices are on vertical profiles, and vertices at each level are on the
  same constant-pressure surface, in every profile,

\item The geodetic surface co\"ordinates $(\lambda,\phi)$ of each vertex,

\item The geodetic height at every level on every profile, computed from a
  temperature profile assuming hydrostatic equilibrium, and

\item A line of sight,

\end{itemize}

Compute

\begin{itemize}

\item The three-dimensional geodetic co\"ordinates $(\lambda,\phi,h)$ of
  all points of intersection of the line of sight with boundary surfaces
  of the QTM, and

\item The interpolation coefficients from vertices of the QTM to those
  points.

\end{itemize}

\end{frame}
%
% -----------------------------------------------------------------------------
%
\newframe{Line of sight}

The line of sight consists of two parts:

\begin{itemize}

\item From the instrument to the tangent or reflection point, and

\item From the tangent or reflection point to the other side of the
atmosphere.

\end{itemize}

\end{frame}
%
% -----------------------------------------------------------------------------
%
\newframe{Line of sight}

A line is represented by a vector to a point on the line, and a multiple
of an unit vector that is parallel to the line:
%
\begin{equation}
\mathcal{L} = \mathbf{C} + s\, \mathbf{U}
\end{equation}
%
For the part of the line from the instrument to the tangent or reflection
point, $\mathbf{C}$ is interpolated from {\tt ScECR} (MIF quantity),
$\mathbf{U}$ is interpolated from the third column of {\tt ECRtoFOV} (MIF
quantity), and $s$ is path length in meters from the instrument.

For the part of the line from the tangent or reflection point to the other
side of the atmosphere,
%
\begin{equation}\begin{array}{ll}
\mathcal{L} = \mathbf{T} + s\, \mathbf{U} \text{ (no reflection) } &
\mathcal{L} = \mathbf{T} + s\, \mathbf{R} \text{ (reflection) } \\
\end{array}
\end{equation}
%
where $\mathbf{T}$ is the tangent or reflection point, the unit vector
$\mathbf{R}$ in the direction of reflection is calculated from
$\mathbf{U}$ as described in \h{wvs-133}, and $s$ is path length in
meters from the tangent or reflection point.

\end{frame}
%
% -----------------------------------------------------------------------------
%
\newframe{Line of sight reflected vector}

Assume that $\mathbf{N}$ is an unit vector that is normal to the Earth
surface at the point where the line of sight reflects.  Let $\mathbf{R}$
be the unit vector in the direction of the reflection.  The projections of
$\mathbf{U}$ and $\mathbf{R}$ into the tangent plane are the same, and
their projections into the normal are reversed.  Therefore, as derived in
\h{wvs-133},

\begin{equation}
\mathbf{R} = \mathbf{U} - 2 (\mathbf{U}\cdot\mathbf{N})\,\mathbf{U}
\end{equation}

\begin{centering}
\includegraphics[scale=0.6,clip=true]{wvs-160-2}\\
\end{centering}

\end{frame}
%
% -----------------------------------------------------------------------------
%
\newframe{Line of sight}

The line of sight is calculated by {\tt Get\_Lines\_of\_Sight} in the
module {\tt Interpolate\_MIF\_to\_Tan\_Press\_m} using

\begin{itemize}

\item The instrument position, taken from {\tt ScECR}, a minor-frame
quantity, and

\item The field-of-view vector, taken from the third column of the {\tt
ECRtoFOV} matrix, a minor-frame quantity,

\end{itemize}

by interpolating from MIF tangent pressures to a tangent pressure taken
from the temperature {\tt vGrid}.

\end{frame}
%
% -----------------------------------------------------------------------------
%
\newframe{Line of sight}

\begin{centering}
\hspace*{-0.5in}
\includegraphics[width=4.5in,keepaspectratio,clip=true]{wvs-160-1}\\
$R^\oplus_{\text{eq}}$ is from the center of the equivalent circular Earth.\\
{\tt SCECR} is from the center of the true Earth.\\
\end{centering}

\end{frame}
%
% -----------------------------------------------------------------------------
%
\newframe{Difference from EMLS}

\begin{itemize}

\item In 2-D, the reflected line of sight reflects from the pressure
reference surface.

\item In QTM, the reflected line of sight reflects from the Earth surface.

\end{itemize}

\begin{itemize}

\item In 2-D, the tangent or reflection point is provided by Level 1.

\item In QTM, the tangent or reflection point is calculated in Level 2.

\end{itemize}

\end{frame}
%
% -----------------------------------------------------------------------------
%
\newframe{Tangent or reflection point}

The tangent or reflection point is the point where the line of sight
either is nearest to the surface of the Earth, or intersects the surface
of the Earth.

Write the line of sight as $\mathbf{p}(s) = \mathbf{C} + s\,\mathbf{U}$
and substitute into the equation for an ellipsoid

\begin{equation}
\left( \mathbf{M}\, \mathbf{p}(s) \right)^T \cdot
 \left( \mathbf{M}\, \mathbf{p}(s) \right) = r^2 \text{ where }
\mathbf{M} = \text{diag} \left[ 1, 1, \frac1{\sqrt{1-e^2}} \right] \,.
\end{equation}

If $r = a$, this is the equation for the surface of the Earth. Otherwise,
it is the equation for an ellipsoid confocal with the Earth ellipsoid (see
\h{wvs-030}).

\end{frame}
%
% -----------------------------------------------------------------------------
%
\newframe{Tangent or reflection point}

After substituting $\mathbf{p}(s)$ into the equation for the ellipsoid,
write the result as a polynomial in $s$, \emph{viz}.
%
\begin{equation}\label{s-poly}
(\mathbf{M}\, \mathbf{U})^T \cdot (\mathbf{M}\, \mathbf{U})\, s^2 +
2 (\mathbf{M} \mathbf{C} )^T \cdot (\mathbf{M}\, \mathbf{U})\, s +
(\mathbf{M} \mathbf{C} )^T \cdot (\mathbf{M} \mathbf{C} ) = r^2 \,.
\end{equation}

This equation has one solution for $s$ if and only if its discriminant is
zero, i.e.,
%
\begin{equation}\label{discriminant}
\left( 2 (\mathbf{M} \mathbf{C} )^T \cdot (\mathbf{M}\, \mathbf{U}) \right)^2
- 4 (\mathbf{M}\, \mathbf{U})^T \cdot (\mathbf{M}\, \mathbf{U})\, 
\left[ (\mathbf{M} \mathbf{C} )^T \cdot (\mathbf{M} \mathbf{C} ) - r^2 \right]
 = 0 \,.
\end{equation}

Assuming the discriminant is zero, the solution of Equation (\ref{s-poly})
is
%
\begin{equation}\label{tv2}
s =
-\frac{\left(\mathbf{M}\mathbf{C}\right)^T \cdot
       \left(\mathbf{M}\mathbf{U}\right)}
      {\left(\mathbf{M}\mathbf{U}\right)^T \cdot
       \left(\mathbf{M}\mathbf{U}\right)} \,.
\end{equation}

\end{frame}
%
% -----------------------------------------------------------------------------
%
\newframe{Tangent or reflection point}

The value of $r$ for which the discriminant, Equation
(\ref{discriminant}), is zero is
%
\begin{equation}\label{discriminant-solution-oblate}
r^2 = \frac{ \left( \mathbf{M}^{-1} \mathbf{N} \right)^T \cdot
             \left( \mathbf{M}^{-1} \mathbf{N} \right) }
           { (1-e^2) \left( \mathbf{M}\mathbf{U} \right )^T \cdot
                 \left( \mathbf{M}\mathbf{U} \right ) }
    = \frac{ | \mathbf{M}^{-1} \mathbf{N} |^2}
           { (1-e^2) | \mathbf{M}\mathbf{U} |^2 }
    = \left( \frac{a}{c} \, \frac{ | \mathbf{M}^{-1} \mathbf{N} | }
                        { | \mathbf{M} \mathbf{U} | } \right)^2 \,,
\end{equation}

where $\mathbf{N} = \mathbf{C} \times \mathbf{U}$ and we used the identity
%
\begin{equation}\begin{split}\label{cross-norm}
| \mathbf{a} \times \mathbf{b} |^2
 = \,& |\mathbf{a}|^2 \, |\mathbf{b}|^2 \sin^2\theta
 = |\mathbf{a}|^2 \, |\mathbf{b}|^2 (1 - \cos^2\theta ) \\
 = \,& |\mathbf{a}|^2 \, |\mathbf{b}|^2 - (\mathbf{a}\cdot\mathbf{b})^2 \,.
\end{split}\end{equation}

If $r \geq a$, the point on $\mathbf{p}(s)$ where $s$ is the solution of
Equation (\ref{s-poly}) is the tangent point; otherwise, the line of sight
intersects the Earth surface at the points that are solutions for $s$ in
Equation (\ref{s-poly}) with $r=a$.

\end{frame}
%
% -----------------------------------------------------------------------------
%
\newframe{Heights and hydrostatic equilibrium}

The heights at specified pressures, on profiles at vertices adjacent to
the line of sight, i.e., vertices of facets under the line of sight, are
calculated from temperature profiles using hydrostatic
equilibrium.\\[10pt]

The heights are needed not only corresponding to the specified {\tt
vGrid}, but also on the intermediate pressure grid corresponding to the
points needed for Gauss-Legende quadrature.\\[10pt]

Quadratures are calculated with $\zeta = -\log_{10} P$, not path length,
as the variable of integration.\\[10pt]

Heights are not needed on profiles that are not adjacent to the line of
sight.

\end{frame}
%
% -----------------------------------------------------------------------------
%
\newframe{Lines of sight}

The lines of sight for all the tangent pressures are determined.\\[5pt]

The set of facets under each line of sight, and the vertices of those
facets, are recorded in the variable {\tt F\_And\_V} in the full forward
model.

\begin{centering}
\includegraphics[scale=0.5,keepaspectratio]{wvs-160-facets}\\
\end{centering}

\end{frame}
%
% -----------------------------------------------------------------------------
%
\newframe{Lines of sight}

\begin{itemize}

\item The union of the sets of facets under the lines of sight is
  determined.

\item If the number of vertices in the union is not greater than twice the
  maximum number of vertices adjacent to any one line of sight, the union
  is used instead of the individual lines of sight, and heights are
  computed using hydrostatic equilibrium only once, at profiles on the
  vertices in the union, before the radiative-transfer calculation is made
  along any line of sight.

\item Otherwise, heights are computed using hydrostatic equilibrium
  separately, at profiles on vertices adjacent to each line of sight,
  before the radiative-transfer calculation is made along each line of
  sight.  (There is an opportunity, not yet exploited, to keep track of
  the vertices at which heights have been calculated, and use those
  results instead of computing them anew. Profiling will reveal whether
  this is important.)

\end{itemize}

\end{frame}
%
% -----------------------------------------------------------------------------
%
\newframe{QTM surfaces}

Every prism of the QTM is bounded by five surfaces with which the line of
sight might intersect:

\begin{itemize}

\item Two roughly-horizontal surfaces of constant pressure,

\item One vertical surface, a cone of constant latitude, and

\item Two vertical surfaces, planes defined by a vertical line that is at
  the polar QTM vertex and a QTM vertex that is not the polar vertex.

\end{itemize}

\end{frame}
%
% -----------------------------------------------------------------------------
%
\newframe{QTM surfaces -- surface of constant pressure}


\begin{minipage}{0.40\textwidth}
\begin{centering}
\includegraphics[scale=.65,keepaspectratio]{wvs-160-surf}\\
\end{centering}
\end{minipage}
%
\begin{minipage}{0.58\textwidth}
The height of a surface of constant pressure within one QTM facet is
modeled as the surface of a sphere that includes the three vertices
$(\mathbf{P}_1,\mathbf{P}_2,\mathbf{P}_3)$ at one pressure level (not
necessarily at the same heights) above a facet of the QTM.  Its radius is
the radius of mean curvature of the surface of the Earth at the geodetic
$(\lambda,\phi)$ surface co\"ordinates of the circumcenter $\mathbf{C}$ of
the triangle, plus the average of the heights at
$(\mathbf{P}_1,\mathbf{P}_2,\mathbf{P}_3)$.

Alternatively, the height at $\mathbf{C}$ could be computed using
barycentric interpolation.
\end{minipage}

\end{frame}
%
% -----------------------------------------------------------------------------
%
\newframe{Digression: Barycentric interpolation in 3-D}

Section \ref{128-Barycentric} in \h{wvs-128} describes how to compute
barycentric interpolation coefficients within a 2-D triangle.

Let $\eta_i$ be the ratio of the area of the triangle $(
\mathbf{P}_j,\mathbf{P}_k,\mathbf{X} )$ to the area of the entire
triangle, with $i \neq j \neq k$.  Then the interpolated value at
$\mathbf{X}$ is $f(\mathbf{X}) = \sum_i \eta_i
\, f ( \mathbf{P}_i )$.

The area of a triangle can be expressed using two edge lengths and the
angle between them.  Let $\mathbf{e}_{ij} = \mathbf{P}_j - \mathbf{P}_i$
be the vector along the edge of the triangle from $\mathbf{P}_i$ to
$\mathbf{P}_j$, and let $\mathbf{e}_{xi} = \mathbf{X} - \mathbf{P}_i$. 
Then twice the area is

\begin{equation}\begin{split}
A = \,& |\mathbf{e}_{ij}| \, |\mathbf{e}_{ik}| \sin\theta =
| \mathbf{e}_{ij} \times \mathbf{e}_{ik} | \text{ and} \\
\eta_i = \,& \frac{ | \mathbf{e}_{xj} \times \mathbf{e}_{xk} | }A \,. \\
\end{split}\end{equation}

(Notice that the factor of 2 cancels.)

\end{frame}
%
% -----------------------------------------------------------------------------
%
\newframe{QTM surfaces -- surface of constant pressure}

The sphere at one level is constructed in three steps:

\begin{enumerate}

\item Compute the circumcenter of the triangle
  $(\mathbf{P}_1,\mathbf{P}_2,\mathbf{P}_3)$ in the plane of the surface
  of the facet, and compute the normal vector to that plane.

\item Compute the mean radius of curvature $R_\text{eq}^\oplus$ of the
  surface of the Earth at the geodetic $(\lambda,\phi)$ surface
  co\"ordinates of the circumcenter of the facet.  This is the inverse of
  the average of meridional curvature and curvature in the prime vertical
  (\h{wvs-146}).

\item Compute the center of the sphere passing through
  $(\mathbf{P}_1,\mathbf{P}_2,\mathbf{P}_3)$ with radius of curvature
  $R_\text{eq}^\oplus$ + the average of the heights of $\mathbf{P}_1$,
  $\mathbf{P}_2$, and $\mathbf{P}_3$.

\end{enumerate}

(see \h{wvs-132}).

\end{frame}
%
% -----------------------------------------------------------------------------
%
\newframe{QTM surfaces -- circumcenter}

The circumcenter $\mathbf{v}$ of a triangle is at the intersection of the
perpendicular bisectors of the sides.

Given the ECR co\"ordinates $(\mathbf{P}_1,\mathbf{P}_2,\mathbf{P}_3)$ of
the three vertices of a surface of constant pressure in a facet of a QTM,
the conditions that the bisectors are perpendicular are

\begin{equation}\label{bisectors}
\left(\mathbf{v} - \frac12(\mathbf{P}_i + \mathbf{P}_j) \right) \cdot
 ( \mathbf{P}_i - \mathbf{P}_j ) = 0
\end{equation}

for $i,j\in\{1,2,3\}$ and $i \neq j$.  These three conditions are not
independent, so a further condition is required: The vector from a vertex,
say $\mathbf{P}_3$, to $\mathbf{v}$, is in the plane of
$(\mathbf{P}_1,\mathbf{P}_2,\mathbf{P}_3)$, \emph{viz.}

\begin{equation}\label{in-plane}
( \mathbf{v} - \mathbf{P}_3 ) \cdot (( \mathbf{P}_1 - \mathbf{P_3} ) \times
 ( \mathbf{P}_2 - \mathbf{P}_3 )) = 0
\end{equation}

\end{frame}
%
% -----------------------------------------------------------------------------
%
\newframe{QTM surfaces -- circumcenter}

Given the ECR co\"ordinates $(\mathbf{P}_1,\mathbf{P}_2,\mathbf{P}_3)$ of
the three vertices of a surface of constant pressure in a facet of a QTM,
compute vectors for two edges, and the normal to the plane, \emph{viz}.

\begin{equation}\begin{split}
\mathbf{A} = \,& \mathbf{P}_1 - \mathbf{P}_3 \\
\mathbf{B} = \,& \mathbf{P}_2 - \mathbf{P}_3 \\
\mathbf{N} = \,& \mathbf{A} \times \mathbf{B}
            \text{ (normal to plane)} \\
\end{split}\end{equation}

In matrix form, Equations (\ref{bisectors}) and (\ref{in-plane}) are

\begin{equation}
%\arraystretch{1.25}
\mathbf{M} \, \mathbf{v} =
\left[ \begin{array}{lll} n_1 & n_2 & n_3 \\[3pt]
                          a_1 & a_2 & a_3 \\[3pt]
                          b_1 & b_2 & b_3 \\
\end{array} \right] \left[ \begin{array}{l} v_1 \\[3pt] v_2 \\[3pt] v_3 \\
\end{array} \right] = \left[ \begin{array}{l}
 0 \\[3pt] \frac12 \mathbf{A} \cdot \mathbf{A} \\[3pt]
           \frac12 \mathbf{B} \cdot \mathbf{B}
\end{array} \right]
\end{equation}

\end{frame}
%
% -----------------------------------------------------------------------------
%
\newframe{QTM surfaces -- circumcenter}

Notice that $\text{det}\, (\mathbf{M}) = \mathbf{N} \cdot \mathbf{N} =
|\mathbf{N}|^2$, and the numerator of each row of $\mathbf{M}^{-T}$ is the
cross product of the other two rows of $\mathbf{M}$. In particular, the
numerator of column 1 of $\mathbf{M}^{-1} = \mathbf{N}$.

The solution for $\mathbf{v}$ (vector from $\mathbf{P}_3$ to
circumcenter $\mathbf{C}$) is

\begin{equation}\begin{split}
\mathbf{v} = \,& \frac{\mathbf{D} \times \mathbf{N}}
                      { 2 |\mathbf{N}|^2 } \text{ where}\\
\mathbf{D} = \,& ( \mathbf{A} \cdot \mathbf{A} ) \mathbf{B} -
                 ( \mathbf{B} \cdot \mathbf{B} ) \mathbf{A} \\
\end{split}\end{equation}

\end{frame}
%
% -----------------------------------------------------------------------------
%
\newframe{QTM surfaces -- constant pressure surface}

The center $\mathbf{p}(t)$ of the sphere lies upon the line from the
circumcenter $\mathbf{C} = \mathbf{p}(0) = \mathbf{P}_3 + \mathbf{v}$ that
is normal to the plane defined by
$(\mathbf{P}_1,\mathbf{P}_2,\mathbf{P}_3)$, at a distance $t$, to be
computed, from that plane:
%
\begin{equation}
\mathbf{p}(t) = \mathbf{p}(0) + t\, \mathbf{N} = \mathbf{P}_3 + \mathbf{v}
 + t\, \mathbf{N} \,.
\end{equation}

The distance from any vertex, say $\mathbf{P}_3$, to the center of the
sphere is $r$.  Therefore
%
\begin{equation}
| \mathbf{p}(t) - \mathbf{P}_3 |^2 = | \mathbf{v} + t\, \mathbf{N} |^2 = |
\mathbf{v} |^2 + 2\,t\, \mathbf{v} \cdot \mathbf{N} + t^2\,
| \mathbf{N} |^2 = r^2 \,.
\end{equation}

Because $\mathbf{v}$ is perpendicular to $\mathbf{N}$ (by definition),
$\mathbf{v} \cdot \mathbf{N} = 0$, and
%
\begin{equation}
t = \pm \sqrt{\frac{r^2 - |\mathbf{v}|^2}{|\mathbf{N}|^2}}
\end{equation}

with the sign chosen to minimize $|\mathbf{p}(t)|$, i.e., to make
$\mathbf{p}(t)$ nearest to the center of the Earth.

\end{frame}
%
% -----------------------------------------------------------------------------
%
\newframe{QTM surfaces -- constant pressure surface}

The radius of curvature at the Earth surface is the inverse of the mean
curvature, which is the average of the principal curvatures, (Section
\ref{146-Mean} in \h{wvs-146}):
%
\begin{equation}\begin{split}
R_\text{eq}^\oplus = \,&
 \frac1H\, \text{, where } H = \frac12(\kappa_1 + \kappa_2) \,,
 \\
\kappa_2 = \,& R_N = \frac{a}{\sqrt{1-e^2 \sin^2 \phi}} \,, \\
\kappa_1 = \,& R_M = \frac{1-e^2}{a^2} R_N^3 =
 \frac{a(1-e^2)}{(1 - e^2 \sin^2 \phi )^\frac32} \\
\end{split}\end{equation}
%
(Equations (\ref{146-RN}) and (\ref{146-RM}) in \h{wvs-146}),
$R_M$ is the meridional curvature, $R_N$ is the curvature in the
prime vertical section, $e = \sqrt{1-\frac{c^2}{a^2}}$ is eccentricity,
$a$ is the equatorial radius of the Earth, $c$ is the polar radius, and
$\phi$ is geodetic latitude.

\end{frame}
%
% -----------------------------------------------------------------------------
%
\newframe{Difference from EMLS}

In EMLS, the line of sight might intersect

\begin{itemize}

\item A roughly-horizontal surface of constant pressure (actually a
curve), and

\item A vertical line of constant geodetic orbit angle $\phi$.

\end{itemize}

EMLS {\tt Height\_Metrics} in {\tt metrics\_m} calculates the height and
orbit geodetic angle only at intersections of the line of sight with the
constant-pressure curve (\h{wvs-048}).  The height between the points of
constant pressure on adjacent profiles is approximated by a straight line,
not a segment of a circle. Intersections with the profiles are not
determined.\\[10pt]

Although {\tt Metrics\_3D\_QTM} in {\tt Metrics\_3D\_m} can calculate
intersections with all three surfaces, for consistency with EMLS, it is
referenced from {\tt FullForwardModel} in such a way as to calculate only
intersections with the constant-pressure surface.

\end{frame}
%
% -----------------------------------------------------------------------------
%
\newframe{Intersection with spherical surface of constant pressure}

The equation for a sphere with its center at the origin, in vector
notation, is $\mathbf{p} \cdot \mathbf{p} = r^2$.  Representing a line in
the usual way

\begin{equation}\label{line-for-sphere}
\mathbf{p}(s) = \mathcal{L}(s) = \mathbf{C} + s \, \mathbf{U}
\end{equation}

and substituting into the equation for a sphere gives an equation for $s$,
\emph{viz}.

\begin{equation}\label{s-sphere}
\mathbf{U} \cdot \mathbf{U} \, s^2 + 2\, \mathbf{C} \cdot \mathbf{U}\, s +
\mathbf{C} \cdot \mathbf{C} = r^2
\end{equation}

(Equation (\ref{131-sphere}) in \h{wvs-131}).

If there are no real solutions for $s$, the line does not intersect the
sphere. If there is one solution, the line is tangent to the sphere. 
Otherwise, it intersects the sphere twice.

\end{frame}
%
% -----------------------------------------------------------------------------
%
\newframe{Intersection with spherical surface of constant pressure}

For every facet in {\tt F\_And\_V\%Facets}

\begin{enumerate}

\item Compute the spherical surfaces of constant pressure.

\item Solve for $s$ in Equation (\ref{s-sphere}) and evaluate
      $\mathbf{p}(s)$ using Equation (\ref{line-for-sphere}).

\item Compute the centroid {\tt cc =} $\frac13 ( \mathbf{P}_1 +
      \mathbf{P}_2 + \mathbf{P}_3 )$ of the triangle bounded by $(
      \mathbf{P}_1 + \mathbf{P}_2 + \mathbf{P}_3 )$.

\item Compute the geodetic surface co\"ordinates $(\lambda,\phi)$ of {\tt
      cc}.

\item Determine the facet in which the point $(\lambda,\phi)$ appears
      using {\tt QTM\_Tree\%Find\_Facet(cc)}.

\item Discard the intersection if it's not within the facet being
      considered.

\end{enumerate}

We use the centroid to find the facet instead of the circumcenter
previously calculated because in ZOT coordinates the facet at the surface
is a right triangle, and therefore the circumcenter is on the midpoint of
the hypotenuse. The facet would be ambiguous.

\end{frame}
%
% -----------------------------------------------------------------------------
%
\newframe{Interpolation in spherical surface of constant pressure}

Interpolation coefficients within the surface of constant pressure are
computed using barycentric interpolation within the spherical triangle
bounded by $( \mathbf{P}_1,\mathbf{P}_2,\mathbf{P}_3 )$.

If the coefficients are computed using barycentric interpolation within
the planar triangle $( \mathbf{P}_1,\mathbf{P}_2,\mathbf{P}_3 )$, or
within the triangle projected to the plane tangent at the intersection,
distortions result, and either unitarity or linear precision is lost.

To interpolate to $\mathbf{X}$ using planar barycentric interpolation, the
coefficient associated with $\mathbf{P}_i$ is the ratio of the area of the
triangle $( \mathbf{P}_j,\mathbf{P}_k,\mathbf{X} )$ to the area of the
entire triangle, with $i \neq j \neq k$.

Barycentric interpolation on the surface of a sphere requires computing
the areas of spherical triangles. This is a very expensive calculation
(see \h{wvs-161}).

\end{frame}
%
% -----------------------------------------------------------------------------
%
\newframe{Complication using spherical surface of constant pressure}

{\parskip 8pt
Assuming facets in which the distance from the polar vertex to the edge at
constant latitude is about $1.5^\circ$, the maximum distance from the
planar triangle $(\mathbf{P}_1,\mathbf{P}_2,\mathbf{P}_3)$ to the spherical
triangle is about 0.56 km.

Computing the intersection of the line of sight with a plane (slide
\pageref{plane-slide}) is simpler than computing the sphere of constant
pressure and computing the intersection of the line sight with that
sphere.

The line of sight might intersect the sphere of constant pressure at two
points within the facet. This would introduce some complication in the
full forward model because the extents of arrays representing quantities
on the line of sight depend upon the number of pressure surfaces.

Therefore it might be preferable to compute the intersection of the line
of sight with the planar triangle
$(\mathbf{P}_1,\mathbf{P}_2,\mathbf{P}_3)$.
}
\end{frame}
%
% -----------------------------------------------------------------------------
%
\newframe{Intersection with cone of constant latitude}

In ECR, the surface of a cone with its vertex on the Earth axis at a
distance $v$ northward $(+z)$ from the center, and its axis parallel to
the Earth axis, is given by

\begin{equation}\label{cone}
(\mathbf{X}-\mathbf{V})^T \mathbf{M} (\mathbf{X}-\mathbf{V}) = 0 \,,
\end{equation}

where $\mathbf{V} = [ 0,0,v ]^T$, $\mathbf{M} = \mathbf{D} \mathbf{D}^T
-\sin^2 \psi\, \mathbf{I}$, $\mathbf{D} = [ 0,0,1 ]^T$, $\psi$ is the
latitude (geocentric or geodetic) where the cone intersects the Earth
surface ($90^\circ - \psi$ is the cone angle), and $\mathbf{I}$ is the
identity matrix. (See \h{wvs-129}.)

\end{frame}
%
% -----------------------------------------------------------------------------
%
\newframe{Intersection with cone of constant latitude}

Define a line in the usual way by
%
\begin{equation}\label{line-for-cone}
\mathbf{X} = \mathbf{C} + s\, \mathbf{U}
\end{equation}

where $\mathbf{C}$ is a vector to a point on the line and $\mathbf{U}$ is a
vector along the line, and substitute into Equation (\ref{cone}) giving
%
\begin{equation}\begin{split}\label{line-cone}
&
( \mathbf{C} + s\, \mathbf{U} - \mathbf{V} )^T \mathbf{M} \,
( \mathbf{C} + s\, \mathbf{U} - \mathbf{V} ) = 0\,\, \text{ or} \\
&
\mathbf{U}^T \mathbf{M} \mathbf{U} \,s^2 +
2 \mathbf{U}^T \mathbf{M}\, ( \mathbf{C} - \mathbf{V} ) \,s +
(\mathbf{C} - \mathbf{V})^T \mathbf{M}\, (\mathbf{C} - \mathbf{V}) = 0
\,.
\end{split}\end{equation}

The points on the line where it intersects the cone are given by values
of Equation (\ref{line-for-cone}), with values of $s$ that are roots of
Equation (\ref{line-cone}).

If $\psi$ is geocentric latitude $\gamma$ then $v=0$.  If $\psi$ is
geodetic latitude $\phi$ then (Equation (\ref{146-zeta}) in \h{wvs-146})
%
\begin{equation}
v = -\frac{a\, e^2 \sin\phi}{\sqrt{1-e^2 \sin^2 \phi}}\,.
\end{equation}

\end{frame}
%
% -----------------------------------------------------------------------------
%
\newframe{Intersection with cone of constant latitude}

For each $s$ that is a root of Equation (\ref{line-cone})

\newcounter{enumTemp}
\begin{enumerate}

\item If $s$ is not within the range of the line of sight, discard the
      intersection.

\item Determine the facet in which $\mathbf{X}$ from Equation
      (\ref{line-for-cone}) appears.

\item If the facet is not a leaf facet (facet at a leaf of the QTM
      quadtree), discard the intersection (it's not within the polygon).

\setcounter{enumTemp}{\theenumi}

\end{enumerate}

\end{frame}
%
% -----------------------------------------------------------------------------
%
\newframe{Intersection with cone of constant latitude}

\begin{minipage}{0.56\textwidth}
\begin{centering}
\includegraphics[scale=0.7]{wvs-160-cone}\\[5pt]
$\mathbf{X} = \mathbf{C} + s\, \mathbf{U}$\\
\end{centering}
\end{minipage}
%
\begin{minipage}{0.42\textwidth}
\begin{enumerate}
\small
\setcounter{enumi}{\theenumTemp}

\item\label{heights-cone} Interpolate to the maximum and minimum heights,
      at the longitude of $\mathbf{X}$, for minimum and maximum levels in
      the facet.

\item If the height of the intersection is not between the interpolated
      heights, discard the intersection.

\item Repeat step \ref{heights-cone} for every pair of levels to determine
      the levels between which $\mathbf{X}$ appears.

\end{enumerate}
\end{minipage}

\end{frame}
%
% -----------------------------------------------------------------------------
%
\newframe{Intersection with vertical plane}\label{plane-slide}

The equation of a plane can be expressed in vector notation as the set of
points $\mathbf{p}$ such that (see \h{wvs-130})

\begin{equation}\label{plane}
(\mathbf{p} - \mathbf{p}_0) \cdot \mathbf{n} = 0\,,
\end{equation}

where $\mathbf{p}_0$ is a point on the plane, and $\mathbf{n}$ is a normal
vector to it.  A line can be expressed as the set of points

\begin{equation}\label{line-for-plane}
\mathbf{X} = \mathbf{C} + s\, \mathbf{U}\,.
\end{equation}

Substituting $\mathbf{p} = \mathbf{X}$ into Equation (\ref{plane}) gives

\begin{equation}\begin{split}\label{line-plane}
&
(\mathbf{C} + s\, \mathbf{U} - \mathbf{p}_0) \cdot \mathbf{n} = 0
\,\,\text{ or}\\
&
s\, \mathbf{U} \cdot \mathbf{n} + ( \mathbf{C} - \mathbf{p}_0) \cdot
\mathbf{n} = 0 \\
\end{split}\end{equation}

\end{frame}
%
% -----------------------------------------------------------------------------
%
\newframe{Intersection with vertical plane}

If $\mathbf{U} \cdot \mathbf{n} = 0$ the line and plane are parallel. 
Further, if $\mathbf{C} = \mathbf{p}_0$, choose a different
$\mathbf{p}_0$; then, if $(\mathbf{p}_0 - \mathbf{C} ) \cdot \mathbf{n} =
0$, the line is in the plane, and any point will do. Otherwise there is no
intersection.

If $\mathbf{U} \cdot \mathbf{n} \neq 0$ the line intersects the plane at
$\mathbf{C} + s\, \mathbf{U}$, where solving for $s$ in Equation
(\ref{line-plane}) gives

\begin{equation}\label{line-intersects-plane}
s = \frac{(\mathbf{p}_0 - \mathbf{C} ) \cdot \mathbf{n}}
         {\mathbf{U} \cdot \mathbf{n}} \,.
\end{equation}

\end{frame}
%
% -----------------------------------------------------------------------------
%
\newframe{Intersection with vertical plane}

For every facet, and every range between two adjacent constant pressure
surfaces, and both vertical planes that include the profile at the polar
vertex

\begin{enumerate}

\item Compute the ECR co\"ordinates of the four corners of the plane face
      of the height range of the facet.

\item Compute the intersection using Equation
      (\ref{line-intersects-plane}).

\item If the latitude of the intersection is outside the latitude range of
      the face, discard the intersection.
\setcounter{enumTemp}{\theenumi}

\end{enumerate}

\end{frame}
%
% -----------------------------------------------------------------------------
%
\newframe{Intersection with vertical plane}

\begin{minipage}{0.56\textwidth}
\begin{centering}
\includegraphics[scale=0.6]{wvs-160-plane}\\[5pt]
$\mathbf{X} = \mathbf{C} + s\, \mathbf{U}$\\
\end{centering}
\end{minipage}
%
\begin{minipage}{0.42\textwidth}
\small
\begin{enumerate}
\setcounter{enumi}{\theenumTemp}

\item\label{heights-plane} Interpolate in the heights of the top and
      bottom boundaries of the plane to the latitude of the intersection.

\item If the height of intersection is not within the range of
      interpolated heights, discard the intersection.

\item Repeat step \ref{heights-plane} for every pair of levels to determine
      the levels between which $\mathbf{X}$ appears.

\end{enumerate}
\end{minipage}

\end{frame}
%
% -----------------------------------------------------------------------------
%
\newframe{Intersection with extrapolated surface of constant height}

After intersections with horizontal, conical, and vertical plane faces of
QTM facets are computed, the intersections are sorted on $s$.  Parts of
the line of sight that are outside the QTM might intersect surfaces of
constant height outside the QTM.

If the intersection with the face at the edge of the QTM is a vertical
plane face or a conical face, compute two horizontal interpolation
coefficients to the point of intersection.  Otherwise, it is a surface of
constant pressure; compute three interpolation coefficients.

For each pressure surface index in the {\tt vGrid}, interpolate height
horizontally on the edges of facets (if there are two interpolation
coefficients), or within the facet if there are three interpolation
coefficients.

Compute the point where the line of sight intersects a surface at the
interpolated height above the surface of the Earth, using the method
described in \h{wvs-134}.

\end{frame}
%
% -----------------------------------------------------------------------------
%
\newframe{Finally (Whew!)}

\begin{centering}
Sort the intersections on $s$.\\
\end{centering}

\end{frame}
%
% ----------------------------------------------------------------------------
%
\end{document}

% $Log$
% Revision 1.3  2020/06/10 21:00:58  pwagner
% Builds OK now
%
% Revision 1.2  2020/06/10 00:35:37  vsnyder
% Correct a typo -- already
%
% Revision 1.1  2020/06/10 00:18:18  vsnyder
% Initial commit
%
