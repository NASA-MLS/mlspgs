\documentclass[11pt]{article}
\usepackage{alltt}
\usepackage[fleqn]{amsmath}
\usepackage{floatflt}
\usepackage{graphicx}
\usepackage{longtable}
\usepackage[strings]{underscore}

\textwidth 6.5in
\oddsidemargin -0.25in
%\evensidemargin -0.5in
\topmargin -0.5in
\textheight 9in

\newcommand{\docname}{wvs-134}
\newcommand{\docdate}{15 March 2016}

\ifx\pdfoutput\undefined
  \pdfoutput=0
  \usepackage[hypertex,plainpages,hyperindex=true]{hyperref}
  \hypersetup{%
    hypertexnames=false%
  }
  % Specify the driver for the color package
  \ExecuteOptions{dvips}
  %\ExecuteOptions{xdvi}
\else
  \ifnum\pdfoutput>0
    \usepackage[pdftex,plainpages,hyperindex=true,pdfpagelabels]{hyperref}
    \hypersetup{%
      hypertexnames=false,%
      colorlinks=true,%
      linktocpage=true,%
    }
    % Specify the driver for the color package
    \ExecuteOptions{pdftex}
  \else
    \usepackage[hypertex,plainpages,hyperindex=true]{hyperref}
    \hypersetup{%
      hypertexnames=false%
    }
    % Specify the driver for the color package
    \ExecuteOptions{dvips}
    %\ExecuteOptions{xdvi}
  \fi
\fi

\hyperbaseurl{}
\newcommand\hr[1]{\href{#1.dvi}{dvi}, \href{#1.pdf}{pdf}}
\newcommand\h[1]{#1 (\hr{#1})}

\begin{document}

%\tracingcommands=1
\newlength{\hW} % heading box width
\newlength{\pW} % page number field width
\settowidth{\hW}{\bf\docname}
\settowidth{\pW}{Page \pageref{lastpage}\ of \pageref{lastpage}}
\ifdim \pW > \hW \setlength{\hW}{\pW} \fi
\makeatletter
\def\@biblabel#1{#1.}
\newcommand{\ps@twolines}{%
  \renewcommand{\@oddhead}{%
    \docdate\hfill\parbox[t]{\hW}{{\hfill\bf\docname}\newline
                          Page \thepage\ of \pageref{lastpage}}}%
\renewcommand{\@evenhead}{}%
\renewcommand{\@oddfoot}{}%
\renewcommand{\@evenfoot}{}%
}%
\makeatother
\pagestyle{twolines}

\vspace{-10pt}
\begin{tabbing}
\phantom{References: }\= \\
To: \>Van\\
Subject: \>Intersection of line with surface at constant height above Earth
reference ellipsoid\\
From: \>Van Snyder\\
Reference: \> wvs-030 \hr{wvs-030}, wvs-131 \hr{wvs-131} \\
\end{tabbing}

\parindent 0pt \parskip 6pt
\vspace{-20pt}

\section{Using derivatives in a Newton iteration}

The Cartesian co\"ordinates of points on a surface at height $h$ above the
Earth reference ellipsoid in the direction normal to it are given in terms
of geodetic co\"ordinates by the vector $\mathbf{V}_e = [ x, y, z ]^T$
where

\begin{equation}\begin{split}
x = \,& p \cos \phi \cos \lambda \,, \\
y = \,& p \cos \phi \sin \lambda \,, \\
z = \,& q \sin \phi \,, \\
\end{split}\end{equation}

$\phi$ is geodetic latitude, $\lambda$ is longitude,

\begin{equation}\begin{split}
p = \,& a^2 d + h \,, \\
q = \,& b^2 d + h \,,  \\
d = \,& \frac1{\sqrt{a^2 \cos^2 \phi + b^2 \sin^2 \phi}} \,, \\
\end{split}\end{equation}

$h$ is height, and $a$ and $b$ are the equatorial and polar radii of the
Earth, respectively.

The Cartesian co\"ordinates of points on a line are given by the vector
$\mathbf{V}_l(s) = \mathbf{V}_0 + s \mathbf{V}_1$, where $\mathbf{V}_0 =
[x_0, y_0, z_0]^T$ is a vector to a point on the line, $\mathbf{V}_1 =
[x_1, y_1, z_1]^T$ is a unit vector along the line, and $s$ is the
distance along the line from $\mathbf{V}_0$ in the direction
$\mathbf{V}_1$.

The intersection of the line with the surface is given by the solution
$\mathbf{S} = [ \phi, \lambda, s ]^T$ of the equation

\begin{equation}
\mathbf{F} = \mathbf{V}_e - \mathbf{V}_l(s) = \left[ \begin{array}{l}
p \cos \phi \cos \lambda - x_0 - x_1 s \\
p \cos \phi \sin \lambda - y_0 - y_1 s \\
q \sin \phi - z_0 - z_1 s \\
\end{array} \right] = 0 \,.
\end{equation}

A solution can be obtained by a Newton iteration

\begin{equation}
\mathbf{J}\, \delta \mathbf{S} = - \mathbf{F} \text{ where}
\end{equation}

\begin{equation}
\mathbf{J} = \left[ \begin{array}{ccc}
p^\prime \cos\phi \cos\lambda
 - p \sin\phi \cos\lambda & -p \cos\phi \sin\lambda & -x_1 \\
p^\prime \cos\phi \sin\lambda
 - p \sin\phi \sin\lambda & p \cos\phi \cos\lambda & -y_1 \\
q^\prime \sin\phi + q \cos\phi & 0 & -z_1 \\
\end{array} \right] \,,
\end{equation}

\begin{equation}\begin{split}
p^\prime = \,& a^2 d^\prime + h \\
q^\prime = \,& b^2 d^\prime + h \text{ and}\\
d^\prime = \,& \frac{\text{d}d}{\text{d}\phi} =
 \cos\phi \sin\phi\, (a^2-b^2)\, d^3 \,.
\end{split}\end{equation}

There can be zero, one, or two solutions. To allow for the possibility of
zero solutions, i.e., that the line does not intersect the surface, a
least-squares Newton iteration

\begin{equation}
\mathbf{J}^T \, \mathbf{J} \, \delta \mathbf{S} = - \mathbf{J}^T \,
\mathbf{F}
\end{equation}

can be used.  If $|\mathbf{F}|$ is large, there is no intersection.

Starting points can be produced by computing the intersections of the line
with the ellipsoid

\begin{equation}\label{eight}
\frac{x^2}{(a+h)^2} + \frac{y^2}{(a+h)^2} + \frac{z^2}{(b+h)^2} = 1 \,,
\end{equation}

which produces the exact solution at the poles and on the equator.
As described in wvs-131 \hr{wvs-131}, this involves solving a quadratic
equation

\begin{equation}\label{nine}
(\mathbf{M} \mathbf{V}_1)^T (\mathbf{M} \mathbf{V}_1) s^2 +
(\mathbf{M} \mathbf{V}_0)^T (\mathbf{M} \mathbf{V}_1) s +
(\mathbf{M} \mathbf{V}_0)^T (\mathbf{M} \mathbf{V}_0) = 0
\end{equation}

where

\begin{equation}
\mathbf{M} = \left[ \begin{array}{ccc}
\frac1{a+h} & 0 & 0 \\
0 & \frac1{a+h} & 0 \\
0 & 0 & \frac1{b+h} \\
\end{array} \right] \,.
\end{equation}

This solution produces only values of $s$, from which values of $\phi$ and
$\lambda$ can be determined by Fukushima's algorithm [1] to compute
geodetic co\"ordinates $[\phi, \lambda, h]^T$ from Cartesian
co\"ordinates.  If the value of $h$ produced by Fukushima's algorithm is
sufficiently close to the given value of $h$, the Newton iteration is not
necessary.

If there is no solution of Equation (\ref{nine}), the closest point to the
ellipsoid can be determined, as described in wvs-030 \hr{wvs-030}.  If the
distance at the closest point is greater than $h$, there is no
intersection between the line and the surface.  Using this desideratum,
the least-squares Newton iteration would not be necessary.

The reason to solve simultaneously for $\phi$, $\lambda$, and $s$, even
though $\phi$ and $\lambda$ can be computed from $\mathbf{V}_l(s)$ using
Fukushima's algorithm, is that we do not have a formula in Cartesian
co\"ordinates for the surface at the constant ellipsoidal height $h$, and
we do not have a formula in geodetic co\"ordinates for $\mathbf{V}_l(s)$.

\section{Without using derivatives}

An alternative iteration, that is simpler but might not converge as
quickly, consists of

\begin{enumerate}

\item\label{step1} Compute $s_0$, the value of $s$ at an intersection, if
      any, with the ellipsoid defined by Equation (\ref{eight}).

\item\label{step2} If step \ref{step1} does not produce an intersection,
      compute the value of $s_0$ at which the line defined by
      $\mathbf{V}_l(s)$ is nearest to the Earth reference ellipsoid using
      the method described in wvs-030 \hr{wvs-030}.  Use Fukushima's
      algorithm to compute $h_0$ at $s_0$.  If $h_0 > h$ there is no
      intersection; otherwise

{\bf for} $i = 0$, \dots

  \begin{enumerate}

  \item\label{step2a} Evaluate $\mathbf{V}_l(s_i)$.

  \item\label{step2b} Using Fukushima's algorithm, compute $\phi_i$,
        $\lambda_i$, and $h_i$ at $\mathbf{V}_l(s_i)$.

        Step \ref{step2a} and this step can be skipped on the first
        iteration if no intersection was found in step \ref{step1}; use
        the values computed in step \ref{step2} instead.

  \item If $|h-h_i| \leq h_{\text{tol}}$ {\bf exit}.

  \item\label{step2d} Compute
  %
        \begin{equation*}
        s_{i+1} = s_i + \frac{\delta h}
         {\text{sign}(\max(|\cos\alpha_i|, \frac12),\cos\alpha_i)}
        \end{equation*}
  %
        where $\cos\alpha_i = \mathbf{N}_i \cdot \mathbf{V}_1$, and
        $\mathbf{N}_i =[ \cos\phi_i \cos\lambda_i, \cos\phi_i
        \sin\lambda_i, \sin\phi_i ]^T$ is the unit normal to the ellipsoid
        at $\mathbf{V}_l(s_i)$.

  \end{enumerate}

\end{enumerate}

The subroutine DZERO from the Math77 library could be used for steps
\ref{step2a}--\ref{step2d}.  Instead of computing $s_i$ as in \ref{step2d}
above, DZERO specifies it.  DZERO needs two values of $s$ and $h$ to get
started.  To get DZERO started in the correct direction, compute $s_1$ as
in step \ref{step2d} above.

Fukushima's algorithm uses a Halley iteration.  One iteration suffices to
produce a value of $\phi_i$ that is accurate to within 30 micro arcseconds
if $h < 30,000$ km, so step \ref{step2b} is fast.

[1] Toshio Fukushima, \emph{Transformation from Cartesian to geodetic
coordinates accelerated by Halley's method}, {\bf J. Geod. 79}, 12
(February 2006) pp 689--693.  DOI 10.1007/s00190-006-0023-2.

\vspace{0.25in}
\includegraphics{wvs-134-1}

\label{lastpage}
\vspace*{-0.1in} % Somehow, this causes lastpage to be defined
\end{document}

% $Id$

% $Log$
