\documentclass[11pt]{article}

\usepackage{alltt}
\usepackage[fleqn]{amsmath}
\usepackage{floatflt}
\usepackage{graphicx}
\usepackage{longtable}
\usepackage[strings]{underscore}

\textwidth 6.5in
\oddsidemargin -0.25in
%\evensidemargin -0.5in
\topmargin -0.5in
\textheight 9in
\parindent 0pt
\parskip 6pt

\newcommand{\docname}{wvs-159}
\newcommand{\docdate}{18 March 2020}

\ifx\pdfoutput\undefined
  \pdfoutput=0
\fi
\ifnum\pdfoutput>0

  \usepackage[pdftex,plainpages,hyperindex=true,pdfpagelabels]{hyperref}
  \hypersetup{%
    hypertexnames=false,%
    colorlinks=true,%
    linktocpage=true,%
  }
  % Specify the driver for the color package
  \ExecuteOptions{pdftex}
\else
  \usepackage[hypertex,plainpages,hyperindex=true]{hyperref}
  \hypersetup{%
    hypertexnames=false%
  }
  % Specify the driver for the color package
  \ExecuteOptions{dvips}
  %\ExecuteOptions{xdvi}
\fi

\hyperbaseurl{}
\newcommand\hr[1]{\href{#1.dvi}{dvi}, \href{#1.pdf}{pdf}}
\newcommand\h[1]{#1 (\hr{#1})}

\begin{document}

%\tracingcommands=1
\newlength{\hW} % heading box width
\newlength{\pW} % page number field width
\settowidth{\hW}{\bf\docname}
\settowidth{\pW}{Page \pageref{lastpage}\ of \pageref{lastpage}}
\ifdim \pW > \hW \setlength{\hW}{\pW} \fi
\makeatletter
\def\@biblabel#1{#1.}
\newcommand{\ps@twolines}{%
  \renewcommand{\@oddhead}{%
    \docdate\hfill\parbox[t]{\hW}{{\hfill\bf\docname}\newline
                          Page \thepage\ of \pageref{lastpage}}}%
\renewcommand{\@evenhead}{}%
\renewcommand{\@oddfoot}{}%
\renewcommand{\@evenfoot}{}%
}%
\makeatother
\pagestyle{twolines}

\renewcommand{\d}{\text{d}}
\newcommand{\T}{\mathcal{T}}

\vspace{-10pt}
\begin{tabbing}
\phantom{References: }\= \\
To: \>Nathaniel, Paul, Bill, Alyn, Mike\\
Subject: \>Converting Geocentric Cartesian (ECR) to Geodetic co\"ordinates\\
From: \>Van Snyder\\
References: \>\h{wvs-004} \h{wvs-146} \\
\end{tabbing}

\section*{Introduction}

There are a very large number of methods to convert geocentric Cartesian
(ECR) co\"ordinates to geodetic co\"ordinates, assuming an ellipsoid of
rotation, i.e., only two different axes, not three.

\section*{Converting Geodetic to Geocentric Cartesian co\"ordinates}

The equations to convert geodetic to geocentric Cartesian co\"ordinates are

\begin{equation}\label{one}\begin{split}
x = \,& ( N(\phi) + H ) \cos \lambda \cos \phi \\
 \,& \\
y = \,& ( N(\phi) + H ) \sin \lambda \cos \phi \\
 \,& \\
z =  \,& ( ( 1-e^2 ) N(\phi) + H ) \sin \phi =
  \left( \frac{b^2}{a^2} N(\phi) + H \right) \sin \phi \\
\end{split}\end{equation}

where $\lambda$ is longitude, $\phi$ is geodetic latitude, $H$ is geodetic
height above the ellipsoid, $e^2 = 1 - \frac{b^2}{a^2}$ is the
eccentricity, $a$ is the equatorial (semi-major) axis, $b$ is the polar
(semi-minor axis), $N(\phi) = \frac{a^2}d$ is the radius of curvature in
the prime normal section (the curve in the east-west plane that intersects
the Earth and contains the zenith -- see \h{wvs-146}), and $d^2 = a^2
\cos^2 \phi + b^2 \sin^2 \phi = a^2(1-e^2\sin^2\phi)$.

Determining the longitude from Cartesian co\"ordinates is easy: $\lambda =
\tan^{-1} \frac{y}x$.

\section*{Bowring's method}

Start by estimating the geodetic latitude
%
\begin{equation*}
\phi = \tan^{-1} \left(\frac{az}{bs} \right)
\end{equation*}

where $s = \sqrt{x^2+y^2}$.  Then iterate the following until $\phi$
converges:
%
\begin{equation*}\begin{split}
t = \,& \frac{b z + ( a^2-b^2 ) \sin^3 \phi}
             {a s - ( a^2-b^2 ) \cos^3 \phi} \\
\,& \\
\phi = \,& \tan^{-1} \left( t \right) \\
\end{split}\end{equation*}

The geodetic height is then

\begin{equation*}\begin{split}
H = \,& s \cos \theta + ( z + e^2 N(\theta) \sin \theta )
    \sin \theta - N(\theta) \\
  = \,& s \cos \theta + z \sin \theta - N(\theta)
    ( 1 - e^2 \sin^2 \theta ) \\
  = \,& s \cos \theta + z \sin \theta - d \\
\end{split}\end{equation*}

where $\theta = \tan^{-1} \left( \frac{a}b t \right)$.

\section*{Another iterative method}

Estimate $\phi_0 = \tan^{-1} \frac{a^2 z}{b^2 s}$.  Then

\begin{enumerate}

\item Evaluate $N(\phi_n)$

\item Evaluate $\phi_{n+1} =
      \tan^{-1}\frac{z + e^2 N(\phi_n) \sin \phi}{s}$

\end{enumerate}

until $|\phi_{n+1} - \phi_n|$ is small enough.  Then evaluate
$H = \frac{s}{\cos\phi_n} - N(\phi_n)$.

\section*{A Newton method}

Let $\phi_n$ and $H_n$ be estimates of $\phi$ and $H$ after $n$
iterations. Let $\mathbf{C}_n$ be a vector consisting of values of
$z_n$ and $s_n$, with $z_0$ and $s_0$ the given values, and later values
taken as functions of $\phi_n$ and $H_n$.  Let $\mathbf{G}_n$ be the
estimated values of $\phi_n$ and $H_n$.

A Newton method to compute $(H, \phi)$ from $(s, z)$ where $s =
\sqrt{x^2 + y^2} = ( N(\phi) + H) \cos \phi$ would require the
Jacobian

\begin{equation*}
\mathbf{J}(\mathbf{C},\mathbf{G}) =
 \frac{\partial (s,z)}{\partial ( H, \phi)} =
  \left( \begin{array}{ccc}
   \cos\phi & & \cos\phi \, \frac{\partial N(\phi)}
                                       {\partial \phi} -
                (N(\phi) + H ) \sin\phi \\
   & \\
   \frac{b^2}{a^2} \sin\phi & &
     \frac{b^2}{a^2} \sin\phi \, \frac{\partial N(\phi)}
                                    {\partial \phi} +
        \left( \frac{b^2}{a^2} N(\phi) + H \right) \cos\phi \\
 \end{array}\right)
\end{equation*}

at each iteration, where

\begin{equation*}
\frac{\partial N(\phi)}{\partial \phi} =
 \frac{a^2 \sin\phi \cos\phi ( a^2-b^2)}{d^3}
\end{equation*}

First, estimate $\mathbf{G}_0 = ( \phi_0, H_0 ) ^T$:

\begin{equation*}\begin{split}
\phi_0 = \,& \tan^{-1} \frac{a^2 z}{b^2 s} \\
\,& \\
H_0 = \,& \sqrt{z^2+s^2} -
 \frac{ab}{\sqrt{a^2 \sin^2 \phi + b^2 \cos^2 \phi}} \\
\end{split}\end{equation*}

then iterate

\begin{equation*}\begin{split}
\mathbf{G}_{n+1} = \,& \mathbf{G}_n - \mathbf{J}^{-1}
  ( \mathbf{C}_0 - \mathbf{C}_n ) \text{ or, written as a system to solve}
  \\
  \,& \\
\mathbf{J}\, \delta \mathbf{G} = \,& \mathbf{C}_0 - \mathbf{C}_n \\
\end{split}\end{equation*}

\section*{Fukushima's method}

In the Journal of Geodesy (2006) 79: 689-693, Toshio Fukushima presented 
a method based on Halley's third-order iteration for the tangent of the
reduced latitude.  Rather than iterating on the tangent, which requires
divisions in each iteration, Fukushima iterates on denormalized quantities
related to the sine and cosine, thereby not requiring any divisions or
evaluations of trigonometric functions during the iteration.

\begin{equation*}\begin{split}
s = \,& \sqrt{x^2 + y^2} \\
e_c = \,& \frac{b}a \\
E = \,& e^2 = 1 - \frac{b^2}{a^2} = 1 - e_c^2\\
P = \,& \frac{s}a \\
S_0 = \,& z \\
C_0 = \,& e_c P \\
A_n = \,& \sqrt{S_n^2+C_n^2} \\
B_n = \,&  1.5 E S_n C_n^2 [ ( P S_n - z C_n ) A_n - E S_n C_n ] \\
B_0 = \,& 1.5 E^2 P S_0^2 C_0^2 ( A_0 - e_c ) \\
D_n = \,& z A_n^3 + E S_n^3 \\
F_n = \,& P A_n^3 - E C_n^3 \\
S_{n+1} = \,& D_n F_n - B_n S_n \\
C_{n+1} = \,& F_n^2 - B_n C_n \\
C_c = \,& e_c C_{n+1} \\
\phi = \,& \text{signum}(z) \, \tan^{-1} \frac{S_{n+1}}{C_c} \\
h = \,& \frac{s C_c + |z| S_{n+1} - b A_{n+1}}
             {\sqrt{C_c^2 + S_{n+1}^2}} \\
\end{split}\end{equation*}

Fukushima's tests show that the method is accurate to within a few micro
arcseconds for heights $<$ 30,000 km, and faster than other methods, with
only one iteration, i.e., only S_1 and C_1 are computed.

\label{lastpage}
\end{document}

% $Id$

% $Log$
