\documentclass[11pt]{article}
\usepackage[fleqn]{amsmath}\textwidth 6.5in
\oddsidemargin -0.25in
%\evensidemargin -0.5in
\topmargin -0.25in
\textheight 9in

\newcommand{\docname}{\bf wvs-090r4}
\newcommand{\docdate}{17 April 2019}

\begin{document}

%\tracingcommands=1
\newlength{\hW} % heading box width
\newlength{\pW} % page number field width
\settowidth{\hW}{\docname}
\settowidth{\pW}{Page \pageref{lastpage}\ of \pageref{lastpage}}
\ifdim \pW > \hW \setlength{\hW}{\pW} \fi
\makeatletter
\def\@biblabel#1{#1.}
\newcommand{\ps@twolines}{%
  \renewcommand{\@oddhead}{%
    \docdate\hfill\parbox[t]{\hW}{{\hfill\docname}\newline
                          Page \thepage\ of \pageref{lastpage}}}%
\renewcommand{\@evenhead}{}%
\renewcommand{\@oddfoot}{}%
\renewcommand{\@evenfoot}{}%
}%
\makeatother
\pagestyle{twolines}

\vspace{-10pt}
\begin{tabbing}
\phantom{References: }\= \\
To: \>Van\\
Subject: \>Derivation of radiative transfer equation used in Forward Model\\
From: \>Van Snyder\\
Reference: \>wvs-063
\end{tabbing}

\parindent 0pt \parskip 6pt
\vspace{-10pt}

\newcommand{\MT}{\mathcal{T}}

The radiative transfer equation for the special case of local thermodynamic
equilibrium without scattering for isotropic radiation is given by

\begin{equation}\label{one}
\frac{\text{d} I(s)}{\text{d} s} + \beta(s) I(s) = \beta(s) B(s)
\end{equation}

where $\beta(s)$ is the absorption coefficient for one chemical species at one
frequency, and $B(s)$ is the Planck radiation function given by

\begin{equation}
B(s) = \frac{h \nu}
            {k \left( \exp \left[ \frac{h \nu}{k T(s)} \right] -1 \right)}\,.
\end{equation}

The solution of Equation (\ref{one}) is usually written

\begin{equation}\label{three}
I(s) = I(s_0)\, \MT(s) + \int_{s_0}^s B(\sigma)\, \beta(\sigma)
        \,\MT(\sigma)\, \text{d}\sigma\,.
\end{equation}

where

\begin{equation}
\MT(s) = \exp\left(-\int_{s_0}^s \beta(s^\prime) \,\text{d}s^\prime \right)
\end{equation}

When several chemical species are involved, $\beta(s)$ is replaced by
$\alpha(s) = \sum_k f_k(s) \,\beta_k(s)$, where $f_k(s)$ is the volume
mixing ratio of the $k^\text{th}$ chemical species.  Observing that
$\frac{\text{d}\MT(s)}{\text{d}s} = -\beta(s) \MT(s)$, Equation
(\ref{three}) can be written

\begin{equation}
I(s) = I(s_0) \MT(s_0) + \int_{s_0}^s B(\sigma)\, \beta(\sigma)\,
 \MT(\sigma) \,\text{d}\sigma =
I(s_0) \MT(s_0) - \int_{s_0}^s B(\sigma)\,
 \frac{\text{d}\MT(\sigma)}{\text{d}\sigma} \text{d}\sigma\,.
\end{equation}

This is further transformed by integrating by parts, making $B(s)$ the variable
of integration, giving

\begin{equation}\begin{split}
I(s) = \,& I(s_0) \MT(s_0) -B(\sigma) \MT(s) + B(s_0) \MT(s_0) +
\int_{s_0}^{s} \MT(B) \, \frac{\text{d} B(\sigma)}{\text{d} \sigma}\,
 \text{d} \sigma \\
     = \,& I(s_0) \MT(s_0) -B(s) \MT(s) + B(s_0) \MT(s_0) +
\int_{B(s_0)}^{B(\sigma)} \MT(B) \,\text{d} B \,.
\end{split}\end{equation}

This is evaluated as a quadrature in the form

\begin{equation}
I(s) = \sum_{i=1}^{2N} \Delta B_i \MT_i
\end{equation}

where $2N$ is the path length and the constant of integration is absorbed into
the scheme by writing

\begin{equation}
\begin{array}{ll}
\Delta B_1 = \frac{B_1+B_2}2 & \Delta B_i = \frac{B_{i+1}-B_{i-1}}2\,.
\end{array}
\end{equation}

In order to account for the possibility that the path might be reflected from
the surface of the Earth, the midpoint of the path at $t=N$ is handled as a
zero-thickness layer, giving

\begin{equation}
\begin{array}{ll}
\Delta B_t = \frac{B_{t+1}-B_{t-1}}2 &
\Delta B_{2N - t + 1} = \frac{B_{2N-t+2}-B_{2N-t}}2 \\
& \\
\Delta B_{2N - i + 1} = \frac{B_{2N-i+2}-B_{2N-i}}2 &
\Delta B_{2N} = I_0 - \frac{B_{2N-1} + B_{2N}}2
\end{array}
\end{equation}

where $I_0$ is space radiance.

$\MT$ is computed using

\begin{equation}
\begin{split}
\text{for } i \leq \,& t, \\
\MT_1 = \,& 1, \\
\MT_i = \,& \exp \left( - \sum_{j=2}^i \Delta\delta_{j \rightarrow j-1} \right), \\
\text{for } t < \,& i < 2N - t + 1, \\
\MT_i = \,& 0, \\
\text{for } i = \,& 2N - t + 1, \\
\MT_{2N-t+1} = \,& \Upsilon \MT_t, \\
\text{for } i > \,& 2N - t + 1, \\
\MT_i = \,& \MT_{2N-t+1} \exp \left( - \sum_{j=2N-t+2}^i
 \Delta\delta_{j \rightarrow j-1} \right), \\
\end{split}
\end{equation}

where $\Upsilon$ is Earth reflectivity if the ray intersects the Earth surface
and 1 otherwise,

\begin{equation}
 \Delta\delta_{j \rightarrow j-1} = \sum_k \Delta\delta_{j \rightarrow j-1}^k
\end{equation}

and

\begin{equation}
 \Delta\delta_{j \rightarrow j-1}^k =
  \frac{\Delta s_{j \rightarrow j-1}^\text{refr}}
       {\Delta s_{j \rightarrow j-1}}
  \int_{s_i}^{s_{i-1}} f^k(s) \beta^k(s)\, \text{d} s\,'\,.
\end{equation}

\label{lastpage}
\end{document}

% $Id$

% $Log$
% Revision 1.4  2017/05/11 02:25:28  vsnyder
% Explain Equation (4) better, expand Equation (5), repair a typo
%
% Revision 1.3  2012/09/19 02:11:27  vsnyder
% Correct a typo
%
% Revision 1.2  2012/03/30 20:41:42  vsnyder
% Repair graphics stuff
%
% Revision 1.1  2010/04/09 02:27:01  vsnyder
% Initial commit
