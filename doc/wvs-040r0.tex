\documentclass[11pt]{article}
\usepackage[fleqn]{amsmath}\textwidth 6.5in
\oddsidemargin -0.25in
%\evensidemargin -0.5in
\topmargin -0.2in
\textheight 9.2in

\newcommand{\docname}{\bf wvs-040}
\newcommand{\docdate}{13 July 2006}

\usepackage{longtable}

\begin{document}

%\tracingcommands=1
\newlength{\hW} % heading box width
\newlength{\pW} % page number field width
\settowidth{\hW}{\docname}
\settowidth{\pW}{Page \pageref{lastpage}\ of \pageref{lastpage}}
\ifdim \pW > \hW \setlength{\hW}{\pW} \fi
\makeatletter
\def\@biblabel#1{#1.}
\newcommand{\ps@twolines}{%
  \renewcommand{\@oddhead}{%
    \docdate\hfill\parbox[t]{\hW}{{\hfill\docname}\newline
                          Page \thepage\ of \pageref{lastpage}}}%
\renewcommand{\@evenhead}{}%
\renewcommand{\@oddfoot}{}%
\renewcommand{\@evenfoot}{}%
}%
\makeatother
\pagestyle{twolines}

\vspace{-10pt}
\begin{tabbing}
\phantom{References: }\= \\
To: \>Bill\\
Subject: \>Calculating $\phi$ for $\zeta_{\text{max}}$\\
From: \>Van Snyder\\
\end{tabbing}

\parindent 0pt \parskip 6pt
\vspace{-20pt}

We wish to find a point along the line-of-sight path such that
$\frac{\text{d}\zeta}{\text{d}s} = 0$.  Using $s = \sqrt{h^2 - h_t^2}$ and
therefore $\frac{\text{d}s}{\text{d}h} = \frac{h}s$ we can write this as

\begin{equation}\label{one}
\frac{\text{d}\zeta}{\text{d}s} =
 \frac{\text{d}h}{\text{d}s} \frac{\text{d}\zeta}{\text{d}h} =
 \frac{\sqrt{h^2 - h_t^2}}h \frac{\text{d}\zeta}{\text{d}h} = 0
\end{equation}

Denoting the height at which $\frac{\text{d}\zeta}{\text{d}s} = 0$ by $h_m$,
letting  $\phi_m$ correspond to
$h_m$, and
expanding $\frac{\text{d}\zeta}{\text{d}h}$ to first order in $\phi$, we have

\begin{equation}
\left.\frac{\text{d}\zeta}{\text{d}h}\right|_{h=h_m} =
\left.\frac{\text{d}\zeta}{\text{d}h}\right|_{h=h_t} +
(\phi_m-\phi_t)
\left.\frac{\text{d}^2\zeta}{\text{d}h\text{d}\phi}\right|_{h=h_t,\,\phi=\phi_t}
\end{equation}

Using $\frac{h_t}{h_m} = \cos(\phi_m-\phi_t)$, Equation (\ref{one}) becomes

\begin{equation}
\frac{\text{d}\zeta}{\text{d}s} =
\sqrt{1-\frac{h_t^2}{h_m^2}} \left.\frac{\text{d}\zeta}{\text{d}h}\right|_{h=h_m} =
\sin(\phi_m-\phi_t)
 \left(\left.\frac{\text{d}\zeta}{\text{d}h}\right|_{h=h_t} +
(\phi_m-\phi_t)
\left.\frac{\text{d}^2\zeta}{\text{d}h\text{d}\phi}\right|_{h=h_t,\,\phi=\phi_t}
\right) = 0
\end{equation}

Solving for $\Delta \phi = \phi_m-\phi_t$ there are two possibilities.  If
$\frac{\text{d}^2\zeta}{\text{d}h\text{d}\phi} = 0$ then
$\sin \Delta \phi = 0$ or equivalently $\Delta \phi = 0$.  Otherwise

\begin{equation}
\Delta\phi =
-\frac{\left.\frac{\text{d}\zeta}{\text{d}h}\right|_{h=h_t}}
      {\left.\frac{\text{d}^2\zeta}{\text{d}h\text{d}\phi}\right|_{h=h_t,\,\phi=\phi_t}}
\,,\,\text{ or }
\phi_m = \phi_t
-\frac{\left.\frac{\text{d}\zeta}{\text{d}h}\right|_{h=h_t}}
      {\left.\frac{\text{d}^2\zeta}{\text{d}h\text{d}\phi}\right|_{h=h_t,\,\phi=\phi_t}}
\,.
\end{equation}

Having $\phi_m$, compute $h_m$ and $\zeta_m$ as in {\tt wvs-038}.

Is this correct?  The {\tt hydrostatic} routine produces
$\frac{\text{d}\zeta}{\text{d}h}$, but where do I get
$\frac{\text{d}^2\zeta}{\text{d}h\text{d}\phi}$?

How about just interpolating the $\zeta$ grid that {\tt metrics} produces
using a cubic spline and finding a maximum in that?

Once I have $\phi_m$, $h_m$ and $\zeta_m$, should I just insert them into
the grids that {\tt metrics} produces, or is there more to do?  Should I
also put new ones symmetrically placed about the tangent point so that
the tangent point is still in the middle of the path?

\label{lastpage}
\end{document}
% $Id$
