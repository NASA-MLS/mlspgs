\documentclass[11pt]{article}
\usepackage{alltt}
\usepackage[fleqn]{amsmath}
\usepackage{floatflt}
\usepackage{graphicx}
\usepackage{longtable}
\usepackage[strings]{underscore}

\textwidth 6.5in
\oddsidemargin -0.25in
%\evensidemargin -0.5in
\topmargin -0.5in
\textheight 9in

\newcommand{\docname}{wvs-125r1}
\newcommand{\docdate}{24 August 2015}

\ifx\pdfoutput\undefined
  \pdfoutput=0
  \usepackage[hypertex,plainpages,hyperindex=true]{hyperref}
  \hypersetup{%
    hypertexnames=false%
  }
  % Specify the driver for the color package
  \ExecuteOptions{dvips}
  %\ExecuteOptions{xdvi}
\else
  \ifnum\pdfoutput>0
    \usepackage[pdftex,plainpages,hyperindex=true,pdfpagelabels]{hyperref}
    \hypersetup{%
      hypertexnames=false,%
      colorlinks=true,%
      linktocpage=true,%
    }
    % Specify the driver for the color package
    \ExecuteOptions{pdftex}
  \else
    \usepackage[hypertex,plainpages,hyperindex=true]{hyperref}
    \hypersetup{%
      hypertexnames=false%
    }
    % Specify the driver for the color package
    \ExecuteOptions{dvips}
    %\ExecuteOptions{xdvi}
  \fi
\fi

\hyperbaseurl{}
\newcommand\hr[1]{\href{#1.dvi}{dvi}, \href{#1.pdf}{pdf}}
\newcommand\h[1]{#1 (\hr{#1})}

\begin{document}

%\tracingcommands=1
\newlength{\hW} % heading box width
\newlength{\pW} % page number field width
\settowidth{\hW}{\bf\docname}
\settowidth{\pW}{Page \pageref{lastpage}\ of \pageref{lastpage}}
\ifdim \pW > \hW \setlength{\hW}{\pW} \fi
\makeatletter
\def\@biblabel#1{#1.}
\newcommand{\ps@twolines}{%
  \renewcommand{\@oddhead}{%
    \docdate\hfill\parbox[t]{\hW}{{\hfill\bf\docname}\newline
                          Page \thepage\ of \pageref{lastpage}}}%
\renewcommand{\@evenhead}{}%
\renewcommand{\@oddfoot}{}%
\renewcommand{\@evenfoot}{}%
}%
\makeatother
\pagestyle{twolines}

\newcommand{\TS}{T_\text{scat}}
\newcommand{\TSs}[1]{T_{\text{scat}_{#1}}}
\newcommand{\DB}{\Delta B}
\newcommand{\oDB}{\overline{\DB}}
\newcommand{\MT}{\mathcal{T}}
\newcommand{\hMT}{\MT^s}
\newcommand{\IF}[1]{\,\mathcal{A}_n\!\left(#1\right)} % Interpolation Function

\vspace{-10pt}
\begin{tabbing}
\phantom{References: }\= \\
To: \>Nathaniel, Paul, Bill, Mike, Van\\
Subject: \>PhiWindow is now a tuple or a number\\
From: \>Van Snyder\\
%Reference: \>
\end{tabbing}

\parindent 0pt \parskip 6pt
\vspace{-20pt}

The {\tt PhiWindow} field of the {\tt Fill} statement or the {\tt
ForwardModel} statement is now allowed to be a number or a pair of numbers
separated by a colon.

A unit must be specified, and must be either {\tt degrees}, {\tt radians},
or {\tt profiles}.  If a tuple is specified, either both must have the
same units, or one may be dimensionless.

None of the values may be negative.  Zero or 0:0 (with units) indicates a
1-D problem.

If a tuple is specified, and its units are {\tt profiles}, the first
element specifies the number of profiles before the tangent point, and the
second specifies the number after.

If a tuple is specified, and its units are angles, the first element
specifies the angle before the tangent point, and the second specifies the
angle after.

If a single number $n$ is specified, and its units are {\tt profiles},
$\lfloor(n-1)/2\rfloor$ profiles are put after the tangent point, and the
rest before.  For odd $n$, the effect is to put the same number before and
after.  For even $n$ the effect is to put one more before than after.

If a single number $\phi$ is specified, and its units are angles, $\phi/2$
degrees are put before the tangent point, and $\phi/2$ are put after.

\label{lastpage}
\vspace*{-0.1in} % Somehow, this causes lastpage to be defined
\end{document}

% $Id$

% $Log$
% Revision 1.2  2015/08/25 17:48:50  vsnyder
% Add CVS log line
%
