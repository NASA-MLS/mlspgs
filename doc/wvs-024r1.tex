\documentclass[11pt]{article}
\usepackage[fleqn]{amsmath}\textwidth 6.25in

\oddsidemargin -0.25in
%\evensidemargin -0.5in
\topmargin -0.5in
\textheight 9.00in

\newcommand{\docname}{\bf wvs-024r1}
\newcommand{\docdate}{3 October 2005}

\begin{document}

%\tracingcommands=1
\newlength{\hW} % heading box width
\newlength{\pW} % page number field width
\settowidth{\hW}{\docname}
\settowidth{\pW}{Page \pageref{lastpage}\ of \pageref{lastpage}}
\ifdim \pW > \hW \setlength{\hW}{\pW} \fi
\makeatletter
\def\@biblabel#1{#1.}
\newcommand{\ps@twolines}{%
  \renewcommand{\@oddhead}{%
    \docdate\hfill\parbox[t]{\hW}{{\docname}\newline
                          Page \thepage\ of \pageref{lastpage}}}%
\renewcommand{\@evenhead}{}%
\renewcommand{\@oddfoot}{}%
\renewcommand{\@evenfoot}{}%
}%
\makeatother
\pagestyle{twolines}

\vspace{-10pt}
\begin{tabbing}
\phantom{References: }\= \\
To: \>Bill\\
Subject: \>PFA derivatives\\
From: \>Van Snyder\\
Reference: \>wvs-017r4\\
\end{tabbing}

\parindent 0pt \parskip 3pt
\vspace{-20pt}

For non-PFA radiances for a particular channel $c$, we have

\begin{equation}\label{lbl}
I_c \approx \sum_{n=1}^{N_f} \phi_n \Delta \nu_n
               \sum_{i=1}^{N_p} \Delta B_{in} \tau^s_{in}
\end{equation}

where $N_f$ is the number of frequencies in the channel, $\phi$ is the
channel's filter function, $N_p$ is line-of-sight path length, and $s$
indicates ``Strong (line-by-line) calculation.''

To combine PFA and non-PFA radiances for a particular channel $c$, we use

\begin{equation}\label{combined}
I_c \approx \sum_{n=1}^{N_f} \phi_n \Delta \nu_n
               \sum_{i=1}^{N_p} \Delta B_{ic} \tau^s_{in} \tau^w_{ic}
\end{equation}

where $w$ indicates ``Weak-line (PFA) calculation''.  Approximating
$\Delta B_{in}$ by $\Delta B_{ic}$, where $\Delta B_{ic}$ is evaluated at
the channel center frequency, is acceptable because the change in
frequency across a channel is negligible compared to the channel center
frequency.

Observing that $\tau_{in}^s = \exp(-\sum_{j=1}^i\delta_{jn}^s)$ and
differentiating Equation (\ref{lbl}), we have

\begin{equation}
\begin{split}
\frac{\partial I_c}{\partial x} \approx\,&
 \sum_{n=1}^{N_f} \phi_n \Delta \nu_n
  \sum_{i=1}^{N_p} \frac{\partial \Delta B_{in}}{\partial x} \tau^s_{in} +
  \Delta B_{in} \frac{\partial \tau^s_{in}}{\partial x} \\
=\,&
 \sum_{n=1}^{N_f} \phi_n \Delta \nu_n
  \sum_{i=1}^{N_p} \left( \frac{\partial B_{in}}{\partial x} - \Delta B_{in}
   \sum_{j=1}^i \frac{\partial \delta^s_{jn}}{\partial x} \right ) \tau^s_{in}
\end{split}
\end{equation}

To combine PFA and non-PFA derivatives we use a similar definition for
$\tau_{jc}^w$ and differentiate Equation (\ref{combined}):

\begin{equation}\label{final}
\begin{split}
\frac{\partial I_c}{\partial x}
\approx\,& \sum_{n=1}^{N_f} \phi_n \Delta \nu_n \sum_{i=1}^{N_p}
  \frac{\partial \Delta B_{ic}}{\partial x} \tau^s_{in} \tau^w_{ic} +
  \Delta B_{ic} \frac{\partial \tau^s_{in}}{\partial x} \tau^w_{ic} +
  \Delta B_{ic} \tau^s_{in} \frac{\partial \tau^w_{ic}}{\partial x} \\
=\,& \sum_{n=1}^{N_f} \phi_n \Delta \nu_n \sum_{i=1}^{N_p}
 \left(
  \frac{\partial \Delta B_{ic}}{\partial x} -
  \Delta B_{ic}
   \sum_{j=1}^i \left( \frac{\partial \delta^s_{jn}}{\partial x} +
                      \frac{\partial \delta^w_{jc}}{\partial x}
                \right)
  \right) \tau^s_{in} \tau^w_{ic}
\end{split}
\end{equation}

In Equations (\ref{combined}) and (\ref{final}) it is acceptable to
include $\tau^w_{ic}$ and $\frac{\partial \delta^w_{jc}}{\partial x}$
within the frequency averaging, even though they are already averages for
a channel, because $\sum_{n=1}^{N_f} \phi_n = 1$.

\label{lastpage}
\end{document}
% $Id$
