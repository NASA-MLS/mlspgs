\documentclass[11pt]{article}
\usepackage{alltt}
\usepackage[fleqn]{amsmath}
\usepackage{floatflt}
\usepackage{graphicx}
\usepackage{longtable}
\usepackage[strings]{underscore}

\textwidth 6.5in
\oddsidemargin -0.25in
%\evensidemargin -0.5in
\topmargin -0.5in
\textheight 9in

\newcommand{\docname}{wvs-123r1}
\newcommand{\docdate}{28 September 2015}

\ifx\pdfoutput\undefined
  \pdfoutput=0
  \usepackage[hypertex,plainpages,hyperindex=true]{hyperref}
  \hypersetup{%
    hypertexnames=false%
  }
  % Specify the driver for the color package
  \ExecuteOptions{dvips}
  %\ExecuteOptions{xdvi}
\else
  \ifnum\pdfoutput>0
    \usepackage[pdftex,plainpages,hyperindex=true,pdfpagelabels]{hyperref}
    \hypersetup{%
      hypertexnames=false,%
      colorlinks=true,%
      linktocpage=true,%
    }
    % Specify the driver for the color package
    \ExecuteOptions{pdftex}
  \else
    \usepackage[hypertex,plainpages,hyperindex=true]{hyperref}
    \hypersetup{%
      hypertexnames=false%
    }
    % Specify the driver for the color package
    \ExecuteOptions{dvips}
    %\ExecuteOptions{xdvi}
  \fi
\fi

\hyperbaseurl{}
\newcommand\hr[1]{\href{#1.dvi}{dvi}, \href{#1.pdf}{pdf}}
\newcommand\h[1]{#1 (\hr{#1})}

\begin{document}

%\tracingcommands=1
\newlength{\hW} % heading box width
\newlength{\pW} % page number field width
\settowidth{\hW}{\bf\docname}
\settowidth{\pW}{Page \pageref{lastpage}\ of \pageref{lastpage}}
\ifdim \pW > \hW \setlength{\hW}{\pW} \fi
\makeatletter
\def\@biblabel#1{#1.}
\newcommand{\ps@twolines}{%
  \renewcommand{\@oddhead}{%
    \docdate\hfill\parbox[t]{\hW}{{\hfill\bf\docname}\newline
                          Page \thepage\ of \pageref{lastpage}}}%
\renewcommand{\@evenhead}{}%
\renewcommand{\@oddfoot}{}%
\renewcommand{\@evenfoot}{}%
}%
\makeatother
\pagestyle{twolines}

\newcommand{\TS}{T_\text{scat}}
\newcommand{\TSs}[1]{T_{\text{scat}_{#1}}}
\newcommand{\DB}{\Delta B}
\newcommand{\oDB}{\overline{\DB}}
\newcommand{\MT}{\mathcal{T}}
\newcommand{\hMT}{\MT^s}
\newcommand{\IF}[1]{\,\mathcal{A}_n\!\left(#1\right)} % Interpolation Function

\vspace{-10pt}
\begin{tabbing}
\phantom{References: }\= \\
To: \>Bill, Nathaniel, Paul, Michael, Van\\
Subject: \>Magnetic field specification in l2cf\\
From: \>Van Snyder\\
Reference: \> \h{wvs-121}, \h{wvs-122}
\end{tabbing}

\parindent 0pt \parskip 6pt
\vspace{-10pt}

There are restrictions and assumptions concerning how the magnetic field
is specified and filled in the {\tt l2cf}

If the tangent point geocentric altitude quantity and the spacecraft
velocity quantity are specified, the magnetic field is evaluated at
specified altitudes on vectors computed from the tangent position by
rotation in the viewing plane by angles specified in the cross-angles
field, if the field is specified; otherwise, it is computed at the
tangent point.  See \h{wvs-122}.  Cross-track viewing is assumed, even
if the tangent point is in the orbit plane and there are no nonzero
values for the cross-angles field.  Otherwise, it is calculated at
positions specified by its {\tt hGrid} and {\tt vGrid} fields.

There was a time when the horizontal grid of the magnetic field quantity
could not be shared, but this is no longer required.  The latitude,
longitude, and surfaces fields of quantity templates are now allocatable,
not pointers.  This  means there is a separate one in each quantity
template, not just one that is referenced by several pointers.

\begin{itemize}

\item The vertical co\"ordinate of the magnetic field must be {\tt
  zeta}, geodetic altitude ({\tt geodAltitude}), or geocentric height
  ({\tt geocAltitude}).

\item If there is any position in the cross-track direction, other than
  the tangent point, at which the magnetic field is desired, this is
  specified by the {\tt xGrid} field of the {\tt quantity} statement. 
  The {\tt xGrid} field must specify an {\tt hGrid} of explicit type. 
  If cross-track viewing is desired, but the magnetic field is needed
  only at the tangent point, the {\tt xGrid} field need not be
  specified.

\item If a spacecraft velocity quantity ({\tt ScVelECR}) and a tangent
  point geocentric altitude quantity ({\tt geocAltitudeQuantity}) are
  specified in the {\tt fill} statement, they must be compatible.  If
  they are specified, cross-track viewing is assumed, even if the
  tangent point is actually in the orbit plane.  Either both must be
  provided, or neither can be provided.

  There is a boolean {\tt regular} field in the {\tt fill} statement that
  only has meaning when filling the magnetic field, and then only if the
  tangent point geocentric altitude is provided.  If the {\tt regular}
  field is false, the magnetic field positions are calculated by rotating
  the MAF according to the {\tt xGrid} angles.  The positions are within
  the ruled surface defined by the lines connecting the spacecraft
  position to tangent points.  This magnetic field can be dumped or
  written using {\tt DirectWrite}, but the forward model cannot use it. 

  If the {\tt regular} field is true (which can be specified using {\tt
  /regular}), a stacked and coherent quantity is created.  The zero
  cross-angle position is at some MIF of the MAF nearest to the profile. 
  This reference MIF can be specified by the {\tt ReferenceMIF} field of
  the {\tt fill} command, which is only relevent to this case.  If no
  reference MIF is specified, the middle MIF is used.  If one is specified
  without units, it is assumed to be a MIF number.  Otherwise, the units
  must be length (e.g. {\tt km}), and the vertical co\"ordinate for the
  magnetic field must be geodetic altitude.  The field is then calculated on
  vertical profiles rotated from the reference position according to the
  {\tt xGrid}.

  In both these cases, the {\tt ForwardModelGeometry} geometry calculation
  used by the {\tt fill} statement reallocates the latitude and longitude
  components of the magnetic field quantity template, and calculates
  values for them. If the vertical co\"ordinate is geocentric altitude,
  the computed latitude is geocentric latitude; otherwise it is geodetic
  latitude.

  If the tangent point geocentric altitude is not provided, the magnetic
  field is computed using the grid originally specified in the magnetic
  field quantity.

\item If the vertical co\"ordinate of the magnetic field is {\tt
  zeta}, a GPH quantity must be provided in the {\tt fill} statement,
  and a tangent point geocentric altitude quantity ({\tt
  geocAltitudeQuantity}) cannot be provided in the {\tt fill} statement.
  That is, a {\tt zeta} vertical co\"ordinate cannot be used with
  cross-track viewing.

  We cannot calculate the geolocations of the positions in the magnetic
  field until the relationship between zeta and geocentric height is
  computed, but we cannot calculate the geolocations at which to compute
  this relationship until the desired geolocations in the magnetic field
  have been computed.

\item The numbers of instances of the magnetic field quantity and the
  tangent point geocentric altitude quantity ({\tt geocAltitudeQuantity})
  must be equal.  The tangent point geocentric altitude is a minor-frame
  quantity, and the magnetic field is a profile quantity.  If we ever want
  to process more than one MAF per chunk, this problem must be addressed.

\item If the magnetic field quantity has a cross-angles grid ({\tt
  xGrid}) and any of the angles are not zero, then both a tangent point
  geocentric altitude quantity ({\tt geocAltitudeQuantity}) and a
  spacecraft velocity quantity ({\tt ScVelECR}) must be specified in the
  {\tt fill} statement.

\end{itemize}

Here are some examples:

{\tt\begin{verbatim}

  vGridMag: vGrid, type=Linear, coordinate=geodAltitude, $
    start=-10 km, stop=150 km, number=17

  CrossGrid: hGrid, type=explicit, $
    geodangle=[-3.0 degrees, -1.5 degrees, 0 degrees, 1.5 degrees, 3.0 degrees]

  magneticField: quantity, type=magneticField, vGrid=vGridMag, $
    hGrid=hGridStandard, xGrid=crossGrid, stacked=true

  Fill, quantity=apriori.magneticField, method=magneticModel, $
    scVelECR=apriori.scVelECR, geocAltitudeQuantity=apriori.tngtGeocAltGHz, $
    /regular, referenceMIF=112km
\end{verbatim}}

\label{lastpage}
\vspace*{-0.1in} % Somehow, this causes lastpage to be defined
\end{document}

% $Id$

% $Log$
% Revision 1.1  2015/04/30 17:48:12  vsnyder
% Initial commit
%
