\documentclass[11pt]{article}
\usepackage{alltt}
\usepackage[fleqn]{amsmath}\textwidth 6.5in
\oddsidemargin -0.25in
%\evensidemargin -0.5in
\topmargin -0.25in
\textheight 9.0in

\newcommand{\docname}{\bf wvs-086r1}
\newcommand{\docdate}{16 October 2009}

\begin{document}

%\tracingcommands=1
\newlength{\hW} % heading box width
\newlength{\pW} % page number field width
\settowidth{\hW}{\docname}
\settowidth{\pW}{Page \pageref{lastpage}\ of \pageref{lastpage}}
\ifdim \pW > \hW \setlength{\hW}{\pW} \fi
\makeatletter
\def\@biblabel#1{#1.}
\newcommand{\ps@twolines}{%
  \renewcommand{\@oddhead}{%
    \docdate\hfill\parbox[u]{\hW}{{\hfill\docname}\newline
                          Page \thepage\ of \pageref{lastpage}}}%
\renewcommand{\@evenhead}{}%
\renewcommand{\@oddfoot}{}%
\renewcommand{\@evenfoot}{}%
}%
\makeatother
\pagestyle{twolines}

\vspace{-10pt}
\begin{tabbing}
\phantom{References: }\= \\
To: \>Van\\
Subject: \>Periodic cubic spline interpolation with unequally spaced abscissae\\
From: \>Van Snyder\\
\end{tabbing}

\parindent 0pt \parskip 10pt
\vspace{-20pt}

\section{The problem}

Consider the piecewise cubic polynomial

\begin{equation}
S_i(x) = a_i (x-x_i)^3 + b_i (x-x_i)^2 + c_i (x-x_i) + d_i\,.
\end{equation}

Denote $h_i = x_{i+1}-x_i$ and require the polynomial to interpolate the data
$\{y_i\}$, i.e., $S_i(x_i) = d_i = y_i$, with the following additional
conditions:

\begin{equation}\begin{split}\label{two}
S_i(x_{i+1}) = a_i h_i^3 + b_i h_i^2 + c_i h_i + d_i
 =\,& S_{i+1}(x_{i+1}) = d_{i+1} \\
S_i^\prime(x_{i+1}) = 3 a_i h_i^2 + 2 b_i h_i + c_i
 =\,& S_{i+1}^\prime(x_{i+1}) = c_{i+1} \\
S_i^{\prime\prime}(x_{i+1}) = 6 a_i h_i + 2 b_i
 =\,& S_{i+1}^{\prime\prime}(x_{i+1}) = 2 b_{i+1} \\
\end{split}\end{equation}

i.e., $C^0$, $C^1$, and $C^2$ continuity conditions.  From the third relation
in Equation (\ref{two}) we have

\begin{equation}\label{three}
a_i = \frac{b_{i+1}-b_i}{3 h_i}
\end{equation}

Solving the first relation in Equation (\ref{two}) for $c_i$ and substituting
Equation (\ref{three}) into that result we have

\begin{equation}\label{four}
c_i = \frac{\rho_i}3 - b_i h_i - a_i h_i^2
    = \frac{\rho_i}3 - \frac{h_i}3 ( b_{i+1} + 2 b_i )
    \text{ where } \rho_i = 3 \left( \frac{d_{i+1}-d_i}{h_i} \right)
\end{equation}

Substituting Equations (\ref{three}) and (\ref{four}) into the second relation
in Equation (\ref{two}) and rearranging terms we have

\begin{equation}\label{five}
h_i b_i + 2 ( h_i + h_{i+1} ) b_{i+1} + h_{i+1} b_{i+2}
 = \rho_{i+1} - \rho_i
\end{equation}

This gives us $n-2$ equations to solve for $\{b_1,\,b_2,\,\dots,\,b_n\}$.

If we know the data are periodic, we can impose the additional conditions

\begin{equation}\begin{split}\label{six}
S_1^\prime(x_1) = c_1 =& S_n^\prime(x_n) = c_n \\
S_1^{\prime\prime}(x_1) = 2 b_1 =& S_n^{\prime\prime}(x_n) = 2 b_n \\
\end{split}\end{equation}

Substituting the second relation in Equation (\ref{two}), Equation (\ref{four}),
and Equation (\ref{three}) into the first of these gives

\begin{equation}\label{seven}
c_1 = \frac{\rho_1}3 - \frac{h_1}3 ( 2 b_1 + b_2 ) = c_n =
 \frac{h_{n-1}}3 ( b_{n-1} + 2 b_n ) - \frac{\rho_{n-1}}3
\end{equation}

Collecting terms we have

\begin{equation}\label{eight}
h_1 ( 2 b_1 + b_2 ) + h_{n-1} ( b_{n-1} + 2 b_n )
 = \rho_1 + \rho_{n-1}
\end{equation}

Collecting the second row of Equation (\ref{six}), Equation (\ref{eight}),
and Equation (\ref{five}) into a matrix and writing $\sigma_i = 2(h_i +
h_{i+1})$, we have

\begin{equation}\label{nine}
\left[ \begin{array}{lllllllll}
1     & 0        & 0        & 0     & \dots & 0       & 0       & 0            & -1 \\               
2 h_1 & h_1      & 0        & 0     & \dots & 0       & 0       & h_{n-1}      & 2 h_{n-1} \\ 
h_1   & \sigma_1 & h_2      & 0     & \dots & 0       & 0       & 0            & 0 \\                
0     & h_2      & \sigma_2 & h_3   & \dots & 0       & 0       & 0            & 0 \\
\dots & \dots    & \dots    & \dots & \dots & \dots   & \dots   & \dots        & \dots \\
0     & 0        & 0        & 0     & \dots & h_{n-3} & \sigma_{n-3} & h_{n-2} & 0 \\
0     & 0        & 0        & 0     & \dots & 0       & h_{n-2} & \sigma_{n-2} & h_{n-1} \\
\end{array} \right]
%
\left[ \begin{array}{c}
b_1 \\
b_2 \\
b_3 \\
b_4 \\
\dots \\
b_{n-1}\\
b_n \\
\end{array} \right]
%
 = \left[ \begin{array}{c}
0 \\
\rho_1 - \rho_{n-1} \\
\rho_2 - \rho_1 \\
\rho_3 - \rho_2 \\
\dots \\
\rho_{n-2} - \rho_{n-3} \\
\rho_{n-1} - \rho_{n-2} \\
\end{array} \right]
\end{equation}

\section{Tedious solution}

The three nonzeros above the diagonal can be eliminated using Gaussian
elimination.  Alternatively, an explicit LU factorization can be developed,
which is useful if splines over the same abscissae $\{x_i\, |\, i = 1, \dots, n
\}$ are to interpolate different data (the matrix depends only upon the
abscissae).

Assuming $n \geq 4$, start by partitioning the matrix of Equation
(\ref{nine}) and its $LU$ decomposition as

\begin{equation}
A =
\left[ \begin{array}{cc} \hat{A}_1 & \hat{A}_2 \end{array} \right] =
\left[ \begin{array}{cc} \hat{L}_1 & \hat{L}_2 \end{array} \right]
\left[ \begin{array}{cc} \hat{U}_{11} & \hat{U}_{12} \\
                          0           & \hat{U}_{22} \end{array} \right] =
\left[ \begin{array}{cc} \hat{L}_1 \hat{U}_{11} &
                         \hat{L}_1 \hat{U}_{12} + \hat{L}_2 \hat{U}_{22}
                         \end{array} \right]
\end{equation}

where $\hat{A}_2$ and $\hat{L}_2$ are $n \times 2$ blocks, and
$\hat{U}_{22}$ is a $2 \times 2$ block.  Since $\hat{A}_1$ is lower
triangular, $\hat{L}_1 \hat{U}_{11}$ is also lower triangular.  Since
$\hat{L}_1$ is lower triangular, necessarily $\hat{U}_{11} = I$, and therefore
$\hat{L}_1 = \hat{A}_1$.

Now repartition as

\begin{equation}
A =
\left[ \begin{array}{ccc} A_{11} & A_{12} =0 & A_{13}     \\
                          A_{21} & A_{22}    & A_{23} = 0 \\
                          A_{31} & A_{32}    & A_{33} \end{array} \right] =
\left[ \begin{array}{ccc} L_{11} & 0         & 0      \\
                          L_{21} & L_{22}    & 0      \\
                          L_{31} & L_{32}    & L_{33} \end{array} \right]
\left[ \begin{array}{ccc} U_{11} & U_{12}    & U_{13} \\
                          0      & U_{22}    & U_{23} \\
                          0      & 0         & U_{33} \end{array} \right],
\end{equation}

where the first and last row and column of every matrix is of extent 2, and
identify

\begin{equation}
\hat{A}_1 =
\left[ \begin{array}{cc} A_{11} & 0      \\
                         A_{21} & A_{22} \\
                         A_{31} & A_{32} \end{array} \right],\,
\hat{L}_1 =
\left[ \begin{array}{cc} L_{11} & 0      \\
                         L_{21} & L_{22} \\
                         L_{31} & L_{32} \end{array} \right]
\text{ and }
\hat{U}_{11} =
\left[ \begin{array}{cc} U_{11} & U_{12} \\
                         0      & U_{22} \end{array} \right],
\end{equation}

from which, since $\hat{A}_1$ = $\hat{L}_1$ their corresponding blocks are
equal, and since and $\hat{U}_{11}=I$, $U_{11} = I$, $U_{22} = I$, and
$U_{12} = 0$.  There is no solution for $n < 4$.  When $n = 4$ the blocks
with either subscript equal to 2 are not present and all remaining blocks
are $2 \times 2$ blocks.  When $n > 5$, $A_{31} = 0$. This leaves the
three additional relations

\begin{equation}\begin{array}{ll}\label{thirteen}
A_{13} = L_{11} U_{13} & \Rightarrow U_{13} = A_{11}^{-1} A_{13}
= \left[\begin{array}{cc} 0 & -1 \\
                         r & 2 (1+r)
  \end{array} \right]
\text{ where } r = \frac{h_{n-1}}{h_1}
 \\
A_{23} = 0 = A_{21} U_{13} + A_{22} U_{23}
                       & \Rightarrow U_{23} = -A_{22}^{-1} A_{21} U_{13} \\
A_{33} = A_{31} U_{13} +  A_{32} U_{23} + L_{33} U_{33}
 & \Rightarrow L_{33} U_{33} =
 \left\{ \begin{array}{ll} A_{33} - A_{31} U_{13} & n = 4 \\
                           A_{33} - A_{32} U_{23} - A_{31} U_{13} & n = 5 \\
                           A_{33} - A_{32} U_{23} & n > 5\,. \end{array}\right.
\end{array}\end{equation}

Summarizing,

\begin{equation}
A = LU =
\left[ \begin{array}{ccc} A_{11} & 0      & A_{13}     \\
                          A_{21} & A_{22} & 0          \\
                          A_{31} & A_{32} & A_{33} \end{array} \right] =
\left[ \begin{array}{ccc} A_{11} & 0      & 0      \\
                          A_{21} & A_{22} & 0      \\
                          A_{31} & A_{32} & L_{33} \end{array} \right]
\left[ \begin{array}{ccc} I      & 0      & A_{11}^{-1} A_{13} \\
                          0      & I      & -A_{22}^{-1} A_{21} U_{13} \\
                          0      & 0      & U_{33} \end{array} \right].
\end{equation}

$L_{33}$ and $U_{33}$ result from the $LU$ decomposition of a $2 \times 2$
block:

\begin{equation}\label{fifteen}
\left[ \begin{array}{cc} a & b \\ c & d \end{array} \right] =
\left[ \begin{array}{cc} a & 0 \\ c & \frac{ad-bc}a \end{array} \right]
\left[ \begin{array}{cc} 1 & \frac{b}a \\ 0 & 1 \end{array} \right]
\end{equation}

Computing $U_{23} = -A_{22}^{-1} A_{21} U_{13}$ is not difficult or
expensive because $A_{22}$ is lower triangular with at most two nonzero
diagonals below the main diagonal.

$U_{23}$ is empty in the $n=4$ case.

In the $n>4$ case, let $B = A_{21} U_{13}$.  From the first relation of
Equation (\ref{thirteen}), the first two rows of $B$ are
$[\sigma_1 r,\, -h_1 + 2 \sigma_1 (1+r)]$ and $[ h_2 r,\, 2 h_2(1+r)]$, and the
remaining rows are zero.

Other than $A_{13}$, $A$ is lower triangular with only two diagonals below the
main diagonal.  The sub-subdiagonal is equal to the main diagonal, except for
the first and last two elements.  Thus we can represent $A$ (except for
$A_{13}$) using two arrays {\tt D(1:n)} and {\tt S(2:n)} where
%
{\tt D} = [$1, h_1, h_2, \dots, h_{n-1}$], and
{\tt S} = [$2 h_1, \sigma_1, \sigma_2, \dots, \sigma_{n-2}$].

Let {\tt U(1:n,1:2)} represent $[U_{13}, U_{23}, U_{33}]^T$.  Having computed
{\tt U(1:2,1:2)} from Equation (\ref{thirteen}) and $B$, we can compute
{\tt U(3:n-2,:)} using

\begin{alltt}
  R = D(n) / D(2)
  U(3,:) = [ S(3) * R, -D(2) + 2 * S(2) * ( 1 + R ) ] / D(3)
  if ( n > 5 ) then
    U(4,:) = ( D(3) * [ R, S(2) * ( 1 + R ) ] - S(4) * U(3,:) ) / D(4)
    do I = 5, n-2
      U(I,:) = ( - U(I-2,:) * D(I-1) - U(I-1) * S(I) / D(I) ! B(5:n,:) == 0
    end do
  end if
\end{alltt}

We can them form $L_{33} U_{33}$ according to the last relation in Equation
(\ref{thirteen}) and factor it according to Equation (\ref{fifteen}).  This
completes the construction of $L$ and $U$.

Having $L$ and $U$ we can write Equation (\ref{nine}) as $A b = L U b= \zeta$
where $\zeta_1=0$, $\zeta_2 = \rho_1 + \rho_{n-1}$ and otherwise $\zeta_i =
\rho_{i-1} - \rho_{i-2}$. We can solve for $b$ by first solving $Ly=\zeta$ for
$y$, and then solving $Ub=y$ for $b$.

Since $L$ (other than $L_{33}$) is equal to the lower triangle of $A$ (other
than $A_{33}$), it can also be represented by two arrays, which are the same as
the arrays {\tt D} and {\tt S} above, with {\tt D(n-1:n)} replaced by $a$ and
$(a d - b c) / a$ from Equation (\ref{fifteen}), and {\tt S(n)} replaced by $c$
from Equation (\ref{fifteen}).  Denote these arrays by {\tt DL} and {\tt SL}. 
Let {\tt Y(1:n)} represent $y$ and {\tt Z(1:n)} represent $\zeta$.  Then we can
solve for $y$ using

\begin{alltt}
  Y(1) = 0
  Y(2) = Z(2) / D(2)
  do I = 3, n
    Y(I) = ( Z(I) - Y(I-2) * DL(I-1) - Y(I-1) * SL(I) ) / DL(I)
  end do
\end{alltt}

Using the array {\tt U} computed above to represent the last two columns of
$U$, remembering that the diagonal elements of $U$ are 1, letting {\tt B}
represent $b$, and noticing that $b_1 = b_n$ (from the first row of Equation
(\ref{nine}), we can solve for $b$ using

\begin{alltt}
  B(n) = Y(n)                             ! U(n,2) = 1
  B(n-1) = Y(n-1) - U(n-1,2) * B(n)       ! U(n-1,1) = 1
  do I = n-2, 2, -1
    B(I) = Y(I) - U(I,1) * B(I+1) - U(I,2) * B(I+2)
  end do
  B(1) = B(n)
\end{alltt}

Remembering that $S_n = S_1$ because the data are periodic, we have $a_n =
a_1$, $c_n = c_1$, and can compute $a_i$ and $c_i$ for $i = 1, 2, \dots, n-1$
from Equations (\ref{three}) and (\ref{four}).

To evaluate the interpolating spline at $\xi$, compute $\hat{\xi} = x_1 +
\xi\!\!\mod (x_n-x_1)$ and evaluate $S_j(\hat{\xi})$ where $x_j \leq \hat{\xi}
< x_{j+1}$.

\section{Easier solution}

In Equation (\ref{nine}), use the first equation to eliminate $b_n$ and
write $\sigma_0 = 2 (h_1+h_{n-1})$, giving

\begin{equation}\label{sixteen}
\left[ \begin{array}{llllllll}
\sigma_0 & h_1      & 0        & 0     & \dots & 0       & 0       & h_{n-1}      \\
h_1      & \sigma_1 & h_2      & 0     & \dots & 0       & 0       & 0            \\             
0        & h_2      & \sigma_2 & h_3   & \dots & 0       & 0       & 0            \\
\dots    & \dots    & \dots    & \dots & \dots & \dots   & \dots   & \dots        \\
0        & 0        & 0        & 0     & \dots & h_{n-3} & \sigma_{n-3} & h_{n-2} \\
h_{n-1}  & 0        & 0        & 0     & \dots & 0       & h_{n-2} & \sigma_{n-2} \\
\end{array} \right]
%
\left[ \begin{array}{c}
b_1 \\
b_2 \\
b_3 \\
b_4 \\
\dots \\
b_{n-1}\\
\end{array} \right]
%
 = \left[ \begin{array}{c}
\rho_1 - \rho_{n-1} \\
\rho_2 - \rho_1 \\
\rho_3 - \rho_2 \\
\dots \\
\rho_{n-2} - \rho_{n-3} \\
\rho_{n-1} - \rho_{n-2} \\
\end{array} \right]
\end{equation}

This is a symmetric system for which we can calculate an $L U$ factorization,
with all elements of the diagonal of $U$ equal to 1.

Partition $A$ as

\begin{equation}
\left[\begin{array}{ll} A_{11} & A_{12} \\
                        A_{21} & A_{22} \end{array}\right]
 = \left[\begin{array}{ll} L_{11} & 0 \\
                           L_{12} & L_{22} \end{array}\right]
   \left[\begin{array}{ll} U_{11} & U_{12} \\
                           0      & 1 \end{array}\right]
\end{equation}

in which $A_{11}$, $L_{11}$ and $U_{11}$ are $n-2 \times n-2$ blocks, and
$A_{22}$ and $L_{22}$ are $1 \times 1$ blocks.  Equating blocks, we have

\begin{equation}\begin{split}\label{eighteen}
A_{11} =\,& L_{11} U_{11} \\
A_{12} =\,& L_{11} U_{12} \\
A_{21} =\,& L_{21} U_{11} \\
A_{22} =\,& L_{21} U_{12} + L_{22}
\end{split}\end{equation}

The first of these requires the $L U$ decomposition of an $n-2 \times n-2$
symmetric tridiagonal matrix.  Since $A_{11}$ is tridiagonal, $L_{11}$ and
$U_{11}$ are bidiagonal.  Let {\tt SIGMA(0:n-2)} represent the diagonal of
$A$, and {\tt H(1:n-2)} represent the subdiagonal. Let {\tt D(1:n-1)}
represent the diagonal $d$ of $L$, with {\tt D(n-1)} representing
$L_{22}$, and let {\tt U(1:n-2)} represent the superdiagonal $u$ of
$U_{11}$.  The subdiagonal $s$ of $L_{11}$ is the same as the subdiagonal
{\tt H(1:n-3)} of $A_{11}$.  We can calculate $L_{11}$ and $U_{11}$ thus:

\begin{alltt}
  D(1) = SIGMA(0)
  do k = 2, n-2
    U(k-1) = H(k-1) / D(k-1)
    D(k) = SIGMA(k-1) - H(k-1) * U(k-1)
  end do
\end{alltt}

Having $L_{11}$ we can solve the second relation $A_{12} = L_{11} U_{12}$
in Equation (\ref{eighteen})

\begin{equation}
 \left[ \begin{array}{l} h_{n-1}\\ 0\\ \dots\\ 0\\ h_{n-2}
 \end{array} \right] = 
 \left[ \begin{array}{lllll} d_1   & 0     & 0     & \dots   & 0 \\
                             h_1   & d_2   & 0     & \dots   & 0 \\
                             0     & h_2   & d_3   & \dots   & 0 \\
                             \dots & \dots & \dots & \dots   & 0 \\
                             0     & 0     & \dots & h_{n-3} & d_{n-2}
        \end{array} \right]
 \left[ \begin{array}{l} c_1 \\ c_2 \\ c_3 \\ \dots \\ c_{n-2}
 \end{array} \right]
\end{equation}

for $c = U_{12}$.  Let {\tt C(1:n-2)} represent $c$. Then

\begin{alltt}
  C(1) = H(n-1) / D(1)
  do k = 2, n-3
    C(k) = - H(k-1) * C(k-1) / D(k)
  end do
  C(n-2) = ( H(n-2) - H(n-3) * C(n-3) ) / D(n-2)
\end{alltt}

Having $U_{11}$ we can solve the third relation $A_{21} = L_{21} U_{11}$
in Equation (\ref{eighteen})


\begin{equation}
[ \begin{array}{lllll}
           h_{n-1} & 0 & \dots & 0 & h_{n-2}
         \end{array} ] =
 [ \begin{array}{lllll} r_1 & r_2 & \dots & r_{n-3} & r_{n-2}
   \end{array} ]
 \left[ \begin{array}{lllll}
   1     & u_1   & 0     & \dots & 0 \\
   0     & 1     & u_2   & \dots & 0 \\
   \dots & \dots & \dots & \dots & 0 \\
   0     & 0     & \dots & 1     & u_{n-3} \\
   0     & 0     & \dots & 0     & 1
 \end{array} \right]
\end{equation}


for $r = L_{21}$.  Let {\tt R(1:n-2)} represent $r$.  

\begin{alltt}
  R(1) = H(n-1)
  do k = 2, n-3
    R(k) = - R(k-1) * U(k-1)
  end do
  R(n-2) = H(n-2) - R(n-3) * U(n-3)
\end{alltt}

Finally, from the last relation in Equation(\ref{eighteen}),
$L_{22} = A_{22} - L_{21} U_{12}$.  Remember that $A_{22}$
and $L_{22}$ are $1 \times 1$ blocks, so

\begin{alltt}
  D(n-1) = SIGMA(n-2) - dot_product ( R, C )
\end{alltt}

We can solve $ A\, b = L\,U\,b =$

\begin{equation}
 \left[ \begin{array}{llllll}
 d_1   & 0     & 0     & \dots   & 0       & 0 \\
 h_1   & d_2   & 0     & \dots   & 0       & 0 \\
 0     & h_2   & d_3   & \dots   & 0       & 0 \\
 \dots & \dots & \dots & \dots   & \dots   & \dots \\
 0     & 0     & 0     & \dots   & d_{n-2} & 0 \\
 r_1   & r_2   & 0     & \dots   & r_{n-2} & L_{22}
 \end{array} \right]
 \left[ \begin{array}{llllll}
 1     & u_1   & 0     & \dots   & 0     & c_1 \\
 0     & 1     & u_2   & \dots   & 0     & c_2 \\
 0     & 0     & 1     & \dots   & 0     & c_3 \\
 \dots & \dots & \dots & \dots   & \dots &\dots \\
 0     & 0     & 0     & \dots   & 1     & c_{n-2} \\
 0     & 0     & 0     & \dots   & 0     & 1
 \end{array} \right] \,b = \zeta
\end{equation}

for $b$ by first solving $L\,y = \zeta$ for $y$, and then solving $U\,b =
y$ for $b$.  Let {\tt Y(1:n-1)} represent $y$, {\tt B(1:n-1)} represent
$b$, and {\tt Z(1:n-1)} represent $\zeta$.  Then

\begin{alltt}
  Y(1) = Z(1) / D(1)
  do k = 2, n-2
    Y(k) = ( Z(k) - H(k-1) * Y(k-1) ) / D(k)
  end do
  Y(n-1) = ( Z(n-1) - dot_product(R,Y(1:n-1)) ) / L22

  B(n-1) = Y(n-1)
  B(n-2) = Y(n-2) - C(n-2) * B(n-1)
  do k = n-3, 1, -1
    B(k) = Y(k) - U(k) * B(k+1) - C(k) * B(n-1)
  end do
\end{alltt}

Combining all of this into one lump, we can factor $A$ and solve for $y$
in one loop, and solve for $b$ in a second loop:

\begin{alltt}
  D(1) = SIGMA(0)
  C(1) = H(n-1) / D(1)
  R(1) = H(n-1)
  Y(1) = Z(1) / D(1)
  do k = 2, n-3
    U(k-1) = H(k-1) / D(k-1)
    D(k) = SIGMA(k-1) - H(k-1) * U(k-1)
    C(k) = - H(k-1) * C(k-1) / D(k)
    R(k) = - R(k-1) * U(k-1)
    Y(k) = ( Z(k) - H(k-1) * Y(k-1) ) / D(k)
  end do
  U(n-3) = H(n-3) / D(n-3)
  D(n-2) = SIGMA(n-3) - H(n-3) * U(n-3)
  C(n-2) = ( H(n-2) - H(n-3) * C(n-3) ) / D(n-2)
  R(n-2) = H(n-2) - R(n-3) * U(n-3)
  D(n-1) = SIGMA(n-2) - dot_product ( R(1:n-2), C )
  Y(n-2) = ( Z(n-2) - H(n-3) * D(n-3) ) / D(n-2)
  Y(n-1) = ( Z(n-1) - dot_product(R,Y) ) / D(n-1)

  B(n-1) = Y(n-1)
  B(n-2) = Y(n-2) - C(n-2) * B(n-1)
  do k = n-3, 1, -1
    B(k) = Y(k) - U(k) * B(k+1) - C(k) * B(n-1)
  end do
\end{alltt}

\label{lastpage}
\end{document}
% $Id$

% $Log$
% Revision 1.2  2009/10/17 00:53:57  vsnyder
% Correct sign error
%
% Revision 1.1  2009/10/15 22:57:22  vsnyder
% Initial commit
%
