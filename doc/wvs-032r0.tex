\documentclass[11pt]{article}
\usepackage[fleqn]{amsmath}\textwidth 6.25in
\oddsidemargin -0.25in
%\evensidemargin -0.5in
\topmargin -0.5in
\textheight 9.00in

\newcommand{\docname}{\bf wvs-032}
\newcommand{\docdate}{14 March 2006}

\begin{document}

%\tracingcommands=1
\newlength{\hW} % heading box width
\newlength{\pW} % page number field width
\settowidth{\hW}{\docname}
\settowidth{\pW}{Page \pageref{lastpage}\ of \pageref{lastpage}}
\ifdim \pW > \hW \setlength{\hW}{\pW} \fi
\makeatletter
\def\@biblabel#1{#1.}
\newcommand{\ps@twolines}{%
  \renewcommand{\@oddhead}{%
    \docdate\hfill\parbox[t]{\hW}{{\docname}\newline
                          Page \thepage\ of \pageref{lastpage}}}%
\renewcommand{\@evenhead}{}%
\renewcommand{\@oddfoot}{}%
\renewcommand{\@evenfoot}{}%
}%
\makeatother
\pagestyle{twolines}

\vspace{-10pt}
\begin{tabbing}
\phantom{References: }\= \\
To: \>Bill, Nathaniel\\
Subject: \>Incoherent-to-coherent fill\\
From: \>Van Snyder\\
\end{tabbing}

\parindent 0pt \parskip 6pt
\vspace{-20pt}

\section{The problem}

Given a set of data $\{\zeta(\xi_i,\eta_i)\}$ where
$\{(\xi_i,\eta_i)\}$ are not necessarily on a rectangular grid, produce
another set of data $\{z_{jk}\} =
\{\left.z(x,y)\right|_{(x,y)=(x_j,y_k)}\}$ that approximate the given
data in some least-squares sense, where $\{(x_j,y_k)\}$ are on a
rectangular but not necessarily regular grid, i.e., $\{(x_j,y_k)\} =
\{x_j\} \times \{y_k\}$, where $|\{x_j\}|=m$ and  $|\{y_k\}|=n$.  The
discussion assumes $x_j < x_{j+1}$ and $y_k < y_{k+1}$ but does not
assume $x_{j+1}-x_j=x_j-x_{j-1}$ or $y_{k+1}-y_k=y_k-y_{k-1}$.

\section{Solutions}

The solutions proposed here all use low-order Taylor expansions. 
Higher-order expansions could be used.

\subsection{Local least-squares}\label{local}

Construct a least-squares approximation of the form

\begin{equation}\label{unweighted}
\zeta(\xi_i,\eta_i) \simeq z_{jk} +
\frac{\partial z_{jk}}{\partial x}(\xi_i-x_j) +
\frac{\partial z_{jk}}{\partial y}(\eta_i-y_k) +
\frac12\frac{\partial^2 z_{jk}}{\partial x \partial y} (\xi_i-x_j)(\eta_i-y_k)\,,
\end{equation}

using only data such that

\begin{equation}\label{bounds}
x_{j-p} \leq \xi_i \leq x_{j+p} \text{ and }
y_{k-q} \leq \eta_i \leq y_{k+q}
\end{equation}

and solve for $z_{jk}$ and its derivatives.  If $p=q=1$ this is
equivalent to the ``Gandalf's hat'' interpolation used in the full
forward model.

This solution method can be formulated as $mn$ (dense) least-squares
problems, each with four columns and as many rows as satisfy
Equation(\ref{bounds}), or as one least-squares problem of $4mn$
columns.  In the latter case, the problem is block diagonal, with each of
the $mn$ blocks corresponding to a particular $(x_j,y_k)$ and having four
columns and a number of rows equal to $|\{(\xi,\eta): x_{j-p} \leq \xi_i
\leq x_{j+p} \text{ and } y_{k-q} \leq \eta_i \leq y_{k+q}\}|$.  The
number of rows might be as large as $4pqmn|\{(\xi,\eta)\}|$.

\subsection{Global unweighted least squares}

Construct a least-squares approximation of the same form as Equation
(\ref{unweighted}), but using all data. If this is formulated as $mn$
least-squares problems, each has four columns and a number of rows equal
to $|\{(\xi,\eta)\}|$. If this is formulated as one least-squares
problem, it has $4mn$ columns and a number of rows equal to
$mn|\{(\xi,\eta)\}|$, but it is not sparse.

\subsection{Global weighted least squares}

Each equation in the global weighted least squares method is
Equation(\ref{unweighted}), multiplied by a weight

\begin{equation}
\exp\left(-\left(\frac{x_j-\xi_i}{s_x}\right)^2-
    \left(\frac{y_k-\eta_i}{s_y}\right)^2\right)\,,
\end{equation}

where $s_x$ and $s_y$ are scale factors that expresses one's confidence
that a low-order Taylor approximation is appropriate over the distances
$x_j-\xi_i$ and $y_k-\eta_i$.  One possible default for the scale factors
is

\begin{equation}\label{scales}
s_x = \frac{x_m-x_1}{m-1} \text{ and } s_y = R\, \frac{y_n-y_1}{n-1}
\end{equation}

where $R$ converts $x$ and $y$ to the same units.  The default for $R$ is
1.0.

\subsection{Spline}

By dividing the $(x_j,y_k)$ grid into cells, say midway between grid
lines, imposing continuity constraints along the cell boundaries, and
considering only data $\zeta(\xi_i,\eta_i)$ within each cell, a
least-squares problem similar to the second formulation in \ref{local}
can be formed.  Instead of being block diagonal, this problem is banded,
with the band arising from the continuity constraints.  One usually
thinks of a bicubic approximation and constraints on both the value and
derivatives of the solution, instead of Equation (\ref{unweighted}), but
the paradigm works equally well using Equation (\ref{unweighted}) and
continuity constraints only on the value of the solution.

This is not yet implemented.

\section{Modifications to {\tt fill} command}

Three new fill methods have been implemented, known as {\tt LSLocal},
{\tt LSGlobal} and {\tt LSWeighted} to specify the local, global
unweighted or global weighted methods described above, respectively. 
These fill methods are specified similarly to {\tt BinMax}, {\tt
BinMean}, {\tt BinMin} and {\tt BinTotal}.  In addition, if {\tt
method=LSWeighted} the values of $s_x$, $s_y$ and $R$ in Equation
(\ref{scales}) can be specified by {\tt ScaleInsts}, {\tt ScaleSurfs} and
{\tt ScaleRatio} fields, respectively.  If any of these is not specified,
the default in Equation (\ref{scales}) is used.  $s_y$ is always
multiplied by $R$, even if {\tt ScaleSurfs} is specified.

\label{lastpage}
\end{document}
% $Id$
