\documentclass[11pt]{article}
\usepackage{alltt}
\usepackage[fleqn]{amsmath}
\usepackage{floatflt}
\usepackage{graphicx}
\usepackage{longtable}
\usepackage[strings]{underscore}
\usepackage{wasysym}

\textwidth 6.5in
\oddsidemargin -0.25in
%\evensidemargin -0.5in
\topmargin -0.5in
\textheight 9in

\newcommand{\docname}{wvs-128r1}
\newcommand{\docdate}{2 December 2015}

\ifx\pdfoutput\undefined
  \pdfoutput=0
  \usepackage[hypertex,plainpages,hyperindex=true]{hyperref}
  \hypersetup{%
    hypertexnames=false%
  }
  % Specify the driver for the color package
  \ExecuteOptions{dvips}
  %\ExecuteOptions{xdvi}
\else
  \ifnum\pdfoutput>0
    \usepackage[pdftex,plainpages,hyperindex=true,pdfpagelabels]{hyperref}
    \hypersetup{%
      hypertexnames=false,%
      colorlinks=true,%
      linktocpage=true,%
    }
    % Specify the driver for the color package
    \ExecuteOptions{pdftex}
  \else
    \usepackage[hypertex,plainpages,hyperindex=true]{hyperref}
    \hypersetup{%
      hypertexnames=false%
    }
    % Specify the driver for the color package
    \ExecuteOptions{dvips}
    %\ExecuteOptions{xdvi}
  \fi
\fi

\hyperbaseurl{}
\newcommand\hr[1]{\href{#1.dvi}{dvi}, \href{#1.pdf}{pdf}}
\newcommand\h[1]{#1 (\hr{#1})}

\begin{document}

%\tracingcommands=1
\newlength{\hW} % heading box width
\newlength{\pW} % page number field width
\settowidth{\hW}{\bf\docname}
\settowidth{\pW}{Page \pageref{lastpage}\ of \pageref{lastpage}}
\ifdim \pW > \hW \setlength{\hW}{\pW} \fi
\makeatletter
\def\@biblabel#1{#1.}
\newcommand{\ps@twolines}{%
  \renewcommand{\@oddhead}{%
    \docdate\hfill\parbox[t]{\hW}{{\hfill\bf\docname}\newline
                          Page \thepage\ of \pageref{lastpage}}}%
\renewcommand{\@evenhead}{}%
\renewcommand{\@oddfoot}{}%
\renewcommand{\@evenfoot}{}%
}%
\makeatother
\pagestyle{twolines}

\vspace{-10pt}
\begin{tabbing}
\phantom{References: }\= \\
To: \>Nathaniel, Bill, Paul\\
Subject: \>Quaternary Triagular Mesh\\
From: \>Van Snyder\\
References: \>\h{wvs-127}
\end{tabbing}

\parindent 0pt \parskip 6pt
\vspace{-20pt}

\section{Introduction}

In {\bf A Hierarchical co\"ordinate System for Geoprocessing and
Cartography}, Geoffrey Dutton described a method to develop a mesh on a
sphere that has the property that, as the mesh is refined, the ratio of
the area of the largest facet to the area of the smallest facet is bounded
by about 11/6.

The method begins by inscribing an octahedron within the sphere.  If the
edges of the octahedron are projected onto the sphere, they divide the
surface into eight triangular facets, called \emph{octants}.

To develop smaller facets, each facet is divided into four triangles by
joining the midpoints of the edges of the larger facet.  Therefore, at all
levels of refinement, facets are triangles.  Then, the boundary points and
edges of the new facets are projected to the surface of the sphere. 

\begin{figure}[h]
\hfil\includegraphics[scale=0.6]{wvs-128-QTM-1}\hfill
\caption{From Dutton's book, showing uniform refinement to the
fourth level}\label{Fig-1}
\end{figure}

By construction, the number of facets at level $l$, where level one is the
initial octahedron, is $f_l = 8\times4^{l-1} = 2^{2l+1}$.  Because smaller
facets are formed by bisecting each edge, the number of vertices is $v_l =
v_{l-1}+e_{l-1}$.  From Euler's formula for polyhedra, $e_l = f_l + v_l -
2$.  Solving these equations, we have $v_l = 2^{2l}+2$ and $e_l = 3 \times
2^{2l}$.  Counting the poles and equator, the vertices are on $2^l+1$
equally-spaced parallels of latitude.  Along the $k^{\text{th}}$ parallel
from each pole, vertices are at $4 k$ equally-spaced longitudes.

The identifier of a facet consists of its octant number, followed by a
sequence of two-bit numbers that describe which sub-facet of a larger
facet contains the next smaller facet.  Since each two-bit number can be
considered to be a base-four digit, or quaternary digit, and every facet
is a triangle, the system is called a \emph{Quaternary Triangular Mesh},
or \emph{QTM}.  The octant number, followed by the sequence of quaternary
digits, is called a \emph{QTM ID} or \emph{QID}.

Within a binary QID, the octant number is represented by a four-bit number
in the range $8\cdots15$.  This allows to determine unambiguously how many
digits are in the representation, and thereby the resolution of the
facet.  Using a four-bit octant number and a two-bit facet number allows
the identifiers for facets of a mesh refined up to thirteen levels to be
represented within a 31-bit integer.  At a refinement level of thirteen,
the length of an edge of a facet is bounded by about 1.2 kilometers.  This
ought to be adequate for MLS.  If not, 63-bit integers could be used,
allowing a refinement level of 29, at which the resolution is about 4
centimeters.

\section{ZOT projection and facet numbering}

Dutton developed a simple way, called the \emph{Zenithial OrthoTriangular}
projection, or ZOT projection, to compute positions within a QTM.

\begin{figure}[h]
\hfil\includegraphics[scale=0.675]{wvs-128-QTM-2}\hfill
\caption{ZOT projections of refinements to levels 2 and 3, from Dutton's
book}\label{Fig-2}
\end{figure}

The heavy lines are edges of the octahedron.  The outer boundary consists
of southern hemisphere meridians of the octahedron, and the halves of each
outer boundary are aliased.  For example, the points labeled 3 on the top
edge are the same point.  The north pole is the center.  The south pole is
the outer corners.  The equator is the inscribed square with diagonal
edges.  Diagonal lines are parallels of latitude.  The narrow lines are
boundaries of the level-2 refinement.  Dashed lines are boundaries of the
level-3 refinement.

Vertices of the mesh are not uniquely identified (see Section
\ref{vertices}).  At each level of refinement, they are assigned a
\emph{basis number} in the range $1\cdots3$.  Initially, the poles are
assigned 1, the 0$^\circ$ and 180$^\circ$ equatorial vertices of the
octahedron are assigned 3, and the 90$^\circ$ and 270$^\circ$ vertices are
assigned 2.  Thereafter, as each edge is bisected, the basis number of the
midpoint is $B_n = 6 - ( B_a + B_b )$, where $B_a$ and $B_b$ are the basis
numbers of the end points.  The QID of each facet at level 1 is its octant
number.  When a facet is subdivided, the QID of the central facet is the
QID of the parent facet, with zero appended.  The QID of each other facet
is the QID of the parent facet, with the basis number of the vertex
opposite the common edge appended.  This is also the basis number of the
vertex of the larger facet that is a vertex of the smaller facet.

In the ZOT projection, every triangle in the mesh is an isosceles right
triangle.  The right angle is called the \emph{polar angle} because it is
always nearest one of the poles of the sphere.  One edge incident on the
pole node is horizontal, and the other is vertical.

ZOT co\"ordinates are in the range $-1\cdots1$, with the north pole at
$(0,0)$, and the south pole at $(\pm1,\pm1)$.  The mapping from longitude
and latitude to ZOT co\"ordinates is done in the $\ell_1$ metric
(sometimes called the ``Manhattan'' or ``taxicab'' metric), and is best
described by reference to the function {\tt Geo_to_ZOT} in the module {\tt
QTM_m}.  To compute the QID containing a point, one computes the ZOT
co\"ordinates of the point, and uses its octant number as the initial
QID.  If the level of facet identified does not have sufficient
resolution, the identity of a sub-facet is determined by computing the
distances $dx$ and $dy$ of the $x$ and $y$ ZOT co\"ordinates from the pole
node.  Let the length of a horizontal or vertical edge of a facet be $s$.
If $|dx| + |dy| < s/2$, append the basis number of the pole node to the
QID of the parent facet.  Otherwise if either $|dx| > s/2$ or $|dy| > s/2$
append the basis number of the $x$ or $y$ node respectively.  Otherwise,
append zero.  If $|dx| > s/2$ and $|dy| > s/2$, you made a mistake along
the way and you're working in the wrong facet.  No square roots or
trigonometric functions are needed.  The process is repeated if the size
of the facet is not sufficiently small.  The number of repetitions is
equal to the degree of refinement of the mesh.

\begin{figure}[h]
\hfil\includegraphics{wvs-128-QTM-3}\hfill
\caption{Calculation of sub-facet and QID, from Dutton's
book}\label{Fig-3}
\end{figure}

\section{Identifying vertices}\label{vertices}

The QID identifies a facet.  Each facet is bounded by three vertices. 
Each vertex that is not a vertex of the octahedron is a member of six
facets, while each vertex of the octahedron is a member of four facets.

In MLS, every vertex in the mesh needs to be uniquely identified because a
sequential index of a vertex in the mesh is used to calculate a column
subscript in the Jacobian, and a subscript in the state vector.

There are at least four methods to identify vertices.

\subsection{Dual mesh based upon centroids of facets}

The centroid of each facet is uniquely identified by the QID.  If
centroids of facets that share an edge are connected, the result is a dual
mesh, in which each facet is hexagonal if the mesh is refined to the same
level everywhere, except at the vertices of the icosahedron, where each
facet is square.  Each hexagon could then be divided into four triangles,
and each square into two triangles.  A vertex within a QTM facet is within
one of three facets of the dual mesh.

\subsection{QID of pole node of a facet}

A vertex can be identified by the QID of one of the facets of which it is
a pole node.  For all vertices other than the poles, the vertex at the
polar angle is a member of two facets.  The QID of the facet that has the
polar angle at a more northern latitude than its other vertices is unique,
except along the meridional edges of the octahedron, i.e., where the
longitude mod 90 is zero, or the $x$ or $y$ ZOT co\"ordinate is zero; in
this case, a simple choice would be to use the one with the even (or odd)
octant number.  The poles could be identified by the facet with the
smallest octant number.  The QID, sans octant number, of the facets
surrounding the poles always consist of quaternary digits having the value
1, so the poles would be identified by QIDs $11\cdots1$ and $51\cdots1$.

\subsection{Enumerate the vertices}

If a QTM is refined to the same level everywhere, its vertices can be
enumerated, starting with the north pole being vertex one, then counting
from the prime meridian eastward around each parallel of latitude.  In the
southern hemisphere (excluding the equator) use the same scheme with
negative indices.  Let $h = 2^{l-1}$.  The index of the last vertex on the
equator would be $4h(2h+1)+1$.  At the $k^{\text{th}}$ parallel of
latitude counted from the pole, with $k \leq h$ ($k < h$ in the southern
hemisphere), the absolute values of the vertex indices would be in the
range $[2k(k-1)+2, 2k(k+1)+1]$.  If the vertex number is $n$ and
$2k(k-1)+2 \leq |n| \leq 2k(k+1)+1$, the latitude is $90^\circ \left( 1 -
\frac{k}{h}\right)$ with the sign of the vertex index, and the east
longitude is $90^\circ ( |n| - 2k(k-1) - 2 ) / k$.

\subsection{Construct an integer from the ZOT co{\"o}rdinates}

ZOT co\"ordinates are equally spaced in $-1\dots1$.  At level $l$ there
are $2^l+1$ values of each co\"ordinate, with spacing $2^{1-l}$. 
Therefore, $2^{l-1}(1+x)$ and $2^{l-1}(1+y)$, where $x$ and $y$ are ZOT
co\"ordinates in a level $l$ refinement, are each integers in the range
$0\cdots2^l$.  Therefore, $2^{l-1} ( 1 + x + ( 1 + 2^l ) ( 1 + y ) )$ is
an identifier in the range $0\cdots 2^{2l}+2^{l+1}$ for each ZOT
co\"ordinate.  Since there are only $2^{2l}+2$ vertices, this is too many
identifiers.  The reason is that the outer edges of the ZOT projection are
edges of the lower half of the octahedron; the negative half of each outer
edge represents the same points as the positive half.  Therefore, ZOT
co\"ordinates need to be disambiguated by using $|y|$ if $|x| = 1$ and
$|x|$ if $|y| = 1$.  This reduces the number of identifiers to $2^{2n}+2$,
and reduces the range to $2^{l-1} \cdots 2^{2l}+2^{l+1}$.  Since $l$ is
restricted so that a QID fits in an integer, and a QID identifies a facet,
and there are $2^{2l+1}$ facets, this identifer also fits in an integer. 
If it is desired to represent ZOT co\"ordinates by this sort of
identifier, and not retain the original representation, it is easier to
invert the identifier to recover ZOT co\"ordinates if  $2^{l-1} ( 1 + x +
2^{l+1} ( 1 + y ) ) = 2^{l-1} ( 1 + x ) + 2^{2l} ( 1 + y )$ is used.  This
puts identifiers into the range $2^{l-1} \cdots 2^{2l+1}$.  The value of
$l$ used ought to be the maximum QTM refinement level, regardless of the
level to which the mesh is actually refined.  This identification method
works even if the mesh is not refined to the same level everywhere.

\section{Using QTM in MLS}

Since the linear extent of a facet is roughly 20,000/$2^l$ km, a
refinement level of 7 (156 km or 1.4$^\circ$) would be appropriate to SMLS
if it has roughtly the same orbit and same scan rates as Aura MLS.  A
finer refinement might be appropriate to A-SMLS.

Once one has identified a triangle and its vertices, interpolation to a
point within it is easy:  One computes the normalized barycentric
co\"ordinates (Section \ref{Barycentric}) of the point.  The interpolation
coefficients are those co\"ordinates.

The {\tt mlsl2} program will not be used to retrieve atmospheric state
over the entire Earth: At a refinement level of 7 (159 km) there would be
16,386 profiles.  Solving for 20 quantities on 72 levels, there would be
23,595,840 columns in the Jacobian.  Therefore, a smaller mesh will be
constructed.  For A-SMLS, a grid could be constructed consisting of a
swath having the instrument path along its center, and a width such that
the longest line of sight extends to a specified altitude, say 100 km.  An
``interesting'' region would be defined by a polygon.  On a sphere, the
concepts of ``inside'' and ``outside'' of a polygon are ambiguous, so an
additional point defined to be inside the polygon is necessary.  Having
the polygon and a point defined to be within it, determining whether
another point is within the polygon, or constructing a set of points
within the polygon, is not terribly difficult.  As vertices are
identified, a hash table could be used to eliminate duplicates.  QTM
vertices within the interesting region of the mesh could thereby be
given serial numbers.

To construct a QTM for MLS

\begin{enumerate}

\item Construct a polygon that encloses the area of interest, and specify
      a point inside it.  The polygon could, but should not, have holes,
      and it need not be convex.  Following steps are carried out in the
      ZOT projection, so if a polygon has an edge that crosses a
      southern-hemisphere meridian of the original octahedron, decompose
      the polygon into several that do not have this property. Only one
      ``inside'' point is needed.  Remember that in Figure \ref{Fig-2},
      the outside edges of the ZOT projection square are
      southern-hemisphere meridians of the original octagon.  Therefore,
      in the ZOT projection, a polygon that crossed one of these meridians
      would not be closed; it would be two (or more) open polygons.

\item Construct the QTM top down, starting with level 1.  If the facet has
      a point in any polygon, or a polygon has a point in the facet, or an
      edge of any polygon intersects an edge of the facet, and the level
      of refinement is not sufficient, refine that facet.

\item When a facet is constructed, determine whether its vertices that are
      within the polygon are already in the set of vertices that are
      geolocations within a polygon.  If not, add them to the set and
      assign serial numbers to them.  The serial number of a vertex is
      then used to compute a column subscript in the Jacobian, and a
      subscript in the state vector.

\item Retain a quadtree that describes the QTM, with serial numbers of
      geolocations within the area of interest at each vertex, and zero as
      the serial number for vertices that are outside the area of
      interest.

\item There are values at vertices, taken from the state vector, only for
      points within the polygon.  Therefore, the number of vertices of a
      facet for which we have values could be in the range $0\cdots3$.  In
      one dimension (i.e., vertically in the orbit plane), our practice is
      to use constant extrapolation outside the region.  In two
      dimensions, a point outside the polygon might be close to vertices
      of the polygon that have different values, i.e., ``constant
      extrapolation'' isn't uniquely defined.  Therefore, to get values at
      all vertices, do the following:

      {\tt\parskip 3pt
      Specify values at all vertices of the facet that are within the
      polygon using values from the state vector.

      call get_values ( facet )
      }
      
      where {\tt get_values} is

      {\tt\parskip 3pt
      recursive subroutine get_values ( facet )

      \hspace*{0.1in}if there are values at all three vertices, return.

      \hspace*{0.1in}if there are values at two vertices,\\
        \hspace*{0.2in}assign a value at the third vertex that is the
        average of the other two.

      \hspace*{0.2in}return.

      \hspace*{0.1in}choose an unexamined adjacent facet in the direction
        of the polygon.

      \hspace*{0.1in}call get_values ( new facet )

      \hspace*{0.1in}assign values to vertices that do not have values if
      they are vertices\\
      \hspace*{0.1in}of the new facet (which has values at every vertex).

      end
      }

\item When it is necessary to interpolate in the QTM to a point, specified
      by its longitude and latitude, determine its ZOT co\"ordinates, and
      from those determine its QID to the desired level of refinement. 
      Then use digits of the QID to traverse the QTM tree to find the
      facet.  The cost of both of these steps is proportional to the level
      of refinement, not the size of the QTM.  When the facet is found,
      the vertex of the QTM tree identifies the vertices of the facet.

\end{enumerate}

ZOT polygons are not conformal to spherical polygons.  If a point is
within (outwith) a ZOT polygon, it might be outwith (within) the
corresponding spherical polygon -- depending upon how boundaries of
spherical polygons are defined and how close the point is to the polygon
boundary.  Polygons on the sphere are usually defined to have edges that
are segments of great circles, but this precludes edges that follow
parallels of latitude.  Polygons in the ZOT projection are considered
always to have straight edges.  Assuming the polygon of interest is
defined on the sphere using longitude and latitude, rather than worry
about how the boundaries of spherical polygons are defined, and whether
points within ZOT polygons are or are not within corresponding spherical
polygons, a slightly larger polygon that encloses the interesting points
(and maybe a few others) should be chosen.

\section{Barycentric co\"ordinates and interpolation}\label{Barycentric}

The unnormalized barycentric co\"ordinates $\hat\lambda_{s_i}$ of a point
$(x, y)$ within a triangle having vertices $(x_{s_i}, y_{s_i})$, where $i
\in \{1,2,3\}$ and $s_i$ is the serial number of vertex $i$ of the QTM
facet, are

\begin{equation*}
\hat\lambda_{s_k} = (y_{s_j}-y_{s_i})(x-x_{s_i}) -
                    (x_{s_j}-x_{s_i})(y-y_{s_i})
\end{equation*}

where $i$, $j$, $k$ are distinct and in $\{1,2,3\}$.  The signs depend
upon whether the vertices are numbered in clockwise or anticlockwise
order.  Normalized co\"ordinates $\lambda_{s_i}$ are obtained by dividing
the unnormalized co\"ordinates by their sum.  Therefore, it doesn't matter
in which order the vertices are numbered.  If any normalized barycentric
co\"ordinate is negative, the point $(x,y)$ is outside the triangle.

Assuming $(x,y)$ is within the facet, the interpolated value at $f(x,y)$
from $\{f(x_{s_i},y_{s_i})\}$ is

\begin{equation*}
f(x,y) = \sum_{i=1}^3 \lambda_{s_i} f(x_{s_i},y_{s_i})\,.
\end{equation*}

The system was introduced in 1827 by August Ferdinand M{\"o}bius.

Barycentric co\"ordinates can be computed using either longitude and
latitude, or ZOT co\"ordinates, which would result in slightly different
values for $\lambda_{s_i}$ and $f(x,y)$.

In MLS, for three-dimensional problems, the state vector is represented
using a stacked and coherent grid.  That is, the same QTM is used at every
level, and the set of vertical mesh points is the same on every vertical
line that passes through corresponding QTM vertices.  To interpolate to a
point within the state vector, given its longitude, latitude, and
altitude, its barycentric co\"ordinates are computed.  Let $z$ be the
altitude of a point with $z_k \leq z \leq z_{k+1}$ and $k, k+1$ are serial
numbers of levels in the vertical mesh.  Using linear interpolation in the
vertical direction, there are two weights $\mu_k =
(z_{k+1}-z)/(z_{k+1}-z_k)$ and $\mu_{k+1} = 1 - \mu_k$.  This results in
six interpolation weights $\eta_{s_{i}j} = \lambda_{s_i}\mu_j$ where $i
\in \{1,2,3\}$ and $j \in \{k, k+1\}$.  In three dimensions, interpolation
would therefore be

\begin{equation*}
f(x,y,z) = \sum_{i=1}^3 \sum_{j=k}^{k+1} \lambda_{s_i}\mu_j
                                         f(x_{s_i},y_{s_i},z_j)
         = \sum_{i=1}^3 \sum_{j=k}^{k+1} \eta_{s_{i}j}
                                         f(x_{s_i},y_{s_i},z_j)\,,
\end{equation*}

and

\begin{equation*}
\frac{\partial f(x,y,z)}{\partial f(x_{s_i},y_{s_i},z_j)} = \eta_{s_{i}j}
\,.
\end{equation*}

\label{lastpage}
\vspace*{-0.1in} % Somehow, this causes lastpage to be defined
\end{document}

% $Id$

% $Log$
% Revision 1.1  2015/11/11 16:46:58  vsnyder
% Initial commit
%
