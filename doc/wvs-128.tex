\documentclass[11pt]{article}
\usepackage{alltt}
\usepackage[fleqn]{amsmath}
\usepackage{floatflt}
\usepackage{graphicx}
\usepackage{longtable}
\usepackage[strings]{underscore}
\usepackage{wasysym}

\textwidth 6.5in
\oddsidemargin -0.25in
%\evensidemargin -0.5in
\topmargin -0.5in
\textheight 9in

\newcommand{\docname}{wvs-128}
\newcommand{\docdate}{11 November 2015}

\ifx\pdfoutput\undefined
  \pdfoutput=0
  \usepackage[hypertex,plainpages,hyperindex=true]{hyperref}
  \hypersetup{%
    hypertexnames=false%
  }
  % Specify the driver for the color package
  \ExecuteOptions{dvips}
  %\ExecuteOptions{xdvi}
\else
  \ifnum\pdfoutput>0
    \usepackage[pdftex,plainpages,hyperindex=true,pdfpagelabels]{hyperref}
    \hypersetup{%
      hypertexnames=false,%
      colorlinks=true,%
      linktocpage=true,%
    }
    % Specify the driver for the color package
    \ExecuteOptions{pdftex}
  \else
    \usepackage[hypertex,plainpages,hyperindex=true]{hyperref}
    \hypersetup{%
      hypertexnames=false%
    }
    % Specify the driver for the color package
    \ExecuteOptions{dvips}
    %\ExecuteOptions{xdvi}
  \fi
\fi

\hyperbaseurl{}
\newcommand\hr[1]{\href{#1.dvi}{dvi}, \href{#1.pdf}{pdf}}
\newcommand\h[1]{#1 (\hr{#1})}

\begin{document}

%\tracingcommands=1
\newlength{\hW} % heading box width
\newlength{\pW} % page number field width
\settowidth{\hW}{\bf\docname}
\settowidth{\pW}{Page \pageref{lastpage}\ of \pageref{lastpage}}
\ifdim \pW > \hW \setlength{\hW}{\pW} \fi
\makeatletter
\def\@biblabel#1{#1.}
\newcommand{\ps@twolines}{%
  \renewcommand{\@oddhead}{%
    \docdate\hfill\parbox[t]{\hW}{{\hfill\bf\docname}\newline
                          Page \thepage\ of \pageref{lastpage}}}%
\renewcommand{\@evenhead}{}%
\renewcommand{\@oddfoot}{}%
\renewcommand{\@evenfoot}{}%
}%
\makeatother
\pagestyle{twolines}

\vspace{-10pt}
\begin{tabbing}
\phantom{References: }\= \\
To: \>Nathaniel, Bill, Paul\\
Subject: \>Quaternary Triagular Mesh\\
From: \>Van Snyder\\
References: \>\h{wvs-127}
\end{tabbing}

\parindent 0pt \parskip 6pt
\vspace{-20pt}

\section*{Introduction}

In {\bf A Hierarchical Coordinate System for Geoprocessing and
Cartography}, Geoffrey Dutton described a method to develop a mesh on a
sphere that has the property that, as the mesh is refined, the ratio of
the area of the largest facet to the area of the smallest facet is bounded
by about 11/6.

The method begins by inscribing an octahedron within the sphere.  If the
edges of the octahedron are projected onto the sphere, they divide the
surface into eight triangular facets, called \emph{octants}.

To develop smaller facets, each facet is divided into four triangles by
joining the midpoints of the edges of the larger facet.  Therefore, at all
levels of refinement, facets are triangles.  Then, the boundary points and
edges of the new facets are projected to the surface of the sphere. 
Here's a picture from Dutton's book, showing subdivision to the fourth
level:

\hfil\includegraphics[scale=0.6]{wvs-128-QTM-1}\hfill

By construction, the number of facets at level $l$, where level one is the
initial octahedron, is $f_l = 8\times4^{l-1} = 2^{2l+1}$.  Because smaller
facets are formed by bisecting each edge, the number of vertices is $v_l =
v_{l-1}+e_{l-1}$.  From Euler's formula for polyhedra, $e_l = f_l + v_l -
2$.  Solving these equations, we have $v_l = 2^{2l}+2$ and $e_l = 3 \times
2^{2l}$.  Counting the poles and equator the vertices are on $2^l+1$
equally-spaced parallels of latitude.  Along the $k^{\text{th}}$ parallel
from each pole, vertices are at $4 k$ equally-spaced longitudes.

The identifier of a facet consists of its octant number, followed by a
sequence of two-bit numbers that describe which sub-facet of a larger
facet contains the next smaller facet.  Since each two-bit number can be
considered to be a base-four digit, or quaternary digit, and every facet
is a triangle, the system is called a \emph{Quaternary Triangular Mesh},
or \emph{QTM}.  The octant number, followed by the sequence of quaternary
digits, is called a \emph{QTM ID}.

Within a binary QTM ID, the octant number is represented by a four-bit
number in the range $8\cdots15$.  This allows to determine unambiguously
how many digits are in the representation, and thereby the resolution of
the facet.  Using a four-bit octant number and a two-bit facet number
allows the identifiers for facets of a mesh subdivided up to thirteen
times to be represented within a 31-bit integer.  At thirteen
subdivisions, the length of an edge of a facet is bounded by 1.2
kilometers.  This ought to be adequate for MLS.  If not, 63-bit integers
could be used, allowing 29 subdivisions, at which the resolution is 4
centimeters.

\section*{ZOT projection, facet numbering, and vertex numbering}

Dutton developed a simple way, called the \emph{Zenithial OrthoTriangular}
projection, or ZOT projection, to compute positions within a QTM.  Here's
an example of a ZOT projection of a refinement of QTM to levels 2 and 3,
from Dutton's book:

\hfil\includegraphics[scale=0.675]{wvs-128-QTM-2}\hfill

The heavy lines are edges of the octahedron.  The outer boundary consists
of southern hemisphere meridians, and the halves of each outer boundary
are aliased.  For example, the points labeled 3 on the top edge are the
same point.  The north pole is the center.  The south pole is the outer
corners.  The equator is the inscribed square with diagonal edges. 
Diagonal lines are parallels of latitude.  The narrow lines are boundaries
of the level-2 refinement.  Dashed lines are boundaries of the level-3
refinement.

Vertices of the mesh are not uniquely identified.  At each level of
refinement, they are assigned a \emph{basis number} in the range
$1\cdots3$.  Initially, the poles are assigned 1, the 0$^\circ$ and
180$^\circ$ equatorial vertices of the octahedron are assigned 2, and the
90$^\circ$ and 270$^\circ$ vertices are assigned 3.  Thereafter, as each
edge is bisected, the basis number of the midpoint is $B_n = 6 - ( B_a +
B_b )$, where $B_a$ and $B_b$ are the basis numbers of the end points. 
The QTM ID of each facet at level 1 is its octant number.  When a facet is
subdivided, the QTM ID of the central facet is the QTM ID of the parent
facet, with zero appended.  The QTM ID of each other facet is the QTM ID
of the parent facet, with the basis number of the vertex opposite the
common edge appended.

In the ZOT projection, every triangle in the mesh is an isosceles right
triangle.  The right angle is called the \emph{polar angle} because it is
always nearest one of the poles of the sphere.  One edge incident on the
pole node is horizontal, and the other is vertical.

ZOT coordinates are in the range $-1\cdots1$, with the north pole at
$(0,0)$, and the south pole at $(\pm1,\pm1)$.  The mapping from longitude
and latitude to ZOT coordinates is done in the $\ell_1$ metric, and is
best described by reference to the function {\tt Geo_to_ZOT} in the module
{\tt QTM_m}.  To compute the QTM ID containing a point, one computes the
ZOT coordinates of the point, and uses its octant number as the initial
QTM ID.  If the level of facet identified does not have sufficient
resolution, the identity of a sub-facet is determined by computing the
distances $dx$ and $dy$ of the $x$ and $y$ ZOT coordinates from the pole
node.  Let the length of a horizontal or vertical edge of a facet be $s$.
If $|dx| + |dy| < s/2$, append the basis number of the pole node to the
QTM ID of the parent facet.  Otherwise if either $|dx| > s/2$ or $|dy| >
s/2$ append the basis number of the $x$ or $y$ node respectively. 
Otherwise, append zero.  If $|dx| > s/2$ and $|dy| > s/2$, you made a
mistake along the way and you're working in the wrong facet.  No square
roots or trigonometric functions are needed.  The process is repeated if
the size of the facet is not sufficiently small.  The number of
repetitions is equal to the degree of refinement of the mesh.

\hfil\includegraphics{wvs-128-QTM-3}\hfill

\section*{Using QTM in MLS}

Since the linear extent of a facet is roughly 20,000/$2^l$ km, a
refinement level of 7 (156 km) would be appropriate to Aura MLS.  A finer
refinement might be appropriate to A-SMLS.

In MLS, every vertex in the mesh needs to be uniquely identified because a
sequential index of a vertex in the mesh is used to calculate a column
subscript in the Jacobian, and a subscript in the state vector.  Dutton's
QTM method provides unique identifiers for every facet.  Each vertex that
is not a vertex of the octahedron is a member of six facets, while each
vertex of the octahedron is a member of four facets.

The centroid of each facet is uniquely identified by the QTM ID.  If
centroids of facets that share an edge are connected, the result is a dual
mesh, in which each facet is hexagonal, except at the vertices of the
icosahedron, where each facet is square.  Each hexagon could then be
divided into four triangles, and each square into two triangles.  A vertex
within a QTM facet is within one of three facets of the dual mesh.

An alternative to a dual mesh is to identify vertices uniquely by only one
of the identifiers of the QTM facets of which it is a member.  One
candidate is to identify a vertex of the mesh by the QTM ID of one of the
facets of which it is a polar node.  For all vertices other than the
poles, the vertex at the polar angle is a member of two facets.  The QTM
ID of the facet that has the polar angle at a more northern latitude than
its other vertices is unique, except along the meriodonal edges of the
octahedron, i.e., where the longitude mod 90 is zero, or the $x$ or $y$
ZOT coordinate is zero; in this case, a simple choice would be to use the
one with the even (or odd) octant number.  The poles could be identified
by the facet with the smallest octant number.  The QTM ID, sans octant
number, of the facets surrounding the poles always consist of quaternary
digits having the value 1, so the poles would be identified by QTM IDs
$11\cdots1$ and $51\cdots1$.

An alternative either to a dual mesh or using a QTM ID for a vertex is to
index the vertices, starting with the north pole being vertex one, then
counting from the prime meridian eastward around each parallel of
latitude.  In the southern hemisphere (excluding the equator) use the same
scheme with negative indices.  Let $h = 2^{l-1}$.  The index of the last
vertex on the equator would be $q = 4h(2h+1)+1$.  At the $k^{\text{th}}$
parallel of latitude counted from the pole, with $k \leq h$ ($k < h$ in
the southern hemisphere), the absolute values of the vertex indices would
be in the range $[2k(k-1)+2, 2k(k+1)+1]$.  If the vertex number is $n$ and
$2k(k-1)+2 \leq |n| \leq 2k(k+1)+1$, the latitude is $90^\circ \left( 1 -
\frac{k}{h}\right)$ with the sign of the vertex index, and the east
longitude is $90^\circ ( |n| - 2k(k-1) - 2 ) / k$.

Each of these methods allows a fast search for the triangle containing a
given coordinate and the indices of the surrounding vertices, and
therefore their coordinates.

As each vertex in the QTM is identified, it would be assigned a
consecutive serial number, which is in turn used to compute a column
subscript in the Jacobian, and a subscript in the state vector.

Once one has identified a triangle and its vertices, interpolation to a
point within it is easy:  One computes the normalized barycentric
coordinates of the point.  The interpolation coefficients are those
coordinates.  The barycentric coordinates of a point are its orthogonal
distances to the edges of the triangle in which it appears.  Barycentric
coordinates are normalized by dividing each one by the sum of all three.

The {\tt mlsl2} program will not be used to retrieve atmospheric state
over the entire Earth: At a refinement level of 7 (159 km) there would be
16,386 profiles.  Solving for 20 quantities on 72 levels, there would be
23,595,840 columns in the Jacobian.  Therefore, a smaller mesh will be
constructed.  For A-SMLS, a grid could be constructed consisting of a
swath having the instrument path along its center, and a width such that
the longest line-of-sight extends to a specified altitude, say 100 km.  An
``interesting'' region would be defined by a polygon.  On a sphere, the
concepts of ``inside'' and ``outside'' of a polygon are ambiguous, so an
additional point defined to be inside the polygon is necessary.  Having
the polygon and a point defined to be within it, determining whether
another point is within the polygon, or constructing a set of points
within the polygon, is not terribly difficult.  A dense set of points
within that swath would then be used to compute QTM IDs, with a hash table
used to discover duplicates.  QTM vertices within the interesting region
of the mesh would thereby be identified.

\label{lastpage}
\vspace*{-0.1in} % Somehow, this causes lastpage to be defined
\end{document}

% $Id$
