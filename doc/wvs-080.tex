\documentclass[11pt]{article}
\usepackage[fleqn]{amsmath}\textwidth 6.5in
\oddsidemargin -0.25in
%\evensidemargin -0.5in
\topmargin -0.25in
\textheight 9.0in

\newcommand{\docname}{\bf wvs-080r1}
\newcommand{\docdate}{25 August 2009}

\begin{document}

%\tracingcommands=1
\newlength{\hW} % heading box width
\newlength{\pW} % page number field width
\settowidth{\hW}{\docname}
\settowidth{\pW}{Page \pageref{lastpage}\ of \pageref{lastpage}}
\ifdim \pW > \hW \setlength{\hW}{\pW} \fi
\makeatletter
\def\@biblabel#1{#1.}
\newcommand{\ps@twolines}{%
  \renewcommand{\@oddhead}{%
    \docdate\hfill\parbox[u]{\hW}{{\hfill\docname}\newline
                          Page \thepage\ of \pageref{lastpage}}}%
\renewcommand{\@evenhead}{}%
\renewcommand{\@oddfoot}{}%
\renewcommand{\@evenfoot}{}%
}%
\makeatother
\pagestyle{twolines}

\vspace{-10pt}
\begin{tabbing}
\phantom{References: }\= \\
To: \>Van\\
Subject: \>TScat and derivatives\\
From: \>Van Snyder\\
\end{tabbing}

\parindent 0pt \parskip 10pt
\vspace{-20pt}

We calculate $T_\text{scat}$ by propagating a ray from space through a sample
atmosphere to a specified point $s$.  The points on that ray are called $r$ in
the sequel.

At $s$ we have the phase function $P_s(\theta)$, $\frac{\partial
P_s(\theta)}{\partial x_s}$, radiance $I_s$ resulting from radiative transfer along
the ray to $s$, $\frac{\partial I_s}{\partial x_s}$,
and $\frac{\partial I_s}{\partial x_r}$, where $x$ is temperature, IWC, or a
mixing ratio.  The phase function is normalized

\begin{equation*}
\overline{P_s}(\theta) =
 \frac{P_s(\theta)}
      {\int_{-\pi}^\pi P_s(\theta) \sin\theta \, \text{d}\theta}
\end{equation*}

where $\theta$ is the angle between the incident ray and the scattered ray that
originates at $s$.

Derivatives of $\overline{P_s}(\theta)$ with respect to auxiliary parameters $x$
are

\begin{equation*}
\frac{\partial\overline{P_s}(\theta)}{\partial x} =
  \frac1{\int_{-\pi}^\pi P_s(\theta) \sin\theta \, \text{d}\theta}
  \left(\frac{\partial P_s(\theta)}{\partial x}
  - \overline{P_s}(\theta)
      \int_{-\pi}^\pi \frac{\partial P_s(\theta)}{\partial x}
     \sin\theta \, \text{d}\theta
  \right)
\end{equation*}

$T_\text{scat}$ at $s$ is defined as

\begin{equation*}
T_{\text{scat}_s} =
 \int_{-\pi}^\pi \overline{P_s}(\theta) I_s(\theta) \sin\theta
  \, \text{d}\theta\,.
\end{equation*}

From this

\begin{equation*}
\frac{\partial T_{\text{scat}_s}}{\partial x_s} =
 \int_{-\pi}^\pi
  \left( \frac{\partial \overline{P_s}(\theta)}{\partial x_s} I_s(\theta) +
         \overline{P_s}(\theta) \frac{\partial I_s(\theta)}{\partial x_s} \right)
  \sin\theta \, \text{d} \theta
\end{equation*}

and

\begin{equation*}
\frac{\partial T_{\text{scat}_s}}{\partial x_r} =
 \int_{-\pi}^\pi 
  \left( \frac{\partial \overline{P_s}(\theta)}{\partial x_r} I_s(\theta) +
          \overline{P_s}(\theta) \frac{\partial I_s(\theta)}{\partial x_r} \right)
  \sin\theta \, \text{d} \theta =
 \int_{-\pi}^\pi \overline{P_s}(\theta) \frac{\partial I_s(\theta)}{\partial x_r}
  \sin\theta \, \text{d} \theta
\end{equation*}

since $\overline{P_s}(\theta)$ does not depend upon $x_r$.

\label{lastpage}
\end{document}
% $Id$
