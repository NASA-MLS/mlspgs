\documentclass[11pt]{article}
\usepackage{alltt}
\usepackage[fleqn]{amsmath}
\usepackage{floatflt}
\usepackage{graphicx}
\usepackage{longtable}
\usepackage[strings]{underscore}

\textwidth 6.5in
\oddsidemargin -0.25in
%\evensidemargin -0.5in
\topmargin -0.5in
\textheight 9in

\newcommand{\docname}{wvs-112r1}
\newcommand{\docdate}{24 April 2013}

\ifx\pdfoutput\undefined
  \pdfoutput=0
  \usepackage[hypertex,plainpages,hyperindex=true]{hyperref}
  \hypersetup{%
    hypertexnames=false%
  }
  % Specify the driver for the color package
  \ExecuteOptions{dvips}
  %\ExecuteOptions{xdvi}
\else
  \ifnum\pdfoutput>0
    \usepackage[pdftex,plainpages,hyperindex=true,pdfpagelabels]{hyperref}
    \hypersetup{%
      hypertexnames=false,%
      colorlinks=true,%
      linktocpage=true,%
    }
    % Specify the driver for the color package
    \ExecuteOptions{pdftex}
  \else
    \usepackage[hypertex,plainpages,hyperindex=true]{hyperref}
    \hypersetup{%
      hypertexnames=false%
    }
    % Specify the driver for the color package
    \ExecuteOptions{dvips}
    %\ExecuteOptions{xdvi}
  \fi
\fi

\hyperbaseurl{}
\newcommand\hr[1]{\href{#1.dvi}{dvi}, \href{#1.pdf}{pdf}}
\newcommand\h[1]{#1 (\hr{#1})}

\begin{document}

%\tracingcommands=1
\newlength{\hW} % heading box width
\newlength{\pW} % page number field width
\settowidth{\hW}{\bf\docname}
\settowidth{\pW}{Page \pageref{lastpage}\ of \pageref{lastpage}}
\ifdim \pW > \hW \setlength{\hW}{\pW} \fi
\makeatletter
\def\@biblabel#1{#1.}
\newcommand{\ps@twolines}{%
  \renewcommand{\@oddhead}{%
    \docdate\hfill\parbox[t]{\hW}{{\hfill\bf\docname}\newline
                          Page \thepage\ of \pageref{lastpage}}}%
\renewcommand{\@evenhead}{}%
\renewcommand{\@oddfoot}{}%
\renewcommand{\@evenfoot}{}%
}%
\makeatother
\pagestyle{twolines}

\renewcommand{\d}{\text{d}}
\newcommand{\T}{\mathcal{T}}
\newcommand{\M}{\mathcal{M}}

\vspace{-10pt}
\begin{tabbing}
\phantom{References: }\= \\
To: \>Bill, Van\\
Subject: \>New representation of temperature derivatives\\
From: \>Van Snyder\\
Reference: \> \h{wvs-093} \\
\end{tabbing}

\parindent 0pt \parskip 6pt
\vspace{-10pt}

The full forward model mitigates singularities at the tangent point using
Equation (10.13) in the 19 August 2004 ATBD:

\begin{equation}\begin{split}\label{10.13}
\int_{\zeta_i}^{\zeta_{i-1}} G(\zeta) \,
 \frac{\d s}{\d h} \frac{\d h}{\d\zeta} \, \d\zeta = \,&
G(\zeta_i) \int_{\zeta_i}^{\zeta_{i-1}}
 \frac{\d s}{\d h} \frac{\d h}{\d\zeta} \, \d\zeta +
\int_{\zeta_i}^{\zeta_{i-1}} \left[ G(\zeta) - G(\zeta_i) \right]
 \frac{\d s}{\d h} \frac{\d h}{\d\zeta} \, \d\zeta \\
= \,& G(\zeta_i) \Delta s_{i \rightarrow i-1} +
\int_{\zeta_i}^{\zeta_{i-1}} \left[ G(\zeta) - G(\zeta_i) \right]
 \frac{\d s}{\d h} \frac{\d h}{\d\zeta} \, \d\zeta
\,,
\end{split}\end{equation}

where here, and throughout, $h$ is height, not Planck's constant.
The temperature derivative of Equation (\ref{10.13}) is

\begin{equation}\begin{split}\label{10.13_dT}
\frac{\d}{\d T} \int_{\zeta_i}^{\zeta_{i-1}}
\,&  G(\zeta) \,
 \frac{\d s}{\d h} \frac{\d h}{\d\zeta} \, \d\zeta =
\frac{\d}{\d T} \left[ G(\zeta_i) \Delta s_{i \rightarrow i-1} \right] +
\frac{\d}{\d T} \left[ \int_{\zeta_i}^{\zeta_{i-1}} \left[ G(\zeta)
                       - G(\zeta_i) \right]
 \frac{\d s}{\d h} \frac{\d h}{\d\zeta} \, \d\zeta \right] \\
= \,& \frac{\d G(\zeta_i)}{\d T} \Delta s_{i \rightarrow i-1} +
      G(\zeta_i) \frac{\d}{\d T} \Delta s_{i \rightarrow i-1} \, + \\
\,&
 \int_{\zeta_i}^{\zeta_{i-1}}
  \frac{\d}{\d T} \left[ G(\zeta) - G(\zeta_i) \right]
  \frac{\d s}{\d h} \frac{\d h}{\d\zeta} \, \d\zeta +
 \int_{\zeta_i}^{\zeta_{i-1}} \left[ G(\zeta) - G(\zeta_i) \right]
  \frac{\d}{\d T} \left[ \frac{\d s}{\d h} \frac{\d h}{\d\zeta} \right]
   \, \d\zeta
\,.
\end{split}\end{equation}

The forward model uses the Pythagorean theorem to compute

\begin{equation}\label{one-a}
\Delta s_{i \rightarrow i-1} =
\sqrt{h_i^2 -h_t^2} - \sqrt{h_{i-1}^2 -h_t^2} =
  \Delta s_{i \rightarrow t} - \Delta s_{i-1 \rightarrow t}
\end{equation}

where $h_t$ is the tangent height.

An alternative is to use the cosine rule:

\begin{equation}\label{two-a}
\Delta s_{i \rightarrow i-1}
= \sqrt{h_{i-1}^2 + h_i^2 - 2 h_{i-1} h_i
           \cos \left( \delta\phi_{i \rightarrow i-1} \right)}
\end{equation}

where $\delta\phi_{i \rightarrow i-1} = \phi_i - \phi_{i-1}$ is the
difference in orbit geodetic angles at points $i$ and $i-1$ on the path.

Equation (\ref{two-a}) is roughly the same amount of work as Equation
(\ref{one-a}), and experiments show that they produce the same values, at
least up to six printed digits.

One reason to want to use Equation (\ref{two-a}) is to replace the
temperature derivative of path length (the integral in Equation (10.18) in
the 19 August 2004 ATBD), i.e.,

\begin{equation}\label{five}
\frac\d{\d T} \left( \Delta s_{i \rightarrow i-1} \right)
= \frac{h_i \frac{\d h_i}{\d T} -
                  h_t \frac{\d h_t}{\d T}}
                 {\sqrt{h_i^2-h_t^2}} -
            \frac{h_{i-1} \frac{\d h_{i-1}}{\d T} -
                  h_t \frac{\d h_t}{\d T}}
                 {\sqrt{h_{i-1}^2-h_t^2}}
\end{equation}

with the derivative of Equation (\ref{two-a}), \emph{viz}.

\begin{equation}\begin{split}\label{six}
\frac\d{\d T} \left( \Delta s_{i \rightarrow i-1} \right) =\,&
 \frac{h_i \frac{\d h_i}{\d T} + h_{i-1} \frac{\d h_{i-1}}{\d T} -
       2 \left( h_i \frac{\d h_{i-1}}{\d T} + h_{i-1} \frac{\d h_i}{\d T}
         \right) \cos \left(\delta \phi_{i \rightarrow i-1} \right) }
      {\Delta s_{i \rightarrow i-1}}
\,.
\end{split}\end{equation}

Equation (\ref{six}) is slightly more work than Equation
(\ref{five}), so it might not look like much of an improvement, but
its use would have a salubrious effect in {\tt drad_tran_dT}.

Equation (\ref{five}) is used in the derivative of the
singularity-amelioration scheme in Equation (10.13) in the ATBD.  As such,
it is needed only where Gauss-Legendre quadrature has been used on a path
segment.  A complicated decision process is used within {\tt drad_tran_dT}
to avoid evaluating Equation (\ref{five}) where it's not needed.  Where
it is needed, both terms in the right-hand side of Equation (\ref{five})
are needed.  The decision procedure further avoids evaluating one of them
twice in the event Gauss-Legendre quadrature has been used on the previous
path segment.

Furthermore, the quadratures in the full forward model use $\Delta s_{i
\rightarrow i-1}$, but do not retain the two terms in Equation
(\ref{one-a}), \emph{viz.}, $\Delta s_{i \rightarrow t}$ and $\Delta s_{i-1
\rightarrow t}$ separately. These terms are needed separately, however, in
Equation (\ref{five}).  At the initial point on the path, the total
distance to the tangent point, $\Delta s_{1 \rightarrow t}$, is
calculated, and then $\Delta s_{i \rightarrow i-1}$ is subtracted from
this for each path segment, up to the tangent point.  Then $\Delta s_{t+1
\rightarrow t+2}$ is used for the first segment after the tangent point,
and $\Delta s_{i \rightarrow i+1}$ is added for each segment up to the end
of the path.

Neither of these complications would be needed if Equation (\ref{six})
were used.

Neither Equation (\ref{five}) nor (\ref{six}) is needed, however.  Using
the Pythagorean theorem,

\begin{equation}\label{seven}
s = \sqrt{h^2-h_t^2}
\end{equation}

wherein both $h$ and $h_t$ (the tangent height) depend upon temperature.

From Equation (\ref{seven}),

\begin{equation}\label{eight}
\frac{\d s}{\d h} = \frac{h}{\sqrt{h^2-h_t^2}} = \frac{h}s
\,.
\end{equation}

From Equation (\ref{eight}),

\begin{equation}\label{nine}
\frac{d}{\d T}\frac{\d s}{\d h}
= \frac1s \frac{\d h}{\d T} - \frac{h}{s^2} \frac{\d s}{\d T}
= \frac1s \frac{\d h}{\d T} - \frac{h^2}{s^3} \frac{\d h}{\d T}
 + \frac{h h_t}{s^3} \frac{d h_t}{\d T}
= \frac{h_t}{s^3} \left( h \frac{\d h_t}{\d T} - h_t \frac{\d h}{\d T} \right)
\,.
\end{equation}

From the definition on page 47 of the 19 August 2004 ATBD, immediately
before Equation (10.16),

\begin{equation}\label{ten}
\frac{\d h}{\d \zeta} = \frac{h^2 k T \ln 10}{g_0 R_0 \mathcal{M}}
\end{equation}

wherein $\mathcal{M}$ is the mean molecular mass of the atmosphere, not
the position of the instrument.  From Equation (\ref{ten}),

\begin{equation}\label{eleven}
\frac{\d}{\d T} \frac{\d h}{\d \zeta} = \frac{h k \ln 10}{g_0 R_0 \mathcal{M}}
 \left( 2 T \frac{\d h}{\d T} + h \right)
= \left( \frac2h \frac{\d h}{\d T} + \frac1T \right) \frac{\d h}{\d \zeta}
\,.
\end{equation}

From Equations (\ref{nine}) and (\ref{eleven})

\begin{equation}\begin{split}\label{twelve}
\frac{\d}{\d T} \left( \frac{\d s}{\d h} \frac{\d h}{\d \zeta} \right) = \,&
\left( \frac{\d}{\d T} \frac{\d s}{\d h} \right) \frac{\d h}{\d \zeta} +
\frac{\d s}{\d h} \frac{\d}{\d T} \frac{\d h}{\d \zeta} =
\frac1s \frac{\d h}{\d \zeta}
 \left[ \left( 2 - \frac{h_t^2}{s^2} \right) \frac{\d h}{\d T} +
  \frac{h}T +\frac{h h_t}{s^2} \frac{\d h_t}{\d T} \right] \\
 = \,&
\left[ \frac1h \left( 2 - \frac{h_t^2}{s^2} \right) \frac{\d h}{\d T} +
  \frac1T +\frac{h_t}{s^2} \frac{\d h_t}{\d T} \right]
  \frac{\d s}{\d h} \frac{\d h}{\d \zeta} =
  \mathcal{D}(\zeta) \frac{\d s}{\d h} \frac{\d h}{\d \zeta}
\,.
\end{split}\end{equation}

Therefore

\begin{equation}\begin{split}\label{thirteen}
\frac{\d}{\d T} \int_{\zeta_i}^{\zeta_{i-1}} \mathcal{G}(\zeta)
 \frac{\d s}{\d h} \frac{\d h}{\d\zeta} \,\d \zeta = \,&
\int_{\zeta_i}^{\zeta_{i-1}} \frac{\d}{\d T}\, \mathcal{G}(\zeta)
 \frac{\d s}{\d h} \frac{\d h}{\d\zeta} +
 \mathcal{G}(\zeta) \frac{\d}{\d T}
  \left( \frac{\d s}{\d h} \frac{\d h}{\d\zeta} \right) \,\d \zeta \\
= \,&
 \int_{\zeta_i}^{\zeta_{i-1}}
  \left( \mathcal{G}(\zeta) \mathcal{D}(\zeta) +
         \frac{\d}{\d T}\, \mathcal{G}(\zeta) \right)
  \frac{\d s}{\d h} \frac{\d h}{\d\zeta}\d \zeta \\
= \,& \left( \mathcal{G}(\zeta_i) \mathcal{D}(\zeta_i) +
         \frac{\d}{\d T}\, \mathcal{G}(\zeta_i) \right)
      \Delta s_{i \rightarrow i-1} \\
 + \,&
  \int_{\zeta_i}^{\zeta_{i-1}}
   \left( \mathcal{G}(\zeta) \mathcal{D}(\zeta) -
          \mathcal{G}(\zeta_i) \mathcal{D}(\zeta_i) +
          \frac{\d}{\d T}\, \mathcal{G}(\zeta) -
         \frac{\d}{\d T}\, \mathcal{G}(\zeta_i) \right)
    \frac{\d s}{\d h} \frac{\d h}{\d\zeta} \,\d \zeta
\,.
\end{split}\end{equation}

\label{lastpage}
\end{document}

% $Id$

% $Log:
