\documentclass[11pt]{article}
\usepackage{alltt}
\usepackage[fleqn]{amsmath}
\usepackage{floatflt}
\usepackage{graphicx}
\usepackage{longtable}
\usepackage[strings]{underscore}

\textwidth 6.5in
\oddsidemargin -0.25in
%\evensidemargin -0.5in
\topmargin -0.5in
\textheight 9in

\newcommand{\docname}{wvs-112}
\newcommand{\docdate}{10 April 2013}

\ifx\pdfoutput\undefined
  \pdfoutput=0
  \usepackage[hypertex,plainpages,hyperindex=true]{hyperref}
  \hypersetup{%
    hypertexnames=false%
  }
  % Specify the driver for the color package
  \ExecuteOptions{dvips}
  %\ExecuteOptions{xdvi}
\else
  \ifnum\pdfoutput>0
    \usepackage[pdftex,plainpages,hyperindex=true,pdfpagelabels]{hyperref}
    \hypersetup{%
      hypertexnames=false,%
      colorlinks=true,%
      linktocpage=true,%
    }
    % Specify the driver for the color package
    \ExecuteOptions{pdftex}
  \else
    \usepackage[hypertex,plainpages,hyperindex=true]{hyperref}
    \hypersetup{%
      hypertexnames=false%
    }
    % Specify the driver for the color package
    \ExecuteOptions{dvips}
    %\ExecuteOptions{xdvi}
  \fi
\fi

\hyperbaseurl{}
\newcommand\hr[1]{\href{#1.dvi}{dvi}, \href{#1.pdf}{pdf}}
\newcommand\h[1]{#1 (\hr{#1})}

\begin{document}

%\tracingcommands=1
\newlength{\hW} % heading box width
\newlength{\pW} % page number field width
\settowidth{\hW}{\bf\docname}
\settowidth{\pW}{Page \pageref{lastpage}\ of \pageref{lastpage}}
\ifdim \pW > \hW \setlength{\hW}{\pW} \fi
\makeatletter
\def\@biblabel#1{#1.}
\newcommand{\ps@twolines}{%
  \renewcommand{\@oddhead}{%
    \docdate\hfill\parbox[t]{\hW}{{\hfill\bf\docname}\newline
                          Page \thepage\ of \pageref{lastpage}}}%
\renewcommand{\@evenhead}{}%
\renewcommand{\@oddfoot}{}%
\renewcommand{\@evenfoot}{}%
}%
\makeatother
\pagestyle{twolines}

\renewcommand{\d}{\text{d}}
\newcommand{\T}{\mathcal{T}}
\newcommand{\M}{\mathcal{M}}

\vspace{-10pt}
\begin{tabbing}
\phantom{References: }\= \\
To: \>Bill, Van\\
Subject: \>More on derivatives of radiative transfer\\
From: \>Van Snyder\\
Reference: \> \h{wvs-093} \\
\end{tabbing}


\parindent 0pt \parskip 6pt
\vspace{-10pt}

From \h{wvs-093}, the radiance measured at the instrument $\M$, in the
form used in the full forward model (i.e., after integrating the usual
form of solution by parts) is
%
\begin{equation}\label{one}
I(s_\M) = \T(s_0,S_\M)(I(s_0)-B(s_0))+B(s_\M) -
 \int_{\zeta(s_0)}^{\zeta(s_\M)} \T(s,s_\M) \frac{\d B}{\d s}
  \frac{\d s}{\d h} \frac{\d h}{\d\zeta} \, \d\zeta
\end{equation}

where $s_0$ is deep space away from the instrument, $s_\M$ is the position
of the instrument, and
%
\begin{equation}\label{two}
\T(s,s_\M) = \exp \left( - \int_{\zeta(s)}^{\zeta(s_\M)} \alpha(\sigma)
 \frac{\d\sigma}{\d h} \frac{\d h}{\d \zeta} \, \d\zeta \right) \,.
\end{equation}

From Equation (\ref{two}),
%
\begin{equation}\label{three}
\frac{\partial \T}{\partial x} =
 - \T \frac{\partial}{\partial x}
    \int_{\zeta(s)}^{\zeta(s_\M)} \alpha(\sigma)
     \frac{\d\sigma}{\d h} \frac{\d h}{\d \zeta} \, \d\zeta
=
 - \T \int_{\zeta(s)}^{\zeta(s_\M)} \left[
     \frac{\partial \alpha}{\partial x}
     \frac{\d\sigma}{\d h} \frac{\d h}{\d \zeta} +
     \alpha(\sigma) \frac{\partial}{\partial x}
      \left( \frac{\d\sigma}{\d h} \frac{\d h}{\d \zeta}
       \right) \right] \, \d\zeta \,.
\end{equation}

From Equation (\ref{one}), since the derivatives of $I(s_0)$, $B(s_0)$,
and $B(s_\M)$ are zero,
%
\begin{equation}\label{four}
\frac{\partial I}{\partial x} =
 \frac{\partial \T(s_0,s_\M)}{\partial x} (I(s_0)-B(s_0))+B(s_\M) -
  \frac\partial{\partial x} \int_{\zeta(s_0)}^{\zeta(s_\M)} \T(s,s_\M)
   \frac{\d B}{\d s} \frac{\d s}{\d h} \frac{\d h}{\d\zeta} \, \d\zeta
\end{equation}

The integal in Equation (\ref{four}), after the derivative is carried
into the integration, is
%
\begin{equation}
\int_{\zeta(s_0)}^{\zeta(s_\M)} \left[
\left( \frac{\partial \T}{\partial x} \frac{\d B}{\d s} +
 \T \frac{\partial^2 B}{\partial s \partial x} \right)
  \frac{\d s}{\d h} \frac{\d h}{\d\zeta} +
 \T \frac{\d B}{\d s} \frac{\partial}{\partial x}
  \left( \frac{\d s}{\d h} \frac{\d h}{\d\zeta} \right)
 \right] \, \d \zeta
\end{equation}

If $x \neq T$, $\frac\partial{\partial x} \left( \frac{\d s}{\d h}
 \frac{\d h}{\d\zeta} \right) = 0$ and
 $\frac{\partial^2B}{\partial s \partial x} = 0$.

The integrals are actually carried out in two stages, from $\zeta(s_0)$ to
$\zeta(s_t)$, and from $\zeta(s_t)$ to $\zeta(s_\M)$, where $t$ denotes
the tangent point, because the correspondence between $s$ and $\zeta$ is
monotone over those intervals, but not over the entire interval.

From section 10.4.2 of the 4 August 2004 ATBD,
%
\begin{equation}
\frac{\d s}{\d h} = \frac{h(\zeta)}{\sqrt{h(\zeta)^2-h(t)^2}}
\text{ and }
\frac{\d h}{\d \zeta} = \frac{h(\zeta)^2 k T(\zeta) \ln 10}
                             {g_0 R_0^2 \M}
\end{equation}

where $\M$ here is the mean molecular mass, not the position of the
instrument.  Therefore there is a singularity in each integrand at $\zeta
= \zeta(s_t)$.  The reduction in quadrature accuracy due to these
singularities is ameliorated using Equation (10.13) in the ATBD.  This
involves adding and subtracting the factor multiplying $\frac{\d s}{\d h}
\frac{\d h}{\d \zeta}$, evaluated at the lower limit, then taking it out
of one of the integrals, leaving behind that factor multiplied by $\Delta
s$.  The derivative of $\Delta s$ is Equation (10.18).

The forward model then uses the Pythagorean theorem to compute
%
\begin{equation}\label{one-a}
\Delta s_{i \rightarrow i-1} =
\sqrt{H_i^2 -H_t^2} - \sqrt{H_{i-1}^2 -H_t^2} =
  \Delta s_{i \rightarrow t} - \Delta s_{i-1 \rightarrow t}
\end{equation}

where $H_t$ is the tangent height.

An alternative is to use the cosine rule:
%
\begin{equation}\label{two-a}
\Delta s_{i \rightarrow i-1}
= \sqrt{H_{i-1}^2 + H_i^2 - 2 H_{i-1} H_i
           \cos \left( \delta\phi_{i \rightarrow i-1} \right)}
\end{equation}

where $\delta\phi_{i \rightarrow i-1} = \phi_i - \phi_{i-1}$ is the
difference in orbit geodetic angles at points $i$ and $i-1$ on the path.

Equation (\ref{two-a}) is roughly the same amount of work as Equation
(\ref{one-a}), and experiments show that they produce the same values, at
least up to six printed digits.

One reason to want to use Equation (\ref{two-a}) is to replace the integral
in Equation (10.18) in the 19 August 2004 ATBD:
%
\begin{equation}\label{three-a}
\frac\d{\d T} \left( \Delta s_{i \rightarrow i-1} \right)
= \frac{H_i \frac{\d H_i}{\d T} -
                  H_t \frac{\d H_t}{\d T}}
                 {\sqrt{H_i^2-H_t^2}} -
            \frac{H_{i-1} \frac{\d H_{i-1}}{\d T} -
                  H_t \frac{\d H_t}{\d T}}
                 {\sqrt{H_{i-1}^2-H_t^2}}
\end{equation}

with the derivative of Equation (\ref{two-a}), \emph{viz}.
%
\begin{equation}\begin{split}\label{four-a}
\frac\d{\d T} \left( \Delta s_{i \rightarrow i-1} \right) =\,&
 \frac{H_i \frac{\d H_i}{\d T} + H_{i-1} \frac{\d H_{i-1}}{\d T} -
       2 \left( H_i \frac{\d H_{i-1}}{\d T} + H_{i-1} \frac{\d H_i}{\d T}
         \right) \cos \left(\delta \phi_{i \rightarrow i-1} \right) }
      {\Delta s_{i \rightarrow i-1}}
\end{split}\end{equation}

Equation (\ref{four-a}) is slightly more work than Equation
(\ref{three-a}), so it might not look like much of an improvement , but
its use would have a salubrious effect in {\tt drad_tran_dT}.

Equation (\ref{three-a}) is used in the derivative of the
singularity-amelioration scheme in Equation (10.13) in the ATBD.  As such,
it is needed only where Gauss-Legendre quadrature has been used on a path
segment.  A complicated decision process is used within {\tt drad_tran_dT}
to avoid evaluating Equation (\ref{three-a}) where it's not needed.  Where
it is needed, both terms in the right-hand side of Equation (\ref{three-a})
are needed.  The decision procedure further avoids evaluating one of them
twice in the event Gauss-Legendre quadrature has been used on the previous
path segment.

Furthermore, the quadratures in the full forward model use $\Delta s_{i
\rightarrow i-1}$, but do not retain the two terms in Equation
(\ref{one-a}), \emph{viz.}, $\Delta s_{i \rightarrow t}$ and $\Delta s_{i-1
\rightarrow t}$ separately. These terms are needed separately, however, in
Equation (\ref{three-a}).  At the initial point on the path, the total
distance to the tangent point, $\Delta s_{1 \rightarrow t}$, is
calculated, and then $\Delta s_{i \rightarrow i-1}$ is subtracted from
this for each path segment, up to the tangent point.  Then $\Delta s_{t+1
\rightarrow t+2}$ is used for the first segment after the tangent point,
and $\Delta s_{i \rightarrow i+1}$ is added for each segment up to the end
of the path.

Neither of these complications would be needed if Equation (\ref{four-a})
were used.

\label{lastpage}
\end{document}

% $Id$

% $Log:
