\documentclass[11pt]{article}
\usepackage[fleqn]{amsmath}\textwidth 6.5in
\oddsidemargin -0.25in
%\evensidemargin -0.5in
\topmargin -0.25in
\textheight 9.2in

\newcommand{\docname}{\bf wvs-041}
\newcommand{\docdate}{20 July 2006}

\usepackage{longtable}

\begin{document}

%\tracingcommands=1
\newlength{\hW} % heading box width
\newlength{\pW} % page number field width
\settowidth{\hW}{\docname}
\settowidth{\pW}{Page \pageref{lastpage}\ of \pageref{lastpage}}
\ifdim \pW > \hW \setlength{\hW}{\pW} \fi
\makeatletter
\def\@biblabel#1{#1.}
\newcommand{\ps@twolines}{%
  \renewcommand{\@oddhead}{%
    \docdate\hfill\parbox[t]{\hW}{{\hfill\docname}\newline
                          Page \thepage\ of \pageref{lastpage}}}%
\renewcommand{\@evenhead}{}%
\renewcommand{\@oddfoot}{}%
\renewcommand{\@evenfoot}{}%
}%
\makeatother
\pagestyle{twolines}

\vspace{-10pt}
\begin{tabbing}
\phantom{References: }\= \\
To: \>Bill\\
Subject: \>Maximum $\zeta$\\
From: \>Van Snyder\\
\end{tabbing}

\parindent 0pt \parskip 6pt
\vspace{-20pt}

We are given $\phi_i$, $\zeta(\phi_i)$ and
$\left.\frac{\text{d}\zeta}{\text{d}\phi}\right|_{\phi=\phi_i}$ for $i = 1,
\dots, n$.  We wish to find $\phi_m$ such that $\zeta(\phi_m)$ is maximum in
$\phi_1 < \phi_m < \phi_n$, and add $\zeta(\phi_m)$ and $\phi_m$ to
$\{\zeta_i\}$ and $\{\phi_i\}$ if $\inf\{|\phi_m-\phi_i|\} >$ tol.

From the given quantities, form Hermite interpolating polynomials in each
region $[\phi_i,\phi_{i+1}]$.  Scaling each region onto $x = [0,1]$ we can
write a polynomial

\vspace{-15pt}\begin{equation}\label{zeta}
\zeta=c_0 + c_1 x + c_2 x^2 + c_3 x^3
\end{equation}

and $\frac{\text{d}\zeta}{\text{d}x} = c_1 + 2 c_2 x + 3 c_3 x^2$.  Denoting
the given $\zeta$ and $\frac{\text{d}\zeta}{\text{d}x}$ at $x=0$ and $x=1$ by
$\zeta_0$, $\zeta^\prime_0$, $\zeta_1$ and $\zeta^\prime_1$ for each region
$[\phi_i,\phi_{i+1}]$, we have

\vspace{-15pt}\begin{equation}
\left[ \begin{array}{llll}
 1 & 0 & 0 & 0 \\
 0 & 1 & 0 & 0 \\
 1 & 1 & 1 & 1 \\
 0 & 1 & 2 & 3 \\
\end{array}\right]
\left[ \begin{array}{l}
 c_0 \\
 c_1 \\
 c_2 \\
 c_3 \\
\end{array}\right] =
\left[ \begin{array}{l}
\zeta_0 \\
\zeta^\prime_0 \\
\zeta_1 \\
\zeta^\prime_1
\end{array}\right]
\end{equation}

from which

\vspace{-15pt}\begin{equation}\begin{split}
c_0 =\,& \zeta_0 \\
c_1 =\,& \zeta^\prime_0 \\
c_2 =\,& -3 \zeta_0 + 3 \zeta_1 -2 \zeta^\prime_0 - \zeta^\prime_1 \\
c_3 =\,& 2 \zeta_0 - 2 \zeta_1 + \zeta^\prime_0 + \zeta^\prime_1 \,. \\
\end{split}\end{equation}

We want $\frac{\text{d}\zeta}{\text{d}\phi} =
\frac{\text{d}\zeta}{\text{d}x}= 0$, or equivalently

\vspace{-15pt}\begin{equation}\label{x_0}
{x_0}_i = x|_{\zeta=0} = \frac{-c_2 \pm \sqrt{c_2^2 - 3 c_1 c_3}}{3 c_3} \,,
\end{equation}

in each region $[\phi_i,\phi_{i+1}]$ unless $c_3 = 0$, in which case the
extremum is at ${x_0}_i = -\frac{c_1}{2 c_2}$, and it is a maximum if $c_2 <
0$.  If $c_3 = 0$ and $c_2 > 0$ the extremum in that region is at either $x=0$
or $x=1$, which isn't interesting since we already have $\phi_i$, $\phi_{i+1}$,
$\zeta(\phi_i)$ and $\zeta(\phi_{i+1})$.

Assuming $c_3 \neq 0$, and seeking a maximum, we want

\vspace{-15pt}\begin{equation}
\left.\frac{\text{d}^2\zeta}{\text{d}x^2}\right|_{x={x_0}_i} = 2(c_2 + 3 c_3
{x_0}_i) = \pm 2 \sqrt{c_2^2 - 3 c_1 c_3} < 0\,, \end{equation}

so we choose the negative sign in Equation (\ref{x_0}).  If ${x_0}_i < 0$ or
${x_0}_i > 1$, the maximum in the region is at $x=0$ or $x=1$, which is again
uninteresting.

If $\sup\{\zeta({x_0}_i)\} > \sup\{\zeta(x_i)\}$, add $\phi({x_0}_i)$ and
$\zeta({x_0}_i)$ to $\{\phi_i\}$ and $\{\zeta_i\}$.

\label{lastpage}
\end{document}
% $Id$
