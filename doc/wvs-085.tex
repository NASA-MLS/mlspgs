\documentclass[11pt]{article}
\usepackage[fleqn]{amsmath}\textwidth 6.5in
\oddsidemargin -0.25in
%\evensidemargin -0.5in
\topmargin -0.25in
\textheight 9.00in

\newcommand{\docname}{\bf wvs-085}
\newcommand{\docdate}{16 September 2009}

\usepackage{graphicx}

\usepackage{floatflt}

\begin{document}

%\tracingcommands=1
\newlength{\hW} % heading box width
\newlength{\pW} % page number field width
\settowidth{\hW}{\docname}
\settowidth{\pW}{Page \pageref{lastpage}\ of \pageref{lastpage}}
\ifdim \pW > \hW \setlength{\hW}{\pW} \fi
\makeatletter
\def\@biblabel#1{#1.}
\newcommand{\ps@twolines}{%
  \renewcommand{\@oddhead}{%
    \docdate\hfill\parbox[t]{\hW}{{\hfill\docname}\newline
                          Page \thepage\ of \pageref{lastpage}}}%
\renewcommand{\@evenhead}{}%
\renewcommand{\@oddfoot}{}%
\renewcommand{\@evenfoot}{}%
}%
\makeatother
\pagestyle{twolines}

\vspace{-10pt}
\begin{tabbing}
\phantom{References: }\= \\
To: \>Van\\
Subject: \>Scattered brightness temperature\\
From: \>Van\\
\end{tabbing}

\parindent 0pt \parskip 4pt
\vspace{-20pt}

\section{Introduction}

We wish to calculate the intensity of radiation scattered from a point to
an observer.

The intensity at the scattering point is measured by brightness
temperature, denoted $T_b$.  This is considered to be incident upon the
scattering point from all directions.

The amount of radiation scattered to the observer from a particular
direction is the product of the incident $T_b$ from that direction and
the Mie phase function $P(\xi)$, where $\xi$ is the angle between the
incident ray and the ray scattered to the observer.  The radiation
observed is the integral over the unit sphere of this product, i.e.,

\begin{equation}
\frac1{4\pi} \oint_S \text{d} \Omega\, T_b(\Omega) P(\xi)\,.
\end{equation}

\section{One obvious co\"ordinate system}

The co\"ordinate system is a polar co\"ordinate system, oriented so the
observer is on the positive $x$ axis and the north pole is on the
positive $z$ axis.  The connection with MLS geometry is that the $x-y$
plane is the orbit plane, the positive $x$ axis is the direction of
the limb tangent, and the positive $y$ axis is the zenith direction.

Consider first the case of integrating around a circle of latitude, then
sweeping the latitudes from pole to pole.

Let $\phi$ be the angle about the $z$ axis, in the $x-y$ plane, measured
anticlockwise from the positive $x$ axis as viewed from the positive $z$
axis, i.e., longitude.  Let $\theta$ be the complement of the angle from
the positive $z$ axis, i.e., latitude. In this co\"ordinate system,  the
surface area element d$\Omega = \text{d}\phi\,\text{d}\theta\cos\theta$
and a point on the sphere is given by the vector $(\cos\theta\cos\phi,
\cos\theta\sin\phi, \sin\theta)^T$.  The cosine of the angle between this
vector and the vector to the observer at $(1,0,0)^T$ is the inner product
of these two vectors, i.e., $\cos\theta\cos\phi$, so $\xi=
\cos^{-1}(\cos\theta \cos\phi)$.  Retaining $\xi$ without explicit
$\theta$ and $\phi$ dependence, and formally exchanging the order of
integration, the integral becomes

\begin{equation}\label{two}
\frac1{4\pi} \oint_S \text{d} \Omega\, T_b(\Omega) P(\xi) =
\frac1{4\pi} \int_{-\pi}^\pi \text{d} \phi
 \int_{-\frac\pi2}^\frac\pi2 \text{d} \theta\, T_b(\theta,\phi) P(\xi) \cos\theta\,.
\end{equation}

We are only able to calculate $T_b$ in the $x-y$ plane, so we must assume
some sort of symmetry.

From the point of view of the scattering point, each integration along a
circle of latitude projects that circle into a circle in the equatorial
plane, where we can calculate $T_b(0,\phi)$.

When $\theta=0$ we want $T_b$ to be as calculated in the $x-y$ plane. 
When $\theta=\pm\frac\pi2$ we want the average contribution of $T_b$
throughout the $x-y$ plane.  Between these extremes we want a monotone
interpolation.  Thus a symmetric form for $T_b$ is

\begin{equation}
T_b(\theta,\phi) \mapsto T_b(\phi) f(\theta) + \frac{1-f(\theta)}{2\pi}
 \int_{-\pi}^\pi \text{d} \phi^\prime\, T_b(\phi^\prime)
\end{equation}

where $f(0)=1$, $f\left(\frac\pi2\right)=0$ and $f(\theta) = f(-\theta)$. 
With this substitution in Equation (\ref{two}) we have

\begin{equation}\begin{split}
&\frac1{4\pi} \int_{-\pi}^\pi \text{d}\phi\, T_b(\phi)
  \int_{-\frac\pi2}^\frac\pi2 \text{d}\theta\, f(\theta) P(\xi)
  \cos\theta +\\
&\frac1{8\pi^2}\int_{-\pi}^\pi \text{d}\phi^\prime T_b(\phi^\prime)
 \int_{-\pi}^\pi \text{d}\phi\,
  \int_{-\frac\pi2}^\frac\pi2 \text{d}\theta\, (1-f(\theta))
   P(\xi) \cos\theta\\
\end{split}\end{equation}

One reasonable choice is $f(\theta)=\cos\theta$.  An alternative, but
probably unphysical symmetry, is $f(\theta)=1$.

\section{Another co\"ordinate system}

If we think of the integration as first around the ``equator,'' then
rotate about the $x$ axis (so we now have a different equator for each
pole position), the argument for $P$ is always the longitude along the
current equator from the observer.  Writing this with the ``around the
equator'' integration on the inside, and then formally exchanging the
order of integration, we have

\begin{equation}
\frac1{4\pi} \int_{-\frac\pi2}^\frac\pi2 \text{d}\theta\,
 \int_{-\pi}^\pi \text{d}\phi\, T_b(\theta,\phi) P(\phi) \cos\theta =
\frac1{2\pi} \int_{-\pi}^\pi \text{d}\phi\, P(\phi)\,
 \frac12 \int_{-\frac\pi2}^\frac\pi2 \text{d}\theta\, T_b(\theta,\phi)
  \cos\theta\,,
\end{equation}

where the normalizations for the inner and outer integrals are separated
in the right-hand side.  The problem here is that we can only calculate
$T_b(0,\phi) = T_{b0}(\phi)$, so we need to make a symmetry assumption
that allows us to estimate $T_b(\theta,\phi)$.

We are not assuming a horizontally isotropic atmosphere, so we cannot
assume $T_{b0}(\phi) = T_{b0}(\pi-\phi)$.  When $\theta=0$ we want
$T_b(\theta,\phi) \mapsto T_{b0}(\phi) = T_b(0,\phi)$.  When
$\theta=\frac\pi2$ we want some linear combination of $T_{b0}(0)$ and
$T_{b0}(\pi)$.  In particular when $\phi=\frac\pi2$ we want
$T_b(\frac\pi2,\frac\pi2) = \frac12(T_{b0}(0) + T_{b0}(\pi))$.  For
intermediate points, we want some linear combination of
$T_{b0}(\phi^\prime)$ and $T_{b0}(\pi-\phi^\prime)$, where $\phi^\prime$
is at the same angle from the spacecraft zenith (the positive $y$ axis in
our coordinate system) as $\phi$, i.e., $\phi^\prime =
\sin^{-1}(\sin\phi\cos\theta)$, in the same quadrant as $\phi$.  Putting
everything together we have

\begin{equation}
\frac1{2\pi} \int_{-\pi}^\pi \text{d}\phi\, P(\phi)\,
 \overline{T}_{b0}(\phi)\,
\end{equation}

where the inner integral is

\begin{equation}
\overline{T}_{b0}(\phi)=
  \frac1{4\pi^2}\int_{-\frac\pi2}^\frac\pi2 \text{d}\theta\,
   \left[
   \cos\theta\, (1+\cos\phi)\, T_{b0}(\phi^\prime) +
   (1-\cos\theta)\, (1-\cos\phi)\, T_{b0}(\pi-\phi^\prime)
   \right] \cos\theta
\end{equation}
\label{lastpage}
\end{document}

% $Id$

% $Log$
