\documentclass[11pt]{article}
\usepackage{alltt}
\usepackage[fleqn]{amsmath}
\usepackage{floatflt}
\usepackage{graphicx}
\usepackage{longtable}
\usepackage[strings]{underscore}

\textwidth 6.5in
\oddsidemargin -0.25in
%\evensidemargin -0.5in
\topmargin -0.5in
\textheight 9in

\newcommand{\docname}{wvs-115}
\newcommand{\docdate}{13 August 2013}

\ifx\pdfoutput\undefined
  \pdfoutput=0
  \usepackage[hypertex,plainpages,hyperindex=true]{hyperref}
  \hypersetup{%
    hypertexnames=false%
  }
  % Specify the driver for the color package
  \ExecuteOptions{dvips}
  %\ExecuteOptions{xdvi}
\else
  \ifnum\pdfoutput>0
    \usepackage[pdftex,plainpages,hyperindex=true,pdfpagelabels]{hyperref}
    \hypersetup{%
      hypertexnames=false,%
      colorlinks=true,%
      linktocpage=true,%
    }
    % Specify the driver for the color package
    \ExecuteOptions{pdftex}
  \else
    \usepackage[hypertex,plainpages,hyperindex=true]{hyperref}
    \hypersetup{%
      hypertexnames=false%
    }
    % Specify the driver for the color package
    \ExecuteOptions{dvips}
    %\ExecuteOptions{xdvi}
  \fi
\fi

\hyperbaseurl{}
\newcommand\hr[1]{\href{#1.dvi}{dvi}, \href{#1.pdf}{pdf}}
\newcommand\h[1]{#1 (\hr{#1})}

\begin{document}

%\tracingcommands=1
\newlength{\hW} % heading box width
\newlength{\pW} % page number field width
\settowidth{\hW}{\bf\docname}
\settowidth{\pW}{Page \pageref{lastpage}\ of \pageref{lastpage}}
\ifdim \pW > \hW \setlength{\hW}{\pW} \fi
\makeatletter
\def\@biblabel#1{#1.}
\newcommand{\ps@twolines}{%
  \renewcommand{\@oddhead}{%
    \docdate\hfill\parbox[t]{\hW}{{\hfill\bf\docname}\newline
                          Page \thepage\ of \pageref{lastpage}}}%
\renewcommand{\@evenhead}{}%
\renewcommand{\@oddfoot}{}%
\renewcommand{\@evenfoot}{}%
}%
\makeatother
\pagestyle{twolines}

\vspace{-10pt}
\begin{tabbing}
\phantom{References: }\= \\
To: \>Nathaniel, Bill, Paul\\
Subject: \>Out-of-orbit-plane modeling\\
From: \>Van Snyder\\
%Reference: \> \\
\end{tabbing}

\parindent 0pt \parskip 6pt
\vspace{-10pt}

For out-of-orbit-plane modeling, for now only for UARS, either the state
vector or the extra vector needs to have a quantity of type {\tt azimuth}.
Only the first element is examined.  It is the angle in degrees,
counterclockwise (looking downward) from the spacecraft velocity vector,
of the plane in which modeling is to take place.

If the angle is not zero (that's all the wrapper cares about so far), and
if the {\tt polarized} field of the configuration is true, the quantities
of type {\tt magneticField}, {\tt RefGPH} and {\tt SCVelECR} must appear
in either the state vector or the extra vector.

The {\tt magneticField} quantity doesn't need to have a value, but it does
need to have {\tt hGrid} and {\tt vGrid}.  The {\tt hGrid} ought to be an
explicit {\tt hGrid}, which needs only to have a {\tt geodAngle} field. 
This is a list of angles at which to evaluate the magnetic field.  The
{\tt vGrid} specifies the vertical co\"ordinate.  The {\tt vGrid}
co\"ordinate type needs to be {\tt zeta}.

The MIF number from which to compute the profile and spacecraft
geolocation (lat,lon) can be specified by a {\tt ReferenceMIF} field in
the forward model configuration.  The default is 1.  The location of the
profile is the (lat,lon) of the reference MIF in the {\tt PTan} quantity,
and the location of the spacecraft is the (lat,lon) of the reference MIF
in the {\tt SCVelECR} quantity, both for the current MAF number, which we
assume L1 will have co\"ordinated correctly.

For now (i.e., for UARS), the model plane is not computed from the {\tt
azimuth} field of the forward model configuration.  Rather, it is the
plane defined by the {\tt PTan} and {\tt SCVelECR} reference MIF
geolocations.  All that's computed in that plane (so far) is the magnetic
field.

The {\tt azimuth} quantity might turn out to be the wrong thing for SMLS. 
Maybe, like UARS, the model plane will be defined by a geolocation in
something like a minor-frame quantity.

The magnetic field model wants $(x,y,z)$, not (lat,lon,$\zeta$), so
vectors are computed in the model plane at positions defined by the {\tt
hGrid} used to define the {\tt magneticField} quantity.  Zero is the
geolocation of the {\tt PTan} reference MIF.  Negative values are nearer
to the spacecraft, along the great circle in the model plane.  The
computed (lat,lon) positions are stored in the magnetic field quantity's
template.  The magnetic field values are computed by the {\tt
UsingMagneticField} subroutine in the {\tt FillUtils_1} module.  It gets
heights from the MAF instance of the {\tt RefGPH} quantity, which we
assume to apply at the {\tt PTan} geolocation and have the same $\zeta$s
as {\tt PTan}, and (since we're doing 1-D retrieval for UARS), also at the
spacecraft geolocation.

Examples:

{\tt\begin{verbatim}
hGridMag: hGrid, type=explicit, geodAngle=[-3.0, -1.5, 0, 1.5, 3.0]
magQ: Quantity, type=magneticField, hGrid=hGridMag, vGrid=vGridStandard
\end{verbatim}}

\label{lastpage}
\end{document}

% $Id$

% $Log$
% Revision 1.1  2013/07/25 00:30:52  vsnyder
% Initial commit
%
