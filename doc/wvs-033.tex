\documentclass[11pt]{article}
\usepackage[fleqn]{amsmath}\textwidth 6.5in
\oddsidemargin -0.25in
%\evensidemargin -0.5in
\topmargin -0.6in
\textheight 9.20in

\newcommand{\docname}{\bf wvs-033}
\newcommand{\docdate}{2 May 2006}

\begin{document}

%\tracingcommands=1
\newlength{\hW} % heading box width
\newlength{\pW} % page number field width
\settowidth{\hW}{\docname}
\settowidth{\pW}{Page \pageref{lastpage}\ of \pageref{lastpage}}
\ifdim \pW > \hW \setlength{\hW}{\pW} \fi
\makeatletter
\def\@biblabel#1{#1.}
\newcommand{\ps@twolines}{%
  \renewcommand{\@oddhead}{%
    \docdate\hfill\parbox[t]{\hW}{{\hfill\docname}\newline
                          Page \thepage\ of \pageref{lastpage}}}%
\renewcommand{\@evenhead}{}%
\renewcommand{\@oddfoot}{}%
\renewcommand{\@evenfoot}{}%
}%
\makeatother
\pagestyle{twolines}

\vspace{-10pt}
\begin{tabbing}
\phantom{References: }\= \\
To: \>Bill\\
Subject: \>Path integrations\\
From: \>Van Snyder\\
\end{tabbing}

\parindent 0pt \parskip 6pt
\vspace{-20pt}

\section{The problem}

The full forward model currently uses either a trapezoid rule or a
three-point Gauss-Legendre rule to integrate $\beta$ over one layer of
the line-of-sight path, depending upon the relationship between a
specified tolerance and a first-order estimate of the error in the
trapezoid rule. When the Gauss-Legendre rule is used, the function values
at the end points, which determined the value of the trapezoidal rule,
are not used.

\section{Proposed solution}

We could get almost the same result with less computational effort by
using either a trapezoid rule or a four-point Lobatto rule, or get a far
better result with the same computational effort by using a five-point
Lobatto rule.  An $n$ point Lobatto rule uses the function values at the
end points of the interval, that is, the values used to evaluate the
trapezondal rule, plus $n-2$ additional values at  points optimally
selected within the interval.

The algebraic order (the order of polynomial integrated exactly) is $2n$
for an $n$ point Gauss-Legendre rule, or $2n-2$ for an $n$-point Lobatto
rule.  Thus, an $n+1$ point Lobatto rule has the same algebraic order as
an $n$ point Gauss-Legendre rule.

The coefficient of the remainder for an $n$-point Lobatto rule is

\begin{equation}
R_L(n) =
\frac{-n (n-1)^3 2^{2 n - 1} [ ( n-2 )! ]^4}
     {(2 n - 1 ) [ ( 2 n - 2 ) ! ]^3}
\end{equation}

The coefficient of the remainder for an $n$-point Gauss-Legendre rule is

\begin{equation}
R_G(n) =
\frac{2^{2 n + 1} (n!)^4}
     {( 2 n + 1 ) ( [ ( 2 n ) ! ]^3}
\end{equation}

Thus $R_L(4)/R_G(3)$ is $-4/3$, that is, the Lobatto rule has the same
algebraic order (6) and almost the same error coefficient, with less work
(four function evaluations instead of five), and $R_L(5)/R_G(3)$ is
$-5/882$, that is, the Lobatto rule has a higher algebraic order (8
instead of 6) and a far smaller error coefficient, with the same amount
of work (five function evaluations in both cases).

The singularity-correction scheme

\begin{equation}
\int_{\zeta_i}^{\zeta_{i+1}} G(\zeta) \frac{\text{d}s}{\text{d}h}
  \frac{\text{d}h}{\text{d}\zeta}\, \text{d} \zeta =
G(\zeta_i) \int_{\zeta_i}^{\zeta_{i+1}} \frac{\text{d}s}{\text{d}h}
  \frac{\text{d}h}{\text{d}\zeta}\, \text{d} \zeta +
\int_{\zeta_i}^{\zeta_{i+1}} (G(\zeta)-G(\zeta_i)) \frac{\text{d}s}{\text{d}h}
  \frac{\text{d}h}{\text{d}\zeta}\, \text{d} \zeta
\end{equation}

continues to work, because it is based
upon a rectangle-rule estimate.

\label{lastpage}
\end{document}
% $Id$
