\documentclass[11pt]{article}
\usepackage[fleqn]{amsmath}

\textwidth 6.5in
\oddsidemargin -0.25in
%\evensidemargin -0.5in
\topmargin -0.25in
\textheight 9.0in

\newcommand{\docname}{\bf wvs-099}
\newcommand{\docdate}{3 December 2010}

\ifx\pdfoutput\undefined
  \pdfoutput=0
  \usepackage[hypertex,plainpages,hyperindex=true]{hyperref}
  \hypersetup{%
    hypertexnames=false%
  }
  % Specify the driver for the color package
  \ExecuteOptions{dvips}
  %\ExecuteOptions{xdvi}
\else
  \ifnum\pdfoutput>0
    \usepackage[pdftex,plainpages,hyperindex=true,pdfpagelabels]{hyperref}
    \hypersetup{%
      hypertexnames=false,%
      colorlinks=true,%
      linktocpage=true,%
    }
    % Specify the driver for the color package
    \ExecuteOptions{pdftex}
  \else
    \usepackage[hypertex,plainpages,hyperindex=true]{hyperref}
    \hypersetup{%
      hypertexnames=false%
    }
    % Specify the driver for the color package
    \ExecuteOptions{dvips}
    %\ExecuteOptions{xdvi}
  \fi
\fi

\begin{document}

%\tracingcommands=1
\newlength{\hW} % heading box width
\newlength{\pW} % page number field width
\settowidth{\hW}{\docname}
\settowidth{\pW}{Page \pageref{lastpage}\ of \pageref{lastpage}}
\ifdim \pW > \hW \setlength{\hW}{\pW} \fi
\makeatletter
\def\@biblabel#1{#1.}
\newcommand{\ps@twolines}{%
  \renewcommand{\@oddhead}{%
    \docdate\hfill\parbox[u]{\hW}{{\hfill\docname}\newline
                          Page \thepage\ of \pageref{lastpage}}}%
\renewcommand{\@evenhead}{}%
\renewcommand{\@oddfoot}{}%
\renewcommand{\@evenfoot}{}%
}%
\makeatother
\pagestyle{twolines}

\renewcommand{\d}{\text{d}}
\newcommand{\T}{\mathcal{T}}

\vspace{-10pt}
\begin{tabbing}
\phantom{References: }\= \\
To: \>Van\\
Subject: \>Variational derivatives of first-order linear differential equations\\
From: \>Van Snyder\\
\end{tabbing}

\parindent 0pt \parskip 5pt
\vspace{-20pt}

\section{Mathematical generalities}

Consider a function $F(X)$ where $X = \{x_1, \dots, x_n\}$ is a finite set
of continuous independent variables (also known as ``discrete
parameters''), with Maclaurin series $c + \sum_j a_j x_j \Delta j +
\sum_{jk} b_{jk} x_j x_k \Delta j \Delta k + \dots$, where $\Delta j=1$
and $\Delta k=1$ are inserted to make the parallel with Equation
(\ref{VAR}) clearer.  The variation $\delta F(X)$ due to variation $\delta
X = \{\delta x_j\}$ of elements of its argument set is $\delta F(X) =
\sum_j a_j \delta x_j \Delta j + O(\delta x_j^2)$.  From this, in the
limit $\delta x_j \rightarrow 0$, it is obvious that the sensitivities of
$F$ due to variations in its arguments are the partial derivatives

\begin{equation}
\frac{\delta F(X)}{\delta x_j} \equiv
 \frac{\partial F(X)}{\Delta j\partial x_j} = a_j\,.
\end{equation}

If we carry this over to a functional $F[X(\xi)]$ of a continuum $X(\xi)$
of independent variables instead of a function of the finite set
$\{x_1,\dots,x_n\}$, i.e., $X(\xi)$ is a function of the continuous
variable $\xi$ instead of a finite set indexed by the discrete index $j$,
the summations in the Maclaurin series are replaced with integrals,
\emph{viz.}

\begin{equation}
F[X] = c + \int_{\xi_0}^{\xi_1} a(\xi) X(\xi) \,\d \xi +
 \int_{\xi_0}^{\xi_1} \int_{\xi_0}^{\xi_1} b(\xi,\xi^\prime) X(\xi)
 X(\xi^\prime) \d \xi \d \xi^\prime + \dots\,.
\end{equation}

Its variation due to variation $\delta X(\xi)$ of its argument function
(also known as ``continuous parameter'') $X(\xi)$, for small variation
$\delta X(\xi)$ (this is variation of the function $X(\xi)$, not variation
of its value), is

\begin{equation}\label{VAR}
\delta F[X] = \int_{\xi_0}^{\xi_1} a(\xi) \,\delta X(\xi) \,\d \xi
  + O(\delta X(\xi)^2)
 = \lim_{\sup \{\Delta \xi_i\} \rightarrow 0 \atop n \rightarrow \infty}
  \sum_{i=1}^n a(\xi_i) \delta X(\xi_i) \Delta \xi_i + O(\delta X(\xi)^2)
\,.
\end{equation}

Parallel to the case of a function of a discrete set, its sensitivity due
to variation $\delta X(\xi)$ of its argument function $X(\xi)$,
throughout the range $\xi_0 \leq \xi \leq \xi_1$, is

\begin{equation}\label{SENS}
\frac{\delta F[X]}{\delta X(\xi)}
 \equiv \lim_{\Delta \xi \rightarrow 0 \atop \delta X(\xi) \rightarrow 0}
  \frac{\partial F[X]}{\Delta \xi \partial X(\xi)}
 = \lim_{\Delta \xi \rightarrow 0 \atop \delta X(\xi) \rightarrow 0}
  \frac{a(\xi) \delta X(\xi) \Delta \xi}{\delta X(\xi) \Delta \xi}
 = a(\xi)\,,
\end{equation}

where here $\xi$ is the continuous independent variable of the
continuum $X(\xi)$ while above $j$ is the discrete index of the discrete
set $X=\{x_j\}$, and here $a(\xi)$ is a function while above $\{a_j\}$
is a discrete set.  Also see pages 27-29 of {\bf Calculus of Variations}
by I. M. Gelfand and S. V. Fomin.  We are interested in this variation,
not in finding $X(\xi)$ so that the variation is zero, so we do not need
to consider the Euler-Lagrange equation.

%=========================================================================

\section{Linear first-order differential equations}

Consider the linear first-order differential equation

\begin{equation}\label{diff eq}
y^\prime(x) + f(x) y(x) = g(x)\,.
\end{equation}

The solution can be written in the form

\begin{equation}\label{one}
y(x) = \int_{x_0}^x \T(x^\prime,x) g(x^\prime) \, \d x^\prime +
 \T(x_0,x) y(x_0)
\end{equation}

where

\begin{equation}\label{two}
\T(x^\prime,x) = \exp\left( -\int_{x^\prime}^x f(\xi) \,\d \xi \right) \,.
\end{equation}

%=========================================================================

\section{Sensitivity of $y(x)$ to variation of $f(x)$}

The variation of $y(x)$ as a result of variation of $f(x)$ is

\begin{equation}\label{var y}
\delta_f y(x) = \delta_f \T(x_0,x) y(x_0) +
 \int_{x_0}^x \delta_f \T(x^\prime,x) g(x^\prime) \, \d x\,.
\end{equation}

To compute $\delta_f \mathcal{T}(x^\prime)$, replace $f(\xi)$ in Equation
(\ref{two}) by $f(\xi) + \delta f(\xi)$ and compute the
difference from Equation (\ref{two}):

\begin{equation}\begin{split}\label{tau vary}
\delta_f \T(x^\prime,x)
=\,& \exp\left( - \int_{x^\prime}^{s_m} (f(\xi) +
                                       \delta f(\xi) )
         \, \d \xi \right) -
     \exp\left( - \int_{x^\prime}^{s_m} f(\xi)
         \, \d \xi \right) \\
=\,& \exp\left( - \int_{x^\prime}^{s_m} f(\xi)
         \, \d \xi \right)
     \exp\left( - \int_{x^\prime}^{s_m} \delta f(\xi)
         \, \d \xi \right) -
     \exp\left( - \int_{x^\prime}^{s_m} f(\xi)
         \, \d \xi \right) \\
=\,& \exp\left( - \int_{x^\prime}^{s_m} f(\xi)
         \, \d \xi \right)
     \left(
       \exp\left( - \int_{x^\prime}^{s_m} \delta f(\xi)
         \, \d \xi \right) -1
     \right) \\
=\,& \exp\left( - \int_{x^\prime}^{s_m} f(\xi)
         \, \d \xi \right)
     \left(1 - \int_{x^\prime}^{s_m} \delta f(\xi) +
           O\left( \left( \int_{x^\prime}^{s_m} \delta f(\xi)
                           \, \d \xi
            \right)^2 \right) - 1 \right)\,.
\end{split}\end{equation}

Therefore, in the limit for small perturbation $\delta f(\xi)$

\begin{equation}\label{tau var}
\delta_f \T(x^\prime,x) =
 \exp \left( - \int_{x^\prime}^{s_m} f(\xi) \,\d \xi \right)
 \cdot
 \left( - \int_{x^\prime}^{s_m} \delta f(\xi) \,\d \xi \right)
=
 -\T(x^\prime,x) \int_{x^\prime}^{s_m} \delta f(\xi) \,\d \xi\,.
\end{equation}

Assume $g(x^\prime)$ depends upon $f(x^\prime)$ and substitute Equation
(\ref{tau var}) into the integral in Equation (\ref{var y}) (exchanging
the order of integration in the penultimate step) to get

\begin{equation}\begin{split}\label{var1}
 \int_{x_0}^{x} \delta_f (\T(\xi,x) \, g(\xi)) \, \d \xi
=\,&
 \int_{x_0}^{x} \T(\xi,x) \delta_f (g(\xi)) + \delta_f (\T(\xi,x)) \, \d
 \xi \\
=\,&
   \int_{x_0}^{x} \d \xi \T(\xi,x)
     \frac{\delta g(\xi)}{\delta f(\xi)} \delta f(\xi)
  -\int_{x_0}^{x} \d \xi \,\T(\xi,x)\,
    g(\xi) 
    \int_\xi^{x} \d \xi^\prime\, \delta
    f(\xi^\prime) \\
=\,&
 \int_{x_0}^{x} \d \xi\, \delta f(\xi)
   \left[ \T(\xi,x) \frac{\delta g(\xi)}{\delta f(\xi)} -
   \int_{x_0}^\xi \d \xi^\prime \, \T(\xi^\prime,x) \,
    g(\xi^\prime) \right] \\
=\,&
 \int_{x_0}^{x} \d \xi\, \delta f(\xi)
   \T(\xi,x) \left[ \frac{\delta g(\xi)}{\delta f(\xi)} -
   \int_{x_0}^\xi \d \xi^\prime \, \T(\xi^\prime,\xi) \,
    g(\xi^\prime) \right]
\,.
\end{split}\end{equation}

Substituting Equation (\ref{tau var}) into the first term for $\delta_f
y$ in Equation (\ref{var y}) gives

\begin{equation}\label{var2}
\delta_f \T(x_0,x) y(x_0)
= \T(x_0,x) y(x_0)
  \int_{x_0}^{x} \delta f(\xi)\, \d \xi\,.
\end{equation}

Collecting Equations (\ref{var1}) and (\ref{var2}) and factoring, the
resulting variation in $y$ due to variation in $f(\xi)$ is

\begin{equation}
\delta_f y(x)
= \int_{x_0}^{x} \d \xi \,\delta f(x)
 \left\{\T(\xi,x)
    \left[ \frac{\delta g(\xi)}{\delta f(\xi)} -
     \int_{x_0}^\xi \T(\xi^\prime,\xi) \,
    g(\xi^\prime)
     \d \xi^\prime
 - \T(x_0,\xi) y(x_0) \right] \right\}
\,.
\end{equation}

According to Equations (\ref{VAR}) and (\ref{SENS}), the sensitivity of
$y$ to variation $\delta f(\xi)$ in $f(\xi)$ is the factor between \{ and
\}, \emph{viz.}

\begin{equation}\label{y var alpha}\boxed{
\frac{\delta y(x)}{\delta f(\xi)} =
 \T(\xi,x) \left[
 \frac{\delta g(\xi)}{\delta f(\xi)} -
  \int_{x_0}^\xi \T(\xi^\prime,\xi) \, g(\xi^\prime) \, \d \xi^\prime
  - \T(x_0,\xi) y(x_0) \right] = \T(\xi,x) \left[
  \frac{\delta g(\xi)}{\delta f(\xi)} -  y(\xi) \right] }\,.
\end{equation}

%=========================================================================

\section{Alternative derivation of sensitivity of $y(x)$ to variation of
$f(x)$}

Taking the partial derivative of Equation (\ref{diff eq}) with respect to
$f(\xi)$ at some single point $\xi$, where $x_0 \leq \xi
\leq x$, so we're not trying to take the derivative with respect to a
function, but rather with respect to a variable, we have

\begin{equation}\label{four}
\frac{\partial}{\partial f(\xi)} \frac{\d}{\d x} y(x) +
 \frac{\partial f(x)}{\partial f(\xi)} y(x) +
  f(x) \frac{\partial y(x)}{\partial f(\xi)}
= \frac{\partial}{\partial f(\xi)} g(x)
= \frac{\partial f(x)}{\partial f(\xi)}
   \frac{\partial g(x)}{\partial f(x)}\,.
\end{equation}

Replacing $\frac{\partial f(x)}{\partial f(\xi)}$ by
the Dirac $\delta$ function $\delta(x-\xi)$ and exchanging the order of
differentiation, we have

\begin{equation}\begin{split}\label{five}
\frac{\d}{\d x} \frac{\partial y(x)}{\partial f(\xi)} +
 \delta(x-\xi) y(x) +
  f(x) \frac{\partial y(x)}{\partial f(\xi)}
=\,& \delta(x-\xi) \frac{\partial g(x)}{\partial f(x)} \text{ or}\\
\frac{\d}{\d x} \frac{\partial y(x)}{\partial f(\xi)} +
 f(x) \frac{\partial y(x)}{\partial f(\xi)}
=\,&
 \delta(x-\xi) \left(
  \frac{\partial g(x)}{\partial f(x)}-y(x) \right)\,.
\end{split}\end{equation}

Using the initial condition $\frac{\partial y(s_0)}{\partial f(\xi)}=0$
the solution for Equation (\ref{five}) can be written

\begin{equation}\begin{split}
\frac{\partial y(x)}{\partial f(\xi)}
=\,& \int_{s_0}^x \delta(x^\prime-\xi)
  \left(\frac{\partial g(x^\prime)}{\partial f(x^\prime)}-y(x^\prime)
  \right)
   \T(x^\prime,x)\, \d x^\prime \\
  \,&\\
=\,& \left ( 
 \frac{\partial g(\xi)}{\partial f(\xi)}-y(\xi) \right )
  \T(\xi,x)\,,
\end{split}\end{equation}

which is the same result as Equation (\ref{y var alpha}).

%=========================================================================

\section{Sensitivity of $y(x)$ to variation of $g(x)$}

The variation of Equation (\ref{one}) as a consequence of variation
$\delta g(\xi)$ of $g(\xi)$ is

\begin{equation}
\delta_g y(x) = \int_{x_0}^x \T(\xi,x) \delta g(\xi) \, \d \xi\,.
\end{equation}

According to Equations (\ref{VAR}) and (\ref{SENS}), the sensitivity of
$y$ to variation $\delta g(\xi)$ in $g(\xi)$ is

\begin{equation}\label{y var g}\boxed{
\frac{\delta y(x)}{\delta g(\xi)} = \T(\xi,x) }\,.
\end{equation}

%=========================================================================

\section{Alternative derivation of sensitivity of $y(x)$ to variation of
$g(x)$}

Taking the partial derivative of Equation (\ref{diff eq}) with respect to
$g(\xi)$ at some single point $\xi$, where $x_0 \leq \xi
\leq x$, so we're not trying to take the derivative with respect to a
function, but rather with respect to a variable, we have

\begin{equation}\label{four g}
\frac{\partial}{\partial g(\xi)} \frac{\d}{\d x} y(x) +
 \frac{\partial f(x)}{\partial g(\xi)} y(x) +
  f(x) \frac{\partial y(x)}{\partial g(\xi)}
= \frac{\partial g(x)}{\partial g(\xi)}\,.
\end{equation}

Replacing $\frac{\partial f(x)}{\partial g(\xi)}$ by zero and
$\frac{\partial g(x)}{\partial g(\xi)}$ by  the Dirac $\delta$
function $\delta(x-\xi)$ and exchanging the order of differentiation,
we have

\begin{equation}\begin{split}\label{five g}
\frac{\d}{\d x} \frac{\partial y(x)}{\partial g(\xi)} +
  f(x) \frac{\partial y(x)}{\partial g(\xi)}
=\,& \delta(x-\xi) \,.
\end{split}\end{equation}

Using the initial condition $\frac{\partial y(s_0)}{\partial g(\xi)}=0$
the solution for Equation (\ref{five g}) can be written

\begin{equation}\begin{split}
\frac{\partial y(x)}{\partial g(\xi)}
=\,& \int_{s_0}^x \delta(x^\prime-\xi)
   \T(x^\prime,x)\, \d x^\prime =\T(\xi,x)\,,
\end{split}\end{equation}

which is the same result as Equation (\ref{y var g}).

%=========================================================================

\section{Sensitivity of $y(x)$ to variation of parameters in $f(x)$ and
$g(x)$}

If we now assume that $f(x)$ and $g(x)$ depend upon a continuous
parameter $p(\xi)$, i.e., $f(x) \mapsto f(x;p(\xi))$ and $g(x) \mapsto
g(x;p(\xi))$, and we want the sensitivity of $y(x)$ to variation in
$p(\xi)$ we have

\begin{equation}\begin{split}
\frac{\delta y(x)}{\delta p(\xi)} =\,&
 \frac{\delta y(x)}{\delta f(x;p(\xi))}
  \frac{\delta f(x;p(\xi))}{\delta p(\xi)}
  +\frac{\delta y(x)}{\delta g(x;p(\xi))}
   \frac{\delta g(x;p(\xi))}{\delta p(\xi)}\\
=\,&
  \left[ -\frac{\delta f(x;p(\xi))}{\delta p(\xi)} y(\xi)
  + \frac{\delta g(x;p(\xi))}{\delta p(\xi)} \right] \T(\xi,x) \,.
\end{split}\end{equation}

If $p(\xi)$ is instead replaced by a set of discrete parameters $P =\{p_1,
p_2, \dots p_n\} = \{p_{\xi_1}, p_{\xi_2}, \dots p_{\xi_n}\}$ we have

\begin{equation}\begin{split}
\frac{\delta y(x)}{\delta p_i} =\,&
 \frac{\delta y(x)}{\delta f(x;p_i)}
  \frac{\partial f(x;p_i)}{\partial p_i}
  +\frac{\delta y(x)}{\delta g(x;p_i)}
   \frac{\partial g(x;p_i)}{\partial p_i}\\
=\,&
  -\left[ \frac{\partial f(x;p_i)}{\partial p_i} y(\xi)
  + \frac{\partial g(x;p_i)}{\partial p_i} \T(\xi,x) \right] \,.
\end{split}\end{equation}

That is, one does not need to integrate

\begin{equation}\begin{split}
\frac{\d}{\d x} \frac{\delta y(x)}{\delta p(\xi)} +
 \frac{\delta f(x;p(x))}{\delta p(\xi)} y(x) +
  f(x;p(x)) \frac{\delta y(x)}{\delta p(\xi)}
=\,& \frac{\delta g(x;p(x))}{\delta p(\xi)} \text{ eqv.} \\
\frac{\d}{\d x} \frac{\delta y(x)}{\delta p(\xi)} +
 f(x;p(x)) \frac{\delta y(x)}{\delta p(\xi)}
=\,& \frac{\delta g(x;p(x))}{\delta p(\xi)} -
 \frac{\delta f(x;p(x))}{\delta p(\xi)} y(x)
\end{split}\end{equation}

or

\begin{equation}\begin{split}
\frac{\d}{\d x} \frac{\partial y(x)}{\partial p_i} +
 \frac{\partial f(x;P)}{\partial p_i} y(x) +
  f(x;P) \frac{\partial y(x)}{\partial p_i}
=\,& \frac{\partial g(x;P)}{\partial p_i} \text{ eqv.} \\
\frac{\d}{\d x} \frac{\partial y(x)}{\partial p_i} +
 f(x;P) \frac{\partial y(x)}{\partial p_i}
=\,& \frac{\partial g(x;P)}{\partial p_i} -
 \frac{\partial f(x;P)}{\partial p_i} y(x)
\end{split}\end{equation}

in order to compute $\frac{\delta y(x)}{\delta p(\xi)}$ or $\frac{\partial
y(x)}{\partial p_i}$.

\label{lastpage}
\end{document}

% $Id$

% $Log$
