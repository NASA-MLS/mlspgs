\documentclass[11pt]{article}
\usepackage{alltt}
\usepackage[fleqn]{amsmath}
\usepackage{floatflt}
\usepackage{graphicx}
\usepackage{longtable}

\textwidth 6.5in
\oddsidemargin -0.25in
%\evensidemargin -0.5in
\topmargin -0.5in
\textheight 9in

\newcommand{\docname}{wvs-096r1}
\newcommand{\docdate}{11 August 2010}

\ifx\pdfoutput\undefined
  \pdfoutput=0
  \usepackage[hypertex,plainpages,hyperindex=true]{hyperref}
  \hypersetup{%
    hypertexnames=false%
  }
  % Specify the driver for the color package
  \ExecuteOptions{dvips}
  %\ExecuteOptions{xdvi}
\else
  \ifnum\pdfoutput>0
    \usepackage[pdftex,plainpages,hyperindex=true,pdfpagelabels]{hyperref}
    \hypersetup{%
      hypertexnames=false,%
      colorlinks=true,%
      linktocpage=true,%
    }
    % Specify the driver for the color package
    \ExecuteOptions{pdftex}
  \else
    \usepackage[hypertex,plainpages,hyperindex=true]{hyperref}
    \hypersetup{%
      hypertexnames=false%
    }
    % Specify the driver for the color package
    \ExecuteOptions{dvips}
    %\ExecuteOptions{xdvi}
  \fi
\fi

\hyperbaseurl{}
\newcommand\hr[1]{\href{#1.dvi}{dvi}, \href{#1.pdf}{pdf}}
\newcommand\h[1]{#1 (\hr{#1})}

\begin{document}

%\tracingcommands=1
\newlength{\hW} % heading box width
\newlength{\pW} % page number field width
\settowidth{\hW}{\bf\docname}
\settowidth{\pW}{Page \pageref{lastpage}\ of \pageref{lastpage}}
\ifdim \pW > \hW \setlength{\hW}{\pW} \fi
\makeatletter
\def\@biblabel#1{#1.}
\newcommand{\ps@twolines}{%
  \renewcommand{\@oddhead}{%
    \docdate\hfill\parbox[t]{\hW}{{\hfill\bf\docname}\newline
                          Page \thepage\ of \pageref{lastpage}}}%
\renewcommand{\@evenhead}{}%
\renewcommand{\@oddfoot}{}%
\renewcommand{\@evenfoot}{}%
}%
\makeatother
\pagestyle{twolines}

\newcommand{\TS}{T_\text{scat}}
\newcommand{\TSs}[1]{T_{\text{scat}_{#1}}}
\newcommand{\DB}{\Delta B}
\newcommand{\MT}{\mathcal{T}}
\newcommand{\IC}[1]{\,\mathcal{I}_j\!\left(#1\right)} % Interpolation Function
\newcommand{\IF}[1]{\,\mathcal{A}_n\!\left(#1\right)} % Interpolation Function
\newcommand{\FA}[1]{\,\mathcal{F}_c\!\left(#1\right)} % Frequency averaging
\newcommand{\ER}[1]{Equation (\ref{#1})} % Equation reference

\vspace{-10pt}
\begin{tabbing}
\phantom{References: }\= \\
To: \>Bill, Nathaniel, Van\\
Subject: \>Combining LBL, PFA and $\TS$\\
From: \>Van Snyder\\
Reference: \>\h{wvs-026}, \h{wvs-027}, \h{wvs-095}
\end{tabbing}

\parindent 0pt \parskip 6pt
\vspace{-10pt}

\section{Combining LBL and PFA}

Recall from wvs-027 that we combine LBL and PFA calculations by

\begin{equation}\label{combined}
I_c \approx \sum_{n=1}^{n_c} \phi_{nc} \Delta \nu_{nc}
               \sum_{i=1}^{n_p} \IF{\DB_{ij} \MT^s_{ij}} \MT^w_{ic}
\end{equation}

where $\IF{\cdot}$ interpolates from the $\{\nu_j\}$ results for
frequencies at which the LBL calculation is carried out to the $\{\nu_n\}$
frequencies at which the channel response is tabulated, superscript $s$
refers to strong lines (processed by LBL calculations), and superscript
$w$ applies to weak lines or tails of distant strong lines (processed by
PFA calculations).

Since we can't always know the channel to which the result of a
single-frequency path integration contributes, the order of summation in
\ER{combined}\ is exchanged, with the result that the incremental
radiance $\DB_{ij} \MT_{ij}^s$ for LBL is frequency averaged at every
point on the path, giving a frequency-averaged incremental radiance

\begin{equation}\label{incrad}
\overline{I^s_{ic}} \approx
 \sum_{n=1}^{n_c} \phi_{nc} \Delta \nu_{nc} \IF{\DB_{ij} \MT^s_{ij}}
 =
 \Delta B_{ic} \FA{\MT^s_{ij}}\,,
\end{equation}

where$\FA{\cdot} = \sum_{n=1}^{n_c} \phi_{nc} \Delta \nu_{nc} \IF{\cdot}$
denotes frequency averaging. Replacing $\DB_{ij}$ in \ER{combined}\ by
$\DB_{ic}$ and then factoring it out of the frequency averaging is
reasonable because it varies slowly within one channel.

This is then combined with PFA $\MT^w_{ic}$ and integrated along the path
to get channel-averaged combined LBL and PFA radiance, giving a revision
of \ER{combined}\ of the form

\begin{equation}\label{three}
I_c \approx \sum_{i=1}^{n_p} \overline{I^s_{ic}} \MT^w_{ic}.
\end{equation}

\section{Adding scattering}

When scattering is taken into consideration, \ER{combined}\ becomes

\begin{equation}\label{scattering}
I_c \approx \sum_{n=1}^{n_c} \phi_{nc} \Delta \nu_{nc}
               \sum_{i=1}^{n_p}
                \IF{(\DB^g_{ij} + \DB^\sigma_{ij}) \MT^s_{ij}}
                \MT^w_{ic}\,,
\end{equation}

where superscript $g$ refers to gas-phase calculations and superscript
$\sigma$ ($s$ in wvs-095) refers to scattering calculations. For the
gas-phase part of the calculation,

\begin{equation}\label{DBg}
\DB_{ij}^g = \frac12\left(( 1 - \omega_{0_{i+1,j}} ) B_{i+1,j} -
                 ( 1 - \omega_{0_{i-1,j}} ) B_{i-1,j}\right),
                 i = 2 \dots n_p-1
\end{equation}

(ignoring special treatment at the ends of the path).  Since
$\omega_{0_{ij}}$ depends upon $\alpha_{\text{gas}_{ij}}$ it does not
necessarily vary slowly within one channel, and therefore $\DB_{ij}^g$
does not necessarily vary slowly within one channel.

Rewriting \ER{DBg}\ as

\begin{equation}
\DB^g_{ij} = \DB_{ij} - \frac12 \left(
 \omega_{0_{i+1,j}} B_{i+1,j} - \omega_{0_{i-1,j}} B_{i-1,j} \right)
\end{equation}

and substituting into \ER{incrad}\ we have

\begin{equation}\begin{split}\label{seven}
\overline{I^s_{ic}}
=\,& \DB_{ic} \FA{\MT^s_{ij}} - \\
   & \frac12 \sum_{n=1}^{n_c} \phi_{nc} \Delta \nu_{nc}
     \IF{\omega_{0_{i+1,j}} B_{i+1,j} \MT^s_{ij}} -
     \IF{\omega_{0_{i-1,j}} B_{i-1,j} \MT^s_{ij}} \\
=\,& \DB_{ic} \FA{\MT^s_{ij}} -
     \frac12 \left ( B_{i+1,c} \FA{\omega_{0_{i+1,j}} \MT^s_{ij}} -
                     B_{i-1,c} \FA{\omega_{0_{i-1,j}} \MT^s_{ij}} \right )
\end{split}\end{equation}

where again we justify factoring $B_{i\pm1,j}$ out of frequency averaging.

For the scattering part of the calculation, $\DB_{ij}$ is replaced by

\begin{equation}
\DB_{ij}^\sigma = \frac12\left( \omega_{0_{i+1,j}} \TSs{i+1,c} -
                 \omega_{0_{i-1,j}} \TSs{i-1,c}\right),
                 i = 2 \dots n_p-1
\end{equation}

(again ignoring special treatment at the ends of the path).  The
scattering part of \ER{scattering}\ becomes

\begin{equation}\begin{split}\label{nine}
I^\sigma_c \approx\,& \sum_{n=1}^{n_c} \phi_{nc} \Delta \nu_{nc}
               \sum_{i=1}^{n_p} \IF{\DB_{ij}^\sigma \MT^s_{ij}} \MT^w_{ic}
               \\
=\,&
 \sum_{n=1}^{n_c} \phi_{nc} \Delta \nu_{nc}
  \sum_{i=1}^{n_p} \IF{\frac12\left( \omega_{0_{i+1,j}} \TSs{i+1,c} -
   \omega_{0_{i-1,j}} \TSs{i-1,c}\right) \MT^s_{ij}} \MT^w_{ic} \\
=\,&
 \frac12 \sum_{n=1}^{n_c} \phi_{nc} \Delta \nu_{nc}
  \sum_{i=1}^{n_p} \IF{\omega_{0_{i+1,j}} \TSs{i+1,c} \MT^s_{ij}} -
   \IF{\omega_{0_{i-1,j}} \TSs{i-1,c} \MT^s_{ij}} \MT^w_{ic} \\
=\,&
 \frac12 \sum_{i=1}^{n_p} \left(
  \TSs{i+1,c} \FA{\omega_{0_{i+1,j}} \MT^s_{ij}} -
  \TSs{i-1,c} \FA{\omega_{0_{i-1,j}} \MT^s_{ij}}
  \right)  \MT^w_{ic}
\end{split}\end{equation}

(again ignoring special treatment at the ends of the path).  That is, the
single-frequency quantities $\omega_{0_{i\pm1,j}} \MT^s_{ij}$ must be
frequency averaged at every point along the path before being combined
with the channel quantities $\TSs{i\pm1,c}$ and $\MT^w_{ic}$.

Substituting \ER{seven}\ and \ER{nine}\ into \ER{three}, we have

\begin{equation}\begin{split}\label{final}
\overline{I^s_{ic}}
=\,&
 \DB_{ic} \FA{\MT^s_{ij}} - \\
&
     \frac12 \left( B_{i+1,c} \FA{\omega_{0_{i+1,j}} \MT^s_{ij}} -
                    B_{i-1,c} \FA{\omega_{0_{i-1,j}} \MT^s_{ij}} \right)
                    + \\
&
     \frac12 \left( \TSs{i+1,c} \FA{\omega_{0_{i+1,j}} \MT^s_{ij}} -
                    \TSs{i-1,c} \FA{\omega_{0_{i-1,j}} \MT^s_{ij}} \right)
                    \\
=\,&
 \DB_{ic} \FA{\MT^s_{ij}} + \\
&
     \frac12 \left( \left( \TSs{i+1,c} - B_{i+1,c} \right)
                          \FA{\omega_{0_{i+1,j}} \MT^s_{ij}} -
                    \left( \TSs{i-1,c} - B_{i-1,c} \right)
                          \FA{\omega_{0_{i-1,j}} \MT^s_{ij}}
             \right)
\end{split}\end{equation}

(again ignoring special treatment at the ends of the path).

Recall the definitions

\begin{equation}\begin{split}
\MT_{ij} =\,& \exp \left( - \sum_{l=2}^i ( \alpha^s_{lj} + \alpha^w_{lj} )
 \Delta s_l \right) \text{, and} \\
\MT^s_{ij} =\,& \exp \left( - \sum_{l=2}^i \alpha^s_{lj} \Delta s_l \right)
\text{ and }
\MT^w_{ij} = \exp \left( - \sum_{l=2}^i \alpha^w_{lj} \Delta s_l \right)
\text{ from which}\\
\MT_{ij} =\,& \MT^s_{ij} \MT^w_{ij}\,.
\end{split}\end{equation}

\section{Problem with $\omega_0$}

In \ER{final}\ we need\footnote{We use $\beta_{c\_s_i}$ and
$\beta_{c\_e_i}$ instead of $\beta_{c\_s_{ij}}$ and $\beta_{c\_e_{ij}}$
because they vary slowly with frequency; we simply interpolate in the
table for them that is closest in frequency to $\nu_j$.}

\begin{equation}\label{twelve}
\omega_{0_{i\pm1,j}} \MT_{ij} =
 \frac{\beta_{c\_s_{i\pm1}}}
     {\alpha^s_{i\pm1,j} + \alpha^w_{i\pm1,c} + \beta_{c\_e_{i\pm1}}}
 \exp \left( - \sum_{l=2}^i ( \alpha^s_{lj} + \alpha^w_{lc} )
  \Delta s_l \right)\,.
\end{equation}

This quantity cannot be factored as functions only of $\alpha^s_{lj}$ and
of $\alpha^w_{lc}$, and therefore we cannot separately frequency average
a quantity that is a function only of $\alpha^s_{lj}$, giving a channel
quantity that we can then multiply by a function only of $\alpha^w_{lc}$.

If $\alpha^w_{ic}$ is sufficiently smaller than $\alpha^s_{ij} +
\beta_{c\_e_i}$ that we can neglect $\alpha^w_{ic}$ in the definition of
$\omega_{0_{ij}}$ we can use \ER{final}.  Otherwise,

\begin{equation}
\alpha^s_{ic} \approx
 \frac{\ln \left( \FA{\MT^s_{ij}} \right) -
       \ln \left( \FA{\MT^s_{i-1,j}} \right)}{\Delta s_i}
\end{equation}

might be used in place of $\alpha^s_{ij}$ in the definition of
$\omega_{0_{ij}}$, giving instead an approximation to $\omega_{0_{ic}}$:

\begin{equation}\label{omega_c}
\omega_{0_{ic}} \approx
 \frac{\beta_{c\_s_i}}
      {\frac{\ln \left( \FA{\MT^s_{ij}} \right) -
             \ln \left( \FA{\MT^s_{i-1,j}} \right)}{\Delta s_i} +
       \alpha^w_{ic} + \beta_{c\_e_i}} =
 \frac{\beta_{c\_s_i} \Delta s_i}
      {\ln \left( \FA{\MT^s_{ij}} \right) -
       \ln \left( \FA{\MT^s_{i-1,j}} \right) +
       \Delta s_i ( \alpha^w_{ic} + \beta_{c\_e_i} ) }.
\end{equation}

Since $\omega_{0_{ic}}$ is a channel quantity, $\FA{\omega_{0_{i\pm1,j}} \MT^s_{ij}}$ becomes $\omega_{0_{i\pm1,c}}
\FA{\MT^s_{ij}}$, resulting in the need to frequency average only
$\MT^s_{ij}$ at every point on the path,\footnote{The forward model
currently averages $\DB_{ij} \MT^s_{ij}$ to combine LBL and PFA; it could
average $\MT^s_{ij}$ and combine that with $\DB_{ic}$ during the PFA step
because $\DB_{ij}$ changes very slowly with frequency.  That is,
\ER{incrad} would become $\overline{I^s_{ic}} \approx \FA{\MT^s_{ic}}$,
and \ER{three} would become $\sum_{i=1}^{n_p} \DB_{ic} \FA{\MT^s_{ic}}
\MT^w_{ic}$.} and \ER{final}\ becomes simpler:

\begin{equation}
\overline{I^s_{ic}}
=
\FA{\MT^s_{ij}} \left[ \DB_{ic} + \Delta W_{ic} \right]
\text{ where }
W_{ic} = (\TSs{ic} - B_{ic}) \omega_{0_{ic}}
\end{equation}

and $\Delta W_{ic} = \frac12 (W_{i+1,c} - W_{i-1,c})$ (again ignoring
special treatment at the ends of the path).

The derivatives of $\omega_{0_{i\pm1,c}}$ become more complicated because
they depend upon $\FA{\MT^s_{ij}}$ and $\Delta s_i$:

\begin{equation}\label{d_omega}
\frac{\partial \omega_{0_{ic}}}{\partial x}
\approx
\frac{\omega_{0_{ic}}}{\beta_{c\_s_i}} \left[
 \frac{\alpha^s_{ic}}{\Delta s_i}
  \frac{\partial \Delta s_i}{\partial x} +
 \frac{\partial \beta_{c\_s_i}}{\partial x} -
\omega_{0_{ic}}
  \left( \frac1{\Delta s_i} \Delta
         \frac{\partial \ln \left( \FA{\MT^s_{ij} }\right)}{\partial x} +
         \frac{\partial \alpha^w_{ic}}{\partial x} +
         \frac{\partial \beta_{c\_e_i}}{\partial x}
  \right)
 \right]
\end{equation}

where $\Delta\frac{\partial \ln \left( \FA{\MT^s_{ij} }\right)}{\partial
x} = \frac{\partial \ln \left( \FA{\MT^s_{ij} }\right)}{\partial
x} - \frac{\partial \ln \left( \FA{\MT^s_{i-1,j} }\right)}{\partial
x}$.  (The derivative $\frac{\partial \Delta s_i}{\partial x}$ is zero if
$x \neq T$).

Since $\MT^s_{ij} = \exp(-\delta^s_{ij})$, $\frac{\partial
\MT^s_{ij}}{\partial x} = -\MT^s_{ij} \frac{\partial
\delta^s_{ij}}{\partial x}$.  Therefore

\begin{equation}
\frac{\partial \ln \left( \FA{\MT^s_{ij} }\right)}{\partial x} =
\frac1{\FA{\MT^s_{ij}}}
 \frac{\partial \FA{\MT^s_{ij}}}{\partial x} =
\frac1{\FA{\MT^s_{ij}}}
 \FA{\frac{\partial \MT^s_{ij}}{\partial x}} =
-\frac1{\FA{\MT^s_{ij}}}
 \FA{\MT^s_{ij} \frac{\partial \delta^s_{ij}}{\partial x}}.
\end{equation}

If the approximation
$\FA{\MT^s_{ij} \frac{\partial \delta^s_{ij}}{\partial x}} \approx
 \FA{\MT^s_{ij}} \FA{\frac{\partial \delta^s_{ij}}{\partial x}}$
is justified then
$\frac{\partial \ln \left( \FA{\MT^s_{ij} }\right)}{\partial x} \approx
 -\FA{\frac{\partial \delta^s_{ij}}{\partial x}}$ and

\begin{equation}
\Delta \frac{\partial \ln \left( \FA{\MT^s_{ij} }\right)}{\partial x}
\approx
-\Delta \frac{\partial \FA{\delta^s_{ij}}}{\partial x} =
-\frac{\partial \FA{ \Delta \delta^s_{ij}}}{\partial x} =
-\frac{\partial \FA{ \alpha^s_{ij}}}{\partial x} =
-\FA{\frac{\partial \alpha^s_{ij}}{\partial x}},
\end{equation}

and \ER{d_omega} becomes

\begin{equation}
\frac{\partial \omega_{0_{ic}}}{\partial x}
\approx
\frac{\omega_{0_{ic}}}{\beta_{c\_s_i}} \left[
 \frac{\alpha^s_{ic}}{\Delta s_i}
  \frac{\partial \Delta s_i}{\partial x} +
 \frac{\partial \beta_{c\_s_i}}{\partial x} +
\omega_{0_{ic}}
  \left( \frac1{\Delta s_i}
         \FA{\frac{\partial \alpha^s_{ij}}{\partial x}} -
         \frac{\partial \alpha^w_{ic}}{\partial x} -
         \frac{\partial \beta_{c\_e_i}}{\partial x}
  \right)
 \right]
\end{equation}

where $\frac{\partial \alpha^s_{ij}}{\partial x}$ is given in wvs-095.
The derivative $\frac{\partial I_c}{\partial x}$ where $I_c$ is given by
\ER{three} is calculated as explained as in wvs-027.

In wvs-095, the quantity

\begin{equation}
I^\sigma_c = \sum_{n=1}^{n_c} \phi_{nc} \Delta \nu_{nc}
  \IF{\sum_{i=1}^{n_p} \DB_{ij}^\sigma \MT_{ij}^s}
\end{equation}

was rearranged using partial summation to give

\begin{equation}
I^\sigma_c = \sum_{i=1}^{n_p} \TSs{ic} \sum_{n=1}^{n_c} \phi_{nc} \nu_{nc}
  \IF{\omega_{0_{ij}} \Delta \MT_{ij}^s} \text{ where }
  \Delta \MT_{ij}^s = \frac12 \left( \MT_{i-1,j} - \MT_{i+1,j} \right)
\end{equation}

(again ignoring special treatment at the ends of the path).  From
\ER{final} it is clear that the
formulation based upon $\Delta \MT_{ij}$ is no longer useful.

If the derivatives of $\omega_{0_{ij}}$ can be calculated sufficiently
accurately by ignoring $\alpha^w_{ic}$, the derivative expressions in
Equations (12) and (13) in wvs-095 remain useful in the context of
\ER{final}.  Otherwise, those derivative expressions will need to be
formulated in terms of \ER{omega_c}\ above, the strong and weak parts
separated, and the strong parts separately frequency averaged.

\section{Alternative approach}

The filter functions are not perfect square waves, and therefore the
``tail'' of one channel might overlap with the ``tail'' of an adjacent
one, causing some frequencies to appear in two channels.  Since we do not
always know the channel to which a LBL quantity pertains, above, we
frequency average LBL quantities to combine them with channel quantities.

Instead of frequency averaging LBL quantities to combine them with
channel quantities, we can instead interpolate channel quantities to the
frequencies at which LBL quantities are evaluated, combine them with LBL
quantities, and frequency average the result.

This strategy could be applied to the combination of LBL and PFA even
without consideration of $\TS$, but reorganization of the treatment of PFA
in the full forward model would be required.

Let $\IC{\cdot}$ denote frequency interpolation from a set of channel
quantities to frequency $\nu_j$.  Then Equation (2) in wvs-027 becomes

\begin{equation}
I_c = \FA{\sum_{i=1}^{n_p}\DB_{ij} \MT^s_{ij} \IC{\{\MT^w_{ic}\}}}
\end{equation}

where by $\{\MT^w_{ic}\}$ we mean a set of $\MT^w_i$ from several
channels.  Rather than keeping $\left\{\FA{\MT^s_{ij}}\right\}$ the full
forward model would need to keep $\{\MT^w_{ic}\}$.  This is the same
amount of space, that is, an array with extent path length $\times$
number of channels, but a different organization.  The order of computing
LBL and PFA would also need to be exchanged.  The results would almost
certainly be different, but hopefully not by much.

If the result of interpolating a product is well approximated by the
product of separately interpolated factors,

\begin{equation}
\frac{\partial I_c}{\partial x} \approx
\FA{\sum_{i=1}^{n_p} \MT^s_{ij} \IC{\{\MT^w_{ic}\}} \left[
 \frac{\partial \DB_{ij}}{\partial x} -
  \DB_{ij} \left( \frac{\partial \delta^s_{ij}}{\partial x} +
                  \IC{ \frac{\partial \delta^w_{ic}}{\partial x}}
 \right) \right]}.
\end{equation}

Otherwise

\begin{equation}
\frac{\partial I_c}{\partial x} \approx
\FA{\sum_{i=1}^{n_p} \MT^s_{ij} \left\{
 \IC{\{\MT^w_{ic}\}} \left[
 \frac{\partial \DB_{ij}}{\partial x} -
  \DB_{ij} \frac{\partial \delta^s_{ij}}{\partial x} \right] -
  \DB_{ij} \IC{\left\{\MT^w_{ic} \frac{\partial \delta^w_{ic}}{\partial x}
               \right\}}
 \right\}}.
\end{equation}

LBL and $\TS$ can be combined by a similar strategy.  \ER{final} becomes


\begin{equation}\begin{split}\label{LBL_Tscat}
\overline{I^s_{ic}}
=\,&
 \FA{\DB_{ij} \MT^s_{ij}} - \\
&
     \frac12 \left( \FA{B_{i+1,j} \omega_{0_{i+1,j}} \MT^s_{ij}} -
                    \FA{B_{i-1,j} \omega_{0_{i-1,j}} \MT^s_{ij}} \right)
                    + \\
&
     \frac12 \left( \FA{\IC{\{\TSs{i+1,c}\}} \omega_{0_{i+1,j}} \MT^s_{ij}} -
                    \FA{\IC{\{\TSs{i-1,c}\}} \omega_{0_{i-1,j}} \MT^s_{ij}} \right)
                    \\
=\,&
 \FA{\DB_{ij} \MT^s_{ij} + \DB^\sigma_{ij} \MT^s_{ij}} \text{ where} \\
\DB^\sigma_{ij}
=\,&
    \frac12 \left(
     \left( \IC{\{\TSs{i+1,c}\}} - B_{i+1,j} \right)
        \omega_{0_{i+1,j}} -
     \left( \IC{\{\TSs{i-1,c}\}} - B_{i-1,j} \right)
        \omega_{0_{i-1,j}}
     \right).
\end{split}\end{equation}

This value of $\overline{I^s_{ic}}$ can be used in \ER{three} to combine
LBL and $\TS$ with PFA.  This would not require a restructuring of the
PFA calculation.

Alternatively, LBL, $\TS$ and PFA could be combined using

\begin{equation}
I_c = \FA{\sum_{i=1}^{n_p}
 ( \DB_{ij} + \DB^\sigma_{ij} ) \MT^s_{ij} \IC{\{\MT^w_{ic}\}}}.
\end{equation}

This removes the requirement to retain a frequency-averaged quantity along
the path, but would require restructuring of the PFA calculation.

Using $I^s_{ij} = \DB_{ij} \MT^s_{ij} + \DB^\sigma_{ij} \MT^s_{ij}$ to
denote the final argument of $\FA{\cdot}$ in \ER{LBL_Tscat} we can write
Equation (6) from wvs-026 as

\begin{equation}
\frac{\partial I^c}{\partial x} =
 \FA{ \sum_{i=1}^{n_p}
  \frac{\partial I^s_{ij}}{\partial x}} -
 \sum_{i=1}^{n_p} \overline{I^s_{ic}} \MT^w_{ic}
  \sum_{l=1}^i \frac{\partial \delta^w_{jc}}{\partial x}
\end{equation}

where, like wvs-026, the first term is the LBL+$\TS$ derivative and the
second is a PFA correction.  The first term in $\frac{\partial
I^s_{ij}}{\partial x}$, \emph{viz.}, $\frac{\partial (\DB_{ij}
\MT^s_{ij})}{\partial x}$, is just the LBL derivative; the second is a
$\TS$ correction.  $\frac{\partial \omega_{0_{ij}}}{\partial x}$, which
is necessary to compute $\frac{\partial \DB^\sigma_{ij}}{\partial x}$, is
described by Equations (11-13) in wvs-095.

\label{lastpage}
\end{document}

% $Id$

% $Log$
% Revision 1.1  2010/08/11 03:18:32  vsnyder
% Initial commit
%
