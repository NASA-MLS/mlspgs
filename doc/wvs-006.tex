\documentclass[11pt]{article}
\textwidth 6.25in
\oddsidemargin -0.25in
%\evensidemargin -0.5in
\topmargin -0.5in
\textheight 9.00in

\begin{document}

%\tracingcommands=1
\newlength{\hW} % heading box width
%\settowidth{\hW}{\bf wvs-006}
\settowidth{\hW}{Page \pageref{lastpage}\ of \pageref{lastpage}}
\makeatletter
\def\@biblabel#1{#1.}
\newcommand{\ps@twolines}{%
  \renewcommand{\@oddhead}{%
    4 May 2000\hfill\parbox[t]{\hW}{{\bf wvs-006}\newline
                          Page \thepage\ of \pageref{lastpage}}}%
\renewcommand{\@evenhead}{}%
\renewcommand{\@oddfoot}{}%
\renewcommand{\@evenfoot}{}%
}%
\makeatother
\pagestyle{twolines}
%\thispagestyle{empty}
%{\hfill\bf J3/00-103}
%\pagenumbering{arabic}

\vspace{-10pt}
\begin{tabbing}
\phantom{References: }\= \\
To: \>Nathaniel\\
Subject: \>Changes I've made or suggest for the L2CF format, and questions
 about it\\
From: \>Van Snyder\\
\end{tabbing}

\parindent 0pt \parskip 5pt
\vspace{-20pt}

I've changed the syntax of the ``range'' operator from ``..'' to ``:''.
This simplified the lexical analysis of the input substantially.

I've not implemented the possibility to specify a semi-infinite range by
``a:'' or ``:b'' because it is ambiguous:  We allow ``name = expr expr
....'' If the first ``expr'' is ``a:'' and the second is ``b'', this
could equally well be interpreted as just one expr, ``a : b''.

The interpretation of a specification is not consistent.  In some
sections, e.g. {\tt Construct}, a specification is introduced by an
initial name known to the processor e.g. vGrid,, followed by a comma, and
the identifier of the specification is provided by a phrase of the form
{\tt name =} \emph{identifier}.  In other sections, e.g. {\tt
ReadApriori} the initial name (preceeding the comma) is an identifier for
the specification, chosen by the user, and the name of the specification
is implied by the section name.

I would like to make the syntax consistent.  This would ease parsing and
interpretation.  I propose that in a specification, a name followed by a
comma should always introduces the specification, and the name of the
specification should be known to the processor, e.g. ``quantity''.  In
order to give an identifier of the user's choosing to the specification,
I suggest beginning the specification with a name followed by a colon. 
Thus the {\tt ReadApriori} and {\tt MergeApriori} sections would have
specifications of the form

{\tt aprioriGPH: apriori, source=GPH, length=200km, ...} and\\
{\tt merge, apriori=aprioriGPH, source=NCEP, species=temperature, ...}

while the {\tt Construct} section would have specifications of the form

{\tt VGridHiRes: VGrid, coordinate=pressure, values=log 1000 mb 60 12\\
temp: quantity, VGrid=VGridHiRes, ...}

Is the set of ``$<$name$>$='' notations for a given specification known
and fixed, or chosen by the user?  For example, in the {\tt
ProfileLayout} specification, I see {\tt type=fractional, fraction=0.5}. 
Are the names {\tt type} and {\tt fractional} known to the program?  If
so, is there an exhaustive list for each specification?

Are the identifiers that appear after the ``='' in ``$<$name$>$=''
notations fixed, or chosen by the user?  For example, in the previous
paragraph, is the name ``{\tt fractional}'' known to the processor?  If
so, is there an exhaustive list for each notation in each specification?

\label{lastpage}
\end{document}
% $Id$
