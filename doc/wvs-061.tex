\documentclass[11pt]{article}
\usepackage[fleqn]{amsmath}\textwidth 6.5in
\oddsidemargin -0.25in
%\evensidemargin -0.5in
\topmargin -0.5in
\textheight 9.4in

\newcommand{\docname}{\bf wvs-061}
\newcommand{\docdate}{7 November 2007}

\begin{document}

%\tracingcommands=1
\newlength{\hW} % heading box width
\newlength{\pW} % page number field width
\settowidth{\hW}{\docname}
\settowidth{\pW}{Page \pageref{lastpage}\ of \pageref{lastpage}}
\ifdim \pW > \hW \setlength{\hW}{\pW} \fi
\makeatletter
\def\@biblabel#1{#1.}
\newcommand{\ps@twolines}{%
  \renewcommand{\@oddhead}{%
    \docdate\hfill\parbox[t]{\hW}{{\hfill\docname}\newline
                          Page \thepage\ of \pageref{lastpage}}}%
\renewcommand{\@evenhead}{}%
\renewcommand{\@oddfoot}{}%
\renewcommand{\@evenfoot}{}%
}%
\makeatother
\pagestyle{twolines}

\vspace{-10pt}
\begin{tabbing}
\phantom{References: }\= \\
To: \>Bill, Nathaniel, Alyn, Dong\\
Subject: \>Trouble with path length in the forward model\\
From: \>Van Snyder\\
Reference: \>wvs-060
\end{tabbing}

\parindent 0pt \parskip 6pt
\vspace{-10pt}

As noted in wvs-060, the expressions
%
\begin{equation}
\frac{\int \text{d}s}
     {\int \frac{\text{d}s}{\text{d}h} \frac{\text{d}h}{\text{d}\zeta}
      \,\text{d} \zeta}
\text{ and }
\frac{\int \frac{\text{d}s}{\text{d}h} \,\text{d}h}
     {\int \frac{\text{d}s}{\text{d}h} \frac{\text{d}h}{\text{d}\zeta}
      \,\text{d} \zeta}
\end{equation}
%
integrated along each segment of the line of sight, are exactly 1, but the
approximation
%
\begin{equation}\label{two}
\frac{\delta s}{\delta \zeta \sum_{i=1}^{n_{\text{GL}}} \frac{\text{d}s}{\text{d}h}
\frac{\text{d}h}{\text{d}\zeta}\, \omega_i}\,,
\end{equation}
%
which ought to
be 1, is computationally usually not.  In a few short tests I ran, this ratio
varies between 0.7 and 1.9.

The rectangular and trapezoidal approximations to the line-of-sight integral
$\int \alpha(s)\, \text{d}s$ are carried out in $s$ coordinates, while the GL
corrections are carried out in $\zeta$ coordinates.  The GL corrections include
a scheme to reduce the strength of a singularity at the tangent point that
involves subtracting the rectangular estimate.  Since the ratio in Equation
(\ref{two}) is not a good approximation to 1.0, the rectangular and GL
estimates are not consistent.

I've added a switch {\tt /pathNorm} to the {\tt forwardModel} specification,
which normalizes path lengths in the forward model so that when computing
%
\begin{equation}
\int \alpha(\zeta) \frac{\text{d}s}{\text{d}h} \frac{\text{d}h}{\text{d}\zeta}
\,\text{d} \zeta
\text{ and }
\int \alpha(s)\, \text{d}s
\end{equation}
%
the path length is the same.  I.e., I have introduced the compensation
%
\begin{equation}\begin{split}
\int \frac{\text{d}s}{\text{d}h}
\frac{\text{d}h}{\text{d}\zeta}\,( \alpha(\zeta) - \alpha(s_j) )\,
\text{d}\zeta +\,&
\int \alpha(s_j)\, \text{d}s =
\left[
\frac{\int \text{d}s}
     {\int \frac{\text{d}s}{\text{d}h}\frac{\text{d}h}{\text{d}\zeta}
     \,\text{d} \zeta}
\right]
\int \frac{\text{d}s}{\text{d}h}
\frac{\text{d}h}{\text{d}\zeta}\,( \alpha(\zeta) - \alpha(s_j) )\,
\text{d}\zeta +
\int \alpha(s_j)\, \text{d}s\\
\approx\,&
\left[
\frac{\delta s}{\delta \zeta \sum_{i=1}^{n_{\text{GL}}} \frac{\text{d}s}{\text{d}h}
\frac{\text{d}h}{\text{d}\zeta}\, \omega_i}
\right]\,
\delta \zeta \sum_{i=1}^{n_{\text{GL}}} \frac{\text{d}s}{\text{d}h}
\frac{\text{d}h}{\text{d}\zeta}\, \omega_i ( \alpha(\zeta_i) - \alpha(s_j) ) +
\alpha(s_j) \delta s\,.
\end{split}\end{equation}
%
described in wvs-060 when this switch is set. Setting this switch makes
profound changes in L2PC and SIDS results.  I haven't run any retrievals with
it set.  I leave it to you to experiment with it, and decide whether to use it.

\label{lastpage}
\end{document}
% $Id$
