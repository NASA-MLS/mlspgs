\documentclass[11pt]{article}
\usepackage[fleqn]{amsmath}\textwidth 6.25in
\usepackage{longtable}
\oddsidemargin -0.25in
%\evensidemargin -0.5in
\topmargin -0.5in
\textheight 9.00in

\begin{document}

%\tracingcommands=1
\newlength{\hW} % heading box width
%\settowidth{\hW}{\bf wvs-021}
\settowidth{\hW}{Page \pageref{lastpage}\ of \pageref{lastpage}}
\makeatletter
\def\@biblabel#1{#1.}
\newcommand{\ps@twolines}{%
  \renewcommand{\@oddhead}{%
    10 March 2005\hfill\parbox[t]{\hW}{{\bf wvs-021}\newline
                          Page \thepage\ of \pageref{lastpage}}}%
\renewcommand{\@evenhead}{}%
\renewcommand{\@oddfoot}{}%
\renewcommand{\@evenfoot}{}%
}%
\makeatother
\pagestyle{twolines}

\vspace{-10pt}
\begin{tabbing}
\phantom{References: }\= \\
To: \>Bill, Dong, Michael, Nathaniel\\
Subject: \>Pre-frequency-averaged radiance calculation\\
From: \>Van Snyder\\
\end{tabbing}

\parindent 0pt \parskip 3pt
\vspace{-20pt}

\section{Introduction}

I've implemented Bill's ideas for pre-frequency-averaged (PFA) radiance
calculation in MLSL2.  This requires to calculate tables of frequency-averaged
betas for each channel and species, which tables can then be used to calculate
betas, and thence radiances, in the forward model.  For species that have many
weak lines or tails within a band, this can result in significant cost
savings.  For example, for HNO$_3$ in band 6, the run time is reduced by a
factor of eight.

\section{Calculating PFA tables}

PFA tables can be calculated by MLSL2.  The advantage of this approach is that
we use the same software as is used to calculate betas in the line-by-line
(LBL) forward model.

A very simple L2CF is all that is necessary to calculate PFA betas.  The only
sections required are Signals, Spectroscopy and GlobalSettings.  The Signals
and Spectroscopy sections are the same as for any other L2 run.  In the
GlobalSettings section, one needs to read the filter shapes, define pressure
and temperature grids, and request to calculate the PFA tables.  The filter
shapes are read by a {\tt ForwardModelGlobal} command.  In an usual L2 run,
this has {\tt antennaPatterns, filterShapes} and {\tt pointingGrids} fields.
All that is necessary for PFA calculation is the {\tt filterShapes} field. 
Here's one I used:
{\tt\begin{verbatim}
  ForwardModelGlobal, filterShapes= $
    "/testing/emls/l2cal/MLS-Aura_L2Cal-Filters_v1-5-0_0000d000.txt"
\end{verbatim}}

The pressure grid is defined by a {\tt vGrid} command, specifying logarithmic
type and a {\tt zeta} coordinate.  Here's one I used:
{\tt\begin{verbatim}
pfaVgrid: vGrid, type=Logarithmic, coordinate=Zeta, $
    start=(10 ^ (3+1/6))mb, formula=[87:12]
\end{verbatim}}

The temperature grid is defined by a {\tt tGrid} command.  This generates a
logarithmic grid with a {\tt theta} coordinate.  You can't specify that ---
it's built in.  Here's one I used:
{\tt\begin{verbatim}
pfaTgrid: tGrid, number = 7, start = 180 K, step = log(300/180)/6
\end{verbatim}}
Notice that the step is in log temperature!

Once the filter shapes and grids are defined, the PFA tables can be calculated
by using a {\tt makePFA} command.  It has five fields, all required:

\begin{longtable}{llll}
Field              & Data type            & Shape  & Meaning \\
\hline
{\tt losvel}       & Numeric              & Scalar & Line-of-sight velocity \\
{\tt molecules}    & {\tt molecule}       & Array  & List of molecules \\
{\tt signals}      & Signal string        & Array  & List of signals \\
{\tt temperatures} & Label of {\tt tGrid} & Scalar & Temperature grid \\
{\tt vGrid}        & Label of {\tt vGrid} & Scalar & Pressure grid \\
\end{longtable}

The {\tt makePFA} command makes PFA tables for the Cartesian product of the
specified signals and molecules.  The signal nomenclature strings are expanded,
so it isn't necessary to specify every channel separately.  If there are
several {\tt makePFA} commands, their results are cumulative.

Here's one I used:
{\tt\begin{verbatim}
pfagenAll: makePFA, $
 signals = [ $
  "B27LM", "B27UM", "B2LF.S0", "B2UF.S0", "B3LF.S2", "B3UF.S2", $
  "B4LF.S0", "B4UF.S0", "B5LF.S0", "B5UF.S0", "B6LF.S0", "B6UF.S0" ], $
 molecules = [ $
  CH3CN, CL_35_O, CL_37_O, CL_37_O_V1, H2O, H2O_17, H2O_17_V2, $
  H2O_18, H2O_18_V2, H2O_R2, H2O_V2, H2O2, HC_13_N, HCN_15, HCN_V2, HNO3, $
  HNO3_V5, HNO3_V6, HNO3_V7, HNO3_V8, HNO3_V9, HO2, N2, N_15_NO, N2O, $
  N2O_17, N2O_18, N2O_2V2, N2O_V1, N2O_V2, NN_15_O, O_17_O, O_18_H, $
  O_18_O, O_18_O_V1, O2, O2_V1, O3, O3_2V2, O3_ASYM_O_17, O3_ASYM_O_18, $
  O3_ASYM_O_18_V2, O3_R2, O3_SYM_O_17, O3_SYM_O_18, O3_SYM_O_18_V2, $
  O3_V1_3, O3_V1_3_V2, O3_V2, OC_13_S_32, OCS_32, OCS_34, S_32_O2, $
  S_32_O2_V2, S_33_O2, S_34_O2 ], $
 temperatures=pfaTgrid, vGrid=pfaVgrid, losVel= -6.80kms
\end{verbatim}}

If the {\tt makePFA} command has a label, it refers to the last table
generated.  This might be useful if you want to dump a table, using
the {\tt dump} command, \emph{viz.}

\hspace*{0.25in}{\tt dump, details=2, pfaData=pfagenAll}

This would dump the PFA table for channel 25 of B6UF and S\_34\_O2.
The {\tt dump} command can also dump all of the generated PFA data, \emph{viz.}

\hspace*{0.25in}{\tt dump, /allPFA}

This only dumps the molecule names, grid names, and signals.  If you want to
see the tables, increase the details level.

\section{Writing PFA tables to a file}

One can either calculate PFA tables and use them in the same MLSL2 run, or
calculate them and write them to an HDF5 file for use in future MLSL2 runs (or
both).  To write PFA tables, use the {\tt writePFA} command in the {\tt
globalSettings} section.  It has three fields:

\begin{longtable}{lllll}
Field & Data type & Shape & Optional? & Meaning \\
\hline
{\tt file}    & String                 & Scalar & No  & The name of the HDF5 file \\
{\tt pfaData} & Label of {\tt makePFA} & Array  & Yes & List of PFA data to write \\
{\tt allPFA}  & Boolean                & Scalar & Yes & If set, means ``Write all PFA data''\\
\end{longtable}

Either {\tt pfaData} shall be specified, or {\tt allPFA} shall be specified
with the value {\tt true} (which can alternatively be specified as {\tt
/allPFA}), but not both.  Here's one I used:
{\tt\begin{verbatim}
  writePFA, file="/user5/vsnyder/mlspgs/l2/runs/PFAData.h5", /allpfa
\end{verbatim}}

Each time the {\tt writePFA} command is executed, it creates a new file; it is
not possible to write several sets of data to the same file by using several
{\tt writePFA} commands that refer to the same file.  (Actually, I haven't
tried that; maybe it works!  But I didn't explicitly plan for it to work that
way.)

If all you're doing in a MLSL2 run is computing and writing PFA tables, you
might want to add {\tt dump, /stop} at the end of globalSettings, so MLSL2
doesn't snivel about not finding L1BOA data --- which isn't needed for PFA table
calculations.

\section{Reading PFA data from a file}

PFA data written by a {\tt writePFA} command can be read by a {\tt readPFA}
command.  The {\tt readPFA} command reads PFA tables for the Cartesian product
of the specified signals and molecules, from the specified file.  The signal
nomenclature strings are expanded, so it isn't necessary to specify every
channel separately.  If there are several {\tt readPFA} commands, their results
are cumulative.  A {\tt readPFA} command has three fields:

\begin{longtable}{lllll}
Field & Data type & Shape & Optional? & Meaning \\
\hline
{\tt file}      & String   & Scalar & No  & The name of the HDF5 file \\
{\tt molecules} & molecule & Array  & Yes & List of molecules \\
{\tt signals}   & String   & Array  & Yes & list of signals \\
\end{longtable}

If the {\tt molecules} field is omitted, it means ``read PFA data for all
molecules;''  If the {\tt signals} field is omitted, it means ``read PFA data
for all signals.''  The molecules are \emph{not} automatically grouped; if PFA
data are needed for various isotopes or rotational or vibrational states, it is
necessary either to read data for all molecules, or specify every desired
molecule explicitly.

Here's one I used:
{\tt\begin{verbatim}
 readPFA, file="/user5/vsnyder/mlspgs/l2/runs/PFAData.h5", $
    signals="b6f", $
    molecules=[hno3,n2,o2,o3]
\end{verbatim}}

This reads PFA data for all channels in both sidebands of band 6, for HNO$_3$,
N$_2$, O$_2$ and O$_3$.

\section{Using the PFA method in the full forward model}

The PFA method can be used to reduce computation time in the full forward
model.  It is not used in other models.  This is done by specifying for which
molecules radiances are to be calculated using the PFA method instead of the
LBL method.  Two fields are added to the {\tt forwardModel} command to specify
PFA calculations:

\begin{longtable}{llllp{2.5in}}
Field                 & Data type & Shape  & Optional? & Meaning \\
\hline
{\tt lsbPFAmolecules} & molecule  & Array  & Yes       & List of molecules for
                                                         which PFA is to be used
                                                         in the lower sideband\\
{\tt usbPFAmolecules} & molecule  & Array  & Yes       & List of molecules for
                                                         which PFA is to be used
                                                         in the upper sideband \\
\end{longtable}

The {\tt molecules} field retains the same syntax and type requirements as when
no PFA calculations are requested, but the meaning is ``list of all molecules,
both PFA and LBL.''  One then specifies the molecules for which PFA
calculations are to be performed, and in which sideband of the signals
specified by the {\tt signals} field.  If molecules are grouped, it is possible
to specify that radiances for a subset of the group are to be calcualted using
the PFA method; it is not necessary to calculate radiances for the entire group
using either PFA or LBL exclusively.

It isn't necessary (or indeed possible) to specify the identities of the PFA
tables explicitly.  The processor for the forward model configuration finds
them automatically.  If any necessary ones are not found, an error is announced
and program execution is terminated.

Here's a {\tt forwardModel} command I used for a SIDS run:
{\tt\begin{verbatim}
  sidsFwm: forwardModel, type=full,  $
    signals = "b6f", $
    molecules =       [hno3,n2,o2,o3], $
    LSBpfaMolecules = [hno3,n2      ], $
    USBpfaMolecules = [        o2,o3], $
    phiWindow=4.0 degrees, $
    /skipOverlaps, $
    integrationGrid=vGridTangentSids, tangentGrid=vGridTangentSids, $
    /do_conv, $
    /do_freq_avg
\end{verbatim}}

This specifies to calculate radiances for HNO$_3$, N$_2$, O$_2$ and O$_3$ in
band 6.  In the lower sideband, radiances for HNO$_3$ and N$_2$ are calculated
using the PFA method, and radiances for O$_2$ and O$_3$ are calculated using
the LBL method.  In the upper sideband, radiances for HNO$_3$ and N$_2$ are
calculated using the LBL method, and radiances for O$_2$ and O$_3$ are
calculated using the PFA method.

For this particular configuration, the maximum absolute value of the difference
in radiance from a full LBL calculation was less than 50 millikelvins.

\label{lastpage}
\end{document}
% $Id$
