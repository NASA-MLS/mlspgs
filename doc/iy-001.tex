\documentclass[fleqn,10pt]{article}
\usepackage[dvips]{graphics,color}
\usepackage{amscd, amsfonts, amsmath, amssymb}
\usepackage{amsthm}
\usepackage{eucal}
\usepackage{pifont}


\usepackage{epsfig}
\usepackage{algorithm}
\usepackage{algorithmic}
\usepackage{verbatim}

\setlength{\topmargin}{-1.5cm}
\setlength{\headheight}{0.5cm}
\setlength{\headsep}{0.7cm}
\setlength{\topskip}{0.5cm}
\setlength{\textheight}{23.0cm}
\setlength{\footskip}{0.5cm}
\setlength{\oddsidemargin}{0.5cm}
\setlength{\evensidemargin}{0.5cm}
\setlength{\textwidth}{14.5cm}




\setcounter{tocdepth}{5}



%\numberwithin{equation}{section}

%\pagestyle{headings}
%\markboth{Igor}{Yanovsky}



\def\pdf{p}
\def\vx{{\bf x}}
\def\vy{{\bf y}}
\def\vg{{\bf g}}
\def\g{g}
\def\vh{{\bf h}}
\def\vf{{\bf f}}
\def\vu{{\bf u}}
\def\vv{{\bf v}}
\def\vnabla{{\bf \nabla}}
\def\vid{{\bf id}}
\def\va{{\bf a}}
\def\vr{{\bf r}}
\def\vb{{\bf b}}

\def\ff{\mbox f}

\def\D{\mathcal{D}}
\def\E{\mathcal{E}}

\def\trace{\mbox{trace}}
\def\divergence{\mbox{div}}

\def\I{\mathcal{I}}
\def\R{\mathcal{R}}
\def\S{\mathcal{S}}
\def\T{\mathcal{T}}


\def\m{\mbox{m}}
\def\km{\mbox{km}}
\def\nm{\mbox{nm}}
\def\cm{\mbox{cm}}
\def\s{\mbox{s}}
\def\kg{\mbox{kg}}
\def\K{\mbox{K}}
\def\N{\mbox{N}}
\def\J{\mbox{J}}
\def\Pa{\mbox{Pa}}
\def\hPa{\mbox{hPa}}
\def\Hz{\mbox{Hz}}
\def\MHz{\mbox{MHz}}
\def\AMU{\mbox{AMU}}

\def\NH{\mbox{NH}}
\def\NP{\mbox{NP}}

\def\tq{f_q^T}

\def\btau{\mbox{{\mathversion{bold}$\mathcal{T}$}}}

\begin{document}



\title{Radiative Transfer and Second Order Model for the Microwave Limb Sounder (MLS)}
\author{Igor Yanovsky \\
iy-001 \footnote{This document was written as part of MLS NRT task, under supervision of William Read, Van Snyder, Paul Wagner, and Nathaniel Livesey. The principal investigator for the project is Alyn Lambert.} }
%\small{(under supervision of William Read, Van Snyder, and Paul Wagner)}  \\
%\small{(PI: Alyn Lambert )} }


\maketitle

\tableofcontents

%\begin{comment}

\input{iy-001-01}   % Hydrostatic Model

\section{Absorption Coefficient Calculations}

$k$ - molecule \\
$j$ - line

\begin{equation*}
\S_j^k = \I_j^k(300) + \frac{\vh c El_j^k}{\log 10 k} \bigg( \frac{1}{300}-\frac{1}{T} \bigg) + \log \Bigg[ \frac{ Q^k(300) \cdot \tanh\{\vh \nu / (2kT)\} \cdot \Big( 1 + \exp\big\{ -\vh \nu_j^k / (kT) \big\}\Big) }{ Q^k(T) \cdot \Big( 1 - \exp\big\{ -\vh \nu_{0j}^k / (300k) \big\}\Big) } \Bigg]
\end{equation*}

\begin{eqnarray*}
x_j^k(\nu) = \frac{\sqrt{\log 2} (\nu - \nu_j^{'k})}{w_d^k}, \ \ \ \ y_j^k(\nu) = \frac{\sqrt{\log 2} w_{cj}^k P}{w_d^k} \Big( \frac{300}{T} \Big)^{n_{cj}^k}, \ \ \ \ z_j^k(\nu) = \frac{\sqrt{\log 2} (\nu + \nu_j^{'k})}{w_d^k},
\end{eqnarray*}

\begin{eqnarray*}
\nu_j^k &=& \Bigg[ \nu_{0j}^k + \triangle \nu_{0j}^k P\Big( \frac{300}{T} \Big)^{n_{\triangle \nu_{0j}^{k}}} \Bigg], \\
\nu_j^{'k} &=& \nu_j^k v_c, \\
v_c &=& 1 + (-6.8 \km/\s)/ c
\end{eqnarray*}
$T$ is temperature in $\K$, \\
$P$ is pressure in $\hPa$, \\
$\R^k$ is the isotopic fraction for the species, \\
$\I_j^k(300)$ is the logarithm of the integrated intensity in $\nm^2 \MHz$ at 300\K, \\
$\nu_{0j}^{k}$ is the unshifted rest line center frequency in $\MHz$, \\
$El_j^k$ is the ground state energy in $\cm^{-1}$, \\
$Q^k(T)$ is the partition function, \\
$w_d^k$ is the Doppler width in $\MHz$, \\
\ \\
$w_{cj}^k$ is the collision width in $\MHz \hPa^{-1}$ at 300\K, $n_{cj}^k$ is its temperature dependence, \\
$\nu_j^k$ is the line position frequency in $\MHz$ \\
$\nu$ is the radiation frequency in $\MHz$ \\
\ \\
$v_c$ is the Doppler shift of the line position due to the z-axis component of the spacecraft and Earth velocities, $\vv$, \\
$\triangle \nu_{0j}^k$ is the pressure shift parameter in $\MHz \, \hPa^{-1}$, and $n_{\triangle \nu_{0j}^{k}}$ is its temperature dependence. \\



\begin{eqnarray*}
\mbox{LineShape} \Big( x_j^k(\nu), y_j^k(\nu), z_j^k(\nu) \Big) = \bigg( \frac{\nu}{\nu_{0j}^k} \bigg) \bigg\{ \frac{1}{\sqrt{\pi}} \frac{y_j^k(\nu)}{(y_j^k(\nu))^2 + (x_j^k(\nu))^2} + \frac{1}{\sqrt{\pi}} \frac{y_j^k(\nu)}{(z_j^k(\nu))^2 + (y_j^k(\nu))^2} \bigg\}
\end{eqnarray*}

The cross-section $\beta^k$ for the $k$th species is given by
\begin{eqnarray*}
\beta^k = \R^k \sqrt{\frac{\log 2}{\pi}} \frac{10^{-13}}{k} \frac{P}{T \cdot w_d^k} \bigg[ \sum_j 10^{S_j^k} \mbox{LineShape} \Big( x_j^k(\nu), y_j^k(\nu), z_j^k(\nu) \Big) \bigg]
\end{eqnarray*}


% $Id$
   % Absorption

\pagebreak

\input{iy-001-03}   % Radiative Transfer

\input{iy-001-04}   % Derivatives

\pagebreak

\input{iy-001-05}   % Antenna Radiance

\section{Finite Difference Approximations for Jacobians}


\subsection*{Mixing Ratio Jacobian}

\begin{eqnarray*}
\frac{\partial I_l}{\partial f_q^k} &\approx& \frac{I_l(f_q^k + \epsilon_q^k) - I_l(f_q^k)}{f_q^k + \epsilon_q^k - f_q^k} = \frac{I_l(f_q^k + \epsilon_q^k) - I_l(f_q^k)}{\epsilon_q^k},
\end{eqnarray*}
where $\epsilon_q^k = 0.05 f_q^k$, i.e. is about $5\%$ of the value of $f_q^k$. Hence,
\begin{eqnarray*}
\frac{\partial I_l}{\partial f_q^k} &\approx& \frac{I_l(1.05 f_q^k) - I_l(f_q^k)}{0.05 f_q^k}.
\end{eqnarray*}


\subsection*{Temperature Jacobian}

\begin{eqnarray*}
\frac{\partial I_l}{\partial f_q^T} &\approx& \frac{I_l(f_q^T + \epsilon_q^T) - I_l(f_q^T)}{f_q^T + \epsilon_q^T - f_q^T} = \frac{I_l(f_q^T + \epsilon_q^T) - I_l(f_q^T)}{\epsilon_q^T},
\end{eqnarray*}
where $\epsilon_q^T = 0.05 f_q^T$.
\begin{eqnarray*}
\frac{\partial I_l}{\partial f_q^T} &\approx& \frac{I_l(1.05 f_q^T) - I_l(f_q^T)}{0.05 f_q^T}.
\end{eqnarray*}


\subsection*{Antenna Radiance Mixing Ratio Jacobian}

\begin{eqnarray*}
\frac{\partial I^A_l}{\partial f_q^k} &\approx& \frac{I^A_l(f_q^k + \epsilon_q^k) - I^A_l(f_q^k)}{\epsilon_q^k}.
\end{eqnarray*}


\subsection*{Antenna Radiance Temperature Jacobian}

\begin{eqnarray*}
\frac{\partial I^A_l}{\partial f_q^T} &\approx& \frac{I^A_l(f_q^T + \epsilon_q^T) - I^A_l(f_q^T)}{\epsilon_q^T}.
\end{eqnarray*}


\subsection*{Derivative of Height with respect to Temperature}

\begin{eqnarray*}
\frac{\partial h_l}{\partial f_q^T} &\approx& \frac{h_l(f_q^T + \epsilon_q^T) - h_l(f_q^T)}{f_q^T + \epsilon_q^T - f_q^T} = \frac{h_l(f_q^T + \epsilon_q^T) - h_l(f_q^T)}{\epsilon_q^T}.
\end{eqnarray*}


\subsection*{Derivative of Absorption Coefficient with respect to Temperature}

\begin{eqnarray*}
\frac{\partial \beta_l}{\partial f_q^T} &\approx& \frac{\beta_l(f_q^T + \epsilon_q^T) - \beta_l(f_q^T)}{f_q^T + \epsilon_q^T - f_q^T} = \frac{\beta_l(f_q^T + \epsilon_q^T) - \beta_l(f_q^T)}{\epsilon_q^T}.
\end{eqnarray*}


\subsection*{Derivative of Source Function with respect to Temperature (has not been tested)}

\begin{eqnarray*}
\frac{\partial \triangle B_i}{\partial f_q^T} &\approx& \frac{\triangle B_i(f_q^T + \epsilon_q^T) - \triangle B_i(f_q^T)}{f_q^T + \epsilon_q^T - f_q^T} = \frac{\triangle B_i(f_q^T + \epsilon_q^T) - \triangle B_i(f_q^T)}{\epsilon_q^T}.
\end{eqnarray*}


\section{Finite Difference Approximations for Hessians}

\subsection*{Radiance Hessians}

\begin{eqnarray*}
\displaystyle  \frac{\partial^2 I_l}{\partial f_q^2}  
&\approx&  D_0^q \bigg[ \frac{\partial I_l}{\partial f_q}(\vf) \bigg] \ = \ \frac{ \displaystyle  \frac{\partial I_l}{\partial f_q}(f_q + \epsilon_q) - \frac{\partial I_l}{\partial f_q}(f_q - \epsilon_q)}{2 \epsilon_q} \\
&\approx& D_-^q D_+^q \Big[ I_l(\vf) \Big] \ = \ \frac{I_l(f_q + \epsilon_q) - 2 I_l(f_q) + I_l(f_q - \epsilon_q)}{\epsilon_q^2}, \\
\displaystyle  \frac{\partial^2 I_l}{\partial f_q \partial f_{\tilde{q}}} 
&\approx&  D_0^{\tilde{q}} \bigg[ \frac{\partial I_l}{\partial f_q}(\vf) \bigg] \ = \ \frac{ \displaystyle  \frac{\partial I_l}{\partial f_q}(f_q, f_{\tilde{q}} + \epsilon_{\tilde{q}}) - \frac{\partial I_l}{\partial f_q}(f_q, f_{\tilde{q}} - \epsilon_{\tilde{q}})}{2 \epsilon_{\tilde{q}}} \\
&\approx&  D_+^q D_+^{\tilde{q}} \Big[ I_l(\vf) \Big] \ = \ \frac{ I_l(f_q + \epsilon_q,f_{\tilde{q}} + \epsilon_{\tilde{q}}) - I_l(f_q, f_{\tilde{q}} + \epsilon_{\tilde{q}}) - I_l(f_q + \epsilon_q,f_{\tilde{q}}) + I_l(f_q,f_{\tilde{q}}) }{\epsilon_q \epsilon_{\tilde{q}}} \\
&\approx&  D_0^q D_0^{\tilde{q}} \Big[ I_l(\vf) \Big] \ = \ \frac{ I_l(f_q + \epsilon_q,f_{\tilde{q}} + \epsilon_{\tilde{q}}) - I_l(f_q - \epsilon_q, f_{\tilde{q}} + \epsilon_{\tilde{q}}) - I_l(f_q + \epsilon_q,f_{\tilde{q}} - \epsilon_{\tilde{q}}) + I_l(f_q - \epsilon_q,f_{\tilde{q}} - \epsilon_{\tilde{q}}) }{4 \epsilon_q \epsilon_{\tilde{q}}},
\end{eqnarray*}
where $f_q = f_q^k$ for mixing ratios and $f_q = f_q^T$ for temperatures. If antenna radiance Jacobians are computed, then $I=I^A$ in the above formulas.


% $Id$
   % Finite Difference

\pagebreak

\input{iy-001-07}   % Frequency Averaging

\input{iy-001-08}   % Polarized Radiative Transfer

\mbox{   }
\clearpage

\section*{Units}

$p$ - pressure, $\Pa$ \\
$F$ - force, $\N$ \\
$A$ - area, $\m^2$ \\
$\displaystyle k = 1.3806503 \cdot 10^{-23} \; \frac{\m^2 \cdot \kg}{\s^{2} \cdot \K} \bigg[=\frac{\J}{\K}\bigg] =  1.3806503 \cdot 10^{-29} \; \frac{\km^2 \cdot \kg}{\s^{2} \cdot \K}$ - Boltzmann constant \\
$\displaystyle \vh = 6.62606885 \cdot 10^{-34} \; \J \cdot \s$  - Planck constant \\
$\displaystyle c = 299 792 458 \; \m / \s$ - speed of light

\begin{equation*}
1 \AMU = 1.66053886 \cdot 10^{-27} \kg
\end{equation*}
\begin{equation*}
1\J = 1 \N\cdot \m = 1 \frac{\kg \cdot \m^2}{\s^2}
\end{equation*}

\begin{equation*}
p = \frac{F}{A}
\end{equation*}
\begin{equation*}
1 \J = 1 \N \cdot \m = 1 \frac{\kg \cdot \m^2}{\s^2}
\end{equation*}
\begin{equation*}
1 \Pa = 1 \frac{\N}{\m^2} = 1 \frac{\J}{\m^3} = 1 \frac{\kg}{\m \cdot \s^2}
\end{equation*}
\begin{equation*}
1 \hPa = 100 \Pa
\end{equation*}

\begin{equation*}
\frac{k}{\vh} = 2.083664 \cdot 10^{10} \, \Hz \cdot \K^{-1} = 20836.74 \, \MHz \cdot \K^{-1}
\end{equation*}

\begin{equation*}
\displaystyle \frac{k \log 10}{\vh c} = 1.60037859 \cdot 10^{5} \frac{1}{\km \cdot \K} = 1.600386 \frac{1}{\cm \cdot \K}
\end{equation*}

\begin{equation*}
\sqrt{\frac{\log 2}{\pi}}\frac{10^{-13}\frac{\m^2 \cdot \cm}{\nm^2 \cdot \km}}{k} = 3.402155052 \cdot 10^{9} \, \frac{\s^{2} \cdot \K}{\m^2 \cdot \kg} \frac{\m^2 \cdot \cm}{\nm^2 \cdot \km} = 3.402155052 \cdot 10^{9} \frac{\K}{\hPa \cdot \nm^2 \cdot \km}
\end{equation*}

\begin{equation*}
\frac{\sqrt{2(\log 2)k}}{c} = 1.45931441 \cdot 10^{-20} \sqrt{\frac{\kg}{\K}} = 3.58116514 \cdot 10^{-7} \sqrt{\frac{\AMU}{\K}}
\end{equation*}


$\displaystyle  \beta (\km^{-1}) = \sqrt{\frac{\log 2}{\pi}} \cdot \frac{I(\nm^2 \MHz) \cdot P(\hPa) \cdot LS (unitless)}{k
\Big( \frac{\m^2 \kg}{\s^2 \K} \Big) \cdot w_d(\MHz) \cdot T(\K)}$

\ \\
Getting rid of common units and extracting the units and number in front
of the $\displaystyle \frac{I \cdot P \cdot LS}{ w_d \cdot T}$ gives

\begin{eqnarray*}
&=& \sqrt{\frac{\log 2}{\pi}} \cdot \frac{\nm^2 \cdot \hPa \cdot \s^2}{(\m^2 \cdot \kg) \cdot 1.38 \cdot 10^{-23}} \\
&&1 \hPa = \frac{\kg}{\cm \cdot \s^2} \\
&=& \sqrt{\frac{\log 2}{\pi}} \cdot \frac{\nm^2}{1.38 \cdot 10^{-23} \cdot \m^2 \cdot \cm} \\
&=& \sqrt{\frac{\log 2}{\pi}} \cdot \frac{\nm^2}{1.38 \cdot 10^{-23} \cdot \m^2 \cdot \cm} \cdot 10^{-18} \frac{\m^2}{\nm^2} \cdot 10^5 \frac{\cm}{\km} \\
&=& \sqrt{\frac{\log 2}{\pi}} \cdot \frac{10^{-13}}{1.38 \cdot 10^{-23}} \frac{1}{\km} = 3.402 \cdot 10^{9} \frac{1}{\km}.
\end{eqnarray*}

\ \\
\ \\
\ \\
\ \\
using this $O_3$ line \\
\ \\
aline =
$\{v0: \ 235709.855d00, \ gse: \ 120.2571, \ ist: \ -3.6755, \ wth: \ 2.263, \ nth: \ 0.655, \ ps: \ 0.0, \ ns: \ 0.0\}$ \\
\ \\
having this partition function and mass \\
mol\_data = \\
$\{qtp:[300.0,225.0,150.0],qpf:[3563.3512,2235.0190,1200.4721],cont:replicate(0.0,6),mass:47.984744\}$
\ \\
T = 240.0 \\
P = 215.443 \\
freqs = 235709.8550 235659.8550 235509.8550 235309.8550 234709.8550 \\
\ \\
I get for $S_j^k$  (argument to the $10^.$) \\
\ \\
-3.53299 \\
\ \\
$\beta^k$ are \\
\ \\
      1197.41 \\
      1185.60 \\
      1055.68 \\
      787.228 \\
      284.450 \\
\ \\
for the 4 frequencies above.


% $Id$
   % Appendix

\section{Glossary}
MIF - minor frame; a smallest unit of L1 data  \\
MAF - major frame; MAF contains 135 MIFs; ; there are 3500 MAFs in a day \\
\ \\
RC info - tells how a row or a column is laid out (this is only information, not data)
 

\begin{comment}

\clearpage

Let $B_i$ denote the bands, $c_i$ the channels, $p_i$ the pointings, and $s_i$ the locations in the atmosphere.

\begin{eqnarray}
I &=& \left[ \begin{array}{ccc}
\left[ \begin{array}{ccc}
c_1p_1 \\ c_1p_2 \\ \vdots \\ c_2p_1 \\ c_2p_2 \\ \vdots \\ c_np_m 
\end{array} \right]  {B_1 \atop \mbox{VectorValue\_T}} \\ \\
\left[ \begin{array}{ccc}
c_1p_1 \\ c_1p_2 \\ \vdots \\ c_2p_1 \\ c_2p_2 \\ \vdots \\ c_np_m 
\end{array} \right]  {B_2 \atop \mbox{VectorValue\_T}} \\
\vdots
\end{array} \right] \mbox{Vector\_T}
\end{eqnarray}

VectorTemplate\_T contains geolocation information, and other information, for a specific Vector\_T.

VectorValue\_T\%values$\underbrace{(:,:)}_{(1:mn,1)}$

\clearpage

\section{Interpreting 1D and 2D Scans}

In 1D case, state vector values are functions of height (of $\xi$) and not functions of angle ($\phi$).  That is, in different locations around the Earch, the quantities are homogeneous at the same heights from the surface of the Earth.  We can not make such an assumption in 2D case.

In 1D, a single scan line of radiances (and convolved radiances) can be thought of being constructed as the space craft circles the earth, sequentially scanning higher tangent lines of a scan line.

In 2D, the radiances at all tangent heights of a single scan line are being calculated via different pointings.  That is, we can assume that the sensor points at different tangent heights from a single position.  As the space craft orbits the Earth, neighboring scan lines are being calculated in a similar manner.

\clearpage

\input{FORTRAN_Forward_Model_Code}

\end{comment}

\end{document}


% $Id$

