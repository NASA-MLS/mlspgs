\section{Compiling and linking to the CFM}
\label{app:Compilation}

This appendix describes the steps to be followed in building and using 
the cfm library.

\subsection{Building the libraries}

In fact, there are two versions of the library, one which is less
complete because it requires additional libraries to be supplied, and
another fuller version which gathers all the compiled objects together.

We assume you are able to obtain the mlspgs software and a suitable
\snippet{.configure} file. You know your mlspgs distribution is complete
and your \snippet{.configure} file is suitable if you are able to build
with it. Ensuring that both are up-to-date may require access to the
\snippet{cvs} utility. We will assume you are able to confirm that both
are up-to-date, too.

The key to building the cfm libraries is the \snippet{make} utility. If
you have not done so already, you should use the pseudotarget
\snippet{update} to build the uniquely-named subdirectories where
\snippet{make} will store the compiled objects.

The command to do this, issued from the root, mlspgs directory, is
\begin{quote}
\snippet{make update}
\end{quote}

The next step is to build and install the libraries. This is easily done by
\begin{quote}
\snippet{make install-cfm}
\end{quote}

\begin{quote}
\snippet{make install-fullcfm}
\end{quote}

In response to these commands \snippet{make} will compile the sources
and add the resulting objects into two library files, whie also creating
a unifying \snippet{.mod} file. These files will be moved into a special
installation directory, normally a subdirectory of your root mlspgs
directory named \snippet{\$MLSCONFG} where the variable
\snippet{MLSCONFG} is set in your .configure file.

We now look into use of the two library files and the unifying
\snippet{.mod} file.

\subsection{Using the unifying \snippet{.mod} file}

When compiling any Fortran module that USEs a second module, the
compiler requires access to the \snippet{.mod} file created when it
compiled that second module. Because of this, a user program that USEs
any of the procedures supplied in the callable forward model can be
compiled only if a \snippet{.mod} file is supplied that was created by
the same Fortran compiler for a library module.  This library-wide
\snippet{.mod} file is our unifying \snippet{.mod}. It is named
\snippet{cfm.mod}. Using our mockup.f90 file for an example, we can
illustrate how to compile, assuming \snippet{cfm.mod} is in
\snippet{\$INSTALLDIR}.
\begin{quote}
\snippet{ifort -c -I \$INSTALLDIR mockup.f90}
\end{quote}

You won't actually have to issue the above command. It's shown for purely for
illustration.

\subsection{Using the \snippet{libcfm.a} file}

The Makefiles already allow the easy implementation of this library. You
need only cd to mlspgs/cfm, substitute the existing \snippet{mockup.f90}
with your own, and then type
\begin{quote}
\snippet{make update\\make}
\end{quote}

in response, you should see Makefile execute the following command (or something close to
it)
\begin{verbatim}
ifort -static -arch ia32 -o mockup mockup.o   \
                  -L./ -lcfm \
    -L../../cloudfwdm/IFC.Linux.ifc11 -lcloud \
    -L../../fwdmdl/IFC.Linux.ifc11 -lfwdmdl \
     \
                  -L../../lib/IFC.Linux.ifc11 -lmls \
    -L../../bin/IFC.Linux.ifc11 -lutctotai \
            -L/software/toolkit/ifc11/toolkit/lib/linux -lPGSTK \
    -L/software/toolkit/ifc11/hdfeos/lib/linux -lhdfeos \
    -L/software/toolkit/ifc11/hdfeos5/lib/linux -lhe5_hdfeos \
            -L/software/toolkit/ifc11/hdf/lib -lmfhdf \
    -ldf \
    -lz \
    -ljpeg -lsz \
            -L/software/toolkit/ifc11/hdfeos/lib/linux -lGctp \
    -L/software/toolkit/ifc11/pvm3/lib/LINUX -lfpvm3 -lgpvm3 -lpvm3 \
     \
     \
    -L../../blas/IFC.Linux.ifc11 -lmlspack \
    -L/software/toolkit/ifc11/hdf5/lib -lhdf5_fortran -lhdf5  \
     -lz  \
\end{verbatim}
Note the second line where \snippet{-L./ -lcfm} tells the linker to
include needed objects from the file \snippet{libcfm.a}.

The result will be to put a newly-built executable named
\snippet{mockup} in the uniquely-named subdirectory
\snippet{\$MLSCONFG}.

\subsection{Using the \snippet{libcfm_all.a} file}

It might be more convenient to build outside the mlspgs/cfm
subdirectory, or it may be that some of the the libraries shown in the
command above conflict with versions needed to build your program. In
this case, you can use the full version of the library,
\snippet{libcfm_all.a}.

An example might be once again our mockup program.

\begin{verbatim}
 ifort -static -arch ia32 -o mockup mockup.o \
    -L$INSTALLDIR -lcfm_all
\end{verbatim}

%%% Local Variables: 
%%% mode: latex
%%% TeX-master: "cfm"
%%% End: 
