\section{Review of overall structure}

This section introduces the various data entities and functionalities
associated with the MLSCFM software.  More details on specific derived type
definitions and function / subroutine invocation are given in the
reference section.

\subsection{MLS radiance measurements and related parameters}

The calibrated MLS radiance measurements are stored, along with radiance
geolocation information in the MLS Level~1B (L1 hereafter) files.  The
L1 files of relevance to MLSCFM applications are the radiance files and
the orbit/attitude file.  These files typically have a daily granularity
running from midnight to midnight UT.

\subsubsection{Major frames and minor frames}

The principal divisions within the MLS L1 files are known as \emph{major
  frames} (MAFs) and \emph{minor frames} (MIFs).  A major frame is one
complete vertical scan of the MLS GHz and THz antennas (including limb
scanning, calibration and retrace activities).  The antennae move
continuously during a MAF and radiances are integrated at 1/6\,s
intervals known as minor frames (MIFs).  During a standard scan limb
views are taken for the first 120 MIFs (114 for the THz antenna).  The
remaining $\sim$26 MIFs in the MAF are spend performing calibration and
antenna retraces.  The on board software attempts to lock the MLS scans
to the Aura orbit, such that scans are made at the same latitudes each
orbit.  In order to accomplish this, occasional `leap MIFs' must be
added in selected MAFs.

\subsubsection{Bands and channels}
The other divisions of relevance for L1 data are band and channel.  MLS
makes measurements in 34 spectral bands, each of which has from 4 to 129
channels.  The bands are described using the MLS `signal designation'
nomenclature
\begin{quote}
\snippet{[Radiometer name].[Band].[Switch].[Spectrometer]}
\end{quote}
Examples are \snippet{R1A:118.B1F:PT.S0.FB25-1}, or
\snippet{R3:240.B7F:O3.S0.FB25-7}.  The complete list of MLS bands is
given in the \snippet{MLS-Aura_L2Cal-Signals_\ldots} file supplied with
the software.  For most bands, the number of channels is quoted in the
\snippet{Spectrometer} clause (\snippet{FB25}, \snippet{MB11} or
\snippet{WF4}) and are numbered 1\ldots$n$.  The exception are the
Digital Autocorellator Spectrometers (DACs) which have 129 channels
numbered 0\ldots128.

\subsubsection{MLS L1 files}

Three radiances files are produced for each day of MLS observations.
The L1BRADG file contains the radiances for the majority of the GHz
bands.  The L1BRADD file contains the radiances for the four MLS digital
autocorrelator spectrometers, also measuring GHz spectral regions.  The
L1BRADT file contains radiances measured by the MLS THz bands.

L1BRAD files are stored in HDF5 (not HDFEOS) and are flat files
containing many HDF SDs.  The most important of these are radiances
which are stored as arrays, named after the signal designation described
above.  The arrays are dimensioned channel, MIF, MAF with channel the
most rapidly changing index.  Precisions are stored for each radiance,
in a separate SD with the name `\snippet{<signal> precision}' (note the
intervening space).  Radiances that should not be used are indicated by
a negative precision values.  The additional quantity `\snippet{<signal>
  Baseline}` values (dimensioned channel, MAF) should be added (for each
MIF per channel/MAF) before radiances are used.  Similarly, the
`\snippet{<signal> Baseline precision}' should be added in quadrature to
the precisions.  (The baseline `AC' and `DC' terms are diagnostic and may
be ignored).  All numbers refer to brightness temperature in Kelvins.

The L1BOA (orbit/attitude) file contains information on spacecraft
location and the locations of the GHz and THz tangent points.  All
angles are degrees and all dimensions are meters.  Quantities are mostly
dimensioned MIF, MAF, with vectors dimensioned 3, MIF, MAF.

Time is the EOS standard TAI93 time (number of seconds, including leap
seconds, since midnight UT on January 1st, 1993.

\subsubsection{A note on \snippet{GeodAngle}}

An important quantity for MLS L1 (and L2) data is the so-called `Orbit
Geodetic Angle' (referred to in the software as \snippet{GeodAngle}).
This is used as a horizontal coordinate for most quantities, and is
defined as the geodetic great circle angle along the Aura orbit track,
increasing with time from the first ascending equator crossing of the
day (defined as 0\degsym).  Unusually for an angle, this number accumulates
beyond 360\degsym.  So, the first complete orbit in the day is
0\degsym\,--\,360\degsym, the second is 360\degsym\,--\,720\degsym\ etc.

\subsection{Constructing the MLS state vector}

The MLS state vector consists of a heterogeneous collection of
quantities, some geophysical, some instrument related.  As with the
measurement vector (which mainly consists of MLS L1 radiance), these are
described structures consisting of `quantities' and `vectors' described
in the following subsection.

The main geophysical quantities are consecutive vertical profiles of
atmospheric temperature and composition on a regular grid in geodetic
angle and pressure (actually $-\log_{10}\left(p/\text{hPa}\right)$).

The most important instrument quantities are the `tangent pressures' for
the MLS GHz and THz limb views, which are dimensioned MIF, MAF.  Others
include terms like `baseline', which describe additional terms used to
capture unexplained spectrally-flat MLS signals.  Note, such terms are
not to be confused with the, unfortunately similarly named, baseline
quantity in the L1 files.

\subsection{Quantities and Vectors in MLSCFM}

\subsection{Sources of information for MLS state vector components}
\mycomment{Where can they get tangent pressure information, extinction etc.?}

\subsection{Defining forward models}

\subsection{Invoking forward models}

%%% Local Variables: 
%%% mode: latex
%%% TeX-master: "cfm"
%%% End: 
