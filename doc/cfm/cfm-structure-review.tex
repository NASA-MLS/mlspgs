\section{Review of overall structure}

This section introduces the various data entities and functionalities
associated with the MLSCFM software.  More details on specific derived type
definitions and function / subroutine invocation are given in the
reference section.

\subsection{MLS radiance measurements and related parameters}

The calibrated MLS radiance measurements are stored, along with radiance
geolocation information in the MLS Level~1B (L1 hereafter) files.  The
L1 files of relevance to MLSCFM applications are the radiance files and
the orbit/attitude file.  These files typically have a daily granularity
running from midnight to midnight UT.

\subsubsection{Major frames and minor frames}

The principal divisions within the MLS L1 files are known as \emph{major
  frames} (MAFs) and \emph{minor frames} (MIFs).  A major frame is one
complete vertical scan of the MLS GHz and THz antennas (including limb
scanning, calibration and retrace activities).  The antennae move
continuously during a MAF and radiances are integrated at 1/6\,s
intervals known as minor frames (MIFs).  During a standard scan limb
views are taken for the first 120 MIFs (114 for the THz antenna).  The
remaining $\sim$26 MIFs in the MAF are spend performing calibration and
antenna retraces.  The on board software attempts to lock the MLS scans
to the Aura orbit, such that scans are made at the same latitudes each
orbit.  In order to accomplish this, occasional `leap MIFs' must be
added in selected MAFs.

\subsubsection{Bands and channels}
The other divisions of relevance for L1 data are band and channel.  MLS
makes measurements in 34 spectral bands, each of which has from 4 to 129
channels.  The bands are described using the MLS `signal designation'
nomenclature
\begin{quote}
\snippet{[Radiometer name].[Band].[Switch].[Spectrometer]}
\end{quote}
Examples are \snippet{R1A:118.B1F:PT.S0.FB25-1}, or
\snippet{R3:240.B7F:O3.S0.FB25-7}.  The complete list of MLS bands is
given in the \snippet{MLS-Aura_L2Cal-Signals_\ldots} file supplied with
the software.  For most bands, the number of channels is quoted in the
\snippet{Spectrometer} clause (\snippet{FB25}, \snippet{MB11} or
\snippet{WF4}) and are numbered 1\ldots$n$.  The exception are the
Digital Autocorrelator Spectrometers (DACs) which have 129 channels
numbered 0\ldots128.

\subsubsection{MLS L1 files}

Three radiances files are produced for each day of MLS observations.
The L1BRADG file contains the radiances for the majority of the GHz
bands.  The L1BRADD file contains the radiances for the four MLS digital
autocorrelator spectrometers, also measuring GHz spectral regions.  The
L1BRADT file contains radiances measured by the MLS THz bands.

L1BRAD files are stored in HDF5 (not HDFEOS) and are flat files
containing many HDF SDs.  The most important of these are radiances
which are stored as arrays, named after the signal designation described
above.  The arrays are dimensioned channel, MIF, MAF with channel the
most rapidly changing index.  Precisions are stored for each radiance,
in a separate SD with the name `\snippet{<signal> precision}' (note the
intervening space).  Radiances that should not be used are indicated by
a negative precision values.  The additional quantity `\snippet{<signal>
  Baseline}` values (dimensioned channel, MAF) should be added (for each
MIF per channel/MAF) before radiances are used.  Similarly, the
`\snippet{<signal> Baseline precision}' should be added in quadrature to
the precisions.  (The baseline `AC' and `DC' terms are diagnostic and may
be ignored).  All numbers refer to brightness temperature in Kelvins.

The L1BOA (orbit/attitude) file contains information on spacecraft
location and the locations of the GHz and THz tangent points.  All
angles are degrees and all dimensions are meters.  Quantities are mostly
dimensioned MIF, MAF, with vectors dimensioned 3, MIF, MAF.

Time is the EOS standard TAI93 time (number of seconds, including leap
seconds, since midnight UT on January 1st, 1993.

\subsubsection{A note on \snippet{GeodAngle}}

An important quantity for MLS L1 (and L2) data is the so-called `Orbit
Geodetic Angle' (referred to in the software as \snippet{GeodAngle}).
This is used as a horizontal coordinate for most quantities, and is
defined as the geodetic great circle angle along the Aura orbit track,
increasing with time from the first ascending equator crossing of the
day (defined as 0\degsym).  Unusually for an angle, this number accumulates
beyond 360\degsym.  So, the first complete orbit in the day is
0\degsym\,--\,360\degsym, the second is 360\degsym\,--\,720\degsym\ etc.

\subsection{Constructing the MLS state vector}

The MLS state vector consists of a heterogeneous collection of
quantities, some geophysical, some instrument related.  As with the
measurement vector (which mainly consists of MLS L1 radiance), these are
described structures consisting of `quantities' and `vectors' described
in the following subsection.

The main geophysical quantities are consecutive vertical profiles of
atmospheric temperature and composition on a regular grid in geodetic
angle and pressure (actually $-\log_{10}\left(p/\text{hPa}\right)$).

The most important instrument quantities are the `tangent pressures' for
the MLS GHz and THz limb views, which are dimensioned MIF, MAF.  Others
include terms like `baseline', which describe additional terms used to
capture unexplained spectrally-flat MLS signals.  Note, such terms are
not to be confused with the, unfortunately similarly named, baseline
quantity in the L1 files.

\subsection{Quantities and Vectors in MLSCFM}

The heterogeneous nature of the MLS state and measurement vectors
dictates that a flexible system be used for their description.  In the
MLSCFM software vectors consist of `templates' and `values'.  Separating
the vector template from its values allows for efficient storage, and
more importantly tracking, of related quantities (e.g., the state, the
initial guess, and the \emph{a priori}, or the measurements and their
forward model estimates).  Vector templates are, in turn, composed of
one or more `quantity templates'.  While quantities do not strictly
speaking exist in isolation from vectors, the term `quantity' is often
loosely used in the software to refer to either a specific quantity
within a vector, and/or its template.

Vector quantities typically describe a single- or multi-valued physical
quantity (temperature profiles, abundances for one species, radiances
for one band).  They can have up to three dimensions, named `channels',
`surfaces' and 'instances' (channels is the fastest changing index,
instances the slowest).  Instance is a time-like or horizontal
dimension, and typically refers to either profiles (for geophysical
quantities) or major frames (for instrument quantities such as measured
radiances).  Similarly, `surfaces' can refer to pressure levels for
geophysical quantities, or minor frames for instrument quantities.  The
channels dimension is typically only used for radiance quantities.  The
actual fortran arrays storing vector quantity values are dimensioned
\snippet{( channels * surfaces, instances )}.  While increasing the
complexity of access for multi-channel quantities, this simplifies the
storage of matrices in the software.

`Coherent' quantities are those where the surfaces are the same for
every instance.  Geophysical quantities, being reported on a fixed
pressure grid, are coherent, while radiance quantities where each MAF
involves views of different altitudes, are incoherent.

`Stacked' quantities are those where the different surfaces in each
instance share the same horizontal geolocation (latitude, longitude,
time etc.).  Again, state vector quantities are typically stacked, with
the quantity describing vertical profiles, while radiance quantities are
unstacked, as each minor frame within the scan has a slightly different
geolocation.

Geophysical quantities are essentially always coherent and stacked,
while `minor frame' (i.e., more instrument related) quantities are
always incoherent and unstacked.  Two `minor frame' vector quantities
describing parameters associated with the same MLS module (GHz, THz or
spacecraft) can, by definition, be assumed to have the same geolocation.
This important assertion is relied upon throughout the forward model
software.  Conversely, in the state vector, geolocation information does
not need to be consistent from quantity to quantity.  The vertical (or
horizontal) grids for temperature need not relate to that for
composition, nor do the grids for all molecules need to be identical.

In addition to geolocation information, quantity templates typically
contain additional information to describe the quantity.  This includes
the molecule for an abundance-related quantity, the signal, radiometer
and module for a radiance-related quantity etc. etc.

\subsection{Matrices in MLSCFM}

Matrices in the CFM software are defined by the vectors that describe
their rows and columns.  For example, the Jacobian matrix relates a
state vector to a measurement vector, while a covariance matrix
describes the covariance of a single vector. The matrices are stored in a blocked manner,
with subblocks relating one instance of one quantity in the rows vector
with an instance of another quantities in the columns vector.  The
tomographic nature of the MLS retrievals \citep{LiveseyEtal06} means
that many of the blocks are all zero and can be omitted for efficiency
of storage and computation.  Similarly, many of the individual blocks
are themselves sparse, and can be stored as such in the software.

The MLSCFM software contains two main modules for dealing with matrices.
\snippet{MatrixModule_0} contains the software for storing and
manipulating individual blocks, while \snippet{MatrixModule_1} contains
the software that describes matrices as a collection of
\snippet{Matrix_0} blocks, and the algorithms for operating upon these
`\snippet{Matrix_1}' entities.  

\subsection{Sources of information for MLS state vector components}

The MLS forward models require a wide variety of vector quantities as
input.  Typically, two state vectors are supplied, one
(\snippet{ForwardModelIn}) containing the quantities of interest (i.e.,
those quantities for which a solution is sought for some or all
elements), and the other (\snippet{ForwardModelExtra}) containing other
(e.g., calibration) quantities, also needed by the forward model.  The
majority of these lesser quantities will be defined and filled for the user by
subroutines supplied as part of the MLSCFM software, and need little
discussion here.

However, an important exception is the tangent pressure information, for
which there are a variety of approaches.  Firstly, the value retrieved
as part of the MLS processing can be read from the L2 `DGM' file.  This
is useful for initial `O--F' studies, but is not feasible when radiance
assimilation is to be performed in real time.  In addition the
production processing's use of GMAO temperature as an \emph{a priori}
introduces potential biases in this approach.

An alternative approach is to estimate the tangent pressure given
knowledge of atmospheric temperature and geopotential height, Aura
orientation and MLS pointing.  However, the pointing knowledge is
imperfect and can also introduce biases.

The best approach is likely to solve for tangent pressure for each minor
frame as part of any radiance assimilation activity.  Whether this
proves to be practical is the subject of further investigations.
(Aside, one approach may be to take GMAO GPH as `truth' and retrieve a
-- hopefully slowly varying -- Aura attitude offset term).

\subsection{Defining forward models}
The forward models must be defined before they can be invoked. The procedure
used to define them is (\snippet{ConstructForwardModelConfig}). Fortunately,
at this stage we anticipate that the necessary calls to this function
will be made automatically during the setup phase, lifting the burden from
the library user. Subsequent versions may be enhanced by allowing the user
to construct and use additional forward models on-the-fly.

In general the forward models are defined according to two classes of
characteristics: global characteristics which are common to all, and specific
characteristics which may vary from model to model. One key characteristic that
varies is whether the forward model is a linearized model or the full model. The
linearized model has the advantage of being quicker to compute, but at the price
of being less accurate.


\subsection{Invoking forward models}
In the example at the end of this document we show how to invoke a forward
model. Especially important is the order of steps leading up to its invocation.
From a very general perspective we first perform an overall setup, loading into
memory all the global data. Then we define the spatial grids. Then we define the
vector and quantity templates. Two vector templates in particular deserve
mention: the state template will hold everything that is known about the
atmosphere including temperatures and molecular abundances; the measurement
template will hold all the radiances. Then we create
instances of the state and measurement vectors. Filling the state vector is a
key task which the user will want to adapt according to need. 
Finally comes the call to the forward model which fills the measurement
vector and optionally the Jacobian matrix.
%%% Local Variables: 
%%% mode: latex
%%% TeX-master: "cfm"
%%% End: 
