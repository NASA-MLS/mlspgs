
\section{Reference}
\subsection{Conventions}
\begin{description}
\item[int i] -- i is an integer
\item[log p] -- p is an logical
\item[str name] -- name is a character variable
\item[typ datatype x] -- x is an instance of user-defined data type datatype
\item[[some\_arg]] -- some\_arg is optional
\item[*some\_arg] -- some\_arg is a pointer
\item[array(:)] -- array is an array
\end{description}

\subsection{Data types}
Each data type will be introduced, defined, and described by its typical use.
\begin{description}
\item[real(r4)] -- real numbers with decimal precision at least 5
\item[real(r8)] -- real numbers with decimal precision at least 13
\item[Range\_T] -- pair of ints: ( Bottom, Top )
\item[ForwardModelConfig\_T] -- forward model configuration
\item[HGrid\_T] -- horizontal grid
\item[FGrid\_T] -- frequency grid
\item[MLSFile\_T] -- file name, location, type, etc.
\item[VGrid\_T] -- vertical grid
\item[QuantityTemplate\_T] -- e.g. Temperature: values, locations, etc.
\item[VectorTemplate\_T] -- container for QuantityTemplate\_Ts
\item[Vector\_T] -- instance of a VectorTemplate\_T
\item[ForwardModelStatus\_T] -- state used when modeled as state machine
\item[Matrix\_T] -- e.g., Jacobian
\item[GriddedData\_T] -- e.g., meteorology
\end{description}


\subsection{Procedures and functions}
Each procedure will be introduced and described in a very general way. The
example near the end of this document should show sufficient detail to make
how and when to use each perfectly clear.

\noindent In presenting the list of procedures, one ordering we
might have followed would have gone from first called to last called.
Another ordering would have been a descending hierarchy. 

\noindent Steering a middle course, we have tried to do both.
\begin{description}
\item[CFM\_MLSSetup] ( int retVal, \newline
  [typ ForwardModelConfig\_T *ForwardModelConfigDatabase(:)] )
\item[typ HGrid\_T CreateExplicitHGrid] ( r8 time, r8 *phi(:), \newline
  r8 *solarTime(:), r8 *solarZenith(:), int error )
\item[CFM\_CreateRegularHGrid] ( typ HGrid\_T hgrid, str hgrid\_name, 

   r8 origin, r8 spacing, int firstMaf, int lastMaf, typ MLSFile\_T L1BFile )
\item[typ QuantityTemplate\_T cfm\_CreateQtyTemplate] ( \newline
    typ MLSFile\_T *l1bFile, int firstMAFIndex, int lastMAFIndex, \newline
    log propertyTable(:,:), int unitstable(:,:), \newline
    [typ FGrid\_T qFGrid], [typ HGrid\_T HGrid], [typ VGrid\_T VGrid], \newline
    [typ QuantityTemplate\_T MifGeolocation], \newline
    [log qLogBasis], [r8 qMinimumV], [str qInstModule], [str qRadiometer], \newline
    [str qMolecule], \newline
    [str qSignal], [str qType] )
\item[ConstructVectorTemplate] ( typ QuantityTemplate\_T quantities(:), \newline
  int selected(:), typ VectorTemplate\_T vectorTemplate, str forwhom )
\item[typ VectorTemplate\_T CreateVecTemplate] ( \newline
  typ QuantityTemplate\_T *quantityTemplates(:) )
\item[ForwardModel] ( typ ForwardModelConfig\_T Config, \newline
  typ Vector\_T FwdModelIn, 
  typ Vector\_T FwdModelExtra, 
  typ Vector\_T FwdModelOut, typ ForwardModelStatus\_T fmStat, \newline
  [typ Matrix\_T Jacobian], [ Vector\_T vectors(:)] )
\item[int InitializeMLSFile] ( typ MLSFile\_T item, [int type], [int access], \newline
  [str content], [str name], \newline
  [str shortName], [int HDFVersion], [int recordLength], \newline
  [typ Range\_T PCFIdRange], [int PCBottom], [int PCTop] )
\item[mls\_openFile] ( typ MLSFile\_T MLSFile, [int error] )
\item[mls\_CloseFile] ( typ MLSFile\_T MLSFile, [int error] )
\item[CFM\_MLSCleanup]
\end{description}

%%% Local Variables: 
%%% mode: latex
%%% TeX-master: t
%%% End: 
