\section{Introduction and background}
\label{sec:Introduction}

This document describes the external interface for, and other details
of, the Aura Microwave Limb Sounder (MLS) `Callable Forward Model' (CFM
hereafter).  More information on Aura MLS can be found in
\citet{WatersEtal06}.  The task of the MLS forward model(s) is to
compute MLS radiances corresponding to supplied profiles of atmospheric
temperature and composition (along with auxiliary quantities,
spectroscopy, and calibration information).  More information on the MLS
forward models can be found in \citet{ReadEtal06} for the `full forward
model' and \citet{SchwartzEtal06} for the `polarized model' used for
Zeeman-split oxygen lines in the mesosphere.  \citet{WuEtal06} describe
the cloud scattering forward model (used mainly for pre-launch
simulations).

The CFM is an extraction of the forward models from the MLS Level~2
processing software (MLSL2 hereafter).  Extracting the forward models
into stand-alone Fortran~90 modules enables their use in a variety of
applications, including:
\begin{itemize}
\item Data assimilation systems evaluating and/or assimilating MLS
  radiances.
\item Off-line calculations to characterize MLS observations and to
  evaluate systematic errors in MLS-like observing systems.
\item OSSE frameworks to further evaluate MLS-like observing systems and
  ascertain their ability to answer specific atmospheric science questions.
\end{itemize}

In its current form, this document is mainly aimed at the first of these
applications.

Section~\ref{sec:Review} reviews the overall structure of the CFM,
including an outline of relevant underlying aspects of the MLS
instrument, operations and data processing algorithms.
Section~\ref{sec:Reference} serves as a reference for the various data
types and procedures that constitute the CFM software.
Appendix~\ref{app:Example} gives examples of code that invokes the CFM
that can serve as a starting point for assimilation studies etc.
Appendix~\ref{app:Compilation} describes the steps necessary to compile
the CFM software.  Finally, Appendix~\ref{app:Lists} provides tables and
lists of relevant terms (units etc.).


%%% Local Variables: 
%%% mode: latex
%%% TeX-master: "cfm"
%%% End: 
