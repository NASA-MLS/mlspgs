
\section{Reference}
\subsection{Conventions}

\subsection{Data types}

\subsubsection{\snippet{HGrid\_T}}
\lstinputlisting{hgridtype.f9h}
\subsubsection{\snippet{VGrid\_T}}
\lstinputlisting{vgridtype.f9h}
\subsubsection{\snippet{Vector\_T}}
\lstinputlisting{vectortype.f9h}
\subsubsection{\snippet{VectorValue\_T}}
\lstinputlisting{vectorvaluetype.f9h}
\subsubsection{\snippet{VectorTemplate\_T}}
\lstinputlisting{vectortemplatetype.f9h}
\subsubsection{\snippet{QuantityTemplate\_T}}
\lstinputlisting{quantitytemplatetype.f9h}
\subsubsection{\snippet{Enumerate types and other types}}
\begin{itemize}
 \item real(r4): real numbers with decimal precision at least 5.
 \item real(r8): real numbers with decimal precision at least 13.
 \item t\_vgridCoordinate: enumerate values are TBD.
 \item t\_quantityType: enumerate values are TBD.
 \item t\_molecule: enumerate values are TBD.
 \item phyq\_invalid: a constant integer to indicate an invalid 
                      physical quantity. Value is TBD.
\end{itemize}

\subsection{Procedures and functions}

\mycomment{Each procedure will be introduced and described in a very
  general way. The example near the end of this document should show
  sufficient detail to make how and when to use each perfectly clear.

  In presenting the list of procedures, one ordering we might have
  followed would have gone from first called to last called.  Another
  ordering would have been a descending hierarchy.

  Steering a middle course, we have tried to do both.}

\subsubsection{\snippet{AddQuantityTemplateToDatabase}}
Puts a given template to a given array (database) by allocating
a new array, and deallocating the old array if database is associated.

Dummy variables:
\begin{itemize}
 \item database: an one-dimensional array of QuantityTemplate\_T objects,
                 or a NULL pointer.
 \item template: a quantity template to be added.
\end{itemize}

\subsubsection{\snippet{CFM\_MLSCleanup}}
Deallocates memory and closes files what CFM\_MLSSetup allocated or opened.

\subsubsection{\snippet{CFM\_MLSSetup}}
Gets the ForwardModelConfig\_T object, the filedatabase, and initializes 
internally used variables.

Dummy variables:
\begin{itemize}
 \item fmConfig: an object to be filled with forward model settings read 
                 from a L2CF
 \item filedatabase: an one-dimensional array of MLSFile\_T objects to be 
                     filled in with necessary and already opened 
                     spectroscopy file(s) and engineering data file(s).
                     This subroutine will allocate the array.
 \item fakeChunk: used internally, to be passed to other subroutines.
 \item error: is set to 0 if this subroutine runs successfully, a non-zero
              number otherwise.
\end{itemize}

\subsubsection{\snippet{CreateQuantityTemplate}}
Creates a QuantityTemplate\_T object.

Dummy variables:
\begin{itemize}
 \item filedatabase: an array of already-opened MLSFile\_T objects.
 \item chunk: a MLSChunk\_T object returned by CFM\_MLSSetup
              for internal used.
 \item qHGrid: the (x,y) coordinates of samples of the spacecraft's
               path.
 \item qVGrid: the z coordinates of samples the spacecraft's path.
 \item qLogBasis: true if points in qHGrid and qVGrid are calculated
                  on a log scale.
 \item qMinimumV:
 \item qInstrumentModule: either "GHz" or "THz."
 \item qMolecule: a string naming the molecule associated with the
                  quantity.
 \item qType: a string naming the type of the quantity.

\end{itemize}
Return: a quantity template.

\subsubsection{\snippet{CreateVector}}
Creates a Vector\_T object.

Dummy variables:
\begin{itemize}
 \item vecTemplate: a VectorTemplate\_T object describing the vector to be
                    returned.
 \item qtyTemplateDatabase: an array of QuantityTemplate\_T objects describing
                            the quantities in the returned vector.
\end{itemize}

Return: an empty vector

\subsubsection{\snippet{CreateVectorTemplate}}
Creates a vector template.


Dummy variables:
\begin{itemize}
 \item qtyTemplateDB: an array of QuantityTemplate\_T objects describing
                      all available quantities.
 \item selected: an array of logicals of the same size as qtyTemplateDB,
                 whose elements are true where the corresponding
                 qtyTemplateDB's elements are used by the vector of
                 this vector template.
\end{itemize}

Return: a VectorTemplate\_T object.

\subsubsection{\snippet{ForwardModel}}
Relates an atmospheric state, representing by a state vector and an optional 
extra state vector, to MLS radiances, also representing by a vector.


Dummy variables:
\begin{itemize}
 \item fmConfig: a ForwardModelConfig\_T object containing the settings for
                 the forward model.
 \item stateVectorIn: a Vector\_T object representing an atmospheric state
 \item stateVectorExtra: a Vector\_T object representing the atmospheric 
                         state's elements that are not captured in 
                         stateVectorIn. There is no constraint on what must 
                         be or must not be in either stateVectorIn or 
                         stateVectorExtra as long as they do not contain 
                         duplicate information.
 \item radiances: the radiances corresponding to the atmospheric state 
                  given by stateVectorIn and stateVectorExtra, to be filled 
                  in by this subroutine.
 \item fmStat: used internally and to be filled by this subroutine.
 \item jacobianMatrix: the matrix equivalent of the forward model with the 
                       specific setting in fmConfig.
\end{itemize}

\subsubsection{\snippet{InitializeMLSFile}}
Initializes an MLSFile\_T object to be opened later.

Dummy variables:
\begin{itemize}
 \item content: TBD for now.
 \item name: the name of the file.
 \item shortName (optional): a short alias for the file.
 \item type: either l\_hdf or l\_ascii, actual values are TBD.
 \item access: one of DFACC_CREATE, DFACC_RDONLY, DFACC_READ, 
               or DFACC_RDWR of HDF library
 \item file: the file object to be initialized
\end{itemize}

\subsubsection{\snippet{mls\_CloseFile}}
Closes an open file.

Dummy variables:
\begin{itemize}
 \item file: the file to be closed in an initialized MLSFile\_T object.
\end{itemize}

\subsubsection{\snippet{mls\_openFile}}
Opens a file.

Dummy variables:
\begin{itemize}
 \item file: the file to be opened in an initialized MLSFile\_T object.
 \item error: a value to store error code in. If the file is opened 
              successfully, the error code is PGS\_S\_SUCCESS.
\end{itemize}

%%% Local Variables: 
%%% mode: latex
%%% TeX-master: "cfm"
%%% End: 
