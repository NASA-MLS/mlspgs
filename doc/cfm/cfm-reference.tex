\section{Reference}
\label{sec:Reference}

\subsection{A note on tables and strings}

The MLSL2 software makes heavy use of tables to hold commonly referred
to entities, such as strings, or entries in the `Level 2 Configuration
File' (L2CF hereafter).  This is witnessed by the use of integers in
most of the derived types below to hold items such as the name of a
quantity template, the type of information it holds (temperature,
composition etc.), the specific molecule for composition quantities, the
specific signal for radiance quantities, etc.  However, the MLSCFM
software presents string arguments to most of the routines supplied that
relieves the calling code of having to be aware of the underlying
tables.

\subsection{Data types}

\subsubsection{\snippet{HGrid\_T}}
The \snippet{HGrid\_T} data type describes the `horizontal' coordinate
system for a vector quantity template.  It is only used for `stacked'
quantities.  Unstacked quantities (those where the different surfaces in
an instance do not share the same geolocation) have their horizontal
coordinate information stored in the quantity template itself.

\lstinputlisting{hgridtype.f9h}

\subsubsection{\snippet{VGrid\_T}}
The \snippet{VGrid\_T} data type describes the `vertical' coordinate
system for a vector quantity template.  It is only used for `coherent'
quantities.  Incoherent quantities (those where the different instances
do not share the same vertical levels) have their vertical coordinate
information stored in the quantity template itself.

\lstinputlisting{vgridtype.f9h}

\subsubsection{\snippet{FGrid\_T}}
The \snippet{FGrid\_T} data type describes the `frequency' coordinate
system for vector quantities that require one. Examples of such quantities
are \snippet{Baselines} and \snippet{Extinctions}.

\lstinputlisting{fgridtype.f9h}

\subsubsection{\snippet{QuantityTemplate\_T}}
The \snippet{QuantityTemplate\_T} type is arguably the most important
derived type in the MLSCFM software.  It describes a collection of
profiles (`instances') for a given quantity (temperature, composition,
tangent pressure, radiance etc.).  See the discussion in
Section~\ref{sec:QuantityTemplates} above.  Typically the geolocation
information is a `shallow copy' (i.e., a pointer to) the source
\snippet{HGrid\_T} and \snippet{VGrid\_T} information or, in the case of
minor frame quantities, a minor frame geolocation repository.

\lstinputlisting{quantitytemplatetype.f9h}

\subsubsection{\snippet{VectorTemplate\_T}}

A vector template is simply a collection of quantity templates, along
with a small amount of summary information.

\lstinputlisting{vectortemplatetype.f9h}

\subsubsection{\snippet{VectorValue\_T}}

The \snippet{VectorValue\_T} type contains the individual values for a
single quantity within a vector.  The \snippet{mask} bitfield contains
element by element flags.  Most of these are for internal use by the
MLSL2 software.

\tbd{Possibly describe the element by element Jacobian computation
  capability within the \snippet{Mask} component if useful.}

\lstinputlisting{vectorvaluetype.f9h}

\subsubsection{\snippet{Vector\_T}}

The \snippet{Vector\_T} describes one instantiation (i.e., collection of
values) for a vector with a given template.

\lstinputlisting{vectortype.f9h}

\subsubsection{\snippet{MatrixElement\_T}}

A \snippet{MatrixElement\_T} type describes one block of a matrix (see
the discussion in Section~\ref{sec:Matrices}).  Matrix blocks can be
`absent', `banded', `column sparse', `full', or, occasionally `unknown',
as described in the comments below.

\lstinputlisting{matrixelementtype.f9h}

\subsubsection{\snippet{RC\_Info}}

The \snippet{RC\_Info} data type describes the rows or columns of a
matrix.  In addition to citing the vector that describes the rows or
columns, the ordering of the subblocks within the matrix is also
defined.

\lstinputlisting{rcinfotype.f9h}

\subsubsection{\snippet{Matrix\_T}}

This type describes a \snippet{Matrix_1} entity -- a collection of
\snippet{Matrix_0} blocks, along with information describing the rows
and columns.

\lstinputlisting{matrixtype.f9h}

\tbd{It is possible more types will need to be detailed here.  The
  \snippet{ForwardModelConfig\_T} is one, but it will probably be
  sufficient to define the routine that will create one.}

\subsection{Procedures and functions}

This subsection describes the main routines it is anticipated a user of
the CFM software will need to invoke.  They are presented in
alphabetical order.  The example given in Appendix~\ref{app:Example}
illustrates their invocation and logical flow.

\subsubsection{\snippet{AddQuantityTemplateToDatabase}}
\lstinputlisting{addqtytemplate-interface.f9h}

\subsubsection{\snippet{CFM\_MLSCleanup}}
\lstinputlisting{cfm-mlscleanup-interface.f9h}

\subsubsection{\snippet{CFM\_MLSSetup}}
\lstinputlisting{cfm-mlssetup-interface.f9h}

\subsubsection{\snippet{CreateFGrid}}
\lstinputlisting{creatfgrid-interface.f9h}

\subsubsection{\snippet{CreateRegularHGrid}}
\lstinputlisting{creathgrid-interface.f9h}

\subsubsection{\snippet{CreateQuantityTemplate}}
\lstinputlisting{creatqtytemplate-interface.f9h}

\subsubsection{\snippet{CreateVector}}
\lstinputlisting{creatvector-interface.f9h}

\subsubsection{\snippet{CreateVectorTemplate}}
\lstinputlisting{creatvectortemplate-interface.f9h}

\subsubsection{\snippet{CreateVGrid}}
\lstinputlisting{creatvgrid-interface.f9h}

\subsubsection{\snippet{Destroy\_Ant\_Patterns\_Database}}
\lstinputlisting{destroy-antenna-interface.f9h}

\subsubsection{\snippet{Destroy\_DACS\_Filter\_Database}}
\lstinputlisting{destroy-DACS-interface.f9h}

\subsubsection{\snippet{Destroy\_Filter\_Shapes_Database}}
\lstinputlisting{destroy-filter-interface.f9h}

\subsubsection{\snippet{Destroy\_Line\_Database}}
\lstinputlisting{destroy-line-interface.f9h}

\subsubsection{\snippet{Destroy\_PFADataBase}}
\lstinputlisting{destroy-pfa-interface.f9h}

\subsubsection{\snippet{Destroy\_Pointing\_Grid\_Database}}
\lstinputlisting{destroy-ptnggrid-interface.f9h}

\subsubsection{\snippet{Destroy\_SpectCat\_Database}}
\lstinputlisting{destroy-speccat-interface.f9h}

\subsubsection{\snippet{DestroyHGridContents}}
\lstinputlisting{destroy-hgrid-content-interface.f9h}

\subsubsection{\snippet{DestroyL2PCDatabase}}
\lstinputlisting{destroy-l2pc-interface.f9h}

\subsubsection{\snippet{DestroyQuantityTemplateDatabase}}
\lstinputlisting{destroy-qtytemplate-db-interface.f9h}

\subsubsection{\snippet{DestroyVectorInfo}}
\lstinputlisting{destroy-vector-info-interface.f9h}

\subsubsection{\snippet{DestroyVectorTemplateInfo}}
\lstinputlisting{destroy-vectemplate-interface.f9h}

\subsubsection{\snippet{DestroyVGridContents}}
\lstinputlisting{destroy-vgrid-content-interface.f9h}

\subsubsection{\snippet{ExplicitFillVectorQuantity}}
\lstinputlisting{explicit-fill-interface.f9h}

\subsubsection{\snippet{ForwardModel}}
\lstinputlisting{forwardmodel-interface.f9h}

\subsubsection{\snippet{GetVectorQtyByTemplateIndex}}
\lstinputlisting{get-vector-qty-interface.f9h}

\subsubsection{\snippet{InitializeMLSFile}}
\lstinputlisting{initmlsfile-interface.f9h}

\subsubsection{\snippet{InitQuantityTemplates}}
\lstinputlisting{init-qtytemplate-interface.f9h}

\subsubsection{\snippet{FillVectorQuantityFromL1B}}
\lstinputlisting{l1b-fill-interface.f9h}

\subsubsection{\snippet{mls\_CloseFile}}
\lstinputlisting{mlsclosefile-interface.f9h}

\subsubsection{\snippet{mls\_openFile}}
\lstinputlisting{mlsopenfile-interface.f9h}

\subsubsection{\snippet{ReadAntennaPatterns}}
\lstinputlisting{read-antenna-interface.f9h}

\subsubsection{\snippet{ReadDACSFilterShapes}}
\lstinputlisting{read-DACS-interface.f9h}

\subsubsection{\snippet{ReadFilterShapes}}
\lstinputlisting{read-filter-interface.f9h}

\subsubsection{\snippet{ReadHDF5L2PC}}
\lstinputlisting{read-l2pc-interface.f9h}

\subsubsection{\snippet{ReadPFAFile}}
\lstinputlisting{read-pfa-interface.f9h}

\subsubsection{\snippet{ReadPointingGrids}}
\lstinputlisting{read-ptnggrid-interface.f9h}

\subsubsection{\snippet{Read\_Spectroscopy}}
\lstinputlisting{read-spectroscopy-interface.f9h}

\subsubsection{\snippet{SpreadFillVectorQuantity}}
\lstinputlisting{spread-fill-interface.f9h}

%%% Local Variables: 
%%% mode: latex
%%% TeX-master: "cfm"
%%% End: 
