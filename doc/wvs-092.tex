\documentclass[11pt]{article}
\usepackage{alltt}
\usepackage[fleqn]{amsmath}\textwidth 6.5in
\oddsidemargin -0.25in
%\evensidemargin -0.5in
\topmargin -0.25in
\textheight 9in

\newcommand{\docname}{\bf wvs-092}
\newcommand{\docdate}{11 February 2010}

\begin{document}

%\tracingcommands=1
\newlength{\hW} % heading box width
\newlength{\pW} % page number field width
\settowidth{\hW}{\docname}
\settowidth{\pW}{Page \pageref{lastpage}\ of \pageref{lastpage}}
\ifdim \pW > \hW \setlength{\hW}{\pW} \fi
\makeatletter
\def\@biblabel#1{#1.}
\newcommand{\ps@twolines}{%
  \renewcommand{\@oddhead}{%
    \docdate\hfill\parbox[t]{\hW}{{\hfill\docname}\newline
                          Page \thepage\ of \pageref{lastpage}}}%
\renewcommand{\@evenhead}{}%
\renewcommand{\@oddfoot}{}%
\renewcommand{\@evenfoot}{}%
}%
\makeatother
\pagestyle{twolines}

\vspace{-10pt}
\begin{tabbing}
\phantom{References: }\= \\
To: \>Nathaniel, Paul, Igor\\
Subject: \>Hessian representation and multiplication\\
From: \>Van Snyder\\
%Reference: \>wvs-063
\end{tabbing}

\parindent 0pt \parskip 6pt
\vspace{-10pt}

Let $I^k$ be the radiance in channel $k$ and $\{f\}$ be the set of state
vector quantities at various positions in the atmosphere.  That is, in
$f^i$, a value of $i$ distinguishes both a position in the atmosphere and
a quantity.

The second-order Taylor-series model, in tensor notation and using the
Einstein summation convention (summation over identical indices of
different factors in the same term, provided one is ``up'' and the other
is ``down''), is
%
\begin{equation}
I^k = \hat I^k + J_i^k ( f^i - \hat f^i ) +
              \frac12 H_{ij}^k  ( f^i - \hat f^i )  ( f^j - \hat f^j )
\end{equation}

where the Jacobian matrix, in tensor notation, is
%
\begin{equation}
J_i^k = \frac{\partial I^k}{\partial f^i}
\end{equation}

and the Hessian matrix, in tensor notation, is
%
\begin{equation}
H_{ij}^k = \frac{\partial^2 I^k}{\partial f^i \partial f^j}
\end{equation}

In the final multiplication
%
\begin{equation}\label{four}
\frac12 H_{ij}^k  ( f^i - \hat f^i )  ( f^j - \hat f^j )
\end{equation}

we want a result indexed by $k$, and we almost surely have at least one
nonzero value, somewhere in $H_{ij}^k$, for every value of $k$, even
though for a given value of $k$ the set of tuples $(i,j)$ for which the
Hessian is nonzero is very small compared to the set of all tuples
$(i,j)$.

The representation of $H_{ij}^k$ is an array of tuples
$(H_{ij}^k,k,i,j)$.  To improve cache locality, which may be irrelevant
if vectors are sort, the tuples should be sorted with $k$ as the major
sort key.  To form the product in Equation (\ref{four}) the tuples are
examined and each $(k,i,j)$ term is added to the $k_\text{th}$ element of
the product.

Suppose $H$ is represented by a variable {\tt H} of type {\tt
Hessian\_0}, having at least an array component named {\tt tuple}, which
is allocated with an extent equal to the number of nonzeroes in $H$.  The
type of {\tt tuple} has components named {\tt H}, {\tt k}, {\tt i} and
{\tt j}.  Then to form the product in an array {\tt P}, given a vector
{\tt F} that represents $f$, the following procedure can be applied:

\begin{alltt}
  P = 0.0
  do n = 1, size(H\%tuple)
    k = H\%tuple(n)\%k
    P(k) = P(k) + 0.5 * H\%tuple(n)\%H * F(H\%tuple(n)\%i) * F(H\%tuple(n)\%j)
  end do
\end{alltt}

\label{lastpage}
\end{document}

% $Id$

% $Log$
%
