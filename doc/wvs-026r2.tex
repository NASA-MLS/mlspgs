\documentclass[11pt]{article}
\usepackage[fleqn]{amsmath}\textwidth 6.25in

\oddsidemargin -0.25in
%\evensidemargin -0.5in
\topmargin -0.5in
\textheight 9.00in

\newcommand{\docname}{\bf wvs-026r2}
\newcommand{\docdate}{17 October 2005}

\begin{document}

%\tracingcommands=1
\newlength{\hW} % heading box width
\newlength{\pW} % page number field width
\settowidth{\hW}{\docname}
\settowidth{\pW}{Page \pageref{lastpage}\ of \pageref{lastpage}}
\ifdim \pW > \hW \setlength{\hW}{\pW} \fi
\makeatletter
\def\@biblabel#1{#1.}
\newcommand{\ps@twolines}{%
  \renewcommand{\@oddhead}{%
    \docdate\hfill\parbox[t]{\hW}{{\docname}\newline
                          Page \thepage\ of \pageref{lastpage}}}%
\renewcommand{\@evenhead}{}%
\renewcommand{\@oddfoot}{}%
\renewcommand{\@evenfoot}{}%
}%
\makeatother
\pagestyle{twolines}

\vspace{-10pt}
\begin{tabbing}
\phantom{References: }\= \\
To: \>Bill, Nathaniel, Mike\\
Subject: \>Simplifying assumption to combine line-by-line and PFA\\
From: \>Van Snyder\\
Reference: \>wvs-024r2, wvs-025\\
\end{tabbing}

\parindent 0pt \parskip 3pt
\vspace{-20pt}

\section{Introduction}

Recall from wvs-024r2 the equations to combine PFA and line-by-line radiances
for a particular channel $c$

\begin{equation}\label{combined}
I_c = \int \text{d}\nu\, \phi_c(\nu)
      \int \text{d} z\, \Delta B(z,\nu) \tau(z,\nu)
 \approx \sum_{n=1}^{N_f} \phi_{nc} \Delta \nu_{nc}
               \sum_{i=1}^{N_p} \Delta B_{ic} \tau^s_{in} \tau^w_{ic}
\end{equation}

and to combine PFA and line-by-line derivatives for that channel

\begin{equation}\label{final}
\frac{\partial I_c}{\partial x_k}
\approx \sum_{n=1}^{N_f} \phi_{nc} \Delta \nu_{nc} \sum_{i=1}^{N_p}
 \left(
  \frac{\partial \Delta B_{ic}}{\partial x_k} -
  \Delta B_{ic}
   \sum_{j=1}^i \left( \frac{\partial \delta^s_{jn}}{\partial x_k} +
                      \frac{\partial \delta^w_{jc}}{\partial x_k}
                \right)
  \right) \tau^s_{in} \tau^w_{ic}\,.
\end{equation}

This isn't precisely representative because the outer sums are actually
applications of the Newton 3/8 quadrature rule, and the inner sums are
evaluated on a different frequency grid from the outer sums (the pointing
frequency grid), and then interpolated to the grid used for the outer sums.

The forward model actually computes ``raw'' radiances and derivatives on the
pointing frequency grid

\begin{equation}\begin{split}\label{raw}
R_n =\,& \sum_{i=1}^{N_p} \Delta B_{in} \tau^s_{in} \tau^w_{ic} \\
\frac{\partial R_n}{\partial x_k} =\,&
\sum_{i=1}^{N_p}
 \left(
  \frac{\partial \Delta B_{in}}{\partial x_k} -
  \Delta B_{ic}
   \sum_{j=1}^i \left( \frac{\partial \delta^s_{jn}}{\partial x_k} +
                      \frac{\partial \delta^w_{jc}}{\partial x_k}
                \right)
  \right) \tau^s_{in} \tau^w_{ic}\,.
\end{split}\end{equation}

It then interpolates these results to the filter function grid and integrates
the product of these results and the filter function using the Newton 3/8 rule:

\begin{equation}\begin{split}
I_c =\,& \sum_{n=1}^{N_f} w_n \phi_{nc} \Delta \nu_{nc} \hat R_n\\
\frac{\partial I_c}{\partial x_k} =\,&
 \sum_{n=1}^{N_f} w_n \phi_{nc} \Delta \nu_{nc}
  \frac{\partial \hat R_n}{\partial x_k}
\end{split}\end{equation}

where $w_n$ is a weight for the Newton 3/8 rule, the hat indicates quantities
interpolated from the pointing frequency grid to the filter function grid, and
$n$ now ranges over the filter grid instead of the pointing frequency grid.

The term $\frac{\partial \delta^s_{jn}}{\partial x_k}$ is a three-dimensional
array that could, in the worst case, require 6 GB of storage per species.  To
reduce this requirement, it is proposed to revise the scheme for combining PFA
and line-by-line radiances and derivatives.

\section{Simplifying assumption for combining line-by-line and PFA
derivatives}

Expand the sum for $\frac{\partial R_n}{\partial x_k}$ in Equation (\ref{raw}):

\begin{equation}\label{expand}
\begin{split}
\frac{\partial R_n}{\partial x_k} =\,& \sum_{i=1}^{N_p}
 \left(
  \frac{\partial \Delta B_{in}}{\partial x_k} -
  \Delta B_{in}
   \sum_{j=1}^i \left( \frac{\partial \delta^s_{jn}}{\partial x_k} +
                      \frac{\partial \delta^w_{jc}}{\partial x_k}
                \right)
  \right) \tau^s_{in} \tau^w_{ic} = \\
&\sum_{i=1}^{N_p}
 \left(
  \frac{\partial \Delta B_{in}}{\partial x_k} -
  \Delta B_{in}
   \sum_{j=1}^i\frac{\partial \delta^s_{jn}}{\partial x_k}
  \right) \tau^s_{in} \tau^w_{ic} -
\sum_{i=1}^{N_p} \Delta B_{ic} \sum_{j=1}^i
 \frac{\partial \delta^w_{jc}}{\partial x_k} \tau^s_{in} \tau^w_{ic}\,.
\end{split}
\end{equation}

If the PFA calculations are done first the $\frac{\partial
\delta^s_{jn}}{\partial x_k}$ quantity need not be represented by an array that
has a frequency dimension:  The first term in the right-hand side can be
evaluated separately for each frequency, as is done in the non-PFA case.  This
requires to store the much smaller array that represents the $\frac{\partial
\delta^w_{jc}}{\partial x_k}$ quantity for use in the second term in the
right-hand side.

By using some representative average value $\overline{\tau^w_c}$ instead
of $\tau^w_{ic}$ in the first term, we have

\begin{equation}\begin{split}\label{simple}
\frac{\partial I_c}{\partial x_k}
\approx\,& \overline{\tau^w_c} \sum_{n=1}^{N_f} \phi_{nc} \Delta \nu_{nc} \sum_{i=1}^{N_p}
 \left(
  \frac{\partial \Delta B_{in}}{\partial x_k} -
  \Delta B_{in}
   \sum_{j=1}^i \frac{\partial \delta^s_{jn}}{\partial x_k}
  \right) \tau^s_{in}\\
  \,& -
 \sum_{n=1}^{N_f} \phi_{nc} \Delta \nu_{nc} \sum_{i=1}^{N_p}
  \Delta B_{ic} \sum_{j=1}^i\frac{\partial \delta^w_{jc}}{\partial x_k}
   \tau^s_{in} \tau^w_{ic}\,.
\end{split}\end{equation}

The first term is simply the line-by-line derivative multiplied by
$\overline{\tau^w_c}$.  As is done in the non-PFA case, the $\frac{\partial
\delta^s_{jn}}{\partial x_k}$ quantity can be computed and discarded for each
frequency in the pointing grid; it need not be saved for use during frequency
averaging.  This still requires that the PFA calculation be done first, and
that $\frac{\partial \delta^w_{jc}}{\partial x_k}$ be saved for use in the
second term.

A further simplification is to approximate $\overline{\tau^w_c} \approx 1$. 
This allows PFA calculations to be done after line-by-line calculations, and
therefore allows $\frac{\partial \delta^w_{jc}}{\partial x_k}$ not to have a
channel dimension.  As is the case for $\frac{\partial \delta^s_{jn}}{\partial
x_k}$ it can be calculated separately for each channel and then discarded.

This introduces a minor inconsistency between the radiance and its derivatives. 
Experiments will show whether this confuses the nonlinear solver.

As explained in wvs-025, it is possible to eliminate the frequency dimension
altogether from array quantities other then the pointing frequency grids
themselves, while preserving the fidelity of the approximations.

The frequency-averaged radiance due to strong lines, in each channel at each
point along the path, is

\begin{equation}\label{rad}
\overline{I^s_{ic}} =
 \sum_{n=1}^{N_f} \phi_{nc} \Delta \nu_{nc} \Delta B_{in} \tau^s_{in}
 \approx
 \Delta B_{ic} \sum_{n=1}^{N_f} \phi_{nc} \Delta \nu_{nc}
  \tau^s_{in}\,.
\end{equation}

Since the frequency variation within a channel is very small compared to the
channel center frequency,  $\Delta B_{ic}$ is very nearly $\Delta B_{in}$, and
this is therefore a good approximation.

Exchanging the order of summation in the second term in Equation (\ref{simple})
and substituting $\overline{I^s_{ic}}$,

\begin{equation}
\frac{\partial I_c}{\partial x_k}
\approx \sum_{n=1}^{N_f} \phi_{nc} \Delta \nu_{nc} \sum_{i=1}^{N_p}
 \left(
  \frac{\partial \Delta B_{in}}{\partial x_k} -
  \Delta B_{in}
   \sum_{j=1}^i \frac{\partial \delta^s_{jn}}{\partial x_k}
  \right) \tau^s_{in}
 -
 \sum_{i=1}^{N_p} \overline{I^s_{ic}} \tau^w_{ic}
  \sum_{j=1}^i\frac{\partial \delta^w_{jc}}{\partial x_k}\,.
\end{equation}

The first term is now simply the line-by-line derivative, and the second term
is a PFA correction.  The inner sum within the second term can be evaluated
within the frequency loop, thereby not needing to provide a $c$ dimension for
$\frac{\partial \delta^w_{jc}}{\partial x_k}$.

The frequency-averaged radiance, including the contribution from PFA (weak)
lines can be gotten by exchanging the order of summation in Equation
(\ref{combined}) and substituting $\overline{I^s_{ic}}$ from Equation
(\ref{rad}):

\begin{equation}
I_c \approx \sum_{i=1}^{N_p} \overline{I^s_{ic}} \tau^w_{ic}\,.
\end{equation}

When PFA is not used, channel radiances can be gotten by assuming $\tau^w_{ic} =
1$, that is, simply by summing $\overline{I_{ic}}$ for each channel.

\label{lastpage}
\end{document}
% $Id$
