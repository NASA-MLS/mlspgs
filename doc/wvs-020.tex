\documentclass[11pt]{article}
\usepackage[fleqn]{amsmath}\textwidth 6.25in
\oddsidemargin -0.25in
%\evensidemargin -0.5in
\topmargin -0.5in
\textheight 9.00in

\begin{document}

%\tracingcommands=1
\newlength{\hW} % heading box width
%\settowidth{\hW}{\bf wvs-020}
\settowidth{\hW}{Page \pageref{lastpage}\ of \pageref{lastpage}}
\makeatletter
\def\@biblabel#1{#1.}
\newcommand{\ps@twolines}{%
  \renewcommand{\@oddhead}{%
    10 March 2005\hfill\parbox[t]{\hW}{{\bf wvs-020}\newline
                          Page \thepage\ of \pageref{lastpage}}}%
\renewcommand{\@evenhead}{}%
\renewcommand{\@oddfoot}{}%
\renewcommand{\@evenfoot}{}%
}%
\makeatother
\pagestyle{twolines}

\vspace{-10pt}
\begin{tabbing}
\phantom{References: }\= \\
To: \>Bill, Dong, Michael, Nathaniel\\
Subject: \>Reading and writing spectroscopy data in MLSL2\\
From: \>Van Snyder\\
\end{tabbing}

\parindent 0pt \parskip 3pt
\vspace{-20pt}

The way we currently read spectroscopy data into MLSL2 is to put it in the
L2CF.  This occupies several tens of thousands of lines of text, several
hundreds of thousands of characters in the character table, several tens of
thousands of entries in the string table, and requires a relatively long time
to read and parse.

A command has been added to the L2CF to write the spectroscopy data onto
an HDF5 file, \emph{viz.}:

\hspace*{0.25in}{\tt writeSpectroscopy, file=\emph{file-name}}

Once this has been done, the entire spectrosopy section can be replaced by
a newly-added command to read that file, \emph{viz.}:

\hspace*{0.25in}{\tt readSpectroscopy, file=\emph{file-name}}

I've already written a spectroscopy file, and put the result into

\hspace*{0.25in}{\tt \~{}vsnyder/mlspgs/l2/big\_spectroscopy.h5}

It probably ought to be done ``officially,'' and the result put into a
standard repository, such as {\tt /testing/...}.

It would also save some time and space to use this method in the next version
we send to Raytheon.

\label{lastpage}
\end{document}
% $Id$
