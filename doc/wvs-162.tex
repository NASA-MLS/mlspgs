\documentclass[11pt]{article}
\usepackage{alltt}
\usepackage[fleqn]{amsmath}
\usepackage{longtable}
\usepackage[strings]{underscore}

\textwidth 6.5in
\oddsidemargin -0.25in
%\evensidemargin -0.5in
\topmargin -0.5in
\textheight 9in

\newcommand{\docname}{wvs-162r1}
\newcommand{\docdate}{13 July 2020}

\ifx\pdfoutput\undefined
  \pdfoutput=0
\fi
\ifnum\pdfoutput>0
  \usepackage[pdftex,plainpages,hyperindex=true,pdfpagelabels]{hyperref}
  \hypersetup{%
    hypertexnames=false,%
    colorlinks=true,%
    linktocpage=true,%
  }
  % Specify the driver for the color package
  \ExecuteOptions{pdftex}
\else
  \usepackage[hypertex,plainpages,hyperindex=true]{hyperref}
  \hypersetup{%
    hypertexnames=false%
  }
  % Specify the driver for the color package
  \ExecuteOptions{dvips}
  %\ExecuteOptions{xdvi}
\fi

\hyperbaseurl{}
\newcommand\hr[1]{\href{#1.dvi}{dvi}, \href{#1.pdf}{pdf}}
\newcommand\h[1]{#1 (\hr{#1})}

\renewcommand{\d}{\text{d}} % for derivatives

\begin{document}

%\tracingcommands=1
\newlength{\hW} % heading box width
\newlength{\pW} % page number field width
\settowidth{\hW}{\bf\docname}
\settowidth{\pW}{Page \pageref{lastpage}\ of \pageref{lastpage}}
\ifdim \pW > \hW \setlength{\hW}{\pW} \fi
\makeatletter
\def\@biblabel#1{#1.}
\newcommand{\ps@twolines}{%
  \renewcommand{\@oddhead}{%
    \docdate\hfill\parbox[t]{\hW}{{\hfill\bf\docname}\newline
                          Page \thepage\ of \pageref{lastpage}}}%
\renewcommand{\@evenhead}{}%
\renewcommand{\@oddfoot}{}%
\renewcommand{\@evenfoot}{}%
}%
\makeatother
\pagestyle{twolines}

\vspace{-10pt}
\begin{tabbing}
\phantom{References: }\= \\
To: \>Paul, Bill, William, Nathaniel\\
Subject: \>Forward Model Configuration\\
From: \>Van Snyder\\
References: \> \h{wvs-017}, \h{wvs-027}, \h{wvs-049}, \h{wvs-065}, \\
  \> \h{wvs-067}, \h{wvs-070}, \h{wvs-107}
\end{tabbing}

\parindent 0pt \parskip 6pt
\vspace{-20pt}

\section*{Summary}

The actions of the forward models are controlled by several factors.

\begin{itemize}

\item Fields of the {\tt ForwardModelGlobal} specification in the {\tt
  GlobalSettings} section of the {\tt l2cf} specify file names:

  \begin{itemize}
  
  \item {\tt AntennaPatterns}
  
  \item {\tt DACSFilterShapes}
  
  \item {\tt FilterShapes}
  
  \item {\tt l2pc}
  
  \item {\tt MieTables}
  
  \item {\tt PFAFiles}
  
  \item {\tt PointingGrid}
  
  \item {\tt Polygon}

  \end{itemize}

\item The {\tt Spectroscopy} section of the {\tt l2cf}

\item The {\tt ForwardModel} specification in the {\tt Construct} section
  of the {\tt l2cf}

\item Quantities in the vectors specified in the {\tt State} and {\tt
  FwdModelExtra} fields in the {\tt Retrieve} or {\tt SIDS} specification
  in the {\tt Retrieve} section of the {\tt l2cf}

  \begin{itemize}

  \item {\tt IsotopeRatio} for a particular molecule.

  \end{itemize}

\item The Jacobian specified in the {\tt Retrieve} or {\tt SIDS}
  specification in the {\tt Retrieve} section of the {\tt l2cf}

\end{itemize}

\section*{ForwardModelGlobal specification}

\subsection*{Syntax}

The {\tt ForwardModelGlobal} specification provides the names of files. It
has eight fields, each of which must have a character string value. They
are all optional. There are no default values.

There is no provision for more than one antenna patterns file, filter
shapes file, DACS filter shape file, or pointing grids file. If several
Mie tables files or several PFA files are needed, their names are specified
as an array in the corresponding field of one {\tt ForwardModelGlobal}
specification. If several {\tt l2pc} files are needed, their names are
specified in separate {\tt ForwardModelGlobal} specifications (or maybe
the subroutine that handles the {\tt l2pc} field accepts an array of
names; Paul might know).

\subsection*{AntennaPatterns file}

The antenna patterns file is read by the {\tt Read_Antenna_Patterns_File}
subroutine in the module {\tt Antenna\-Pat\-terns_m}.

The antenna patterns file contains Fourier coefficients to use when
convolving the forward model results with the angular response of the
antenna.

Each section of the file begins with a signal specification that must
match a signal specified in the {\tt Signals} section of the {\tt l2cf}. A
signal specification might specify several signals, so the number of
signals specified by each signal section is recorded.

The first line after the signal specifies the number of coefficients $n$,
and {\tt lambda}, which is used in the module {\tt FOV_Convolve_m} to
compute the angular step between coefficients, \emph{viz.}

\begin{equation*}
\text{\tt ang_step} = \frac1{2^{11} \lambda}
\end{equation*}

The units of {\tt lambda} are inverse radians. The value in at least one
file is 5.0 where the number of coefficients is 2048, and 5.72803 where
the number of coefficients is 335.

The code computes $\ell = \lceil\log_2 n\rceil$, and computes {\tt
power2}~= $2^{\ell}$. Then there is a line with the comment {\tt Bill
debug} that sets {\tt power2}~= $2^{11}$. The value {\tt 2*power2} is then
used to allocate arrays in the {\tt antennaPatterns} structure 
(coefficients have both real and imaginary parts). Antenna patterns are
ultimately used in the {\tt FOV_Convolve_m} module, where there are arrays
having $2\times2^{11}$ elements. This will need to be changed for a
different instrument.

Following the second line in each section are $n$ pairs of coefficients,
first the real part, then the imaginary part, stored into an array
component named {\tt aaap}. Values between $n+1$ and {\tt 2*power2} are
set to zero. Array components named {\tt d1aap} and {\tt d2aap} are
allocated and filled with zeroes. These are used to compute derivatives of
{\tt aaap} in the module {\tt FOV_Convolve_m}.

\subsection*{DACSFilterShapes file}

The DACS filter shapes file is read by the {\tt
Read_DACS_Filter_Shapes_File} subroutine in the {\tt FilterShapes_m}
module.

Blank lines are allowed.

The section for each filter consists of

\begin{enumerate}

\item Zero or more blank lines or comment lines beginning with ``!''

\item A signal specification that must match a signal specified in the
  {\tt Signals} section of the {\tt l2cf}.

\item Fortran {\tt NAMELIST} input that specifies the characteristics of
  the filter. It begins with {\tt \&filter} (the namelist name) and ends
  with ``/''. Blanks, including blank lines, are permitted before {\tt
  \&filter}.  The namelist input can specify the values of five variables,
  in any order, by giving the variable name (case insensitive), followed
  by ``='', followed by the value. Blanks are permitted around ``=''.
  Name/value pairs are separated by blanks, commas, or ends of lines.

  \begin{itemize}
  
  \item {\tt lhs =} the left-hand side of the filter range, in MHz
  
  \item {\tt rhs =} the right-hand side of the filter range, in MHz
  
  \item {\tt logFilt =} the base-2 logarithm of the size of the pre-filter
    shape array
  
  \item {\tt logNorm =} the base-2 logarithm of the size of the
    normalization array

  \item {\tt logApod =} the base-2 logarithm of the LO apodization array,
    which must be the same as either {\tt logFilt} or {\tt logNorm}

  \end{itemize}

  {\tt logFilt}, {\tt logNorm}, and {\tt logApod} are integer variables.
  {\tt lhs} and {\tt rhs} are real variables.

  The values of {\tt lhs} and {\tt rhs} are used to construct the filter's
  frequency grid. If {\tt lhs} $>$ {\tt rhs}, the array will have
  decreasing frequencies in consecutive elements.

  The default value for {\tt logNorm} is 7. There are no default values
  for any of the other variables. There is no check whether values have
  been specified. Be careful to specify all of them.

  Blank lines, and comments beginning with ``!'' and extending to the end
  of the line, are permitted within namelist input.

\item The filter shape array. Its size is 2**{\tt logFilt}+1.

\item The LO apodization array. Its size is 2**{\tt logApod}+1.

\item The channel normalization array. Its size is 2**{\tt logNorm}+1.

\end{enumerate}

The three arrays after the {\tt \&filter} namelist input are all read
using list-directed (also called free-format) input. Any number of values
can appear on each line. Blank lines are permitted. Values are separated
by blanks, commas, or ends of lines. If consecutive commas appear (which
can be separated by blanks or ends of lines), values are not input for the
intervening array elements. Comments are not permitted. If ``/'' appears
it terminates input even if insufficient values to fill the array are
supplied. Default values are not specified for the arrays, so un-filled
elements become undefined.

\subsection*{FilterShapes file}

The filter shapes file, for filters that are not DACS filters, is read by
the {\tt Read_Filter_Shapes_File} subroutine in the {\tt FilterShapes_m}
module.

Blank lines are allowed.

The section for each filter consists of

\begin{enumerate}

\item Zero or more blank lines or comment lines beginning with ``!''

\item A signal specification that must match a signal specified in the
  {\tt Signals} section of the {\tt l2cf}.

\item Fortran {\tt NAMELIST} input that specifies the characteristics of
  the filter. It begins with {\tt \&filter} (the namelist name) and ends
  with ``/''. Blanks, including blank lines, are permitted before {\tt
  \&filter}. The namelist input can specify the values of three
  variables, in any order, by giving the variable name (case insensitive),
  followed by ``='', followed by the value. Blanks are permitted around
  ``=''. Name/value pairs are separated by blanks, commas, or ends of
  lines.

  \begin{itemize}
  
  \item {\tt lhs =}  the left-hand side of the filter range, in MHz
  
  \item {\tt rhs =} the right-hand side of the filter range, in MHz
  
  \item {\tt number_in_filter =} the number of elements in the filter
    shape array

  \end{itemize}

{\tt number_in_filter} is an integer variable. {\tt lhs} and {\tt rhs} are
real variables.

The values of {\tt lhs} and {\tt rhs} are used to construct the filter's
frequency grid. If {\tt lhs} $>$ {\tt rhs}, the array will have decreasing
frequencies in consecutive elements.

There is no default value for  any variable. There is no check whether
values have been specified. Be careful to specify all of them.

Blank lines, and comments beginning with ``!'' and extending to the end of
the line, are permitted within namelist input.

\item The filter shape array, containing {\tt number_in_filter} elements.

  The array is read using list-directed (also called free-format) input.
  Any number of values can appear on each line. Blank lines are permitted.
  Values are separated by blanks, commas, or ends of lines. If consecutive
  commas appear (which can be separated by blanks or ends of lines),
  values are not input for the intervening array elements. Comments are
  not permitted. If ``/'' appears it terminates input even if insufficient
  values to fill the array are supplied. Default values are not specified
  for the arrays, so un-filled elements become undefined.

  Filter shape data units are normalized response per MHz. That is
  $\int f(\nu)\, \text{d}\nu = 1$, for $\nu$ in MHz.

\end{enumerate}

\subsection*{l2pc file}

I do not know anything about {\tt l2pc} files. Nathaniel will need to
write this section.

\subsection*{MieTables file}

See \h{wvs-065}, \h{wvs-067}, and \h{wvs-070}.

Mie\footnote{Gustav Mie developed equations for electromagnetic scattering
from homogeneous spheres.} tables are used to calculate the TScat tables
used for the TScat approximation to model cloud scattering.

If the file name ends in ``.hdf'' (case insensitive), the Mie tables file
is an HDF file. Otherwise, it is a Fortran unformatted files. The file is
created by the {\tt mlspgs/Mie/Mie_Tables} program.

The first record on the Fortran unformatted file contains eight scalars:

\begin{enumerate}

\item {\tt n_f} Number of frequencies.

\item {\tt n_IWC} Number of ice water content values.

\item {\tt n_T} Number of temeratures.

\item {\tt n_theta} Number of scattering angles.

\item {\tt r_min} Minimum radius of scattering particles ($\mu$m).

\item {\tt r_max} Maximum radius of scattering particles ($\mu$m).

\item {\tt n_cut} Maximum order for Bessel functions.

\item {\tt derivs} A logical variable indicating whether derivatives are
  present.

\end{enumerate}

{\tt n_f}. {\tt n_IWC}, {\tt n_T}, and {\tt n_theta} are used to allocate
arrays. The other scalars are not used by {\tt mlsl2}.

Independent variables in the Fortran unformatted file, in the following
order:

\begin{enumerate}

\item {\tt IWC_s(n_IWC)} Ice water content.

\item {\tt T_s(n_T)} Temperatures.

\item {\tt Theta_s(n_theta)} Scattering angles.

\item {\tt F_s(n_f)} frequencies in GHz.

\end{enumerate}

Dependent variables in the Fortran unformatted file, in the following
order:

\begin{enumerate}\setcounter{enumi}{4}

\item {\tt Beta_c_e(n_t,n_iwc,n_f)} extinction coefficient
  $\beta_{c\_e} = \pi \int_0^\infty n(r)\, r^2\, \xi_e(r) \, \text{d} r$.

\item {\tt Beta_c_s(n_t,n_iwc,n_f)} scattering coefficient
  $\beta_{c\_s} = \pi \int_0^\infty n(r)\, r^2\, \xi_s(r) \, \text{d} r$.

\item {\tt p(n_t,n_iwc,n_theta,n_f)} integrated phase functions\\
   $P(\theta;T,\text{\tt IWC},f) =
   \frac{\lambda^2}{2 \pi \beta_{c\_s}} \int_0^\infty n(r)\,
   p_0(\theta,r;T,\text{\tt IWC},f)
   \, \text{d} r$n.

\item {\tt dBeta_dIWC_e((n_t,n_iwc,n_f)} $\frac{\partial
  \beta_{c\_e}}{\partial \text{\tt IWC}}$.

\item {\tt dBeta_dIWC_s((n_t,n_iwc,n_f)} $\frac{\partial
  \beta_{c\_s}}{\partial \text{\tt IWC}}$.

\item {\tt dBeta_dT_c_e((n_t,n_iwc,n_f)} $\frac{\partial
  \beta_{c\_e}}{\partial T}$.

\item {\tt dBeta_dT_c_s((n_t,n_iwc,n_f)} $\frac{\partial
  \beta_{c\_s}}{\partial T}$.

\item {\tt dP_dIWC((n_t,n_iwc,n_theta,n_f)} $\frac{\partial P}{\partial
  \text{\tt IWC}}$.

\item {\tt dP_dT((n_t,n_iwc,n_theta,n_f)} $\frac{\partial P}{\partial T}$.

\end{enumerate}

In the Fortran unformatted file, each of the dependent value data sets, in
addition to the value, includes the error estimate from the quadrature,
the number of function values for the quadrature, and the maximum order of
the Patterson formula.

The HDF file has the above data sets. The error estimate, number of
function values, and maximum order, are not read by {\tt mlsl2}.

In addition, the HDF file includes data sets

\begin{itemize}

\item {\tt Beta_c_a(n_t,n_iwc,n_f)} absorption coefficient $\beta_{c\_a} =
  \beta_{c\_e} - \beta_{c\_e}$
  

\item {\tt dBeta_dIWC_c_a(n_t,n_iwc,n_f)} $\frac{\partial
  \beta_{c\_a}}{\partial\text{\tt IWC}}$

\item {\tt dBeta_dT_c_a(n_t,n_iwc,n_f)} $\frac{\partial
\beta_{c\_a}}{\partial T}$

\end{itemize}

\subsection*{PFAFiles}

PFA files contain data for the PFA approximation for weak lines, or weak
tails of lines outside a channel -- see \h{wvs-017}.

PFA files are HDF5 files that are constructed by {\tt mlsl2}, as described
in \h{wvs-049}.

Each file contains a data set named {\tt Molecules} that lists all the
molecules for which data are available, and a data set named {\tt Signals}
that lists all the signals for which data are available.

For every combination of molecules and signals for which data are
available, there is a group named {\tt molecule\%signal}, where {\tt
molecule} is a molecule from the {\tt Molecules} data set, and {\tt
signal} is a signal from the {\tt Signals} data set.

Within each such group are the following attributes:

\begin{description}

\item[{\tt filterFile}] The name of the filter file used to create the
  data set.

\item[{\tt spectroscopyFile}] The name of the spectroscopy file used to
  create the data set.

\item[{\tt signal}] The signal for which the data set was prepared.

\item[{\tt sideband}] The sideband for which the data set was prepared.

\item[{\tt vel-rel}] The line-of-site velocity used when the data set was
  prepared.

\item[{\tt whichLines}] List of lines used to prepare the data set.

\item[{\tt nTemps}] The number of temperatures used to prepare the data
  set.

\item[{\tt tStart}] Starting temperature.

\item[{\tt tStep}] Step between temperatures.

\item[{\tt nPress}] The number of pressures $(\zeta)$ used to prepare the
  data set.

\item[{\tt vStart}] Starting pressure $(\zeta)$.

\item[{\tt vStep}] Step between pressures $(\zeta)$.

\end{description}

There are four data sets, all dimensioned {\tt nTemps} $\times$ {\tt
nPress}:

\begin{description}

\item[{\tt absorption}] $\ln \beta$.

\item[{\tt dAbsDwc}] $\frac{\partial \ln \beta}{\partial w}$ where $w$ is
  line width.

\item[{\tt dAbsDnc}] $\frac{\partial \ln \beta}{\partial n}$ where $n$ is
  line width temperature dependence.

\item[{\tt dAbsDnu}] $\frac{\partial \ln \beta}{\partial \nu}$ where $\nu$ is
  line center frequency.

\end{description}

\subsection*{PointingGrid file}

The pointing grid file specifies the frequencies to be used to integrate
the radiative-transfer equation, for each specified tangent pressure.

The file consists of an arbitrary number of sets of data.

Each data set in the file consists of:

\begin{itemize}

\item An arbitrary number of signals, one per line. The end of the list of
  signals is detected by the first non-blank character on the line being a
  number.

\item Line center frequency in MHz, apparently doppler shifted.

\item Tangent pressure $(\zeta)$, and the number of frequencies to use for
  the specified pressure.

\item Frequencies, relative to the line center frequency, at which to
  integrate the radiative-transfer equation for the specified tangent
  height.

  Frequencies are read using list-directed (also called free-format) input.
  Any number of values can appear on each line. Blank lines are permitted.
  Values are separated by blanks, commas, or ends of lines. If consecutive
  commas appear (which can be separated by blanks or ends of lines),
  values are not input for the intervening array elements. Comments are
  not permitted. If ``/'' appears it terminates input even if insufficient
  values to fill the array are supplied. Default values are not specified
  for the array, so un-filled elements become undefined.

\end{itemize}

\subsection*{Polygon file}

The polygon file specifies the polygon in which to inscribe a QTM.

A polygon file consists of a (lon,lat) pair that specifies a point defined
to be inside the polygon.  This is necessary because the concept ``inside
a polygon'' is ambiguous on the surface of a sphere.

This is followed by however many (lon,lat) pairs as necessary to specify
the boundary.

The units of latitudes and longitudes are degrees.

This is agnostic concerning whether latitudes are geocentric or geodetic,
so long as they are consistent with latitudes and longitudes used for
finding facets in the QTM and interpolating within them.

Everything on a line after \# or ! is ignored. The "inside" point needs to
be on a line of its own. Thereafter, for points on the boundary, as many
(lon,lat) pairs (not enclosed in parentheses) as desired may be placed on
a line, but a (lon,lat) pair cannot be split between lines. Values on each
line are separated by blanks or commas. Consecutive commas result in an
undefined value, so don't do that.

\section*{Spectroscopy Section}

Spectroscopy data are provided in the spectroscopy section in two ways.

\subsection*{{\tt l2cf} statements}

Spectral characteristics for each molecule are specified by any number of
{\tt line} statements, each specifying the characteristics of one spectral
line, followed by a {\tt spectra} statement that specifies the molecule
and the spectral lines for the molecule.

The fields of the {\tt line} statement are

\begin{longtable}{llcc}
Field          & Type  & Array & Required \\
\hline
delta          & numeric &   & * \\
el             & numeric &   & * \\
emlsSignals    & string  & * &   \\
emlsSignalsPol & string  & * &   \\
gamma          & numeric &   & * \\
n1             & numeric &   & * \\
n2             & numeric &   & * \\
ns             & numeric &   & * \\
n              & numeric &   & * \\
ps             & numeric &   & * \\
qn             & numeric & * &   \\
signals        & string  & * &   \\
signalsPol     & string  & * &   \\
str            & numeric & * & * \\
umlsSignals    & string  & * &   \\
v0             & numeric &   & * \\
w              & numeric &   & * \\
\end{longtable}

The fields of the {\tt spectra} statement are

\begin{longtable}{llcc}
Field               & Type                     & Array & Required \\
\hline
continuum           & numeric                    &   &   \\
defaultIsotopeRatio & numeric                    &   &   \\
lines               & {\tt line} statement label & * &   \\
mass                & numeric                    &   &   \\
molecule            & molecule                   &   & * \\
qlog                & numeric                    & * &   \\
\end{longtable}

The {\tt line} and {\tt spectra} statements are prepared from a \LaTeX\
file by one of Bill's programs.

\subsection*{HDF5 file}

A spectroscopy HDF5 file is produced by {\tt mlsl2} using a {\tt
writeSpectroscopy} statement. The only field of the statement is a {\tt
file} field, which specifies the file name.

A spectroscopy HDF5 file is read by {\tt mlsl2} using a {\tt
readSpectroscopy} statement. The only field of the statement is a {\tt
file} field, which specifies the file name.

The spectroscopy section can contain both methods of specifying spectral
characteristics.  It can contain any number of {\tt readSpectroscopy}
statements, and any number of {\tt writeSpectroscopy} statements. It is
possible to add spectroscopy information to a spectroscopy HDF5 file by
reading the file using a {\tt readSpectroscopy} statement, and {\tt line}
and {\tt spectra} statemments to define additional spectroscopy, and a
{\tt writeSpectroscopy} statement. Spectral characteristics can only be
added; the characteristics for a molecule cannot be replaced by this
method.

\section*{{\tt ForwardModel} specification}

The fields of the {\tt ForwardModel} statement are used to configure
forward models.

\newcommand{\N}{$^\dagger$}
\begin{longtable}{llllcl}
Field                     & Type           & Units     & Default & Array & Model$^*$  \\
\hline
\endhead
\hline
\N Statement label \\
\multicolumn{6}{l}{%
$^*$Models: Baseline, Cloud, Full, Linear, Hybrid, Polar linear, switching Mirror, Scan, TScat}
\endfoot
\hline
\N Statement label \\
\multicolumn{6}{l}{%
$^*$Models: Baseline, Cloud, Full, Linear, Hybrid, Polar linear, switching Mirror, Scan, TScat}
\endlastfoot
allLinesForRadiometer     &  boolean       &               & false   &       & F      \\
allLinesInCatalog         &  boolean       &               & false   &       & F      \\
atmos_der                 &  boolean       &               & false   &       & F      \\
atmos_second_der          &  boolean       &               & false   &       & F      \\
binSelectors              &  binSelector\N &               &         &   *   & L P    \\
cloud_der                 &  cloud_der     &               & none    &       & C      \\
default_spectroscopy      &  boolean       &               & false   &       & C      \\
differentialScan          &  boolean       &               & false   &       & S      \\
do_baseline               &  boolean       &               & false   &       & B      \\
do_conv                   &  boolean       &               & false   &       & F      \\
do_freq_avg               &  boolean       &               & false   &       & F      \\
do_1d                     &  boolean       &               & false   &       & S      \\
forceSidebandFraction     &  boolean       &               & false   &       & F L    \\
frqTol                    &  numeric       & frequency     & 600 MHz &       & T      \\
i_saturation              &  i_saturation  &               & clear   &       & C      \\
ignoreHessian             &  boolean       &               & false   &       & L T    \\
incl_cld                  &  boolean       &               & false   &       & F      \\
integrationGrid           &  vGrid\N       &               &         &       & F      \\
linearSideband            &  numeric       & dimensionless & 0       &       & H      \\
lineCenter                &  molecule      &               &         &       & F      \\
lineWidth                 &  molecule      &               &         &       & F      \\
lineWidth_TDep            &  molecule      &               &         &       & F      \\
lockBins                  &  boolean       &               & false   &       & L      \\
lsbLBLMolecules           &  molecule      &               &         &   *   & F      \\
lsbPFAMolecules           &  molecule      &               &         &   *   & F      \\
MIFTangent                &  MIFTangent    &               & ptan    &       & F      \\
module                    &  module\N      &               &         &       & F      \\
moleculeDerivatives       &  molecule      &               &         &   *   & F      \\
moleculeSecondDerivatives &  molecule      &               &         &   *   & F      \\
molecules                 &  molecule      &               &         &   *   & F      \\
nabterms                  &  numeric       & dimensionless & 50      &       & C      \\
nazimuthangles            &  numeric       & dimensionless & 8       &       & C      \\
ncloudspecies             &  numeric       & dimensionless & 2       &       & C      \\
nmodelsurfs               &  numeric       & dimensionless & 640     &       & C      \\
no_dup_mol                &  boolean       &               & false   &       & config \\
noMagneticField           &  boolean       &               & false   &       & none   \\
nscatteringangles         &  numeric       & dimensionless & 16      &       & C T    \\
ncloudspecies             &  numeric       & dimensionless & 2       &       & C      \\
nsizebins                 &  numeric       & dimensionless & 40      &       & T      \\
pathNorm                  &  boolean       &               & false   &       & F      \\
phiWindow                 &  see below     & see below     & [2,2]   &       & F      \\
polarized                 &  boolean       &               & false   &       & F      \\
referenceMIF              &  numeric       & dimensionless & 1       &       & none   \\
refract                   &  boolean       &               & true    &       & F      \\
scanAverage               &  boolean       &               & false   &       & F      \\
signals                   &  string        &               &         &       & all    \\
skipOverlaps              &  boolean       &               & false   &       & SIDS   \\
switchingMirror           &  boolean       &               & false   &       & M      \\
specificQuantities        &  quantity\N    &               &         &       & F      \\
spect_der                 &  boolean       &               & false   &       & F      \\
tangentGrid               &  vGrid\N       &               &         &       & F      \\
temp_der                  &  boolean       &               & false   &       & F      \\
tolerance                 &  numeric       & temperature   & -1.0 K  &       & F      \\
transformMIFextinction    &  boolean       &               & false   &       & F H L P \\
transformMIFRHI           &  boolean       &               & false   &       & none   \\
trapezoid                 &  trapezoid     &               & wrong   &       & F      \\
TScatMIF                  &  numeric       & dimensionless & 1       &       & T      \\
TScatMoleculeDerivatives  &  molecule      &               &         &   *   & T      \\
TScatMolecules            &  molecule      &               &         &   *   & T      \\
type                      &  fwmType       &               &         &       & all    \\
usbLBLMolecules           &  molecule      &               &         &   *   & F      \\
usbPFAMolecules           &  molecule      &               &         &   *   & F      \\
useTScat                  &  boolean       &               & false   &       & F      \\
xStar                     &  vector\N      &               &         &       & L      \\
yStar                     &  vector\N      &               &         &       & L      \\
\end{longtable}

The {\tt type} field is the only required field.

The value of the {\tt lsbLBLMolecules}, {\tt lsbPFAMolecules}, {\tt
usbLBLMolecules}, or {\tt usbPFAMolecules} field can be an array,
including an empty array.

If the type of the field is not a statement label, numeric, boolean, or
string, the list of literals allowed for the type is specified in {\tt
init_tables_module} by a sequence starting with {\tt begin} followed by
{\tt t+t_} and the type name, and ending with {\tt n+n_dt_def}.

\begin{description}

\item[\tt allLinesForRadiometer] if the value is true, it specifies that
  all lines for the radiometer, for each molecule, are to be used, not
  only lines within the frequency range of the channel.

\item[\tt allLinesInCatalog] if the value is true, it specifies that all
  lines in the catalog, for each molecule, are to be used, not only lines
  within the frequency range of the channel.

\item[\tt atmos_der] if the value is true, it specifies that derivatives
  are to be calculated with respect to quantities other than temperature
  and cloud ice water content.

\item[\tt atmos_second_der] if the value is true, it specifies that second
  derivatives are to be calculated with respect to quantities other than
  temperature and cloud ice water content.

\item[\tt binSelectors]

\item[\tt cloud_der] if the value is true, it specifies that derivatives
  are to be calculated with respect to cloud ice water content. This is
  used with the full cloud forward model.

\item[\tt default_spectroscopy]

\item[\tt differentialScan]

\item[\tt do_baseline]

\item[\tt do_conv] if the value is true, it specifies that antenna
  convolution is to be done.

\item[\tt do_freq_avg] if the value is true, it specifies that frequency
  averaging is to be done. The full forward model is run for all the
  frequencies in the pointing frequency grid file, for each tangent
  pressure. If the value of this field is false, the full forward model is
  run only at channel center frequencies.

\item[\tt do_1d] does something for the scan model. If its value is true
  for the {\tt linear} or {\tt polarLinear} model, an ``irrelevant
  parameter'' error is announced.

\item[\tt forceSidebandFraction] if the value is true, it specifies to use
  the {\tt limbSidebandFraction} quantity for antenna convolution. The
  quantity can be taken from either the {\tt state} or {\tt
  fwdModelEx\-tra} vector.

\item[\tt frqTol] specifies the maximum distance from the desired
  frequency when searching Mie tables during generation and use of TScat
  tables.

\item[\tt i_saturation]

\item[\tt ignoreHessian] if the value is true, it specifies to the linear
  and TScat models not to fill Hessian components from L2PC bins. If
  false, the linear model uses a second-order method.

\item[\tt incl_cld] if the value is true, it specifies that the full
  forward model is to invoke the cloud or TScat models.

\item[\tt integrationGrid] provides $\zeta$ values in addition to those in
  the temperature, mixing ratio, extinction, and tangentGrid quantities.

\item[\tt linearSideband]

\item[\tt lineCenter] specifies molecules for which derivatives with
  respect to the line center frequency are to be calculated. The line
  center frequency is not in the state vector, so one cannot solve for the
  line center frequency, which might be useful to estimate the
  line-of-sight component of wind velocity. This is for offline studies.

\item[\tt lineWidth] specifies molecules for which derivatives with
  respect to the line width are to be calculated. The line width is not in
  the state vector, so one cannot solve for the line width. This is for
  offline studies.

\item[\tt lineWidth_TDep] specifies molecules for which derivatives with
  respect to the line width temperature dependence exponent are to be
  calculated. The line width temperature dependence exponent is not in the
  state vector, so one cannot solve for the line width temperature
  dependence exponent. This is for offline studies.

\item[\tt lockBins]

\item[\tt lsbLBLMolecules] specifies the molecules for which the lower
  sideband model is a line-by-line model.

\item[\tt lsbPFAMolecules] specifies the molecules for which the lower
  sideband model is a PFA model.

\item[\tt MIFTangent] is not yet used. I think I added this in
  anticipation of needing it for QTM.

\item[\tt module] used to specify the instrument module when finding
  forward model vector quantities.

\item[\tt moleculeDerivatives] molecules for which derivatives with
  respect to mixing ratio are to be calculated.

\item[\tt moleculeSecondDerivatives] molecules for which second
  derivatives with respect to mixing ratio are to be calculated.

\item[\tt molecules] molecules for which radiance are to be calculated. 
  An element of the {\tt molecules} field array can be an array; if so,
  the first molecule specifies the ``beta group'' name, and other
  molecules in the array specify the molecules in the group. The molecule
  group name must be the same as a molecule in the group.

\item[\tt nabterms]

\item[\tt nazimuthangles]

\item[\tt ncloudspecies]

\item[\tt nmodelsurfs]

\item[\tt no_dup_mol] if the value is true, it indicates that duplicate
  molecule specifications are an error. This affects configuration
  processing, but is not used by any model.

\item[\tt noMagneticField] appears not to be used.

\item[\tt nscatteringangles] specifies the number of scattering angles for
  the full cloud forward model, and the number of scattering angles to use
  when creating TScat tables.

\item[\tt ncloudspecies]

\item[\tt nsizebins]

\item[\tt pathNorm] $\int \frac{\d s}{\d h} \frac{\d h}{\d \zeta}\, \d
  \zeta$ ought to be $\delta\zeta$. If {\tt pathNorm} is true, multiply
  {\tt dhdz_path(i)} by\\
  {\tt del_s(i) / ( sum(dsdz_gw_path(i) * del_zeta(i))} to cause an
  approximation to that expected normalization.

\item[\tt phiWindow] specifies the number of profiles before and after the
  tangent point, or the angular extent before and after the tangent
  point.  The phi window is the part of the {\tt hGrid} used in {\tt
  Fill_Grids_1} in {\tt load_species_data_m}. It can be either a number or
  a range of numbers. If it is a number, it specifies both before and
  after extents. If it is a range, the first specifies the extent before
  the tangent point and the second specifies the extent after. The units
  can be either angles or profiles. If it is a range, the second number
  can be dimensionless, in which case its dimensions are taken from the
  first number.

\item[\tt polarized] If the value is true, it specifies that the full
  forward model or the linearized model is the polarized model. The
  polarized model cannot use the PFA method.

\item[\tt referenceMIF] appears not to be used.

\item[\tt refract] if the value is true, it specifies to apply the phi
  refractive correction defined by Equation (A.6) in Appendix A of the 19
  August 2005 MLS Full Forward Model ATBD. This is different from the
  refractive correction given by Equation (10.12) on page 45 of the 19
  August 2005 MLS Full Forward Model ATBD. The true default can be
  overridden by the {\tt -Snorf} command-line switch.

\item[\tt scanAverage] if the value is true, it specifies to use scan
  averaging during antenna convolution.

\item[\tt signals] specifies the signals the model is to use.

\item[\tt skipOverlaps] if the value is true, it specifies that the SIDS
  module is not to calculate for MAFS in the overlap region. It is not
  used by any model.

\item[\tt switchingMirror]

\item[\tt specificQuantities] used when finding quantities if the quantity
  type and/or molecule is not sufficiently specific to select one quantity.
  Usually used for extinction.

\item[\tt spect_der] if the value is true, it specifies that derivatives
  with respect to spectral properties -- line center frequency, line width,
  and exponent of line width temperature dependence -- are to be calculated.

\item[\tt tangentGrid]  provides $\zeta$ values in addition to those in
  the temperature, mixing ratio, extinction, and integrationGrid
  quantities. If some of these are below the surface, it affects the value
  of the surface tangent index.

\item[\tt temp_der] if the value is true, it specifies that derivatives
  with respect to temperature are to be calculated.

\item[\tt tolerance] is used to decide where Gauss-Legendre quadrature is
  necessary or trapezoidal quadrature is sufficient, when integrating
  incremental optical depth. $\delta = \int_0^s \alpha(s) \, \d s$.  A
  first-order difference is used to estimate $\frac{\d \mathcal{T}}{\d s}
  \sim \exp \delta \frac{\d \delta}{\d s}$. Where $\frac{\d
  \mathcal{T}}{\d s} <$ {\tt tolerance}, Gauss-Legendre quadrature is not
  needed. $\frac{\d \mathcal{T}}{\d s}$ is always positive, so setting
  {\tt tolerance} negative causes Gauss-Legendre quadrature to be used on
  all panels.

\item[\tt transformMIFextinction] if the value is true, it specifies to
  transform MIF extinction quantities to ``molecules'' before running the
  forward model, and transform the ``molecules'' and associated columns of
  the Jacobian to MIF extinction after return. see \h{wvs-107}.

\item[\tt transformMIFRHI] appears not to be used.

\item[\tt trapezoid] specifies which trapezoidal quadrature to use on
  panels that do not need Gauss-Legendre quadrature. Trapezoidal
  quadrature is usually written as $\frac12 ( f(a) + f(b) ) ( b - a )$.
  This is used if the value is {\tt correct}. If the value is {\tt wrong},
  $b-a$ is approximated as $\frac{\d s(a)}{\d h} \frac{\d h(a)}{\d \zeta}
  ( \zeta(b) - \zeta(a) )$. The latter might be more consistent with
  Gauss-Legendre quadrature because therein, the weight at the
  $i^\text{th}$ internal point is multiplied by $\frac{\d s(i)}{\d h} 
  \frac{\d h(i)}{\d \zeta}$ and the interval length is $\zeta(b) -
  \zeta(a)$.

\item[\tt TScatMIF] specifies the MIF index to use for line-of-sight
  velocity and $\phi$ reference when generating TScat tables.

\item[\tt TScatMoleculeDerivatives] specifies the molecules for which
  derivatives of TScat with respect to mixing ratio are to be calculated
  when generating TScat tables.

\item[\tt TScatMolecules] specifies the molecules for which TScat tables
  are to be generated.

\item[\tt type] type of model. Value is {\tt baseline}, {\tt linear}, {\tt
  full}, {\tt cloudFull}, {\tt hybrid}, {\tt scan}, {\tt scan2d}, {\tt
  switch\-ing\-Mirror}, or {\tt polarLinear}.

\item[\tt usbLBLMolecules] specifies the molecules for which the upper
  sideband model is a line-by-line model.

\item[\tt usbPFAMolecules] specifies the molecules for which the upper
  sideband model is a PFA model.
  
\item[\tt useTScat] if the value is true, it specifies that the full
  forward model is to use TScat tables to approximate cloud scattering.

\item[\tt xStar]

\item[\tt yStar]

\end{description}

\section*{Vector quantities}

The following quantities are sought from the vectors specified by the {\tt
State} or {\tt FwdModelEx\-tra} fields of the {\tt Retrieve} or {\tt SIDS}
specification in the {\tt Retrieve} section of the {\tt l2cf}:

\begin{longtable}{lll}
Quantity & Model & Required \\
\hline
baseline             & baseline           & yes       \\
boundaryPressure     & cloud              & depends   \\
cloudIce             & cloud              & optional  \\
earthRefl            & full               & yes       \\
ECRtoFOV             & full               & depends   \\
elevOffset           & full               & yes       \\
fieldAzimuth         & polarLinear        & yes       \\
GPH                  & cloud              & yes       \\
heightOffset         & scan               & yes       \\
isotopeRatio         & full               & optional  \\
limbSidebandFraction & full               & depends   \\
l1bMIF_TAI           & full               & yes       \\
losVel               & full               & yes       \\
lowestRetrievedPressure & all             & depends   \\
magneticField        & full polarLinear   & polarized \\
MIFExtinctionExtrapolation & all          & depends   \\
MIFExtinctionForm    & all                & depends   \\
MIFRadC              & full               & for QTM   \\
MIFDeadTime          & full               & yes       \\
opticalDepth         & linear             & yes       \\
orbIncline           & full scan2d        & depends   \\
phitan               & full scan          & yes       \\
ptan                 & baseline full scan & yes       \\
radiance             & full linear TScat  & yes       \\
refGPH               & full scan          & yes       \\
scanResidual         & scan               & yes       \\
scatteringAngle      & TScat gen          & yes       \\
scECR                & full               & depends   \\
scGeocAlt            & full               & yes       \\
scVelECR             & full               & depends   \\
spaceRadiance        & full               & yes       \\
surfaceHeight        & full               & yes       \\
temperature          & linear scan scan2d & yes       \\
tngtGeocAlt          & full scan          & for QTM   \\
TScat                & TScat gen or use   & yes       \\
vmr all              & full               & as needed \\
vmr h2o              & full scan          & yes       \\
vmr IWC              & TScat gen          & yes       \\
\end{longtable}

\begin{description}

\item [\tt baseline] at least one of a band-based quantity, a
  radiometer-based quantity, or an unspecified quantity must be present.
  If several are present, there is a complicated algorithm in {\tt
  BaselineForwardModel_m} to choose one, or announce an error if there are
  conflicts.

\item [\tt boundaryPressure] is used by the full cloud model if the value
  of the {\tt i_saturation} field of the {\tt ForwardModel} statement is
  {\tt clear}.

\item [\tt cloudIce] if present in the full forward model, it is loaded
  into the {\tt grids_IWC} structure; otherwise, components in that
  structure are allocated with zero size.

\item [\tt ECRtoFOV] is used by the full forward model

  \begin{itemize}
  
  \item to compute the line-of-sight vector when the horizontal basis is
  QTM.
  
  \item to rotate the magnetic field if the value of the {\tt polarized}
  field of the {\tt ForwardModel} statement is true.
  
  \end{itemize}

\item [\tt isotopeRatio] is used by the full forward model if it is
  present for any molecule; otherwise, the default isotope ratio from the
  spectroscopy catalog is used.

\item [\tt limbSidebandFraction] is required in the full forward model if
  the value of the {\tt forceSideband\-Frac\-tion} field of the {\tt
  ForwardModel} statement is true, or if the sideband for the first signal
  is zero.

\item [\tt lowestRetrievedPressure] is used  if the value of the {\tt
  transformMIFExtinction} field  of the {\tt For\-ward\-Model} statement
  is true.

\item [\tt magneticField] is required in the full forward model if the
  value of the {\tt polarized} field of the {\tt ForwardModel} statement is
  true.

\item [\tt MIFExtinctionExtrapolation] is used to get the MIF
  extrapolation exponent if the value of the {\tt transformMIFExtinction}
  field  of the {\tt ForwardModel} statement is true.

\item [\tt MIFExtinctionForm] is used to get the MIF extrapolation form if
  the value of the {\tt transform\-MIF\-Ex\-tinction} field  of the {\tt
  ForwardModel} statement is true.

\item [\tt orbIncline] is used by the full forward model if the value of
  the {\tt  generateTScat} field of the {\tt ForwardModel} statement is
  true, or if the horizontal basis is QTM. The latter use is probably
  obsolete.

\item [\tt scECR] is used by the full forward model to compute the
  line-of-sight vector when the horizontal basis is QTM.

\item [\tt scVelECR] is used in {\tt Viewing_Azimuth_Vec} in
  {ForwardModelGeometry} to compute the angle between the normal to the
  plane defined by the spacecraft position and velocity vectors, and the
  normal to the plane defined by {\tt ptan} and the instrument position,
  when the horizontal basis is QTM. {\tt Viewing_Azimuth_Vec} is not
  referenced in the full forward model. It was developed in anticipation
  of needing it to compute $\chi$ angles.

\item [\tt tngtGeocAlt] is used in {\tt Viewing_Azimuth_Vec} in
  {ForwardModelGeometry} to compute the angle between the normal to the
  plane defined by the spacecraft position and velocity vectors, and the
  normal to the plane defined by {\tt ptan} and the instrument position,
  when the horizontal basis is QTM. {\tt Viewing_Azimuth_Vec} is not
  referenced in the full forward model. It was developed in anticipation
  of needing it to compute $\chi$ angles.

\end{description}

I did not reverse engineer how quantities of types {\tt fieldAzimuth},
{\tt fieldElevation}, {\tt fieldStrength}, {\tt temperature}, {\tt TScat},
and {\tt vmr} are used in {\tt L2PCBins_m}.

\section*{Jacobian}

The Jacobian is required if any derivatives are requested in the {\tt
ForwardModel} statement. The retriever creates a Jacobian if one is not
specified on the {\tt Retrieve} statement. A {\tt SIDS} run doesn't
inherently know a Jacobian is necessary, so it won't create one. If you
want derivatives, for example to create {\tt l2pc} files, a Jacobian must
be specified.

\section*{Command-line options}

Several command-line options affect the full forward model. Command-line
options are case sensitive.

%\newcommand{\lbrack}{[}
%\newcommand{\rbrack}{]}
\begin{description}

\item[\tt -f\lbrack n\rbrack] causes tracing within the full forward
  model. Larger values of n cause more detailed tracing.

\end{description}

The following switches are preceded on the command line by {\tt -S}. Then
can be individually specified, each with {\tt -S}, or several can be
specified in one command-line option beginning with {\tt -S}. All {\tt -S} switches can optionally be
followed by a number $n$. If $n$ does not appear, it is assumed to be
zero. If a switch is not set, the {\tt SwitchDetail} function from the
{\tt MLSStringLists} module returns $-1$.

\begin{description}

\item[\tt 0sl] prints a warning message if there are no lines or continuum
  for at least one species.

\item[\tt ant] dumps the antenna patterns database after reading it in the
  {\tt globalSettings} section.

\item[\tt Azimuth] prints the viewing azimuth for UARS.

\item[\tt binsel] prints information about bin selection when selecting
  {\tt l2pc} bins.

\item[\tt clean] adds the ``-c'' option to several dumps. The effect is
  that array dumps are preceded by $\backslash n$ where $n$ is the array
  size, and lines of the dump do not contain subscripts. This is useful to
  read dumps in {\tt idl} using my {\tt read1d} program.

\item[\tt dacsfil] dumps the DACS filter database after reading it in the
  {\tt globalSettings} section.

\item[\tt deltaupol] causes the {\tt deltau_pol} array to be dumped. {\tt
  deltau_pol} is the exponential of the $2\times2$ polarized incremental
  optical depth matrix.

\item[\tt dmess] causes messages about requesting derivatives for
  quantities that are not in the state vector to be warnings instead of
  errors.

\item[\tt dpri\lbrack n\rbrack\ -SDpri -SDPRI] cause {\tt Rad_Tran_Pol} to
  dump internal arrays {\tt more_inds}, {\tt gl_inds}, {\tt
  gl_delta_polarized}, {\tt incoptdepth_pol_gl}, {\tt incoptdepth_pol},
  and {\tt ref_cor}. If $n = 0$, the dumps occur only if {\tt cs_expmat}
  (exponential of a $2 \times 2$ matrix) detects an overflow. If $n = 1$
  the dump occurs every time {\tt Rad_Tran_Pol} is invoked. If $n > 1$ the
  dump occurs on the first invocation, and {\tt mlsl2} stops. {\tt Dpri}
  is the same as {\tt dpri1}. {\tt DPRI} is the same as {\tt dpri2}. If
  {\tt cs_expmat} detects an overflow, any $n\geq0$ causes the dumps, and
  if $n > 1$ {\tt mlsl2} stops.

\item[\tt drfc\lbrack n\rbrack] dumps diagnostic information about the
  failure of the iteration to solve\\
   $h ( 1 + n_1 \exp ( \epsilon (h - h_1))) = \sqrt{q^2 +
  \mathcal{N}_t^2 H_t^2}$\\
  for $h$, where $q$ is one of the abscissae of the Gauss-Legendre
  quadrature for the integral in Equation (10.12) on page 45 of the 19
  August 2005 MLS Full Forward Model ATBD, $\epsilon =
  \frac{\log(n_2/n_1)}{h_2-h_1}$, $h_1$ and $n_1$ are the values of $h$
  and $\mathcal{N}$ at $\zeta_i$, and $h_2$ and
  $n_2$ are the values of $h$ and $\mathcal{N}$ at $\zeta_{i-1}$.

\item[\tt dsct] causes generated {\tt TScat} values to be dumped.

\item[\tt dtcst]  causes much printing related to derivatives during {\tt
  TScat} table generation.

\item[\tt filt] dumps the filter database after reading it in the {\tt
  globalSettings} section.

\item[\tt ffrq] causes frequencies selected from the point frequency grid
  to be dumped.

\item[\tt fmconf] causes the forward model configuration to be dumped.

\item[\tt fwmd\lbrack n\rbrack\ -SDpri -SDPRI] dumps the forward model
  configuration after adding some things derived from it, and the
  spectroscopy catalog extract. The detail level for dumping the
  spectroscopy catalog extract is the low-order digit of $n$. If $n>9$
  {\tt mlsl2} stops. Once set, the switch retains its value, even if it is
  set by the {\tt switches} field of the {\tt retrieve} statement.

\item[\tt -gMag] dumps the magnetic field quantity that is created by a
  {\tt fill} statement.

\item[\tt Grids] causes the internal structures {\tt grids_f} (mixing
  ratios), {\tt grids_tmp} (temperature) {\tt grids_v} (line center
  frequency), {\tt grids_w} (line width), and {\tt grids_m} (line width
  temperature dependence exponent) to be dumped if the horizontal basis is
  not QTM.

\item[\tt hnan] causes dumping during Hessian convolution.

\item[\tt hndp\lbrack n\rbrack] causes dumping during computation of the
  refractive correction described in Equation (10.12) of the 19 August 2004
  MLS ATBD. If $n\wedge 1 = 1$, dump the arrays if there is trouble.
  If $n\wedge 2 = 1$, dump the arrays and stop. If $n\wedge 4 = 1$, dump
  the iterates.

\item[\tt hphi] causes {\tt h} and {\tt phi} inputs to {\tt
  Height_Metrics} to be dumped.

\item[\tt igsc\lbrack n\rbrack] changes the error message to a warning if
  the scattering point does not appear to be in the path during {\tt
  TScat} table generation, if $n>0$.

\item[\tt incp] causes the species list from the forward model
  configuration, the index of the tangent point in the coarse grid, the
  index of the tangent point in the fine grid, $\beta$ on the coarse path,
  mixing ratios on the coarse path, $\alpha$ on the coarse path, the
  incremental optical depth, $\delta s$, indices of GL points on the fine
  path, and indices of panels on the coarse path needing GL, to be dumped.
  For polarized runs it also causes the $2\times2$ polarized incremental
  optical depth matrix to be dumped. It also causes output during {\tt
  TScat} table generation.

\item[\tt incr] causes {\tt tau} $\left(\exp \int_0^s \alpha \,\d
  s\right)$, {\tt inc_rad_path} ($I(s)$), {\tt t_script} ($\mathcal{T} =
  \Delta B$), frequency, and {\tt radV} ($I(s_{\mathcal{M}})$) to be
  dumped. It also causes output during {\tt TScat} table generation.

\item[\tt lblb\lbrack n\rbrack] causes printing during $\beta$ generation.
  for nonpolarized radiative transfer. Once set, the switch retains its
  value, even if it is set by the {\tt switches} field of the {\tt
  retrieve} statement. {\tt n} has three digits $n_2n_1n_0$. If $n_0>0$
  dump $\beta$. If $n_0>1$ dump temperature and $\text{tanh} \frac{h\nu}{2
  k T}$. If $n_1>0$ dump $\zeta$ instead of $P$. If $n_2>0$ stop after
  dumping. 

\item[\tt mag\lbrack n\rbrack] causes the MIF, MAF, rotation matrix ({\tt
  ECRtoFOV}), heights on the fine grid, the magnitude of the magnetic
  field $H$, the $\cos\theta$, $\sin\theta \cos\phi$, and $\sin\theta
  \sin\phi$ components of the magnetic field, to be dumped. If $n$ is odd
  the magnetic field on the path is also dumped as a matrix.

\item[\tt MagGrid\lbrack n\rbrack] dumps the {\tt grids_mag} structure
  after the magnetic field is generated. The detail level for the dump is
  given by {\tt n}.

\item[\tt metd\lbrack n\rbrack] dumps quantities during {\tt
  Height_Metrics} calculation. Increasing {\tt n} increases the number of
  quantities dumped. If $n>1$ {\tt mlsl2} stops after finishing {\tt
  More_Metrics}.

\item[\tt mhpx] causes output relating to failure of the $h$ -- $\phi$
  iteration in {\tt Height_Metrics}.

\item[\tt nfmpt] (case insensitive) turns off the pointing loop.

\item[\tt path\lbrack n\rbrack] causes path quantities to be printed in a
  readable format. If $n=0$ only $\phi$ and $\zeta$ are printed. If $n>1$
  the volume mixing ratios and $\beta$ are printed. If $n>0$, $\alpha$,
  incremental optical depth, $\tau = \exp \int_0^s \alpha \,\d s$, $I(s)$,
  and temperature are printed.

\item[\tt pfab] prints PFA data during PFA generation.  Once set, the
  switch retains its value, even if it is set by the {\tt switches} field
  of the {\tt retrieve} statement.

\item[\tt pfag] prints progress information during PFA generation.

\item[\tt pfaw] prints the PFA group name when one is created in a PFA HDF
  file.

\item[\tt plane\lbrack n\rbrack] causes printing while calculating the
  viewing plane for UARS.

\item[\tt point] dumps the pointing grid database after reading it in the
  {\tt globalSettings} section.

\item[\tt POLB polb] dumps polarized $\beta$. {\tt POLB} stops after
  dumping. Once set, the switch retains its value, even if it is set by
  the {\tt switches} field of the {\tt retrieve} statement.

\item[\tt polr\lbrack n\rbrack] causes the $2\times2$ complex radiance
  matrix to be printed. If $n>0$, the pointing index is also printed. If
  $n>1$ the frequency index is also printed.

\item[\tt psct] causes much printing during {\tt TScat} table generation.

\item[\tt ptg] prints the pointing ($\chi$) angles before antenna
  convolution.

\item[\tt rad\lbrack n\rbrack] causes tangent pressures and radiance to be
  printed when the full forward model completes. If $n>0$ it also causes
  frequencies and radiances to be printed for each pointing. Causes
  radiances to be printed when the linearized forward model completes.

\item[\tt RAD] causes radiances to be printed when the linearized forward
  model completes, and then {\tt mlsl2} stops.

\item[\tt rcfx\lbrack n\rbrack] prints information if the refractive
  correction computation failed. If $n>0$ and the refractive correction
  computation failed, {\tt mlsl2} stops.

\item[\tt seez] causes many things to be printed. See {\tt
  dump_print_code.f9h}. This started as Zvi's debug printing, and then
  grew.

\item[\tt spec speC] dumps the spectroscopy catalog at the end of the {\tt
  Spectroscopy} section. {\tt speC} stops after dumping the catalog.

\item[\tt tauL] causes line-by-line $\tau = \exp \int_0^s \alpha \,\d s$
  to be printed during a PFA run. This is called $\tau^s_{in}$ in
  \h{wvs-027}.

\item[\tt tauP] causes PFA $\tau = \exp \int_0^s \alpha \,\d s$ to be
  printed during a PFA run. This is called $\tau^w_{in}$ in \h{wvs-027}.

\item[\tt -TScat] causes much printing during {\tt TScat} table
  generation.

\item[\tt zdet] prints details relating to finding a minimum $\zeta$ that
  is not at the tangent point.

\item[\tt ZMOR] causes printing related to finding a minimum $\zeta$ that
  is not at the tangent point.

\item[\tt zmor] causes a search for a minimum $\zeta$ that is not at the
  tangent point. This code has never been exercised; it might not work.

\end{description}

\label{lastpage}
\vspace*{-0.1in} % Somehow, this causes lastpage to be defined
\end{document}

% $Id$

% $Log$
