\documentclass[11pt]{article}
\usepackage[fleqn]{amsmath}\textwidth 6.5in
\oddsidemargin -0.25in
%\evensidemargin -0.5in
\topmargin -0.25in
\textheight 9in

\newcommand{\docname}{\bf wvs-065r4}
\newcommand{\docdate}{5 February 2014}

\ifx\pdfoutput\undefined
  \pdfoutput=0
  \usepackage[hypertex,plainpages,hyperindex=true]{hyperref}
  \hypersetup{%
    hypertexnames=false%
  }
  % Specify the driver for the color package
  \ExecuteOptions{dvips}
  %\ExecuteOptions{xdvi}
\else
  \ifnum\pdfoutput>0
    \usepackage[pdftex,plainpages,hyperindex=true,pdfpagelabels]{hyperref}
    \hypersetup{%
      hypertexnames=false,%
      colorlinks=true,%
      linktocpage=true,%
    }
    % Specify the driver for the color package
    \ExecuteOptions{pdftex}
  \else
    \usepackage[hypertex,plainpages,hyperindex=true]{hyperref}
    \hypersetup{%
      hypertexnames=false%
    }
    % Specify the driver for the color package
    \ExecuteOptions{dvips}
    %\ExecuteOptions{xdvi}
  \fi
\fi

\hyperbaseurl{}
\renewcommand{\d}{\text{d}}
\ifx\dvidir\undefined
  \newcommand\hr[1]{\href{#1.dvi}{dvi} \href{#1.pdf}{pdf}}
\else
  \newcommand\hr[1]{\href{\dvidir/#1.dvi}{dvi} \href{\pdfdir/#1.pdf}{pdf}}
\fi
\newcommand\h[1]{#1 \hr{#1}}
\newcommand\hh[2]{#1 \hr{#2}}
\newcommand\hrpdf[1]{\href{#1.pdf}{pdf}}
\newcommand\hpdf[1]{#1 \hrpdf{#1}}

\begin{document}

%\tracingcommands=1
\newlength{\hW} % heading box width
\newlength{\pW} % page number field width
\settowidth{\hW}{\docname}
\settowidth{\pW}{Page \pageref{lastpage}\ of \pageref{lastpage}}
\ifdim \pW > \hW \setlength{\hW}{\pW} \fi
\makeatletter
\def\@biblabel#1{#1.}
\newcommand{\ps@twolines}{%
  \renewcommand{\@oddhead}{%
    \docdate\hfill\parbox[t]{\hW}{{\hfill\docname}\newline
                          Page \thepage\ of \pageref{lastpage}}}%
\renewcommand{\@evenhead}{}%
\renewcommand{\@oddfoot}{}%
\renewcommand{\@evenfoot}{}%
}%
\makeatother
\pagestyle{twolines}

\vspace{-10pt}
\begin{tabbing}
\phantom{References: }\= \\
To: \>Van\\
Subject: \>Summary of Mie parameters\\
From: \>Van Snyder\\
Reference: \>\h{wvs-058}, \h{wvs-070}
\end{tabbing}

\parindent 0pt \parskip 6pt
\vspace{-10pt}

\section{Mie Efficiencies}

Compute
\begin{equation}\begin{split}
\beta_{c\_e} =\,& \pi \int_0^\infty n(r) r^2 \xi_e(r)\, \text{d}r \text{ and}\\
\beta_{c\_s} =\,& \pi \int_0^\infty n(r) r^2 \xi_s(r)\, \text{d}r
\end{split}\end{equation}
where $\xi_e(r)$ and $\xi_s(r)$ are
%
\begin{equation}
\xi_s = \frac2{\chi^2} \sum_{n=1}^{n_\text{cut}}
      (2 n + 1 ) ( |a_n|^2 + |b_n|^2 ) \text{ and }
\xi_e = \frac2{\chi^2} \sum_{n=1}^{n_\text{cut}}
      (2 n + 1 ) \Re( a_n + b_n )\,,
\end{equation}
%
$\chi = \frac{2 \pi r}{\lambda}$, $r$ is particle radius,
%
$\lambda$ is wavelength,
\begin{equation}
a_n = \frac{(A_n/m + n/\chi) \Re W_n - \Re W_{n-1}}
           {(A_n/m + n/\chi) W_n - W_{n-1}}\,, \,
b_n = \frac{(m A_n + n/\chi) \Re W_n - \Re W_{n-1}}
           {(m A_n + n/\chi) W_n - W_{n-1}}\,,\\
\end{equation}
%
$W_n = \chi h^{(2)}_n (\chi)$, $h^{(2)}_n (\chi)$ is the spherical
Hankel function of the second kind and order $n$ = $j_n(\chi) -
i y_n(\chi)$, $j_n(\chi)$ and $y_n(\chi)$ are spherical Bessel functions
of the first and second kind, respectively,
%
\begin{equation}
A_n = -\frac{n}{m \chi} + \frac{j_{n-1}(m \chi)}
                               {j_n(m \chi)}\,,
\end{equation}
%
$m = \sqrt{\epsilon}$ is the complex refractive index, and $\epsilon$ is
the complex dielectric constant.

Let $D = 2r$.  Then the particle number distribution as a function of
particle size $r$ ($\mu$m) is
\begin{equation}\begin{split}\label{five}
 n(D) =\,& \frac1{\rho_{\text{ice}}} \left\{
           N_1 D \exp(-\alpha D) +
            \frac{N_2}D \exp \left[ -\frac12 \left(
              \frac{\log(D/D_0) - \mu}\sigma \right)^2\right] \right\}
       \text{ where} \\
 N_1 =\,& \frac{\text{IWC}_{<100} \alpha^5}{4 \pi}
 \text{ and }
 N_2 = \frac6{\sqrt{2 \pi^3}}
       \frac{\text{IWC}_{>100}}
            {D_0^3 \rho_{\text{ice}} \sigma \exp( 3 \mu + 4.5 \sigma^2)}\,,
\end{split}\end{equation}
%
\begin{equation}\begin{split}\label{six}
 & \text{IWC}_{<100} = \min[\text{IWC},
                            0.252(\text{IWC}/\text{IWC}_0)^{0.837}]\,, \\
 & \text{IWC}_{>100} = \text{IWC} - \text{IWC}_{<100}\,, \\
 & \alpha = -4.99\times 10^{-3} - 0.0494
              \log_{10} (\text{IWC}_{<100}/\text{IWC}_0)\,, \\
 & \mu = (5.2 + 0.0013 T) + (0.026 - 1.2 \times 10^{-3}T)
          \log_{10} (\text{IWC}_{>100}/\text{IWC}_0)\,, \\
 & \sigma = (0.47 + 2.1 \times 10^{-3}T) + (0.018 - 2.1 \times 10^{-4} T)
             \log_{10} (\text{IWC}_{>100}/\text{IWC}_0)\,,
\end{split}\end{equation}
%
IWC is the ice water content of the atmosphere in g/m$^3$, IWC$_0$ = 1
g/m$^3$, $D_0 = 1 \mu$m, $\rho_{\text{ice}} = 0.91$ g/m$^3$, and $T$ is
the atmospheric temperature in degrees Celsius.\footnote{Equation (5.3) in
the ATBD has an exponent of 1, not 3, for $D_0$ in the denominator of the
expression for $N_2$ in Equation (\ref{five}).  In Equation (4) in Greg M.
McFarquhar, Andrew J. Heymsfield, \emph{Parameterization of Tropical
Cirrus Ice Crystal Size Distributions and Implications for Radiative
Transfer: Results from CEPEX}, {\bf Journal of the Atmospheric Sciences
54}, 1 (Sep 1997) pp.\ 2187-2200 (MH hereafter), the exponent is 3. 
Equation (5.6) in the ATBD has +0.0494 in the expression for $\alpha$ in
Equation (\ref{six}).  Equation (6) in MH has $-$0.0494.  Tables 1 and 2
on page 2193 in MH give coefficients as a function of degrees Celsius.}

\section{Phase function}

The phase function is

\begin{equation}\label{one}
p(\theta,r) = \frac{p_0(\theta,r)}{C(r)}
\end{equation}

where

\begin{equation}\begin{split}\label{two}
p_0(\theta,r) =\,& |S_1(\theta,r)|^2 + |S_2(\theta,r)|^2 =
  \Re (S_1(\theta,r))^2 + \Im (S_1(\theta,r))^2 +
  \Re (S_2(\theta,r))^2 + \Im (S_2(\theta,r))^2 \\
S_1 =\,& \sum_{j=1}^\infty \frac{2j+1}{j(j+1)} \left(
 a_j(r,T) \frac{\d P_j^1(\cos\theta)}{\d \theta} +
 b_j(r,T) \frac{P_j^1(\cos\theta)}{\sin\theta} \right) \\
S_2 =\,& \sum_{j=1}^\infty \frac{2j+1}{j(j+1)} \left(
 a_j(r,T) \frac{P_j^1(\cos\theta)}{\sin\theta} +
 b_j(r,T) \frac{\d P_j^1(\cos\theta)}{\d \theta} \right) \\
C(r) =\,& \frac12 \int_0^\pi p_0(\theta,r) \sin\theta \, \d \theta \,. \\
\end{split}\end{equation}

The integrated phase function is

\begin{equation}\label{nine}
P(\theta) = 
 \frac\pi{\beta_{c\_s}} \int_0^\infty
 r^2 n(r) \xi_s(r)
  \frac{p_0(\theta,r)}{C(r)} \,\d r \,.
\end{equation}

As shown in wvs-070, this may be reduced to a single integral:

\begin{equation}
P(\theta) =
 \frac{\lambda^2}{2 \pi \beta_{c\_s}} \int_0^\infty n(r)\, p_0(\theta,r) \,\d r
\end{equation}

and its derivatives correspondingly simplify to

\begin{equation}\begin{split}
\frac{\partial P(\theta)}{\partial T} = \,&
 \frac{\lambda^2}{2 \pi \beta_{c\_s}} \int_0^\infty\,
  \frac{\partial n(r)}{\partial T}\, p_0(\theta,r) +
  n(r) \frac{\partial p_0}{\partial T} \,\d r -
  \frac{P(\theta)}{\beta_{c\_s}} \frac{\partial \beta_{c\_s}}{\partial T}
\text{ and} \\
\frac{\partial P(\theta)}{\partial IWC} = \,&
 \frac{\lambda^2}{2 \pi \beta_{c\_s}} \int_0^\infty\,
  \frac{\partial n(r)}{\partial IWC}\, p_0(\theta,r) \,\d r -
  \frac{P(\theta)}{\beta_{c\_s}} \frac{\partial \beta_{c\_s}}{\partial IWC}
\,.
\end{split}\end{equation}

\label{lastpage}
\end{document}
% $Id$

% $Log$
% Revision 1.4  2014/01/29 23:16:20  vsnyder
% Correct Equation number in MH footnote
%
% Revision 1.3  2014/01/29 01:35:37  vsnyder
% Add footnote about differences between ATBD and McFarquhar-Heymsfield
%
% Revision 1.2  2013/07/17 02:13:15  vsnyder
% Add phase function section
%
