\documentclass[11pt]{article}
\usepackage[fleqn]{amsmath}\textwidth 6.5in
\oddsidemargin -0.25in
%\evensidemargin -0.5in
\topmargin -0.5in
\textheight 9.4in

\newcommand{\docname}{\bf wvs-062}
\newcommand{\docdate}{7 November 2007}

\begin{document}

%\tracingcommands=1
\newlength{\hW} % heading box width
\newlength{\pW} % page number field width
\settowidth{\hW}{\docname}
\settowidth{\pW}{Page \pageref{lastpage}\ of \pageref{lastpage}}
\ifdim \pW > \hW \setlength{\hW}{\pW} \fi
\makeatletter
\def\@biblabel#1{#1.}
\newcommand{\ps@twolines}{%
  \renewcommand{\@oddhead}{%
    \docdate\hfill\parbox[t]{\hW}{{\hfill\docname}\newline
                          Page \thepage\ of \pageref{lastpage}}}%
\renewcommand{\@evenhead}{}%
\renewcommand{\@oddfoot}{}%
\renewcommand{\@evenfoot}{}%
}%
\makeatother
\pagestyle{twolines}

\vspace{-10pt}
\begin{tabbing}
\phantom{References: }\= \\
To: \>Bill, Nathaniel, Alyn, Dong, Mike\\
Subject: \>Minimum and maximum state vector elements during retrieval\\
From: \>Van Snyder\\
%Reference: \>wvs-060
\end{tabbing}

\parindent 0pt \parskip 6pt
\vspace{-10pt}

I've added optional fields {\tt stateMax} and {\tt stateMin} to the {\tt
retrieve} command.  The field values have to be vectors with the same
template as the state vector.

When the retriever starts, if {\tt stateMax} is specified, it gets filled
with -HUGE(0.0).  If {\tt stateMin} is specified, it gets filled
with HUGE(0.0).

After each Newton move, if {\tt stateMax} is specified it gets replaced
with max({\tt stateMax}, {\tt state}).  If {\tt stateMin} is specified it
gets replaced with  min({\tt stateMin}, {\tt state}).  This includes
rejected moves, so the bounds might be exaggerations of where the solver
actually went successfully --- but it doesn't reject moves very often. 
If you want to see when it accepts and rejects Newton moves, set the {\tt
-Snwt} command-line switch.  If the Newton solver goes too far uphill and
can't get back down again with another Newton move, it will retreat to
the best solution it had and take a gradient move.  If you want to see
the state every time it evaluates the model, set the {\tt -Sxvec}
command-line switch.

Since the retriever initializes these vectors, if you want to see what
got put into them during the iteration, the L2CF needs to write them (or
dump them) after each {\tt retrieve} command that mentions them and
before the next one that mentions them.

\label{lastpage}
\end{document}
% $Id$
