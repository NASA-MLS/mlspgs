\documentclass[11pt]{article}
\usepackage[fleqn]{amsmath}\textwidth 6.5in
\oddsidemargin -0.25in
%\evensidemargin -0.5in
\topmargin -0.35in
\textheight 9.35in

\newcommand{\docname}{\bf wvs-091}
\newcommand{\docdate}{10 February 2010}

\begin{document}

%\tracingcommands=1
\newlength{\hW} % heading box width
\newlength{\pW} % page number field width
\settowidth{\hW}{\docname}
\settowidth{\pW}{Page \pageref{lastpage}\ of \pageref{lastpage}}
\ifdim \pW > \hW \setlength{\hW}{\pW} \fi
\makeatletter
\def\@biblabel#1{#1.}
\newcommand{\ps@twolines}{%
  \renewcommand{\@oddhead}{%
    \docdate\hfill\parbox[t]{\hW}{{\hfill\docname}\newline
                          Page \thepage\ of \pageref{lastpage}}}%
\renewcommand{\@evenhead}{}%
\renewcommand{\@oddfoot}{}%
\renewcommand{\@evenfoot}{}%
}%
\makeatother
\pagestyle{twolines}

\vspace{-10pt}
\begin{tabbing}
\phantom{References: }\= \\
To: \>Van\\
Subject: \>Accessing the TScat Jacobian as a {\tt Matrix\_T} object\\
From: \>Van Snyder\\
%Reference: \>wvs-063
\end{tabbing}

\parindent 0pt \parskip 6pt
\vspace{-10pt}

Matrices in {\tt mlsl2} are two-level structures.  At the outer level a
matrix is composed of blocks.  The inner level describes the blocks. 
Each block is composed of elements, which can be represented as absent,
column sparse, banded, or full.  The details are in the modules {\tt
MatrixModule\_1} (for the outer level) and {\tt MatrixModule\_0} (for the
inner level).  The following table describes symbols used below.

\begin{tabular}{l|l}
Symbol & Index of\\
\hline
{\tt i}$_b$ & a band (signal quantity) \\
{\tt k}$_c$ & a channel in the band indexed by {\tt i}$_b$ \\
$\phi_C$    & the orbit geodetic angle of a cloud scattering point \\
$\zeta_C$   & the pressure ($\zeta$) level of a cloud scattering point \\
{\tt j}$_q$ & a state-vector quantity (species or temperature) \\
$\phi_A$    & the orbit geodetic angle of a point in the atmosphere \\
$\zeta_A$   & the pressure ($\zeta$) level of a point in the atmosphere \\
\end{tabular}

An element of the TScat Jacobian is the derivative of radiance observed
in a specific channel of a specific band, scattered from a specific
point, with respect to a specific quantity (mixing ratio of a specific
species or the temperature) at a particular point in the atmosphere:
%
\begin{equation*}
\frac{\partial I^{\text{\tt i}_b,\text{\tt k}_c}_{\phi_C,\zeta_C}}
     {\partial f^{\text{\tt j}_q}_{\phi_A,\zeta_A}}
\end{equation*}
%
where $f^{\text{\tt j}_q}_{\phi_A,\zeta_A}$ is the $\text{\tt
j}_q^\text{th}$ quantity (either mixing ratio or temperature).

Assuming the blocks are full and the name of the Jacobian is {\tt M}, an
element of the Jacobian can be referenced using {\tt
M\%block(i,j)\%values(k,l)} where
%
\begin{equation*}\begin{split}
\text{{\tt i}} =\,&
 \text{\tt i}_b + (\phi_C-1) \times (\text{number of bands}) \\
\text{{\tt j}} =\,&
 \text{\tt j}_q + (\phi_A-1) \times (\text{number of quantities}) \\
\text{{\tt k}} =\,&
 \zeta_C + ( \text{\tt k}_c -1 ) \times (\text{number of }\zeta_C) \\
\text{{\tt l}} =\,& \zeta_A\,.
\end{split}\end{equation*}

The labels for the rows of {\tt M\%block} are given by a {\tt Vector\_T}
object in {\tt M\%row\%vec\%template}. The number of bands is the number
of quantities in that vector.  The labels for the columns of {\tt
M\%block} are given by a {\tt Vector\_T} object in {\tt
M\%col\%vec\%template}.  The following table describes how the indices
access the labels.

\begin{tabular}{l|l}
Symbol & Indexes \\
\hline
{\tt i}$_b$ & {\tt M\%row\%vec\%template\%quantities} \\
{\tt k}$_c$ & {\tt M\%row\%vec\%template\%quantities(i$_b$)\%channels} \\
$\phi_C$    & {\tt M\%row\%vec\%template\%quantities(i$_b$)\%phi} \\
$\zeta_C$   & {\tt M\%row\%vec\%template\%quantities(i$_b$)\%surfs} \\
{\tt j}$_q$ & {\tt M\%col\%vec\%template\%quantities} \\
$\phi_A$    & {\tt M\%col\%vec\%template\%quantities(j$_q$)\%phi} \\
$\zeta_A$   & {\tt M\%col\%vec\%template\%quantities(j$_q$)\%surfs} \\
\end{tabular}

Assuming the name of the TScat radiance vector is {\tt V}, individual
radiances can be accessed using {\tt
V\%quantities(i$_b$)\%values(k,$\phi_C$)}.

\label{lastpage}
\end{document}

% $Id$

% $Log$
