\documentclass[11pt]{article}
\usepackage[fleqn]{amsmath}\textwidth 6.5in
\oddsidemargin -0.25in
%\evensidemargin -0.5in
\topmargin -0.25in
\textheight 9in

\newcommand{\docname}{\bf wvs-084r1}
\newcommand{\docdate}{24 May 2011}

\begin{document}

%\tracingcommands=1
\newlength{\hW} % heading box width
\newlength{\pW} % page number field width
\settowidth{\hW}{\docname}
\settowidth{\pW}{Page \pageref{lastpage}\ of \pageref{lastpage}}
\ifdim \pW > \hW \setlength{\hW}{\pW} \fi
\makeatletter
\def\@biblabel#1{#1.}
\newcommand{\ps@twolines}{%
  \renewcommand{\@oddhead}{%
    \docdate\hfill\parbox[t]{\hW}{{\hfill\docname}\newline
                          Page \thepage\ of \pageref{lastpage}}}%
\renewcommand{\@evenhead}{}%
\renewcommand{\@oddfoot}{}%
\renewcommand{\@evenfoot}{}%
}%
\makeatother
\pagestyle{twolines}

\vspace{-10pt}
\begin{tabbing}
\phantom{References: }\= \\
To: \>Bill, Dong, Van\\
Subject: \>Total IWC\\
From: \>Van Snyder\\
Reference: \>wvs-066
\end{tabbing}

\parindent 0pt \parskip 6pt
\vspace{-10pt}

\section{Ice particle size distribution}

Quoting from wvs-066:

\newcommand{\I}{\text{IWC}}

Let $D = 2r$ and $n(D) = n_1(D) + n_2(D)$ where
\begin{equation*}\begin{array}{cc}
 n_1(D) = N_1 D \exp(-\alpha D) &
 n_2(D) = \frac{N_2}D \exp \left( -\frac12 \gamma^2\right) \\
 N_1 = \frac{\I_{<100}\, \alpha^5}{4 \pi \rho_{\text{ice}}}
 &
 N_2 = \frac6{\sqrt{2 \pi^3}}
       \frac{\I_{>100}}
            {D_0^3 \rho_{\text{ice}} \sigma \exp( 3 \mu + 4.5 \sigma^2)}\,, \\
 \gamma = \frac{\log(D/D_0) - \mu}\sigma & \\
\end{array}\end{equation*}
%
$D_0 = 1 \mu$m, $\rho_{\text{ice}} = 0.91$ g/cm$^3$.

\begin{equation*}\begin{split}
 & \I_{<100} = \min[\I,0.252(\I/\I_0)^{0.837}] \\
 & \I_{>100} = \I - \I_{<100} \\
 & \alpha = -4.99\times 10^{-3} - 0.0494
              \log_{10} (\I_{<100}/\I_0) \\
 & \mu = (5.2 + 0.0013 T) + (0.026 - 1.2 \times 10^{-3}T)
          \log_{10} (\I_{>100}/\I_0) \\
 & \sigma = (0.47 + 2.1 \times 10^{-3}T) + (0.018 - 2.1 \times 10^{-4} T)
             \log_{10} (\I_{<100}/\I_0)
\end{split}\end{equation*}
%
where IWC$_0$ = 1 g/m$^3$ and $T$ is the atmospheric temperature in Celsius.

\section{Total IWC}

The total ice-water content is defined as

\begin{equation*}
\I_\text{total} = \int_0^\infty r^3 n(D)\, \text{d}r =
\int_0^\infty r^3 n_1(D)\, \text{d} r + \int_0^\infty r^3 n_2(D)\, \text{d} r\,.
\end{equation*}

According to Maple, these integrals can be expressed explicitly as

\begin{equation*}
\int_0^\infty r^3 n_1(D)\, \text{d} r = \frac32\, \frac{N_1}{\alpha^5}\,
 \text{ and }
\int_0^\infty r^3 n_2(D)\, \text{d} r =
 \sqrt{2 \pi}\, \frac{\sigma}{16}\, D_0^3 N_2
  \exp\left( \frac92 \sigma^2 + 3 \mu \right)\,.
\end{equation*}

After substituting $N_1$ and $N_2$ this simplifies to

\begin{equation*}
\I_\text{total} = \frac3{8\pi\rho_\text{ice}} \left( \I_{<100} + \I_{>100}
\right) = \frac3{8\pi\rho_\text{ice}}\, \I
\end{equation*}

\label{lastpage}
\end{document}
% $Id$

% $Log$
