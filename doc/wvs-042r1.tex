\documentclass[11pt]{article}
\usepackage[fleqn]{amsmath}\textwidth 6.5in
\oddsidemargin -0.25in
%\evensidemargin -0.5in
\topmargin -0.25in
\textheight 9.2in

\newcommand{\docname}{\bf wvs-042r1}
\newcommand{\docdate}{12 September 2006}

\usepackage{longtable}

\begin{document}

%\tracingcommands=1
\newlength{\hW} % heading box width
\newlength{\pW} % page number field width
\settowidth{\hW}{\docname}
\settowidth{\pW}{Page \pageref{lastpage}\ of \pageref{lastpage}}
\ifdim \pW > \hW \setlength{\hW}{\pW} \fi
\makeatletter
\def\@biblabel#1{#1.}
\newcommand{\ps@twolines}{%
  \renewcommand{\@oddhead}{%
    \docdate\hfill\parbox[t]{\hW}{{\hfill\docname}\newline
                          Page \thepage\ of \pageref{lastpage}}}%
\renewcommand{\@evenhead}{}%
\renewcommand{\@oddfoot}{}%
\renewcommand{\@evenfoot}{}%
}%
\makeatother
\pagestyle{twolines}

\vspace{-10pt}
\begin{tabbing}
\phantom{References: }\= \\
To: \>Bill\\
Subject: \>$H$--$\phi$ iteration in {\tt metrics}\\
From: \>Van Snyder\\
\end{tabbing}

\parindent 0pt \parskip 6pt
\vspace{-20pt}

The present version of {\tt metrics} uses $H^{\text{ref}}$ and
$\phi^{\text{basis}}$ to estimate $H$ and $\phi$ along the limb ray by using
the following iteration:

Use $H_t$ and $\phi_t$ for initial $H$ and $\phi$ for each $\zeta$ along the
limb ray\\
{\bf do}\\
\hspace*{0.25in} Interpolate in $H^{\text{ref}}$ using $\phi$ and
  $\phi^{\text{basis}}$ to get a new $H$\\
\hspace*{0.25in} Compute new $\phi := \cos^{-1} \frac{H_t}H$\\
{\bf until} $H$ doesn't change much

$H^{\text{ref}}$ is a two-dimensional array, having a number of rows $m$ that
is the same as the maximum size of $\zeta$ and a number of columns $n$ equal to
the size of $\phi^{\text{basis}}$.

Fred and I discussed this problem, and observed that since $H \propto \sec
\phi$, not $\phi$, rather than interpolate in $H^{\text{ref}}$ using $\phi$ and
$\phi^{\text{basis}}$, a new basis coordinate $H^{\text{basis}} := H_t \sec
(\phi^{\text{basis}}-\phi_t) \,\text{sign}(\phi^{\text{basis}}-\phi_t)$ should
be used, because rows of $H^{\text{ref}}$ should be linear in that coordinate. 
Then observing that $H \approx a \zeta + b$ it makes sense to use $H$ instead
of $\phi$ for the ``vertical'' coordinate of $H^{\text{ref}}_{^.}$  This results in
the simpler iteration:

Solve $H^{\text{ref}}_{1,n/2} = a \zeta^{\text{ref}}_1 + b$
and $H^{\text{ref}}_{m,n/2} = a \zeta^{\text{ref}}_m + b$ for $a$ and $b$\\
Compute initial $H = a \zeta^{\text{ref}} + b$\\
{\bf do}\\
\hspace*{0.25in} Interpolate in $H^{\text{ref}}$ using $H$ and $H^{\text{basis}}$
to get a new $H$\\
{\bf until} $H$ doesn't change much\\
Compute $\phi := \cos^{-1} \frac{H_t}H$

This always converges in one iteration, as is to be expected.

When solving for $a$ and $b$, a least-squares solution, perhaps using all
columns of $H^{\text{ref}}$ instead of only the middle one, might be better. 
Perhaps even better would be to compute separate $a$ and $b$ for each row of
$H^{\text{ref}}$, using the two adjacent rows: the shorter the range, the more
accurate the assumption that $H$ and $\zeta$ are linearly related.

In the one test I've run so far (one of the ``gold brick'' L2PC runs), the
computed {\tt h\_grid} is different from the one computed by the old method by
as much as 70 meters.  Fred has suggested to use higher-order interpolation in
$H^{\text{ref}}_{^,}$ at least ``one shot,'' to get an idea which method is correct.

\label{lastpage}
\end{document}
% $Id$
