\documentclass[11pt]{article}
\usepackage[fleqn]{amsmath}
\usepackage{longtable}
\usepackage[strings]{underscore}

\ifx\pdfoutput\undefined
  \pdfoutput=0
  \usepackage[hypertex,plainpages,hyperindex=true]{hyperref}
  \hypersetup{%
    hypertexnames=false%
  }
  % Specify the driver for the color package
  \ExecuteOptions{dvips}
  %\ExecuteOptions{xdvi}
\else
  \ifnum\pdfoutput>0
    \usepackage[pdftex,plainpages,hyperindex=true,pdfpagelabels]{hyperref}
    \hypersetup{%
      hypertexnames=false,%
      colorlinks=true,%
      linktocpage=true,%
    }
    % Specify the driver for the color package
    \ExecuteOptions{pdftex}
  \else
    \usepackage[hypertex,plainpages,hyperindex=true]{hyperref}
    \hypersetup{%
      hypertexnames=false%
    }
    % Specify the driver for the color package
    \ExecuteOptions{dvips}
    %\ExecuteOptions{xdvi}
  \fi
\fi

\hyperbaseurl{}
\ifx\dvidir\undefined
  \newcommand\hr[1]{\href{#1.dvi}{dvi} \href{#1.pdf}{pdf}}
\else
  \newcommand\hr[1]{\href{\dvidir/#1.dvi}{dvi} \href{\pdfdir/#1.pdf}{pdf}}
\fi
\newcommand\h[1]{#1 \hr{#1}}

\oddsidemargin -0.25in
%\evensidemargin -0.5in
\topmargin -0.5in
\textwidth 6.25in
\textheight 9.00in

\newcommand{\docname}{\bf wvs-027r2}
\newcommand{\docdate}{3 October 2011}

\begin{document}

%\tracingcommands=1
\newlength{\hW} % heading box width
\newlength{\pW} % page number field width
\settowidth{\hW}{\docname}
\settowidth{\pW}{Page \pageref{lastpage}\ of \pageref{lastpage}}
\ifdim \pW > \hW \setlength{\hW}{\pW} \fi
\makeatletter
\def\@biblabel#1{#1.}
\newcommand{\ps@twolines}{%
  \renewcommand{\@oddhead}{%
    \docdate\hfill\parbox[t]{\hW}{{\docname}\newline
                          Page \thepage\ of \pageref{lastpage}}}%
\renewcommand{\@evenhead}{}%
\renewcommand{\@oddfoot}{}%
\renewcommand{\@evenfoot}{}%
}%
\makeatother
\pagestyle{twolines}

\vspace{-10pt}
\begin{tabbing}
\phantom{References: }\= \\
To: \>Bill, Nathaniel, Mike\\
Subject: \>Revised method to combine line-by-line and PFA\\
From: \>Van Snyder\\
References: \>\h{wvs-024}, \h{wvs-025}, \h{wvs-026}\\
\end{tabbing}

\parindent 0pt \parskip 3pt
\vspace{-20pt}

\section{Introduction}

The radiative transfer equations used for line-by-line (LBL) calculations
are

\begin{align}
I_c =& \int \text{d}\nu\, \phi_c(\nu)
      \int \text{d} z\, \Delta B(z,\nu) \tau(z,\nu)
 \approx \sum_{n=1}^{N_f} \phi_{nc} \Delta \nu_{nc}
               \sum_{i=1}^{N_p} \Delta B_{in} \tau^s_{in}
      & \text{(averaged)}\label{basic} \\
I_c =& \int \text{d} z\, \Delta B(z,\nu) \tau(z,\nu)
 \approx \sum_{i=1}^{N_p} \Delta B_{in} \tau^s_{in}
      & \text{(monochromatic)}\label{mono}
\end{align}

where $\tau^s_{in}$ is the incremental transmittance due to strong
spectral lines.  Recall from wvs-024r2 that incremental transmittances
from strong and weak (pre-frequency-averaged or PFA) lines are combined
using

\begin{align}
I_c \approx & \sum_{n=1}^{N_f} \phi_{nc} \Delta \nu_{nc}
               \sum_{i=1}^{N_p} \Delta B_{ic} \tau^s_{in} \tau^w_{ic}
               \text{, and} \label{combined}
 && \text{(averaged)} \\
I_c \approx & \sum_{i=1}^{N_p} \Delta B_{ic} \tau^s_{in} \tau^w_{ic} \,.
 && \text{(monochromatic)} \label{combined-mono}
\end{align}

Memo \h{wvs-026} developed a frequency-averaged incremental radiance in
each channel at each point along the path by exchanging the order of
summation in Equations (\ref{basic},\ref{mono}) and then postponing the
outer sum

\begin{align}
\overline{I^s_{ic}} = &
 \sum_{n=1}^{N_f} \phi_{nc} \Delta \nu_{nc} \Delta B_{in} \tau^s_{in}
 \approx
 \Delta B_{ic} \sum_{n=1}^{N_f} \phi_{nc} \Delta \nu_{nc}
  \tau^s_{in}\text{, and}
 && \text{(averaged)} \label{FrqAvgPath} \\
\overline{I^s_{ic}} = &
 \Delta B_{ic} \tau^s_{ic}\,.
 && \text{(monochromatic)} \label{mono-path}
\end{align}

Exchanging the order of summation in Equation (\ref{combined}) we have

\begin{equation}\label{exchanged}
I_c \approx \sum_{i=1}^{N_p} \tau^w_{ic} \Delta B_{ic}
             \sum_{n=1}^{N_f} \phi_{nc} \Delta \nu_{nc} \tau^s_{in}
    = \sum_{i=1}^{N_p} \overline{I^s_{ic}} \tau^w_{ic}
\end{equation}

which is the same if there is no frequency averaging for LBL.  The
special treatment needed at the zero-thickness tangent-point layer is
applied when $\tau^s_{in}$ and $\tau^w_{ic}$ are calculated, so no
special care is needed here.

Memo \h{wvs-026} developed the approximations

\begin{align}
\frac{\partial I_c}{\partial x_k}
\approx &
 \sum_{n=1}^{N_f} \phi_{nc} \Delta \nu_{nc} \sum_{i=1}^{N_p}
 \left(
  \frac{\partial \Delta B_{in}}{\partial x_k} -
  \Delta B_{in}
   \sum_{j=1}^i \frac{\partial \delta^s_{jn}}{\partial x_k}
  \right) \tau^s_{in}
 -
 \sum_{i=1}^{N_p} \overline{I^s_{ic}} \tau^w_{ic}
  \sum_{j=1}^i\frac{\partial \delta^w_{jc}}{\partial x_k} && \text{(avg.)}
  \label{deriv} \\
\frac{\partial I_c}{\partial x_k}
\approx &
 \sum_{i=1}^{N_p}
 \left(
  \frac{\partial \Delta B_{ic}}{\partial x_k} -
  \Delta B_{ic}
   \sum_{j=1}^i \frac{\partial \delta^s_{jc}}{\partial x_k}
  \right) \tau^s_{ic}
 -
 \sum_{i=1}^{N_p} \overline{I^s_{ic}} \tau^w_{ic}
  \sum_{j=1}^i\frac{\partial \delta^w_{jc}}{\partial x_k} &&
  \text{(mono.)} \label{deriv-mono}
\end{align}

to combine LBL and PFA derivatives.

Setting $\frac{\partial \Delta B_{in}}{\partial x_k} \equiv 0$ and
$\frac{\partial \Delta B_{ic}}{\partial x_k} \equiv 0$ because $B$ does
not depend upon mixing ratios, these simplify to

\begin{align}
\frac{\partial I_c}{\partial x_k}
\approx &
 -\sum_{n=1}^{N_f} \phi_{nc} \Delta \nu_{nc} \sum_{i=1}^{N_p}
  \Delta B_{in}
   \sum_{j=1}^i \frac{\partial \delta^s_{jn}}{\partial x_k}
   \tau^s_{in}
 -
 \sum_{i=1}^{N_p} \overline{I^s_{ic}} \tau^w_{ic}
  \sum_{j=1}^i\frac{\partial \delta^w_{jc}}{\partial x_k} && \text{(avg.)}
  \label{deriv0} \\
\frac{\partial I_c}{\partial x_k}
\approx &
 -\sum_{i=1}^{N_p}
  \Delta B_{ic}
   \sum_{j=1}^i \frac{\partial \delta^s_{jc}}{\partial x_k}
   \tau^s_{ic}
 -
 \sum_{i=1}^{N_p} \overline{I^s_{ic}} \tau^w_{ic}
  \sum_{j=1}^i\frac{\partial \delta^w_{jc}}{\partial x_k} &&
  \text{(mono.)} \label{deriv0-mono}
\end{align}

In the LBL-only case

\begin{align}
\frac{\partial I_c}{\partial x_k}
\approx &
 -\sum_{n=1}^{N_f} \phi_{nc} \Delta \nu_{nc} \sum_{i=1}^{N_p}
  \Delta B_{in} \tau^s_{in}
   \sum_{j=1}^i \frac{\partial \delta^s_{jn}}{\partial x_k}
 && \text{(averaged)}
  \label{LBL0} \\
\frac{\partial I_c}{\partial x_k}
\approx &
 -\sum_{i=1}^{N_p}
  \Delta B_{ic} \tau^s_{ic}
   \sum_{j=1}^i \frac{\partial \delta^s_{jc}}{\partial x_k}
 &&
  \text{(monochromatic)} \label{LBL0-mono}
\end{align}

Special care must be taken for the zero-thickness tangent-point layer,
both in Equation (\ref{LBL0-mono}), and in the second term in Equations
(\ref{deriv},\ref{deriv-mono}).  Subroutine {\tt dscrt\_dx} takes this
special care in evaluating Equation (\ref{LBL0-mono}).

By replacing $\Delta B_{ic}$ by $\overline{I^s_{ic}}$ and $\frac{\partial
\delta^s_{jc}}{\partial x_k}$ by $\frac{\partial \delta^w_{jc}}{\partial
x_k}$, Equation (\ref{LBL0-mono}) becomes the second term in Equations
(\ref{deriv},\ref{deriv-mono}).  Therefore, {\tt dscrt\_dx} can evaluate
the second term in Equations (\ref{deriv},\ref{deriv-mono}).

In the PFA-only case, everything reduces to the monochromatic LBL-only
case:

\begin{align}
I_c =\,& \int \text{d} z\, \Delta B(z,\nu) \tau(z,\nu)
 \approx \sum_{i=1}^{N_p} \Delta B_{ic} \tau^s_{ic}\label{LBL}\\
\frac{\partial I_c}{\partial x_k}
 \approx\,& \sum_{i=1}^{N_p}
  \left(
   \frac{\partial \Delta B_{ic}}{\partial x_k} -
   \Delta B_{ic}
    \sum_{j=1}^i \frac{\partial \delta^w_{jn}}{\partial x_k}
   \right) \tau^w_{ic}
 = - \sum_{i=1}^{N_p} \Delta B_{ic} \tau^w_{ic}
       \sum_{j=1}^i \frac{\partial \delta^w_{jn}}{\partial x_k}\,,
   \label{LBL-deriv}
\end{align}

the last result arising because $B$ does not depend upon mixing ratios.

\newpage
\section{Sixteen cases in the Forward Model}

There are sixteen cases in the Forward Model, of which four are
impossible.  Processing is described by the following decision table.
Combinations of possibilities are above the line; actions are below the
line.  Hyphen means ``don't care.''

\begin{longtable}{p{2.5in}|ccccccccccccc}
Frequency avg.      (8) & -   & N & N & - & - & N & N &  Y &  Y &  Y &  Y \\
PFA (4)                 & N   & N & N & Y & Y & Y & Y &  N &  N &  Y &  Y \\
LBL (2)                 & N   & Y & Y & N & N & Y & Y &  Y &  Y &  Y &  Y \\
Derivatives (1)         & -   & N & Y & N & Y & N & Y &  N &  Y &  N &  Y \\
{\tt Frq_Avg_Sel value} & 0,1 & 2 & 3 & 4 & 5 & 6 & 7 & 10 & 11 & 14 & 15 \\
                        & 8,9 &   &   &12 &13 &   &   &    &    &    &    \\
&\\[-9pt]
\hline
&\\[-9pt]
Impossible              & x   &   &   &   &   &   &   &    &    &    &    \\
Radiances = {\tt RadV}  &     & \ref{LBL}
                                  & \ref{LBL}
                                      & \ref{LBL}
                                          & \ref{LBL}
                                              &   &   &    &    &    &    \\
{\tt K = K_frq}         &     &   & \ref{LBL-deriv}
                                      &   & \ref{LBL-deriv}
                                              &   &   &    &    &    &    \\
\raggedright Frequency average path integrated LBL radiance
                        &     &   &   &   &   &   &   & \ref{basic}
                                                           & \ref{basic}
                                                                & \ref{basic}
                                                                     &    \\
\raggedright Combine total path radiances
                        &     &   &   &   &   & \ref{combined-mono}
                                                  & \ref{combined-mono}
                                                      &    &    &    &    \\
\raggedright Frequency average integrated LBL derivative
                        &     &   &   &   &   &   &   &    & \ref{LBL0}
                                                                &    & \ref{LBL0} \\
\raggedright Frequency average radiance along path
                        &     &   &   &   &   &   &   &    &    &    & \ref{FrqAvgPath} \\
\raggedright Combine radiances along path
                        &     &   &   &   &   &   &   &    &    & \ref{exchanged}
                                                                     & \ref{exchanged} \\
\raggedright Combine LBL and PFA derivatives
                        &     &   &   &   &   &   & \ref{deriv0-mono}
                                                      &    &    &    & \ref{deriv0-mono} \\
\end{longtable}


\label{lastpage}
\end{document}
% $Id$

% $Log$
