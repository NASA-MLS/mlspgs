\documentclass[11pt]{article}
\usepackage{alltt}
\usepackage[fleqn]{amsmath}
\usepackage{floatflt}
\usepackage{graphicx}
\usepackage{longtable}
\usepackage[strings]{underscore}

\textwidth 6.5in
\oddsidemargin -0.25in
%\evensidemargin -0.5in
\topmargin -0.5in
\textheight 9in

\newcommand{\docname}{wvs-143}
\newcommand{\docdate}{20 September 2017}

\ifx\pdfoutput\undefined
  \pdfoutput=0
  \usepackage[hypertex,plainpages,hyperindex=true]{hyperref}
  \hypersetup{%
    hypertexnames=false%
  }
  % Specify the driver for the color package
  \ExecuteOptions{dvips}
  %\ExecuteOptions{xdvi}
\else
  \ifnum\pdfoutput>0
   
\usepackage[pdftex,plainpages,hyperindex=true,pdfpagelabels]{hyperref}
    \hypersetup{%
      hypertexnames=false,%
      colorlinks=true,%
      linktocpage=true,%
    }
    % Specify the driver for the color package
    \ExecuteOptions{pdftex}
  \else
    \usepackage[hypertex,plainpages,hyperindex=true]{hyperref}
    \hypersetup{%
      hypertexnames=false%
    }
    % Specify the driver for the color package
    \ExecuteOptions{dvips}
    %\ExecuteOptions{xdvi}
  \fi
\fi

\hyperbaseurl{}
\newcommand\hr[1]{\href{#1.dvi}{dvi}, \href{#1.pdf}{pdf}}
\newcommand\h[1]{#1 (\hr{#1})}

\begin{document}

%\tracingcommands=1
\newlength{\hW} % heading box width
\newlength{\pW} % page number field width
\settowidth{\hW}{\bf\docname}
\settowidth{\pW}{Page \pageref{lastpage}\ of \pageref{lastpage}}
\ifdim \pW > \hW \setlength{\hW}{\pW} \fi
\makeatletter
\def\@biblabel#1{#1.}
\newcommand{\ps@twolines}{%
  \renewcommand{\@oddhead}{%
    \docdate\hfill\parbox[t]{\hW}{{\hfill\bf\docname}\newline
                          Page \thepage\ of \pageref{lastpage}}}%
\renewcommand{\@evenhead}{}%
\renewcommand{\@oddfoot}{}%
\renewcommand{\@evenfoot}{}%
}%
\makeatother
\pagestyle{twolines}

\vspace{-10pt}
\begin{tabbing}
\phantom{References: }\= \\
To: \>Nathaniel, Bill, Van\\
Subject: \>Determining what part of the path to integrate\\
From: \>Van Snyder\\
%Reference: \\
\end{tabbing}

\parindent 0pt \parskip 6pt
\vspace{-20pt}

If the instrument is not within the atmosphere, the entire path through
the atmosphere is integrated.  If the instrument is within the atmosphere,
only the part of the path in the instrument viewing direction should be
integrated.

If the height of the instrument is greater than the highest height in the
{\tt h_ref} array in the Full Forward Model the instrument is considered
not to be within the atmosphere.  The {\tt h_ref} array records the height
of each $\zeta$ surface for each profile, computed using the temperatures
along each profile, assuming hydrostatic equilibrium.

The vectors $\mathbf{S}$ to the instrument and $\mathbf{T}$ to the tangent
point, both in ECR coordinates, together with the matrix $\mathbf{E}$,
which gives the rotation from ECR to FOV coordinates can be used to
determine which part of the path to integrate.

The third column of $\mathbf{E}$, denoted $\mathbf{E}_3$, is an unit
vector in the direction of the line-of-sight of the instrument.  If
$(\mathbf{T} - \mathbf{S}) \cdot \mathbf{E}_3 > 0$ the tangent point is in
the direction of the line of sight, the part of the path that should be
integrated begins at the end opposite the tangent point from the
instrument, and the tangent point is on that path.  Otherwise, the part of
the path that should be integrated begins at the end on the same side of
the tangent point as the instrument.

If the path of integration does not reflect from the Earth's surface, the
problem is somewhat simpler.  Let the endpoints of a path of integration
that passes entirely through the atmosphere be $\mathbf{P}_1$ and
$\mathbf{P}_2$.  If $(\mathbf{P}_1 - \mathbf{S}) \cdot \mathbf{E}_3 > 0$,
the beginning point of the path of integration is $\mathbf{P}_1$, else it
is $\mathbf{P}_2$.

The vectors $\mathbf{S}$ and $\mathbf{T}$ are minor frame quantities in
the forward model.  The MIF used to define them should be chosen such that
$\mathbf{T}$ is nearest to the tangent point on the path of integration.

The path of integration ends at the point on the path of integration that
is nearest to the instrument, which can be computed using the method
described in \h{wvs-142}.

\label{lastpage}
\vspace*{-0.1in} % Somehow, this causes lastpage to be defined
\end{document}

% $Id$

% $Log$
