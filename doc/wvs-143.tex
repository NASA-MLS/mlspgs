\documentclass[11pt]{article}
\usepackage{alltt}
\usepackage[fleqn]{amsmath}
\usepackage{floatflt}
\usepackage{graphicx}
\usepackage{longtable}
\usepackage[strings]{underscore}

\textwidth 6.5in
\oddsidemargin -0.25in
%\evensidemargin -0.5in
\topmargin -0.5in
\textheight 9in

\newcommand{\docname}{wvs-143r3}
\newcommand{\docdate}{2 October 2017}

\ifx\pdfoutput\undefined
  \pdfoutput=0
  \usepackage[hypertex,plainpages,hyperindex=true]{hyperref}
  \hypersetup{%
    hypertexnames=false%
  }
  % Specify the driver for the color package
  \ExecuteOptions{dvips}
  %\ExecuteOptions{xdvi}
\else
  \ifnum\pdfoutput>0
   
\usepackage[pdftex,plainpages,hyperindex=true,pdfpagelabels]{hyperref}
    \hypersetup{%
      hypertexnames=false,%
      colorlinks=true,%
      linktocpage=true,%
    }
    % Specify the driver for the color package
    \ExecuteOptions{pdftex}
  \else
    \usepackage[hypertex,plainpages,hyperindex=true]{hyperref}
    \hypersetup{%
      hypertexnames=false%
    }
    % Specify the driver for the color package
    \ExecuteOptions{dvips}
    %\ExecuteOptions{xdvi}
  \fi
\fi

\hyperbaseurl{}
\newcommand\hr[1]{\href{#1.dvi}{dvi}, \href{#1.pdf}{pdf}}
\newcommand\h[1]{#1 (\hr{#1})}

\begin{document}

%\tracingcommands=1
\newlength{\hW} % heading box width
\newlength{\pW} % page number field width
\settowidth{\hW}{\bf\docname}
\settowidth{\pW}{Page \pageref{lastpage}\ of \pageref{lastpage}}
\ifdim \pW > \hW \setlength{\hW}{\pW} \fi
\makeatletter
\def\@biblabel#1{#1.}
\newcommand{\ps@twolines}{%
  \renewcommand{\@oddhead}{%
    \docdate\hfill\parbox[t]{\hW}{{\hfill\bf\docname}\newline
                          Page \thepage\ of \pageref{lastpage}}}%
\renewcommand{\@evenhead}{}%
\renewcommand{\@oddfoot}{}%
\renewcommand{\@evenfoot}{}%
}%
\makeatother
\pagestyle{twolines}

\vspace{-10pt}
\begin{tabbing}
\phantom{References: }\= \\
To: \>Nathaniel, Bill, Van\\
Subject: \>Definition of path of integration\\
From: \>Van Snyder\\
%Reference: \\
\end{tabbing}

\parindent 0pt \parskip 6pt
\vspace{-20pt}

\section{By interpolating in tangent height}

If the instrument is not within the atmosphere, the entire path through
the atmosphere is integrated.  If the instrument is within the atmosphere,
only the part of the path in the instrument viewing direction should be
integrated.

If the height of the instrument is greater than the highest height in the
{\tt h_ref} array in the Full Forward Model the instrument is considered
not to be within the atmosphere.  The {\tt h_ref} array records the height
of each $\zeta$ surface for each profile, computed using the temperatures
along each profile, assuming hydrostatic equilibrium.

The vectors $\mathbf{S}_m$ to the instrument and $\mathbf{T}_m$ to the
tangent point are minor-frame quantities in ECR co\"ordinates for MIF
number $m$. The orthonormal matrix $\mathbf{E}_m$ is a minor-frame
quantity that gives the rotation from ECR to FOV co\"ordinates.  The third
column of $\mathbf{E}_m$, denoted $\mathbf{E}_{3m}$, is an unit vector in
the direction of the line-of-sight of the instrument, in ECR
co\"ordinates.  Each line of sight can be represented parametrically as
the line whose loci are given by $\mathcal{L}_m(s) = \mathbf{S}_m +
s\,\mathbf{E}_{3m}$.

Assume there are several MIFs, and therefore there is some MIF number $r$
such that $|\mathbf{T}_r| \neq 0$.  The number $r$ might be specified in
the {\tt l2cf}, or if $|\mathbf{T}_r| = 0$ it might be re-calculated
according to some to-be-specified algorithm.

\begin{floatingfigure}{3.1in}
\includegraphics[width=3in]{wvs-143-1}
\end{floatingfigure}
\vspace*{0.1in}

Compute $\mathbf{N}_r = \mathbf{E}_{3r} \times \mathbf{T}_r$, the vector
normal to the plane that would contain all field-of-view vectors if the
instrument moves and scans in that plane.

Compute a vector $\mathbf{T}_g$ to the Earth's surface at the longitude
and geodetic latitude of $\mathbf{T}_r$, in ECR co\"ordinates, and a
vector $\mathbf{V}_r = \mathbf{T}_r - \mathbf{T}_g$, normal to the Earth's
surface at $\mathbf{T}_g$ (every point on $\mathbf{T}_g + s\,
\mathbf{V}_r$ has the same longitude and geodetic latitude).  Compute an
unit vector $\hat{\mathbf{V}}_r = \mathbf{V}_r / |\mathbf{V}_r|$.  For
each $m$ there is a vector $\mathbf{V}_m = \mathbf{T}_m - \mathbf{T}_g$,
and a geodetic tangent height $\hat{h}_m = |\mathbf{V}_m|$ if
$\mathbf{T}_m \cdot \mathbf{V}_m > 0$ or $\hat{h}_m = -|\mathbf{V}_m|$
otherwise.

Compute a set of numbers $\{h_i\}$ where $h_i \geq 0$ are geodetic heights
corresponding to $\{\zeta_i\}$, computed from hydrostatic equilibrium
using temperatures on a geodetic profile at $\mathbf{T}_g$, and $h_i<0$
are arbitrary pre-specified heights.  Use only the subset $\{h_i | h_i
\geq \hat{h}_m\}$.

Compute vectors $\{\mathbf{T}_i = \mathbf{T}_g + h_i \hat{\mathbf{V}}_r
\}$.  Represent the points on path $i$ of integration parametrically as
the loci of points on the line given by $\mathcal{L}_i(s) = \mathbf{C}_i +
s\,\mathbf{U}_i$, where $\mathbf{U}_i = \frac{\mathbf{N}_r \times
\mathbf{T}_i}{|\mathbf{N}_r \times \mathbf{T}_i|}$, and $\mathbf{C}_i =
\mathbf{T}_i$ if $|\mathbf{T}_i| \geq |\mathbf{T}_g|$.  If $|\mathbf{T}_i|
< |\mathbf{T}_g|$ the path of integration has two segments, one with
$\mathbf{U}_i$ computed as above and the other its reflection, and for
both segments $\mathbf{C}_i$ is the point where $\mathcal{L}_i(s)$
intersects the Earth's surface.  This is equivalent to an assumption that
the path and velocity of the instrument, and the scan rate, are such that
path $i$ of integration is orthogonal to $\mathbf{T}_i$.  If $\mathbf{U}_i
\cdot \mathbf{E}_{3r} > 0$, replace $\mathbf{U}_i$ with $-\mathbf{U}_i$,
that is, $\mathbf{U}_i$ is a vector directed toward the instrument.

The $i^\text{th}$ path of integration ends at the point on the path of
integration that is nearest to $\mathbf{S}_m$, where $m$ is either $r$ or
the point for which $\mathbf{T}_m$ is nearest to $\mathbf{T}_i$, or
$|\mathbf{T}_m - \mathbf{T}_g|$ is nearest to $h_i$.  Once $m$ is chosen,
this position can be computed using the method described in \h{wvs-142}.

The $i^\text{th}$ path of integration begins on $\mathcal{L}_i(s)$ at the
point nearest to $-\infty \mathbf{U}_i$.  If the path intersects the
Earth's surface, this might not be on the segment nearest to
$\mathbf{S}_m$.

\section{By interpolating angles}


\begin{floatingfigure}{3.1in}
\includegraphics[width=3in]{wvs-143-2}
\end{floatingfigure}
\vspace*{0.1in}

Let $\mathbf{T}_1$ and $\mathbf{T}_2$, and $\mathbf{S}_1$ and
$\mathbf{S}_2$, be tangent and instrument positions, respectively,  and let
$\mathbf{L}_1$ and $\mathbf{L}_2$ be line-of-sight unit vectors, in ECR
co\"ord\-in\-ates at consecutive MIF numbers $m_1$ and $m_2$.  Assume all
vectors are in a single plane with normal $\mathbf{N} = \mathbf{S}_1
\times \mathbf{L}_1$.

Let $\mathbf{A}_T$, $\mathbf{A}_S$ and $\mathbf{A}_L$ be angles between
$\mathbf{T}_1$ and $\mathbf{T}_2$, $\mathbf{S}_1$ and $\mathbf{S}_2$, and
$\mathbf{L}_1$ and $\mathbf{L}_2$, respectively.  Let $\zeta_1$ and
$\zeta_2$ be tangent pressures at those MIF numbers.  Let $\zeta$ be the
desired tangent pressure.  Let $\mathbf{T}_\zeta$, $\mathbf{S}_\zeta$, and
$\mathbf{L}_\zeta$ be the tangent, instrument position, and line-of-sight
vectors for the desired $\zeta$.

Compute an interplating coefficient $\eta = ( \zeta_2 - \zeta ) / (
\zeta_2 - \zeta_1)$.  Rotate the vectors $\mathbf{T}_1$, $\mathbf{S}_1$,
and $\mathbf{L}_1$ about $\mathbf{N}$ toward $\mathbf{T}_2$,
$\mathbf{S}_2$, and $\mathbf{L}_2$ by angles $\eta\,\mathbf{A}_T$,
$\eta\,\mathbf{A}_S$, and $\eta\,\mathbf{A}_L$, giving $\mathbf{T}_\zeta$,
$\mathbf{S}_\zeta$, and $\mathbf{L}_\zeta$, respectively, as described in
\h{wvs-122}.  Scale the lengths of the vectors $\mathbf{T}_\zeta$,
$\mathbf{S}_\zeta$, and $\mathbf{L}_\zeta$ such that $|\mathbf{T}_\zeta| =
\eta |\mathbf{T}_1| + ( 1 - \eta ) |\mathbf{T}_2|$, and
$|\mathbf{S}_\zeta| = \eta |\mathbf{S}_1| + ( 1 - \eta ) |\mathbf{S}_2|$. 
$\mathbf{L}_\zeta$ remains an unit vector.

\label{lastpage}
\vspace*{-0.1in} % Somehow, this causes lastpage to be defined
\end{document}

% $Id$

% $Log$
% Revision 1.3  2017/09/28 23:12:01  vsnyder
% Need an unit vector in the direction of V_r
%
% Revision 1.2  2017/09/26 23:08:32  vsnyder
% Almost completely revised
%
% Revision 1.1  2017/09/21 23:12:46  vsnyder
% Initial commit
%
