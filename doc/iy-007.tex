\documentclass[fleqn,10pt]{article}
\usepackage[dvips]{graphics,color}
\usepackage{amscd, amsfonts, amsmath, amssymb}
\usepackage{amsthm}
\usepackage{eucal}
\usepackage{pifont}

\usepackage{epsfig}
\usepackage{algorithm}
\usepackage{algorithmic}
\usepackage{verbatim}

\setlength{\topmargin}{-1.5cm}
\setlength{\headheight}{0.5cm}
\setlength{\headsep}{0.7cm}
\setlength{\topskip}{0.5cm}
\setlength{\textheight}{23.0cm}
\setlength{\footskip}{0.5cm}
\setlength{\oddsidemargin}{0.5cm}
\setlength{\evensidemargin}{0.5cm}
\setlength{\textwidth}{14.5cm}

% For pasting into PPT presentation: 
%\setlength{\textwidth}{10.5cm}




\setcounter{tocdepth}{5}



%\numberwithin{equation}{section}

%\pagestyle{headings}
%\markboth{Igor}{Yanovsky}



\def\pdf{p}
\def\vx{{\bf x}}
\def\vy{{\bf y}}
\def\vg{{\bf g}}
\def\g{g}
\def\vh{{\bf h}}
\def\vf{{\bf f}}
\def\vF{{\bf F}}
\def\vu{{\bf u}}
\def\vv{{\bf v}}
\def\vn{{\bf n}}
\def\vmu{\boldsymbol \mu}
\def\vnabla{{\bf \nabla}}
\def\vid{{\bf id}}
\def\va{{\bf a}}
\def\vr{{\bf r}}
\def\vb{{\bf b}}

\def\ff{\mbox f}

\def\D{\mathcal{D}}
\def\E{\mathcal{E}}

\def\trace{\mbox{trace}}
\def\divergence{\mbox{div}}

\def\I{\mathcal{I}}
\def\R{\mathcal{R}}
\def\S{\mathcal{S}}
\def\T{\mathcal{T}}


\def\m{\mbox{m}}
\def\km{\mbox{km}}
\def\nm{\mbox{nm}}
\def\cm{\mbox{cm}}
\def\s{\mbox{s}}
\def\kg{\mbox{kg}}
\def\K{\mbox{K}}
\def\N{\mbox{N}}
\def\J{\mbox{J}}
\def\Pa{\mbox{Pa}}
\def\hPa{\mbox{hPa}}
\def\Hz{\mbox{Hz}}
\def\MHz{\mbox{MHz}}
\def\AMU{\mbox{AMU}}

\def\NH{\mbox{NH}}
\def\NP{\mbox{NP}}

\def\insta{\mathfrak{M}}
\def\instb{\mathfrak{T}}


\hyphenation{atmosphere}
\hyphenation{considered}
\hyphenation{defined}
\hyphenation{Jacobian}
\hyphenation{thermodynamic}
\hyphenation{vector}


\begin{document}

\title{Radiative Transfer: Deriving Integral Equation from Differential Equation}
\author{Igor Yanovsky \\
iy-007}

\date{November 9, 2010}

\maketitle


The radiative transfer equation for a nonscattering atmosphere in local thermodynamic equilibrium is
\begin{eqnarray*}
\frac{\partial I (\nu, \vx(s))}{\partial s} + \alpha (\nu, \vx(s)) I(\nu, \vx(s)) &=& \alpha(\nu,\vx(s)) B(\nu,\vx(s)),
\end{eqnarray*}
where $s$ is the distance coordinate along the path, $\vx$ is the state vector, $I(\nu,\vx(s))$ is the radiance at frequency $\nu$ at position $s$, $\alpha(\nu,\vx(s))$ is an absorption coefficient, and $B(\nu,\vx(s))$ is the Planck radiation function. \\ 
Consider a case with a single frequency: 
\begin{equation}
\frac{\partial I (s)}{\partial s} + \alpha (s) I(s) \ = \ \alpha(s) B(s).
\label{eqn:differential_RT_single_nu}
\end{equation}
Given $s = s_0$ and $s = s_\insta$ are path locations at the top of the atmosphere and at the satellite, respectively, the transmission is defined as
\begin{equation}
\T(s) \ = \ e^{-\int_{s}^{s_\insta} \alpha(x) dx}.
\label{eqn:tau}
\end{equation}
Note:
\begin{eqnarray*}
\T(s_0) &=& e^{-\int_{s_0}^{s_\insta} \alpha(x) dx}, \\
\T(s_\insta) &=& 1.
\end{eqnarray*}
Multiply Equation (\ref{eqn:differential_RT_single_nu}) by $e^{\int_{s_0}^{s} \alpha(x) dx}$:
\begin{eqnarray}
e^{\int_{s_0}^{s} \alpha(x) dx} \bigg( \frac{\partial I (s)}{\partial s} + \alpha (s) I(s) \bigg) &=& e^{\int_{s_0}^{s} \alpha(x) dx} \alpha(s) B(s),
\nonumber \\
\Big( e^{\int_{s_0}^{s} \alpha(x) dx} I (s) \Big)' &=& e^{\int_{s_0}^{s} \alpha(x) dx} \alpha(s) B(s),
\nonumber \\
e^{\int_{s_0}^{s} \alpha(x) dx} I (s) - I(s_0) &=& \int_{s_0}^{s} e^{\int_{s_0}^{s'} \alpha(x) dx} \alpha(s') B(s') ds',
\nonumber
\end{eqnarray}
where $s_0 < s' < s$.  From this, we have
\begin{eqnarray*}
I(s) &=& I(s_0) e^{-\int_{s_0}^{s} \alpha(x) dx} + e^{-\int_{s_0}^{s} \alpha(x) dx} \int_{s_0}^{s} e^{\int_{s_0}^{s'} \alpha(x) dx} \alpha(s') B(s') ds',
\end{eqnarray*}
or
\begin{eqnarray*}
I(s) &=& I(s_0) e^{-\int_{s_0}^{s} \alpha(x) dx} + \int_{s_0}^{s} e^{-\int_{s'}^{s} \alpha(x) dx} \alpha(s') B(s') ds'.
\end{eqnarray*}
When $s=s_{\insta}$, we have
\begin{eqnarray*}
I(s_{\insta}) &=& I(s_0) e^{-\int_{s_0}^{s_{\insta}} \alpha(x) dx} + \int_{s_0}^{s_{\insta}} e^{-\int_{s'}^{s_{\insta}} \alpha(x) dx} \alpha(s') B(s') ds',
\end{eqnarray*}
or
\begin{equation}
I(s_{\insta}) = I(s_0) \T(s_0) + \int_{s_0}^{s_{\insta}} \T(s') \alpha(s') B(s') ds'.
\label{eqn:RT_integral1}
\end{equation}
From equation (\ref{eqn:tau}):
\begin{equation*}
\frac{d\T}{ds} = \alpha(s)\T(s).
\end{equation*}
Thus, we have
\begin{equation}
I(s_{\insta}) = I(s_0) \T(s_0) + \int_{\T(s_0)}^{1} B(s) \, d\T(s).
\label{eqn:RT_integral2}
\end{equation}
Integrating by parts, we get
\begin{equation*}
I(s_{\insta}) \ = \ I(s_0) \T(s_0) \, + \, \Big[ B(s)\T(s) \Big]_{s=s_0}^{s=s_{\insta}}  \, - \, \int_{B(s_0)}^{B(s_{\insta})} \T(s) \, dB(s),
\end{equation*}
or
\begin{equation}
I(s_{\insta}) \ = \ \big( I(s_0) - B(s_0) \big) \T(s_0) \, + \, B(s_{\insta}) \, - \, \int_{B(s_0)}^{B(s_{\insta})} \T(s) \, dB(s).
\label{eqn:RT_integral3}
\end{equation}
\ \\
\ \\
\ \\
{\it Remark:}  No emission assumption results in the following differential equation:
\begin{equation*}
\frac{\partial I (s)}{\partial s} + \alpha (s) I(s) = 0,
\end{equation*}
which has the following solution:
\begin{equation*}
I (s_\insta) = I(s_0)\T(s_0).
\end{equation*}



%\bibliographystyle{ieee}
%\bibliography{/Users/yanovsky/Igor/papers/refslib/yanovsky}


\end{document}

% $Id$