\documentclass[11pt,twoside]{article}
\usepackage[fleqn]{amsmath}\textwidth 6.25in
\oddsidemargin -0.25in
\evensidemargin -0.25in
\topmargin -0.5in
\textheight 9.00in

\begin{document}

%\tracingcommands=1
\newlength{\hW} % heading box width
\settowidth{\hW}{\bf wvs-016r3}
%\settowidth{\hW}{Page \pageref{lastpage}\ of \pageref{lastpage}}
\makeatletter
\def\@biblabel#1{#1.}
\newcommand{\ps@twolines}{%
  \renewcommand{\@oddhead}{%
    21 May 2019\hfill\parbox[t]{\hW}{{\bf wvs-016r3}\newline
                          Page \thepage\ of \pageref{lastpage}}}%
\renewcommand{\@evenhead}{}%
\renewcommand{\@oddfoot}{}%
\renewcommand{\@evenfoot}{}%
}%
\makeatother
\pagestyle{twolines}

\vspace{-10pt}
\begin{tabbing}
\phantom{References: }\= \\
To: \>Mike, Bill, Nathaniel\\
Subject: \>Alternative formulation for polarized radiance derivatives\\
From: \>Fred Krogh, by way of Van Snyder\\
\end{tabbing}

\parindent 0pt \parskip 3pt
\vspace{-20pt}

For polarized radiative transfer, we want to compute

\begin{equation}\label{begin}
 {\bf I} = \sum_{i=1}^n \mathbf{\tau}_i \Delta B_i,
 \text{ where }
 \mathbf{\tau}_i = \mathbf{P}_i \mathbf{P}_i^\dagger,~
 \mathbf{P}_i =  \prod_{k=2}^i \mathbf{E}_k \text{ and }
 \mathbf{E}_i =  \exp \left( - \int_{s_i}^{s_{i-1}} \mathbf{G} (s^\prime)
               \text{d} s^\prime \right )
\end{equation}

and its derivative

\begin{equation}\label{deriv}
  \frac{\partial \bf I}{\partial x} = \sum_{i=1}^n
   \left [ \frac{\partial {\bf \tau}_i}{\partial x} \Delta B_i +
           {\bf \tau}_i \frac{\partial \Delta B_i}{\partial x}\right ],
   \text{ where from equation (\ref{begin}) }
  \frac{\partial {\bf \tau}_i}{\partial x} =
   \frac{\partial {\bf P}_i {\bf P}_i^\dagger}{\partial x} \text{ .}
\end{equation}

$\mathbf{E}$ is a layer quantity.  $\mathbf{E}_1$ is the incremental
transmissivity from the spacecraft to the top of the atmosphere, which is
identity.  Therefore, we start with $\mathbf{E}_2$.

Applying Laplace's differentiation rule to $\mathbf{P}_i = \prod_{k=2}^i
{\bf E}_k$, we had written

\begin{equation}\label{mess}
 \begin{split}
  \frac{\partial\mathbf{\tau}_i}{\partial x}
  =\, & \sum_{k=2}^i \left [ \mathbf{E}_2 \dots \mathbf{E}_{k-1}
       \frac{\partial\mathbf{E}_k}{\partial x} \mathbf{E}_{k+1} \dots
       {\bf E}_i {\bf P}_i^\dagger +
       {\bf P}_i {\bf E}_i^\dagger \dots {\bf E}_{k+1}
       \frac{\partial\mathbf{E}_k^\dagger}{\partial x}
       \mathbf{E}_{k-1}^\dagger \dots \mathbf{E}_2^\dagger \right ]
\\
  =\, & \sum_{k=2}^i \left [ {\bf P}_{k-1}
       \frac{\partial\mathbf{E}_k}{\partial x} \mathbf{P}_k^{-1} {\bf \tau}_i
     + \mathbf{\tau}_i^\dagger \mathbf{P}_k^{-\dagger}
       \frac{\partial\mathbf{E}_k^\dagger}{\partial x} \mathbf{P}_{k-1}^\dagger
       \right ] \text{ .} \\
 \end{split}
\end{equation}

If we instead write $\mathbf{P}_i = \mathbf{P}_{i-1} \mathbf{E}_i$, with
$\mathbf{P}_1 = \mathbf{1}$ (we use $\mathbf{1}$ for the identity matrix
because $\mathbf{I}$ is the radiance), we have

\begin{equation}\label{dP}
 \frac{\partial\mathbf{P}_i}{\partial x} =
  \frac{\partial\mathbf{P}_{i-1}}{\partial x} \mathbf{E}_i +
   \mathbf{P}_{i-1} \frac{\partial\mathbf{E}_i}{\partial x}
\end{equation}

(keep in mind that $\mathbf{P}_1 = \mathbf{1}$ so
$\frac{\partial\mathbf{P}_1}{\partial x} = \mathbf{0}$).  Differentiating
$\mathbf{\tau}_i$ without expanding $\mathbf{P}_i$ gives

\begin{equation}\label{final}
 \frac{\partial\mathbf{\tau}_i}{\partial x} =
  \frac{\partial\mathbf{P}_i\mathbf{P}_i^\dagger}{\partial x} =
  \frac{\partial\mathbf{P}_i}{\partial x} \mathbf{P}_i^\dagger +
  \mathbf{P}_i \frac{\partial\mathbf{P}_i^\dagger}{\partial x} =
  \frac{\partial\mathbf{P}_i}{\partial x} \mathbf{P}_i^\dagger +
  \mathbf{P}_i \left(\frac{\partial\mathbf{P}_i}{\partial x}\right)^\dagger =
  \frac{\partial\mathbf{P}_i}{\partial x} \mathbf{P}_i^\dagger +
  \left(\frac{\partial\mathbf{P}_i}{\partial x} \mathbf{P}_i^\dagger\right)^\dagger
  \text{ .}
\end{equation}

This is a much simpler expression than equation (\ref{mess}), and doesn't
involve calculation of $\mathbf{P}_k^{-1}$.  There is also some speculation
that equation (\ref{mess}) is unstable because of the $\mathbf{P}_k^{-1}$;
using equations (\ref{dP}) and (\ref{final}) avoids this question altogether.

As expected, $\frac{\partial\tau_i}{\partial x}$ is Hermitian, so we only
need to compute the real parts of the diagonal elements, and one off-diagonal
element.

\label{lastpage}
\end{document}
% $Id$

% $Log$
% Revision 1.2  2019/05/21 18:24:18  vsnyder
% Repair a typo in Equation (3)
%
