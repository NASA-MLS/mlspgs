% To make the graphic
% tgif -print -epsi wvs-154-1.obj; epstopdf wvs-154-1.eps
% tgif -print -jpeg wvs-154-1.obj

\documentclass[11pt]{article}
\usepackage{alltt}
\usepackage[fleqn]{amsmath}
\usepackage{floatflt}
\usepackage{graphicx}
\usepackage{longtable}
\usepackage[strings]{underscore}

\textwidth 6.5in
\oddsidemargin -0.25in
%\evensidemargin -0.5in
\topmargin -0.5in
\textheight 9in

\newcommand{\docname}{wvs-154r3}
\newcommand{\docdate}{16 February 2020}

\ifx\pdfoutput\undefined
  \pdfoutput=0
  \usepackage[hypertex,plainpages,hyperindex=true]{hyperref}
  \hypersetup{%
    hypertexnames=false%
  }
  % Specify the driver for the color package
  \ExecuteOptions{dvips}
  %\ExecuteOptions{xdvi}
\else
  \ifnum\pdfoutput>0
   
\usepackage[pdftex,plainpages,hyperindex=true,pdfpagelabels]{hyperref}
    \hypersetup{%
      hypertexnames=false,%
      colorlinks=true,%
      linktocpage=true,%
    }
    % Specify the driver for the color package
    \ExecuteOptions{pdftex}
  \else
    \usepackage[hypertex,plainpages,hyperindex=true]{hyperref}
    \hypersetup{%
      hypertexnames=false%
    }
    % Specify the driver for the color package
    \ExecuteOptions{dvips}
    %\ExecuteOptions{xdvi}
  \fi
\fi

\hyperbaseurl{}
\newcommand\hr[1]{\href{#1.dvi}{dvi}, \href{#1.pdf}{pdf}}
\newcommand\h[1]{#1 (\hr{#1})}

\renewcommand\d{\text{d}}
\newcommand\T{\mathcal{T}}

\begin{document}

%\tracingcommands=1
\newlength{\hW} % heading box width
\newlength{\pW} % page number field width
\settowidth{\hW}{\bf\docname}
\settowidth{\pW}{Page \pageref{lastpage}\ of \pageref{lastpage}}
\ifdim \pW > \hW \setlength{\hW}{\pW} \fi
\makeatletter
\def\@biblabel#1{#1.}
\newcommand{\ps@twolines}{%
  \renewcommand{\@oddhead}{%
    \docdate\hfill\parbox[t]{\hW}{{\hfill\bf\docname}\newline
                          Page \thepage\ of \pageref{lastpage}}}%
\renewcommand{\@evenhead}{}%
\renewcommand{\@oddfoot}{}%
\renewcommand{\@evenfoot}{}%
}%
\makeatother
\pagestyle{twolines}

\vspace{-10pt}
\begin{tabbing}
\phantom{References: }\= \\
To: \>MLS group\\
Subject: \>Angle between two vectors\\
From: \>Van Snyder\\
%Reference: \\
\end{tabbing}

\parindent 0pt \parskip 6pt

The formula usually used to compute the angle between two vectors $x$ and
$y$ is

\begin{equation}\label{one}
\theta = \cos^{-1} z \text{ where }
 z = \frac{ x \cdot y }{ ||x|| \,\, ||y|| } \,.
\end{equation}

In general, to first order, the error in computing a function $f(z)$ is
$\delta f(z) \approx \delta z f^\prime (z)$.

The derivative of $\cos^{-1} z$ is

\begin{equation*}
\frac{ \d \, \cos^{-1} z }{ \d z } = \frac{-1}{\sqrt{1-z^2}} \,,
\end{equation*}

which becomes singular as $|z| \rightarrow 1$, i.e., when $x$ and $y$ are
nearly parallel or nearly antiparallel, and therefore $|\delta \theta|$
becomes arbitrarily large.

In three dimensions, one can use

\begin{equation}\label{two}
\theta = \left\{
 \begin{array}{ll}
  \sin^{-1} \left( \frac{ || x \times y || }{ || x || \, || y || }
            \right) & x \cdot y \geq 0 \\
  \pi - \sin^{-1} \left( \frac{ || x \times y || }{ || x || \, || y || }
                 \right) & x \cdot y < 0
 \end{array} \right.
\end{equation}

but this has the same sort of singularity when $x$ and $y$ are nearly
orthogonal.  This also only works in three dimensions.  Analogs in higher
dimensions have increasing complexity.

For unit vectors $u$ and $v$, it is easy to verify that $u-v$ and $u+v$
are orthogonal:

\begin{equation*}
( u - v ) \cdot ( u + v ) =
 u \cdot u + u \cdot v - v \cdot u - v \cdot v = 0 \,.
\end{equation*}

From

\includegraphics{wvs-154-1}

it is easy to see that the angle $\theta$ between $u$ and $v$ is

\begin{equation*}
\theta = \left \{ \begin{array}{ll}
 2\, \tan^{-1} z & x \cdot y \geq 0 \\
 \pi - 2\, \tan^{-1} z & x \cdot y < 0 \\
 \end{array} \right.
 \text{ where } z = \left( \frac{ \frac12 || u - v || }
                                { \frac12 || u + v || } \right)
                  = \left( \frac{ || u - v || }
                                { || u + v || } \right) \,.
\end{equation*}

If each of the vectors $x$ and $y$ are scaled by the other's length, which
is equivalent to considering unit vectors parallel to $x$ and $y$ but
doesn't require dividing each by its norm (which might be zero), we have

\begin{equation}\label{three}
\theta = \left \{ \begin{array}{ll}
 2\, \tan^{-1} z & x \cdot y \geq 0 \\
 \pi - 2\, \tan^{-1} z & x \cdot y < 0 \\
 \end{array} \right.
 \text{ where }
 z = \left( \frac{ \left|\left|\frac{x}{||x||} - \frac{y}{||y||}\right|\right|}
                 { \left|\left|\frac{x}{||x||} + \frac{y}{||y||}\right|\right|} \right)
   = \left( \frac{ || \, ( x || y || - y || x || ) \,|| }
                 { || \, ( x || y || + y || x || ) \,|| }
     \right) \,.
\end{equation}

If either $||x|| = 0$ or $||y|| = 0$, $\theta = \tan^{-1} \frac00$.  In
this case, one can arbitrarily assign $\theta = 0$ or $\theta =
\frac{\pi}2$ or $\theta = \pm \pi$. To avoid trying to compute $\frac00$,
use the two-argument arctangent function. In Fortran, {\tt atan2(y,0)} is
defined to be a processor-dependent approximation to $\pm\frac{\pi}2$,
regardless of whether {\tt y} is zero.  If the processor distinguishes
between positive and negative zero, the result has the same sign as {\tt
y}.

The derivative of $\tan^{-1} z$,

\begin{equation*}
\frac{\d \tan^{-1} z}{\d z} = \frac1{z^2+1} < 1 \,,
\end{equation*}

doesn't have any singularities for real $z$.  Equation (\ref{three}) is
quite a bit more expensive than Equation (\ref{one}).

Equations (\ref{one}) and (\ref{two}) have the same error at $|z| =
\frac{\sqrt{2}}2 \approx 0.7071$, where the error is $\sqrt{2} \, \delta z
\approx 1.414 \, \delta z$.

Equations (\ref{one}) and (\ref{three}) have the same error at $|z| =
\sqrt{3 - 2\sqrt{3}} \approx 0.6813$, where the error is
$\frac1{\sqrt{4-2\sqrt{3}}} \, \delta z \approx 1.366 \, \delta z$.

For a computational method that approximately minimizes the maximum error,
and has good performance in three dimensions, first compute $z$.  In three
dimensions, use  Equation (\ref{one}) for $z \leq \frac{\sqrt{2}}2$ and
Equation (\ref{two}) otherwise.  For other dimensions, use Equation
(\ref{one}) for $z \leq \sqrt{3 - 2\sqrt{3}}$, and Equation (\ref{three})
otherwise.

For maximum performance and minimum maximum error, use Equation
(\ref{one}) for $|z| \leq \sqrt{3 - 2\sqrt{3}}$, use Equation
(\ref{three}) for $|z| > \sqrt{3 - 2\sqrt{3}}$ in more than three
dimensions, use Equation (\ref{three}) for $\sqrt{3 - 2\sqrt{3}} < |z| <
\frac{\sqrt{2}}2$ in three dimensions, and use Equation (\ref{two}) for
$|z| \geq \frac{\sqrt{2}}2$ in three dimensions.

The difference between the errors in Equation (\ref{one}) at $|z| =
\frac{\sqrt{2}}2$ and $|z| = \sqrt{3 - 2\sqrt{3}}$ is $\approx 0.048 \,
\delta z$, so the same $|z|$ could be used for selecting the method in
three dimensions or other dimensions.

In Equation (\ref{three}), the error is $\frac2{z^2+1}\, \delta z < 2\,
\delta z$. The error in Equation (\ref{one}) is equal to $2\, \delta z$
for $|z| = \frac{\sqrt{3}}2$.  For higher average performance, Equation
(\ref{one}) could be used for $|z| \leq \frac{\sqrt{3}}2 \approx 0.866$,
if errors as large as $2\, \delta z$ are acceptable.

\vspace*{-1pt} \label{lastpage} \end{document}

% $Id$

% $Log$
% Revision 1.4  2019/12/10 23:27:59  vsnyder
% Correct a typo
%
% Revision 1.3  2019/09/18 22:28:32  vsnyder
% Add a graphic, explain derivations
%
% Revision 1.2  2019/09/17 03:20:26  vsnyder
% Explain error calculations
%
% Revision 1.1  2019/09/12 02:51:53  vsnyder
% Initial commit
%
