\documentclass[11pt]{article}
\usepackage{alltt}
\usepackage[fleqn]{amsmath}
\usepackage{floatflt}
\usepackage{graphicx}
\usepackage{longtable}
\usepackage[strings]{underscore}

\textwidth 6.5in
\oddsidemargin -0.25in
%\evensidemargin -0.5in
\topmargin -0.5in
\textheight 9in

\newcommand{\docname}{wvs-154}
\newcommand{\docdate}{11 September 2019}

\ifx\pdfoutput\undefined
  \pdfoutput=0
  \usepackage[hypertex,plainpages,hyperindex=true]{hyperref}
  \hypersetup{%
    hypertexnames=false%
  }
  % Specify the driver for the color package
  \ExecuteOptions{dvips}
  %\ExecuteOptions{xdvi}
\else
  \ifnum\pdfoutput>0
   
\usepackage[pdftex,plainpages,hyperindex=true,pdfpagelabels]{hyperref}
    \hypersetup{%
      hypertexnames=false,%
      colorlinks=true,%
      linktocpage=true,%
    }
    % Specify the driver for the color package
    \ExecuteOptions{pdftex}
  \else
    \usepackage[hypertex,plainpages,hyperindex=true]{hyperref}
    \hypersetup{%
      hypertexnames=false%
    }
    % Specify the driver for the color package
    \ExecuteOptions{dvips}
    %\ExecuteOptions{xdvi}
  \fi
\fi

\hyperbaseurl{}
\newcommand\hr[1]{\href{#1.dvi}{dvi}, \href{#1.pdf}{pdf}}
\newcommand\h[1]{#1 (\hr{#1})}

\renewcommand\d{\text{d}}
\newcommand\T{\mathcal{T}}

\begin{document}

%\tracingcommands=1
\newlength{\hW} % heading box width
\newlength{\pW} % page number field width
\settowidth{\hW}{\bf\docname}
\settowidth{\pW}{Page \pageref{lastpage}\ of \pageref{lastpage}}
\ifdim \pW > \hW \setlength{\hW}{\pW} \fi
\makeatletter
\def\@biblabel#1{#1.}
\newcommand{\ps@twolines}{%
  \renewcommand{\@oddhead}{%
    \docdate\hfill\parbox[t]{\hW}{{\hfill\bf\docname}\newline
                          Page \thepage\ of \pageref{lastpage}}}%
\renewcommand{\@evenhead}{}%
\renewcommand{\@oddfoot}{}%
\renewcommand{\@evenfoot}{}%
}%
\makeatother
\pagestyle{twolines}

\vspace{-10pt}
\begin{tabbing}
\phantom{References: }\= \\
To: \>MLS group\\
Subject: \>Angle between two vectors\\
From: \>Van Snyder\\
%Reference: \\
\end{tabbing}

\parindent 0pt \parskip 6pt

The formula usually used to compute the angle between two vectors $x$ and
$y$ is

\begin{equation}\label{one}
\theta = \cos^{-1} z \text{ where }
 z = \frac{ x \cdot y }{ ||x|| \, ||y|| } \,.
\end{equation}

This can be severely inaccurate if $x$ and $y$ are nearly parallel or
antiparallel because

\begin{equation*}
\frac{ \d \, \cos^{-1} z }{ \d z } = \frac{-1}{\sqrt{1-z^2}}
\end{equation*}

becomes singular as $|z| \rightarrow 1$, i.e., when $x$ and $y$ are nearly
parallel or nearly antiparallel.

In three dimensions, one can use

\begin{equation*}
\theta = \left\{
 \begin{array}{ll}
  \sin^{-1} \left( \frac{ || x \times y || }{ || x || \, || y || }
            \right) & x \cdot y \geq 0 \\
  \pi - \sin^{-1} \left( \frac{ || x \times y || }{ || x || \, || y || }
                 \right) & x \cdot y < 0
 \end{array} \right.
\end{equation*}

but this has the same sort of singularity when $x$ and $y$ are nearly
orthogonal.  This also only works in three dimensions.  Analogs in higher
dimensions have increasing complexity.

For unit vectors $u$ and $v$, it is easy to verify that $u-v$ and $u+v$
are orthogonal:

\begin{equation*}
( u - v ) \cdot ( u + v ) =
 u \cdot u + u \cdot v - v \cdot u - v \cdot v = 0 \,,
\end{equation*}

and therefore the angle between $u$ and $v$ is

\begin{equation*}
\theta = 2 \tan^{-1} \left( \frac{ || u - v || } { || u + v || } \right) \,.
\end{equation*}

If each of the vectors $x$ and $y$ are scaled by the other's length, which
is equivalent to considering unit vectors parallel to $x$ and $y$ but
doesn't require dividing, we have

\begin{equation}\label{two}
\theta = 2 \tan^{-1} \left( \frac{ || \, ( x || y || - y || x || ) \,|| }
                                 { || \, ( x || y || + y || x || ) \,|| }
                     \right) \,,
\end{equation}

which doesn't have any singularities.  This is quite a bit more expensive
than Equation (\ref{one}).  Instead, one might first compute $z$, then use
Equation (\ref{two}) if $ |z| \approx 1 $ and Equation (\ref{one})
otherwise.

\vspace*{-1pt} \label{lastpage} \end{document}

% $Id$

% $Log$
