\documentclass[11pt]{article}
\usepackage[fleqn]{amsmath}

\textwidth 6.5in
\oddsidemargin -0.25in
%\evensidemargin -0.5in
\topmargin -0.25in
\textheight 9.25in

\newcommand{\docname}{\bf wvs-067r5}
\newcommand{\docdate}{28 May 2014}

\ifx\pdfoutput\undefined
  \pdfoutput=0
  \usepackage[hypertex,plainpages,hyperindex=true]{hyperref}
  \hypersetup{%
    hypertexnames=false%
  }
  % Specify the driver for the color package
  \ExecuteOptions{dvips}
  %\ExecuteOptions{xdvi}
\else
  \ifnum\pdfoutput>0
    \usepackage[pdftex,plainpages,hyperindex=true,pdfpagelabels]{hyperref}
    \hypersetup{%
      hypertexnames=false,%
      colorlinks=true,%
      linktocpage=true,%
    }
    % Specify the driver for the color package
    \ExecuteOptions{pdftex}
  \else
    \usepackage[hypertex,plainpages,hyperindex=true]{hyperref}
    \hypersetup{%
      hypertexnames=false%
    }
    % Specify the driver for the color package
    \ExecuteOptions{dvips}
    %\ExecuteOptions{xdvi}
  \fi
\fi

\hyperbaseurl{}
\renewcommand{\d}{\text{d}}
\ifx\dvidir\undefined
  \newcommand\hr[1]{\href{#1.dvi}{dvi} \href{#1.pdf}{pdf}}
\else
  \newcommand\hr[1]{\href{\dvidir/#1.dvi}{dvi} \href{\pdfdir/#1.pdf}{pdf}}
\fi
\newcommand\h[1]{#1 \hr{#1}}
\newcommand\hh[2]{#1 \hr{#2}}
\newcommand\hrpdf[1]{\href{#1.pdf}{pdf}}
\newcommand\hpdf[1]{#1 \hrpdf{#1}}

\begin{document}

%\tracingcommands=1
\newlength{\hW} % heading box width
\newlength{\pW} % page number field width
\settowidth{\hW}{\docname}
\settowidth{\pW}{Page \pageref{lastpage}\ of \pageref{lastpage}}
\ifdim \pW > \hW \setlength{\hW}{\pW} \fi
\makeatletter
\def\@biblabel#1{#1.}
\newcommand{\ps@twolines}{%
  \renewcommand{\@oddhead}{%
    \docdate\hfill\parbox[t]{\hW}{{\hfill\docname}\newline
                          Page \thepage\ of \pageref{lastpage}}}%
\renewcommand{\@evenhead}{}%
\renewcommand{\@oddfoot}{}%
\renewcommand{\@evenfoot}{}%
}%
\makeatother
\pagestyle{twolines}

\vspace{-10pt}
\begin{tabbing}
\phantom{References: }\= \\
To: \>Van\\
Subject: \>Summary of Mie formulae\\
From: \>Van Snyder\\
Reference: \>\h{wvs-058}, \h{wvs-070}
\end{tabbing}

\parindent 0pt \parskip 6pt
\vspace{-10pt}

\section{Mie Efficiencies}

Extinction and scattering cross sections:

\begin{equation}\begin{split}
\beta_{c\_e} =\,&\, \pi \int_0^\infty r^2 n(r)\, \xi_e(r) \text{d} r \\
\beta_{c\_s} =\,&\, \pi \int_0^\infty r^2 n(r)\, \xi_s(r) \text{d} r \\
\end{split}\end{equation}
%
Particle number distribution as a function of particle size $r$ ($\mu$m)
and $T$ (Celsius)\footnote{Equation (5.3) in the ATBD has an exponent of
1, not 3, for $D_0$ in the denominator of the expression for $N_2$.  In
Equation (4) in Greg M. McFarquhar, Andrew J. Heymsfield,
\emph{Parameterization of Tropical Cirrus Ice Crystal Size Distributions
and Implications for Radiative Transfer: Results from CEPEX}, {\bf
Journal of the Atmospheric Sciences 54}, 1 (Sep 1997) pp.\ 2187-2200 (MH
hereafter), the exponent is 3.  Equation (5.6) in the ATBD has +0.0494 in
the expression for $\alpha$.  Equation (6) in MH has $-$0.0494.  Tables 1
and 2 on page 2193 in MH give coefficients as a function of degrees
Celsius.}:
%
\begin{equation}\begin{split}
n(r) =\,& N_1\, r \exp(-2 \alpha r) +
      N_2\, \frac1r\, \exp\left(-\frac12 \gamma^2\right) \text{ where }
\gamma = \frac{\log(2r) - \mu}{\sigma} \\
I_< =\,& \min\left(I,\, 0.252\, I^{0.837}\right) \text{ and }
I_> = I - I_< \\
\alpha =\,& \max\left(0,\,-4.99 \times 10^{-3} - 0.0494 \log_{10} I_< \right) \\
\mu =\,& 5.2 + 1.3 \times 10^{-3}\, T +
     ( 0.026 - 1.2 \times 10^{-3}\, T ) \log_{10} I_> \\
\sigma =\,& 0.47 + 2.1 \times 10^{-3}\, T +
     ( 0.018 - 2.1 \times 10^{-4}\, T ) \log_{10} I_> \\
N_1 =\,& \frac{I_< \alpha^5}{4 \pi \rho} \text{ and }
N_2 = \frac{6 I_>}{\sqrt{2 \pi^3}\, D_0^3\, \rho_{\text{ice}}\, \sigma\,
                   \exp(3 \mu + 4.5 \sigma^2)}
\end{split}\end{equation}
%
Extinction and scattering efficiency factors:
%
\begin{equation}\begin{split}
\xi_e(r) =\,& \frac2{\chi^2} \sum_{n=1}^\infty ( 2 n + 1 ) ( \Re a_n + \Re b_n ) \\
\xi_s(r) =\,& \frac2{\chi^2} \sum_{n=1}^\infty ( 2 n + 1 ) ( |a_n|^2 + |b_n|^2 ) \\
\end{split}\end{equation}
%
where $\chi = \frac{2 \pi r}\lambda$,
%
\begin{equation}
A_n = -\frac{n}{m \chi} + \frac{j_{n-1}(m \chi)}{j_n(m \chi)}
%     =  \frac1{2m\chi} + \frac1{j_n(m\chi)}
%              \frac{\partial j_n(m\chi)}{\partial (m\chi)}
    = -\frac{n}{m \chi} + \left( \frac{n}{m \chi} - A_{n-1} \right)^{-1}
    \!\!\!\!\!,\,\,\,
    A_0 = \cot{m \chi}
\end{equation}
%
\begin{equation}\begin{split}
\frac1{a_n} =\,& \frac{\hat a_n h^{(2)}_n(\chi) - h^{(2)}_{n-1}(\chi)}
                      {\hat a_n j_n(\chi) - j_{n-1}(\chi)}
            =    1 - i \frac{\left(n(1-m^2) j_n(m \chi) -
                                   m \chi j_{n-1} (m \chi) \right) y_n(\chi) +
                              m^2 \chi j_n(m \chi) y_{n-1}(\chi)}
                            {\left(n(1-m^2) j_n (m \chi) +
                                   m \chi j_{n-1}(m \chi) \right) j_n(\chi) +
                              m^2 \chi j_n(m \chi) j_{n-1}(\chi)}
\\
\frac1{b_n} =\,& \frac{\hat b_n h^{(2)}_n(\chi) - h^{(2)}_{n-1}(\chi)}
                      {\hat b_n j_n(\chi) - j_{n-1}(\chi)}
            =    1 - i \frac{m\, y_n(\chi) j_{n-1}(m \chi) -
                             y_{n-1}(\chi) j_n(m \chi)}
                            {m\, j_n(\chi) j_{n-1}(m \chi) -
                             j_{n-1}(\chi) j_n(m \chi)}
            =    1 - i \frac{W(y_n(\chi),j_n(m \chi))}
                            {W(j_n(\chi),j_n(m \chi))}
%             =    1 + i \frac{\frac{\partial}{\partial\chi}
%                              \frac{y_n(\chi)}{j_n(m\chi)}}
%                             {\frac{\partial}{\partial\chi}
%                              \frac{j_n(\chi)}{j_n(m\chi)}}
\\
\hat b_n =\,& m A_n + \frac{n}\chi = m \frac{j_{n-1}(m \chi)}{j_n(m \chi)} \\
\hat a_n =\,& \frac{A_n}m +\frac{n}\chi =
              \frac{\hat b_n}{m^2} + \frac{n(m^2-1)}{m^2 \chi} \\
\end{split}\end{equation}
%
$\lambda$ is wavelength, $T$ is temperature in Celsius, $I$ is ice water content
in g/m$^3$, $\rho = 0.91$ g/cm$^3$ is ice density, $m$ is the complex index of
refraction, which depends upon temperature and frequency, $j_n(\cdot)$ is
the spherical Bessel function of the first kind, $y_n(\cdot)$ is the
spherical Bessel function of the second kind, $h^{(2)}_n(\cdot)$ is the
spherical Hankel function of the second kind, and

\begin{equation}
W(f_1,f_2) = \left|
             \begin{array}{ll}
              f_1 & f_2 \\
              \frac{\partial f_1}{\partial\chi} & \frac{\partial f_2}{\partial\chi}
             \end{array} \right|
           = f_2^2 \frac{\partial}{\partial\chi} \left( \frac{f_1}{f_2} \right)
\text{ is the Wronskian determinant of } f_1 \text{ and } f_2\,.
\end{equation}

\section{Phase function}

The phase function is

\begin{equation}\label{one}
p(\theta,r) = \frac{p_0(\theta,r)}{C(r)}
\end{equation}

where

\begin{equation}\begin{split}\label{two}
p_0(\theta,r) =\,& |S_1(\theta,r)|^2 + |S_2(\theta,r)|^2 =
  \Re (S_1(\theta,r))^2 + \Im (S_1(\theta,r))^2 +
  \Re (S_2(\theta,r))^2 + \Im (S_2(\theta,r))^2 \,, \\
S_1 =\,& \sum_{j=1}^\infty \frac{2j+1}{j(j+1)} \left(
 a_j(r,T) \frac{\d P_j^1(\cos\theta)}{\d \theta} +
 b_j(r,T) \frac{P_j^1(\cos\theta)}{\sin\theta} \right) \,, \\
S_2 =\,& \sum_{j=1}^\infty \frac{2j+1}{j(j+1)} \left(
 a_j(r,T) \frac{P_j^1(\cos\theta)}{\sin\theta} +
 b_j(r,T) \frac{\d P_j^1(\cos\theta)}{\d \theta} \right) \,, \\
C(r) =\,& \frac12 \int_0^\pi p_0(\theta,r) \sin\theta \, \d \theta \,, \\
\end{split}\end{equation}

and $P^1_j(\cos\theta)$ is the Legendre function.  The integrated phase
function is

\begin{equation}\label{six}
P(\theta) = \frac\pi{\beta_{c\_s}} \int_0^\infty
 r^2 n(r) \xi_s(r) p(\theta,r) \, \d r =
 \frac\pi{\beta_{c\_s}} \int_0^\infty
 r^2 n(r) \xi_s(r)
  \frac{p_0(\theta,r)}{C(r)} \,\d r \,.
\end{equation}

As shown in wvs-070, this may be reduced to a single integral:

\begin{equation}
P(\theta) =
 \frac{\lambda^2}{2 \pi \beta_{c\_s}} \int_0^\infty n(r)\, p_0(\theta,r) \,\d r
\end{equation}

and its derivatives correspondingly simplify to

\begin{equation}\begin{split}
\frac{\partial P(\theta)}{\partial T} = \,&
 \frac{\lambda^2}{2 \pi \beta_{c\_s}} \int_0^\infty\,
  \frac{\partial n(r)}{\partial T}\, p_0(\theta,r) +
  n(r) \frac{\partial p_0}{\partial T} \,\d r -
  \frac{P(\theta)}{\beta_{c\_s}} \frac{\partial \beta_{c\_s}}{\partial T}
\text{ and} \\
\frac{\partial P(\theta)}{\partial IWC} = \,&
 \frac{\lambda^2}{2 \pi \beta_{c\_s}} \int_0^\infty\,
  \frac{\partial n(r)}{\partial IWC}\, p_0(\theta,r) \,\d r -
  \frac{P(\theta)}{\beta_{c\_s}} \frac{\partial \beta_{c\_s}}{\partial IWC}
\,.
\end{split}\end{equation}

\label{lastpage}
\end{document}
% $Id$

% $Log$
% Revision 1.5  2014/02/06 00:08:21  vsnyder
% Add McFarquhar's first name to the footnote
%
% Revision 1.4  2014/01/29 23:16:20  vsnyder
% Correct Equation number in MH footnote
%
% Revision 1.3  2014/01/29 01:35:37  vsnyder
% Add footnote about differences between ATBD and McFarquhar-Heymsfield
%
% Revision 1.2  2013/07/17 02:14:17  vsnyder
% Add phase function section
%
