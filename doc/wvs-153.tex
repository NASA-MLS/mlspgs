\documentclass[11pt]{article}
\usepackage{alltt}
\usepackage[fleqn]{amsmath}
\usepackage{floatflt}
\usepackage{graphicx}
\usepackage{longtable}
\usepackage[strings]{underscore}

\textwidth 6.5in
\oddsidemargin -0.25in
%\evensidemargin -0.5in
\topmargin -0.5in
\textheight 9in

\newcommand{\docname}{wvs-153}
\newcommand{\docdate}{26 August 2019}

\ifx\pdfoutput\undefined
  \pdfoutput=0
  \usepackage[hypertex,plainpages,hyperindex=true]{hyperref}
  \hypersetup{%
    hypertexnames=false%
  }
  % Specify the driver for the color package
  \ExecuteOptions{dvips}
  %\ExecuteOptions{xdvi}
\else
  \ifnum\pdfoutput>0
   
\usepackage[pdftex,plainpages,hyperindex=true,pdfpagelabels]{hyperref}
    \hypersetup{%
      hypertexnames=false,%
      colorlinks=true,%
      linktocpage=true,%
    }
    % Specify the driver for the color package
    \ExecuteOptions{pdftex}
  \else
    \usepackage[hypertex,plainpages,hyperindex=true]{hyperref}
    \hypersetup{%
      hypertexnames=false%
    }
    % Specify the driver for the color package
    \ExecuteOptions{dvips}
    %\ExecuteOptions{xdvi}
  \fi
\fi

\hyperbaseurl{}
\newcommand\hr[1]{\href{#1.dvi}{dvi}, \href{#1.pdf}{pdf}}
\newcommand\h[1]{#1 (\hr{#1})}

\renewcommand\d{\text{d}}
\newcommand\T{\mathcal{T}}

\begin{document}

%\tracingcommands=1
\newlength{\hW} % heading box width
\newlength{\pW} % page number field width
\settowidth{\hW}{\bf\docname}
\settowidth{\pW}{Page \pageref{lastpage}\ of \pageref{lastpage}}
\ifdim \pW > \hW \setlength{\hW}{\pW} \fi
\makeatletter
\def\@biblabel#1{#1.}
\newcommand{\ps@twolines}{%
  \renewcommand{\@oddhead}{%
    \docdate\hfill\parbox[t]{\hW}{{\hfill\bf\docname}\newline
                          Page \thepage\ of \pageref{lastpage}}}%
\renewcommand{\@evenhead}{}%
\renewcommand{\@oddfoot}{}%
\renewcommand{\@evenfoot}{}%
}%
\makeatother
\pagestyle{twolines}

\vspace{-10pt}
\begin{tabbing}
\phantom{References: }\= \\
To: \>MLS group\\
Subject: \>Derivation of block Cholesky factorization\\
From: \>Van Snyder\\
%Reference: \\
\end{tabbing}

\parindent 0pt \parskip 6pt

Matrix arithmetic in {\tt mlsl2} uses a two-level representation. At the
top level, the matrix is composed of blocks. At the bottom level there are
four representations of blocks: absent, banded, sparse, or dense.

Cholesky factorization can be organized to exploit this representation. In
particular, absent blocks can be exploited to reduce cost.

Suppose $\mathbf{A}$ is symmetric and positive definite.  Regard
$\mathbf{A} = (A_{ij})$ and its upper-triangular Cholesky factor
$\mathbf{U} = (U_{ij})$ as $N \times N$ matrices with square diagonal
blocks.  By equating $(i,j)$ blocks in the equation $\mathbf{A} =
\mathbf{U^T} \mathbf{U}$ with $i \leq j$, it follows that

\begin{equation*}
 A_{ij} = \sum_{k=1}^i U_{ki}^T U_{kj}
\end{equation*}
Define
\begin{equation*}
S_{ij} = A_{ij} - \sum_{k=1}^{i-1} U_{ki}^T U_{kj}
       = A_{ij} - \sum_{k=1}^i U_{ki}^T U_{kj} + U_{ii}^T U_{ij}
       = U_{ii}^T U_{ij}
\end{equation*}

Rearranging, we have the following algorithm:

{\bf do} $i = 1, N$\\
\hspace*{0.25in} $S_{ii}= A_{ii} - \sum_{k=1}^{i-1} U_{ki}^T U_{ki}$\\
\hspace*{0.25in} $U_{ii} = $ Cholesky factor of $S_{ii}$\\
\hspace*{0.25in} $A_{ii} = U_{ii}$ if you want A factored in-place\\
\hspace*{0.25in} {\bf do} $j = i+1, N$\\
\hspace*{0.5in}    $S_{ij}= A_{ij} - \sum_{k=1}^{i-1} U_{ki}^T U_{kj}$\\
\hspace*{0.5in}    solve $U_{ii}^T U_{ij} = S_{ij}$ for $U_{ij}$\\
\hspace*{0.5in}    $A_{ij} = U_{ij}$ if you want A factored in-place\\
\hspace*{0.25in} {\bf end do} ! j\\
{\bf end do} ! i

Cholesky factorization of a sparse matrix $\mathbf{A}$ might unnecessarily
produce nonzero elements of the factors. This can be avoided by permuting
the rows and columns of $\mathbf{A}$.

Consider the sparse symmetric positive-definite matrix

\begin{equation*}
\mathbf{A} = \left[ \begin{array}{rrrr}
4 & -1 & -1 & -1 \\
-1 & 2 &    &    \\
-1 &   &  2 &    \\
-1 &   &    &  2 \\
\end{array} \right]
\end{equation*}

for which the lower-triangular Cholesky factor (with four units of
precision) is

\begin{equation*}
\mathbf{L_A} = \mathbf{U_A^T} = \left[ \begin{array}{rrrr}
 2.0000 &         &         &         \\
-0.5000 &  1.3229 &         &         \\
-0.5000 & -0.1890 &  1.3093 &         \\
-0.5000 & -0.1890 & -0.2182 &  1.2910 \\
\end{array} \right] \,.
\end{equation*}

As is seen, the lower triangular matrix $\mathbf{L_A}$ is full. The
entries of $\mathbf{L_A}$ that were zero in $\mathbf{A}$ are called
\emph{fill-in}.  In $\mathbf{L_A}$, the (3,2), (4,2), and (4,3) elements
are fill-ins.

If $\mathbf{A}$ is multiplied by the permutation matrix

\begin{equation*}
\mathbf{P} = \left[ \begin{array}{rrrr}
 0 & 1 & 0 & 0 \\
 0 & 0 & 1 & 0 \\
 0 & 0 & 0 & 1 \\
 1 & 0 & 0 & 0 \\
\end{array} \right]
\end{equation*}

on the left and $\mathbf{P^T}$ on the right, the symmetric
positive-definite matrix

\begin{equation*}
\mathbf{B} = \mathbf{P A P^T} = \left[ \begin{array}{rrrr}
  2 &    &    & -1 \\
    &  2 &    & -1 \\
    &    &  2 & -1 \\
 -1 & -1 & -1 &  4 \\
\end{array} \right]
\end{equation*}

is obtained.

The lower-triangular Cholesky factor of $\mathbf{B}$, \emph{viz.}

\begin{equation*}
\mathbf{L_B} = \left[ \begin{array}{rrrr}
  1.4142 &         &         &         \\
         &  1.4142 &         &         \\
         &         &  1.4142 &         \\
 -0.7071 & -0.7071 & -0.7071 &  1.5811 \\
\end{array} \right]
\end{equation*}

has no fill-ins.  The factorization $\mathbf{B} = \mathbf{L_B L_B^T}$ is
clearly preferable to $\mathbf{A} = \mathbf{L_A L_A^T}$.  The solution $x$
of $\mathbf{A}x = b$ can be calculated by first finding the solution of
the linear system $\mathbf{B}y = \mathbf{P}b$ and then setting $x =
\mathbf{P^T} y$.  This is described in detail in {\bf Direct methods for
sparse matrix solution} by Prof. Iain Duff (Rutherford Appleton
Laboratory, Chilton, Oxfordshire, UK) and Bora U\c{c}ar (CNRS and ENS,
Lyon, France), {\tt
http://www.scholarpedia.org/article/Direct_methods_for_sparse_matrix_solution\#Cholesky_factorization}

\vspace*{5pt}
Finding the optimal $\mathbf{P}$ is NP-complete, but there are many heuristics to
reduce fill-in significantly.  One example is \emph{Algorithm 836: A
Column Approximate Minimum Degree Ordering Algorithm}, {\bf ACM
Transactions on Mathematical Software 30}, 3 (September 2004) pp 377--380,
for which the code is available from {\tt
http://www.netlib.org/toms/836.gz}.

\vspace*{-1pt} \label{lastpage} \end{document}

% $Id$

% $Log$
