\documentclass[11pt]{article}
\usepackage[fleqn]{amsmath}\textwidth 6.5in
\oddsidemargin -0.25in
%\evensidemargin -0.5in
\topmargin -0.25in
\textheight 9in

\newcommand{\docname}{\bf wvs-068r2}
\newcommand{\docdate}{16 July 2013}

\ifx\pdfoutput\undefined
  \pdfoutput=0
  \usepackage[hypertex,plainpages,hyperindex=true]{hyperref}
  \hypersetup{%
    hypertexnames=false%
  }
  % Specify the driver for the color package
  \ExecuteOptions{dvips}
  %\ExecuteOptions{xdvi}
\else
  \ifnum\pdfoutput>0
    \usepackage[pdftex,plainpages,hyperindex=true,pdfpagelabels]{hyperref}
    \hypersetup{%
      hypertexnames=false,%
      colorlinks=true,%
      linktocpage=true,%
    }
    % Specify the driver for the color package
    \ExecuteOptions{pdftex}
  \else
    \usepackage[hypertex,plainpages,hyperindex=true]{hyperref}
    \hypersetup{%
      hypertexnames=false%
    }
    % Specify the driver for the color package
    \ExecuteOptions{dvips}
    %\ExecuteOptions{xdvi}
  \fi
\fi

\hyperbaseurl{}
\renewcommand{\d}{\text{d}}
\ifx\dvidir\undefined
  \newcommand\hr[1]{\href{#1.dvi}{dvi} \href{#1.pdf}{pdf}}
\else
  \newcommand\hr[1]{\href{\dvidir/#1.dvi}{dvi} \href{\pdfdir/#1.pdf}{pdf}}
\fi
\newcommand\h[1]{#1 \hr{#1}}
\newcommand\hh[2]{#1 \hr{#2}}
\newcommand\hrpdf[1]{\href{#1.pdf}{pdf}}
\newcommand\hpdf[1]{#1 \hrpdf{#1}}

\begin{document}

%\tracingcommands=1
\newlength{\hW} % heading box width
\newlength{\pW} % page number field width
\settowidth{\hW}{\docname}
\settowidth{\pW}{Page \pageref{lastpage}\ of \pageref{lastpage}}
\ifdim \pW > \hW \setlength{\hW}{\pW} \fi
\makeatletter
\def\@biblabel#1{#1.}
\newcommand{\ps@twolines}{%
  \renewcommand{\@oddhead}{%
    \docdate\hfill\parbox[t]{\hW}{{\hfill\docname}\newline
                          Page \thepage\ of \pageref{lastpage}}}%
\renewcommand{\@evenhead}{}%
\renewcommand{\@oddfoot}{}%
\renewcommand{\@evenfoot}{}%
}%
\makeatother
\pagestyle{twolines}

\vspace{-10pt}
\begin{tabbing}
\phantom{References: }\= \\
To: \>Dong, Igor, Van\\
Subject: \>Mie phase function, integrated scattering, and derivatives\\
From: \>Van Snyder\\
Reference: \>\h{wvs-058}, \h{wvs-066}
\end{tabbing}

\parindent 0pt \parskip 6pt
\vspace{-10pt}

From the cloud ATBD JPL D-19299 (4 June 2004), the phase function is

\begin{equation}\label{one}
p(\theta,r) = \frac{p_0(\theta,r)}{C(r)}
\end{equation}

where

\begin{equation}\begin{split}\label{two}
p_0(\theta,r) =\,& |S_1(\theta,r)|^2 + |S_2(\theta,r)|^2 =
  \Re (S_1(\theta,r))^2 + \Im (S_1(\theta,r))^2 +
  \Re (S_2(\theta,r))^2 + \Im (S_2(\theta,r))^2 \\
S_1 =\,& \sum_{j=1}^\infty \frac{2j+1}{j(j+1)} \left(
 a_j(r,T) \frac{\d P_j^1(\cos\theta)}{\d \theta} +
 b_j(r,T) \frac{P_j^1(\cos\theta)}{\sin\theta} \right) \\
S_2 =\,& \sum_{j=1}^\infty \frac{2j+1}{j(j+1)} \left(
 a_j(r,T) \frac{P_j^1(\cos\theta)}{\sin\theta} +
 b_j(r,T) \frac{\d P_j^1(\cos\theta)}{\d \theta} \right) \\
C(r) =\,& \frac12 \int_0^\pi p_0(\theta,r) \sin\theta \, \d \theta \,. \\
\end{split}\end{equation}

$P_j^1(\cos\theta)$ is the associated Legendre function of the first kind,
which satisfies the following recurrence, from which it can be computed stably:

\begin{equation}
\frac{P_j^1(\cos\theta)}{\sin\theta} = \left\{
\begin{array}{ll}
0 & j = 0 \\
1 & j = 1 \\
 \cos\theta\, \frac{2j-1}{j-1} \frac{P_{j-1}^1(\cos\theta)}{\sin\theta} -
 \frac{j}{j-1} \frac{P_{j-2}^1(\cos\theta)}{\sin\theta} & j > 1\,.\\
\end{array}\right.
\end{equation}

The derivatives of $P_j^1(\cos\theta)$ w.r.t.\ $\theta$ can be computed from the
following relations:

\begin{equation}
\frac{\d P_j^1(\cos\theta)}{\d\theta}  = \left\{
\begin{array}{ll}
0 & j \leq 1 \\
 & \\
\begin{split}
 j\,& \frac{P_{j+1}^1(\cos\theta)}{\sin\theta} -
 (j+1) \cos\theta\, \frac{P_j^1(\cos\theta)}{\sin\theta} \\
=\,&
 j \cos\theta\, \frac{P_j^1(\cos\theta)}{\sin\theta} -
 (j+1)\, \frac{P_{j-1}^1(\cos\theta)}{\sin\theta}
\end{split} & j > 1
\end{array} \right.
\end{equation}

(see Equation 8.5.4 in National Bureau of Standards AMS 55, or Equations
14.10.4-5 in the NIST Handbook of Mathematical Functions).  $P^1_j(\cos m
\pi) = 0$, but $\frac{P^1_j(\cos m\pi)}{\sin(m\pi)}$ is undefined, for $m
= 0, 1, 2, \dots.$

$\theta$ is independent of $T$ and IWC, and $a_j(r,T)$ and $b_j(r,T)$ are
independent of IWC, so derivatives of $S_1$ and $S_2$ w.r.t.\ IWC are zero
and derivatives w.r.t.\ $T$ depend upon derivatives of $a_j(r,T)$ and
$b_j(r,T)$ w.r.t.\ $T$ in an obvious straight-forward way (see {\tt
wvs-066} for $a_j(r,T)$ and $b_j(r,T)$ and their derivatives w.r.t.\ $T$).

From the definition of the quantity $p_0(\theta,r)$ in Equation (\ref{two}),

\begin{equation}\begin{split}
\frac{\partial p_0(\theta,r)}{\partial T} = 2 & \left(
 \Re S_1(\theta,r) \frac{\partial \Re S_1(\theta,r)}{\partial T} +
 \Im S_1(\theta,r) \frac{\partial \Im S_1(\theta,r)}{\partial T} \right.\, + \\
  \,& \left. \phantom{ (\,\,\,}
 \Re S_2(\theta,r) \frac{\partial \Re S_2(\theta,r)}{\partial T} +
 \Im S_2(\theta,r) \frac{\partial \Im S_2(\theta,r)}{\partial T} \right) \,.
  \\
\end{split}\end{equation}

Using this and the definitions of $p(\theta,r)$ and $C(r)$ from Equations
(\ref{one}) and (\ref{two}),

\begin{equation}\begin{split}
\frac{\partial p(\theta,r)}{\partial T} =\,&
 \frac1{C(r)} \left( \frac{\partial p_0(\theta,r)}{\partial T} -
 p(\theta,r) \frac{\partial C(r)}{\partial T} \right) \\
=\,& \frac1{C(r)} \left( \frac{\partial p_0(\theta,r)}{\partial T} -
 \frac{p(\theta,r)}2 \int_0^\pi \frac{\partial p_0(\theta,r)}{\partial T}
 \sin\theta\, \d \theta \right) .
\end{split}\end{equation}

Since $a_j(r,T)$ and $b_j(r,T)$ do not depend on IWC, the phase function does
not depend on IWC.

The integrated phase function is

\begin{equation}
P(\theta) =
 \frac{\pi}{\beta_{c\_s}} \int_0^\infty r^2\, n(r)\, \xi_s(r)\, p(\theta,r)\, \d r
\end{equation}

from which

\begin{equation}\begin{split}
\frac{\partial P(\theta)}{\partial T} =\,&
 \frac{\pi}{\beta_{c\_s}} \int_0^\infty r^2\, n(r)\, \xi_s(r)\, p(\theta,r)\,
 \left( \frac1{n(r)}\, \frac{\partial n(r)}{\partial T} +
        \frac1{\xi_s(r)}\, \frac{\partial \xi_s(r)}{\partial T} +
        \frac1{p(\theta,r)}\, \frac{\partial p(\theta,r)}{\partial T}
 \right) \, \d r \\
 \,&
 -\frac{P(\theta)}{\beta_{c\_s}}\, \frac{\partial \beta_{c\_s}}{\partial T} \\
\end{split}\end{equation}
and
\begin{equation}
 \frac{\partial P(\theta)}{\partial \text{IWC}} =
 \frac{\pi}{\beta_{c\_s}} \int_0^\infty r^2\,
  \frac{\partial n(r)}{\partial \text{IWC}} \,
  \xi_s(r)\, p(\theta,r)\, \d r
 -\frac{P(\theta)}{\beta_{c\_s}}\,
  \frac{\partial \beta_{c\_s}}{\partial\text{IWC}}\,.
\end{equation}

See \h{wvs-066} for derivatives of $\beta_{c\_s}$ and $n(r)$ w.r.t.\ $T$
and IWC, and the derivative of $\xi_s(r)$ w.r.t.\ $T$.

\label{lastpage}
\end{document}
% $Id$

% $Log$
