\documentclass[11pt]{article}
\usepackage[fleqn]{amsmath}\textwidth 6.5in
\oddsidemargin -0.25in
%\evensidemargin -0.5in
\topmargin -0.2in
\textheight 9.2in

\newcommand{\docname}{\bf wvs-039r2}
\newcommand{\docdate}{10 July 2006}

\usepackage{longtable}

\begin{document}

%\tracingcommands=1
\newlength{\hW} % heading box width
\newlength{\pW} % page number field width
\settowidth{\hW}{\docname}
\settowidth{\pW}{Page \pageref{lastpage}\ of \pageref{lastpage}}
\ifdim \pW > \hW \setlength{\hW}{\pW} \fi
\makeatletter
\def\@biblabel#1{#1.}
\newcommand{\ps@twolines}{%
  \renewcommand{\@oddhead}{%
    \docdate\hfill\parbox[t]{\hW}{{\hfill\docname}\newline
                          Page \thepage\ of \pageref{lastpage}}}%
\renewcommand{\@evenhead}{}%
\renewcommand{\@oddfoot}{}%
\renewcommand{\@evenfoot}{}%
}%
\makeatother
\pagestyle{twolines}

\vspace{-10pt}
\begin{tabbing}
\phantom{References: }\= \\
To: \>Bill\\
Subject: \>Calculating $H_m$ in modification to {\tt metrics\_m}\\
From: \>Van Snyder\\
\end{tabbing}

\parindent 0pt \parskip 6pt
\vspace{-20pt}

The basic formula for $H_m$ is $H_m = H_t / \cos \phi_m$, where $H_t$ is
tangent height, which is derived from $\phi_m = \cos^{-1} \frac{H_t}{H_m}$. 
{\tt metrics} thinks in terms of offsets from the equivalent circular earth
radius, and from $\phi_t$, so the latter becomes

\begin{equation}\label{one}
\phi_m = \phi_t +
 \cos^{-1}\frac{H_t+R^\oplus_\text{eq}}{H_m+R^\oplus_\text{eq}}\,,
\end{equation}

or (solving for $H_m$)

\begin{equation}
H_m = \frac{H_t+R^\oplus_\text{eq}}{\cos(\phi_m-\phi_t)} - R^\oplus_\text{eq}\,,
\end{equation}

at least when the ray is not an earth-intersecting ray.

When there is an earth-intersecting ray, Equation (\ref{one}) becomes more
complicated for the second half of the path, \emph{viz.}


\begin{equation}\label{two}
\phi_m = \phi_t +
 \cos^{-1}\frac{H_t+R^\oplus_\text{eq}}{H_m+R^\oplus_\text{eq}} -
 2 \cos^{-1}\frac{N+R^\oplus_\text{eq}}{R^\oplus_\text{eq}}\,,
\end{equation}

where $N$ is the ``negative tangent height.''  Solving for $H_m$ we have

\begin{equation}
H_m = \frac{H_t+R^\oplus_\text{eq}}
      {\cos\left(\phi_m-\phi_t
       +2\cos^{-1}\frac{N+R^\oplus_\text{eq}}{R^\oplus_\text{eq}}\right)}
      -R^\oplus_\text{eq}
\end{equation}

\label{lastpage}
\end{document}
% $Id$
