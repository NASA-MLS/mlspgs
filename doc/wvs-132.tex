\documentclass[11pt]{article}
\usepackage{alltt}
\usepackage[fleqn]{amsmath}
\usepackage{floatflt}
\usepackage{graphicx}
\usepackage{longtable}
\usepackage[strings]{underscore}

\textwidth 6.5in
\oddsidemargin -0.25in
%\evensidemargin -0.5in
\topmargin -0.5in
\textheight 9in

\newcommand{\docname}{wvs-132r3}
\newcommand{\docdate}{15 April 2020}

\ifx\pdfoutput\undefined
  \pdfoutput=0
\fi
\ifnum\pdfoutput>0
  \usepackage[pdftex,plainpages,hyperindex=true,pdfpagelabels]{hyperref}
  \hypersetup{%
    hypertexnames=false,%
    colorlinks=true,%
    linktocpage=true,%
  }
  % Specify the driver for the color package
  \ExecuteOptions{pdftex}
\else
  \usepackage[hypertex,plainpages,hyperindex=true]{hyperref}
  \hypersetup{%
    hypertexnames=false%
  }
  % Specify the driver for the color package
  \ExecuteOptions{dvips}
  %\ExecuteOptions{xdvi}
\fi

\hyperbaseurl{}
\newcommand\hr[1]{\href{#1.dvi}{dvi}, \href{#1.pdf}{pdf}}
\newcommand\h[1]{#1 (\hr{#1})}

\begin{document}

%\tracingcommands=1
\newlength{\hW} % heading box width
\newlength{\pW} % page number field width
\settowidth{\hW}{\bf\docname}
\settowidth{\pW}{Page \pageref{lastpage}\ of \pageref{lastpage}}
\ifdim \pW > \hW \setlength{\hW}{\pW} \fi
\makeatletter
\def\@biblabel#1{#1.}
\newcommand{\ps@twolines}{%
  \renewcommand{\@oddhead}{%
    \docdate\hfill\parbox[t]{\hW}{{\hfill\bf\docname}\newline
                          Page \thepage\ of \pageref{lastpage}}}%
\renewcommand{\@evenhead}{}%
\renewcommand{\@oddfoot}{}%
\renewcommand{\@evenfoot}{}%
}%
\makeatother
\pagestyle{twolines}

\newcommand{\TS}{T_\text{scat}}
\newcommand{\TSs}[1]{T_{\text{scat}_{#1}}}
\newcommand{\DB}{\Delta B}
\newcommand{\oDB}{\overline{\DB}}
\newcommand{\MT}{\mathcal{T}}
\newcommand{\hMT}{\MT^s}
\newcommand{\IF}[1]{\,\mathcal{A}_n\!\left(#1\right)} % Interpolation Function

\vspace{-10pt}
\begin{tabbing}
\phantom{References: }\= \\
To: \>Bill, Van\\
Subject: \>Center of a sphere (or circle) defined by three (or two) points
 and its radius\\
From: \>Van Snyder\\
Reference: \> \h{wvs-128}, \h{wvs-146}
\end{tabbing}

\parindent 0pt \parskip 6pt
\vspace{-10pt}

\section{Circumcenter}

Consider three points defined by ECR vectors $\mathbf{A}$, $\mathbf{B}$,
and $\mathbf{C}$.

The circumcenter $\mathbf{v}$ of a triangle is where the perpendicular
bisectors of the sides meet.

The vector from the midpoint of side $\mathbf{AB}$ to $\mathbf{v}$ is
$\mathbf{v} - \frac12\, ( \mathbf{A} + \mathbf{B} )$.  The condition that it
is perpendicular to the side $\mathbf{AB}$ is
%
\begin{equation}
\left( \mathbf{v} - \frac12\, ( \mathbf{A} + \mathbf{B} ) \right)
  \cdot ( \mathbf{B}-\mathbf{A} ) = 0 \,,
\end{equation}

and similarly for the sides $\mathbf{AC}$ and $\mathbf{BC}$.  This does
not give three conditions to determine the components of $\mathbf{v}$,
because they are dependent.  An additional condition is that $\mathbf{v}$
is coplanar with $\mathbf{A}$, $\mathbf{B}$, and $\mathbf{C}$, \emph{viz.}
%
\begin{equation}
( \mathbf{v} - \mathbf{A} ) \cdot \left(
( \mathbf{B}-\mathbf{A} ) \times ( \mathbf{C}-\mathbf{A} ) \right) = 0
\,.
\end{equation}

Arbitrarily choose $\mathbf{A}$ as a reference point, and define two
vectors
%
\begin{equation}\begin{split}
\mathbf{b} =\,& \mathbf{B} - \mathbf{A} \\
\,& \\
\mathbf{c} =\,& \mathbf{C} - \mathbf{A} \,. \\
\end{split}\end{equation}

Then
%
\begin{equation}
\mathbf{n} = 
( \mathbf{B}-\mathbf{A} ) \times ( \mathbf{C}-\mathbf{A} )
= \mathbf{b} \times \mathbf{c}
\end{equation}

is normal to the plane defined by $\mathbf{A}$, $\mathbf{B}$,
and $\mathbf{C}$.

If we choose $\mathbf{A}$ as the origin, i.e., $\mathbf{A} = 0$, we have
three equations, \emph{viz.}

\begin{equation}\begin{split}
\mathbf{v} \cdot \mathbf{b} = \,& \frac12\, \mathbf{b} \cdot \mathbf{b}
                            = |\mathbf{b}|^2 \\
\mathbf{v} \cdot \mathbf{c} = \,& \frac12\, \mathbf{c} \cdot \mathbf{c}
                            = |\mathbf{c}|^2 \\
\mathbf{v} \cdot \mathbf{n} = \,& 0 \\
\end{split}\end{equation}

The solution of these equations gives the vector from $\mathbf{A}$ to the
circumcenter, \emph{viz.}
%
\begin{equation}
\mathbf{v} = \frac{(|\mathbf{b}|^2 \mathbf{c} - |\mathbf{c}|^2 \mathbf{b})
            \times \mathbf{n}}
           {2 | \mathbf{n} |^2 }\,.
\end{equation}
%
The ECR vector to the circumcenter of the three points is therefore
$\mathbf{p}(0) = \mathbf{A} + \mathbf{v}$.

% ======================================================================
\section{Sphere}

The centers $\mathbf{p}(t)$ of the two spheres defined by$\mathbf{A}$,
$\mathbf{B}$, and $\mathbf{C}$, and the radius $r$, are on the line
passing through the circumcenter of the three points, and normal to the
plane containing the three points, \emph{viz.}
%
\begin{equation}
\mathbf{p}(t) = \mathbf{p}(0) + t \, \mathbf{n} =
 \mathbf{A} + \mathbf{v} + t \, \mathbf{n}\,\,
\end{equation}
%
at a value of $t$ to be computed.  The distance from any of the three
points, say $\mathbf{A}$, to the center of the sphere is $r$.  Therefore
%
\begin{equation}
| \mathbf{p}(t) - \mathbf{A} | ^2 = | \mathbf{v} + t \, \mathbf{n} |^2 =
  |\mathbf{n}|^2 \, t^2 +
   2 \, \mathbf{v} \cdot \mathbf{n} \, t +
  | \mathbf{v} | ^2 =
 r^2 \,.
\end{equation}
%
The vector $\mathbf{n}$ is (by definition) orthogonal to $\mathbf{v}$
because $\mathbf{v}$ is in the plane defined by $\mathbf{A}$,
$\mathbf{B}$, and $\mathbf{C}$.  Therefore the second term above is zero,
i.e., $|\mathbf{n}|^2 \, t^2 +| \mathbf{v} | ^2 = r^2$ and
%
\begin{equation}\label{six}
t = \pm \sqrt{ \frac{r^2 - | \mathbf{v} | ^2}
               { | \mathbf{n} |^2 } }\,.
\end{equation}
%
Since $\mathbf{A}$, $\mathbf{B}$, and $\mathbf{C}$ are given in ECR
co\"ordinates, to select a center of the sphere nearer to the center of
the Earth, choose the sign of $t$ to minimize $|\mathbf{p}(t)|$, which
will depend upon the direction of $\mathbf{n}$, which will in turn depend
upon the order in which the points are considered.

\section{Circle}

The centers of two circles defined by ECR vectors $\mathbf{A}$ and
$\mathbf{B}$ to two points, and the radius $r$,  are on a line
perpendicular to $\mathbf{B}-\mathbf{A}$ at $\mathbf{p}(0) =
\frac12(\mathbf{B}+\mathbf{A})$ and in the plane containing $\mathbf{A}$
and $\mathbf{B}$.

The vector normal to the plane containing $\mathbf{A}$ and $\mathbf{B}$ is
$\mathbf{n}_1 = \mathbf{A} \times \mathbf{B}$.  The vector perpendicular
to $\mathbf{B}-\mathbf{A}$ and in the plane is $\mathbf{n} = \mathbf{n}_1
\times ( \mathbf{B}-\mathbf{A})$.  The vector from $\mathbf{A}$ to
$\mathbf{p}(0)$ is $\mathbf{v} = \frac12(\mathbf{B}-\mathbf{A})$.

The centers of the circles defined by $\mathbf{A}$, $\mathbf{B}$, and $r$
are on the line
%
\begin{equation}
\mathbf{p}(t) = \mathbf{p}(0) + t\,\mathbf{n}
= \frac12(\mathbf{A} + \mathbf{B}) + t\,\mathbf{n}\,,
\end{equation}
%
at a value of $t$ to be computed.  The distance from either point, say
$\mathbf{A}$, to the center of the circle is $r$.  Therefore
%
\begin{equation}
|\mathbf{p}(t) - \mathbf{A}|^2
= \left|\,\frac12(\mathbf{B}-\mathbf{A}) + t\mathbf{n}\,\right|^2
= |\mathbf{v} + t\mathbf{n}|^2
= |\mathbf{n}|^2\,t^2 + 2\, \mathbf{v} \cdot \mathbf{n}\, t + |\mathbf{v}|^2
= r^2 \,.
\end{equation}
%
The vector $\mathbf{n}$ is (by definition) orthogonal to $\mathbf{v}$. 
Therefore the second term above is zero, i.e., $|\mathbf{n}|^2 \, t^2 +|
\mathbf{v} | ^2 = r^2$ and
%
\begin{equation}\label{nine}
t = \pm \sqrt{ \frac{r^2 - | \mathbf{v} | ^2}
               { | \mathbf{n} |^2 } }\,.
\end{equation}
%
Since $\mathbf{A}$ and $\mathbf{B}$ are given in ECR co\"ordinates, to
select a center of the circle nearer to the center of the Earth, choose
the sign of $t$ to minimize $|\mathbf{p}(t)|$, which will depend upon the
direction of $\mathbf{n}$, which will in turn depend upon the order in
which the points are considered.

\section{Application to forward model metrics}

The purpose of forward model metrics is to compute the geometric positions
of intersections of a line of sight with surfaces of constant pressure. 
Heights of points of constant pressure are given on profiles.  If the
height within a region defined by points at the same pressure on three
profiles, or between two such points in a plane, is approximated by a
plane or a line, a small error is introduced.  A more accurate scheme is
to approximate the height between such points by a spherical triangle, or
a circular arc.  Computing the intersection of a line with a sphere or
circle involves a square root, not a Newton iteration as used in
\h{wvs-048}.

If the surface of the sphere is meant to approximate a surface above the
Earth reference ellipsoid within the triangle defined by $\mathbf{ABC}$,
compute the geodetic latitude of $\mathbf{p}(0)$, and use the method
described in \h{wvs-146} to compute $R^{\,\oplus}_{\text{eq}_s}$, either
the radius of Gaussian curvature or the radius of mean curvature of the
Earth's surface at the geocentric latitude $\phi$ at the surface below
$\mathbf{p}(0)$ (which is not the same as the geocentric latitude of
$\mathbf{p}(0)$).  Then in Equation (\ref{six}), use $r =
R^{\,\oplus}_{\text{eq}_s} + \hat{h}$, where $\hat{h}$ is the geodetic
height interpolated from the geodetic heights at $\mathbf{A}$,
$\mathbf{B}$, and $\mathbf{C}$ to $\mathbf{p}(0)$, using barycentric
interpolation as described in \h{wvs-128}.

If the circle is meant to approximate a surface above the Earth reference
ellipsoid within the plane defined by $\mathbf{A}$, $\mathbf{B}$, and the
Earth's center (presumably the orbit plane inclined at the angle $\beta$),
compute the radius of curvature of the Earth in the orbit plane $N(\phi) =
a^2 / \sqrt{a^2 \cos^2 \phi + c^2 \sin^2 \phi}$ at the orbit geodetic angle
$\phi$, midway between the orbit geodetic angles at $\mathbf{A}$ and
$\mathbf{B}$, as described in section 5.2 of the ATBD, where $c$ is given
by Equation (5.3) in the ATBD, and use $r = N(\phi) + \hat{h}$ in Equation
(\ref{nine}), where $\hat{h}$ is the average of the geodetic heights at
$\mathbf{A}$ and $\mathbf{B}$.

\label{lastpage}
\vspace*{-0.1in} % Somehow, this causes lastpage to be defined
\end{document}

% $Id$

% $Log$
% Revision 1.3  2017/10/13 19:05:10  vsnyder
% Add derivation for circle.  Explain application to forward model metrics.
%
% Revision 1.2  2016/02/24 01:33:36  vsnyder
% Correct a typo
%
% Revision 1.1  2016/02/19 01:22:38  vsnyder
% Initial commit
%
