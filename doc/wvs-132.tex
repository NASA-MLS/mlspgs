\documentclass[11pt]{article}
\usepackage{alltt}
\usepackage[fleqn]{amsmath}
\usepackage{floatflt}
\usepackage{graphicx}
\usepackage{longtable}
\usepackage[strings]{underscore}

\textwidth 6.5in
\oddsidemargin -0.25in
%\evensidemargin -0.5in
\topmargin -0.5in
\textheight 9in

\newcommand{\docname}{wvs-132r1}
\newcommand{\docdate}{18 February 2016}

\ifx\pdfoutput\undefined
  \pdfoutput=0
  \usepackage[hypertex,plainpages,hyperindex=true]{hyperref}
  \hypersetup{%
    hypertexnames=false%
  }
  % Specify the driver for the color package
  \ExecuteOptions{dvips}
  %\ExecuteOptions{xdvi}
\else
  \ifnum\pdfoutput>0
    \usepackage[pdftex,plainpages,hyperindex=true,pdfpagelabels]{hyperref}
    \hypersetup{%
      hypertexnames=false,%
      colorlinks=true,%
      linktocpage=true,%
    }
    % Specify the driver for the color package
    \ExecuteOptions{pdftex}
  \else
    \usepackage[hypertex,plainpages,hyperindex=true]{hyperref}
    \hypersetup{%
      hypertexnames=false%
    }
    % Specify the driver for the color package
    \ExecuteOptions{dvips}
    %\ExecuteOptions{xdvi}
  \fi
\fi

\hyperbaseurl{}
\newcommand\hr[1]{\href{#1.dvi}{dvi}, \href{#1.pdf}{pdf}}
\newcommand\h[1]{#1 (\hr{#1})}

\begin{document}

%\tracingcommands=1
\newlength{\hW} % heading box width
\newlength{\pW} % page number field width
\settowidth{\hW}{\bf\docname}
\settowidth{\pW}{Page \pageref{lastpage}\ of \pageref{lastpage}}
\ifdim \pW > \hW \setlength{\hW}{\pW} \fi
\makeatletter
\def\@biblabel#1{#1.}
\newcommand{\ps@twolines}{%
  \renewcommand{\@oddhead}{%
    \docdate\hfill\parbox[t]{\hW}{{\hfill\bf\docname}\newline
                          Page \thepage\ of \pageref{lastpage}}}%
\renewcommand{\@evenhead}{}%
\renewcommand{\@oddfoot}{}%
\renewcommand{\@evenfoot}{}%
}%
\makeatother
\pagestyle{twolines}

\newcommand{\TS}{T_\text{scat}}
\newcommand{\TSs}[1]{T_{\text{scat}_{#1}}}
\newcommand{\DB}{\Delta B}
\newcommand{\oDB}{\overline{\DB}}
\newcommand{\MT}{\mathcal{T}}
\newcommand{\hMT}{\MT^s}
\newcommand{\IF}[1]{\,\mathcal{A}_n\!\left(#1\right)} % Interpolation Function

\vspace{-10pt}
\begin{tabbing}
\phantom{References: }\= \\
To: \>Van\\
Subject: \>Center of a sphere defined by three points and its radius\\
From: \>Van Snyder\\
\end{tabbing}

\parindent 0pt \parskip 6pt
\vspace{-20pt}

The centers of the two spheres defined by three points $\mathbf{A}$,
$\mathbf{B}$, and $\mathbf{C}$ and the radius $r$ are on the line passing
through the circumcenter of the three points, and normal to the plane
containing the three points.

Arbitrarily choose $\mathbf{C}$ as a reference point, and define two
vectors
%
\begin{equation}\begin{split}
\mathbf{a} =\,& \mathbf{A} - \mathbf{C} \\
\,& \\
\mathbf{b} =\,& \mathbf{B} - \mathbf{C} \,. \\
\end{split}\end{equation}

A vector normal to the plane defined by the three points is
%
\begin{equation}
\mathbf{n} = \mathbf{a} \times \mathbf{b} \,.
\end{equation}

The vector from $\mathbf{C}$ to the circumcenter $\mathbf{p}_0$ is given
by
%
\begin{equation}
\mathbf{v} = \frac{(|\mathbf{a}|^2 \mathbf{b} - |\mathbf{b}|^2 \mathbf{a})
            \times \mathbf{n}}
           {2 | \mathbf{n} |^2 }\,.
\end{equation}

The circumcenter of the three points is therefore $\mathbf{p}_0 =
\mathbf{C} + \mathbf{v}$.

The centers of the spheres defined by the three points and the radius are
on the line
%
\begin{equation}
\mathbf{p}(t) = \mathbf{p}_0 + t \, \mathbf{n} =
 \mathbf{v} + \mathbf{C} + t \, \mathbf{n}\,.
\end{equation}

The distance from any of the three points, say $\mathbf{C}$, to the center
of the sphere is $r$.  Therefore
%
\begin{equation}
| \mathbf{p}(t) - \mathbf{C} | ^2 = | \mathbf{v} + t \, \mathbf{n} |^2 =
  |\mathbf{n}|^2 \, t^2 +
   2 \, \mathbf{v} \cdot \mathbf{n} \, t +
  | \mathbf{v} | ^2 =
 r^2 \,.
\end{equation}

Since $\mathbf{v}$ is in the plane defined by the three given points,
$\mathbf{v}$ is orthogonal to $\mathbf{n}$, and therefore the second term
above is zero.  Therefore $|\mathbf{n}|^2 \, t^2 +| \mathbf{v} | ^2 = r^2$
and
%
\begin{equation}
t = \pm \sqrt{ \frac{r^2 - | \mathbf{v} | ^2}
               { | \mathbf{n} |^2 } }\,.
\end{equation}

Assuming the three points are given in ECR co\"ordinates, to select a
center of the sphere nearer to the center of the Earth, choose the sign of
$t$ to minimize $|\mathbf{p}(t)|$, which will depend upon the direction of
$\mathbf{n}$, which will in turn depend upon the order in which the points
are considered.

\label{lastpage}
\vspace*{-0.1in} % Somehow, this causes lastpage to be defined
\end{document}

% $Id$

% $Log$
% Revision 1.1  2016/02/19 01:22:38  vsnyder
% Initial commit
%
