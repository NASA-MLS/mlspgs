\documentclass[11pt]{article}
\usepackage{alltt}
\usepackage[fleqn]{amsmath}
\usepackage{floatflt}
\usepackage{graphicx}
\usepackage{longtable}
\usepackage[strings]{underscore}

\textwidth 6.5in
\oddsidemargin -0.25in
%\evensidemargin -0.5in
\topmargin -0.5in
\textheight 9in

\newcommand{\docname}{wvs-103}
\newcommand{\docdate}{7 February 2011}

\ifx\pdfoutput\undefined
  \pdfoutput=0
  \usepackage[hypertex,plainpages,hyperindex=true]{hyperref}
  \hypersetup{%
    hypertexnames=false%
  }
  % Specify the driver for the color package
  \ExecuteOptions{dvips}
  %\ExecuteOptions{xdvi}
\else
  \ifnum\pdfoutput>0
    \usepackage[pdftex,plainpages,hyperindex=true,pdfpagelabels]{hyperref}
    \hypersetup{%
      hypertexnames=false,%
      colorlinks=true,%
      linktocpage=true,%
    }
    % Specify the driver for the color package
    \ExecuteOptions{pdftex}
  \else
    \usepackage[hypertex,plainpages,hyperindex=true]{hyperref}
    \hypersetup{%
      hypertexnames=false%
    }
    % Specify the driver for the color package
    \ExecuteOptions{dvips}
    %\ExecuteOptions{xdvi}
  \fi
\fi

\hyperbaseurl{}
\newcommand\hr[1]{\href{#1.dvi}{dvi}, \href{#1.pdf}{pdf}}
\newcommand\h[1]{#1 (\hr{#1})}

\begin{document}

%\tracingcommands=1
\newlength{\hW} % heading box width
\newlength{\pW} % page number field width
\settowidth{\hW}{\bf\docname}
\settowidth{\pW}{Page \pageref{lastpage}\ of \pageref{lastpage}}
\ifdim \pW > \hW \setlength{\hW}{\pW} \fi
\makeatletter
\def\@biblabel#1{#1.}
\newcommand{\ps@twolines}{%
  \renewcommand{\@oddhead}{%
    \docdate\hfill\parbox[t]{\hW}{{\hfill\bf\docname}\newline
                          Page \thepage\ of \pageref{lastpage}}}%
\renewcommand{\@evenhead}{}%
\renewcommand{\@oddfoot}{}%
\renewcommand{\@evenfoot}{}%
}%
\makeatother
\pagestyle{twolines}

\renewcommand{\d}{\text{d}}
\newcommand{\T}{\mathcal{T}}

\vspace{-10pt}
\begin{tabbing}
\phantom{References: }\= \\
To: \>Nathaniel, Bill\\
Subject: \>Error in temperature derivative of oxygen continuum\\
From: \>Van Snyder\\
Reference: \>\h{wvs-093}, \h{wvs-101}, JPL D-18130 \\
\end{tabbing}

\parindent 0pt \parskip 6pt
\vspace{-10pt}

In the {\tt Abs_CS_O2_Cont_dT} subroutine in the {\tt get_beta_path_m}
module in the forward model, we compute the oxygen continuum cross section

\begin{equation}
\beta = c_1 p^2 \nu^2 \exp(c_2 \theta) D
\end{equation}

where $D = 1 / (\nu^2 + f)$ and $f = \left( c_3 p \exp(c_4 \theta)
\right)^2$.

Observing that $\frac{\d\theta}{\d T} = -\frac{\theta}T$ the code computes

\begin{equation}
\frac{\beta}T ( c_4 f D - c_2 )
\end{equation}

and calls it $\frac{\partial \beta}{\partial T}$.

I believe the correct expression is

\begin{equation}
\frac{\partial \beta}{\partial T} = \frac{\beta \theta}T ( 2 c_4 f D - c_2 )
\end{equation}

\label{lastpage}
\end{document}

% $Id$

% $Log$
