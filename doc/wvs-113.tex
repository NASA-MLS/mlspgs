\documentclass[11pt]{article}
\usepackage{alltt}
\usepackage[fleqn]{amsmath}
\usepackage{floatflt}
\usepackage{graphicx}
\usepackage{longtable}
\usepackage[strings]{underscore}

\textwidth 6.5in
\oddsidemargin -0.25in
%\evensidemargin -0.5in
\topmargin -0.5in
\textheight 9in

\newcommand{\docname}{wvs-113}
\newcommand{\docdate}{30 May 2013}

\ifx\pdfoutput\undefined
  \pdfoutput=0
  \usepackage[hypertex,plainpages,hyperindex=true]{hyperref}
  \hypersetup{%
    hypertexnames=false%
  }
  % Specify the driver for the color package
  \ExecuteOptions{dvips}
  %\ExecuteOptions{xdvi}
\else
  \ifnum\pdfoutput>0
    \usepackage[pdftex,plainpages,hyperindex=true,pdfpagelabels]{hyperref}
    \hypersetup{%
      hypertexnames=false,%
      colorlinks=true,%
      linktocpage=true,%
    }
    % Specify the driver for the color package
    \ExecuteOptions{pdftex}
  \else
    \usepackage[hypertex,plainpages,hyperindex=true]{hyperref}
    \hypersetup{%
      hypertexnames=false%
    }
    % Specify the driver for the color package
    \ExecuteOptions{dvips}
    %\ExecuteOptions{xdvi}
  \fi
\fi

\hyperbaseurl{}
\newcommand\hr[1]{\href{#1.dvi}{dvi}, \href{#1.pdf}{pdf}}
\newcommand\h[1]{#1 (\hr{#1})}

\begin{document}

%\tracingcommands=1
\newlength{\hW} % heading box width
\newlength{\pW} % page number field width
\settowidth{\hW}{\bf\docname}
\settowidth{\pW}{Page \pageref{lastpage}\ of \pageref{lastpage}}
\ifdim \pW > \hW \setlength{\hW}{\pW} \fi
\makeatletter
\def\@biblabel#1{#1.}
\newcommand{\ps@twolines}{%
  \renewcommand{\@oddhead}{%
    \docdate\hfill\parbox[t]{\hW}{{\hfill\bf\docname}\newline
                          Page \thepage\ of \pageref{lastpage}}}%
\renewcommand{\@evenhead}{}%
\renewcommand{\@oddfoot}{}%
\renewcommand{\@evenfoot}{}%
}%
\makeatother
\pagestyle{twolines}

\vspace{-10pt}
\begin{tabbing}
\phantom{References: }\= \\
To: \>Nathaniel, Bill\\
Subject: \>Considerations for out-of-orbit-plane modeling\\
From: \>Van Snyder\\
%Reference: \> \\
\end{tabbing}

\parindent 0pt \parskip 6pt
\vspace{-10pt}

\paragraph*{Introduction.}

The simplest modification of the forward model that does not require the
solutions to be reported in the orbit plane is to assume that the
solutions are reported on a plane that is parallel to the orbit plane, and
corresponds on average to the locus of tangent points.  If there is one
out-of-orbit scan, as in UARS, there is one plane, but if there are
several, as for SMLS, there would be several planes, but the forward model
could consider them one at a time.  There are (at least) four
possibilities concerning where in that plane to report the solutions.

The forward model currently develops interpolation coefficients between
points on each integration path and the set of points on which solutions
are to be reported.  These interpolation coefficients appear in the
expressions for derivatives.  Since solutions are reported in the orbit
plane, there are two directions of interpolation, one in $\phi$ and the
other in $\zeta$.  A third interpolation direction, call it azimuth, will
be needed in some of the possible solution-reporting methods.

In all cases, azimuthal interpolation coefficients might be needed to
connect the MIF tangent points to the model tangent points, for purposes
of antenna convolution.

\paragraph*{Vertical profile, fixed viewing azimuth.}

The first possibility is to report vertical profiles, which means each
profile has one (lat,lon) and several $\zeta$s.  If the lines of sight are
at a constant azimuth from the orbit plane, since the position of each
tangent point in the solution plane advances as the spacecraft proceeds in
its orbit, the locus of tangent points is not a vertical line.  This means
that the forward model needs azimuthal interpolation coefficients from the
tangent points' positions to the profile positions, in order to compute
derivatives of radiance with respect to quantities on the reporting
profile.

\paragraph*{Vertical profile, variable viewing azimuth.}

The second possibility is also to report vertical profiles but to
integrate along lines of sight from the spacecraft position to the
vertical profile line, i.e., the tangent points are roughly on a vertical
line, but each line of sight is at a different azimuth.  The azimuth of
each line of sight would be computed from the profile (lat,lon) and the
spacecraft (lat,lon).  The profile (lat,lon) could, in turn, be computed
by specifying the viewing azimuth from the spacecraft (lat,lon) at some
representative average orbit angle during the MAF.  This wouldn't require
azimuthal interpolation coefficients.

\paragraph*{Slanted profile.}

The third possibility is to report profiles that are slanted to correspond
to the positions of the tangent points.  In this case, the forward model
doesn't need azimuthal interpolation coefficients, and the lines of sight
are all at the same azimuth from the orbit plane.  The points in the
profile are located by (lat,lon,$zeta$) for each solution point. 
Alternatively, the normal to the solution plane, and the plane's
displacement from the center of the earth, can be reported.  Then, only
$\zeta$ and an angular coordinate within this plane need be reported for
each solution point.

\paragraph*{Solutions at MIF tangent points.}

The fourth possibility is to assume a constant azimuth and report the
solutions along the path in the reporting plane that is formed by where
the MIFs' lines of sight pierce it, but not necessarily at those points. 
This is very much like the \emph{slanted profile} report, but it does not
assume that the locus of those points is a straight line.  This also
requires (lat,lon,$\zeta$), or solution plane plus (angle,$\zeta$) for
each solution, but no azimuth interpolation coefficients.

\paragraph*{Further considerations.}

The most general solution, applicable in all cases, is to provide for
three-dimensional interpolation from the set of reporting points to the
forward model tangent points, and allow the forward model to combine those
coefficients with the ones it currently develops to interpolate to the
lines of sight.  This allows the solutions to be reported on an arbitrary
grid.  It might be better to remove this calculation from the forward
model in all cases.  This would require removing the calculation of the
set of $\zeta$s used by the forward model from the forward model as well. 
Some benefit to performance might be gained if the forward model knows the
dimensionality of interpolation.  Alternatively, a general system of
sparse interpolation, based upon the matrix and vector modules, might be
developed.

If we assume that the forward model operates on one vertical scan, the
lines of sight still form a 2-D surface in all cases, but in the second
and fourth cases they are not in a plane.

If we do not assume homogeneity along the line of sight, a 3-D temperature
field is needed for the hydrostatic routine to compute height and pressure
on this surface.

In the SMLS case a 3-D temperature field would be needed, but presumably
we would have it anyway.

The existing metrics routine could probably be used, or would require at
most modest revisions.

\label{lastpage}
\end{document}

% $Id$

% $Log$
