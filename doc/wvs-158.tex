\documentclass[11pt]{article}
\usepackage{alltt}
\usepackage[fleqn]{amsmath}
\usepackage{floatflt}
\usepackage{graphicx}
\usepackage{longtable}
\usepackage[strings]{underscore}

\textwidth 6.5in
\oddsidemargin -0.25in
%\evensidemargin -0.5in
\topmargin -0.5in
\textheight 9.1in

\newcommand{\docname}{wvs-158r1}
\newcommand{\docdate}{11 December 2019}

\ifx\pdfoutput\undefined
  \pdfoutput=0
  \usepackage[hypertex,plainpages,hyperindex=true]{hyperref}
  \hypersetup{%
    hypertexnames=false%
  }
  % Specify the driver for the color package
  \ExecuteOptions{dvips}
  %\ExecuteOptions{xdvi}
\else
  \ifnum\pdfoutput>0
   
\usepackage[pdftex,plainpages,hyperindex=true,pdfpagelabels]{hyperref}
    \hypersetup{%
      hypertexnames=false,%
      colorlinks=true,%
      linktocpage=true,%
    }
    % Specify the driver for the color package
    \ExecuteOptions{pdftex}
  \else
    \usepackage[hypertex,plainpages,hyperindex=true]{hyperref}
    \hypersetup{%
      hypertexnames=false%
    }
    % Specify the driver for the color package
    \ExecuteOptions{dvips}
    %\ExecuteOptions{xdvi}
  \fi
\fi

\hyperbaseurl{}
\newcommand\hr[1]{\href{#1.dvi}{dvi}, \href{#1.pdf}{pdf}}
\newcommand\h[1]{#1 (\hr{#1})}

\renewcommand\d{\text{d}}
\newcommand\T{\mathcal{T}}
\newcommand{\M}{\mathcal{M}}

\begin{document}

%\tracingcommands=1
\newlength{\hW} % heading box width
\newlength{\pW} % page number field width
\settowidth{\hW}{\bf\docname}
\settowidth{\pW}{Page \pageref{lastpage}\ of \pageref{lastpage}}
\ifdim \pW > \hW \setlength{\hW}{\pW} \fi
\makeatletter
\def\@biblabel#1{#1.}
\newcommand{\ps@twolines}{%
  \renewcommand{\@oddhead}{%
    \docdate\hfill\parbox[t]{\hW}{{\hfill\bf\docname}\newline
                          Page \thepage\ of \pageref{lastpage}}}%
\renewcommand{\@evenhead}{}%
\renewcommand{\@oddfoot}{}%
\renewcommand{\@evenfoot}{}%
}%
\makeatother
\pagestyle{twolines}

\vspace{-12pt}
\begin{tabbing}
\phantom{References: }\= \\
To: \>Bill Read\\
Subject: \>``Correct'' $\vec{d}$ vector -- ATBD Equation (5.22)\\
From: \>Van Snyder\\
References: \>\h{wvs-146} \\
\end{tabbing}

\parindent 0pt \parskip 4pt

\vspace*{-15pt}

From the ATBD before Equation (5.11), the normal radius at orbit geodetic
angle $\phi$, that is, the distance from the surface to the minor axis of
the ellipse defined by the intersection of the surface of the Earth and
the orbit plane, in the direction normal to the ellipse, is

\begin{equation}
N(\phi) = \frac{a^2}{\sqrt{a^2 \cos^2 \phi + c^2 \sin^2 \phi}} 
 = \frac{a^2}{\sqrt{D}} = \frac{a}{\sqrt{1-e^2 \sin^2 \phi}} \,,
\end{equation}

where $a$ is the Earth equatorial radius, $c$ is the semi-minor axis
of the orbit-projected ellipse and

\begin{equation}
D = a^2 \cos^2 \phi + c^2 \sin^2 \phi = a^2(1-e^2 \sin^2 \phi) \,.
\end{equation}

From ATBD Equation (5.10), the geocentric sub-tangent point is:

\begin{equation}\begin{split}\label{two}
\vec{R}^\oplus = \,&
 N(\phi) \left[ \cos \phi, \frac{c^2}{a^2} \sin \phi \right]
 = \left[ \frac{a^2 \cos \phi}{\sqrt{D}},
          \frac{c^2 \sin \phi}{\sqrt{D}} \right]
 = \frac{a}{\sqrt{1-e^2 \sin^2 \phi}}
   \, [ \cos \phi, (1-e^2) \sin \phi ]
\text{ and } \\[5pt]
|\vec{R}^\oplus|
% = \sqrt{\frac{a^4 \cos^2 \phi + c^4 \sin^2 \phi}
%                              {a^2 \cos^2 \phi + c^2 \sin^2 \phi}}
 = \,&\sqrt{\frac{a^4 \cos^2 \phi + c^4 \sin^2 \phi}{D}}
 = a \sqrt{ \frac{1+e^2 ( e^2-2) \sin^2 \phi}{1-e^2 \sin^2 \phi}} \,.
\end{split}\end{equation}

Tangent vector:

\begin{equation}\begin{split}\label{three}
\vec{T} = \,& \frac{\d \vec{R}^\oplus}{\d \phi} =
 \frac{a^2 c^2}{D^\frac32} \, [\, -\sin \phi, \cos \phi \, ] =
 g \, [\, -\sin \phi, \cos \phi \, ] =
 a\,\frac{a^2(1-e^2)}{(1-e^2 \sin^2 \phi)^\frac32} \,
  [\, -\sin \phi, \cos \phi \, ] \\
 = \,& N(\phi) \frac{a^2(1-e^2)}{D} [\, -\sin \phi, \cos \phi \, ] \,,
\end{split}\end{equation}

where $g = \mathbf{n} \cdot \frac{\partial^2 \mathbf{r}}{\partial \phi^2}$
is the third term in the second fundamental form for an ellipsoid.

Inward normal vector:

\begin{equation}
\vec{G} = [ \vec{T}_2, -\vec{T}_1 ]
         = a\,\frac{a^2(1-e^2)}{(1-e^2 \sin^2 \phi)^\frac32} \,
  [\, \cos \phi, \sin \phi \, ]
 = N(\phi) \frac{a^2(1-e^2)}{D}  [\, \cos \phi, \sin \phi \, ] \,.
\end{equation}

Substituting $[x,y] = \vec{R}^\oplus$ from Equation (\ref{two}) into

\begin{equation}
\nabla E = \nabla \left( \frac{x^2}{a^2} + \frac{y^2}{c^2} \right )
         = \nabla \frac1{a^2} \left( x^2 + \frac{y^2}{1-e^2} \right)
         = \frac2{a^2} \left[ x, \frac{y}{1-e^2} \right]
         = \frac2{\sqrt{D}} \left[ \cos \phi, \sin \phi \right]\,,
\end{equation}

results in
\begin{equation}
\frac{|\nabla E|}{|\vec{G}|} = 2\, \frac{1 - e^2 \sin^2 \phi}{a^2(1-e^2)}
 = 2\,\frac{D}{a^4 ( 1-e^2 )} \,.
\end{equation}

$\vec{d}$ vector:

\begin{equation}\begin{split}\label{d}
\vec{d} = \vec{R}^\oplus - \vec{G} = \,&
 N ( \phi ) \,
 \left[ \left(1 - \frac{c^2}D \right) \cos \phi,
        \left( \frac{c^2}{a^2} - \frac{c^2}D \right) \sin \phi
 \right] \\
 = \,& a\frac{e^2}{(1-e^2 \sin^2 \phi)^\frac32}
     [\, \cos^3 \phi, \, (1-e^2) \sin^3 \phi \, ] \\
 =  \,& N(\phi) \frac{a^2\,e^2}{D}\, [\, \cos^3 \phi,\, (1-e^2) \sin^3 \phi\, ] \,.
\end{split}\end{equation}

ATBD Equation (5.21):

\begin{equation}\label{5.21}
R_\text{eq}^\oplus \equiv H_t^\oplus = | \vec{T} |
 = N(\phi_t) \sqrt{ \sin^2 \phi_t + \frac{c^4}{a^4} \cos^2 \phi_t}
 = \frac{\sqrt{a^4 \sin^2 \phi_t + c^4 \cos^2 \phi_t} }
        {\sqrt{D}}
\end{equation}

was developed from Equation (\ref{two}) using

\begin{equation}
\vec{T} = \frac{\d \vec{R}^\oplus}{\d \phi}
 = N(\phi) \frac{\d}{\d \phi}
    \left[ \cos \phi, \frac{c^2}{a^2} \sin \phi \right] +
   \frac{\d N(\phi)}{\d \phi}
    \left[ \cos \phi, \frac{c^2}{a^2} \sin \phi \right] \,,
\end{equation}

but the $\frac{\d N(\phi)}{\d \phi}$ term was neglected.  Equation
(\ref{three}) is the correct equation for $\vec{T}$.  The correct equation
for $R^\oplus_\text{eq} = | \vec{G} |$ is

\begin{equation}
R^\oplus_\text{eq} \equiv H_t^\oplus = | \vec{G} |
 = N(\phi_t) \frac{a^2 ( 1-e^2 )}{D}
 = a\, \frac{1-e^2}{(1-e^2 \sin^2 \phi_t)^\frac32}
 = \frac{a^2 \,c^2}{(a^2 \cos^2 \phi_t + c^2 \sin^2 \phi_t)^\frac32} \,,
\end{equation}

which is the same as the meridional curvature of an oblate spheroid,
derived in \h{wvs-146}.

Using Equation (\ref{5.21}) results in ATBD Equation (5.22):

\begin{equation}\begin{split}
\vec{d} = \,& \left[ \left( N(\phi_t) - H_t^\oplus \right) \cos \phi_t,
                 \left( \frac{c^2}{a^2} 
                        N(\phi_t) - H_t^\oplus \right) \sin \phi_t \right]
        \\
 = \,& N(\phi_t) \left[ \left( 1 -
                    \sqrt{ \sin^2 \phi_t + \frac{c^4}{a^4} \cos^2 \phi_t}
                    \right) \cos \phi_t,
                    \left( \frac{c^2}{a^2} -
                    \sqrt{ \sin^2 \phi_t + \frac{c^4}{a^4} \cos^2 \phi_t}
                    \right) \sin \phi_t
             \right] \\
\end{split}\end{equation}

being different from Equation (\ref{d}), and also being incorrect.

\label{lastpage}
\end{document}

% $Id$
